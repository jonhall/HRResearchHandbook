\documentclass[a4paper,12pt,oneside,landscape]{memoir}

\usepackage{xparse}

%%Footnotes
\usepackage[perpage,side,marginal,symbol*]{footmisc}
\usepackage{ textcomp }%%For text circle

\DefineFNsymbols{hallrapanotti}{{\color{red}\textbullet}{\color{green}\textbullet}{\color{blue}\textbullet}{\color{orange}\textbullet}{\color{gray}\textbullet}{\color{purple}\textbullet}{\color{black}\textbullet}}
\setfnsymbol{hallrapanotti}

%%Temporary margin change
\usepackage{changepage}

%%Temporary margin change
\usepackage{changepage}

%%options interpretation
\usepackage{ifdraft}

\usepackage{amssymb}
%%For ordinals – \nth
\usepackage{fmtcount}

%%Colors
\usepackage[svgnames]{xcolor}

%%Bibtex stuff
\usepackage[
	backend=bibtex,
	style=authoryear,
	autocite=inline,
	sortcites,
	backref=true,
]{biblatex}
\addbibresource{MY-BIB-CLEAN}
\addbibresource{collatedBib}

%%Ensure bibliography always included, even if \end{document} early
\AtEndDocument{%
\ifoptionfinal{\printsolutions}{}
	\printbibliography}

%%Extended list environments
\usepackage[inline,shortlabels]{enumitem}


%%Xrefs
\usepackage{hyperref}
\usepackage{xurl}

%%Graphics
\usepackage{graphicx}
\graphicspath{{Figures/}}

%%To control paragraph separation
\usepackage{parskip}

%%Numbering options to subsection
\setcounter{secnumdepth}{3}
\setcounter{tocdepth}{2}

%%Quotes
\usepackage[threshold=15,thresholdtype=words,autopunct]{csquotes}
\AddToHook{env/quote/before}{\textit}


%%Underconstruction logo visible from overview
\newcommand{\UCJ}{\ {\color{red}\rule{2cm}{2cm}}}
\newcommand{\UCL}{\ {\color{pink}\rule{2cm}{2cm}}}

%%Todo notes
\usepackage[textsize=tiny,obeyFinal]{todonotes}
% 2022-10-11: so that todos don't eat following spaces.
% https://tex.stackexchange.com/questions/11802/todo-note-steals-space
\usepackage{xpatch}
\makeatletter
\xpretocmd{\todo}{\@bsphack}{}{}
\xapptocmd{\todo}{\@esphack}{}{}
\makeatother

%%%For Activities and the like
\newcommand{\exampleColor}{CornflowerBlue!40!white}
\newcommand{\activityColor}{Pink!80!white} 
\newcommand{\discussionColor}{\activityColor} 
\newcommand{\tipColor}{Green!20!white} 
\newcommand{\quoteColor}{Linen} 
\newcommand{\frameColor}{black!50!white}
\usepackage[many]{tcolorbox}
\newtcolorbox{activity}{
%  enhanced,
  breakable,
  pad at break=2mm,
  left=2mm,
  right=2mm,
  colback=\activityColor,
  colframe=\frameColor,
%  drop fuzzy midday shadow=black!50!yellow, 
  width={\textwidth}, 
  frame hidden, 
  segmentation hidden,
  before=\par\vspace*{2mm},
  after=\par\bigskip,
	title=%\textbf
		{\GetQuestionProperty{subtitle}{\CurrentQuestionID}\hfill\#\CurrentQuestionID}}

\newtcolorbox{discussion}{
%  enhanced,
  breakable,
  pad at break=2mm,
  left=2mm,
  right=2mm,
  colback=\discussionColor,
  colframe=\frameColor,
%  drop fuzzy midday shadow=black!50!yellow, 
  width={\textwidth}, 
  frame hidden, 
  segmentation hidden,
  before=\par\vspace*{2mm},
  after=\par\bigskip,
title={Discussion\hfill \#\CurrentQuestionID},
}

 %%Guidance environment
\newenvironment{guidance}{\tcbsubtitle%[before skip=\baselineskip]%
          {\textbf{Guidance}}}
    {}

\newtcolorbox{tip}{
%  enhanced,
  breakable,
  pad at break=2mm,
  left=2mm,
  right=2mm,
  colback=\tipColor,
  colframe=\frameColor,
%  drop fuzzy midday shadow=black!50!yellow, 
  width={\textwidth}, 
  frame hidden, 
  segmentation hidden,
  before=\par\vspace*{2mm},
  after=\par\bigskip,
title={Tip},
}

\newtcolorbox{example}[1]{
%  enhanced,
  breakable,
  pad at break=2mm,
  left=2mm,
  right=2mm,
  colback=\exampleColor,
  colframe=black!50!white,
%  drop fuzzy midday shadow=black!50!yellow, 
  width={\textwidth}, 
%  frame hidden, 
  segmentation hidden,
  before=\par\vspace*{2mm},
  after=\par\bigskip,
  title=\textbf{#1}}%
  
\newtcolorbox{running}[1]{
%  enhanced,
  breakable,
  pad at break=2mm,
  left=2mm,
  right=2mm,
  colback=\exampleColor,
  colframe=black!50!white,
%  drop fuzzy midday shadow=black!50!yellow, 
  width={\textwidth}, 
%  frame hidden, 
  segmentation hidden,
  before=\par\vspace*{2mm},
  after=\par\bigskip,
  title=\textbf{#1}}%

%%Quote environment
\renewtcolorbox{quote}{
%  enhanced,
  breakable,
  pad at break=2mm,
  left=2mm,
  right=2mm,
  colback=\quoteColor,
  colframe=\frameColor,
%  drop fuzzy midday shadow=black!50!yellow, 
  width={\textwidth}, 
%  frame hidden, 
  segmentation hidden,
  before=\par\vspace*{2mm},
  after=\par\bigskip,
  title=\textbf{Quote}}%

%%%For question solution pairs, with delayed printing
\usepackage{exsheets}
\SetupExSheets{
	headings=empty,
	question/pre-hook=\begin{activity},
	question/post-hook=\end{activity},
	solution/pre-hook=\IfInsideQuestionTF{\tcbsubtitle{Discussion}}{\begin{discussion}},
	solution/post-hook=\IfInsideQuestionTF{}{\end{discussion}},
	solution/print=true,
	}
\ifoptionfinal{\SetupExSheets{solution/print=false}}{}	

%%Answer environment
%%Is this used?
\newenvironment{answer}{\textbf{Answer}\begin{trivlist}
\item}
    {\end{trivlist}}

%%For layout
\usepackage{geometry}
\geometry{
%  showframe,
%  paperwidth=6in,
%  paperheight=9in,
%	asymmetric,
  left=1cm,
  right=1cm,
  top=1.5cm,
  bottom=1.5cm,
  marginparsep=8mm,
  marginparwidth=75mm,
  includemp,
  includehead,
}
  
%%https://tex.stackexchange.com/questions/94845/problems-with-toprule-and-midrule-in-a-table
\usepackage{booktabs}%%For fancy lines in tables
\usepackage{multicol}%%For multiple columns

%%For additional caption options
%%Allows captions in longtables
\usepackage{caption}
\usepackage{float}

%%Move to embedded table.tex

%%Tables and figures labelled consecutively – TODO: not working
\RequirePackage{chngcntr}
\counterwithout*{table}{part}
\counterwithout*{figure}{part}

%%Migrating to tabularray as for tabular
\RequirePackage{tabularray}
\UseTblrLibrary{varwidth}%%For itemize in entries
\UseTblrLibrary{booktabs}%%For lines in entries
\UseTblrLibrary{siunitx}%%For aligned decimal points
%\SetTblrOuter{expand=\frameColor}
%\SetTblrOuter{expand=\activityColor}

%%Define the widest tables width
%\newlength{\widetablewidth}
%\newlength{\narrowtablewidth}
%\newcommand{\widetablewidth}{\the\dimexpr(\textwidth+\marginparsep+\marginparwidth)\relax}
%\newcommand{\narrowtablewidth}{\the\dimexpr(\textwidth-\marginparwidth/8)\relax}
\newcommand{\widetablewidth}{\textwidth+\marginparsep+\marginparwidth}
\newcommand{\narrowtablewidth}{\textwidth-\marginparwidth/4}%Ensure - isn't –

%Table colours – not used
\newcommand{\tableBackgroundColor}{black!90!white}

%%Create new column type for ragged right in body.
\NewColumnType{R}[1][1]{X[#1,preto={\raggedright}]}

%%Simple N column table
\NewDocumentEnvironment{SimpleNColTable}{
		m% 					% label
		m%					% number of columns
		O{\widetablewidth} 	% tablewidth (default: \widetablewidth)
		m% 					% caption
		O{XXXXXXXXXX|}%	% optional colspec
		O{}					% Any other options to pass
		+b%					% body – *{10} is "upto 10 repeats, depending on how many entries the table has.
		}%
		{
		\begin{tblr}[
				long,
	 			label={#1},
				caption={#4}
%				expand=\activityColor
			]{
		  		colspec = {#5},
	 			width = {#3},
				stretch=-1,%%Allows lists in body, with no padding
				hlines = {fg = \frameColor, wd=2pt}, %column{2} = {co=1}, 
				vline{Z} = {fg = \frameColor, wd=2pt}, %column{2} = {co=1}, 
				colsep = 5pt,
%TODO change colours to neutral/transparent, if possible.
				column{1} = {fg = white, bg=\frameColor},%, font=\sffamily},
				row{1} = {font=\bfseries, fg = white, bg=\frameColor},%, font=\bfseries\sffamily},
				rows = {m,rowsep = 3mm},
				rowhead = 1,
				#6, % Any other options passed directly
				measure=vbox, 
			}
			#7
		\end{tblr}
	}{}

%%Complex table definitions
%%All ResActTables follow this format
\NewTableCommand{\ResActTableHeader}{&Deliverable&{Writing Outcome:\\ by the end of this stage you will}&Effort&Supervisor Interaction}
\NewTableCommand{\ResActChHeader}{\SetCell[c=5]{l}}
\NewTableCommand{\ReportLineHeader}{\SetCell[c=3]{l}}

\NewDocumentEnvironment{ResActTable}{O{Missing caption}+b}{
	\begin{tblr}[
			long,
			caption={#1}
		]{
			measure=vbox, 
			stretch=-1,%%Allows lists in body, with no padding
 			width=\widetablewidth,
	  		hlines = {fg = \frameColor, wd=2pt}, %column{2} = {co=1}, 
			vline{Z} = {fg = \frameColor, wd=2pt}, %column{2} = {co=1}, 
			colsep = 5pt,
			column{1} = {fg = white, bg=\frameColor},%, font=\sffamily},
			row{1} = {fg = white, bg=\frameColor},%, font=\bfseries\sffamily},
			rowhead = 1,
			colspec = {lX[8]X[13]X[-1]X[5]|}}
%Header					
	&Deliverable&{Writing Outcome:\\ by the end of this stage you will}&Effort&Supervisor Interaction\\
%Body
		#2
	\end{tblr}}{}
	
%%All ReportTables follow this format
\NewTableCommand{\ReportTableHeader}{&Deliverable&{Writing Outcome:\\ by the end of this stage you will}&Effort&Supervisor Interaction}
\NewTableCommand{\ReportChHeader}{\SetCell[c=5]{l}}

\NewDocumentEnvironment{ReportTable}{
		m					%label
		O{Missing caption}	%caption
		+b					%body
		}{
	\begin{tblr}[long,caption={#2},label={#1}]{measure=vbox, stretch=-1,%%Allows lists in body, with no padding
 				 	width=\widetablewidth,
			  		hlines = {fg = \frameColor, wd=2pt}, %column{2} = {co=1}, 
					vline{Z} = {fg = \frameColor, wd=2pt}, %column{2} = {co=1}, 
					colsep = 5pt,
					column{1} = {fg = white, bg=\frameColor},%, font=\sffamily},
					row{1} = {fg = white, bg=\frameColor},%, font=\bfseries\sffamily},
					rowhead = 1,
%					cell{1}{1}={c=1}{l, font=\bfseries},
%					cell{1}{2}={c=1}{l, font=\bfseries},
					colspec={X[8,l]X[18]X[4,preto={\raggedright}]},
					}
%Header					
	Part/Structure&Guidance\\
%Body
		#3
	\end{tblr}}{}
	
%%ReportTable auxiliaries
\newcommand{\basedOn}[1]{that is based on the work you completed in~\Cref{#1}}
%%repeated report table messages
\newcommand\tablenocontent{No content at this stage}
\newcommand\tablecites{You should include those references that you cite in new and/or update chapters}
%\newcommand\tableconsiderprogress{Ensure all cited references are listed in this section}
\newcommand\tablefinalise[1]{After completing your work for this state, finalise your #1}
\newcommand\tableminor[1]{Reread and make any minor modifications to your #1}
\newcommand\tablecheck[1]{You should check your #1 for correctness and completeness}
	
	
%%ReportTable list formats 
\usepackage{enumitem}
\newlist{reportenum}{enumerate}{2}
\setlist[reportenum]{nosep}
\setlist[reportenum,1]{label={\arabic*}}
%\setlist[reportenum,2]{resume}
\newlist{reportitem}{itemize}{1}
\setlist[reportitem]{nosep,label=\textbullet}

\usepackage{ifthen}

\newcommand{\ordColour}[0]{gray8}

\newcommand{\ritem}[2]{%
	\ifthenelse{\equal{#1}{!}}
		%%new section
		{\item[\color{red}{\footnotesize new}]\color{white}}
		{\ifthenelse{\equal{#1}{*}}
			%%updated section
			{\item[\color{white}\textbullet]\color{white}}
			%%not yet included						
			{\ifthenelse{\equal{#1}{?}}
				%%major update section
				{\item[\color{orange}{\footnotesize update}]\color{white}}
				%%not yet included						
				{\item[\color{\ordColour}\textbullet]\color{\ordColour}}
			}
		}
	#2}

\newcommand{\titem}[2]{%
	\ifthenelse{\equal{#1}{*}}%
		{\color{white}}%
		{\color{\ordColour}}%
	#2}

\newcommand\ReportTitle[1]{\ReportLineHeader\titem{#1}{Title}}

\newcommand\ReportAbstract[5]
	{\ReportLineHeader\titem{#1}{Abstract}
	\begin{reportenum}
		\ritem{#2} {Research problem, its context and significance}
		\ritem{#3} {Research aim}
		\ritem{#4} {Research methodology}
		\ritem{#5} {Knowledge contribution and its implications}
	\end{reportenum}
}
	
\newcommand\ReportIntroduction[4]%
	{\ReportLineHeader\titem{#1}{Chapter 1: Introduction}
	\begin{reportenum}
		\ritem {#2} {Background to the research}
		\ritem {#3} {Justification for the research}
		\ritem {#4} {Definitions}
	\end{reportenum}
}

\newcommand\ReportLitRev[3]%
	{\ReportLineHeader\titem{#1}{Chapter 2: Literature review}
	 \begin{reportenum}
		\ritem {#2}{Review of existing relevant knowledge}
		\ritem {#3}{Critical summary, including knowledge gap to be addressed by the research}
	\end{reportenum}
}

\newcommand\ReportResearchDef[4]%
	{\ReportLineHeader\titem{#1}{Chapter 3: Research Definition}
		 \begin{reportenum}
			\ritem{#2}  {Problem statement}
			\ritem{#3}  {Aim and objectives} 
			\ritem{#4}  {Knowledge contribution}
		\end{reportenum}
	}

\newcommand\ReportResearchDes[5]%
	{\ReportLineHeader  \titem{#1}{Chapter 4: Research design}
		 \begin{reportenum}
			\ritem{#2}  {Data and evidence} 
			\ritem{#3}  {Research strategy and methods} 
			\ritem{#4}  {Research procedures} 
			\ritem{#5}  {Ethics and regulations} 
		\end{reportenum}
	}

\newcommand\ReportAnalysisInterp[4]%
	{\ReportLineHeader \titem{#1}{Chapter 5: Analysis and interpretation}
		 \begin{reportenum}
			\ritem{#2}  {Summary and analysis of data} 
			\ritem{#3}  {Summary of key findings} 
			\ritem{#4} {Interpretation in relation to aim and objectives} 
		\end{reportenum}
	}

\newcommand\ReportEvalConc[7]%
	{\ReportLineHeader  \titem{#1}{Chapter 6: Evaluation and conclusion}
	 \begin{reportenum}
		\ritem{#2}  {Evaluation against aim and objectives} 
		\ritem{#3}  {Evaluation against related work in the literature} 
		\ritem{#4}  {Validity of the research} 
		\ritem{#5}  {Future work} 
		\ritem{#6}  {Implications for practice} 
		\ritem{#7}  {Reflexive account} 
	\end{reportenum}
}

\newcommand\ReportRefs[1]{\ReportLineHeader\titem{#1}{References}}


\newcommand\ReportProgressTracking[6]%
	{\ReportLineHeader\titem{#1}{Appendix - Progress tracking}
	\begin{reportenum}
		\ritem{#2}{Fit}
		\ritem{#3}{Feasibility}
		\ritem{#4}{Risk assessment}
		\ritem{#5}{Work schedule}
		\ritem{#6}{Further reading to be completed}
	\end{reportenum}
}

\newcommand\ReportReflection[2]%
	{\ReportLineHeader\titem{#1}{Appendix - Reflection and reflexivity}
	\begin{reportenum}
		\ritem{#2}{Personal statement}
	\end{reportenum}
}

\newcommand\ReportDissertationAppendices[2]%
	{\ReportLineHeader\titem{#1}{Dissertation appendices}
	\begin{reportenum}
		\ritem{#2}{Raw data, questionnaires, code, etc.}
	\end{reportenum}
}

\newcommand\ReportAppendices[1]{\ReportLineHeader\titem{#1}{Appendices}}



\newcommand{\WOCaption}[2]{Stage #1{} Writing Outcomes (#2\% of stage effort)}
\newcommand{\RActivitiesTableCaption}[1]{Stage #1{} Research Activities}

\endinput
%%%For tables – intention is to migrate to tabularray
%\usepackage{longtable,tabulary,tabularray}
%\UseTblrLibrary{varwidth}%%For itemize in entries
%\UseTblrLibrary{counter}%%For counters in tables

%%tabulary gives L column
%%David Carlisle's mix of tabulary and longtable from https://tex.stackexchange.com/questions/78075/multi-page-with-tabulary
\makeatletter
\def\ltabulary{%
	\def\endfirsthead{\\}%
	\def\endhead{\\}%
	\def\endfoot{\\}%
	\def\endlastfoot{\\}%
	\def\tabulary{%
		\def\TY@final{%
			\def\endfirsthead{\LT@end@hd@ft\LT@firsthead}%
			\def\endhead{\LT@end@hd@ft\LT@head}%
			\def\endfoot{\LT@end@hd@ft\LT@foot}%
			\def\endlastfoot{\LT@end@hd@ft\LT@lastfoot}%
%JGH Not yet working	\let\caption\LT@caption%%JGH 2023-10-13: Added to allow ltabulary captions
%			\let\LT@mcol\multicolumn%%JGH 2023-10-13
			\longtable}%
		\let\endTY@final\endlongtable
		\TY@tabular}%
%	\dimen@\columnwidth
%	\advance\dimen@-\LTleft
%	\advance\dimen@-\LTright
	\tabulary}%\dimen@}

\def\endltabulary{\endtabulary}

%%so that tables extend into margin
\newdimen\tablewidth
\setlength{\tablewidth}{1.4\textwidth}

\makeatother

%%Hack for @{}lllll@{} instead of @{}LLLL@{}
%%Remove when all tables properly formatted
\newcolumntype{l}{L}

%%Reinstate when in final version
%\usepackage{fontspec}
%
%\usepackage{iffont}
%\settofirstfound{\mainfont}{
%	Lato Light,
%	Arial,
%	}
%\settofirstfound{\boldfont}{
%	Lato Bold,
%	} 
%\setmainfont[BoldFont=\boldfont]{\mainfont}


%%\begin{stolen} to identify uncredited material
\newenvironment{stolen}[1]{\bgroup\color{red}\marginpar{Material stolen from #1}}{\egroup}
\newenvironment{adapted}[1]{\bgroup\color{blue}\marginpar{Material adapted from #1}}{\egroup}

%%For progress indicator
\usepackage{lastpage}
\usepackage{refcount}
\usepackage{intcalc}

\title{Master Research\\ Handbook}
\author{Lucia Rapanotti and Jon Hall}
\def\mytitle{Masters Research Handbook}
\def\myauthor{Lucia Rapanotti and Jon Hall}
\def\mycopyright{Copyright 2023, Lucia Rapanotti and Jon Hall. All rights reserved.}
\title{ \mytitle }
\author{ \myauthor }
\begin{document}
\maketitle
\setcounter{tocdepth}{3}
\tableofcontents

\chapter*{Foreword}
You have just about embark in your very first Masters research project. That's both exciting and daunting.

Exciting because you will be able to focus on a topic of your own choice and to investigate in depth an issue or problem which is of particular interest to you, either personally or professionally. In doing so, you will acquire a deep knowledge of that topic, conduct a unique and novel research study, and develop and apply a wide range of research skills.

Daunting because your success will depend on you demonstrating your mastering of the topic and of the research process, that you can exercise competently a wide range of research skills, and can communicate your work effectively through an academic dissertation, possibly the largest and most demanding piece of writing you will ever undertake.

The aim of this handbook is to support you in taking your first steps into academic research at Masters level. It will provide you with a solid scaffolding for you to become a competent and confident researcher. It will demystify the language around academic research and, through practical advice and activities, will help you plan and manage your project work successfully from start to finish.

But the benefits don't stop with the completion of your Masters project. In succeeding in your research project, you will have also gained and demonstrated a wide range of skills which are professionally relevant and valued by employers, from problem solving, effective communication, digital and information literacy, self-management and resilience. These transferable skills will serve you well in your profession and your life, regardless of the path you will take after your Masters course.

\chapter{So, you want to do a research project!}
Well, you've come to the right place{\ldots}

\section{What is academic research?}
In its most general sense, research is investigation, fact-finding, exploration and analysis. However, academic research is first and foremost about the \emph{generation of new knowledge} with respect to what is already known in a field of study. New knowledge is what academic research must deliver.

Academic research is therefore about \emph{the process of knowledge generation} and how it is executed. The expectation is that of a \emph{systematic and rigorous process} of collecting, analysing and interpreting data, of drawing well-founded conclusions from evidence and explicitly stated assumptions, and of presenting findings in a clear and logical manner. There is also an expectation that your work is legal and ethical in all respects, both in the treatment of other participants and in any sensitive information, and in the way you report your work. It is this rigorous process of academic research that you will learn and practise by following the guidance in this handbook.

Academic research is also a \emph{collaborative endeavour} within a field of study: even if you are conducting your project on your own, you will build on what researchers have done before -- standing on the shoulders of giants as one great scientist once said-- , adding one more piece to the jigsaw puzzle so that future researchers can build on what you have done -- by standing on yours!

With the completion of your Masters, as a new researcher you will join a community of peers.

\begin{question}[subtitle={Activity: What is academic research?}] You\footnote{This book is full of activities like this. Each one is designed to help your personal research process about your Masters project. Each asks you to think about something and then write down your answer, which you should always do. Writing down your answer means you can revisit it later.} will have been involved in a process of personal research already in your life, probably daily, discovering knowledge that is \emph{new to you}. Every time you search the web and find information about something you didn't know, for instance, your own personal knowledge grows.

Take five minutes to think about how this personal research process differs from academic research. Write down your answer.

\begin{solution}If the knowledge\footnote{Of course, we`re talking reputable sources. All that is on the web isn't knowledge! Your studies will lead to new \emph{reputable} knowledge.} you seek is something you can easily find in existing sources, e.g., books, articles, the web, etc., then it is already part of the existing body of knowledge, so there is no new knowledge generated in a general sense. It is also unlikely you will have followed a rigorous and systematic process in collecting and analysing data to draw your conclusions, possibly because data are incomplete or of poor quality, or your analysis is not very rigorous. The issue you are investigating may also be of no relevance to a wider community. Rather than academic research, this is a process of personal learning.
\end{solution}\end{question}
%%Hack to correct tcbox behaviour
\color{black}

\subsubsection{Masters level research}
Now that you know what academic research is about, you may be wondering what is special about Masters level research. This is research in the context of a Masters programme, such as that done as part of a capstone project\footnote{A \emph{capstone} is a large flat stone that completes a tomb. A capstone project completes a degree or other programme of study. The two are not related:)}. Typically, MScs (Master of Science) have capstone projects as the final module.

Masters research is no different from any other kind of academic research in that it you will be expected to generate some original knowledge within your field of study, perhaps by addressing a relevant, non-trivial issue or problem within that field. It will be conducted in a rigorous, systematic and ethical manner, by applying accepted research methods and techniques.

However what makes your project Masters level compared to a Doctorate or post-doctoral research is primarily its scope: you must complete your Masters project within the time constraint of your Masters course, which will be a small fraction\footnote{A capstone module is usually 600 hours of study. A Doctorate could be as much as 10 times that.} of the time you would have to spend on a Doctorate. Therefore, at Masters level you must trade off your ambition as a researcher with what you can realistically achieve within the available time.

Because of this, Masters degrees leave very little room for errors in those early choices. In particular, it is critical to identify a scope appropriate for your intended research very early in your studies and might involve choosing a topic proposed by an academic who will become your supervisor. Often Masters projects take an existing idea or approach and apply it in a novel way, while a Doctorate project may come up with a ground-breaking, completely new approach.

In addition, your choice of topic will be limited by the degree for which you are studying. If you're doing a Masters in Geography, for instance, a capstone project in the area of information security is not going to be relevant.

Lastly, the range of methods and techniques\footnote{You'll hear a lot about methods and techniques that apply at Masters level in Section ? The skills you will build on top of them -- critical thinking, problem solving, reflection, for instance -- are generally useful, particularly in your professional career. So, there are very good reasons to study them well, even if you might get a bit sick of them during your project.} you will apply at Masters level is likely to be narrower than those used by a Doctoral researcher, as some of those approaches are just too time consuming to be feasible. This, too, limits the type of the research problems you will be able to address in your project.

\begin{example}{Example}The recent explosion of AI solutions has been driven by the creation of technology which is able to do human-like things, for instance the creation of images and text that is sometimes indistinguishable from what you or I might write. Some might go as far as to say that AI is creating new knowledge!

A good topic for a Masters' project might be to investigate the extent to which AI technology can be said to be `doing research'.

The originality of such project comes from the novel combination and application of existing techniques, or might build a framework for distinguishing new knowledge from the simpler re-hash of old knowledge.

This project might come up with new Intellectual Property\footnote{Intellectual Property (IP) is the practical part of new knowledge. Many Masters projects have led to IP that has been developed, franchised, and\slash or sold. Many areas, including law, health, software engineering, and others, recognise that IP is an important way of delivering economic benefits to a client base, business, or public body. Your university will have an IP policy that you should know about before signing up if there`s any chance your IP will be valuable. There's more -- much more -- about this in Section ?} (IP) capturing ways of distinguishing AI- and human-knowledge generation that leads to practical tools for use by business and education.
\end{example}

\section{What you will have achieved as a Masters' graduate}

While there may be differences in the specifics of each Masters, there are benchmarks which define the knowledge and skills all Masters' graduates must attain. Many countries use standard frameworks to define their academic qualifications. In the UK, for instance, the UK Frameworks for Higher Education Qualifications align to the Framework for Qualifications of the European Higher Education Area, bringing together standards across the European Union to ensure comparability between European degrees.\footnote{The ENIC--NARIC Networks are notable in ensuring that the academic recognition of qualifications across 55 countries. \href{https://www.enic-naric.net/page-framework-qualifications-europe-and-north-america-region}{https:\slash \slash www.enic-naric.net\slash page-framework-qualifications-europe-and-north-america-region}}

The UK has adopted the following definitions:

\begin{quote}
Master's degrees are awarded to students who have demonstrated:

\begin{enumerate}
\item a systematic understanding of knowledge, and a critical awareness of current problems and\slash or new insights, much of which is at, or informed by, the forefront of their academic discipline, field of study or area of professional practice

\item a comprehensive understanding of techniques applicable to their own research or advanced scholarship

\item originality in the application of knowledge, together with a practical understanding of how established techniques of research and enquiry are used to create and interpret knowledge in the discipline

\item conceptual understanding that enables the student:

\begin{itemize}
\item to evaluate critically current research and advanced scholarship in the discipline

\item to evaluate methodologies and develop critiques of them and, where appropriate, to propose new hypotheses.

\end{itemize}

\end{enumerate}

Typically, holders of the qualification will be able to:

\begin{enumerate}
\item deal with complex issues both systematically and creatively, make sound judgements in the absence of complete data, and communicate their conclusions clearly to specialist and non-specialist audiences

\item demonstrate self-direction and originality in tackling and solving problems, and act autonomously in planning and implementing tasks at a professional or equivalent level

\item continue to advance their knowledge and understanding, and to develop new skills to a high level.

\end{enumerate}

And holders will have:

\begin{enumerate}
\item the qualities and transferable skills necessary for employment requiring:

\begin{itemize}
\item the exercise of initiative and personal responsibility

\item decision-making in complex and unpredictable situations

\item the independent learning ability required for continuing professional development.

\end{itemize}

\end{enumerate}

\end{quote}


\subsubsection{}
This definition establishes key \emph{learning outcomes}\footnote{A \emph{learning outcome} defines what a student should be able to do after their study. Learning outcomes are tested in assessment, exams and, in the case of research, the dissertation or thesis. Although, in this case, they are based on a standard, they may be described differently in for your Masters -- often useful detail is added. Knowing the learning outcomes for the Masters degree you are studying can give you a real advantage when it comes to writing!} for a Masters graduate, primarily:

\begin{itemize}
\item Advanced knowledge at the leading edge of an academic discipline

\item Critical thinking, in evaluating existing knowledge, research methods and their application to generate new knowledge, and in making judgements and deriving conclusions

\item Originality in applying knowledge and techniques to address complex problems and generate new knowledge

\item Effective communication to diverse audiences

\item Self direction, autonomy and independent learning

\end{itemize}

\section{The role of your supervisor}
For your Masters, you will be assigned an academic supervisor to support you throughout the project --- a research expert highly knowledgeable in the subject area of your research. It may even be the case that your project topic comes from suggestions from, or previous research conducted by, your supervisor.

There's a hint in the title -- your supervisor is there to supervise your studies: to guide, make suggestions, and engage in discussion on all aspects of your research. They are unlikely to want to micro-manage your project (but it does sometimes happen!). But, whatever, ensuring that you make the required progress and meet the required standards remains your responsibility.

It's sometimes that case that, as you make progress in your research your knowledge and expertise will grow to rival that of your supervisor. However, your supervisor, as an expert in the \emph{process} of research, will continue to provide invaluable guidance and support throughout your project.

Depending on your course regulations, your supervisor may have additional roles. For instance, your supervisor may be required formally to assess whether you are making the necessary progress as you reach each milestone, or may be required to validate that the research carried out is indeed your own work. Missing their assessment and validation is very likely to slow progress, and in the worst case, stop you completing your Masters.

So, to discharge their various roles, your supervisor must have sight of your work and how it develops throughout the project; engaging with your supervisor regularly, discussing your ideas and their development in detail will be a very important part of conducting your project.

It is therefore essential to your success that you develop a close working relationship with your supervisor as early as is possible and engage in an active dialogue on all aspects of your research on a continuing basis.

You supervisor may have more than one student conducting research with them. This means that there will also an expectation that \emph{you} drive and manage the interaction with your supervisor.

\begin{question}[subtitle={Activity: Managing the interaction with your supervisor }] If you haven't already done so, get in touch with your supervisor to introduce yourself and discuss how they want to work together.

After the meeting, write down what you have agreed and create diary appointments in your favourite calendar app accordingly.

Make sure your supervisor formally agrees -- although, these days, this is usually just a matter of RSVPing to a calendar invitation.

\begin{guidance}Before contacting your supervisor, you should think about \emph{your} availability for a 30 minute meeting every week, for instance, or 60 minutes every fortnight. This will make it easier for you to identify and agree with your supervisor a regular slot to discuss your ideas and work progress. You should also account for periods when you or your supervisor may not be available.

It is important to book regular diary slots for the duration of the project. It may not be that you need every meeting you arrange, especially in the later stages of your research, but it will be good to know that the time is there if you do need it.

This activity is particularly important if you are a part-time student, possibly juggling study, work and family commitments.
\end{guidance}\end{question}
%%Hack to correct tcbox behaviour
\color{black}

\section{What is expected of you}
There are key differences between doing a research project and studying a taught module. Awareness of those differences will help you prepare better for the demands of Masters research.

These differences arise because of the higher demands of an academic research project to demonstrate i) self-direction, ii) critical thinking, iii) time and task management, and iv) information management.

\subsubsection{Self-direction}
A Masters' research project requires you to be in the driving seat\footnote{This includes ``managing'' the interaction with your supervisor, as mentioned earlier.}, including identifying relevant materials you must study and skills you need to develop and apply to carry out your project.

While in your previous studies, most of the materials were likely to be provided for you alongside detailed advice on how and when to study them, on a research project those decisions are left up to you to a large extent. In particular, you will need to decide what to read and what to ignore in relation to your chosen topic, determining its relevance and how to use it in your research.

You will also be responsible for understanding the academic standards required, and for organising your research work. This is part of demonstrating that you can exercise self-direction in planning, executing, and critically reviewing your work and that you are truly an independent learner.

This handbook will provide plenty of advice as to how direct your own research.

\begin{question}[subtitle={Activity -- self direction}] Write down three areas of your life where you need to be self-directed. For each, write down how you go about being self-directed, from how you start to how you finish. What does it feel like to be self-directed? Is there some emerging pattern?

\begin{guidance}Those areas might be from your work, your previous Masters' studies, your personal investments, hobbies, or some other aspects of your life.
\end{guidance}\end{question}
%%Hack to correct tcbox behaviour
\color{black}

\subsubsection{Critical thinking}
Your Masters research will require you to think critically about all aspects of your work. In essence, critical thinking is about asking questions systematically to look for evidence and good reasons in order to form your own judgements, instead of accepting what you read or hear `at face value'.

At the heart of critical thinking is an ability to maintain an `objective' position by weighing up all sides of an argument, evaluating its strengths and weaknesses, and testing how sound the claims made and their supporting evidence are. This will require you to scrutinise arguments in great detail and with some degree of skepticism. Such an objective stance will help you judge good and bad arguments, regardless of whether you agree with them.

In your project, you will be expected to think critically about all aspects of your research and to capture such thinking in your writing. As the author of a dissertation, critical thinking will benefit your ability to build stronger arguments, avoid bias and link your claims to appropriate supporting evidence. As a reader of the academic literature, critical thinking will help you to assess the strength of other researchers' arguments and identify unsupported claims and illogical reasoning.

Critical thinking is both an attitude and a skill essential in academic research, which this handbook will help you develop. It is also beneficial to your professional and private life. In fact, it is likely that you already think critically in many aspects of your life.

\begin{question}[subtitle={Activity -- critical thinking}] Write down three areas of your life where you need to think critically. For each, write down how you go about thinking critically, from how you start to how you finish. What does it feel like to think critically? Is there some emerging pattern?

\begin{guidance}These areas might be the same areas as for the previous activity, or might be different. You might even find that they are parts of previous activities.
\end{guidance}\end{question}
%%Hack to correct tcbox behaviour
\color{black}

\subsubsection{Time and task management}
Although, for your research project, you may be provided with some broad guidelines and deadlines\footnote{Deadlines for research tend to be absolute as they affect the cycle of Masters' assessment and validation.} around which you should conduct your project in general, however, the expectation will be that it is down to you to organise your time, and plan and execute your work in detail to meet those deadlines.

Within the set duration of your project, you will need to plan, organise and execute your research and submit your final dissertation. It will be up to you to develop and keep under review your personal research plan, allocate time to tasks, and meet deadlines and milestones\textbf{.}

You will find that one of the challenges of conducting academic research is that it is open ended. Therefore you will need to learn how to set boundaries and use your limited research time effectively to meet the requirements of a Masters.

This handbook will help you plan your project work based on a 5-stage framework we have developed for this purpose.

\begin{question}[subtitle={Activity -- Time and task management}] Write down three areas of your life where you need to manage your time and tasks. For each, write down how you go about managing tasks and time, from how you start to how you finish. What does it feel like to manage tasks and time? Is there some emerging pattern?

\begin{guidance}These areas might be the same areas as for the previous activity, or might be different. You might even find that they are parts of previous activities.
\end{guidance}\end{question}
%%Hack to correct tcbox behaviour
\color{black}

\subsubsection{Information management }
The volume of materials you will need to read and study is considerable: you will need to read extensively around the topic of your choice, as well as research approaches to apply in your research. This will result in a large amount of information that you will need to gather, organise, store and make sense of, including making copious notes as you go along.

You will also need to submit a substantial final dissertation. In the UK, for instance, a Masters research dissertation is usually in the range of 10,000 to 15,000 words. This is likely to be much longer than any other written work required by your previous studies. Therefore you will also need to structure your narrative appropriately and present your work in compliance with standards and conventions for academic writing.

All this will require you to develop and apply a range of new skills, which this handbook will help you develop.

\begin{question}[subtitle={Activity -- Information management}] Write down three areas of your life where you need to manage information. For each, write down how you go about managing information, from how you start to how you finish. What does it feel like to manage information? Is there some emerging pattern?

\begin{guidance}Again, these might be the same areas as for the previous activities, or might be different. You might even find that it is part of previous activities.
\end{guidance}\end{question}
%%Hack to correct tcbox behaviour
\color{black}

\section{Key skills}
In this book, there are some basic skills that you will practice over and over again in your project. As they are critical to your success, it's worth considering them upfront. One benefit of thinking about them early is that, if you feel that you're lacking in any of them, you can prioritise their study early on.

\subsubsection{Active reading and note taking}
Throughout your project, and particularly when reviewing the literature, you will read lots of materials, and you are likely to spend more time reading and assimilating new content than you may have experienced in your previous studies.

It is important, therefore, that you become both effective and efficient at reading and note taking. The key points you need to keep in mind are:

- You take notes as you read.

- You won't need to study in depth all that you read, so you should develop reading techniques both to grasp the essence of an article very quickly, and techniques to dig deeper into content and meaning.

- You should be disciplined and systematic in your note taking, so that you can easily locate and review your notes when you need to during your project.

These practices will help you become an \textbf{active reader}, that is one who engages with the materials and is able to assimilate the important points in an effective manner.

\begin{question}[subtitle={Activity: Your note taking practice}] Think of how you take notes when reading new materials. Write down key practices you apply and how effective they are in helping you assimilate new materials.

\begin{solution}These are things I usually do. While reading, I highlight definitions and word or phrases of interest, and make annotations in the margin next to paragraphs which make significant points, or provide useful summaries or raise interesting questions. If reading a physical book, I attach post-it notes to pages which are of particular interest, so that I can return to them easily.

After reading a whole article or a chapter in a book, I jot down some bullet points summarising key insights from my reading, my overall judgement of the whole reading, and how it may be useful, or otherwise, for my work.

I usually keep my notes and summaries with the physical or digital copy of what I have read, and I also keep together related articles within my file system --- in the past, I used to have a physical filing cabinet for this, but these days I work almost exclusively with digital content.

You will have come up with a different set of practices. What matters is that you reflect on how effective they are or where improvements may be needed.

\end{solution}\end{question}
%%Hack to correct tcbox behaviour
\color{black}

\subsubsection{Digital literacy and tools}
Unless you're thinking of handwriting your dissertation\footnote{\emph{And it has been done, although perhaps not that recently, check out \textbackslash{}url\{\href{https://www.reddit.com/r/Handwriting/comments/b9vmss/my_mums_handwritten_thesis_from_1982/}{https:\slash \slash www.reddit.com\slash r\slash Handwriting\slash comments\slash b9vmss\slash my\_mums\_handwritten\_thesis\_from\_1982\slash }\}, for instance!}}, you're going to come across digital tools for text preparation, bibliographic management, note taking, etc.

There is a vast range of tools, some of which you will be already familiar with (but might not be suitable for the very long document you're going to have to write) and some that you are going to have to learn how to use -- and find time for that learning. The most important of this latter group is probably those for \emph{bibliography management}.

So, why don't we start with these?

\subsubsection{Bibliographic management tools}
During your project, it is essential that you keep track of the articles and papers that you read, alongside your notes on their content and relevance, or otherwise, to your project. Some you will access over and over again during your research; others you will only cite as part of your literature review; others still, you may just discard as not relevant. Whatever their final use in your project might be, it is important that you keep track of what you have read and the use you can make of it as you go along.

As the amount of reading grows, keeping track can quickly become overwhelming. This is where a \textbf{Bibliographic Management Tool}\footnote{These are also known as Reference Management Tools` 'Reference management software` or 'Bibliographic management software.' Wikipedia has a comparison of bibliographic management tools here: url\{\href{https://en.wikipedia.org/wiki/Comparison_of_reference_management_software}{https:\slash \slash en.wikipedia.org\slash wiki\slash Comparison\_of\_reference\_management\_software}\}} \textbf{(BMT)} can help you. If you are not already using a BMT, this is the time for you to pick one and master its basic functionalities.

Briefly, a BMT is a software tool that can help you collect and save details of the resources you have used in your research, and from which you can generate easily references, reference lists and bibliographies in a variety of bibliographical styles, to include in your TMAs and dissertation. Hence, its basic functionalities include:

\begin{itemize}
\item a repository into which you can store the details of each document that you have read (authors, title and so on), your notes on the document, and even a copy of the document itself. Where the repository is online, the tool may allow you to share your collections with other users. In addition, in many cases, the reference sharing service will enable you to download many of the necessary citations from the database.

\item an export mechanism to desktop word processors, such as MS Word, Apple Pages or Apache OpenOffice, so that you can include your citations in your text and generate your reference section automatically. The exact mechanism and its accuracy vary between tools, but it is likely to generate a list of references with most (if not all) the required elements of your chosen bibliographical style --- some manual checking and editing may still be needed to ensure that the output conforms to the required style.

\end{itemize}

\begin{question}[subtitle={Activity: Investigating BMTs}] Visit the wikipedia article that compares notable reference management software (the link is in the related note above). Count how many are included, which ones are free or licensed, and which operating systems, devices and work processing software they are compatible with.

Choose a couple which may work with your computing devices and software. Review their key features.

\begin{guidance}If you already know or use a BMT, you may wish to compare its features with those of others in the wikipedia article. If you don't know where to start, you could consider Zotero first.

It is also possible your university has licensed one BMT that you can use for your project, so you should find out whether that is the case and what are its key features.

\end{guidance}\end{question}
%%Hack to correct tcbox behaviour
\color{black}

\subsubsection{Keeping track of your digital assets}
Beside your growing repository of articles from the literature, which you should manage through your chosen BMT, during your project you will be collecting and producing lots of other digital \textbf{research assets} which are needed for your research and will contribute to your dissertation. These may include data, images, tables, your own notes, comments from your supervisor, etc. All these assets will need organising and managing in a disciplined and systematic manner.

The simplest and cheapest way to manage digital assets is to organise your file system appropriately. Here are two basic practices you could apply:

- Establish a naming convention, so that the content of each file is clear from their name

- Establish a coherent folder structure, so that the relation between files within each folder is clear, as is the relation between folders

\begin{question}[subtitle={Activity: Organising your digital assets}] Think of how you currently organise and manage you file system. Write down any rule or convention you usually apply and reflect on the extent it helps you keep track and locate easily the information you need.

\begin{solution}I try to keep my file system well organised and ensure I review it and clean it up on a regular basis. My file system is partitioned between personal and work-related assets. Under each category, I organise assets thematically, using folders and sub-folders structures. For instance, in my work-related folder, I keep a list of my projects organised into current, completed or terminated sub-folders, the latter for things I started but for one reason or other couldn't complete. Under each sub-folder, each project has its own folder, appropriately named with the project name. When folders become to crowded, I tend to refactor them, by creating more sub-folders to group related assets.

You will have come up with a different set of practices. What matters is that you reflect on how effective they are or where improvements may be needed.
\end{solution}\end{question}
%%Hack to correct tcbox behaviour
\color{black}

\subsubsection{Managing document versions}
As you progress in your project, you will generate several versions of parts of your reports, for instance as your literature review grows or of your chosen research methods change based on your growing research knowledge and feedback from your supervisor.

All these versions should be kept and carefully managed to ensure you can always access the latest version, but also refer back easily to previous work should you need to.

Your word processor may already include some functionalities to manage versions. If so, you should make sure you can use them effectively.

Equally there are bespoke \textbf{version control systems} you can use for this purpose, like Git (\href{https://git-scm.com}{https:\slash \slash git-scm.com}), which is free and open source.

However, you can simply use your file system wisely by applying an appropriate versioning convention to keep successive versions of your files in good order.

\begin{question}[subtitle={Activity: Thinking about how you manage document versions}] Think of how you currently manage different versions of your documents or other digital assets. Summarise your practices and reflect on the extent they help you keep track and locate easily different digital versions.

\begin{solution}Part of keeping my file system well organised is to ensure I keep track of the most recent version of any document, but can also locate previous ones easily. While I don't use a separate version control system, I make sure that I keep previous versions of a document I'm working on in a separate folder within my working folder. I use a simple number system to identify versions, that is 1.0, 2.0, 3.0, etc. I only generate a new version when there is a significant departure from the current document, for instance, if I decide to restructure it significantly, or remove a substantial amount of content. I also generate new versions if I share a document with somebody for feedback, while still wanting to carry on working on it.

This is not a perfect system, but it is simple and works for me. You will have come up with a different set of practices. What matters is that you reflect on how effective they are or where improvements may be needed.

\end{solution}\end{question}
%%Hack to correct tcbox behaviour
\color{black}

\subsubsection{Choosing the right word processor}
If you have not written a substantial document, like a dissertation, before you should put some thought into how you are going to write up your project work. Your final dissertation is likely to be between 10,000 and 15,000 words, plus appendices, which makes it a substantial document. By following the advice in this handbook, you will also write reports at each stage, which will build on each other, so that you will be editing, reshaping, extending and re-versioning content you have written as you progress in your project. Therefore, you will need to choose the right tool for the job.

You could start by considering whether any word processors you may be familiar with and possibly used in your previous studies, is appropriate for these requirements. It is possible that you will need to develop more sophisticated word processing skills or use more advanced functionalities of your word processor of choice.

You should also take into consideration the format your course requires you to submit your work in, so you will need to ensure your chosen software can generate documents in that format.

Whichever word processor you choose, you should ensure it allows you to:

- Manage multiple documents

- Structure and process large documents

- Back up your work regularly, possibly automatically, and restore documents which may have been corrupted by system errors or crashes

- Connect to your chosen Bibliographic Management Tool (BMT), so that citations and references can be managed easily and shared automatically between the two

- Generate documents in the formats required for submitting your dissertation and any other required assignment.

\begin{question}[subtitle={Activity: Investigating your word processor's advanced functionalities }] Check the functionalities of your current word processor against the list above. Is it powerful enough to meet the above requirements? Write down the key functionalities you will need to use and make a note of any new skill you will need to develop, then set some time aside to practice them.

\begin{solution}LaTex is my favourite document preparation and typesetting system. It is a very powerful tool, particularly suited to scientific and technical content. It is also free and well supported by a large community of users.

Latex meets all the requirements above. I can use it to structure and process large documents, which I can split up and organise in separate documents: for instance I can have different files for different chapters or different versions of a chapter, which I can then include flexibly in my final document.

The Latex system I use backs up my documents automatically and includes a recovery feature in case of system crash. It also connect with my BMT (I use BibDesk\footnote{\textbackslash{}url\{Insert link\} BibDesk is only available for the Macintosh Operating System}, but could have used Zotero), so that all my citations and references are easy to manage and shared automatically between the two systems.

One of Latex's most powerful features is that it separates output format from source text, so that I can change the style of citations and references, and in fact of the the whole document, very easily and generate outputs in different styles from the same source text.
\end{solution}\end{question}
%%Hack to correct tcbox behaviour
\color{black}

\paragraph{}
Although you might not like what we're about to say, the real Masters pros will use LaTeX, especially if you're writing

\begin{itemize}
\item A mathematical or scientific dissertation

\item A dissertation in which you will have lots of figures or tables

\item A dissertation in which you will be cross-referencing between sections lots

\item A dissertation that has a complex structure

\item A dissertation that has many bullet points or enumerated lists

\item A dissertation with many references and citations

\end{itemize}

You can read and watch why LaTeX is the document preparation system of choice at many blogs, including this one \url{https://blog.orvium.io/latex-over-word/}.

LaTeX has a steep learning curve -- or rather the investment of time that you will use for learning LaTeX will be considerable but it will, if your dissertation has any of the above characteristics, it will save you time and tears in the end, especially when you're making the final touches:

\begin{figure}[htbp]
\centering
\includegraphics[keepaspectratio,width=\textwidth,height=0.75\textheight]{Screenshot2023-05-10at113717.jpg}
\caption{You've got five minutes to submit your dissertation in Word. You want to move that figure a little. What happens to the rest of your dissertation? Source: \url{https://www.reddit.com/r/funny/comments/2glhbp/moving_a_picture_in_microsoft_word/}}
\label{screenshot2023-05-10at113717}
\end{figure}

Currently the best tool for writing LaTeX is \href{http://Overleaf.com}{Overleaf.com}\footnote{\href{http://Overleaf.com}{http:\slash \slash Overleaf.com}}. It's the best because it:

\begin{itemize}
\item Has all the power of LaTeX

\item Is set up for research work

\item Has a fantastic help system

\item Has many, many templates -- and may even have one customised for your university

\item Can be used to track changes between versions

\item Backs up your work

\item Works through a browser, and so will be available on all*\footnote{Probably not your calculator, and may be difficult to see on your smartphone.} your devices.

\end{itemize}

LaTeX also has a Word-like interface, through the LyX tool, which can reduce the learning curve. If you find it easier to use a Word-like interface, but want the benefits of LaTeX, you might like to try LyX.\footnote{LyX is available at \textbackslash{}url\{lyx\}.}

\section{Reflection on learning}
It is time to reflect on what you have learnt so far. Reflection is what makes your learning more effective, and relevant and useful to your own practice, so you shouldn't be surprised that reflection is a common theme in this handbook. You have already been asked to reflect in the practical activities you have carried out so far. In this closing section, you are asked to reflect of what you have learnt in this chapter as a whole.

\begin{question}[subtitle={Activity -- Reflection on learning}] Consider the content of this chapter and write down key things you have learnt or that have surprised you. For each indicate why they are notable or relevant to you, and how you may apply them in new situations or to inform your future learning.

\begin{guidance}tbd
\end{guidance}\end{question}
%%Hack to correct tcbox behaviour
\color{black}

\chapter{The 5-stage Masters project framework}
This handbook gives you a 5-stage framework with which to approach your research project. The framework has been refined over many years of working with hundreds of Masters students at the Open University\footnote{If you know anything about the Open University, you`ll know that it is a distance--learning university, most of its students work at a distance and so don't attend lectures. If you think about it, the ``at--a--distance'' model is precisely what you`ll be doing in your Masters project -- you won't have lectures! This makes the Open University model really appropriate for a Master project.}, UK.

An overview of the framework is provided in this chapter --- we will break it down into stages in the following chapters, each chapter giving detailed guidance for that stage.

\section{What do we mean by framework?}
Writing a Master dissertation is a complex task: the goal is a complete 10,000--15,000 word document that is going to satisfy the examiners. There are many risks that you will face in writing your dissertation, which include:

\begin{itemize}
\item Misunderstanding what is required

\item Running out of time

\item Not having the skills or resources you need

\item Choosing the wrong methods or techniques, or not applying them correctly

\item {\ldots}

\end{itemize}

The framework we give you will allow you to manage these risks to give you the best chance you can possibly have of submitting something that will satisfy the examiners and show your Masters research skills off in their best light.

The framework gives recommended stages and research activities, as well as metrics and guidelines to help you parameterise your project based on your course requirements, derive a work plan that meets your needs, and manage your interaction with your supervisor effectively.

Study within our framework and you'll have your best chance of succeeding.

Because the key research requirement is for new knowledge, every Masters student will be working at the leading edge. One corollary of this is that every dissertation will be different. That's not necessarily the case for a taught Master's module, where there may be a single correct answer.

Your project will allow you to show how your domain expertise in a focussed topic has developed through your studies and your dissertation will reflect that. That's your goal!

For us in defining our framework, it raises a problem -- we cannot possibly know the details of your final dissertation. We can't know even simple things, like how many pages it will have exactly! We can't know what your arguments will be or what your conclusions or future work will contain. Neither can we know which literature you will consult, what the seminal paper in it is.

That would be a problem if our focus was only on the final form your dissertation will take. But we don't!

What we teach in this handbook is the process of arriving at your final dissertation. That's why we structure our framework into stages -- each stage building on the previous, each moving you further down the line to the final dissertation.

And because we have seen literally hundreds of students follow the framework as we have developed it, we know it can work and work well!

If this has convinced you that we're onto something, welcome aboard!

Your first step with us will be to learn about the research process that you will follow.

\section{The research process and its key activities}
A research process is the sequence of activities you undertake when conducting academic research.

Research is messy --- (literally!) looking into the unknown means there is no road map to follow and you will often have to retrace your steps and try different paths. So, moderate your expectations -- you shouldn't expect a linear, orderly research process!

Instead, the research process is an iterative, incremental, and adaptive process of knowledge discovery --- it iterates through its research activities in small incremental steps, adapting as your knowledge grows.

The exact path you will follow in your research project will be unique to you. However, you will go through some widely recognised activities, which are common to all research processes.

\subsubsection{Identifying the research problem}
A key concept in academic research is that of \emph{research problem}. A research problem captures the knowledge gap to be addressed (\emph{the need}) by the research within a particular field of study (\emph{the context}). A well-defined research problem is the foundation of any research project as it clarifies the purpose of the research and its intended outcomes.

In this handbook, you will learn a practical approach to identifying and articulating your chosen research project.

\subsubsection{Reviewing the literature}
One of the major tasks you will have in compiling your dissertation is to give the reader sufficient background that they can check where you're coming from: you will be referring to the work of other authors and researchers throughout your project: this is collectively known as the academic literature. Depending on your degree, you may already have practise in doing this in your previous studies, for instance in writing academic essays for your assignments. For your Masters, however, the more definitive you can be in referencing the academic literature the better. This means that your review of the academic literature will be a significant part of your research project.

Your review has two key functions.

Firstly, it will help you contextualise your research in what is already known and where the knowledge gaps are, so that you can focus your research more tightly on a research problem which is relevant and significant to your field of study. This includes evaluating your own findings against those of others in your field.

Secondly, it will help understand how to conduct your research: by reading your field's literature, you'll learn what is academically acceptable in terms of the methods and techniques which you can apply to address your research problem.

Reviewing the literature is both time consuming and demanding, and will require you to apply many skills which this handbook will help you develop.

\subsubsection{Setting your aim and objectives}
While your research problem helps you establish the contribution to knowledge of your research, your aim and objectives will help you establish the scope and boundaries of your project by stating the specific way in which your research will address the problem.

Scoping your project appropriately is necessary in all research, but is particularly critical at Masters level with its many constraints on the time you have and the methods you can feasibly apply.

\subsubsection{Developing the research design}
Your research design will summarise, explain and justify how your research is conducted for the benefit of your examiner and other readers\footnote{Although we all want to think differently, it is often the case that the examiner is the only person outside of your supervisory relationship that will read your dissertation. We'll talk later about how you can get your family involved -- which may double your readership!}. As with your literature review, it will develop during your project: at the start, it will be a collection of your initial ideas; by the end, it will be an account of what you did.

Your research design will depend on many things. There are some obvious ones, including the type of research problem you are trying to address, the intended outcome of your research, the sort of evidence you will need, and the research strategies and methods that are acceptable within your chosen discipine.

There are others that are less obvious: you'll have to tailor your research design to the resources and expertise you easily have access to, and your preferred research style\footnote{You may have heard of a \emph{learning style}. A research style is similar -- you may prefer, for instance, to talk to people about the practical ways they work as opposed to building theories of how they do so.}. As a key participant in your own research, your own personal views and values will also affect your choices while developing your research design.

There are also some esoteric ones: some philosophical beliefs which are shared by researchers in your field in regard of how knowledge can be generated. Although fascinating, to really understand them well takes decades of study of others' writings, on top of which there will be hours on hours of conversations with other researchers. Clearly, this is not something you will have much time to dwell upon in your Masters project --- you have a supervisor that can advice you, instead.

Research design is a field of study in its own right, one which has grown out of the many different academic traditions and ways of thinking across academic disciplines and subject areas. It's not easy to digest and is far from stable or complete: every so often a new book on research methods will be offered for review by a publisher. Research design is possibly one of the most challenging aspects of doing academic research and can be puzzling\footnote{It can sometimes be puzzling for experienced researchers too!} for those just starting.

Because of this, you should rely on your supervisor for advice, particularly in your initial research design, although your understanding will mature during your project. This handbook will help you develop such an understanding and assess which choices are most appropriate for your project.

\subsubsection{Gathering and analysing evidence}
This is where you use your research design to gather and analyse the data and evidence\footnote{Data is raw information with no interpretation attached, while evidence is information interpreted to support your academic arguments. The two are closely linked, with data forming the basis of evidence, so that they are often used interchangeably. We will return to this topic in Section ??} you need for your project. This is possibly the most exciting part of conducting academic research!

You're going to spend a lot of time doing this, and you'll have to circle back\footnote{One PhD student we know got to within 2 months of finishing their PhD when they realised that their research design led them to incorrect data analysis. That was a real critical moment in their studies, but they got it done in the end. Just.} to this activity as you learn more about your field and as your research develops.

\subsubsection{Interpreting and evaluating findings}
This is where you review your findings critically to establish the extent your research has met its stated aim and objectives. Because it relies on the way you interpret your evidence, it is closely related to and typically influences evidence gathering and analysis, so that you will iterate between these two research activities for a large proportion of your project.

\subsubsection{Reporting}
Throughout your project you will need to report in a critical\footnote{Developing your critical voice is, er, critical to being a successful researcher. On a good capstone project, an effective critical voice will be highly valued by your examiners.} and rigorous manner on all aspects of your research, including both new insights and the process you followed to arrived at them. Therefore, integral to your reporting is a critical appraisal of the strengths and limitations of the research you have conducted, the overall conclusions which can be drawn from it and what the impact on future research might be.

While your course may only require you to submit a final dissertation for assessment, it is essential that you write up reports throughout your project. This will allow you to improve your academic writing skills, share what you have done with your supervisor for feedback and formative assessment, and develop your dissertation incrementally.

\subsubsection{Reflecting}
We have already mentioned how reflection is what makes your learning more effective, and relevant and useful to your own practice. Reflection is important in any kind of learning, but particularly in the experiential learning\footnote{Experiential learning is `learning by doing' and using reflection to think critically about what you have experienced and relate it to knowledge and how it can apply to new situations.} of conducting research.

Our framework therefore emphasises reflection as an essential activity within the research process, to develop and consolidate the knowledge and skills you need to be a competent and confident researchers.

\subsubsection{Planning work}
Some researches will have worked much of their lives on one problem. You don't have that luxury -- or burden, depending on your perspective!

Time-bound research, like that you are just about to start, must be planned carefully as early mistakes and missteps can be very difficult to correct later.

For this reason, we recommend that you build a \emph{work plan}. At its core, a work plan summarises all activities and work to be conducted to complete the research, how they will be organised and executed in the time-span of the project and which milestones must be met.

Standard project management techniques apply, but with the caveat that in academic research you are usually planning for the unknown, so not everything can be figured out upfront and some level of adaptation will be needed. Hence, understanding what might change, related risk and how to manage it is an essential part of work planning.

\subsubsection{Managing risk}
Risk captures the likelihood of something going wrong combined with the impact that will have on your project, both on time, resources and outcome. In theory, both positive and negative impacts can be assessed, but very often the focus is on what can affect your project in a bad way. It is essential that you make an assessment of risk at the start of your project and identify ways to manage it, then monitor and adjust as you go along.

\subsubsection{How the activities relate to each other}
Figure 1 depicts the relations between activities in the research process. Note that reflecting, reporting, work planning and risk management apply to both individual activities, and the process overall, which is why they are separated from the other activities in the figure.

\begin{figure}[htbp]
\centering
\includegraphics[width=898pt,height=474pt]{PastedGraphic.pdf}
\caption{}
\label{pastedgraphic}
\end{figure}

Figure 1 - Research activities and their relations

The main relations between activities are as follows (with reference to the numbering in Figure 1):

1	The research problem helps you identify relevant literature to review

2	The literature helps you identify current knowledge and existing gaps and frame the research problem

3	The research problem informs what aim and objectives should be

4	The research problem identifies the phenomena under study, which inform the choice of research strategy and methods

5	The literatures help you establish which research strategy and methods to apply and how

6	The objectives inform the research strategy and methods to discharge them

7	The research designs tells you how to gather and analyse evidence

8	The analyse evidence feeds your interpretation and evaluation

9	Your interpretation and evaluation may require further evidence gathering and analysis

10	Your interpretation and evaluation assesses the extent aim and objectives are discharged

11	Aim and objectives gives you criteria for your evaluation and interpretation

12	Activities feed your project plan

13 Your work plan inform what you have to do and when

14 Activities give rise to risk to be managed

15 Managing risk will constrain the way you carry out activities

16 Activities and their outputs must be documented in your reports

17 Your reflection on all you do will trigger changes and adjustment to all your activities

The figure highlights the interconnected nature of everything you do during a research project, and how the need to revisit some activities, perhaps has the result of reflection, will trigger adjustments to others which are closely related, so that there is no linear path through that all project will follow. The figure also does not assume there is a set starting point, although, as you will see in the next chapter, identifying the research problem from an initial topic is the way we recommend you get your project started. Equally, research can go on forever as reflecting of what you know may trigger new avenues of enquiry to follow. Of course, in the context of a Masters project you only have a set amount of time to complete your work, so that at some point you will need to decide that you have done enough and finalise your project dissertation.

\section{The 5-stage framework for your research}
This handbook recommends you organise your research project into five major stages. We've developed our 5-stage framework from our experience of working with hundreds of successful\footnote{And a few, a very few, unsuccessful ones{\ldots}} Masters research students, our knowledge of the iterative, incremental and adaptive nature of the research process, and our awareness of the many risks that must be managed to be successful.

Given this, the stages have the following characteristics:

\begin{itemize}
\item Each stage contains many interrelated research activities\footnote{Don`t worry, we're going to go into exhaustive detail about this the remainder of the book!}, so that you will often revisit earlier activities as you learn more about research and move forward in your project. From one stage to the next, however, the balance between those activities will change reflecting your increasing expertise in them and the progress you have made

\item Each stage builds incrementally and adaptively on work from the previous stage(s)

\item Each stage includes critical reflection on what you have achieved and learned, and how this should inform the work ahead

\item Each stage includes some report writing so that your dissertation builds with your expertise

\item Each stage includes re-assessing risk and adapting your work plan.

\end{itemize}

Because it sets up the whole research project, in Stage 1, alongside developing a deeper knowledge of the research process and its activities, you will do lots of prep, including identifying your project's scope and contribution to knowledge, doing an initial risk assessment, outlining an initial work plan, and setting up the relationship with your supervisor.

Given a successful set up of your research in Stage 1, from Stage 2 onwards you'll spend more time on the specifics of your research and less about the process\footnote{Which will probably come as welcome relief:)}. Even so, in each stage, you will need to review your project risk and progress, adjusting your project plan accordingly.

The 5 stages are designed to balance the activities you need to carry out in your project; each is allocated a recommended proportion of the overall project time. The exact balance is influenced by many factors, including topic, any previous research experience and your supervisor's advice, so take the next figure with a pinch of salt -- it's only meant to show how the focus on research activities changes in the stages. We will consider a more detailed description in the next section, including indicative timings, but again, nothing is written in stone.

\begin{figure}[htbp]
\centering
\includegraphics[width=868pt,height=564pt]{Screenshot2023-06-01at124221.pdf}
\caption{Figure 1 --- Change of focus between research activities across the project stages}
\label{screenshot2023-06-01at124221}
\end{figure}

\begin{question}[subtitle={Activity: Considering the stages and activities in the framework }] Consider Figure 1 carefully, taking notice of which activities are more prominent in each stage, and the relative length of each stage within the framework. Jot down a timeline for your project based on your expected study length in weeks, identifying when each stage starts and ends.

\begin{solution}Some activities are more prominent in the early stages, specifically identifying the research problem and reviewing the literature. Others become prominent in later stages, particularly gathering and analysing evidence, and interpreting and evaluating findings. Others still feature throughout the project: this is particularly the case for reflecting and reporting.

Your timeline will be specific to your project, of course. Say, you have 48 weeks in total. Then, Stage 1 will take 15\% of 48 weeks, or just over 7 weeks, Stage 2 will take the same length, Stage 3 will take 20\% of 48 weeks, or just under 10 weeks, etc.

\end{solution}\end{question}
%%Hack to correct tcbox behaviour
\color{black}

\subsubsection{Activity breakdown}
Table 1 gives you the recommended breakdown of activities between and within stages.

This is designed to help organise and plan your project work to match the requirements of your specific Masters course, something you will consider as part of work planning in the next chapter, which focuses on Stage 1.

Table 1 also provides recommendations on critical aspects of your work you should discuss with your supervisor stage by stage. We have already discussed the importance to your project success of a close working relationship with your supervisor, of engaging in an active dialogue on a regular basis, and your responsibility to drive and manage that interaction. In practice, your supervisor will only have limited time to dedicate to your project, therefore it is essential you make an effective use of that time: the advice in the table will help you guide your interaction with your supervisor, ask key questions and make the time you spend together more productive.

Table 1 --- Recommended breakdown of project work between and within stages, alongside suggested focus of interaction with supervisor

\begin{ltabulary}{1.4\textwidth}{@{}LLLLLLLLLLLLL@{}}
\textbf{Research activities} & \textbf{Stage 1~} & \textbf{Suggested focus of your interaction with your supervisor} & \textbf{Stage 2~} & \textbf{Suggested focus of your interaction with your supervisor} & \textbf{Stage 3~} & \textbf{Suggested focus of your interaction with your supervisor} & \textbf{Stage 4} & \textbf{Suggested focus of your interaction with your supervisor} & \textbf{Stage 5} & \textbf{Suggested focus of your interaction with your supervisor} \\
 &  & \textbf{Identifying the research problem} & Develop problem statement and intended contribution to knowledge & Suitability of research problem for academic research and to meet the requirements of specific~ course & Adjust, if needed &

 & Adjust, if needed &

 & Adjust, if needed &

 & Adjust, if needed &

 \\
 &  & \textbf{Reviewing the literature} & Compile initial draft of literature review and plan remaining review & Scope of literature review and possible gaps & Complete full draft of literature review
Draft critical summary of key insights from literature review & Suitability of literature review structure and narrative flow
Appropriate logical argumentation
Critical thinking in driving insights & Adjust, if needed &

 & Adjust, if needed &

 & Adjust, if needed &

 \\
 &  & \textbf{Setting research aim and objectives} & Define aim and~ objectives & Suitability and feasibility of aim and objectives in relation to research problem and project time & Adjust, if needed &

 & Finalise aim and objectives, and define tasks and deliverables & Suitability of tasks and deliverables from objectives & Adjust, if needed &

 & Adjust, if needed &

 \\
 &  & \textbf{Developing the research design} &
Consider elements of~ research design, including data and evidence, and types of research methods

Complete ethics assessment, including applying for permission to proceed, if needed
 &
Consistency of choices in relation to aim and objectives.

Compliance with own university's ethical and legal guidelines
 & Revise research design, with detailed consideration of data and evidence, research strategy and research methods & Any further adjustment needed to the research design & Complete research design, with detailed consideration of data and evidence, research strategy, research methods and procedures & Suitability of research procedures & Adjust, if needed &

 & Adjust, if needed &

 \\
 &  & \textbf{Gathering and analysing evidence} & n/a &

 & n/a &

 & Conduct pilot work to test aspects of your research design & Scope of your pilot work & Conduct initial data/evidence generation and analysis & Initial application of collection and analysis methods, and any improvements required~ & Complete data/evidence generation and analysis & Overall quality and quantity of data/evidence and their analysis \\
 &  & \textbf{Interpreting and evaluating findings} & n/a &

 & n/a &

 & n/a &

 & Critically assess findings up to this point & Critical thinking in assessing findings, and any improvements required~ & Critically assess all findings and the overall research conducted & Overall critical thinking in assessing findings \\
 &  & \textbf{Reporting} & Write up Stage 1 report & Demonstration of critical thinking and good academic writing, and any improvements required & Write up Stage 2 report & Any further improvements required & Write up Stage 3 report & Any further improvements required & Write up Stage 4 report & Any further improvements required & Write up dissertation & Overall quality of dissertation \\
 &  & \textbf{Reflecting} & At stage end, think critically about experiential learning in relation to the research process and its activities &

 & At stage end, think critically about experiential learning in relation to the research process and its activities &

 & At stage end, think critically about experiential learning in relation to the research process and its activities &

 & At stage end, think critically about experiential learning in relation to the research process and its activities &

 & At stage end, think critically about experiential learning in relation to the research process and its activities & Overall critical thinking in assessing experiential learning \\
 &  & \textbf{Planning work} & Draw your initial project timeline and a detailed plan for Stage 1 & Appropriateness of initial work plan & At stage start, plan Stage 2 work in detail, including any revision or additional work needed from the previous stage. & Any major adjustment required & At stage start, plan Stage 3 work in detail, including any revision or additional work needed from the previous stage. & Any major adjustment required & At stage start, plan Stage 4 work in detail, including any revision or additional work needed from the previous stage. & Any major adjustment required & At stage start, plan Stage 5 work in detail, including any revision or additional work needed from the previous stage. & Any major adjustment required \\
 &  & \textbf{Managing risk} & Assess project risk & Consideration of major risk & Review project risk and make any required adjustments for the next stage & Any major adjustment required & Review project risk and make any required adjustments for the next stage & Any major adjustment required & Review project risk and make any required adjustments for the next stage & Any major adjustment required &  &
\end{ltabulary}

\section{Critical success factors}

We conclude this chapter by summarising critical success factors you should keep in mind during your Masters research:

\textbf{Making good use of the 5-stage framework} --- The framework in this handbook is the result of decade of practice, helping Masters students like you succeeding in they first academic project. However, it is not a straightjacket, and you should adapt it to your own needs. The framework was developed with the novice academic researcher in mind, but it may be that you already have some research experience, in which case some of the the timing suggested by the framework may be too generous. Equally, your own course may require you to submit some interim report as formative assessment, in which case the stages length may need altering to match your course requirements. Nevertheless, the framework provides a scaffolding to help you take control when planning and conducting your project.

\textbf{Making good use of this handbook and its activities} --- This handbook is designed to accompany you in your first journey into academic research, so you should follow its stage by stage advice to guide your project work. The handbook is also designed to be very practical, so that there are plenty of activities for you to do. Most of the activities are there to help you make steady progress with essential work for your project: we have indicated them as XXXX and we strongly advise you complete them systematically. Others are designed to help you develop your research skills, so you should use them at your own discretion, although we strongly advise you at least consider them as you go along: those activities are marked as XXXX.

\textbf{Your self-drive and commitment} --- The course you are studying may provide you with some structure to help your progress your project, including, for instance formal assessment points. However, you will be in the driving seat most of the time. You must choose your topic and research problem, how to investigate it, and how to organise your research time in detail; it is also up to you to understand the academic standards required and to assess which research skills you must develop further to meet them. Above all, it is up to you to commit to a sustained effort for the duration of your project, which may be several months.

\textbf{Continuous effort} --- Successful research requires continuity, so that you will need to set aside a sufficient and regular time for your research project every week, ensuring you keep making progress as you go along. Long breaks are incompatible with conducting research: while in your previous studies you may have been able to stop and start, and possibly cram lots of work around assignment deadlines, conducting research requires lots of time for reflection and for ideas to develop and mature, something you can't compress close to a deadline. Conducting research is a marathon, not a sprint! It requires endurance, perseverance and continuous effort.

\textbf{Your working relation with your supervisor} --- Your supervisor is your strongest study ally, a research expert who can guide and advise you on all aspects of your project, with whom you can discuss your research ideas, and who can assess that you are making sufficient progress at each point in your journey. It is essential that you develop a good working relationship with your supervisor, meet regularly and have an open and honest dialogue throughout your project.

\textbf{Thinking and writing} --- There is a crucial interaction between reading, thinking, and writing in research: reading informs your thinking; your thinking is what you try to express in your writing; your writing helps you make sense of what you have read, and hence of what you think, and informs more reading, thinking and writing. Because of this, while you will spend a lot of time reading and thinking, it is also essential that you write as you go along. The more regularly you write, the easier it will be for you to develop academic writing skills and the more critical your thinking will be.

\textbf{Your dissertation} --- It is very likely that your dissertation will be the only way your research is assessed. To be successful, your dissertation must demonstrate that you have mastered a wide range of research skills and can communicate your academic research effectively in writing. In fact, this is usually more important than any feature your proposed solution to your research problem might have. At Masters level, that you `solve' your research problem may not even be necessary! What matters most is your scholarly and critical attitude to each element of your research and your grasp of what academic research entails as demonstrated by your written dissertation.

\textbf{Making good use of your study support} --- Last but not least, it is important that you assess your progress as you go along, and make the most of the support which is provided to you by your supervisor, and any other adviser who may be made available to you as part of your course. There will be times when you will find working on your project very challenging and may lose confidence in your ability to complete it. Those are the times when it is going to be particularly important for you to reach out and ask for extra help, even if your first instinct may be to hide. All researchers experience such feelings at one time or another, so do not be discouraged: talk to an adviser and work through the difficulties you are having to overcome them.

\section{Reflection on learning}
We close this chapter with some reflection on what you have learnt.

\begin{question}[subtitle={Activity -- Reflection on learning}] Consider the content of this chapter and write down key things you have learnt or that have surprised you. For each indicate why they are notable or relevant to you, and how you may apply them in new situations or to inform your future learning.

\begin{guidance}tbd
\end{guidance}\end{question}
%%Hack to correct tcbox behaviour
\color{black}

\chapter{Stage 1: Preparing your research proposal}
At some point -- not too far from starting your project! -- you'll have completed a well articulated research proposal. As you'll see below, a good research proposal has many characteristics and these make it a difficult document to get right.

Starting your proposal and getting it into a workable state is the focus of this chapter: Stage 1 -- Preparing your research proposal.

\section{What is a research proposal?}
A research proposal is simply a detailed outline of your intended research project. It'll be 8 to 10 pages long and its main focus will be to convince the reader -- your supervisor, your examiner, and perhaps any family members or friends that help you -- that a research problem\footnote{Remember that research is a quest for new knowledge. Your research proposal must be convincing that current knowledge doesn't already have this covered.} actually exists and you have some initial ideas of how to approach it.

To be able to do this, your research proposal must be appropriately contextualised in the academic literature\footnote{Together, these give answers to the ``What?'', ``Why'', and ``How?'' questions that motivate your research.} and, if relevant, also in professional practice, and should identify the knowledge gap you are going to address. It should also include some early considerations of the intended research design, together with an initial risk assessment.

The research proposal is \emph{your} understanding of where you want your research to go.

\subsubsection{Do I really have to do a research proposal?}
It may be that you have chosen to research a topic proposed by your supervisor(s) which offers a perfectly good research proposal associated with it.

It might even be that you've chosen the topic because you know that there's already a perfectly good research proposal written!

Even if a research proposal is not required in your studies, we strongly recommend you still go through the stages of developing one, even if it's just to rewrite a pre-existing one in your own language.

There are some very good reasons to do this:

\begin{itemize}
\item It will provide a foundation for your project that you understand, making it more solid

\item It will help you clarify your initial thinking --- which you're going to have to do anyway

\item It will show your supervisor that you have something distinct to contribute to the research

\item It will force you to think about the risks you face and how you will handle them -- again, you're going to have to do this anyway

\item Most importantly, it will then be yours -- you'll take ownership of your own research.

\end{itemize}

So, our recommendation is that you develop your own research proposal even if one exists as it will help you think deeply about what you are about to embark on.

\subsubsection{Welcome to Stage 1!}
Stage 1 is fundamental to your projects as it gives you the solid foundation for the whole of your research. As such it is an intense project stages, where you will be expected to undertake a wide range of research activities. You can have an early peek at the activities that we recommend in Table $<$\$n\#table$>$, which also provides some guidelines for your interaction with your supervisor during this stage.

Table 1 Stage 1 activities, including their relative effort

\begin{ltabulary}{1.3\textwidth}{lllllllllllll}
\textbf{Research activities} & \textbf{Stage 1~(15\% of project length)} & \textbf{Effort within stage} & \textbf{Suggested focus of your interaction with your supervisor} &  &  &  &  &  &  &  &  &  \endfirsthead
\textbf{Identifying the research problem} & Develop problem statement and intended contribution to knowledge & 15\% & Suitability of research problem for academic research and to meet the requirements of your specific course &  &  &  &  &  &  &  &  &  \\
\textbf{Reviewing the literature} & Compile initial draft of literature review and plan remaining review & 30\% & Scope of literature review and possible gaps &  &  &  &  &  &  &  &  &  \\
\textbf{Setting research aim and objectives} & Define aim and~ objectives & 5\% & Suitability and feasibility of aim and objectives in relation to research problem and project timeline &  &  &  &  &  &  &  &  &  \\
\textbf{Developing the research design} &
Consider elements of~ research design, including data and evidence, and types of research methods

Complete ethics assessment, including permission to proceed, if needed
 & 10\% &
Consistency of choices in relation to aim and objectives.

Compliance with own university's ethical and legal guidelines
 &  &  &  &  &  &  &  &  &  \\
\textbf{Gathering and analysing evidence} & n/a & 0\% &

 &  &  &  &  &  &  &  &  &  \\
\textbf{Interpreting and evaluating findings} & n/a & 0\% &

 &  &  &  &  &  &  &  &  &  \\
\textbf{Reporting} & Write up Stage 1 report & 20\% & Demonstration of critical thinking and good academic writing, and any improvements required &  &  &  &  &  &  &  &  &  \\
\textbf{Reflecting} & At stage end, think critically about experiential learning in relation to the research process and its activities & 10\% &

 &  &  &  &  &  &  &  &  &  \\
\textbf{Planning work} & Draw your initial project timeline and a detailed plan for Stage 1 & 5\% & Appropriateness of initial work plan &  &  &  &  &  &  &  &  &  \\
\textbf{Managing risk} & Assess project risk & 5\% & Consideration of major risk &  &  &  &  &  &  &  &  &  \\
\end{ltabulary}

You'll notice that Table ? has entries for each of the eight research activities that were introduced in Section ??? (Figure 2.1?) and, from the \% column, you'll notice that 7 of the 9 activities are covered in Stage 1. We did say that Stage 1 was intense!

Of the 7 activities covered in this stage, \textbf{Reviewing the literature} to compile an initial draft for your dissertation is the most demanding at 30\%, closely followed by \textbf{Identifying the research problem} at 15\% and \textbf{Reporting} at 20\%, the latter consisting in putting your research proposal together after you have developed its constituent parts.

Together, reviewing and literature and identifying the research problems are your first steps to understanding what your contribution to knowledge will be and so is good to invest a lot of time in these. Setting aim and objectives (5\%) and starting to develop your research design (10\%) follow from them.

We have already mentioned the importance of reflection in research, and how it is essential for you to reflect on your increasing understanding of the research process and your own research practice as you go along. Unsurprisingly, therefore, the framework accounts for 10\% of your time spent doing just that.

Equally, we have also stressed the importance of considering risk in your research and to manage your time effectively, therefore 5\% of your time should be devoted each to risk management and to work planning.

In this chapter, you will start by thinking about how you will organise your work in this stage, then look at the other activities in turn.

\section{Planning your work for Stage 1}
As a Masters researcher, you will be expected to manage your project to a large extent, including planning your project work in some detail, and monitoring your progress and making appropriate adjustments to your plan as you go along.

Key activities for project planning and to keep on track include:

- Identify key milestones for your project, and what you are required to accomplish or deliver at each milestone

- Identify the work you will need complete to reach each milestone

- Break down that work into discrete tasks

- Schedule those tasks in the time available

- Making an efficient use of the time you have available.

As an important caveat, you should not expect your plan to be cast in stone! There are too many unknowns in a research project for you to be able to predict upfront exactly what is going to happen. Therefore, you shouldn't spend too much time trying to plan every single thing, and you should review your plan often to monitor progress and make adjustments as needed. In summary, your project plan should keep you on track, but should not be a straight-jacket.

In this section, you will use the 5-stage framework as a tool to help you put together an initial plan your project work, focusing on the activities required in Stage 1. You will extend this plan, stage by stage, adding details and making necessary adjustments as you go along.

\begin{question}[subtitle={Activity: Breaking down your study time}] Consider the timeline for your project that you defined as part of Activity XX in the previous chapter. Based on the number of weeks you allocated to Stage 1 in your timeline and the number of study hours per week at your disposal, use the percentages in Table XX to calculate the actual study time for each of its activities.

\begin{guidance}Assuming you have allocated 6 weeks to Stage 1, with an average of 20 hours of study per week, then you have a total of 6 x 20 = 120 hours of study for the whole stage. Of these, you will need to spend 10 hours on identifying your research problem (15\%), 36 hours on reviewing the literature (30\%), 6 hours on setting your aim and objectives (5\%), etc., including studying the relevant parts of this handbook and carrying out the activities within.
\end{guidance}\end{question}
%%Hack to correct tcbox behaviour
\color{black}

\subsubsection{Milestones, deliverables and tasks}

The 5-stage framework identifies for you key milestones and deliverables at a general level: each stage is broken down into research activities, each with clear deliverables, and a written report on the work completed is expected by the end of each stage. Together, these are your main milestones and deliverables.

Within each stage there is much detailed work to be done to reach such milestones. Therefore, at the beginning of each stage, you will be encouraged to plan such work carefully, by matching the generic activities in the framework to specific tasks in your project.

\begin{question}[subtitle={Activity: Identifying your project tasks}] Consider the generic activities in Table XX and write down how they correspond to tasks in your own project. Allocate time to those tasks based on your study time estimate in our previous activity.

\begin{guidance}Feel free to break tasks down into sub-tasks, but be wary of your plan becoming too detailed at this point.
\end{guidance}\end{question}
%%Hack to correct tcbox behaviour
\color{black}

\subsubsection{Producing a project plan for Stage 1}
You can now produce a project plan based on the milestones, deliverable, tasks and their required time you have identified.

This is no different from any other kind of project, so that many planning charts and tools are available which you can choose; in fact, you may be already familiar with some of them. Gantt charts are one of them, which we recommend you use.

Briefly, \textbf{Gantt charts} are scheduling charts that you can use to plan and organise your project work. A Gantt chart is usefully to get an overview of how your work will be broken down and organised over time, including an indication of how much time you will spend on each task, when tasks should start and end, and which tasks might overlap at any point in your project. As such, it is also useful to communicate your project plan to third parties. As a generic project management tool, a Gantt chart can be quite sketchy or very detailed: you should aim for something quite light, but still including all main tasks and deliverables of your project.

\begin{question}[subtitle={Activity: Investigating Gantt charts}] If you unfamiliar with Gantt charts, conduct a web search on introductory materials and examples. Select and review a small subset of resources you have found. Write down key points about using Gantt charts.

\begin{guidance}In your web search , you'll likely find several links to digital tools supporting Gantt charts, to tutorials on how to develop one, and even to spreadsheet templates for download. You should focus on materials which can help you construct your own project Gantt chart.
\end{guidance}\end{question}
%%Hack to correct tcbox behaviour
\color{black}

\begin{question}[subtitle={Activity: Constructing your Gantt chart}] Based on the outcome of your previous activities, create a Gantt chart for your own project in relation to Stage 1.

\begin{guidance}If you already use a different project scheduling approach for projects in your own practice, then feel free to use that instead, as long as it can be used to organise your project work effectively and in a way which is easy to communicate with your supervisor, with whom you should share and discuss your plan.

You will augment and adjust your plan in all stages of your project.
\end{guidance}\end{question}
%%Hack to correct tcbox behaviour
\color{black}

\begin{tip}{Top tips for managing your time efficiently}Top tips for managing your time efficiently ****Your plan is more likely to work if you make an effective use of your time. Here is some key practices you should keep in mind:

\textbf{Finding time for your research} --- you need to make a realistic assessment of how much time you will have for your project, considering other commitments you may have, wether professional and personal: this is crucial for part-time study. You course will expect you to spend an average number of hours per week on your project work throughout its duration, so you need to ensure you can dedicate that time to your project on a regular basis. But it is not just about quantity: you should choose your most productive time to dedicate to your research--- in may be that you are a morning person, or you can focus better at night. In either case, it is important you come to your work with a fresh mind, so that you can concentrate on the tasks at hand.

\textbf{Ensuring continuity of effort} --- research requires continuity of effort. While in previous studies you may have been able to start and stop, and possibly `cram' much of the work around assignment deadlines, in your project you need the time to develop your critical thinking and other research skills, and that's a continuous process. If you fall behind, you may find it difficult to catch-up. For your project, you must therefore ensure that your study time is arranged evenly and regularly over its duration and that any study break you may take is short, not to affect your continuous progress.

\textbf{Making your spare time productive} --- there will be much to read through your project, so you should always keep something to read with you, which you can look at while waiting for something, say a bus, or in a queue or travelling on public transports. You may be surprised of how much reading you can do while waiting.

\textbf{Avoiding postponing and procrastinating} --- there will be tasks you may find harder than others, so it would be natural to put them off or engage in displacement activities. You should avoid that and focus on what must be done to meet your project milestones.

\textbf{Scheduling extra time} --- inevitably things don't always go to plan or activities may take longer than estimated, so you should always factor in extra time for the things that may go wrong. If you find you don't need it, you can always allocate the time to other tasks or take a longer break, but at least you won't fall behind in subsequent tasks.

\textbf{Being adaptive} --- research is about looking into the unknown, so you can't expect to be able to plan everything upfront and in great detail. Your planning should keep you on track to reach your major milestones, but should not be a straightjacket. Don't spent too much time trying to plan every single thing to do: keep your planning light and ensure you return to your plan often to make adjustments as your project unfolds.

\textbf{Investing in the right tools for the job} --- while you may find it a nuisance to have to spend time learning new tools, such as a Bibliographic Management Tool (BMT) or some advanced word processing features, doing so will save you time in the long run, so make sure you invest in setting up your new systems and learning new skills earlier on in your project.\end{tip}

\begin{question}[subtitle={Activity: Reflecting on your study practices}] Assess on your own study habits and practices in light of the above advice. Write down things you could or should change for your research project.

\begin{guidance}You should take a balance stance and consider both on your current strengths and weaknesses: build on the former and put some effort in addressing the latter.
\end{guidance}\end{question}
%%Hack to correct tcbox behaviour
\color{black}

\section{Identifying the research problem}
A research project gives you an opportunity to carry out a focused piece of academic research in the subject area of your degree and on a topic of your own choice.

It's a complex thing to do, but we'll walk you through it! You've got this!

\subsubsection{Choosing a topic for your project}
There is no single way to get started and inspiration can come from many places, from previous studies, any professional experience you may have, articles you have read, or even suggestions from your supervisor.

\begin{question}[subtitle={Activity: Possible topics to investigate}] Write down possible topics to investigate for your research project.

\begin{guidance}Think back\footnote{This might be your opportunity to find out more about them through research.} over your studies to this point. Did anything stick in your mind as very interesting, something you'd like to return to? What study topics did you particularly enjoy?

Next, think about what might be interesting within your industry. What are you industrial colleagues struggling with at the moment?

Next look in the supporting materials that your course provides for your Masters, particularly those associated with possible Masters supervisors. It may be that you will find topic suggestions for Master projects.

Next, think about what has been happening in your discipline recently. Go to Google Scholar and search for recent topic that might interest you.

You should now have a long list of general possibilities for research.

Unless you already have something very specific in mind, you may start with a broad selection, then narrow down your choice to one or two candidate topics for further investigation.
\end{guidance}\end{question}
%%Hack to correct tcbox behaviour
\color{black}

\subsubsection{}
Before investing lots of your time looking into a particular topic, however, you should assess whether it is appropriate for Masters research. Here are some things to consider before you make your final choice.

\subsubsection{Qualification fit}
If your project is the capstone of a taught degree, you should ensure that the topic is appropriate to the degree you are studying. In many instances, this relevance will be obvious -- you may well be exploring a concept that is a mainstream aspect of your subject and that you have met during previous studies.

However, given the cross-disciplinary nature of many postgraduate programmes, it is equally possible that your topic will span boundaries between disciplines. In such a case, it is important for you to demonstrate that the emphasis of the research will be appropriate for the Masters degree you are aiming for.

\begin{question}[subtitle={ACTIVITY: Considering qualification fit}] For each of the study options you identified in the last Activity, write down how it relates to your previous degree studies, and particularly to its core modules. Identify those topics that are most suitable for your degree.

\begin{guidance}You should make a note of key ideas, theories, approaches or principles covered in those modules, which are particularly relevant to the topic you have chosen, and identify specific materials you may like to revise or apply in your project.
\end{guidance}\end{question}
%%Hack to correct tcbox behaviour
\color{black}

\subsubsection{Professional fit}
At Masters level, your research should be of interest -- and of potential value -- to at least one of the following groups of people:

\begin{itemize}
\item Professionals\footnote{This may include both public and private enterprises -- local and national government and agencies -- as well as commercial organisations.} in organisations making up a particular industrial or economic sector that is within the scope of your research. This should be the case even if your research will be focused on a single organisation -- some element of the research you produce must be relevant and applicable to organisations doing broadly similar things.

\item Professionals in other sectors than that on which your research is focused but who experience similar problems or issues. A problem area for one industry may have relevance beyond that industry, or there may be implications for those setting policy or creating legislative frameworks.

\item Professional researchers in the field -- their interests are represented by the publications to be found in academic journals. Will your research contribute value to their future research?

\end{itemize}

Although you\footnote{And so it's really worthwhile thinking through professional fit very early on{\ldots}} will certainly need to argue the case for the professional fit and relevance of your research in your dissertation, it may also be the case that early and\slash or intermediate reports will require you to do this too. This will be especially important if your primary investigation is exclusively within a single organisation, in which case it may not be obvious that the results are applicable elsewhere. It is not enough just to state that the relevance exists -- you must provide some evidence in the form of logical argument or citation of reputable sources identifying a problem common across a range of organisations or an entire industry. Your review of the academic literature plays an important role: as we will see, a key purpose of such review is to identify gaps in existing knowledge that your research is designed to fill. This alone may demonstrate clearly enough the relevance to the wider academic and professional community.

\begin{question}[subtitle={ACTIVITY: Professional fit }] Consider your chosen topic. Write down who may be interested\slash benefit from research in this topic and why.

\begin{guidance}You should also indicate any evidence do you have to support your thinking. Ideally, you should talk to some of those people to explain what you intend to do and gain early feedback on the extent this may be of interest\slash benefit to them.

\end{guidance}\end{question}
%%Hack to correct tcbox behaviour
\color{black}

\subsubsection{Personal fit}
Through your Masters project, you will be living with your chosen research topic for many months\footnote{It may even feel longer:)}, so you \emph{must} choose a topic that will retain your interest over that period. A deep-seated interest in a question\footnote{Moreover, great research usually arises from the researcher's passion for the topic.} can carry you through the tough bits when your motivation might be flagging.

And\footnote{The timescale for a Masters research project is usually relatively short and you will be very busy without having to learn completely new subject areas or master completely new skills.}, although your topic should interest you, even a passionate interest can only take you so far: you should also avoid topics in which you have little existing knowledge, for instance, or that will require you to learn and master major new skills.

\begin{question}[subtitle={ACTIVITY: Personal fit }] Consider your chosen topic and write down the reasons why are you interested in it.

\begin{guidance}You should also make a list any new skills, if any, you may need to develop in order to research such topic, and indicate how you will develop them and in the time available.

\end{guidance}\end{question}
%%Hack to correct tcbox behaviour
\color{black}

\subsubsection{Organisational fit}
Your Masters course may or may not require that you associate your research with a specific organisation. Particularly in part-time studies, work-inspired research is a good way of getting value out of your research beyond your studies, and to gain support or even sponsorship from your employer.

In case of employer-sponsored projects, however, a note of caution is needed to avoid too narrow a focus for research to be possible. A common misconception is that the research is an opportunity to complete a piece of work for the sponsoring organisation. That should not be the case: the organisation may well benefit from your research, but the research itself must address a \emph{bona fide} research problem, i.e., one that has wider appeal and is beyond the needs of any single organisation. So, be careful that your research isn't linked too closely to the fortunes or objectives of a specific company or department.

If you are in the happy position of your employer offering sponsorship, make sure that you discuss with them the outcomes they expect from your research. Tell them about the broad focus that you will need to be successful. And make sure they understand that that broad focus will not stop you contributing to their desired outcomes. If needs be, you could always suggest that the Intellectual Property\footnote{Intellectual property is Intellectual property (IP) is property of things people create with their own intellect, such as an invention, an artwork, a design, etc. We will return to IP in Section XX} that you create through your research may lead to:

\begin{itemize}
\item Solutions to their specific problems;

\item New products and\slash or services that might generate a revenue stream that you can share;

\item The skills and attitudes that you will develop during the course of the research and your findings will be available within your organisation

\end{itemize}

\begin{enumerate}
\item even after this conversation, they remain unconvinced, then it may be better to choose a research topic that is not of immediate interest to the sponsor. It is a hard choice to give up sponsorship, but at least it will mean that you won't be constrained in conducting your research.

\end{enumerate}

\begin{question}[subtitle={ACTIVITY: Considering organisational fit }] If you're considering asking your employer to sponsor your Masters research project, or if they're already sponsoring your studies, write down how your chosen topic fits with their expectations, and how these compare to the requirements of your course.

\begin{guidance}You should list possible constraints from your sponsor which may prevent you from conducting your academic research, and how you intend to deal with those constraints.
\end{guidance}\end{question}
%%Hack to correct tcbox behaviour
\color{black}

\subsubsection{What is a research problem}

Within your chosen topic, you will need to identify a specific research problem that your project is going to address. Your research problem narrows down your focus from a whole topic -- which might be quite broad -- to something very specific: a context, a knowledge gap and a justification.

A well-defined research problem is the foundation of any research project, clarifying the research purpose and intended outcomes, something all good academic research requires. As well as using it to drive much of the process of arriving at your dissertation, you'll also include a description of your research problem in your dissertation to tell a reader what you are trying to achieve.

All research problems you'll find in this handbook stem from the following research problem template:

\emph{In the context C, with phenomena P to address knowledge gap G. This matters to W because R.}

An example of this could be:

\emph{In the context of the food industry, to address ``The replacement of plastic containers and wrappers,'' where} plastic, containers\emph{,} wrappers \emph{and} the process to replace them \emph{are all phenomena of interest. This matters: to the food industry because 10\% of all costs can be attributed to the use of plastic; and to society as plastic is highly polluting.}

As written, this research problems may appear a little contrived, but this is only the starting point to identify its important constituents, that is:	•	C: in which Context will the research take place?

\begin{itemize}
\item P: which are the Phenomena of interest?

\item G: what is the knowledge Gap?

\item W: to Whom does this matter{\ldots}

\item R: {\ldots}and for which Reasons?

\end{itemize}

It may be that in addressing the research problem, there may be other things generated: it might be, for instance, that a new process for manufacturing food containers is invented, which means that intellectual property will have been generated, or that other subsidiary questions are answered too, such as better container designs. However, these are not part of the research project based on this research problem. Of course, you may wish to follow them up later, after your project has finished\footnote{And it may be that your project can help you to do this{\ldots}}, but typically you wouldn't consider them otherwise.

Ok, back to the template. Let's unpack each component in turn.

\subsubsection{The context and phenomena of interest}
The context in which the research will take place contribute to constrain the project scope, alongside the phenomena that will be considered, what is of interest about them, what they influence, how they are measured, how they are observed, etc. Different contexts and phenomena therein will typically lead the researcher in different directions: for instance, using wifi in a home will lead to different considerations of its behaviour that using wifi in a hospital, where it could interfere with delicate medical devices.

As we shall see, contexts can be embedded in other contexts or can overlap each other, so that, it's important to identify precisely the context of the research project.

\begin{example}{Clara's research}Clara works for a large multinational engineering company in the production planning department of one of its small plants in Denmark. She finds that her planning department doesn't seem able to forecast resources and time accurately, so that production costs often escalate, reducing profit margins and lowering competitiveness. She has been asked to do some research to improve the situation.
\end{example}

\begin{question}[subtitle={Activity: Clara's context}] Identify the context for Clara's research problem.

\begin{solution}Clara's problem is embedded in many contexts, including:

\begin{itemize}
\item Denmark, the country,

\item a large multinational company,

\item a small plant belonging to that company, and

\item the production planning department.

\end{itemize}

If the scope was Denmark, then there would be crazily many phenomena that Clara could think about: choosing Denmark is just too large. The large multinational is beginning to be a more realistic scope, but there are other hints that make even more sense. She's been asked to investigate forecasting resources and time in her plant, so that the small plant and it production planning department are a most suitable context for her research, particularly if they share similarities with other plants within the company or in the industry sector.

\end{solution}\end{question}
%%Hack to correct tcbox behaviour
\color{black}

\paragraph{}
Let's consider phenomena next. Technically, a phenomenon is an element of the world, the occurrences of which are observable. They can be a material thing, transient, like an event or a situation, or information-based, like a fact or a concept; like we said, anything you can observe.

Some phenomena arise naturally and others artificially: many proteins are phenomena that occurs naturally through protein synthesis, whereas the technologies underpinning social media are constructed by software engineers.

If you look around the room in which you're sitting, every object you perceive -- each light, each fly, your computer and its (virtual) files -- every event that occurs -- that light turning on, the creak of your chair, a bird flying into your window -- is a phenomenon.

Phenomena can be really simple --- a speck of dust, or very complex --- the whole sequence\footnote{Although without specialist equipment, you might not be able to observe it} of steps that your computer goes through to connect to your wifi is a phenomenon. Complex phenomena are usually made of various observable elements, each a phenomenon too.

Phenomena\footnote{We don't claim that all research problems relate to phenomena, but suspect that all the good ones do.} are a very rich source of research problems. They have many and various characteristics, not all of which may be fully known: these call for more research.

There are phenomena which haven't even been identified yet! At one point, although we knew how to send data over radio signals, wifi didn't exist and so the devices and protocols associated with it had to be invented. New phenomena are being created and\slash or discovered all the time. That means new knowledge is needed and so research.

The existence of phenomena in one field might also suggest their existence in another -- more research needed.

In summary, a phenomenological basis for research problems means there's a very rich seam for researchers to mine.

\begin{question}[subtitle={Activity: Clara's phenomena of interest}] Identify phenomena of interest in Clara's research context.

\begin{solution}From the description above, the \emph{forecasting of resources and time} and its \emph{accuracy} are of interest: this is a complex phenomenon, possibly including a process, some techniques and some measurements. As part of Clara's research, each of its constituent parts (other phenomena!) may need investigating. Also of interest is its relation to \emph{profit margins} and \emph{competitiveness}, whose investigation too may require consideration of further phenomena.
\end{solution}\end{question}
%%Hack to correct tcbox behaviour
\color{black}

\subsubsection{The knowledge gap}
We have already discussed how academic research aims to generate new knowledge in a field of study. It is therefore essential that your research problem captures the knowledge gap your research intends to address. To help you start thinking about the knowledge gap, let's go back to Clara's scenario once more. You'll remember that Clara has been asked to investigate forecasting resources and time in her plant which is found to be inaccurate.

\begin{question}[subtitle={Activity: Clara's candidate knowledge gap}] Given the context and phenomena identified in the previous activities, write down a possible knowledge gap that Clara might address with her research.

\begin{guidance}If you're having difficulty finding the question, focus on what Clara has been asked to do.

\begin{solution}A possible knowledge gap we've come up with is:

\emph{``How to improve the accuracy of forecasting resources and time in the small plant.''}

Yours is probably similar. If it isn't, look again at the example initial description and focus on what Clara has been asked to do.
\end{solution}
\end{guidance}\end{question}
%%Hack to correct tcbox behaviour
\color{black}

\paragraph{}
It is important to know that this is only a \emph{candidate knowledge gap}: what we don't know at this point is whether Clara's planning department is just bad at forecasting, possibly not employing appropriate processes and techniques, or whether there is effectively a knowledge gap in the sector in that accurate techniques are unknown. To be able to judge, Clara will need to do more work, including looking at the literature, but we are not there quite yet.

\subsubsection{The justification}
Some research is motivated by the researcher' pure curiosity and desire to advance knowledge in a field of study, which is good justification to conduct research!

However, most research, particularly when located in a real-world context, matters to other people too, usually referred to as the stakeholders or beneficiaries\footnote{We will use the term beneficiary when they benefit from the research; a stakeholder means they will be affected by the research, but not necessarily positively.} (these include the researcher, of course!) and may have measurable real-world impact.

Measuring\footnote{Universities these days are assessed on their real world impact, so it is possible that your project will have some form of impact assessment in it.} the real-world benefit delivered through research is an easy way of showing that it has value. Having beneficiaries also means that you can more easily conclude that the research problem has been solved in context -- as you can ask the beneficiaries. Asking why the research is important is another good thing to know as you can then use their criteria is assess the value of research.

\begin{question}[subtitle={Activity: Who will benefit and why?}] In Clara's case, identify who may benefit and how. Also consider who may be able to judge whether Clara's research has addressed the problem.

\begin{solution}Clara's research should lead to more accurate forecasting, which would save her organisation money and make them more competitive.

Her colleagues in the planning department should be able to help her work out why the current process is less effective than it could be, and how innovations from her research could make a difference.

\end{solution}\end{question}
%%Hack to correct tcbox behaviour
\color{black}

\paragraph{}
\emph{It is very important to note, however, that it isn't the beneficiaries of the research that determine whether the research is actually research; that's a judgement based on the fact that new knowledge has been generated and this can only be judged by the larger research community.}

For a Masters research project, that larger research community might be represented by a very small group of people: your supervisor, a second marker, the external examiner. In exceptional cases,\footnote{Many of our research students have done this.} a Masters student might go on to report their research findings to a larger community of scholars at an academic conference, for instance, or even through a scientific journal: conference or journal publication is the pinnacle of validation that knowledge is new.

\subsubsection{Getting to the initial research problem}

It is now time to write down Clara's problem, by using the template we have provided, and based on what we have found out so far, which is summarised in Table X.

Table X - Elements of Clara's problem

\begin{ltabulary}{\textwidth}{ll} \hline
\textbf{Context C} & Clara's engineering plant and planning department \endfirsthead \hline
\textbf{Phenomena P} & \textcolor[rgb]{0.984,0,0.09}{F}orecasting of time and resources and its accuracy \\ \hline
\textbf{Knowledge gap G} & How to improve the accuracy of forecasting \\ \hline
\textbf{Stakeholders S} & Clara's company \\ \hline
\textbf{Reason}\textcolor[rgb]{0.984,0,0.09}{\textbf{s}}\textbf{ R} & Inaccurate forecasting increases production cost and reduces profit margins and competitiveness \\ \hline
\end{ltabulary}

\begin{question}[subtitle={Activity: Clara's initial problem}] Based on the information in the table and the template, write down Clara's initial research problem.

\begin{solution}This is what we have come up with:

\emph{To improve the accuracy of resources and time forecasting within Clara's engineering plant and planning department. This matters to Clara's company because inaccurate forecasting increases production cost and reduces profit margins and competitiveness.}

You may have written something similar.

\end{solution}\end{question}
%%Hack to correct tcbox behaviour
\color{black}

\paragraph{}
This problem formulation is only the starting point, as Clara has yet to establish that a knowledge gap actually exists. To do so, she will need to look into the literature and revise her problem statement accordingly. This is true in general: your initial problem will help you scope your review of the literature, which in turn will inform which knowledge gaps exist, and will help you reformulate your problem as a proper research problem.

In the next section we will look in detail at how you can do that, but first let's look more closely at diverse types of research problem.

\section{Types of research problems}
Research problems originated from our template depend critically on phenomena. And it won't come as any surprise then, that you can classify the types of research problem by the things you can do with those phenomena. And that leads to the different types of research that you can do, which we consider in this section.

\subsubsection{Descriptive problems}
Descriptive problems aim to describe phenomena of which we have little knowledge, accurately and systematically. The goal is to describe phenomena for the first time or to render existing descriptions more detailed or accurate.

\begin{example}{Examples of descriptive problems}To characterise the present state of the European commercial synthetic biology ecosystem. This matters to both producers and consumers in that ecosystem because this knowledge can be used to inform effective routes to commercialisation.

To determine how UK organisations are using the ITIL framework to manage cloud-based IT services. This matters to IT managers within those organisations because it can help them improve their service management practices.
\end{example}

\begin{question}[subtitle={Activity - Descriptive problems}] For each example above, identify its constituent parts with reference to the research problem template we have provided. Jot down any similarity between the problems.

\begin{solution}The table provides our mapping of the problem statements onto the template:

%Table in Latex
\begin{ltabulary}{\textwidth}{@{}LLLL@{}}
\textbf{Context C} & European commercial synthetic biology ecosystem & UK organisations                     & \endfirsthead
\textbf{Phenomena P}                                                & The current state of the ecosystem                                             & Those organisations' management of cloud-based IT services         &   \\
\textbf{Knowledge gap G}                                            & A characterisation of the present state the ecosystem                          & How they use the ITIL framework to manage cloud-based IT services  &   \\
\textbf{Stakeholders S}                                             & Consumers and producers within the ecosystem                                   & Those organisations' managers                                      &   \\
\textbf{Reason}\textbf{s}\textbf{ R} & To inform effective routes to commercialisation & Can help improve service management &
\end{ltabulary}

In both cases, the goal is to provide a description: of a state-of-the-art in the first problem, and of the use of a framework in the second problem. For the research to address a knowledge gap, such descriptions should currently be lacking.
\end{solution}\end{question}
%%Hack to correct tcbox behaviour
\color{black}

\subsubsection{Exploratory problems}
Exploratory problems aim to investigate phenomena of which little is known. The goal is typically to generate new ideas, hypotheses, theories, models or predictions which can be investigated in further research.

Lots of academic research is exploratory as it strives to develop a deep understanding of natural or social phenomena of which little is known. For instance, physicists try to explain the working of our natural world by making empirical observations of natural phenomena and conjecturing cause-and-effect relations, usually expressed as mathematical formulae or theories. This is the way natural sciences have developed over the centuries. This is also true of social scientists conductive observations in social settings in order to develop theories on human behaviour.

\begin{example}{Examples of exploratory problems}Within public service providers, to investigate the relation between demographic and behavioural factors and end-users' awareness of cyber security threats. This matters to those providers as successful cyber security attacks can result in loss of confidential information. This also matters to their customers as such information may include customers' data.

To investigate the possible use of social media in the management of a natural disaster by government organisations. Given the spread of social media, this matters to those organisations, which could integrate them within their critical communication infrastructures.
\end{example}

\begin{question}[subtitle={Activity - Exploratory problems}] For each example above, identify its constituent parts with reference to the research problem template we have provided. Jot down any similarity between the problems.

\begin{solution}The table provides our mapping of the problem statements onto the template:

\begin{tabulary}{\textwidth}{@{}LLLL@{}}
\textbf{Context C} & Public service providers & Government organisations dealing with natural disasters &  \\
\textbf{Phenomena P} & End-users' demographic and behavioural factors;~ cyber security threats & Social media; the management of natural disasters &  \\
\textbf{Knowledge gap G} & To investigate the relation between demographic and behavioural factors and end-users' awareness of cyber security threats & To investigate the possible use of social media in the management of a natural disaster &  \\
\textbf{Stakeholders S} & The providers and their customers & Those government organisations &  \\
\textbf{Reasons R} & Successful cyber security attacks can result in loss of confidential information & Social media could be integrated within critical communication infrastructures &
\end{tabulary}

In both cases, the goal is to explore a situation for which little understanding is currently available (this being the knowledge gap): in the first problem, the focus are possible effects of demographic and behavioural factors on awareness of cyber security threat; in the second, whether social media could be used effectively for communication in the management of a natural disaster.
\end{solution}\end{question}
%%Hack to correct tcbox behaviour
\color{black}

\subsubsection{Explanatory Problems}
Explanatory problems aim to identify why certain phenomena occur or are related. The goal is often to test a conjecture, hypothesis, theory or model.

\begin{example}{Example of explanatory problems}In the context of software development companies, to identify which learning strategies adopted by software engineers allow them to develop new skills fast and efficiently. These matters to those companies as software technology changes rapidly, so that software engineers need upskilling and retraining very frequently.

Within the public sector, to establish which actions taken by organisations and employees lead to an effective and efficient use of teleconferencing technology. This matters to public sector organisations due to the recent significant increase in flexible and home working, so that much of their business is now conducted online.\\
\end{example}

\begin{question}[subtitle={Activity - Explanatory problems}] For each example above, identify its constituent parts with reference to the research problem template we have provided. Jot down any similarity between the problems.

\begin{solution}The table provides our mapping of the problem statements onto the template:

\begin{tabulary}{\textwidth}{@{}LLLL@{}}
\textbf{Context C} & Software development companies & The public sector &  \\
\textbf{Phenomena P} & Learning strategies; skills development & Use of teleconferencing technology &  \\
\textbf{Knowledge gap G} & Which learning strategies adopted by software engineers allow them to develop new skills fast and efficiently & Which actions taken by organisations and employees lead to an effective and efficient use of teleconferencing technology &  \\
\textbf{Stakeholders S} & Software development companies and software engineer & Public sector organisations &  \\
\textbf{Reasons R} & Software technology changes rapidly, so that software engineers need upskilling and retraining very frequently & More and more business is conducted online &
\end{tabulary}

In both cases, the goal is to explore how and why certain phenomena are related. In the first problem, the focus is why certain learning strategies lead to more effective upskilling; in the second problem, why certain actions lead to more efficient use of teleconferencing technology.
\end{solution}\end{question}
%%Hack to correct tcbox behaviour
\color{black}

\subsubsection{Predictive Problems }
Predictive problem aim to predict future phenomena. The goal is usually to define a model, extrapolating from current knowledge, that allows predictions to be made together with an assessment of how accurate they might be.

\begin{example}{Examples of predictive problems}To quantify the potential impact on fresh water consumption of recycling domestic bathroom water in the UK within the next decade. This is important both from an economical and an environmental perspective, as fresh water is becoming a scarce resource, so its preservation is imperative.

To use demographic and curriculum data to predict which current students are at risk of dropping out in the context of higher education. This matters to higher education institutions as retaining their students is their statutory duty, and because students not completing their degree results in fewer skilled people entering the job market, which may impact productivity.
\end{example}

\begin{question}[subtitle={Activity - Predictive problems}] For each example above, identify its constituent parts with reference to the research problem template we have provided. Jot down any similarity between the problems.

\begin{solution}The table provides our mapping of the problem statements onto the template:

\begin{tabulary}{\textwidth}{@{}lll@{}} \toprule
 \textbf{Context C} & European commercial synthetic biology ecosystem & UK organisations \\
\midrule

 \textbf{Phenomena P} & The current state of the ecosystem & Those organisations' management of cloud-based IT services \\
 \textbf{Knowledge gap G} & A characterisation of the present state the ecosystem & How they use the ITIL framework to manage cloud-based IT services \\
 \textbf{Stakeholders S} & Consumers and producers within the ecosystem & Those organisations' managers \\
 \textbf{Reasons R} & To inform effective routes to commercialisation & Can help improve service management \\
\bottomrule

\end{tabulary}

Not completing degrees has a negative impact on institutions for lack of compliance with their statutory requirements, and on the job market due to scarcity of skilled workers

In both cases, the goal is to make predictions. In the first problem, on the impact of recycling water on fresh water availability; in the second problem, on the likelihood of students dropping out.
\end{solution}\end{question}
%%Hack to correct tcbox behaviour
\color{black}

\subsubsection{Evaluative Problems }
Evaluative problems aim to establish whether phenomena that have been introduced have achieved the desired outcome. These might be concepts, theories, products, technology, etc., and the goal is to establish how they have performed.

Evaluative problems may focus on testing the strength of an academic theory or model to ensure its long-lasting validity and scope of applicability. They may place an existing theory or model in a new context, or apply them to new situations, evaluating the effects of a change of conditions, what works, what does not and why, possibly leading to new theories or improved models.

Evaluative research may also take the form of a well-founded critique of diverse explanatory theories or potentially conflicting evidence put forward by previous researchers. A systematic and critical analysis of such evidence may lead to new knowledge on which theories are most reliable, so that future researchers may focus on those.

\begin{example}{Examples of evaluative problems}To evaluate the effectiveness of Artificial Intelligence (AI) to reduce the occurrence and impact of successful cyber-security attacks within the financial sector. This matters to the sector because over £3 billion are lost every year due to successful attacks, and efforts to prevent them are also very costly.

Within non-clinical laboratories, to determine if lean techniques can improve process flow in the presence of unpredictable demand and supply. Inefficiencies in process flow are costly to non-clinical laboratories, so that improving it is beneficial to them and their customers.
\end{example}

\begin{question}[subtitle={Activity - Evaluative problems}] For each example above, identify its constituent parts with reference to the research problem template we have provided. Jot down any similarity between the problems.

\begin{solution}The table provides our mapping of the problem statements onto the template:

\begin{ltabulary}{\textwidth}{@{}lll@{}}
 \textbf{Context C} & The financial sector & Non clinical laboratories \\
 \textbf{Phenomena P} & AI techniques and cyber security attacks & Lean techniques; process flow; demand and supply \\
 \textbf{Knowledge gap G} & To evaluate the effectiveness of AI in reducing successful cyber-security attacks & To determine if lean techniques can improve process flow in the presence of unpredictable demand and supply \\
 \textbf{Stakeholders S} & Operators in the sector and their customers & Those laboratories and their customers \\
 \textbf{Reasons R} & Significant financial losses are due to such attacks every year, and the cost of preventing them is high & Inefficiencies in process flow are costly \\
\end{ltabulary}

In both cases, the goal is to evaluate how approaches applied within a certain context perform. In the first problem, the focus is on the performance of AI to counter cyber security attacks; in the second problem, the focus is on the effect of lean techniques on the flow of processes with unpredictable demand and supply.
\end{solution}\end{question}
%%Hack to correct tcbox behaviour
\color{black}

\subsubsection{Design problems}
Design problems aim to create artefacts which embed new knowledge. The term `artefact' is meant in its widest meaning, and includes tangible and intangible products, processes, novel combinations of ideas or technologies, etc. The goal is to embed within the artefact new knowledge which extends human and\slash or organisational capabilities, including improved ways of doing something.

\begin{example}{Examples of design problems}
In the context of data visualisation tools, to devise algorithms able to generate automatically text summaries of line graphs depicting multiple time series for the benefit of sight-impaired users. This matters to both tool providers and sight-impaired users as it would improve the accessibility of such tools.

To define a hybrid project management framework, based on complexity and volatility characteristics, to reduce project failure in the development of embedded software systems. Current project failure across the sector is above 70\%, resulting in a high financial burden.
\end{example}

\begin{question}[subtitle={Activity - Design problems}] For each example above, identify its constituent parts with reference to the research problem template we have provided. Jot down any similarity between the problems.

\begin{solution}The table provides our mapping of the problem statements onto the template:

\begin{ltabulary}{\textwidth}{@{}LLL@{}} \\
 \textbf{Context C} & Data visualisation tool development industry & Embedded software systems development sector \\
\textbf{Phenomena P} & text summaries; multiple time series line graphs; algorithms & complexity and volatility characteristics; project failure; project management \\
 \textbf{Knowledge gap G} & To devise algorithms able to generate automatically text summaries of line graphs depicting multiple time series & To define a hybrid project management framework, based on complexity and volatility characteristics, to reduce project failure \\
 \textbf{Stakeholders S} & Tool providers and sight-impaired end-users & Project managers in the sector \\
 \textbf{Reasons R} & To would improve the accessibility of such tools & Project failure rate is high, resulting in a high financial burden for the sector \\
\end{ltabulary}

In both cases, the goal is to design something new. In the first problem, new algorithms for the generation of text summaries; in the second problem, a framework for hybrid project management.
\end{solution}\end{question}
%%Hack to correct tcbox behaviour
\color{black}

\paragraph{}
Note that an essential characteristic of a design research problem is that the to-be-designed artefact is a genuine contribution to knowledge in that it augments what is currently possible. Some of the most exciting research in technology-related subjects is about developing new artefacts. However, for such development to fit academic research it is important to establish what the contribution to knowledge is, and assess its originality and significance. In particular, it is important to distinguish the very first development of an innovative artefact from the routine of implementation of known systems: the former augments knowledge; the latter does not.

\subsubsection{Masters-appropriate research problems}

Regardless of the type of problem you intend to address with your research, you will only have a limited amount of time to complete your project and write your dissertation, most likely no more than one year. It is therefore important that you think carefully about your research problem from the start as changing direction later on in your project could be very challenging and increase your risk of not completing it successfully.

In particular, you should focus on the following criteria:

\begin{itemize}
\item Generality: your chosen research problem should not be so limited in scope to be irrelevant or of no interest to others. While your inspiration may well be something you have observed directly in your personal or professional life, your research problem should address a knowledge gap which is shared within your field of study, is of academic relevance and possibly of interest to professional practice, so that your findings can be generalised to some extent. Reviewing the literature is a way to ensure that what you have in mind meets these requirements.

\item Complexity: you will only have limited resources for your project, so your problem should not be so complex that it cannot be addressed within those constraints. Some research problems are just too ambitious for Masters research, e.g., ``To combat social inequality in the UK.'' However, there may potentially be suitable sub-problems you could consider, e.g., ``To understand the role of food banks in alleviating poverty in the UK in the last decade.''

\item Volatility: your chosen problem should remain current for at least the duration of your project and, ideally, present further opportunities for research in future. It is therefore important that you choose a problem which is not likely to become irrelevant too quickly. This can happen in technology-oriented projects that focus on specific products or tools, and their features. It could be avoided by considering if a more general problem can be found of which the original problem is an instance. For example, instead of ``how to visualise data effectively in tool X'', the problem could be generalised to ``which design principles apply for effective data visualisation'', regardless of the specific tool used.

\end{itemize}

\begin{running}{Assessing Clara's problem}We left our example with Clara's problem expressed as follows:

\emph{To improve the accuracy of resources and time forecasting within Clara's engineering plant and planning department. This matters to Clara's company because inaccurate forecasting increases production cost and reduces profit margins and competitiveness.}

Let's apply the criteria above to work out whether this is a suitable research problem or more work is needed.

In terms of generality, this problem is too specific to Clara's own company, so may not be indicative of a knowledge gap --- for instance, it is possible that Clara's company applies outdated estimation approaches, and better approaches are known and applied elsewhere. Therefore, for Clara, the next step will be to do some initial reading of the academic literature to establish:

\begin{itemize}
\item What, if anything, is already known about this problem in its wider context, e.g., other engineering companies or industries

\item Which specific aspects of this problem, if any, could be investigated to make a novel contribution to knowledge.

\end{itemize}

In terms of complexity, the problem appears tightly focusses and doable as part of a Masters project, so there are not particular concerns at this point. Similarly, it is unlikely the problem will cease to be relevant in the time span of the project, so volatility is not a concern either.

\end{running}

\begin{question}[subtitle={Activity: Appropriateness of research problem statements}] Consider the following research problem statements and discuss the extent you think they are appropriate for Masters research based on the above criteria.

1) To analyse possible differences in people's web search strategies to inform the design of search algorithms in software. Some individuals are better at conducting online searches than others and this knowledge may lead to more efficient algorithms.

2) To establish which information should be displayed to end users to help them identify energy waste within their heating systems. Energy waste impacts negatively household finances and the environment, so raising end-users' awareness could help reduce waste.

3) To improve coding skills of students at a distance. Coding skills are in demand and there is shortage on job market.

\begin{solution}This is our assessment of each of them. Yours may be different, of course.

1) Web search algorithms are a very active field of study, so it is likely that a contribution to knowledge can be made which is of wide interest to both academia and industry, and likely to remain so for the foreseeable future. Some thought will be needed to ensure that the work can be conducted within the constraints of the project, for instance by containing the number of participants in the study or the length of observations of their search strategies. Therefore, overall, with appropriate adjustments, this could be suitable for Masters research.

2) This problem is both topical and specific, so it has a good chance to meet the criteria. Some work on contextualising the problem in the literature will be necessary to establish the extent of the potential contribution to knowledge, particularly if design guidelines already exist, in which case it would be important to establish how these are lacking. Overall, this too could lead to a suitable Masters project.

3) This problem is too broad and open-ended for a Masters project --- it is a topic, rather that a specific research problem. A more suitable problem could be defined, for instance, by identifying which specific aspects of improving coding skills at a distance are problematic with reference to the literature, so that a much narrower knowledge gap can be established.
\end{solution}\end{question}
%%Hack to correct tcbox behaviour
\color{black}

\subsubsection{Formulating your initial research problem}
It is time for you to have a go at formulating your initial research problem, based on what you have learnt in this section.

\begin{question}[subtitle={Activity: Your initial research problem }] Apply the template provided to write down your initial research problem. Discuss what type of problem it is and assess it in terms of generality, complexity and volatility.

\begin{solution}Make sure you include all the elements of the template, i.e., C, P, G, S and R, and explain how your problem aligns with its problem type's goal. Argue concisely why your problem is appropriate for Masters research in terms of generality, complexity and volatile, or indicate further steps you will take to ensure that that's the case.

\end{solution}\end{question}
%%Hack to correct tcbox behaviour
\color{black}

\section{Starting your literature review}
In this section, you will start your literature review, an activity you will continue in Stage 2. In Stage 1 your focus will be on scoping and gathering what to read, and reading, analysing and assimilating relevant content. In Stage 2, your focus will be on completing a synthesis of all your reading to produce a full draft of your literature review.

\subsubsection{The role of the literature in research}
Academic research does not take place in a vacuum. The academic researcher relies on a body of knowledge in their field of study, the cumulative result of collective research efforts over long periods of time. They then use their creativity to add to it, and collaborate with other researchers to develop new ideas and technologies. The main vehicle for the codification and sharing of that knowledge is the academic literature. So, reviewing the literature and adding to it are intrinsic to academic research.

Your literature review will help you frame, contextualise and justify your chosen topic and research problem and to investigate the methods and modes of research that are considered relevant to your discipline. It will also help you demonstrate\footnote{As always, you`re trying to demonstrate this to the examiners and so it's their standards that you're trying to meet.} your understanding of the state-of-the-art, and to highlight knowledge gaps --- the unknowns --- to which your research will contribute.

Towards the end of the project, your literature review will also help you establish precisely which knowledge you have created and so evaluate what you have achieved in relation to other published related work.

Our framework recommends you focus on your literature review in Stages 1 and 2\footnote{Stage 2 start on page \textbackslash{}pageref\{???\}}, although you will be maintaining your literature review throughout your project -- incorporating newly published papers, adding or removing detail from what you've already written to support your dissertation as you progress.

Examiners will be looking for a literature review based on between 30 and 60 academic articles, and in the range of 2,000--3000 words.\footnote{Or something like 8 to 10 pages.}

\subsubsection{How to access the literature}
As a Masters student, your university library will be your first point of call for the academic literature -- probably online, unless you are able to travel to it every day\footnote{Most university libraries have single occupancy study rooms if you need to get away from the bustle of a home or work. Ask the duty librarian what facilities are available at your library.}. It is highly likely that alongside their collection of printed materials, your library will provide you with online access to a wide range of digital resources and selection of bibliographical databases. The latter are collections of ejournals, ebooks and articles that can all be accessed online and searched at the same time. Among those resources you will will probably find most, if not all, of what you need for your project.

\begin{question}[subtitle={Activity: Investigating your library resources and services}] Investigate your library resources and services to identify those which are going to be particularly useful for your project.

\begin{guidance}You should pay particular attention to those services which allow you to access resources online at any time and from everywhere.
\end{guidance}\end{question}
%%Hack to correct tcbox behaviour
\color{black}

\paragraph{}
Another source of articles is Google Scholar\footnote{Google Scholar is available at http:\slash \slash scholar.google.com} which has collected together over 100 million\footnote{This is wikipedia's estimate: \href{https://en.wikipedia.org/wiki/Google_Scholar}{https:\slash \slash en.wikipedia.org\slash wiki\slash Google\_Scholar}} academic articles in English, with direct links to each. Google Scholar also has bibliographic citations for download for each article that you can cut and paste directly into your bibliographic database.

Google Scholar provides access to the full-text of an article, particularly for published articles from commercial publishers, although these are usually behind a pay-wall\footnote{A pay wall is an electronic way of protecting an electronic document.}. In addition, Google Scholar collects together `versions' of the same paper so that, even if you don't have access to a paid service version, there may be a pre-print version of the article also available. If there is no version available except a pay-for-access one, you can link Google Scholar to any university library that you have electronic access to: this is an incredibly useful service which gives you direct access to those articles found in Google Scholar that are included in your library subscriptions.

Google Scholar can also help you access the so-called \emph{grey literature}\textbf{,} which is the collection of information produced by organisations whose primary or commercial remit is not publishing, such as as academia, government bodies, or non-publishing businesses and industries. It includes pre-publication and non-peer-reviewed articles, theses and dissertations, research and committee reports, government reports, conference papers, accounts of ongoing research, etc.

Google Scholar is an amazing resource for the researcher.

\begin{question}[subtitle={Activity: Connecting Google Scholar to your library services}] Investigate whether your library provides the facility to integrate Google Scholar alongside their proprietary search engines.

\begin{solution}Our university library provides some instructions on how to set up Google Scholar to connect to the databases the library subscribes to, so that direct links to the full body articles appear as part of the results of a Google Scholar search.
\end{solution}\end{question}
%%Hack to correct tcbox behaviour
\color{black}

\paragraph{}
A third way to get to specific articles which you can't access from your library or via Google Scholar, is to contact the authors directly: academic authors are usually keen to have their work read and should be able to share pre-publication versions of their articles or even point you to other relevant publications which they have produced. Contact details of academic authors can usually be found in the header of the articles they have published, or from their university's web pages. Alternatively, you may be able to contact them via professional networks, such as ResearchGate\footnote{ResearchGate is an online network specifically created to connect researchers around the world. It currently has a community of over 20 million researchers which use ResearchGate to ``connect, collaborate, and share their work.''} or LinkedIn\footnote{LinkedIn is the largest professional network worldwide, used to make professional connections and improve one's career. It is not specific to academia, but many academics are part of that community, so it is a useful place to go to make first contact.}.

\subsubsection{How to read an article}
In this section, we'll be looking at how to read an academic paper. It may seem strange to think about trying \footnote{Searching comes next, promise!}to read an academic article without having found any first! It's such an important skill, however, that it needs introducing early, and in a way that we can coach you through it first time.

This an approach we have used many hundreds of times with our students and is based on an excellent paper ``''How to read a paper'' by Srinivasan Keshav (2007). Keshav suggests a practical workflow for reading an academic paper. The workflow has up to three `passes', not all of which need to be used, with each having a specific aim:

\begin{itemize}
\item the first pass gives\footnote{You should still be adding it to your bibliography, however, who knows when it might become relevant in future -- even a long time after your masters studies are complete.} you a general idea of what the paper is saying. The first pass may take only 5 minutes and will allow you to disregard the paper if you find out it isn't relevant.

\item the second pass gives you a better grasp of the content in outline of a relevant. The second pass may take 20 minutes or so, and it gives you the opportunity to annotate the paper\footnote{We mean write your notes on it that capture your growing understanding.} as you begin to understand it. You can stop after the second pass if you need to -- perhaps you have a large pile of papers that you wish to sort for relevance.

\item the third pass deepens your understanding of the paper to that point that you can reproduce its main arguments and conclusions -- Keshav calls this ``Virtually reimplementing the paper''! Clearly, this form of understanding is truly deep and can take many hours, days, months, or even years! You can read a paper over and over again in the third pass, and your annotation at the end may grow to be as big as the paper itself. There will be key papers in your Masters project that have this status but they will be few.

\end{itemize}

Kershav's approach is summarised in Figure 1.

\begin{table}[htbp]
\begin{minipage}{\linewidth}
\setlength{\tymax}{0.5\linewidth}
\centering
\small
\caption{Summary of Keshav's approach --- adapted from the 2013 revision of Keshav (2007)}
\label{summaryofkeshavsapproach---adaptedfromthe2013revisionofkeshav2007}
\begin{tabulary}{\textwidth}{@{}lllll@{}} \toprule
 \# & Time & What and how to read or do & Objectives & Suitability \\
\midrule

 1 & 5--10 minutes & * Title, abstract, introduction * Section and sub-heading * Glance at mathematical content (if any) * Conclusions * Glance over references & * Category\slash type: measurement, analysis, description? * Context: related other papers? Theoretical basis? * Correctness: assumptions valid? * Contributions: main contribution? * Clarity: well written & Papers not in your research area but may someday prove relevant \\
 2 & 1 hour & * Jot down key points, comments, questions * Figures\slash diagrams: check axes, etc * Mark relevant unread references & * Grasp the content * Main thrust of the paper with supporting evidence & Papers of interest but not lying in your research speciality. \\
 3 & $>$1hour\footnote{We did day it could take a long time:)} & * Virtually reimplement the paper * Identify\slash challenge assumptions * How should I present it? & * Innovation, hidden failings\slash assumptions * Insight into complex arguments and presentation techniques * Jot down ideas for future work * Reconstruct the structure of the paper from memory & Strong\slash weak points, implicit assumptions, missing citations, potential issues. \\
\bottomrule

\end{tabulary}
\end{minipage}
\end{table}

\begin{question}[subtitle={Activity: Applying Kershaw's approach}] The reference you need to investigate is to a paper in a print journal:

Keshav, S. (2007) `How to read a paper', ACM SIGCOMM Computer Communication Review, 37(3), pp. 83--4.

Use google search to look up the article's title and then download the paper\footnote{Remember to record it in your bibliography; you can find most formats available through Google scholar under the cite link.}.

Start reading Keshav's paper using the guidance above.

\begin{guidance}You should ensure you make appropriate annotations as you go along, then develop and include appropriate notes and summaries in your BMT.

The figure below shows the annotations from our three readings for comparison with yours.
\end{guidance}\end{question}
%%Hack to correct tcbox behaviour
\color{black}

\subsubsection{}
Reading Keshav's paper using Keshav's workflow\footnote{It's also satisfyingly recursive!} will give you a head start on the following.

\begin{figure}[p]
%\includegraphics[width=\textwidth]{keshav2007readannotatedpdfpdf.pdf}
\end{figure}

\subsubsection{}
We've learned a lot in the time Keshav's paper has been in use by our students:

Firstly, the time suggested for each pass is only an approximation, and will depend on your own skills and previous experience of reading the academic literature, whether the subject matter or methodology is new to you, or even that the article may be poorly written and difficult to follow. Secondly, three passes may not strictly be enough: for some articles, particularly the most critical to your research, the third may actually consist of multiple passes.

\emph{Vice-versa}, not all papers require a third pass and you should be careful in deciding whether a particular paper should have one at any point in time: you should never totally disregard a paper -- unless it is out of scope -- as the importance of understanding a paper may only become apparent later: it may be a highly cited paper that needs to be understood so that other papers become accessible, for instance.

Moreover, we've found the following to be very important: at each pass, Keshav stresses the importance of making annotations as you read. Whether you print the article or work to annotate an electronic copy\footnote{A tablet of some form with appropriate note taking software is a great way of doing this.}, effective annotations will help you identify, and access more easily later on, those elements of the paper which are of particular interest to you. In the second pass, your annotations should include comments or queries on elements you find particularly relevant, interesting, or unclear, while in the third pass, they may include ideas for future work which may provide some inspiration for your own research.

Keshav also suggests you should highlight references upon which the arguments of the paper rest and that you may like to read later on, and comment on possible relation to other articles you may have already read. This will help you contextualise the article in the wider literature you are reviewing.

Your annotations will then help you write up your own notes and summaries.

Electronic copies of notes may, of course, be stored in your bibliographic database.

Ok, now it's time to give you something to try Keshav's workflow on!

\subsubsection{How to review the literature}
At this stage of your project, the main objective of your literature review is to demonstrate sufficient understanding of the topic you have chosen to be able to justify why your chosen research problem is worth investigating, i.e., why it will lead to new knowledge. In particular, your literature review should help you consider its wider significance beyond your personal or professional interest.

The gold standard of academic publications are peer-reviewed articles, that is articles which have been rigorously scrutinised by academic experts in the field (the `peers') to ensure high scientific quality. Academic journal and conference articles are usually peer-reviewed\footnote{The peer review process may be more or less stringent so you need to take care when selecting articles and you should treat each publication on its own merits. You can check though your university library if an article is peer--reviewed or not.}.

Peer-review is a practice unique\footnote{Well, almost unique.} to academia and it reflects the critical need in the academic community that work contributes new knowledge, that it is the work of those claiming it, and that the paper has been written to be accessible\footnote{This doesn't necessarily mean it will be easy to read!} to the community.

Most references in your literature review should be peer-reviewed, although there is scope for using other non-academic sources, such as trade magazines and the like -- articles that are professionally relevant, even if not peer-reviewed. Other sources of information can also be useful in your Masters research include government and other official reports, and although non peer-reviewed, they may have undergone some level of public scrutiny and still contain reliable information. Books and even websites can also be helpful, but they are unlikely to have been peer-reviewed or scrutinised, so you should treat their claims to be new or definitive knowledge with appropriate caution.

Reviewing the literature is a process of knowledge discovery\footnote{To know where there are gaps in knowledge, you need to know what is already known -- hence knowledge discovery. A poor literature review may leave you unable to claim a contribution to knowledge.}, and it is both iterative\footnote{Iterative = it never finishes -- although it's better than it sounds: later iterations will typically make fewer changes than earlier ones} and incremental\footnote{Incremental = you'll add to it over time.}. To help you access and review the academic literature in an organised and systematic manner, we recommend you follow the process\footnote{2023--05--31, 10:26It's not yet a process:)} outlined in Figure 1, whose activities are summarised in Table 1.

In developing your literature review, it will be up to you to select work which is relevant and should be included, to provide the necessary level of detail from work of other researchers, while at the same time synthesising and critically appraising ideas from different sources, establishing links between studies and their findings, and drawing on strengths and limitations of published research.

In practice, you will need to make use critically and creatively of the content of articles, books and other literature sources you have reviewed to articulate and support your own arguments and demonstrate your knowledge of the subject, including central arguments, ideas, trends and what remains unknown. \includegraphics[width=700pt,height=501pt]{Screenshot2022-09-12at110221.pdf}

Figure 1 --- LR - digram to redraw A process to review the literature

\begin{table}[htbp]
\begin{minipage}{\linewidth}
\setlength{\tymax}{0.5\linewidth}
\centering
\small
\caption{Table 1 --- Activities for reviewing the literature, including key skills required [LR- need to discuss the entries]}
\label{table1---activitiesforreviewingtheliteratureincludingkeyskillsrequiredlr-needtodiscusstheentries}
\begin{tabulary}{\textwidth}{@{}llllll@{}} \toprule
 \textbf{Activity} & \textbf{Aim} & \textbf{Key skills} & \textbf{Inputs} & \textbf{Outputs} & \textbf{Section} \\
\midrule

 \textbf{Planning} & To identify topics and sub-topics to investigate & Critical analysis of topic and subtopics & Research Problem & Search keywords & Key skills: ?? Outputs: ?? \\
 \textbf{Searching and gathering} & To identify and collect articles to review & Systematic searching of library resources\slash google scholar Bibliographic management & & List of articles with electronic (or paper) copies & Key skills: ?? Outputs: ?? \\
 \textbf{Processing} & To undertstand relevance of searched articles for further consideration & Skim reading, note taking, bibliographic management & & Populated bibliographic database, article annotations, notes, article ranking & Key skills: ?? Outputs: ?? \\
 \textbf{Assimilating} & To assimilate the content of the academic articles you have selected for in-depth review & Assimilating academic content, critical reading, further note taking and Bibliographic Management & & More detailed, article annotations, notes, short paper summaries, diagrams, etc leading to understanding & Key skills: ?? Outputs: ?? \\
 \textbf{Synthesising} & To consolidate what you have learnt as you start summarising your understanding based on your ongoing reading & Critical thinking, including identifying key ideas\slash gaps\slash relationships, and comparing articles Academic writing, including developing arguments and organising your narrative & & Structures and arguments into which articles logically fit. & Key skills: ?? Outputs: ?? \\
\bottomrule

\end{tabulary}
\end{minipage}
\end{table}

$<$$<$REVISIT AFTER REDRAWING: probably needs a subsection for each table entry.$>$$>$As well iterating the whole process, you should also expect some level of iteration between activities. For instance, it is possible that the searches you have planned do not return any articles, or the articles returned are not relevant and you do not wish to pursue them further. As already explained, there is much trial-and-error throughout the research process.

\subsubsection{Starting your literature search}
A literature review starts\footnote{And will change if your research problem change -- this is one reason for putting as much effort into your research problem as possible so that downstream activities, such as the literature review, will change as little as possible as you progress.} with your research problem. As we've explained, a research problem has the following elements: a context including some phenomena of interest, a knowledge gap in relation to those phenomena, and a justification in terms of who the stakeholders are and the reasons why the problem matters to them. Your quest for knowledge about each of them will drive your literature search.

\paragraph{Identifying search terms}

The number of peer-reviewed academic articles was estimated to have passed 50 million way back in 2010. So even without the grey literature\footnote{Look back at section ? for the grey literature definition.}, focussing on the most relevant articles for your research could be a challenge, if not approached systematically.

A good place to start is to brainstorm a set of search terms\footnote{A search term is a word or a combination of words that you can key into a search engine.} that bear a relationship to each of the elements of your research problem. Let's see how this could be done on our running example.

\begin{running}{Search terms from a research problem}Let us consider the following research problem:

\emph{To evaluate the effectiveness of techniques to test embedded safety-critical software while the hardware is still unavailable. Full software testing can only occur once the software is embedded in its hardware. However, in many safety-critical avionics systems, this can happen very late in the development process, leading to expensive re-design if errors are found. Early software testing while the hardware is still unavailable could reduce such late occurrence of errors and expensive re-design.}

We can take each problem element in turn to identify possible search terms. In doing so, we have come up with those in the table below. This initial set should be sufficient to perform our initial search.

\begin{tabulary}{\textwidth}{@{}LLL@{}}
 Element & From problem statement & Brainstormed search terms \\
 Context & Avionics systems development & Avionic systems; safety-critical systems \\
 Phenomena & Embedded software; safety-critical applications; testing approaches & Embedded software; software testing \\
 Knowledge gap & To evaluate the effectiveness of testing techniques when the hardware is not available & Testing techniques; effectiveness \\
 Stakeholders & Software developers, safety engineers within the context & Software developers; software engineers \\
 Reasons & Early software testing while the hardware is still unavailable could reduce such the late occurrence of errors and expensive re-design. & Late errors; re-design
\end{tabulary}
\end{running}

\begin{question}[subtitle={Activity: Search terms brainstorming}] Using your research problem, fill in the following table with up few search terms for each research problem element.

\begin{tabulary}{\textwidth}{@{}lll@{}} \toprule
 Element & What you wrote & Brainstormed keywords \\
\midrule

 Context & & \\
 Knowledge gap & & \\
 Phenomena & & \\
 Stakeholders & & \\
 Reasons & & \\
\bottomrule

\end{tabulary}

\begin{guidance}The aim of this activity is to arrive at an initial set of search terms you can use to start your own literature search. You should start by words in your problem descriptions, but you could also include synonyms, particularly if you are not sure which technical terms are used in the literature.

You shouldn't worry too much if you have come up with too many or too few in relation to each element: you will have an opportunity to review and refine your choices later as this, too, is an iterative and incremental process.
\end{guidance}\end{question}
%%Hack to correct tcbox behaviour
\color{black}

\paragraph{Combining search terms}
Your brainstormed search terms will provide a starting point for searching the literature. Initially, you should be prepared for the fact that very large numbers of articles might be returned with many of them irrelevant to your final goal of understanding your chosen topic. Therefore, you are likely to need to refine your search terms and also come up with ways to narrow down your search results to manageable set of relevant articles.

Search terms are often used in combination to create more complex search terms which may increase the likelihood of finding articles which are close to your topic of interest. Basic Boolean operators, as those described in Table X, are used for this purpose.

\begin{table}[htbp]
\begin{minipage}{\linewidth}
\setlength{\tymax}{0.5\linewidth}
\centering
\small
\begin{tabulary}{\textwidth}{@{}ll@{}} \toprule
 Boolean operator & Effect of the search \\
\midrule

 ``Term 1'' AND ``Term 2'' & Materials containing both Term 1 and 2 will be returned. This is used to narrow down a search. \\
 ``Term 1'' OR ``Term 2'' & Materials containing only Term 1 or only Term 2 or both will be returned. This is used to expand a search. \\
 NOT ``Term 1'' & Materials not containing Term 1 will be returned \\
\bottomrule

\end{tabulary}
\end{minipage}
\end{table}

\begin{question}[subtitle={Activity: Your first literature search}] Conduct a search in Google Scholar from the search terms you brainstormed out of your research problem elements, combining them using the operators above.

For each, record both the combination you have used and how many articles Google Scholar returns as hits. Reflect on how using different terms and operators may change the number of hits and how that may inform ways to widen or narrow down your searches.

\begin{guidance}You could start by using the AND operator, then repeat the search trying different combinations of search terms, different operators, adding and removing terms, or even using synonyms of your search terms.

You should ensure you record both the combinations you've used and the resulting number of hits.
\end{guidance}

\begin{solution}Using the search terms from our example, we have conducted the following searches.

Firstly, we typed in:

``safety critical systems'' and ``software testing''

to Google Scholar and this returned 5310 results: that's a lot of papers to read, so we knew that we needed to narrow the search down.

We therefore added another term and typed in:

``safety critical systems'' and ``software testing'' and ``avionics systems''

which returned 389 results: more manageable, but still a very large number.

To narrow the search further, we typed in:

``safety critical systems'' and ``software testing'' and ``avionics systems'' and ``testing techniques''

and found 90 results.

We then tried:

``embedded software'' and ``testing techniques'' or ``testing methods'' and ``avionics systems''

Which returned 35 results.

In this example, adding ``avionics systems'' helped us narrow down the search, presumably because only avionics applications were returned, rather than any kind of safety critical system; similarly, adding ``testing techniques'' presumably helped the search engine focus on techniques rather than, say testing processes or other. However, when we added ``testing methods'' using the OR operator, the hit list got even smaller, which was surprising. This demonstrate that search engines do not always behave as we would expect, so that it is important to try different approaches to arrive to a desirable outcome.

\end{solution}\end{question}
%%Hack to correct tcbox behaviour
\color{black}

\paragraph{}
As you may have gathered from this activity, to a large extent your initial search will be a trial-and error-process\footnote{Be sure to be systematic and record which search terms you have used: this will save a lot of pain later when you're trying to remember which search term you used to pick a particular article!} and you may end up iterating several times. It's not wasted effort, though. You can make the most of the time spent by:

\begin{itemize}
\item Generating more eUsing Boolean combinations of search terms wisely to narrow and widen your searches

\item Particularly if only few hits are returned, replacing your search terms with new terms -- often synonyms for existing ones.

\item Creating more compl

\end{itemize}

Finally, although in this section we have focussed on Google Scholar, the same techniques apply to other bibliographical database search engines, so you could repeat the activities on your database of choice.

\paragraph{What if there are still too many hits?}
Even if you spend your search time wisely, you may still end up with a large number of hits to consider. An important thing to recognise, is that not all of the hits returned will be relevant, so you will need to make an initial assessment of what to exclude from further consideration. When the number of search hits is high, this will take time, so you will need to select which articles to look at first.

We recommend you start by considering recent review articles. Review articles are academic articles which summarise the current understanding of a specific topic or phenomenon. They are usually the result of analysing and synthesising academic literature or bringing together findings from large surveys. Any recent review articles that appear in your search offer a very easy way to start your literature review: not only do they collect together recent relevant papers, but they may have an overview or a precis of each.

If no review articles are included in your hits, you could repeat your search by adding AND ``Review'' or AND ``Survey'' to the end of your search term. Alternatively, you could ask your supervisor if they can suggest a review article you can start with. It's a great question to ask at an early meeting with them. Your supervisor may also suggest other seminal papers for you to look at.

By `recent' we mean within 2 to 5 years, particularly for fast changing disciplines, such as Computing. For slow changing disciplines, the time frame of publication may be less important. Reducing the timeframe of publication is an effective way to cut down the returns considerably. Google Scholar allows you to select articles within a specified timeframe, and similar facilities are available in most bibliographical database search engines.

Second thing to recognise is that not all scholar hits are accurate -- they may have the wrong date on them for instance -- so care is warranted.

\begin{question}[subtitle={Activity: Narrowing down your hit list}] Consider the outcome of one of your searches. Identify any recent review paper which may be included. If necessary, revise your search term by including AND ``Review'' or AND ``Survey'' and try again, or set a specific timeframe to reduce the number of outputs.

\begin{guidance}In a previous activity, we conducted a search for Clara's problem using:

``engineering plant'' and ``planning'' and ``forecasting'' and ``accuracy'' and ``production cost''

which returned 78 hits.

By considering the last 5 years as timeframe for publication, these were reduced to 17 hits. Among them, we found 3 articles which included some form of review.
\end{guidance}\end{question}
%%Hack to correct tcbox behaviour
\color{black}

\subsubsection{Identifying relevant articles for further consideration}
Although you will need to read all of the papers that you find relevant, fortunately you won't need to read all of paper to find out if a paper is relevant!

Sometimes\footnote{Google Scholar, for instance, includes a short part of the paper that caused it to match your search term.} it's sufficient to read what accompanies the paper on a search service.

Otherwise, clicking on the paper will reveal its abstract which you can quickly skim to check whether the paper is relevant. If so, add it to your bibliographic database as something for further consideration.

Don't be too picky at this point, it's better to include something in your database that you don't use later, than to discard something that you find later might have been relevant.

Once you have a few articles, between 5 and 10 say, you should begin processing them. How to do that is coming right up{\ldots}

Ok, so you've found a relevant paper. That's going to start a new process{\ldots}

\paragraph{How will I know something is relevant?}
Understanding the relevance of articles is an important skill that develops over time by reading them.

You'll know something is relevant if the phenomena that are mentioned in a paper bear some relation to your project topic and research problem. There is no standard way of referring to phenomena, so having an open mind -- at least initially -- as to what could constitute the observables would be good.

Key to that skill is knowing which authors\footnote{Knowing which authors to look for can also help in narrowing down the literature to search. Care is needed, however, to ensure that other core literature is not excluded.} are the core contributors to your subject, although getting started with knowing authors means reading the literature -- which is circular. Your supervisors may be able to help by suggesting which authors have made core contributions.

You'll know something is relevant if the phenomena\footnote{Remember, phenomena were defined very generally as observables -- so having that list of phenomena you created in Section ? around while you read the literature would be a good idea.} that are mentioned in a paper. To distinguish observables from other things referred to in a paper, observables can be referred to in the following ways:

\begin{itemize}
\item Directly: statements such as ``the prices of software services were \emph{observable} in this study'' -- the observables are the various prices of software services.

\item Indirectly: statement such as ``the prices of software services were \emph{not directly observable} in this study, so we used the proxy of hours worked accumulated across the team of developers. The observables were the number of hours worked.''

\item As measurables: statements such as ``the number of bugs in the software were given through the collection of bug reports''

\item As inferences from other events: statements such as ``the number of bugs in the software were estimated from the number of bug reports''

\item More here, from the literature, add actual examples from papers?

\end{itemize}

\begin{example}{Going back to our example, one of the review articles we found in our search...}Going back to our example, one of the review articles we found in our search is:

Vahid Garousi, Michael Felderer, Çağrı Murat Karapıçak, Uğur Yılmaz (2018). Testing embedded software: A survey of the literature, Information and Software Technology, Volume 104, Pages 14--45, ISSN 0950--5849, https:\slash \slash doi.org\slash 10.1016\slash j.infsof.2018.06.016.

Its abstract\footnote{Increasingly, and very helpfully, many journals and conferences research articles are beginning to use \emph{structured abstracts}, such as the one in this article. A structured abstract has sections which bring out the context, objectives and other information.} reads:

''Context

Embedded systems have overwhelming penetration around the world. Innovations are increasingly triggered by software embedded in automotive, transportation, medical-equipment, communication, energy, and many other types of systems. To test embedded software in an effective and efficient manner, a large number of test techniques, approaches, tools and frameworks have been proposed by both practitioners and researchers in the last several decades.

Objective

However, reviewing and getting an overview of the entire state-of-the-art and the --practice in this area is challenging for a practitioner or a (new) researcher. Also unfortunately, as a result, we often see that many companies reinvent the wheel (by designing a test approach new to them, but existing in the domain) due to not having an adequate overview of what already exists in this area.

Method

To address the above need, we conducted and report in this paper a systematic literature review (SLR) in the form of a systematic literature mapping (SLM) in this area. After compiling an initial pool of 588 papers, a systematic voting about inclusion\slash exclusion of the papers was conducted among the authors, and our final pool included 312 technical papers.

Results

Among the various aspects that we aim at covering, our review covers the types of testing topics studied, types of testing activity, types of test artifacts generated (e.g., test inputs or test code), and the types of industries in which studies have focused on, e.g., automotive and home appliances. Furthermore, we assess the benefits of this review by asking several active test engineers in the Turkish embedded software industry to review its findings and provide feedbacks as to how this review has benefitted them.

Conclusion

The results of this review paper have already benefitted several of our industry partners in choosing the right test techniques \slash  approaches for their embedded software testing challenges. We believe that it will also be useful for the large world-wide community of software engineers and testers in the embedded software industry, by serving as an ``index'' to the vast body of knowledge in this important area. Our results will also benefit researchers in observing the latest trends in this area and for identifying the topics which need further investigations.''

\end{example}

\begin{question}[subtitle={Activity: the relevance of early hits to your search terms}] For five of the hits to the search you conducted in Activity ??, fill in the relevancy matrix below, and score them for relevance.

\begin{ltabulary}{\textwidth}{@{}LLL@{}}
\textbf{Type}&\textbf{Comment}&\textbf{Score}\endhead
 Direct & TBD& TBD\\
 Indirect & & \\
 Measurable & & \\
 Inferences & & \\
Complete this...& &
\end{ltabulary}

\begin{solution}TBD
\end{solution}\end{question}
%%Hack to correct tcbox behaviour
\color{black}

\paragraph{What to do when you find a relevant paper}
First thing you should do is to record it in your bibliographic database. You should record it with the search term that you used to find it so that you can rerun the search to:

\begin{itemize}
\item Find similar papers that might later have added relevance to your work

\item Find new papers that have been added to the literature since you last looked

\end{itemize}

In the latter case, you should also record the date\footnote{Many bibliographic management tools will do this for you. Check whether yours does, perhaps even by reading the manual{\ldots}} that you found it so you can check.

\paragraph{Surely that's not all I should do?}
Of course not, you need to read the paper too and begin to understand it -- for the first few times, you might like to refer back to the Keshav workflow in Section ??.

\subsubsection{Working towards an in-depth understanding of an academic article}
Initially, getting the most fro\footnote{Your supervisor may also be able to help with this task.}m an academic paper means understanding what it contributes to your literature review. Getting to that understanding quickly is important as your collection of articles grows as then you can determine which are the most important papers to include.

Honestly\footnote{And not just for your masters project. Accessing the academic literature is a professional skill that not many professionals have.}, it's going to take some time to pick up the skills of doing so, but it's an investment worth making. Given this, we're going to start slow and coach you through the process on a single paper that is of relevance to everyone studying for a Masters -- it's a paper on how to read an academic paper.

To understand an academic paper is to understand the contribution to knowledge that it makes. While you won't have to understand in depth everything you read, there will be some articles which are so fundamental to your research that you will need to study them carefully. This will require a substantial investment of your time and can be very challenging: even the most experienced researcher is unlikely to be able to read an academic paper once and immediately understand it in its entirety.

\paragraph{Assimilating and iterating}
In this section we give you a number of techniques to help you engage with the content of the articles you have selected. Your goal here should be to identify each paper's key features and their relationships so to be able to address the following questions and their contributions to your understanding:

\begin{ltabulary}{\textwidth}{@{}ll@{}} \toprule
 Questions & Contribute to \\
\midrule

 * When does the problem occur? * Where does it occur? * How is it caused? & Articulating the problem in its context \\
 * Whom does it affect? * How serious is it? * What are the benefits of addressing it? & Establishing its significance \\
 * What has been done so far to address it? * What else could be done to address it? & Establishing the potential contribution to knowledge \\
\bottomrule
\end{ltabulary}

\paragraph{Tracking key features of academic articles}
As the number of papers you have read grows, it will be increasingly difficult to keep track of what each article is about without continually having to refer back to your annotations. Even more difficult will be to compare quickly the key features of different articles.

To help you keep track of each article, you can use a summary-comparison matrix\footnote{Note that an SCMatrix must adapt to your needs as they change. Table 1 is a good starting point, but you'll probably add further columns as you progress.} (Sastry and Mohammed, 2013 (add \textbackslash{}cite)), SCMatrix, a basic form for which is shown in Table 2. In an SCMatrix you record the key features of each article, such as the research problem addressed, the key contribution made, the research methods applied, etc.

\begin{ltabulary}{\tablewidth}{@{}LLLLLLL@{}} \toprule
 \textbf{Reference (link to your bibliography)} & \textbf{Research problem\slash question\slash hypothesis} & \textbf{Research methods\slash approach} & \textbf{Key findings} & \textbf{Key conclusions} & \textbf{Implications for future research} & \textbf{Implications for practice (if relevant)} \\
\midrule

 & TBD & & & & & \\
 & & & & & & \\
 & & & & & & \\
 & & & & & & \\
 & & & & & & \\
 & & & & & & \\
 & & & & & & \\
\bottomrule
\end{ltabulary}

Example summary-comparison matrix

\begin{question}[subtitle={Activity: Constructing your summary-comparison matrix for Keshav's paper}] Complete the above summary-comparison matrix for Keshav's paper.

\paragraph{GUIDANCE}
$<$TO BE REDONE$>$This activity will require you to go back to your notes on each article and possibly skim through the article again to extract relevant points.

For each paper you read you should return to your SCMatrix and update it, looking for links between the papers you have read. With a few papers, this is simple to do, but don't underestimate the time it will take -- or the value that it will have for your literature review -- as the number of papers grows.

$<$Perhaps examples from students, if we have access?$>$\end{question}
%%Hack to correct tcbox behaviour
\color{black}

\paragraph{Identifying relevant themes}

As the number of articles you read grows, you will find recurrent themes emerging within your chosen discipline. These might be of many types:

\begin{itemize}
\item Subject relevant

\item Questions asked

\item Hypotheses\footnote{We'll say more about hypotheses later, but for now think just think of them as candidate answers to questions.} created

\item Research Methods

\item Ways of thinking through problems

\item Important terms and other nomenclature

\item Add more here

\end{itemize}

That these elements\footnote{Of course, forming a judgement on this is something you will have to do, perhaps with the help of your supervisor.} recur in papers in your chosen area should suggest to you that they are important themes.

You will also notice that, over time, certain of these develop as issues within your discipline as more is understood about it -- some will grow in importance while some reduce, perhaps even disappearing.

\paragraph{TIP: Any new themes that are introduced and recur in very recent papers would be good to make a special note of -- these might reveal current hot research areas.}
Given that you wish to make a contribution to the knowledge in your discipline, it is important that you're able to track the discipline's development as they will help you in your critical summaries, where you will need to bring together and compare and contrast ideas from different authors and articles, and in your own research, where they will indicate paths that are still to be trodden.

Here, another simple tool becomes important, the theme identification matrix (citation?), or TIMatrix, an example of which is shown in Figure 1 around the topic of safety critical systems --- it is not necessary for you to be familiar with that topic!

The rows represent articles you have read, so that their citations are included in the first column; the remaining columns correspond to the emerging themes --- concepts, questions or ideas --- you have encountered while reading them; the ticks give you an indication of how often and where each of them recurs in the literature you have reviewed.

\begin{table}[htbp]
\begin{minipage}{\linewidth}
\setlength{\tymax}{0.5\linewidth}
\centering
\small
\caption{Actual(?) theme identification matrix}
\label{actualthemeidentificationmatrix}
\begin{tabulary}{\textwidth}{@{}llllll@{}} \toprule
 Citations vs topics & Definition of safety critical system & Safety analysis & Safety analysis within development & Safety anomalies in development & COTS in safety critical systems \\
\midrule

 Paige, and McDermid, 2010 & x & & & & \\
 IEC, 2022 & & x & x & & \\
 De Lemos, Saeed, and Anderson, 1998 & & & x & & \\
 Martin and Muniak, 2002 & & & x & & \\
 Ellis, 1995 & & & & Xreference 6$<$????meaning? & \\
 Profeta, Adrianos, and Yu, 1996 & & & & & x \\
 Dawkins and Kelly, 1997 & & & & & x \\
\bottomrule

\end{tabulary}
\end{minipage}
\end{table}

$<$Is this from Rich's dissertation? All the better!$>$

\begin{question}[subtitle={Activity: Identifying recurring themes}] Consider the articles your have read so far, and your notes and summaries, to identify key recurrent themes. $<$I don't think I could do this with the reading I would have had to have done to get to this point -- 2023--06--12: might be better now$>$

\begin{guidance}You should construct a TIMatrix similar to the in Figure 1. You should also reflect on the most common themes and make notes on how they are approached by the different articles.
\end{guidance}\end{question}
%%Hack to correct tcbox behaviour
\color{black}

\paragraph{Identifying key ideas, gaps and relationships}
Having identified a number of relevant themes, you should begin to reflect on their relevance to your research and to assess the extent to which they should be addressed in your literature review. In particular, you need to start going deeper into those themes to explore underlying ideas, trends, links to other relevant themes and especially knowledge gaps.

To do so, we suggest you make use of a simple theme analysis table, like the one illustrated in Table 1: you should adapt it in ways which suit your project.

\begin{table}[htbp]
\begin{minipage}{\linewidth}
\setlength{\tymax}{0.5\linewidth}
\centering
\small
\begin{tabulary}{\textwidth}{@{}llllll@{}} \toprule
 References (link to MTB) & Theme considered & Key ideas\slash contributions\slash arguments & Relevance to my research & Knowledge gaps & Links to other themes \\
\midrule

 Reference 1 & & & & & \\
 Reference 2 & & & & & \\
 Reference 3 & & & & & \\
 Reference 4 & & & & & \\
 Reference 5 & & & & & \\
 {\ldots} & & & & & \\
\bottomrule

\end{tabulary}
\end{minipage}
\end{table}

Example theme analysis matrix or TAMatrix

\begin{question}[subtitle={Activity: Constructing your theme analysis matrix}] Starting from your current TIMatrix, select a theme which you think is relevant to your research and that you wish to investigate further.

For all the articles in your TIMatrix addressing that theme, fill in an entry in the table: this will require you to go back to your notes on each article and possibly skim through the article again to extract relevant points.

Do this activity for all the themes you wish to investigate further.

\begin{guidance}In the first column, you should include a full citation\footnote{Your BMT might allow you to directly link to an article for easy retrieval of your notes, or the whole article.}.

In the second column you should record the theme you are investigating.

The third column gives you an opportunity to summarise the main ideas and contributions of each article in relation to the theme you are investigating.

In the fourth column you should include your assessment of how each article relates to your own research and may inform your work, which also justifies why it should be included in your literature review.

The fifth column should help you explore the unknowns, and the ways in which your research may address existing knowledge gaps.

The sixth final column is for you to establish how ideas may relate to other themes, also helping you select which themes to consider next.

This is a substantial activity which, depending on how much material you have, may take you several hours or a full day of work. It's also an activity that you'll have to loop back on, given that themes in one paper will overlap\footnote{Perhaps not completely{\ldots}} with those in others.
\end{guidance}\end{question}
%%Hack to correct tcbox behaviour
\color{black}

\begin{guidance}There are other tools that will help you, too, such as mind maps. Mind maps are graphical devices for linking ideas together and are very popular with some people. However, there is a learning curve to be climbed and, if you aren't already familiar with Mind Maps, you may like to stick with the simpler matrix version.
\end{guidance}

\paragraph{}
Once you have completed your table, you will have a number of references associated with each of the themes you have selected, alongside a set of key ideas, contributions and gaps. In doing so, you should have a grasp of existing research around your chosen topic, and this should help shape your own thinking on how your research may contribute new knowledge.

\begin{question}[subtitle={Activity: Considering sources of information for your project}] Other than academic articles , list other sources of information which may be useful to inform your project. $<$ I don't think I'd be able to do this$>$

\begin{guidance}You should also make notes of how you will use them to inform your research.
\end{guidance}\end{question}
%%Hack to correct tcbox behaviour
\color{black}

\subsubsection{How much to read}

You will need to read extensively during your research and you will read a lot\footnote{When you write, you will focus only on the most relevant of the papers your search turned up.} more than your final selection of articles cited in your research proposal. So, a natural question is ``How do I know when I have read enough and have the references I need?''

Your final selection, simply, will be sufficient to convince the examiner that you have identified a gap in knowledge. The following diagram might help.

(Add Pacman diagram, showing the detail of the mouth opening)

You might like to think of it as trying to locate a particular rock pool\footnote{Might not be a good metaphor:)} starting from a map of your country. For places extremely far away from the rock pool, you don't need any description at all, other than mentioning the country. But, as you get closer, you might want to pick out a particular town to give an approximate location. Then, you need to give more and more precise descriptions as your reader gets closer and closer to the rock pool, finally giving them a really precise description so they can lock onto it.

In detail, you want your literature to identify the area in which your contribution to knowledge is going to be made, precisely and with support for the significant points in your reasoning.

You'll build up to your full literature review in your dissertation by first building a pilot review for your research proposal, one that supports your arguments in relation to your choice of research problem and the potential contribution to knowledge of your research with reference to the state-of-the-art.

Given this, if there are significant points which are not linked to the relevant literature, and not supported by other evidence otherwise, then you don't have enough references; conversely, if there are references which don't relate to any significant point, then they may well be superfluous. You can apply this rule of thumb at each stage of your research to assess whether you have done enough to progress to the next stage.

There is a caveat. A literature review is never really finished because as as you gain insights from the literature, those insights will in turn point you toward other reading; once you have answered a question, new questions arise, and so on. It is a iterative process you will go through till your final dissertation submission, and even then you will have new insights to follow-up or unanswered questions to address. But that's good, as it simply points to the fact that more can come from your work that other researchers may well pick up and so can contribute to your conclusions and further work, which examiners always like to see.

\section{Research activity: Setting research aim and objectives}

\subsubsection{Articulating your research aim}

Exciting times! Now that you have identified your research problem, and have supported it with an initial review of the literature, you can start defining your overall research aim, which is a concise statement of the intended outcome of your research.

A good way of thinking of the research aim is the it will describe the new knowledge generated by the project. This also tells you how important your research aim is: your overall research goal is to make a contribution to knowledge. That you can describe to some level of detail that contribution means that you will have made some great progress!

That you're at this point means it's the perfect time to think about your research aim.

\begin{example}{Clara's research }In our example, if your recall, Clara had identified the following problem after her initial review of the literature:

\emph{how to apply ML to improve the accuracy of resources and time forecasting in the context of small engineering plants. This matters to engineering companies because conventional methods are often inaccurate, and inaccurate forecasting increases production cost and reduces profit margins and competitiveness.}

Clara's initial research aim formulation is:

\textbf{Aim}: to identify which ML techniques improve the accuracy of resource and time forecasting in the context of small engineering plant.

In which she identifies the new knowledge: which specific ML techniques would be effective in addressing the need of improving forecasting accuracy in her chosen context.

Why is this his new knowledge? It may be -- indeed it is very likely -- that lots of research has been completed and published in the area of suing ML to forecast resources and time for large organisations. It may even be the case that this extends to published knowledge about large or medium sized engineering plant. However, through her initial literature review Clara will have reached the conclusion that this knowledge specifically doesn't apply to small engineering plant.

This way, Clara has identified the knowledge she aims to contribute through her research.
\end{example}

\subsubsection{}
Note that the difference between research problem and research aim can be very subtle. To distinguish them, you should think of a research problem as focusing on the knowledge gap to address, while the aim of the intended way that gap could be addressed.

\begin{question}[subtitle={Activity: Drafting your research aim }] Consider your current problem statement. Draft a possible related research aim.

\begin{guidance}Your research aim formulation doesn't need to be perfect at this stage, and will evolve with your research problem as you progress through your project. However, it is important for you to think about what intended knowledge contribution your research will make, and expressing your research aim is a way to do so.

\end{guidance}\end{question}
%%Hack to correct tcbox behaviour
\color{black}

\subsubsection{Choosing a title}
With your chosen research problem and aim you should now have a good idea of what you are hoping your research will focus on -- the need you will address in context -- and deliver -- the intended knowledge outcome. From this you can choose a representative title for your project. \footnote{The title is the first contribution you will have made to your dissertation! That first page is no longer blank. Congrats, you're on your way!}

The title provides the first indication to your reader of what you propose to research. It may change as the research progresses, so it is important to review it from time to time to check its current relevance. At this stage, the title can only be your best attempt at anticipating later developments in your research, so don't agonise too much over it.

Here are some guidelines for you to follow:

\begin{itemize}
\item A good title should succinctly convey elements of both research problem and aim, specifically, the focus and intended outcome of your project

\item A good title is around 8--20 words. Here are some examples\footnote{An experienced supervisor may have other examples too.}:

\begin{itemize}
\item Integrated process improvement strategies in small\slash medium-sized manufacturing enterprises in the UK fabricated metals industry

\item Cost-effective greenhouse gas mitigation measures for the UK livestock industry -- a risk assessment of the impact on water footprints

\item Critical Success Factors for enabling Packaged Software to realise the potential Business Benefits

\end{itemize}

\item Avoid titles that are very short and enigmatic, or titles that are long and rambling

\item Do not include acronyms or obscure technical terms, except those which are likely to be widely understood (e.g., UK, USA, IT, or WWW)

\item Do not pose a question in your title

\item A title is not a sentence, so it does not require a full stop at the end.

\end{itemize}

\begin{question}[subtitle={Activity: Choosing a title }] Use the guidelines above together with your research question and aim, to write down an appropriate title for your research proposal.

\begin{guidance}You will have many opportunities to refine your title, so at this point, spend no more than 20 minutes on this.

You should indicate how your chosen title convey elements of both your research problem and aim.

$<$Q: Is this something to do in the title, or in other notes?$>$
\end{guidance}\end{question}
%%Hack to correct tcbox behaviour
\color{black}

\subsubsection{Articulating your research objectives}
Research objectives help you break down your aim into the steps you should take to achieve it.

While your aim will be expressed in broadish terms, your research objectives must be specific and feasible: specific enough to indicate exactly what you will achieve, so to be able to tell whether you have achieved them by the end of your project; feasible in the sense that you must have the knowledge, skills, time and resources to achieve them.

$<$This might contradict the above statement -- changes to help?$>$Research objectives can still be quite broad, though, so that you will only need, 3 to 5 or them. As they contribute to reaching your aim, usually later research objectives will build on earlier ones.

\begin{example}{Example}Clara's research aim is $<$Check that this is still the case$>$:

\emph{to apply Machine Learning (ML) to improve the accuracy of resources and time forecasting in the context of small engineering plants}

Clara has decided that the `identify-assess-recommend' research objective pattern works for her. Using this pattern she comes up with the following incremental research objectives:

Objective 1. To \emph{identify} ML techniques that are applicable to resource and time forecasting in the context of small engineering plants

Objective 2. To \emph{assess} the accuracy of forecasting of those techniques in the context of small engineering plants

Objective 3. To \emph{recommend} how to integrate the most appropriate techniques effectively in engineering practice in order to improve forecasting accuracy

Objective 1 directs Clare to identify specific ML techniques to be used in the project, to ensure the work is feasible within the time-frame of the project.

Objective 2 directs Clara to assess how accurate the chosen techniques are in the context of her small engineering plants.

Objective 3 directs Clara to draw some conclusions from the research conducted and make recommendations to improve professional practice.

Objective 3 could have been split into two research objectives -- the first being to draw conclusions, the second being to make recommendations. Clara didn't do this as she understands the process of making recommendations will require her to arrive at some conclusions about applicability of what she has found.

Note how those objectives build on each other and, if successfully completed, they contribute to meet Clara's overall research aim.
\end{example}

\subsubsection{}
The `identify-assess-recommend' pattern the Clara used is a very useful pattern to know if the knowledge you create as part of your research will have real world application, which will be the lion's share of masters research projects.

Even if you apply this pattern, identifying the best form for your research objectives will require some creativity on your part, as there is no magic formula to do so. The example breaks down the aim into research objectives following an i\emph{dentify-assess-recommend} pattern, which is quite common, but by no means universal. If your research is more formal, mathematical, clinical, or of some other form, there are other research objective patterns that will be applicable. Examples are:

$<$Insert patterns, perhaps from students' work$>$

You can try to apply this pattern, but are not required to do so. In all cases, your should discuss how to break down your aim into objectives with your supervisor.

At this point in your project your objectives are still speculative, expressing the intention of your research: they are likely to change during your project, so that you you will need to review them at each study stage.

\begin{question}[subtitle={Activity: Articulating your research objectives}] Consider your current research aim. Write down 3 to 5 possible objectives, explaining how they relate to each other and how they contribute, if successfully completed, to meet your research aim.

Also comment on how specific and feasible they are, the latter in relation to your own knowledge, skills and resources, and your project time span.

\begin{guidance}Your aim and objectives don't need to be perfect at this stage and will evolve with your research problem as you progress through your project. However, it is important for you to think about what concrete contribution your research will make: expressing aim and objectives is a way to do this.

\end{guidance}\end{question}
%%Hack to correct tcbox behaviour
\color{black}

\section{Research activity: A toolbox for research design}

\subsubsection{Research design}
Although not all aspects of research are manageable, having a design for your research will provide many benefits, including the ability to summarise, explain, and justify it. The design will, like most other aspects of your project, will evolve: at the start, it will be a collection of your initial ideas; by the end, it will be a detailed account of what you have actually done. It will be a touchstone for you to refer to at times of difficulty and allow you to plot your progress against your objectives.

Your research design will depend on many factors, including the type of research problem you are trying to address, the intended outcome of your research, the sort of evidence you will need, the resources and expertise you\footnote{As well as accepted research strategies and methods applied by other researchers in your field to tackle similar problems.} have, and the philosophical beliefs which motivate them. As you are a key participant in your own research, your personal views and values will also affect the choices you make while developing your research design.

Research design is also a field of study in its own right, one which has grown out of many diverse academic traditions and ways of thinking across academic disciplines and subject areas, and which is still evolving\footnote{In this young research area, there is still a lot of post--rationalisation of a particular course of research as authors looks for generalisable themes.}. As such, it is not an easy topic to digest and is one of the most challenging aspects of doing academic research. It can be puzzling for students embarking on academic research for the first time.

For this reason, in Stage 1, we will not consider research design in detail, that will be started in Stage 2 instead. However, so that you can start to think about research design, in this section, we introduce some basic definitions around evidence and research methods, and then focus on ethical and legal matters in research.

\subsubsection{Types of evidence and data}
The phenomena upon which your research will be based must be observed and this gives rise to data. Data can be interpreted to give information and evidence.

Thus, most academic research will be based on data and evidence. Data is the raw observations with no interpretation attached --- anything you may collect, observe or gather in your research. Evidence is information interpreted to support (or otherwise!) your academic arguments. Indeed, data forms the basis of evidence, so the two concepts are closely linked and often used interchangeably. This section recalls briefly the main types of data and evidence used in academic research.

\textbf{Quantitative data} are data that can be quantified or measured, and be given numerical values. They include the following types:

\begin{itemize}
\item \textbf{Numerical} data are numbers\footnote{Yes, they are!}, such as the number of students registered on a module or the temperature in the UK in July (continuous). Simplifying a little, when numerical data has a whole-number values it is called discrete, otherwise it is continuous\footnote{Given the fundamental nature of energy, and the vagaries of quantum physics, it may be that we`re incorrect in stating that real--world temperature is actually a continuous variable. However, even if it isn't, its values lie on a continuous scale.}. In either case, appropriate mathematical and statistical operations can be applied apply.

\item \textbf{Ordinal} data can be arranged in an order, but are not necessarily numerical. An example is the very widely used Likert scale\footnote{Almost certainly, the most recent survey you completed would have used the 5--point Likert scale mentioned here.} often used in questionnaires to elicit opinions. An example of a 5-point Likert scale is that ranging from Strongly disagree to Strongly agree with Disagree, Neither Disagree nor Agree, and Agree in the middle. While these values can be arranged in order\footnote{This might be done by giving Strongly disagree the numerical value 1, Disagree the numerical value 2, and so on.}, mathematical and statistical operations can only be applied with care, for instance, taking the mean (or average) score.

\item \textbf{Interval} data are ordinal data, but for which we can know the distance between any two data points. For instance, calendar dates are interval data in the sense that we can calculate the time interval between two given dates, e.g., the number of days in between.

\end{itemize}

\textbf{Qualitative data, on the other hand,} are descriptive in nature and defy ordering. Sentences, words, images, sounds, etc., are all examples of qualitative data. An important subclass of qualitative data is \textbf{categorical} data\footnote{Categorial data, also called nominal data, because data is categorised or named. An example might be Dog, Cat, Alligator, Rock.} which is data that cannot be ordered and on which mathematical operations and functions don't apply.

Data and evidence are also classed as:

- \textbf{primary,} when newly generated or collected during research; or

- \textbf{secondary}, when already available from previous research, and re-used during new research.

The academic literature that will be at the core of your literature review\footnote{A bit or a hint for the next Activity:)} is secondary evidence, as are all other published academic and non academic documents, e.g., laws, policies and procedures, official reports, etc.

\begin{question}[subtitle={Activity: Considering types of data and evidence}] In the context of your project, consider which primary and secondary data and evidence you may need, where it may come from, and what type it is, either quantitative or qualitative. Write down your answer. $<$Check: I don't know if I could do this, given where we are.$>$

\begin{guidance}If you think you might use surveys in your research, you might like to think about the data that will come from the participants.

If you're designing something, think about the forms in which a description of the problem that your design will solve will be in.

If you're {\ldots}
\end{guidance}

\begin{solution}Taking Clara's example?
\end{solution}\end{question}
%%Hack to correct tcbox behaviour
\color{black}

\subsubsection{Classes of research methods}
\textbf{Research methods} are the means used in research to collect, analyse, synthesise or present data and evidence, and to derive findings from them. Their purpose is to help you conduct your research in a systematic, rigorous, repeatable and reliable fashion.

Research methods can be classes based on their purpose into:

\begin{itemize}
\item \textbf{data} \textbf{collection methods}, used to gather data and evidence

\item \textbf{data analysis methods}, used to analyse data and evidence

\item \textbf{modelling methods}, used to build models of complex real-world situations, where many interrelated phenomena are at play and a holistic understanding is needed.

\end{itemize}

Methods are also classed based on the type of data and evidence they handle into:

\begin{itemize}
\item \textbf{quantitative methods}, which -- unsurprisingly -- are used when dealing with qualitative data;

\item \textbf{qualitative methods}, which -- again unsurprisingly -- are used for qualitative data.

\end{itemize}

Broadly speaking, quantitative methods are widely applied in the natural sciences, with their focus on measurement, natural phenomena and their simpler cause-and-effect relations, while qualitative methods are widely applied in the social sciences, with their focus on understanding human behaviour. In practice, however, this distinction is not as stark and often quantitative and qualitative methods are mixed in research, particularly when research spans several academic disciplines. Instead, modelling methods are often associated with design, computing, engineering and more generally the so-called `sciences of the artificial' (Simon, 1969), which consider technology and its development in its wider social context, focusing on addressing complex, messy socio-technical problems.

Within these broad classes of methods, you will encounter several techniques, some of which are considered in Stage 2.

\begin{question}[subtitle={Activity: Considering research methods}] In the context of your project, write down which classes of methods you are likely to apply.

\begin{guidance}Look back to your answer to the previous activity. You'll be using qualitative methods if your data is qualitative, quantitative methods if your data is qualitative, and mixed methods if there are both. For each, you should indicate their purpose in your research, and relation to the data and evidence you will need in your project.
\end{guidance}

\begin{solution}Taking Clara's example?
\end{solution}\end{question}
%%Hack to correct tcbox behaviour
\color{black}

\subsubsection{Ethical and legal considerations}
All research must be carried out legally and ethically\footnote{People who carry out unethical and illegal research cannot share their results within the academic community without being called out. They cannot, therefore, be called researchers. Examples include those who researched the effects of tobacco and found they caused cancers, but didn`t share their 'research' outside of their sponsoring organisations.$<$Is this ok?$>$}, so that it is essential you consider your proposed research and research design from these standpoints. There are some key considerations in conducting conducting research ethically and legally, which we outline in this section.

\subsubsection{Personal data in research}
In your research, may also wish to collect data about your participants. The collection and use of this kind of data is usually regulated by law, although the specifics may change from country to country, and the university with which you're studying might add specific guidance too.

Within the European Union, the EU General Data Protection Regulation (GDPR) applies.\footnote{At the time of writing{\ldots}} GDPR defines \textbf{personal data} as any information which may identify a living person, be that a name or a personal identification number, or a combination of physical characteristics, or cultural or social identities, and establishes rules for the use of such data.

It also establishes particular legal protection or safeguards for \textbf{sensitive personal data}, that is, data which may reveal:

\begin{itemize}
\item racial or ethnic origin

\item political beliefs

\item religious or philosophical beliefs

\item trade union membership

\item genetic or biometric data

\item physical or mental health

\item sex life or sexual orientation

\item criminal convictions and offences.

\end{itemize}

Depending on your university or course regulation, you may not be allowed to handle sensitive personal data in your research, and you are likely to need to follow strict protocols when handling other personal data. You may even have to apply for permission to conduct the research you wish to do, including allowing an ethical committee to see any surveys that you wish to conduct.

Moreover, in reporting your research, you should \textbf{anonymise} any personal data, that is remove any information which may, directly or indirectly, lead to a living person being identified.

\subsubsection{The rights of research participants<Is it worth asking Clara for examples of ethical research including animals?>}
In your project, you may call on other people to take part in your research, for instance people you intend to interview or observe, or who may complete questionnaires you design, or provide you with documentary evidence you require. These people have a number of rights you must respect, primarily:

\begin{itemize}
\item The right not to participate -- no-one should be pressured to take part in your research

\item The right to withdraw -- they can change their mind at any point

\item The right to give informed consent -- they should be given sufficient information on your research and their role in it for them to decide whether they wish to participate or not

\item The right to anonymity -- their identity should not be disclosed unless they give you explicit permission to do so

\item The right to confidentiality -- the data\slash information you obtain from them should be kept private if they ask you not to disclose it

\item The right to privacy -- you should not intrude unnecessarily into their lives

\item The right to protection from harm, i.e., you must take steps to minimise the risk of harm, either physical or psychological, to all participants.

\end{itemize}

\begin{question}[subtitle={Activity}] By asking your supervisor, searching on your university's intranet, or otherwise, find out what ethical guidelines you will have to conduct your research under. Add any documents that you find to your bibliographic database so that they are easy to find again.

\begin{solution}Our university has the following ethical guidelines:

$<$Add OU guidelines here$>$

\end{solution}\end{question}
%%Hack to correct tcbox behaviour
\color{black}

\subsubsection{Processing personal data}
\textbf{The `processing' of `processing personal data'} refers to any action involving personal data, including obtaining, recording, analysing, and\slash or destroying data from which a living individual can be identified.

The GDPR sets out six principles\footnote{Why not set a reminder in your diary to revisit these principles whenever your research design changes.} that must be observed when processing personal data. Some refer to the collection and intended use of personal data; others to the way personal data are stored. Even if you think GDPR will not apply to your research, the six principles are useful as a reference as your research design develops, as that development might include personal data. Table 1 summarises the six principles, alongside our recommendations on how your can embed them in your project.

Table 1--- Data protection principles and how to apply them in your project$<$are these our creation? Do they have to be as a table?$>$

\begin{tabular}{@{}p{0.7\textwidth}p{0.7\textwidth}@{}} \toprule
 Data Protection Act principles & Recommendations for your project \\
\midrule
\begin{itemize}
\item Personal data processing must be lawful and fair
\item The purposes of personal data processing must be specified, explicit and legitimate
\item Personal data collected must be adequate, relevant and not excessive
\item Personal data must be accurate and kept up to date
\end{itemize} &
\begin{itemize}
\item Introduce yourself to your participants, e.g., ``I'm Karl Poppler, a student with University XXX''
\item Explain the reason why you are collecting the data, e.g., ``as part of my final Masters project I am researching topic YYY''
\item Assure them that the data will only be used for academic research purposes
\item Explain to them that the data will form part of an anonymised research report
\item Assure them that the data will not be passed on to any third party. The only exception is your research supervisor, which is bound by confidentiality
\end{itemize}\\

 \begin{itemize}
 \item Personal data must not be kept for longer than is necessary.
 \item Personal data must must be processed in a secure manner
\end{itemize}&
\begin{itemize}
\item Inform your participants of how long you will retain the data. Although GDPR does not set a limit, you shouldn't retain personal data for much longer than the duration of your project and submission of your dissertation. If you wish to retain some of the data for longer, you should anonymise them, so that personal information cannot be leaked accidentally
\item Inform your participants of how the data will be stored securely, including the use of encryption techniques for digital data, to prevent accidental or unauthorised access to, or destruction, loss, use, modification or disclosure
\item Inform your participants if you use an online data collection service which may store the data outside the EU, possibly in counties with weaker data protection legislation
 \end{itemize}
\\
\bottomrule
\end{tabular}

The six principles apply to personal data from which a living person can be identified. This is a severe hindrance on research that could only be done with live human data. Because of this, many techniques have been created to allow the use of live human data without comprising the six principles. These generally anonymise the data, i.e., the data is processed so that any link to a living person is removed.

If you do need to work with live human data, it is therefore worthwhile to understand which techniques exist for anonymising it. If you dod need to do this, at any point of your research project, the following activity will give you up-to-date information on how to achieve anonymised live human data.

\begin{question}[subtitle={Activity: Anonymising personal data}] Do a web search to identify techniques to anonymise personal data. List and summarise the main techniques you have found.

\begin{solution}Among others, you may have encountered the following common approaches to anonymising personal data:

\begin{itemize}
\item hiding, which refers to removing personal data from a dataset, for instance, the name and address of participants in a study. Those categories are completely removed from the data set

\item masking, which refers to obfuscating personal data by replacing values with certain characters, for instance replacing all names and addresses with asterisks. As are result, the specific values are not visible, but their categories are retained in the data set.

\item pseudonymisation, which refers to replacing identifying data with made-up identification data, for instance replacing names and addressed with faked ones

\item Generalisation, which refers to replacing certain data with more general equivalent, for instance, replacing an exact address with an area code.

\end{itemize}

\end{solution}\end{question}
%%Hack to correct tcbox behaviour
\color{black}

\paragraph{}
These various ways of anonymising data\footnote{A supervisor that has conducted research in the area of your project may have used data anonymising techniques before. It's always worth checking with them what level of anonymisation is needed, and which techniques could be used to achieve this.} have the specific uses depending on how the data will be used, with those later preserving more information than those earlier. For instance, hiding, which is the most destructive method, will stop a participant in data collection being tracked over time. This is good for anonymisation, but in a clinical study, for instance, you might need to identify the outcomes of a series of tests wit a particular individual. If you have removed all identifying information from the personal data, this isn't possible.

In this case, masking -- in which participant ``Cruella Deville'' is given an identifying label ''P1' -- allow tracking over time without revealing identity.

Masking does remove all category information; pseudonymisation -- in which ``Cruella Deville'' is replaced by ``Barry Norman'' allows a named individual to be referred to later, which might help a reader to stay more engaged.

Generalisation, in the other hand, could be used with other ways of identifying a person, so that their postcode could be replaced by the first component only, for instance, MK7 6AA -- which identifies a street location -- could be replaced by the less specific MK7 which is much less specific, but without removing all location information.

\subsubsection{Equity, Diversity and Inclusion in research}
Equity, diversity, and inclusion (EDI for short) are important ethical considerations in research Your university is likely to have policies that guide research towards good EDI. Not only is EDI important for society, but good research EDI leads to better research as your and others' biases can be compensated for.

Such policies may be aligned to national or international guidance. For instance, in the UK, national EDI definitions and principles have been established across the academic sector by UK Research and Innovation, which is a government body that brings together all UK research councils responsible for supporting research and knowledge exchange in UK higher education institutions.

As stated by the UKRI EDI principles, ``research and innovation should be `by everyone, for everyone' -- a dynamic, diverse and inclusive research and innovation system in the UK is an integral part of society and should give everyone the opportunity to participate and to benefit.'' And also that ``By valuing all, we recognise that a diversity of ideas, opinions, knowledge and people enriches our work and enlarges our knowledge economy.'' [Add source]

Equity\footnote{It is worth noting that both equity and equality are often used in the literature as the `E' in EDI. There is, however, a difference: equality refers to treating everybody in the same way, while equity acknowledges specific barriers or obstacles which affect certain individuals or groups of individuals, and seeks to remove them. The latter is now considered the better definition of the two.}, Diversity and Inclusion are as follows:

\begin{itemize}
\item \textbf{Equity,} which relates to fairness and justice, in the sense of removing barriers or bias which may prevent individuals, or groups of individuals, from having equality of access, opportunity or outcomes.

\item \textbf{Diversity}, which refers to the full spectrum of differences and similarities between individuals, whether socio-demographic, such as age, gender, race, ethnicity, etc., or in terms of beliefs and values, life experiences or personal preferences.

\item \textbf{Inclusion}, which concerns ensuring that all individuals feel welcome, valued and confident to be treated fairly and respectfully. Inclusion is often paraphrased as `diversity becoming normal'.

\end{itemize}

In thinking about EDI in your research project, you should consider:

\begin{itemize}
\item Yourself as a researcher: how do your own cultural perspectives and preferences affect your interactions with others or influence the way you consider data and evidence in your project?

\item Your research context: what insights do you have on the diversity of people you will interact with or may be affected by your research?

\item Your research activities: how will you account for equity and diversity in your research?

\end{itemize}

\begin{question}[subtitle={Activity: Considering EDI in your project}] Which EDI considerations apply to each of the following:

\begin{itemize}
\item Inviting human participants to take part in a research project

\item Designing a novel artefact, say a new technology or process

\end{itemize}

Write down your answers.

\begin{solution}Your answer may include some of the following:

\begin{itemize}
\item When inviting research participants, it is essential to consider their diversity to ensure they form a representative sample of the population under study.

\item When designing a novel artefact, it is essential to take the end-users diversity into account, so not to disadvantage some groups of individuals, for instance in terms of accessibility.

\end{itemize}

\end{solution}\end{question}
%%Hack to correct tcbox behaviour
\color{black}

\subsubsection{Intellectual property}
WORK IN PROGRESS

Intellectual property (IP) is the owned property\footnote{In this definition `property 'has the legal meaning of a collection of rights applying to the things which have been created, rather than the things themselves. If you find this confusing, think about a house or a car you may own. In legal terms, you have the property of that house or car, which gives you certain rights in law. However, in common language you may refer to your house or car as your own property. The two meanings are often confused in the wider literature on intellectual property. In this handbook we will always use the term IP to refer to the system of rights.} created through intellect, such as an invention, an artwork, a formula for some chemical compounds, etc.

Some IP gives rise to specific rights which are protected by law. For instance, the UK UK Intellectual Property Act 2014 \footnote{`\textbackslash{}url\{\href{https://www.legislation.gov.uk/ukpga/2014/18/contents/enacted}{https:\slash \slash www.legislation.gov.uk\slash ukpga\slash 2014\slash 18\slash contents\slash enacted}\}`\{=latex\}} defines a number of IP rights, establishing whom they belong to and what they allow the owner to do.

\begin{question}[subtitle={Activity: Types of IP rights}]
Conduct a web search to identify way of protecting intellectual property. For each, write down a definition and comment on whether they may arise from academic research.

\begin{solution}Your answer may include the following:

\begin{itemize}
\item Copyrights --- A copyright is an IP right that applies to any original work that is expressed in a tangible form, say a poem or a painting. It gives legal right to its creator(s) to reproduce, publish and distribute the IP, and to transfer it to a new owner. A creator acquires copyright automatically. A research dissertation, and any part within, gives rise to copyrights.

\item Patents --- A patent is an IP that prevents anybody by the owner from making, using, or selling an invention for a limited period of time. Patents are usually associated with products and processes, e.g., a microchip in a mobile device, or the process to engineer bacteria that digest plastic. For the owner's rights to be protected in law, a patent must be registered with a patent office, and different regulations apply in different counties. Academic research can generate patents, but the protection is not automatic: it requires appropriate registration to take effect.

\item Trademarks --- A trademark applies to something which can be used to distinguish commercial products and services of a trader from all others within the same market, e.g., the `apple' used on all products by Apple inc. A trademark must be registered, but there is no time limit to the protection it provides. Academic research may lead to new products or services which could be protected by registered trademarks.

\end{itemize}

This is not an exhaustive list and different countries' legislation include different forms of IP. For instance, the UK IP Act mentioned above includes `design' as a form of IP, which relates to the appearance, shape or configuration of a product, rather than the product itself (as the case for patents). $<$Check$>$ You may have also encountered `Trade Secrets', which prevents anybody other than the owner to disclose them: the famous CocaCola formula is proceed in this way.\footnote{Merriam Webster has this definition: something (such as a formula) which has economic value to a business because it is not generally known or easily discoverable by observation and for which efforts have been made to maintain secrecy}\end{solution}\end{question}
%%Hack to correct tcbox behaviour
\color{black}

\paragraph{}
Your university is likely to have an IP policy which establishes, among others, which IP applies in the context of academic research and Masters studies, defining both ownership and related rights.

\begin{question}[subtitle={Activity: Understanding your university's IP policy}] Using your university's intranet look up any IP policy you university has. Write a brief summary of what you have found out in relation to possible rights which apply to your Masters project work.

\begin{solution}At The Open University in the UK, the IP policy establishes rights in relation to the ownership, development, protection and exploitation of IP arising from all types of research and scholarship carried out in the institution. Although the policy changes depending on the type of invention, in relation to Masters research projects which are part of a taught course of studies, our university assigns the IP to the student, as long as they have paid their university fees. However, in case of fee waivers or bursary from the university, the IP will belong to the university$<$Check$>$.
\end{solution}\end{question}
%%Hack to correct tcbox behaviour
\color{black}

\subsubsection{Bias in research}
Bias\footnote{Several forms of bias have been recognised in research, with some becoming more and more prominent with the rise of AI research and applications.} can damage research findings, giving outcomes which are unreliable, casting doubt on the claimed knowledge contribution. As well as being experts in their subject, expert examiners are also experts in the forms of bias that can damage research in their field. You should therefore act to remove all bias from your research, and to use every resource at your disposal to identify which forms of bias might be detrimental.

\begin{question}[subtitle={Activity: Types of bias}] You can do some preliminary work on identifying general biases that might affect you by looking on the web. Conduct a web search on types of bias in research in general, and in relation to data in particular. List and summarise the main kinds you find, and measures that can be applied to mitigate them.

\begin{solution}Among others, you may have encountered the following:

\begin{itemize}
\item Confirmation bias, which is the tendency to favour the selection, analysis and interpretation of data which support the researcher's prior beliefs.

\item Observation bias\footnote{Observation bias is also known as the Hawthorne effect. You can read about the original Hawthorne experiment here: \textbackslash{}url\{\href{https://en.wikipedia.org/wiki/Hawthorne_effect}{https:\slash \slash en.wikipedia.org\slash wiki\slash Hawthorne\_effect}\}}, which is the tendency of participants being observed during a study to change their behaviour, possibly to please the researcher or provide the answers they think the research is seeking.

\item Selection bias, which occurs when data is selected subjectively, leading to samples which are not representative of the population under study.

\item Recall bias, which is the tendency of people to recall certain types of events more vividly than others, and can affect the outcomes of research which relies on participants' memories of the past.

\item Algorithmic bias, particularly in AI applications, which occurs when an algorithm produces outcomes systematically and repeatably which disadvantage one group of individuals over others.

\end{itemize}

Bias is subtle. We are all affected by it. It is easy to see in others, but extremely difficult to see in oneself. To be a good researcher you should assume that you are affected by many different forms of bias and work to eliminate them from your work. The upside is that you will be a better researcher for doing this.
\end{solution}\end{question}
%%Hack to correct tcbox behaviour
\color{black}

\paragraph{}
All bias can damage your research. However, confirmation bias is very easy to fall into, can be very disruptive, and can arise when you have too personal a stake in the research beyond the project itself. For example, you may be the manager of a process you are seeking to improve. This means that you will have a stake in the research subject beyond meeting your Masters academic requirements. Although this can be a strength, in that, since you already have an interest in, and knowledge of, the research subject and context, you are likely to have insights into the causes of problems and the factors that have an impact on them, it can also be a weakness in that it may lead you to focus on evidence that confirms any existing beliefs that you might have about how the process should improve, while dismissing evidence that does not support them\footnote{In the worst case, you might find that your current approach to improvement is wrong which might be visible to others within your organisation. If you find yourself in this situation and, in our experience, it does happen, it can be very hard to deal with this sort of bias.}. This may result in a lack of objectivity, and a tendency to make subjective judgements instead of an evaluation based on sound evidence.

Another effect of confirmation bias is that you clearly see a problem where no-one in your readership does. So much so that you don't feel you have to explain the problem or convince others that it exists.

In our experience, both of these can hamper progress. The latter is especially pernicious: that you see a problem is great, but can you find sufficient evidence in the academic literature?

\begin{question}[subtitle={Activity: Potential bias in your research}] Make a note of any potential bias that you may bring to your research. Conduct a web search to identify possible measure you could adopt to prevent them from affecting your objectivity.

\begin{solution}Your answer will depend on the king of bias you have been focusing one. For instance, for confirmation bias you may have found the following measures:

•	having participants, colleagues and\slash or your peers review your arguments and results;

•	making use of diverse data\slash evidence sources;

•	intentionally looking for alternative explanations as part of your data\slash evidence analysis.

LR --- LOOK UP OTHER MEASURES FOR OTHER KINDS OF BIAS
\end{solution}\end{question}
%%Hack to correct tcbox behaviour
\color{black}

\subsubsection{Your responsibility as an ethical researcher
}
To summarise, your responsibilities as a researcher are to:

\begin{itemize}
\item Behave with integrity, i.e., respect the rights of your participants, be open and honest about how you have conducted your research and about your results, including not committing plagiarism, ensure validity and accuracy in the collection and reporting of data, and disclose any conflict of interest, e.g. personal interests or relations with research participants which may compromise your judgement .

\item Comply with ethical standards in research, including EDI, laid down by appropriate bodies, including your university, and possibly to other codes set out by professional bodies in your field of study.

\item Comply with legal requirements in relation to health and safety$<$do we mention these?$>$, and data protection.

\item Guard against all form of bias in your research

\end{itemize}

In addition, there may be further guidelines established by your own university or course of study. For instance, you may be prevented from conducting research in which participants who cannot provide fully informed consent, or which requires you to collect sensitive personal data or commercial purposes. There may be also circumstances in which you will need an explicit permission, for instance if you require your participants to discuss sensitive issues, or be subject to prolonged interviewing, testing or observation.

\begin{question}[subtitle={Activity: Considering your university's ethical and legal guidelines}] Look up your university and course regulations to check any specific ethical and legal guidelines or constraints which apply to your project. Write a brief summary of what you have found out.

\begin{solution}Our own institution, The Open University, UK, has an extensive set of ethics and legal policies and guidelines related to research, alongside processes to gain approval when dealing with human participants and personal data. In addition, our Masters courses put further restrictions on the kind of research which can be conducted, including not allowing research involving minors or vulnerable adults, or the collection of sensitive data.
\end{solution}\end{question}
%%Hack to correct tcbox behaviour
\color{black}

\subsubsection{Sketching your research design}
Let's get a first sketch of the elements of your project research design.

\begin{question}[subtitle={Activity: Summarising elements of your research design}] Based on your answers to the previous activies, address each of the following questions:

\begin{itemize}
\item Which evidence and data you will need in your project and why?

\item Where will such data\slash evidence come from, and how will you ensure access?

\item Which kind of research methods do you intend to apply and why?

\item Which ethical and legal issues are relevant to your project, and how you will address them?

\end{itemize}

\begin{guidance}You should provide an explicit rationale with reference to your research problem, and intended aim and objectives. You should also comment on why you think those choices are feasible within the constraints of your project.

To address feasibility, you should consider the extent your choices are:

\begin{itemize}
\item Effective --- they should produce the data\slash evidence you need.

\item Manageable --- you should be able to apply them within the time available

\item Efficient --- they should produce data\slash evidence that you can process with the skills, resources and time available.

\end{itemize}

\begin{solution}What's our answer for Clara?

\end{solution}
\end{guidance}\end{question}
%%Hack to correct tcbox behaviour
\color{black}

\section{Research activity Managing risk}
``What could possibly go wrong in a flourishing concern like the Brixleigh Bank\footnote{Later in the book, Brixleigh's Bank is shown to be a Ponsi scheme -- a gigantic fraud -- which fails.}? My dear, you are too fond of conjuring up imaginary evils.'' -- Matilda Mary Pollard's \emph{Cora: Three Years of a Girl's Life}, 1882

Pollard's imaginary evils are now called risks. They are no longer seen as imaginary evils, but as things that need managing.

\subsubsection{Risk in your project}
Risk captures the likelihood of something going wrong combined with the impact that will have on your project, both on time, resources and outcome. In theory, both positive and negative impacts should be considered, but very often the focus is on what can affect your project in a bad way, letting the good stuff roll.

The management of risk is an important discipline in its own right -- you do not need all of the tools that that discipline offers.

So, in analysing risk for your project, you should focus on the following dimensions:

\begin{itemize}
\item Specific risks: What sort of things can go wrong? For example, you may not be able to recruit sufficient respondents to a survey to gain direct access to key evidence.

\item Impact: What are the consequences if things do go badly? How severe might those be? For instance, not obtaining key evidence will invalidate your all research, so this would be very severe in terms of your project outcome.

\item Likelihood: How likely is it that things will go wrong?

\item Mitigation\slash contingency: What can you do to reduce likelihood or impact? For example, you may have lined up a secondary source of evidence, which may not be as authoritative or useful as what you had in mind originally, but would be easier to access and still allow you to derive some interesting results.

\end{itemize}

In a research project there will be risk which is very specific to what you intend to do, but there are also risk categories which are common to all project, which we consider next.

\subsubsection{Technical skills}
Your intended project may require you to apply expert technical skills, for instance, coding or advanced statistical analysis. Early on in your research project, it may even be possible that the details of precisely which technical skills you will need are not clear -- will conducting a survey, for instance, require sophisticated statistical skills?

Your risk analysis should recognise this:

Specific risks: that your technical skill level isn't sufficient to be able to exercise those skills;

Impact: a lack of appropriate skills might mean that you are not able to analyse your data as well as you would like, losing its value.

Likelihood\footnote{Some approaches to risk ignore likelihood, assuming Murphy's Law, i.e., that anything that can go wrong will go wrong. This makes risk}: how likely is it that skills that you don't possess will be needed?

Mitigation\slash contingency: to reduce the impact (to manage the risk) you can take a course that helps you understand what level of skills you will need, and the love that you actually possess. This is an item that you can including in your project plan.

\begin{question}[subtitle={Activity: Risk in relation to technical skills}] Consider whether there are bespoke technical skills you do not possess, but are essential to your intended research. Perform a risk analysis in relation to their development and write down the outcome.

\begin{guidance}You should record the specific skill development risk, including likelihood, impact, and any mitigation\slash contingency.

On mitigation: if you don't think you will have developed all necessary technical skills in good time, then you should consider whether you have made the right choice of project -- this is also a way of managing risk! At Masters level, it is a lot safer to focus on research which makes good use of the technical skills you already possess.
\end{guidance}\end{question}
%%Hack to correct tcbox behaviour
\color{black}

\subsubsection{Study time}
Consider the time which is required for your project as indicated in your course guidance, and the fact that you should sustain a continuous effort throughout, with little scope for making time up when you are not able to or taking long breaks.

\begin{question}[subtitle={Activity: Risk in relation to study time }] Consider your current personal and professional commitments, and study practices. Perform a risk analysis on whether you will be able to dedicate sufficient time to your project on a regular basis and write down the outcome.

\begin{guidance}Under mitigation\slash contingency, you may include any adjustment to your current study practices and patterns that may be needed. If substantial, that you will need to assess very carefully how feasible it is for you to make such changes. $<$Here and later: should we be adding questions to ask? Do we have them?$>$

\end{guidance}\end{question}
%%Hack to correct tcbox behaviour
\color{black}

\subsubsection{Resources}
Depending on your chosen research problem and aim, you may need access to participants, organisational information, third-party data, industrial case studies, etc., or you may need to acquire specialised software or hardware.

It is important for you to assess how likely it is that you will be able to gain access to or acquire such resources and, should there be any cost involved, whether you can afford it.

If conducting research with your current employer, you should also consider the extent the data you require are confidential and non disclosable, as well as the possibility of changing jobs, and the extent you may be able to retain access or make alternative arrangements in such a case.

\begin{question}[subtitle={Activity: Risk in relation to resources }] Consider resources you are likely to need to conduct your research. Perform a risk assessment in relation to your access to those resources for the duration of your project and write down the outcome.

\begin{guidance}If you don't think you will have access to all necessary resources, then you should consider refocusing your project, so that you can make best use of resources you already have or will find easier to access.

\end{guidance}\end{question}
%%Hack to correct tcbox behaviour
\color{black}

\subsubsection{Ethical and legal issues}
In your initial development of your research design, you should have identified ethical and legal issues which are pertinent to your project. Here you are asked to consider any related risk.

\begin{question}[subtitle={Activity: Risk in relation to ethical and legal issues }] Consider the ethical and legal issues you identified as relevant to your project. For each, perform a risk analysis and write down the outcome.

\begin{guidance}You should pay particular attention to issues of health and safety, data regulations, permissions to access data or documentation from third-party, and the need for explicit permission to proceed from your own university.

It is essential you identify any ethical or legal impediments to your intended research earlier on, as these will prevent you from conducting your chosen research
\end{guidance}\end{question}
%%Hack to correct tcbox behaviour
\color{black}

\subsubsection{Summarising your project risk}
You can use Table 1 to summarise your risk analysis in relation to your project.

Table 1 --- Project risk analysis table

\setlength\tymin{2.5cm}%ensure small columns are not squeezed
\begin{tabulary}{\tablewidth}{@{}LLLLLL@{}} \toprule
 \textbf{Risk category} & \textbf{Specific instance} & \textbf{Likelihood} & \textbf{Impact} & \textbf{Mitigation\slash contingency} & \textbf{Guidance} \\
\midrule

 \textbf{Lack of technical skills for primary research} & & & & & You should consider essential technical skills needed in your research and whether you will be able to develop them in good time to meet your project milestones \\
 \textbf{Lack of study time} & & & & & You should consider your current personal and professional commitments, and study patterns and how these may fit the study assumptions fro your project \\
 \textbf{Lack of access to resources needed, including participants and secondary data} & & & & & You should consider both what you will need for your project and how you will ensure access \\
 \textbf{Ethical and legal constraints} & & & & & You should pay particular attention to health and safety, data regulations, and permissions from third parties \\
 \textbf{Other (specific to your project)} & & & & & You should consider if there is other risk which is specific to your proposed research and not covered by the other categories \\
\bottomrule
\end{tabulary}

\begin{question}[subtitle={Activity: Risk assessment for your project }] Complete your project risk assessment by filling in the entries in Table 1.

\begin{guidance}You should already have most of the required content from your previous activities.

You should ensure you consider carefully any specific risk to you project which is not covered by the generic risk categories we have included in the table.
\end{guidance}\end{question}
%%Hack to correct tcbox behaviour
\color{black}

\section{Research activity: Reflecting}
[TBC]You might be thinking: ``Why, if I've finished a task, should I cause myself grief by reflecting on it -- in the best case nothing will change. In the worst case, I'll have to do it again.''

We feel exactly the same -- reflecting on what you've written causes a lot of grief, especially when you realise that it's not as good as it could have been\footnote{We`ve reflected on the materials for this book many many times{\ldots} that's meant redrafting, adding extra and -- hurtfully -- having to remove stuff that just wasn`t good enough. It's changed -- for the better we hope -- because of it. You are free to disagree, and we'd welcome your reflection on it too.} and there are very good reasons to improve it.

At the same time, we've learned a lot from that reflection. For instance, we've been able to split up complex issues into more digestible chunks; we've identified links between topics that we didn't know about; we've read more about what the students we have worked with to understand more precisely their contributions; we've taken apart the materials we have written to ensure their validity in the master's research context; and, last but not least, we've been able to generalise much of what we know of this topic to be more applicable across the board, while at the same time realising that there are special topics that will affect only a small part of our readership.

There is a context into which all research fits that colours the knowledge that is its primary contribution making it less valuable, less distinct, less out there. That context is a hegemony of received wisdom, of `common sense thinking', of uncritical investigation, none of which are necessarily knowledge. Reflection -- ''the voluntary disobedience of thought and reasoned undocility'' according to Foucault (1985) -- is your way of breaking that mould of stepping out of conventionalism and of shaking up the world.

Reflection is the thinking of radical thoughts. Improving your research through ever deeper reflection as you progress is the best way to engage (and impress) your reader\footnote{{\ldots}and your examiner, who will want to know that you have reflected on what you have done.} and keep them reading to the end.

\begin{question}[subtitle={Activity}]
$<$Needs assessing for content and structuring into activity + guidance$>$

This activity has four parts: the first is something you should be doing regularly, but won't make you into a disobedient or indocile thinker. The second, third and fourth may help you get started and keep going.

Part 1: Think about your study this far -- using this book and anything you've done for your dissertation in parallel -- as a journey. More from elsewhere, including £!!.

Part 2: think about yourself and the way you think. How does your desk look? Is it messy or tidy? Do the same for your computer desktop. Is it empty or are there hundreds of files strewn across it? Do you think your tidiness or untidiness will affect the way you do your research? How about how you keep your -- critically important -- bibliographic database which may contain up to a hundred academic\footnote{It's not unknown to have more than a hundred.} and other articles by the time you're finished?

Part 3: think about the context of your research. Which professional pressures are there on you to succeed in solving your research problem? Pressures could come in many forms: financial -- there's a promotion for you at the end of it; peer -- your colleagues know that you are studying will have good expectations of your result and you'll want to prove them right\footnote{Or wrong, depending on the colleague!}. Are you sponsored by your employer? Will you be able to report a negative outcomes to your research, for instance, that there is no solution to our problem using the current technology stack? A negative result is a very good research outcome, even if it tends to satisfy fewer non-academics than a positive result.

Which family pressures do you feel? It's' not unusual that you will have given up a paying role to study, moving the responsibility to provide onto another member of your family. What are their expectations?

Part 4: what's that thought nagging at the back of your mind? Is it ``How will I start?'' Or ``Will I be able to dedicate enough time to this?'' Or ``Can I really do this?''. Or ''Is ``shouldn't I be bringing in a wage rather than studying?''

You may be one of the lucky ones that doesn't have such negative thoughts, but negative thoughts are a very natural part of steps into the unknown. And research is precisely that, a step into the unknown. At least if you are aware of the doubts you naturally have, you can manage them. Think about making even the tiniest of steps forward in your research visible and celebrated! Work with Kansan boards where progress is encouragingly visible as you move a task from the inbox to the outbox. If you have concerns about managing your time, start using one of the many tools out there that break time up into manageable units and help manage it for you. If your concerns are about how to organise your thoughts, look into mind maps, lists, todo lists.

Thinking early and often through reflection is a powerful way of doing better. Do it well and your final report will be better than you will have expected.

It's worth saying that, at the end of what could be an exhausting journey, you will not fully appreciate your achievements. That realisation may have to wait until you are rested, graduated, or some distant time later.

But it will come.

\begin{guidance}Something here
\end{guidance}\end{question}
%%Hack to correct tcbox behaviour
\color{black}

\section{Research activity: Reporting}
It may not feel like it, but you're now ready to write a substantial contribution to your research project: your full research proposal.

We recommend that you write a report at the end of each stage to consolidate the work you have carried out, regardless of whether your course may require to do so. Writing such reports will help you develop your dissertation incrementally, and provide good practice to improve your academic writing skills as you go along.

\subsubsection{Putting your research proposal together}

Here, in Stage 1, your report\footnote{Although not all what you write here will end up in your actual dissertation, substantial parts of it will. You're definitely started now and that has to feel good!} will consist of your full research proposal, which we recommend you structure as indicated in Table 1 --- subsequent reports will build on this structure by adding further elements.

Table 1 -- Research proposal structure and guidance

\begin{ltabulary}{\tablewidth}{@{}LL@{}} \toprule
 \textbf{Report template} & \textbf{Guidance} \endhead\midrule
 Proposed title & Your title should capture succinctly research problem and aim \\
 Sect 1 - Introduction 1.1 Background to the research 1.2 Justification for the research & This section should provide an introduction to your research topic in its wider context (as background) and your justification of why the research is worth pursuing. It should be well articulated and supported by evidence \\
 Sect 2 - Literature review 2.1 Review of existing relevant knowledge 2.2 Planned literature review & Your review should provide a critical summary of your in-depth engagement with the academic (and other) relevant literature to date, including identifying key trends, ideas and possible knowledge gaps. Most of your citations should point to academic articles. Your planned review should identify further reading you will undertake in the next stage. Both coverage and depth of your review matter. You should ensure that your review is well structured, with a logical narrative flow and your arguments are well supported by evidence \\
 Sect 3 - Research definition 3.1 Problem statement 3.2 Aim and objectives 3.3 Knowledge contribution & You should ensure that your research problem is well articulated and appropriate for your course and your personal and professional circumstances, that your and objectives are consistent with research problem, and that the intended knowledge contribution of your research is clearly articulated \\
 Sect 4 - Research design 4.1 Evidence and data 4.2 Research methods 4.3 Ethical and legal considerations & This section should demonstrated your initial engagement with research design, particularly that you have thought about the kind of evidence and methods you may need, appropriately justified in relation to your research problem, aim and objectives. It should also demonstrate your careful consideration of ethical and legal matters, and that your research will comply with your course and university requirements \\
 Sect 5 - Assessment of your proposed research 5.1 Qualification fit 5.2 Personal and professional fit 5.3 Technical skills and resources needed 5.4 Statement of feasibility & In this section you should argue how your research is a good fit across all criteria. You should provide a clear rationale as to why you think what you are proposing is feasible \\
 Sect 6 - Planning, scheduling and risk assessment 6.1 Statement of progress 6.2 Key priorities in follow-up stage 6.3 Risk assessment & In this section you should reflect on the progress you have made in Stage 1 and establish your priorities for the next stage. You should also summarise the outcome of your risk assessment, focusing on your major risk and how you intend to manage it \\
 References & You should keep your references in good order and ensure you apply the required bibliographical style consistently. Ideally, you should use a BMT to generate and integrate your references within your report \\
 Appendix - Work schedule & If you have created a work schedule, you could include it as an appendix for reference \\
 Appendix - Risk assessment table & You could include your filled-in risk table as an appendix for reference \\
\bottomrule
\end{ltabulary}

\begin{question}[subtitle={Activity: Putting your report together}] Using your word processor of choice, create a report with the structured indicated in Table 1, and fill it in by following the guidance provided in the table, making good use of your notes and summaries from, and reflection on, all related activities you have carried out so far.

\begin{guidance}In this first pass at putting together your report you should focus primarily on completeness, ensuring that each section includes at least draft content.
\end{guidance}\end{question}
%%Hack to correct tcbox behaviour
\color{black}

\subsubsection{Assessing, Iterating and finalising}
After you have filled in your report with as much material as you can, you should review and revise it until you are happy with your account, and ready to move on. This may take more than one iteration, but you should ensure you do not delay your work for the follow-up stage.

In the next activity, you will use Table 1 to assess whether your report is of good standard.

Table 1 - Criteria to review your report

\begin{tabulary}{\tablewidth}{@{}LL@{}} \toprule
 \textbf{Criteria} & \textbf{Prompts} \\
\midrule
 \textbf{Completeness} & Are all sections of the suggested structure completed in line with the guidance provided? \\
 \textbf{Good academic writing practices} & Have you applied good academic writing practices throughout? \\
 \textbf{Logical structure and flow} & Have you structured your narrative appropriately to ensure a logical flow of arguments? \\
 \textbf{Supporting references or evidence} & Are your key arguments supported by appropriate references or other evidence? \\
 \textbf{Citation and reference style} & Do all your citations and references comply with the required bibliographical style? \\
 \textbf{Avoiding plagiarism} & Have you acknowledged the work of others and distinguished it from your own appropriately? \\
 \textbf{Standard of English (or any modern language you use)} & Have you proof-read your report carefully to remove all typos and grammatical errors? \\
\bottomrule
\end{tabulary}

\begin{question}[subtitle={Activity: Reviewing your report}] Apply the criteria in Table 1 to review and reflect on your current report and write up a summary of your assessment.

\begin{guidance}For each criteria, consider the related prompts to help you assess your report overall, and write down any further work needed for your next stage.
\end{guidance}\end{question}
%%Hack to correct tcbox behaviour
\color{black}

\subsubsection{}
Writing up your report is an excellent way to communicate the work you have completed and still planning to do, and is something tangible you can share with your supervisor for comment and other formative feedback.

\chapter{Stage 2: First research increment}

\section{Introducing stage 2}
Stage 2 assumes that you have completed your research proposal at Stage 1, and possibly discussed it with your supervisor, who should have helped you assess whether it is appropriate for your course. If your proposal still requires some `remedial' work, you should carry that out before moving on.

With reference to our 5-stage framework, the activities which are in focus in Stage 2 are recalled in Table 1, which also provides some guidelines for your interaction with your supervisor during this stage.

Stage 2 assumes that you have now a fair idea of how the research process works, having been through one iteration in Stage 1, and mastered sufficient basic research skills to be able to carry on your research more independently from now on, albeit still with your supervisor's support.

Therefore from now on, this handbook will will not take you through the research process step by step as it did in Stage 1, but focus on additional knowledge and skills you'll require instead.

Table 1 - Stage 2 activities

\begin{table}[htbp]
\begin{minipage}{\linewidth}
\setlength{\tymax}{0.5\linewidth}
\centering
\small
\begin{tabulary}{\textwidth}{@{}llll@{}} \toprule
 \textbf{Research activities} & \textbf{Stage 2} \textbf{(15\% of project length)} & \textbf{Effort within stage} & \textbf{Suggested focus of your interaction with your supervisor} \\
\midrule

 \textbf{Identifying the research problem} & Adjust, if needed & 3\% & \\
 \textbf{Reviewing the literature} & Complete full draft of literature review Draft critical summary of key insights from literature review & 25\% & Suitability of literature review structure and narrative flow, including appropriate logical argumentation, and demonstrating critical thinking in deriving insights \\
 \textbf{Setting research aim and objectives} & Adjust, if needed & 2\% & \\
 \textbf{Developing the research design} & Develop research design, with detailed consideration of data and evidence, research strategy and research methods, including an brief review of the related academic literature Review ethical assessment & 40\% & Well thought-out, appropriate and coherent research design overall, showing good understanding of chosen research strategies and methods, supported by related citations, and appropriate justification in terms of research aim and objectives \\
 \textbf{Gathering and analysing evidence} & n\slash a & 0\% & \\
 \textbf{Interpreting and evaluating findings} & n\slash a & 0\% & \\
 \textbf{Reporting, critical reflection and conclusions} & Assess research progress and write up Stage 2 report & 25\% & Any further improvements required, particularly in relation to critical thinking and academic writing \\
 \textbf{Work planning and risk management} & At stage start, re-assess risk and work plan and adjust as needed If you have received feedback from supervisor on your previous stage work, adjust plan to include any revision recommended & 5\% & Any major adjustment required to address deficiencies or manage risk \\
\bottomrule

\end{tabulary}
\end{minipage}
\end{table}

\begin{question}[subtitle={Activity: Understanding the effort needed in this stage}] Consider Table 1 carefully, taking notice of the entries in the `Effort within stage' column. Make a note of the activities which are most prominent in this stage and what is expected under each.

\begin{solution}Developing the research design will constitute your major effort in this stage (40\% of the study time), with completing your literature review still requiring some significant effort.
\end{solution}\end{question}
%%Hack to correct tcbox behaviour
\color{black}

\section{Literature review in stage 2}
In Stage 2, you should continue to read and critically summarise the literature to reach a complete draft, building on your draft literature review and further literature review plan from Stage 1.

This means that the work on your literature review in this stage will be similar to what you have done already, and the knowledge, process and techniques you acquired and applied in Stage 1 will still be relevant (if needed, refresh your understanding by returning to Section XX in the previous chapter).

\subsubsection{Synthesising: consolidating what you have learnt and you start summarising your understanding based on your ongoing reading}
The road from the literature to your dissertation is long and winding, but needn't be unduly so -- by which we mean you can start writing straight away refining what you have over time to reflect your understanding. This brings us to synthesis -- the act of bringing ideas together into a cogent whole.

That cogent whole will eventually be your narrative for the dissertation -- the spell-binding story you create to convince your examiner\footnote{Eventually, your examiners, first your supervisor, family, friends, anyone else that will read it{\ldots}} that you have made a contribution to knowledge. Of course, \emph{at this point,} you don't have an actual contribution to write about\footnote{Of course, you may have guiding ideas{\ldots}}.

What you can start writing about at this point is the literature and the knowledge gaps that you have found there. This is the literature review, discussed next.

\subsubsection{Synthesising}
One you have selected, read and understood\footnote{At this point, you may have only a rudimentary understanding, but that will depend as you bring your synthesis together.} a number of relevant articles, you can start develop the narrative to include in your literature review.

There are two key skills to demonstrate through this task.

Firstly, you will need to apply and demonstrate critical thinking, that is maintaining an objective position by weighing up all sides of an argument, evaluating their strengths and weaknesses, and testing how sound the claims made and their supporting evidence are. This is a substantial task.

At its essence, critical thinking is the skill of systematically asking questions that get you ``under the hood'' of the research -- perhaps into the hidden crevices into which no one has looked before. Part of this is searching for a lack of evidence or poor reasoning behind an argument, experiment or the application of other research tools, instead of accepting what you read `at face value.'

Critical thinking develops with time and practice. The trick is to balance of scepticism: choosing the arguments or experiments to take issue with -- those that are less solid -- and leaving the rest.

\begin{tip}{TIP:}TIP: TIP: you may not believe this, but there's a notion in software engineering of ``coding bad smells'' grew in popularity around the turn of the century. Equally unbelievable is that coding bad smells arose from the parenthood experiences of one famous software engineer\footnote{Which you can read about here: \href{http://www-public.tem-tsp.eu/~gibson/Teaching/Teaching-ReadingMaterial/BeckFowler99.pdf}{http:\slash \slash www--public.tem--tsp.eu\slash \ensuremath{\sim}gibson\slash Teaching\slash Teaching--ReadingMaterial\slash BeckFowler99.pdf}\footnote{\href{http://www-public.tem-tsp.eu/~gibson/Teaching/Teaching-ReadingMaterial/BeckFowler99.pdf}{http:\slash \slash www-public.tem-tsp.eu\slash \ensuremath{\sim}gibson\slash Teaching\slash Teaching-ReadingMaterial\slash BeckFowler99.pdf}}}, he ``was under the influence of the odors\emph{[sic]} of his newborn daughter at the time{\ldots}''. The point was that an experienced coding specialists just know when something is wrong with code and needs to be looked at. At the beginning of your research career, you're yet to have that experience. But read, anyway, and see if anything `smells' to you. $<$This really isn't very good$>$\end{tip}

\paragraph{}
As an academic author, critical thinking will, later, benefit your ability to build stronger arguments, avoid bias and link your claims to appropriate supporting evidence so it is well worth picking it up as a skill.

Secondly, your ability to do more than summarise an article by drawing links between it and other articles that you have found.

This is more than summarising a single article. You will, at least initially, have to identify the things that two articles agree on and disagree on. Then you'll do it for three, then four, etc, grouping articles together under various different themes that you find, the themes being relevant to your research problem.

At some point, you'll find some more or less definitive unifying topics to which you can assign the papers you are reading: ``paper 1 works with $<$Example here, from a student?$>$''

At this point, you've moved from a paper-based process to a topic-based process and your synthesis will be really coming together. So much so that, per\footnote{We`re getting a little ahead of ourselves here -- you won't have a complete literature review at this point, but you may start to feel that you can see how, eventually, it could come together{\ldots}}haps, the topics that you have found may even be used as the titles of section\footnote{Or, more likely, subsections{\ldots}} of your literature review with your critical reflection on the literature being the content of those sections which continue to grow and grow into the completed literature review.

If so, you will have adequately demonstrated your ability to synthesis from the academic literature.

Having got to this point, your completed literature review should be a self-contained piece of academic writing, which shows your critical thinking and mastering of academic writing skills, and through which you:

\begin{itemize}
\item demonstrate your understanding of key ideas and their significance to your research, particularly to framing and justifying your research problem

\item relate different ideas to each other, including arguments and counter-arguments,

\item reason through the evidence to argue the possible contribution to knowledge your research can make

\end{itemize}

\paragraph{Descriptive vs critical writing}
Many authors of fiction use descriptive writing to give vivid, detailed descriptions of their characters in the hope you will feel some empathy for them. Being able to write descriptively is a great skill for a fiction author to have.

Descriptive writing is also an essential part of academic writing, too, but its use is very different: in academic writing description is used to set the context and to provide existing evidence behind an argument you are developing. The key word\footnote{You might use different words to those used in the original, for instance, to shorten the original, but your intention will be to retain the original meaning.} is \emph{existing}: descriptive writing does not try to augment what someone else has written.

A \footnote{You summarise to introduce, typically to those not experts. A summary is likely to be little more than a sentence of introduction for any paper. You'll see examples next{\ldots}}certain amount of description is therefore necessary in any academic writing. But to present new knowledge, your examiner will be looking for something more than description -- they will be looking for a critical approach, by which you will build\footnote{You will be able to add in your generated evidence too, of course.} new arguments using what has gone before by, for instance, analysing, synthesising, and evaluating.

There perfect critique of an article will have the following components:

\begin{itemize}
\item your introduction to the paper, by saying what is involved, where it takes place, or under which circumstances. \textbf{Description answers the questions: what? who? where? when?}

\item $<$Add 1 or 2 examples from students$>$

\item your analysis, which gives your perspective on the paper, perhaps highlighting how the paper comments on your focus in reading, including its strengths and weaknesses from the perspective of the topic. \textbf{Analysis answers the questions: how? why? what if?}

\item $<$Add 1 or 2 examples from students$>$

\item Your synthesis, which explains how the parts fit into your research context, giving reasons, making comparisons, and highlighting relationships with other papers. \textbf{Synthesis answers the questions: where else? which relationships?}

\item $<$Add 1 or 2 examples from students$>$

\item your evaluation of the strengths and weaknesses of the paper from your perspective, the implications that can be made for your purposes, and the impact and value to your research. \textbf{Evaluation answers the questions: so what? what next?}

\item $<$Add 1 or 2 examples from students$>$

\end{itemize}

The boundary between these is sometimes fuzzy, of course, in the sense that is that in-depth descriptions may well start to be analytical and synthetic, and some analysis and synthesis may include a level of evaluation. Table 1, inspired by \textcite{Cottrell2005} provides some practical examples which may help you distinguish between description, analysis, synthesis and evaluation in academic writing.

Table 1 - Distinguishing between description, analysis, synthesis and evaluation academic writing

\begin{table}[htbp]
\begin{minipage}{\linewidth}
\setlength{\tymax}{0.5\linewidth}
\centering
\small
\begin{tabulary}{\textwidth}{@{}lll@{}} \toprule
 \textbf{Descriptive} & \textbf{Analytic\slash synthetic} & \textbf{Evaluative} \\
\midrule

 What happens or what something is like & Makes comparisons & Identifies significance \\
 Tells a story or the order in which things occur & Explain why things work the way they do & Demonstrate relevance \\
 Explains how to do something & Gives reasons for choices & Draws conclusions \\
 What a theory says or how something works & Structures information based on established criteria & Weighs pieces of information against one another \\
 Lists things, alternatives and options, etc. & Shows relations between pieces of information, parts of a system & Highlights strengths and weaknesses \\
 Describes a system or its components & Demonstrates how a theory works & Considers wider implications \\
\bottomrule

\end{tabulary}
\end{minipage}
\end{table}

\paragraph{Good practices in academic writing}
Academic writing follows core practices you should be aware of and apply consistently in your project work. They are:

\textbf{Arguments over descriptions}

Most of your writing should focus on well-formed arguments, i.e. on claims that are well supported by evidence, on comparing and evaluating alternative arguments, and on forming judgements on the basis of the evidence. Your writing will therefore favour analysis, synthesis and evaluation over descriptions.

In your descriptions you should also focus on essential details and keep general background information to a minimum.

\textbf{Clarity and precision}

Your writing must make it easy for the reader to follow your arguments and grasp the points you are trying to make. You should:

\begin{itemize}
\item avoid long, over-complicated and poorly punctuated sentences

\item avoid jargon, and vague or inaccurate language

\item clearly define terms and concepts that may be open to more than one interpretation to avoid misunderstanding

\item keep your audience in mind when using technical terms and explain any that your readers are unlikely to be familiar with.

\end{itemize}

\textbf{Ordering}

You need to consider carefully which information your readers need to read first to make best sense of an argument. Often, it's good to present the points that support your own argument first so that you establish your case early in the mind of the reader.

Getting the order right will take time and perhaps many iterations. As you write, you should take time to step back and think about whether your arguments follow logically from each other, or whether moving them around may make your line of reasoning clearer to the reader.

\textbf{Grouping similar points}

It is harder for the reader to navigate your text if similar points are scattered in your writing -- they may get the impression that you are repeating yourself, or they may miss important connections in your arguments.

You should strive to bring similar points together, including any analysis and evaluation that relates to them, as well as comparisons and counter arguments.

\textbf{Signposting}

Signposting avoids the reader getting lost while reading your work. It applies both to the overall structure of your document and to individual arguments within each of its sections.

Common signposting practices at document level are:

\begin{itemize}
\item a judicious choice of headings and sub-headings

\item `setting the scene' at the start of a chapter or major section to provide a roadmap of what comes next

\item summarising key points at the end of a section\slash chapter.

\end{itemize}

At an individual argument level, the use of appropriate `signal words' (see Table 1) can help the reader understand where they are in an argument.

\textbf{Good English (or other modern language you are using in your writing)}

This is an essential characteristic of all written work you are expected to produce. You should therefore proofread carefully all your writing before submission to remove as many grammatical errors and typos as possible.

\textbf{Appropriate format}

You should pay attention to stylistic requirements. For instance, all your citations and references should comply with the bibliographical stye required by your course, if any, you dissertation pages should be numbered, and all your figures and tables should also be numbered and accompanied by appropriate captions. You should check your course requirements to ensure your dissertation meets expectations.

\begin{tabulary}{\textwidth}{@{}LL@{}}
\textbf{Function} & \textbf{Possible words to use} \\
\textbf{introducing an argument, a description, a section or a chapter} & first, firstly, first of all, to begin with, initially \\
\textbf{reinforcing similarities/arguments} & similarly, equally, in the same way, also \\
\textbf{adding further evidence/arguments} & furthermore, moreover, in addition \\
\textbf{introducing alternative evidence/arguments} & alternatively, however, on the other hand, differently \\
\textbf{highlighting choices} & either/or, neither/nor \\
\textbf{contrasting ideas/arguments} & instead, by contrast, conversely, on the one hand [...] on the other \\
\textbf{drawing conclusions} & therefore, as a result, as a consequence, in conclusion, consequently, because of this \\
\end{tabulary}

Table - `Signal words' used in academic writing, inspired by Cottrell (2017).

\paragraph{Avoiding plagiarism}
To distinguish your work and contribution to knowledge from that of others, in your writing you will need to acknowledge clearly all sources you have used. This is known as referencing and its role is to give credit to others, where it is due, avoiding any possible accusation of plagiarism, that is passing off the work of others as if it were your own. We distinguish between:

\begin{itemize}
\item a citation, which is a short-cut that appears in the main body of the text to refer to a specific source

\item a reference, which is the full bibliographic information of a source you cite in your text. References are usually collected in a section at the end of a dissertation, article, report, etc.,

\end{itemize}

Intentional plagiarism implies a deliberate attempt to deceive, something that both universities and publishers take very seriously and may have severe consequences. Plagiarism, however, can also be the result of poor practices in academic writing, for instance failing to highlight in quotation marks text which has been copied \emph{verbatim} from an article, even if the article is cited. To avoid any form of plagiarism, here are two simple practices you should apply diligently in your work:

\begin{itemize}
\item If you want to include the exact sequence of words that appear in a source, then you must include such word sequence in quotation marks and add both a citation next to it and a full reference in your references section. It is, however, important that you use quotations selectively as to avoid constructing sections entirely made up of quotations, which don't make for a fluent or compelling narrative.

\item Instead of quotations, you should consider rewriting ideas and information from your sources in your own words. In this case, even if you are using your own words, you are still reporting on somebody else's work, so you must acknowledge your source with both a citation and a full reference.

\end{itemize}

If you follow these two simple rules consistently, you will avoid most plagiarism and your work is more likely to comply with your course requirements and expectations.

\begin{question}[subtitle={ACTIVITY: Looking up your University's plagiarism policy}] Look up your university policy on plagiarism and any disciplinary process related to it.

\begin{solution}Our university has strict policy on plagiarism. Intentional plagiarism can lead to severe disciplinary actions, from failing study modules to be expelled from a course. Poor academic practices are usually addressed by providing extra study support, although repeated offences can also incur in disciplinary action.

\end{solution}\end{question}
%%Hack to correct tcbox behaviour
\color{black}

\paragraph{Developing your arguments}
Academic arguments are at the core of any academic writing. But what exactly is an academic argument? Booth, Colomb and Williams (1995) suggest that an academic argument is made of 5 parts, which are connected in the way illustrated in Figure 1; they are:

\begin{itemize}
\item A claim, which is the point you are trying to make

\item The reason, which is an explanation of why the claim is made; there could be many

\item The evidence, which provides the grounds on which the claim is made

\item The warrant, which explains how the reason is relevant to the claim

\item The qualifications, which are concessions which may limit what is being claimed, for instance by acknowledging objections, alternatives, etc.

\end{itemize}

\begin{figure}[htbp]
\centering
\includegraphics[width=618pt,height=399pt]{Screenshot2022-06-20at093914.pdf}
\caption{}
\label{screenshot2022-06-20at093914}
\end{figure}

Figure 1 - The five elements of an academic argument, adapted from Booth et al., 1995. \textbf{LR: Needs redrawing as it is currently a merger from a couple of illustrations in that book}

Description

The figure summarises the five elements of an academic argument and their relationship.

End of description

Booth, Colomb and Williams (1995) distinguish between reason and evidence, and provide the following illustrative example.

\begin{example}{Argument form}\emph{Claim}: TV violence can have harmful psychological effects on children

\emph{Reason}: because frequent exposure to violent images makes them unable to distinguish fantasy from reality

\emph{Evidence}: Smith (1997) found that children ages 5--7 who watched more than three hours of violent television a day were 25 percent more likely to say that what they saw on television was ''really happening.
\end{example}

\begin{question}[subtitle={Acticity: Differentiating between reason and evidence}] Consider the example above and write down what you think the difference between reason and evidence is.

\begin{solution}According to Booth, Colomb and Williams (1995), reasons are things we think up in our mind, while evidence is somewhat ``out there'' for everybody to see and examine. While in everyday casual conversation, we can often support a claim with just a reason, that should not be the case in academic research where reasons should be backed up by evidence, as your research audience is unlikely to accept your reasons at face value.
\end{solution}\end{question}
%%Hack to correct tcbox behaviour
\color{black}

\begin{question}[subtitle={Acticity: Constructing academic arguments}] Consider your current theme analysis table. Consider some key points you wish to include in your literature review, and try to express them following Booth et al.'s structure.

\begin{guidance}Don't worry if you can't include all the five elements for each argument, but try to capture both your claim and any reasons and evidence in support.

Constructing academic arguments becomes easier with practice and we suggest you spend up to 1 hour on this activity.

\end{guidance}\end{question}
%%Hack to correct tcbox behaviour
\color{black}

\paragraph{Organising your narrative}
Your literature review should tell a story to the reader, supported by evidence from previous research and the work of others. It is therefore useful to start by creating an overview of the main arguments you wish to make, organised in a logical manner.

\begin{question}[subtitle={Activity: Developing your literature review outline}] Based on the arguments you developed in the previous activity, write an outline of your literature review which indicates which arguments to include and in which order. Review and revise your outline, until you are satisfied that all the arguments you wish to make are included and they follow logically from each other.

\begin{guidance}There is no single way to go about doing this. Here are some techniques you may like to try if you find it difficult to get started:

- Draw a mind map of the arguments you have developed and how they relate to each other. Use the map to group and sequence the arguments in your outline

- Write a sentence\slash bullet point for each argument, then re-order them to ensure they follow from one another logically

- Try different ways of grouping your arguments, for instance by theme or chronologically

- Try some free writing, then read, review and re-organise to ensure there is a logical flow

You should spend up to 2 hours on this activity.
\end{guidance}\end{question}
%%Hack to correct tcbox behaviour
\color{black}

\paragraph{}
Once you are happy with your outline, you can start filling in the details. Depending on how much you have read and the arguments you have been able to make, you should end up with a substantial initial draft of you literature review by the end of Stage 1.

As your literature review grows you may find that you will need to structure it further, for instance, by breaking it down into sections and sub-sections. This might be the case if you are covering many different topics, or some related topics should be presented together.

Setting up headings and sub-headings in a table of contents will help you think of the detailed structure of your literature review, even if some of the headings are just placeholders and will be reviewed and finalised later on.

\begin{question}[subtitle={Activity: Choosing headings and sub-headings}] Consider the current draft of your literature review and the further reading you plan to undertake. Consider if breaking it down into sections and sub-sections may help clarity and logical flow, and, in that case, draw a possible table of contents including appropriate headings and sub-headings.

Once you are happy, fit the content of your literature review to that structure.

\begin{guidance}You choice of headings and sub-headings should result in coherent sections and should:

•	indicate, in outline, to the reader how the `story' develops through the document

•	express the purpose of each section

•	accurately reflect the content of each section

•	be concise

In other words, your choice of headings and sub-headings should signpost to the reader where they are in your `story', help them follow the thread of your arguments, and allow them to locate efficiently specific content they wish to return to.

Once you fitted your content to the structure, you should check that your arguments continue to follow logically from one another.

Depending on the amount of materials you have, this might be a substantial activity, so you should set 8 hours aside to complete it.
\end{guidance}\end{question}
%%Hack to correct tcbox behaviour
\color{black}

\begin{question}[subtitle={ACTIVITY: Completing your literature review }] Consider your current draft literature review draft and plan for further review from Stage 1, also taking into account any feedback you may have received from your supervisor.

Identify and carry out all remaining work to develop a full draft of your literature review. Conclude your review with a critical summary of the key insights you have gained, which clearly identifies the knowledge gap your project is going to address, so as to justify your research problem, aim and objectives.

\begin{guidance}In this stage, synthesis, critical reflection and academic writing will be more prominent than searching, reading and assimilating content, so that you will need to pay particular attention at how the structure and narrative of your literature review develops as you incorporate more materials.

As you go along, you should continue to take notes of key insights and knowledge gaps you identify, and keep your theme analysis table and summary-comparison matrix up to date.

This is a very substantial activity and you should plan to spend about 25\% of your study time for Stage 2 to complete it.
\end{guidance}\end{question}
%%Hack to correct tcbox behaviour
\color{black}

\subsubsection{Assessing your literature review}
The following activity will help you assess your progress with your literature review and guide any further work required.

\subsubsection{Assessing scope, planning and iterating<Could go later{\ldots}>}
Even if you hit gold first time with your literature search, you are likely to require several iterations of the literature review process in order to develop your full review. In our 5-stage framework, we recommend you develop a comprehensive initial draft of your literature survey by the end of Stage 2, with the bulk of the work on gathering and assimilating articles completed in Stage 1 to allow more time for synthesis in Stage 2. At each iteration\footnote{You'll study how to reflect on things like this later in this stage.}, you should assess the work that still needs to be done and decide what should come next.

So, whether you're trying to find more to read, or updating\footnote{It`s good that you took notes of what you searched for{\ldots} you did, didn't you?} your literature review after some time away, in order to identify further articles to review, you can use different approaches. For instance, you could consider:

\begin{itemize}
\item topics currently under-explored in your theme analysis table $<$WHAT IS THIS, is it the keyword table?$>$

\item interesting ideas or sub-topics you have come across in your previous reading

\item keywords used in relevant articles you have already reviewed

\item articles cited in relevant work you have already reviewed, or that cite that work

\end{itemize}

The last point provides two complementary searching approaches which help you explore the relation of an article to other articles: backwards, to those that came before and it relies upon (its references); and forwards, to those that come after and rely on it (by citing it).

Whichever approach you follow for your next bibliographical search, you need to ensure that you keep the activity focussed on what you are trying to achieve, that is to be able to answer the questions at the start of this section to help you articulate and justify your research problem, and establish your potential contribution to knowledge. There will be opportunities later on in your project to widen your reading of the literature for other purposes.

\begin{question}[subtitle={Activity: Widening your literature review}] Select a number of topics for further exploration and gather a number of articles for further review. Then assimilate and synthesise their content as you did previously, while also updating your theme identification matrix, theme analysis table and summary-comparison matrix as you go along.

\begin{guidance}Apply any of the approaches above to select your new topics. Remember to record all new entries in your BMT, alongside your notes and summaries.

This is a substantial activity which, depending on how much material you have already gathered, may take you several days, if not weeks.

\end{guidance}\end{question}
%%Hack to correct tcbox behaviour
\color{black}

\begin{question}[subtitle={Activity: Assessing your literature review}] Assess your current draft of the literature review by applying the criteria in Table 1.

\begin{guidance}For each criterium, use the prompts to write down your own assessment and to record what is still missing: the latter will help you identify further work you will need to carry out.
\end{guidance}\end{question}
%%Hack to correct tcbox behaviour
\color{black}

\subsubsection{}
Table 1 --- Criteria for assessing your literature review

\begin{table}[htbp]
\begin{minipage}{\linewidth}
\setlength{\tymax}{0.5\linewidth}
\centering
\small
\begin{tabulary}{\textwidth}{@{}ll@{}} \toprule
 \textbf{Criteria} & \textbf{Prompts} \\
\midrule

 \textbf{Research problem underpinning} & To which extent does it demonstrate your understanding of different facets of your research problem? To which extent does it demonstrate the generality of the research problem? \\
 \textbf{Research problem justification} & To which extent does it argue that the research problem is worth investigating? \\
 \textbf{Potential contribution to knowledge} & How clearly does it articulate what the knowledge gap? How clearly does it articulate why it is significant to address it? \\
 \textbf{Logical progression} & To which extent does it include a logical progression of arguments? \\
 \textbf{Critical writing} & Are connections between ideas appropriately explored? Is there a good balance between description, analysis and evaluation? \\
 \textbf{Supporting references} & To which extent are all key arguments supported by appropriate references? \\
 \textbf{Format and proof reading} & Have you reviewed your writing carefully to remove typos and grammatical errors? Are all citations and references in correct bibliographical style? \\
\bottomrule

\end{tabulary}
\end{minipage}
\end{table}

\section{Research design in stage 2}

In this section we consider some fundamental concepts in research design to help you develop your understanding and inform your project choices. As you learnt in Stage 1, your research design should summarise, explain and justify how your research is conducted, developing into a detailed account of what you have done by the end of your project.

Figure 1 illustrates the relation between the basic building blocks of research design which you will encounter in this section. The \textbf{research} \textbf{methods} are the techniques you will apply in your research to collect, analyse, synthesise and present data and evidence, and to derive findings. Those methods will need to come together within a coherent \textbf{research strategy}, which makes systematic their use to address your research problem, and to meet the aim and objectives of your research. Research strategies are influenced by \textbf{philosophical traditions}, which embody sets of beliefs on the nature of what we can study, how knowledge can be generated, and what is of value in research.

\begin{figure}[htbp]
\centering
\includegraphics[width=288pt,height=333pt]{Screenshot2023-04-20at072215.pdf}
\caption{}
\label{screenshot2023-04-20at072215}
\end{figure}

Figure 1 --- How research design building blocks are related

Description

The figure illustrates how the basic building blocks of research design relate to each other: research methods are techniques applied within a system defined by the chosen research strategy, with such choice informed by underlying sets of of beliefs.

End of description

\subsubsection{Research methods}
\textbf{Research methods} are the means used in research to collect, analyse, synthesise or present data and evidence, and to derive findings from them. It is worth noticing that there is a subtle distinction between data and evidence. Data is raw information with no interpretation attached --- anything you may collect, observe or gather in your research. Evidence is information interpreted to support your academic arguments. Indeed, data form the basis of evidence, so the two concepts are closely linked and often used interchangeably.

The purpose of research methods is to help you conduct your research in a systematic, rigorous, repeatable and reliable fashion. So, research methods are important in research design because they underpin the validity of the research you carry out.

Research methods are many and vary greatly, which can be confusing to the novice researcher --- and even the seasoned one at times! One source of confusion is that the same term is often used to indicate both specific techniques and procedures, and broad \textbf{research strategies} combining many.

In this handbook we will use the term research method as a synonym for research technique\slash procedure, but be aware that you may encounter other meanings while reading the academic literature.

In this section, we recall a wide range of research methods commonly applied in research, particularly at Masters level.

\subsubsection{Data collection methods}

\paragraph{Questionnaires}
A \textbf{questionnaire} is a fixed set of questions organised in a particular order used to gather answers. It can be delivered face-to-face or distributed to respondents to gather their answers. The respondents' answers constitute the generated data that is subsequently analysed by the researcher.

Questionnaires are a data generation technique applicable when:

- you wish to obtain standardised data from many people

- you seek relatively brief information from your respondents

- you expect your respondents to be able to understand and interpret the questions in a straightforward manner.

Questionnaires can be \textbf{self-administered}, in the sense that the respondents complete the questionnaire without the researcher being present, or \textbf{researcher-administered}, in which case the researcher asks the questions and writes down the responses.

\begin{question}[subtitle={Activity: Considering questionnaires}] In relation to your project aim and objectives, reflect on the extent questionnaires may be useful and which form of questionnaire is more likely to fit your needs. Jot down your answer.

\begin{guidance}It may well be that you don't consider this technique useful for your project, in which case you should articulate the reason.
\end{guidance}\end{question}
%%Hack to correct tcbox behaviour
\color{black}

\paragraph{Interviews}
An \textbf{interview} is a form of conversation between the researcher and one or more interviewees, designed by the researcher to gain insights and opinions on a specific topic. The researcher guides and controls the conversation and asks the questions. The interviewees' answers constitute the generated data that is subsequently analysed by the researcher.

An interview is a technique for data generation applicable when you wish to:

- obtain detailed information on a specific issue or topic

- ask open-ended, complex questions, which may be tackled or interpreted differently by different interviewees

- investigate sensitive issues or privileged information that interviewees may not be willing to commit to writing.

Interviews can be one-on-one, between the researcher and one interviewee at a time, or can happen in a group, with several interviewees being interviewed together by the researcher. The latter is referred to as a \textbf{focus groups}.

Interviews can be fully planned or quite open-ended. The former are termed \textbf{structured} and use pre-determined, identical questions with all the interviewees, while the latter are termed \textbf{unstructured} and typically start by introducing a topic but then let the interviewee talk freely around their ideas, experience and beliefs. Somewhere in between are \textbf{semi-structured} interviews, where the researcher selects some themes and related questions upfront, but then adapt them depending on how the conversation with the interviewee develops.

\begin{question}[subtitle={Activity: Considering interviews}] In relation to your project aim and objectives, reflect on the extent interviews may apply and which form of interview is more likely to fit your needs. Jot down your answer.

\begin{guidance}It may well be that you don't consider this technique useful for your project, in which case you should articulate the reason.
\end{guidance}\end{question}
%%Hack to correct tcbox behaviour
\color{black}

\paragraph{Delphi technique}
With the \textbf{Delphi} technique, a group of experts are consulted with a view of obtaining a consensus on a particular issue or topic. It involves an iterative process of collecting, synthesising and circulating anonymous judgements from those experts to eventually arrive at a consensual view. More precisely, each subject expert is initially consulted separately by the researcher, who then anonymises and collates the group responses and circulate them to the same group of experts. The process is repeated until a consensus is reached.

This technique is based on the idea that a group of people are more likely to arrive at an informed and valid position than an individual, with anonymity preventing interpersonal relationships from influencing the outcome. The judgements and consensus gathered constitute the generated data that is subsequently analysed by the researcher.

The Delphi technique is particularly suited to situations in which the researcher wishes to improve their understanding of an under-explored problem or issue in order to inform decision-making.

\begin{question}[subtitle={Activity: Considering the Delphi technique}] In relation to your project aim and objectives, reflect on the extent the Delphi technique may apply. Jot down your answer.

\begin{guidance}It may well be that you don't consider this technique useful for your project, in which case you should articulate the reason.

\end{guidance}\end{question}
%%Hack to correct tcbox behaviour
\color{black}

\paragraph{Observations and measurements}
\textbf{Observations} are used in research to find out what people actually do or what actually happens in a particular context, rather than what has been reported about it. Observations can be of people's behaviour or interactions, e.g., observing a formal meeting in an organisation, or of events and processes, for instance observing a queue at the post-office or a computer-controlled production plant. As such, observations can generate all kinds of data, qualitative and quantitative. Quantitative observations are often referred to as \textbf{measurements}, e.g., the length of time a particular customer has spent waiting in the queue at the post office.

There are two main types of research observation, \textbf{systematic} vs. \textbf{participant}. The former is when the researcher decides in advance what to observe, the schedule of observations and what to record. For example, the observation of a queue at the post office could be planned to take place over a certain week or month, and recordings may include time of arrival and departure of each customer, average and maximum length of the queue, average service time, etc.

In participant observations, the researcher participates directly in the situation under study and produces a rich description of what happens based on what they experience. For instance, in relation to the previous example, the researcher might join the queue at the post office and record their experience in great detail, or even join the staff in the post office to understand why queues are longer or shorter for certain tasks.

\begin{question}[subtitle={Activity: Considering observations and measurements}] In relation to your project aim and objectives, reflect on the extent observations may apply and which type of observations may suit your needs. Jot down your answer.

\begin{guidance}It may well be that you don't consider this technique useful for your project, in which case you should articulate the reason.
\end{guidance}\end{question}
%%Hack to correct tcbox behaviour
\color{black}

\paragraph{Using existing documents or data sets}
The previous techniques can be used to generate primary evidence.

Academic research, however, can also use secondary evidence as its starting point. This can be represented by existing documents of may forms, from academic articles to documents found in organisations, e.g. laws, policies and procedures, reports, formal minutes of meetings, informal communications, etc. Similarly, there are plenty of publicly available data sets that can be used, created by academic communities or private and public organisations, such as business financial data or statistical data from the UK Office for National Statistics or data from social network platforms. As already noted, the academic literature at the core of your literature review is a form of secondary evidence.

\begin{question}[subtitle={Activity: Considering reusing existing evidence}] In relation to your project aim and objectives, reflect on the extent you may be able to reuse secondary evidence, which form this may take and which sources of secondary evidence are available to you. Jot down your answer.

\begin{guidance}It may well be that you won't use secondary evidence (other than the academic literature) for your project, in which case you should articulate the reason.

\end{guidance}\end{question}
%%Hack to correct tcbox behaviour
\color{black}

\subsubsection{Data analysis methods}

\paragraph{Spreadsheets, tables and charts}
Spreadsheets, tables, charts and graphs are the bread and butter of data analysis. They are applicable to all kinds of data, and can be used to summarise and visualise data, and identify interesting patterns. I will assume you are already familiar with this topic, so that this is only a brief overview to refresh your knowledge.

A \textbf{spreadsheet} is a digital tool you can use to capture, display and manipulate data arranged in tables, that is arranged in rows and columns. Common spreadsheets include Microsoft Excel, Apple Numbers and Google Sheets. Spreadsheets are among the most used digital tools, so it is likely you are already familiar with at least their basic functionalities. Spreadsheets have become quite sophisticated tools, including all sort of charts and graphs, as well as programmatic capabilities which allow you to code quite complex data manipulation functions. Some of those advanced functionalities could be advantageous to your research, so it is worth spending some time considering what they can offer to your project. There are plenty of tutorials and other documentation online you can use to learn more.

\begin{question}[subtitle={Activity: Understanding features of spreadsheets}] Conduct a web search on your spreadsheet of choice to find summaries and tutorials on core and advanced functionalities which may be useful in your project.

\begin{guidance}You should make a note of how you could use such functionality in your research, and bookmark web pages of particular relevance to which you may want to return to later on.
\end{guidance}\end{question}
%%Hack to correct tcbox behaviour
\color{black}

\paragraph{}
\textbf{Charts} and \textbf{graphs} are visual representations of data, usually employed to summarise data, and highlight or identify interesting patterns. All modern spreadsheets provide a wide range of charts and graphs which allow you to visualise data from spreadsheet tables.

\begin{question}[subtitle={Activity: Investigating visualisation functionalities}] Investigate the visualisation functionalities of your spreadsheet of choice, identifying those which may be useful in your project.

\begin{guidance}You should make a note of how you may use such visualisations in your research and keep track of useful examples and instructions to which you may return later on.
\end{guidance}\end{question}
%%Hack to correct tcbox behaviour
\color{black}

\paragraph{}
Alongside spreadsheets a growing number of \textbf{data analytics} tools are also available: these are sophisticated digital tools which extend spreadsheet capabilities for collating and visualising data to include some degree of automated analysis, both statistical and based on Machine Learning algorithms. Tools like Tableau and Power BI are notable examples: both are available in free versions for community use and for study.

\begin{question}[subtitle={Activity: Understanding features of data analytics tools}] Conduct a web search on Tableau and Power BI to identify tutorials and other documentation which illustrate their key functionalities. Consider if such tools may be of use in your project, and jot down their potential use.

\begin{guidance}You should identify which of such tools, if any, you may be able to use in your research. You should consider the time which you will require to become proficient in their use, and bookmark web pages of particular relevance to which you may want to return to later on.
\end{guidance}\end{question}
%%Hack to correct tcbox behaviour
\color{black}

\paragraph{Statistical analysis}
\textbf{Statistical analysis} is a broad collection of techniques used to investigate trends, patterns, and relationships in quantitative data. It is a well established field of study with wide application across all kinds of research, and well developed tool support. IN particular, both spreadsheets and data analytics tools include functionalities which allow you to calculate statistical measures on data, check statistical relationships between variables and data sets, and generate basic statistical models. Bespoke statistical tools also exist for more advanced statistical analysis and modelling, like IBM SSPS or Minitab, which are also available in free versions for students.

T802 will not require you to use any advanced statistics, and will assume you are already familiar with some of the basic concepts, should you need to apply them, particularly descriptive statistics and correlation. However, should your project require advanced statistical analysis, then you will need to become proficient in good time to carry out your analysis and interpretation of findings in the second half of your project work.

\begin{question}[subtitle={Activity: Considering statistical analysis}] In the context of your project and in relation to the types of data you will require, reflect on whether you may apply statistical analysis, either basic or advanced techniques. Jot down your answer.

\begin{guidance}It may well be that you don't consider these techniques useful for your project, in which case you should articulate the reason.

\end{guidance}\end{question}
%%Hack to correct tcbox behaviour
\color{black}

\paragraph{Thematic analysis}
\textbf{Thematic analysis} is a way of analysing qualitative data, particularly texts, e.g., transcriptions of interviews or answers to questionnaires or existing text documents, in order to find out something about people's views, opinions, knowledge, etc.

At its core is the identification by the researcher of recurring themes, their definition and relationships: this relies on the researcher's judgement and it is quite subjective.

There are two basic types of thematic analysis, inductive and deductive. In \textbf{inductive thematic} \textbf{analysis} the themes emerge from the data and are not pre-defined, while in \textbf{deductive thematic} \textbf{analysis} some themes are established upfront, possibly based on an existing theory or previous research, which are then used to analyse the qualitative data.

\begin{question}[subtitle={Activity: Considering thematic analysis}] In the context of your project and in relation to the types of data you will require, reflect on whether you may apply thematic analysis. Jot down your answer.

\begin{guidance}It may well be that you don't consider this technique useful for your project, in which case you should articulate the reason.
\end{guidance}\end{question}
%%Hack to correct tcbox behaviour
\color{black}

\paragraph{Content analysis}
\textbf{Content analysis} is a way to investigate certain words, themes, or concepts in qualitative data, whether text, images, videos, or other.

It can be either quantitative, where the focus is on, for instance, counting the occurrence of those words, themes or concepts, or qualitative, where the focus is on interpreting and understanding their meaning and relationships. As such it can be used for many purposes, from discovering and understanding patterns, to looking at intentions behind what is expressed, or to highlight differences of use in different contexts.

\begin{question}[subtitle={Activity: Considering content analysis}] In the context of your project and in relation to the types of data you will require, reflect on whether you may apply content analysis. Jot down your answer.

\begin{guidance}It may well be that you don't consider this technique useful for your project, in which case you should articulate the reason.
\end{guidance}\end{question}
%%Hack to correct tcbox behaviour
\color{black}

\subsubsection{Modelling methods}
At its essence, a \textbf{model} is an abstraction or representation of something, be that a system, a structure or a behaviour. Modelling is used across many disciplines, so a vast repertoire of modelling techniques exist.

Possibly the most important thing you must remember about modelling is expressed by the following oft-cited aphorism:

``All models are wrong, some are useful'' (Box, 1976)

which makes clear that a model should not be regarded as a faithful replication of some reality, but as a tool to investigate some aspects of that reality.

In this section, I will cover a very small set of modelling techniques, which are particularly relevant to T802 projects. A lot more can be found in the academic literature and beyond.

\textbf{Reference}

Box, George E. P. (1976), ``Science and statistics'' (PDF), Journal of the American Statistical Association, 71 (356): 791--799, doi:10.1080\slash 01621459.1976.10480949.

\paragraph{Systems diagrams}
You can use \textbf{systems diagrams} to help you understand the structure of a situation of interest that can be rendered as a system. The term `system' is meant in its widest possible meaning of a set of components interconnected for a purpose. This is a very general and versatile technique that you can apply to all sorts of real-world situations. If you have studied a systems thinking and practice module for your qualification, you will be already familiar with this technique.

There are many different kinds of systems diagrams. For examples, \textbf{systems maps} allow you to sketch the structure of a system by identifying key components and sub-systems. They can be extended to show how those elements influence each other, in which case they are called \textbf{influence diagrams}. On the other hand, \textbf{causal loop diagrams} are used to capture cause-and-effect relations in a system, hence model certain dynamics of that system, particularly underlying feedback structures. They can be turned into \textbf{stock and flow diagrams} by adding quantitative information, so that this type of diagram is useful both for analysis and simulation of systems behaviour.

\textbf{Study Note}: The system thinking diagramming tutorials on Open Learn are a good starting point to look up these techniques. Link: \href{https://www.open.edu/openlearn/science-maths-technology/across-the-sciences/systems-thinking-diagramming-tutorials}{https:\slash \slash www.open.edu\slash openlearn\slash science-maths-technology\slash across-the-sciences\slash systems-thinking-diagramming-tutorials}. For an application of these techniques on a particular case study you can also consider \href{https://www.open.edu/openlearn/science-maths-technology/computing-ict/diagramming-development-1-bounding-realities/content-section-0?active-tab=description-tab}{https:\slash \slash www.open.edu\slash openlearn\slash science-maths-technology\slash computing-ict\slash diagramming-development-1-bounding-realities\slash content-section-0?active-tab=description-tab}

System diagrams have an accepted structure, format and notation but what you choose to describe and include within a system and its components will depend on your own viewpoint. Systems diagrams can be shared with others as learning devices to promote more understanding of a situation.

\begin{question}[subtitle={Activity: Considering systems diagrams}] Reflect on whether you may use systems diagrams in your research, which particular kind may be of particular use and for which purpose. Justify your answer. Jot down your answer.

\begin{guidance}It may well be that you don't consider this technique useful for your project, in which case you should articulate the reason.
\end{guidance}\end{question}
%%Hack to correct tcbox behaviour
\color{black}

\paragraph{UML modelling}
UML (\textbf{Unified Modeling Language}) is a graphical language for visualising, specifying or documenting various artefacts in the process of developing software systems. If you have studied a software engineering module for your qualification, you may be already familiar with UML.

UML can be used as a `sketching' language, to capture elements of systems informally, like you can do with systems diagrams, or as a `blueprinting' language to specify precisely how elements of a software system will be developed. Different kinds of UML diagrams exist, so that you can model elements of software systems both in terms of their structures and behaviours, and their interactions with end-users.

In a T802 project, UML can be used to help you understand existing systems in context, or plan the development of new innovative artefacts.

\begin{question}[subtitle={Activity: Considering ML modelling}] In the context of your project, reflect on whether you may apply UML modelling and for which purpose. Jot down your answer.

\begin{guidance}It may well be that you don't consider this technique useful for your project, in which case you should articulate the reason.
\end{guidance}\end{question}
%%Hack to correct tcbox behaviour
\color{black}

\paragraph{Problem diagrams}
\textbf{Problem diagrams} have their roots in software requirements engineering as a diagrammatic technique to capture requirements in a real-world context to inform the specification of a new software system to satisfy them. They have been subsequently generalised for application to general engineering problems for which some novel solution artefact is to be developed in a real-world context, to guide design and ensure fitness-for-purpose. Problem diagrams span the problem-solution space divide by focusing on those phenomena that characterise a problem and constrain its solution.

In the context of T802 projects, problem diagrams can help you develop a good understanding of real-world requirements in context and explore both constraints and effects of designing new systems for that context.

\begin{question}[subtitle={Activity: Considering problem diagrams}] In the context of your project, reflect on whether you may use problem diagrams and for which purpose. Jot down your answer.

\begin{guidance}It may well be that you don't consider this technique useful for your project, in which case you should articulate the reason.

\end{guidance}\end{question}
%%Hack to correct tcbox behaviour
\color{black}

\paragraph{Statistical modelling}
Many \textbf{statistical modelling techniques} exist. The most commonly applied include those used to model relations between variables, e.g., how crop yields relate to environmental factors, such as soil quality or meteorological conditions, or to model real-world processes, e.g., the spreading of a disease in a population. Once a statistical model is defined, it can then be used to make predictions of what might happen in the real world.

As per advanced statistical analysis, statistical modelling requires well developed statistical knowledge and skills, which you should already possess if you are considering their use in your T802 project.

\begin{question}[subtitle={Activity: Considering statistical modelling}] In the context of your project, reflect on whether you may need statistical modelling and for which purpose. Jot down your answer.

\begin{guidance}It may well be that you don't consider this technique useful for your project, in which case you should articulate the reason.
\end{guidance}\end{question}
%%Hack to correct tcbox behaviour
\color{black}

\subsubsection{Research strategies}

A \textbf{research strategy} is a systematisation of a set of research methods, applied in order to address research problems of a particular kind. The term \emph{methodology} is also sometimes used in the literature with a similar meaning, although methodology also means the study of research methods, so it is an overloaded term this handbook will avoid.

This section provides an overview of some of the best known and most commonly applied research strategies, particularly at Masters level. Note that this not an exhaustive account: for instance it does not cover observational research and ethnography, which are often applied in the social sciences, to observe participants and phenomena in their natural setting usually over long periods of time, so that they are not suited to the time constraints of a Masters project.

\subsubsection{Survey research}
\textbf{Survey research} aims to gain insights which are valid across a target population, by collecting data from a predefined sample in a standardised and systematic way.

A typical application of survey research is to predict the outcome of an upcoming general election by polling data from a representative sample of voters.

For your data collection, you need to identify upfront which data you will collect in a standardised matter, your target population and sample. So questionnaires or structured interviews are usually used for data collection.

In your data analysis, you seek patterns in the sample data collected to arrive at generalisations to the wider population. Statistical analysis is usually applied, possibly complemented by some thematic analysis, if open-ended questions are also included.

For survey research to be successful, you must be able to access an appropriate sample and generate a sufficient volume of data.

The advantages of this strategy are that it can produce a lot of data in a relatively short time, and you can replicate your data collection process on different samples or on the same sample at a later time. However, among its disadvantages are the depth in the data that can sometimes be lacking, its focus on what can be measured, the fact that it cannot reveal cause-and-effect relationships, and can only provide a snapshot at a particular time.

\begin{question}[subtitle={Activity: Considering survey research}] In the context of your project, reflect on whether this strategy might be appropriate and justified, and if so, which data collection and analysis methods you may adopt. Jot down your answer.

\begin{guidance}It may well be that you won't use this strategy for your project, in which case you should articulate the reason.
\end{guidance}\end{question}
%%Hack to correct tcbox behaviour
\color{black}

\subsubsection{Experimental research}
\textbf{Experiments} are used to investigate cause and effect relationships between factors by testing hypothesis or proving\slash disproving causal links.

For instance, you may run an experiment to test ways in which the use of mobile phones just before going to sleep affect people's sleeping patterns.

There are two main kinds of experiments: \textbf{laboratory experiments}, which are carried out in closed environments, such as a laboratory; and \textbf{field experiments}, which are conducted in the `real world'. Laboratory experiments are often applied in engineering and computer science research, while \textbf{field experiments} are usually applied when people are involved.

Possibly the best known kind of field experiment are clinical trials, widely applied in medicine. However, field experiments are also very popular in research which investigates technology in its social context or application of use.

In experiments, for data collection, you would need first to state the \textbf{hypothesis} to be tested: this is a tentative statement about the relationship between phenomena to be tested in the experiment. In the example above, a hypothesis to be tested might be that ``the blue light emitted by a mobile phone reduces the production of melatonin.'' As melatonin is the hormone which controls a person's sleep-wake cycle, its reduction is likely to disrupt a person's sleeping pattern. After formulating the hypothesis, you would then make detailed observations and measurements of outcomes, e.g., the amount of melatonin released by the body, and any changes that take place when particular factors are introduced or removed, e.g., the length of exposure to the blue light.

In analysing your experimental data you seek to explain causal links between factors under study, looking at your observations and measurements under different experimental conditions. Statistical analysis is widely used for data analysis.

For experiments to be successful you must be able to control factors which can affect the outcome. This is possible in laboratory experiments, while the level of control in field experiments is diminished.

Experimental research has well established processes and protocols and is particularly well suited to the consideration of cause-and-effect relations. However, it has its pros and cons. Laboratory experiments are very reliable due to the high level of control, but can be very artificial, with little or no relation to a real-world context. The opposite is true for field experiments.

\begin{question}[subtitle={Activity: Considering experimental research}] In the context of your project, reflect on whether this strategy might be appropriate, and if so, which hypotheses you may wish to test and which type of experiment you may apply. Jot down your answer.

\begin{guidance}It may well be that you won't use this strategy for your project, in which case you should articulate the reason.
\end{guidance}\end{question}
%%Hack to correct tcbox behaviour
\color{black}

\subsubsection{Design science research}
\textbf{Design science} seeks to generate new knowledge about a significant problem or its solution via the design of an artefact. It simultaneously generates knowledge about the problem, the artefact and the method used to design it. Artefact indicates anything made by humans, so this is a very broad definition, encompassing all that does not exist in nature.

Lots of research in Computing is an expression of design science, for instance designing new algorithms able to emulate human cognition.

More than data collection and analysis, in design science you need to follow a process of articulating the problem, and designing, constructing and evaluating a solution artefact. In doing so, you shed new insights on the problem, and argue how the solution and solution process contribute new knowledge. As a result, modelling techniques are widely applied, possibly informed by data collection techniques, like reviewing existing documents or interviews with stakeholders and experts. \textbf{Prototyping} is often used to produce proof-of-concept artefacts to test or demonstrate the design.

For design science research to be successful you must be able to argue that it is not `normal' design, that is you are not simply re-implementing a solution to a well-known problem through a well-known development process.

An advantage of design science research is that it leads to tangible artefacts which fit real-world contexts, and it is particularly suited to emerging and rapidly changing technology-related fields of study, where new problems emerge all the time and known solutions are sparse or become rapidly obsolete. The latter is also a disadvantage, of course, as new solutions may be short lived. Also, it may be difficult to generalise outcomes to different real-world settings. Depending on the nature of the artefact being designed, advanced technical skills may be required.

\begin{question}[subtitle={Activity: Considering design science research}] In the context of your project, reflect on whether this strategy might be appropriate, and if so, in which ways would the artefact design be novel and contribute new knowledge. Jot down your answer.

\begin{guidance}It may well be that you won't use this strategy for your project, in which case you should articulate the reason.
\end{guidance}\end{question}
%%Hack to correct tcbox behaviour
\color{black}

\subsubsection{Case study research}
A \textbf{case study} is used to investigate in great depth a notable instance of what is under study, in its real-world context. Case studies focus on the `how?' and `why?', and what you seek can span from exploring possible questions or hypotheses for follow-up research, to providing a detailed account of a phenomenon in its natural context, to explaining why certain outcomes or phenomena have occurred.

For instance, an example of case study could be a detailed investigation of the US Equifax social security breach of 2017, in which 143 million of their consumer records were stolen by hackers. This may be descriptive of the chain of events that took place or explicative of why things happened the way they did.

Case studies require you to collect data from a great variety of sources, and to focus on depth rather than breadth. Therefore, all data collection techniques which allow you to do so may be used, from interviews to observations to studying existing documents forensically. This will lead to much qualitative data, so that qualitative methods are often needed for the analysis of the evidence.

For a case study to be conducted successfully you must be able to analyse the chosen instance holistically and in its real-world context.

Case studies allow you to study a complex situation where several factors are at play, and to explore alternative meanings and explanations. However, case studies are time-consuming, difficult to perform rigorously and with limited generalisation beyond the particular instance under study.

\begin{question}[subtitle={Activity: Considering case study research}] In the context of your project, reflect on whether this strategy might be appropriate, and which data collection and analysis methods you may adopt. You should also think of whether you will have sufficient time to collect and analyse the amount to evidence required. Jot down your answer.

\begin{guidance}It may well be that you won't use this strategy for your project, in which case you should articulate the reason.
\end{guidance}\end{question}
%%Hack to correct tcbox behaviour
\color{black}

\subsubsection{Systemic inquiry}
\textbf{Systemic inquiry} is used to explore complex, messy problematic situations involving multiple and often contrasting perspectives, with the aim of transforming the situation for social betterment.

Systemic inquiry is based on concepts and principles of systems thinking and systems practice, so that it is the research strategy of choice for T802 projects related to the Systems Thinking in Practice (STiP) Masters degree.

Situations for systemic inquiry can range from local to global. So, it may equally apply to exploring changes in practice within a local organisation, and to international responses to disruptive events such as climate change. Of course, it is highly unlikely that your T802 project will tackle a situation at a global scale!

In systemic inquiry, you must be able to articulate your personal stake in the situation, for example, a deeply felt interest or active involvement , rather than assuming and claiming unbiased passive `neutral' observation. You must also keep your own journal during the course of your research inquiry, tracking changes in your own viewpoint and how you adapted your research as a result. In some sense, a systemic inquiry is a conceptualisation of your own learning system and how it adapts to change during the research. For Masters level research you will not be expected to be fully embedded in the situation, but a successful systemic inquiry should demonstrate \textbf{reflexivity} -- reflecting on your own changing viewpoint and impact on the wider research situation.

To conduct your systemic inquiry successfully you must articulate your research problem with reference to one or more systems of interest pertaining to the problematic situation under study, and frame the research in terms of possible systems change. You must also have access to sources of different perspectives on the situation under study in order to generate your evidence. This may include both primary evidence from people involved and affected by the situation, and secondary evidence from official and grey literature associated with the situation; your own research journal will also be a source of evidence. In terms of methods, a systemic inquiry is essentially a qualitative endeavour in which you can complement traditional research methods with other tools and techniques from your existing repertoire of expertise and professional tradition: this is known as \textbf{bricolage} research (Kincheloe, 2011).

Study Note: Grey literature is the term used for the collection of information produced by organisations whose primary or commercial remit is not publishing, such as as academia, government bodies, or non-publishing businesses and industries. It includes pre-publication and non-peer-reviewed articles, theses and dissertations, research and committee reports, government reports, conference papers, accounts of ongoing research, etc.

Systemic inquiry helps you make sense of complex situations of change and uncertainty, while acknowledging that the research will not deliver `certainty' in terms of `problem-solving' associated with complex situations. Instead, with its STiP principles of humility, empathy, and inevitable fallibility, reflexivity can support development of trust amongst participants, including trust with the researcher in co-exploring the situation.

\begin{question}[subtitle={Activity: Considering systemic inquiry}] In the context of your project, reflect on whether a systemic inquiry might be appropriate, and which data collection\slash analysis\slash modeling methods you may adopt. Jot down your answer.

\begin{guidance}It may well be that you won't use this strategy for your project, in which case you should articulate the reason.
\end{guidance}\end{question}
%%Hack to correct tcbox behaviour
\color{black}

\subsubsection{}
\textbf{References}

Kincheloe, J. L. (2011). Describing the bricolage: Conceptualizing a new rigor in qualitative research. In Key works in critical pedagogy (pp. 177--189). Brill.

\subsubsection{Mixed methods research}
\textbf{Mixed methods research} combines quantitative and qualitative methods to gain different perspectives on phenomena of interest, by exploring connections and contradictions between quantitative and qualitative data.

For instance, in looking at acceptance of a new technology, a mixed methods approach could consider both levels of adoption and demographics , and the reasons behind adoption or otherwise, possibly to inform further development of the technology.

\textbf{Study note}: Mixed-method research should not be confused with multi-method research, which simply indicates the use of many methods, possibly all qualitative or quantitative. It should also not be confused with bricolage -- the adoption and continual adaptation of methods drawing on a practitioner's own experience of using different methods in different situations.

Data collection and analysis will depend on the particular combination of methods selected. An important aspect is the consideration of how connections between findings are established, through comparing and contrasting data from the different methods applied. This is also referred to as \textbf{triangulation}.

The main advantage of mixed methods research is that it can provide a more holistic understanding of the phenomena under study, and facilitate different avenues for exploration. It is particularly suited to situations in which neither quantitative nor qualitative methods alone can provide sufficient insights. However, mixed methods make research design more complex and demanding in terms of execution time, skills required and data variety to handle and analyse.

\begin{question}[subtitle={Activity: Considering mixed methods research}] In the context of your project, reflect on whether this strategy might be appropriate. If so, which methods would you combine in your project? Jot down your answer.

\begin{guidance}It may well be that you won't use this strategy for your project, in which case you should articulate the reason.
\end{guidance}\end{question}
%%Hack to correct tcbox behaviour
\color{black}

\subsubsection{Systematic review research}
A \textbf{systematic review} is a literature review linked to a clearly defined research problem or question. It uses a rigorous set of criteria to identify, select, and critically appraise relevant research from previously published studies in order to generate a scholarly synthesis of the evidence in relation to that problem or question. Such a synthesis is meant to advance a field of research.

For example, a systematic review of randomised controlled trials on the effectiveness of a specific medical treatment could be used to advance evidence-based medicine.

In a systematic review you only use secondary evidence from published studies. You must decide upfront your research problem\slash question and the set of criteria you will use to select, summarise and evaluate those studies. The type of analysis you will conduct will depend on the nature of the evidence you are considering and combining. In \textbf{narrative reviews}, a narrative synthesis is produced, while in \textbf{meta-analysis}, statistical techniques are used to analyse and combine results.

To be successful, a systematic review has to be both systematic and extensive, which requires the researcher to have a very grasp of the subject area, in order to establish appropriate criteria and make a novel contribution to knowledge.

Because of their explicit set of criteria, systematic reviews are considered transparent, reliable, and easy to replicate. However, they can be very time-consuming due to the large body of work to review. Also, in striving to piece together evidence from potentially very different studies, they may obscure important differences. Narrative reviews may also be subject to bias.

\begin{question}[subtitle={Activity: Considering systematic review research}] In the context of your project, reflect on whether this strategy might be appropriate. Jot down your answer.

\begin{guidance}It may well be that you won't use this strategy for your project, in which case you should articulate the reason.
\end{guidance}\end{question}
%%Hack to correct tcbox behaviour
\color{black}

\subsubsection{Philosophical traditions}
Research methods and research strategies are strongly related to \textbf{philosophical traditions}, which are world-views that inform how one should conduct research. Philosophical traditions may sound a bit esoteric, but they matter in that they make explicit assumptions behind research design choices, influencing what a researcher chooses to research and the way they may go about collecting evidence or interpreting findings.

The term \emph{research paradigm} is also used in the literature with a similar meaning.

\paragraph{}
Each philosophical tradition embodies a set of beliefs around three fundamental philosophical issues:

- The nature of our world, which relates to questions such as: What is there? What kind of categories do things belong to? How are those categories related? The part of philosophy dealing with these questions is called \textbf{ontology}. In research design, ontology determines which phenomena are there to be studied as part of the research, and underlies our experience of the world. Hence, ontology is closely connected with the kind of observations we make or evidence we gather.

- How knowledge is acquired, which relates to questions such as: What does it mean to know something? How can one claim to know something? What makes a belief justified? The part of philosophy dealing with these questions is called \textbf{epistemology}. In research design, epistemology is closely related to research methods for knowledge creation and validation.

- What are the values, especially in relation to ethics, which relates to questions such as: What is good or bad? What is right or wrong? Where do values come from? How do we justify our values? The part of philosophy dealing with these questions is called \textbf{axiology}. In research design, axiology is closely related to ethical considerations when planning or executing research.

In what follows, I will provide a brief introduction to some of the better known and most often cited traditions. However, you should be aware that their definitions are not universal, their boundaries not clear-cut, and it is very rarely the case that a research design will fit a specific tradition neatly. You should, instead, consider each of these traditions as a `wrapper' of convenience for a set of beliefs on research practice which have emerged from different disciplines and cultures, and also be aware that such beliefs have changed over time, and continue to do so.

\subsubsection{Positivism}
\textbf{Positivism} is perhaps the oldest tradition, with roots in the natural sciences. It sees the world as ordered and regular, with universal laws governing its functioning, and assumes it can be investigated objectively.

Specifically, positivism encompasses the following set of beliefs:

\begin{itemize}
\item There is a physical world which exists `out there' and can be observed and measured. This also implies that all researchers will observe and measure the same phenomena in exactly the same way.

\item Through observations and measurements, the researcher can produce models of how the world functions, which are `true' explanations of the aspects of the world under study. This also implies that only one true explanation exists.

\item Truths about the world are perfectively objective and independent of the researcher's values or beliefs. This means that all researchers will arrive at the same truth.

\item Research is based on the empirical testing of theories or hypothesis, leading to either confirmation or rejection (a.k.a. `refutation'). As there can only be one truth, either the theory or hypothesis tested is that truth, in which case all subsequent tests will confirm it, or it is not that truth, in which case at some point a test will reject it. The term refutation is used to indicate that a truth, albeit universal, is always tentative: it will be valid until somebody comes up with a test to reject it.

\item Research seeks universal laws and irrefutable facts. This means that re-testing such laws or facts should always confirm them, if they are indeed truths.

\end{itemize}

For instance, starting with the hypothesis that `all swans are white', a positivist researcher would set as a test to look for swans and observe their colour. If all swans are seen white, then the hypothesis would be confirmed, if not, then it would be rejected. If the hypothesis is confirmed, then the truth that `all swans are white' is added to the body of knowledge and will remain so until another test will lead to a rejection --- indeed that's what English people believed until they first spotted a black swan in Australia!

\begin{question}[subtitle={Activity: Summarising positivism}] Given these beliefs, what does positivism assume of the nature of the world (ontology), how knowledge is acquired (epistemology), and what is of value in research (axiology)?

\begin{solution}Ontology: the world exists independently of the researcher, and can be observed and measured objectively.

Epistemology: there are universal truths, which can be acquired by empirical testing of theories and hypothesis. Tests can lead to either confirmation or rejection. Confirmed theories and hypothesis are added to the body of knowledge.

Axiology: positivism values objectivity above all, and dismisses individual's subjective views or experience.
\end{solution}\end{question}
%%Hack to correct tcbox behaviour
\color{black}

\paragraph{}
Positivism has attracted criticism particularly from the social sciences, which consider some of its beliefs untenable, primarily that researchers are totally objective and not influenced by their own values and beliefs, or that knowledge is made of perfectly generalisable truths. This has led to other traditions, which we consider next.

\subsubsection{Interpretivism\slash Constructivism}
With its roots in the social sciences, \textbf{interpretivism} seeks to identify, explore and explain phenomena in social settings, acknowledging that people perceive the world in different ways, mediated by their beliefs, attitudes and values.

Specifically, interpretivism encompasses the following set of beliefs:

\begin{itemize}
\item Different individuals, groups or cultures perceive the world differently and what people consider real is a construction of their mind --- leading to the term \textbf{constructivism} also being used.

\item The researcher is not neutral, and their perceptions of the world are influenced by their values or beliefs. This implies that different researchers can perceive the same phenomena in different ways, and there is no single truth or single explanation of the world.

\item As there are different perceptions of reality, communication among groups of individuals is the only way of constructing some shared meaning or understanding, and this will change over time.

\item As researchers are influenced by their own values and beliefs, they will arrive at different interpretations as a result of their observations. The strengths of their interpretations will depend on the strengths of the evidence and arguments their interpretations are based upon.

\item Research is based on studying people and other phenomena in their `natural' context. Such a context can be unique, so that interpretations based on observations may not be generalisable to other contexts.

\end{itemize}

\begin{question}[subtitle={Activity: Summarising interpretivism/constructivism}] Given these beliefs, what does interpretivism assume of the nature of the world (ontology), how knowledge is acquired (epistemology), and what is of value in research (axiology)?

\begin{solution}Ontology: the researcher acknowledges that they perceive the world based on their belief, values and culture.

Epistemology: the researcher will offer interpretations based on observations in a social context. Different researchers may offer different interpretations. All knowledge is constructed and shared understanding is reached through communication. Interpretations in one context may not be generalisable to other social contexts.

Axiology: The researcher's values and beliefs matter. The strength of their interpretations will depend on the strengths of the evidence and arguments in support.

\end{solution}\end{question}
%%Hack to correct tcbox behaviour
\color{black}

\subsubsection{Critical theory}
Perhaps not as well established as the previous traditions, \textbf{critical theory} originated in the fields of sociology, philosophy and political theory.

Like interpretivism, it assumes multiple interpretations of reality in social contexts. However, it goes a step further by asserting that reality is shaped by those who are powerful, who legitimate particular ways of perceiving the world: `truth' is inherently political, defined by those in charge to the disadvantage of many, and challenged by those who wish to promote equality.

As a result, critical researchers seek to challenge the status quo and perceive research as transformative at a social level, confronting ideology and trying to discover and challenge the mechanisms through which exploitation and disadvantage are perpetuated in society.

\begin{question}[subtitle={Activity: Summarising critical theory}] Given these characteristics, what does critical theory assume of the nature of the world (ontology) , how knowledge is acquired (epistemology), and what is of value in research (axiology)?

\begin{solution}Ontology: reality is the product of power relations, shaped by those who are powerful and there are disadvantages for many.

Epistemology: the researcher confronts ideology and tries to discover the truth of exploitation and the mechanisms by which disadvantage is perpetuated to challenge the status quo and promote social justice and equality.

Axiology: The researcher has the moral responsibility to make things better in society.

\end{solution}\end{question}
%%Hack to correct tcbox behaviour
\color{black}

\subsubsection{Indigeneous}
The traditions I have described so far are attracting increasing criticisms in that they are seen as Western-European centric and often imposed on other indigenous cultures as a result of colonialism.

In counterposition, an indigenous research tradition has emerged with a social and political agenda of decolonisation of indigenous societies. It emphasises the connection between the researcher and their own culture, in the sense that cultural practices and forms of expressions should be reflected in the way the research is conducted, including language, metaphors, oral traditions and knowledge systems. It also advocates an holistic approach which strives to reach a balance between different areas of life, integrating intellectual, social, political, economic, psychological and spiritual dimensions.

\begin{question}[subtitle={Activity: Summarising indigenous traditions}] Given these characteristics, what does the indigenous tradition assume of the nature of the world (ontology), how knowledge is acquired (epistemology), and what is of value in research (axiology)?

\begin{solution}Ontology: reality is determined by the indigenous culture, to which the researcher is strongly connected.

Epistemology: this is determined by indigenous knowledge systems, cultural practices and forms of expressions.

Axiology: The researcher has a social and political agenda of decolonisation of indigenous societies.

\end{solution}\end{question}
%%Hack to correct tcbox behaviour
\color{black}

\subsubsection{Sketching your research design}
It is time for you to put what you have leant into practice to produce a first sketch of the research design for your project. The next set of activities will help you summarise your choices in relation to each of its building blocks.

\subsubsection{Data and evidence}
Table 1 provides a summary of the type of data and evidence that were introduced in Stage 1. In the next activity you will reflect on those which are most relevant to your project.

Table 1 - Common types of data\slash evidence in a research project

\begin{table}[htbp]
\begin{minipage}{\linewidth}
\setlength{\tymax}{0.5\linewidth}
\centering
\small
\begin{tabulary}{\textwidth}{@{}ll@{}} \toprule
 \textbf{Types of data\slash evidence} & \\
\midrule

 \textbf{Quantitative} data that can be quantified or measured, and be given numerical values & \textbf{Numerical} numbers, either discrete or continuous \\
 & \textbf{Ordinal} can be arranged in an order, but are not necessarily numerical \\
 & \textbf{Interval} ordinal data for which we can calculate precisely the interval between any two data points \\
 \textbf{Qualitative} all other data which is descriptive in nature & \textbf{Categorical (or nominal)} correspond to categories which cannot be ordered and on which mathematical operations and function don't apply \\
 & \textbf{Other} e.g., texts, words, images, sounds, etc. \\
\bottomrule

\end{tabulary}
\end{minipage}
\end{table}

\begin{question}[subtitle={ACTIVITY: Detailing data and evidence}] For each type in the table, comment on whether it may be needed in your project.

\begin{guidance}For those needed in your project, you should also give specific examples and indicate the source.
\end{guidance}\end{question}
%%Hack to correct tcbox behaviour
\color{black}

\subsubsection{Research methods}
Table 1 provides a summary of research methods introduced in this section. In the next activity you will reflect on those which are most relevant to your project.

Table 1 - Research methods introduced in this section

\begin{table}[htbp]
\begin{minipage}{\linewidth}
\setlength{\tymax}{0.5\linewidth}
\centering
\small
\begin{tabulary}{\textwidth}{@{}ll@{}} \toprule
 \textbf{Types of method} & \\
\midrule

 \textbf{Data collection} & \textbf{Questionnaires} pre-defined set of questions organised in a particular order, which are distributed to respondents to gather their answers \\
 & \textbf{Interviews} a form of conversation between the researcher and one or more interviewees, designed by the researcher to gain insights and opinions around a specific topic \\
 & \textbf{Delphi} iterative process of collecting and synthesising anonymous judgements from experts to arrive at a consensual view \\
 & \textbf{Observations\slash measurements} direct observation\slash measurement to find out what people actually do or what actually happens in a particular context \\
 & \textbf{Existing documents\slash data sets} reusing secondary data\slash evidence \\
 & \textbf{Other} any other collection method you know which may be applicable \\
 \textbf{Data analysis} & \textbf{Tables and chats} to summarise and visualise data, and identify interesting patterns \\
 & \textbf{Statistical analysis} to investigate trends, patterns, and relationships in quantitative data \\
 & \textbf{Thematic analysis} to identify recurring themes, their definition and relationships \\
 & \textbf{Content analysis} to investigate certain words, themes, or concepts \\
 & \textbf{Other} any other analysis method you know which may be applicable \\
 \textbf{Modelling} & \textbf{Systems diagrams} to help you understand the structure of a situation of interest that can be rendered as a system. Various flavours exist \\
 & \textbf{UML (Unified Modeling Language)} to visualise, specify or document various artefacts in the process of developing software systems \\
 & \textbf{Problem diagrams} to capture requirements in a real-world context to inform the specification of a new software system, or more generally of some novel solution artefact to be designed and develop, and to ensure it fitness-for-purpose \\
 & \textbf{Statistical modelling} to model relations between variables and being able to make predictions \\
 & \textbf{Other} any other modelling method you know which may be applicable \\
\bottomrule

\end{tabulary}
\end{minipage}
\end{table}

\begin{question}[subtitle={ACTIVITY: Detailing research methods}] For each method in the table, comment on whether it is applicable.

\begin{guidance}For those applicable in your project, you should also justify them in terms of the data and evidence needed in your project.
\end{guidance}\end{question}
%%Hack to correct tcbox behaviour
\color{black}

\subsubsection{Research strategies}
Table 1 provides a summary of research strategies introduced in this section. In the next activity you will reflect on those which are most relevant to your project.

Table 1 - Research strategies introduced in this section

\begin{table}[htbp]
\begin{minipage}{\linewidth}
\setlength{\tymax}{0.5\linewidth}
\centering
\small
\begin{tabulary}{\textwidth}{@{}lllllll@{}} \toprule
 \textbf{name} & \textbf{aim} & \textbf{data collection} & \textbf{data analysis} & \textbf{success factors} & \textbf{advantages} & \textbf{disadvantages} \\
\midrule

 \textbf{survey research} & to gain insights which are valid across a target population, by collecting data from a predefined sample in a standardised and systematic way & you need to identify upfront which data you will collect in a standardise matter, your target population and sample & you seek patterns in the sample data collected to devise generalisations to the wider population & you must be able to access an appropriate sample and generate a sufficient volume of data & - can produce a lot of data in a short time - data collection can be replicated on different samples, or the same sample on a later time & * lack of depth * focus on what can be measured * provides a snapshot at a particular time, rather than a longitudinal view * can't reveal cause-and-effect relationships \\
 \textbf{experimental research} & to investigate cause and effect relationships between factors by testing hypothesis or proving\slash disproving causal links & you need first to state the hypothesis to be tested, then make detailed observations and measurements of outcomes and any changes that take place when particular factors are introduced or removed & you seek to explain causal links between factors under study, looking at your observations and measurements under the different experimental conditions & you must be able to control factors which may affect the outcome. This is possible in laboratory experiments, while the level of control in field experiments is diminished. & * there are well established processes and protocols * tailored to the study of causal relations & * Laboratory experiments are very reliable due to the high level of control, but can be very artificial, with little or no relation to a real-world context * The opposite is true for field experiments \\
 \textbf{design science research} & to generate new knowledge about a significant problem or its solution via the design of an artefact. It simultaneously generates knowledge about the problem, the artefact and the method used to design it. By artefact is meant anything made by humans, so this is a very broad definition encompassing all that does not exist in nature & you need to both articulate the problem, and design, construct and evaluate the solution artefact & you need shed new insights on the problem, and argue how solution and solution process contribute new knowledge & you must be able to argue that it is not `normal' design & * leads to tangible artefacts which fit a real-world context * is particularly suited to emerging and rapidly changing technology-related fields of study & * might be difficult to generalise to other real-world settings * may require advanced technical skills * may lead to short shelf-life of the research, particularly in technological volatile fields of study where technology becomes quickly obsolete \\
 \textbf{case study research} & to investigate in great depth a notable instance of what is under study, in its real-world context & you must be able to articulate your personal stake in the situation, for example, a deeply felt interest or active involvement , rather than assuming and claiming unbiased passive `neutral' observation. You must also keep your own journal during the course of your research inquiry, & you may seek to explore questions or hypotheses for follow-up research, or provide a detailed account of a phenomenon in its natural context, or explain why certain outcomes or phenomena have occurred & you must be able to analyse the significant instance holistically and in context & * allows the study of a complex situation where several factors are at play * iallows the researchers to explore alternative meanings and explanations & * can be time-consuming and access may be difficult to obtain * may be perceived as lacking rigour * insights may be difficult to generalise \\
 \textbf{systemic inquiry} & to explore complex, messy situations involving multiple and often contrasting perspectives, with the aim of transforming the situation for social betterment. & - bricolage research applies, in which you can complement traditional research methods with other tools and techniques from your existing repertoire of expertise and professional tradition - your research journal contributes evidence & * mainly a qualitative endeavour * bricolage research applies, in which you can complement traditional research methods with other tools and techniques from your existing repertoire of expertise and professional tradition & * you must articulate your research problem with reference to one of more systems of interest, and frame the research in terms of possible systems change. * you must articulate your personal stake in the situation * You must also have access to sources of different perspectives on the situation under study in order to generate your evidence & * helps you make sense of complex situations of change and uncertainty * support development of trust amongst participants, including trust with the researcher in co-exploring the situation & - acknowledges that the research will not deliver `certainty' in terms of `problem-solving' associated with complex situations \\
 \textbf{mixed methods research} & to gain different perspectives on phenomena of interest, by exploring connections and contradictions between quantitative and qualitative data & will depend on the particular combination of methods selected & a key aspect is consideration of how connections between findings are established, through comparing and contrasting data from the different methods applied & you must be able to apply competently different kinds of methods & * can provide a more holistic understanding of the phenomena under study, and facilitate different avenues for exploration * particularly suited to situations in which neither quantitative nor qualitative methods alone can provide sufficient insights & * add complexity to the research design * can be demanding and time consuming \\
 \textbf{systematic review research} & to generate a scholarly synthesis of evidence in relation to a specific research problem or question & you need to establish and apply a rigorous set of criteria to identify, select, and critically appraise relevant research from previously published studies & you need to generate a critical synthesis of evidence based on the selected set of criteria & - your review must be both systematic and extensive - you need to be a skilled critical thinker and academic writer & - is considered transparent, reliable, and easy to replicate & - can be very time-consuming * is only as reliable as the studies reviewed * can be difficult to synthesise findings from potentially very different studies \\
\bottomrule

\end{tabulary}
\end{minipage}
\end{table}

\begin{question}[subtitle={ACTIVITY: Detailing research strategies }] For each strategy in the table, comment on whether it is applicable to your research.

\begin{guidance}For those applicable to your project, you should say why that's the case, with reference to your research problem, and aim and objectives. You should also comment on which research methods you may apply within, based on your previous research methods analysis, and on how likely you are to be able to apply it successfully.
\end{guidance}\end{question}
%%Hack to correct tcbox behaviour
\color{black}

\subsubsection{Putting it all together}
You should now have enough material to sketch your overall research design.

\begin{question}[subtitle={Activity: Sketching your overall research design}] Based on your judgements, expressed in the previous activities, summarise your research design by addressing each of the following questions:

\begin{itemize}
\item Which evidence and data will you need and why?

\item Where will you source such data\slash evidence from?

\item Which research strategy are you thinking of adopting and why?

\item Which research methods are you thinking of applying for your data collection\slash analysis or modelling within that strategy?

\end{itemize}

Ensure that your answers are justified in term of your research problem, and intended aim and objectives, by indicating explicitly the rationale behind your choices.

\begin{guidance}At this point, your choices will be tentative, but should provide a good starting point for further investigation and a meaningful conversation with your supervisor who will be able to advise you further.

As well as your intended research, you should also keep in mind that you will have limited time and resources to complete your project, so you should limit your choices to be:

•	manageable in terms of their application in the context in which you are going to conduct your research and the time you have available

•	efficient in terms of the data\slash evidence they produce and your ability to process them with the resources and time you have available

•	effective at producing data\slash evidence that you have the skills and expertise to analyse in the time you have available.

\end{guidance}\end{question}
%%Hack to correct tcbox behaviour
\color{black}

\subsubsection{Investigating research strategies and methods further}
The overview provided in this section was designed to help you develop a broad understanding of possible choices you can make to help you sketch your initial research design. It is now time to start looking a little deeper into your most likely research strategies and methods.

\begin{question}[subtitle={Activity: Reviewing your chosen research strategies and methods}] Consider the research strategies and methods you have included in your sketched research design. Conduct a small literature review on each of them to help you confirm that they are indeed suitable for your research, and to help you articulate how and why they are suitable for your project.

\begin{guidance}As this is a literature review, your should follow the process and practices you have already learnt and applied, including recording entries and notes in your BMT.

You should focus on materials which will help you understand how they work, and their strength and weaknesses in relation to your project, something you will return to in the next stage of your project, where you will consider specific procedures for applying them.

Your review does not need to be extensive: a couple of references for each strategy\slash method should suffice, as long as they provide the information required. You can use the annotated reading list in the next section as your starting point, but you should also explore the wider literature. Your supervisor should also be able to suggest literature you could start from.
\end{guidance}\end{question}
%%Hack to correct tcbox behaviour
\color{black}

\subsubsection{Annotated reading list}

\begin{itemize}
\item There are many books and web portals which cover a variety of research strategies and methods. You could start from the following:

\begin{itemize}
\item GO-GN (2020), Research Methods handbook, \href{https://go-gn.net/wp-content/uploads/2020/07/GO-GN-Research-Methods.pdf}{https:\slash \slash go-gn.net\slash wp-content\slash uploads\slash 2020\slash 07\slash GO-GN-Research-Methods.pdf} This is a practical introduction to research methods for phd research, written with contributions from doctoral research students.

\item

\item Oates, B.J. (2006) Researching information systems and computing, SAGE. This is very good read for novice research students, particularly in information systems and computing disciplines. It provides a clear, practical and comprehensive introduction to academic research, including key definitions, methods and techniques.

\item

\item The SAGE research portal at \href{https://methods-sagepub-com.libezproxy.open.ac.uk}{https:\slash \slash methods-sagepub-com.libezproxy.open.ac.uk} contains a great variety of resources on both strategies and methods, from articles to video tutorials

\end{itemize}

\item If you are interested in Design Science Research, then you could start from:

\begin{itemize}
\item vom Brocke J., Hevner A., Maedche A. (2020) Introduction to Design Science Research. In: vom Brocke J., Hevner A., Maedche A. (eds) Design Science Research. Cases, Cham. \href{https://doi.org/10.1007/978-3-030-46781-4_1}{https:\slash \slash doi.org\slash 10.1007\slash 978-3-030-46781-4\_1}, which is possibly the most up-to-date introduction to the topic.

\item The International Conference on Design Science Research in Information Systems and Technology (DESRIST) (since 2005) have tracked the development of this research strategy, with many seminal papers published its proceedings. These can be accessed via the OU Library.

\end{itemize}

\item An of-cited reference on case study research is:

\begin{itemize}
\item Yin, R.K., 2009. Case study research: Design and methods. Applied social research methods series, Vol. 5. Fourth Edition, Sage.

\end{itemize}

\item If you are interested in systemic inquiry you could start here:

\begin{itemize}
\item Ison, R. (2017). Systemic inquiry. Ch. 10 in Part 3 Systems Practice: How to Act (pp. 251--274). Springer, London, which is also available as eBook Reading

\item Simon, G and Chard, A. eds. (2014) Systemic Inquiry: Innovations in Reflexive Practice Research. Farnhill: Everything is Connected Press.

\item Ison, R.L., Collins, K.B. and Iaquinto, B.L., 2021. Designing an inquiry‐based learning system: Innovating in research praxis to transform science--policy--practice relations for sustainable development. Systems Research and Behavioral Science, 38(5), pp.610--624.

\item McClintock, D., Ison, R. and Armson, R., 2003. Metaphors for reflecting on research practice: researching with people. Journal of Environmental Planning and Management, 46(5), pp.715--731.

\end{itemize}

\item Problem diagrams are part of a wider approach to problem oriented engineering, with roots in software development, but more widely applicable to most forms of design and engineering. The following references are a good starting point:

\begin{itemize}
\item Jackson, M., 2005. Problem frames and software engineering. Information and Software Technology, 47(14), pp.903--912.

\item Hall, J., Rapanotti, L. and Jackson, M., 2008. Problem oriented software engineering: Solving the package router control problem. IEEE Transactions on Software Engineering, 34(2), pp.226--241.

\end{itemize}

\item UML is a modelling language with roots in software engineering. It is now an international standard that can be found at:

\begin{itemize}
\item \href{https://www.iso.org/standard/52854.html}{https:\slash \slash www.iso.org\slash standard\slash 52854.html}

\item Many tutorials are available online, so it should be relatively easy for you to find an introductory one. There are also many UML digital modelling tools, some of which are open source and free to use

\end{itemize}

\item If you wish to find out more about working with table, graphs and charts, you could start with the following free resources from The Open University, UK:

\begin{itemize}
\item `Working with charts, graphs and tables' at:

\end{itemize}

\end{itemize}

\href{https://www.open.edu/openlearn/science-maths-technology/mathematics-statistics/working-charts-graphs-and-tables/content-section-0?active-tab=description-tab}{https:\slash \slash www.open.edu\slash openlearn\slash science-maths-technology\slash mathematics-statistics\slash working-charts-graphs-and-tables\slash content-section-0?active-tab=description-tab}

- `More working with charts, graphs and tables' at:

\href{https://www.open.edu/openlearn/science-maths-technology/mathematics-statistics/more-working-charts-graphs-and-tables/content-section-0?active-tab=content-tab}{https:\slash \slash www.open.edu\slash openlearn\slash science-maths-technology\slash mathematics-statistics\slash more-working-charts-graphs-and-tables\slash content-section-0?active-tab=content-tab}

or the UK BBC Skillswise site at:

\href{https://www.bbc.co.uk/teach/skillswise/graphs/zmkpqp3}{https:\slash \slash www.bbc.co.uk\slash teach\slash skillswise\slash graphs\slash zmkpqp3}

\section{Reporting in Stage 2}
At the end of Stage 2, we recommend you complete a report, extending that of Stage 1, covering the work you have carried on in this stage. As already indicated, writing end-of-stage reports throughout your project will help you consolidate your work, develop your dissertation incrementally, and improve your critical thinking and academic writing skills as you go along.

The recommended structure and content for the Stage 2 report is indicated in Table 1.

Table 1 -- Report structure and content guidance

\begin{table}[htbp]
\begin{minipage}{\linewidth}
\setlength{\tymax}{0.5\linewidth}
\centering
\small
\begin{tabulary}{\textwidth}{@{}ll@{}} \toprule
 \textbf{Structure} & \textbf{Content guidance} \\
\midrule

 Proposed title & Your title should continue to capture succinctly research problem and aim \\
 Sect 1 - Introduction 1.1 Background to the research 1.2 Justification for the research & This section should provide an introduction to your research topic in its wider context (as background) and your justification of why the research is worth pursuing. It should be well articulated and supported by evidence \\
 Sect 2 - Literature review 2.1 Review of existing relevant knowledge 2.2 Critical summary, including knowledge gap to be addressed by the research & Your review should provide a critical account of your in-depth engagement with the academic (and other) relevant literature, including identifying key trends, ideas and possible knowledge gaps. Most of your citations should point to academic articles. Your critical summary should highlight key insights from your review and provide a strong justification for your proposed research. Both coverage and depth of your review matter. You should ensure that your review is well structured, with a logical narrative flow and your arguments are well supported by evidence \\
 Sect 3 - Research definition 3.1 Problem statement 3.2 Aim and objectives 3.3 Knowledge contribution & You should ensure that your research problem is well articulated and appropriate for your course and your personal and professional circumstances, that your aim and objectives are consistent with research problem, and that the intended knowledge contribution of your research is clearly articulated \\
 Sect 4 - Research design 4.1 Evidence and data 4.2 Research strategy and methods 4.3 Ethical, legal and EDI considerations & This section should demonstrated your critical engagement with the key elements of research design, particularly your understanding of the type of evidence, research methods and strategies you will to apply, also supported by your reading of the related literature, and appropriately justified in relation to your research problem, aim and objectives. It should also demonstrate your careful consideration of ethical and legal matters, and that your research will comply with your course and university requirements \\
 Sect 5 - Assessment of your proposed research 5.1 Qualification fit 5.2 Personal and professional fit 5.3 Technical skills and resources needed 5.4 Statement of feasibility 5.5 Personal reflection on research process & In this section you should continue to argue how your research is a good fit across all criteria. You should provide a clear rationale as to why you think what you are proposing is feasible. You should also reflect on your growing understanding of the research process, including key learning and aspects you have found particularly challenging. \\
 Sect 6 - Planning, scheduling and risk assessment 6.1 Statement of progress 6.2 Key priorities in follow-up stage 6.3 Risk assessment & In this section you should reflect on the progress you have made in Stage 2 and establish your priorities for the next stage. You should also review your risk assessment as appropriate. \\
 References & You should keep your growing references in good order and ensure you apply the required bibliographical style consistently. Ideally, you should use a BMT to generate and integrate your references within your report \\
 Appendix - Work schedule & You should include your revised work plan as an appendix \\
 Appendix - Risk assessment table & You should include your updated risk table as an appendix \\
\bottomrule

\end{tabulary}
\end{minipage}
\end{table}

\begin{question}[subtitle={Activity: Putting your report together}] Using your word processor of choice, and starting from your Stage 1 report, create a report with the structured indicated in Table 1, and fill it in by following the guidance provided in the table, and making good use of your notes and summaries from all related activities you have carried out so far.

\begin{guidance}In this first pass at putting together your report, you should focus primarily on completeness, ensuring that each section includes at least draft content.
\end{guidance}\end{question}
%%Hack to correct tcbox behaviour
\color{black}

\paragraph{}
After you have filled in your report with as much material as you can, you should review and revise it until you are happy with your account, and ready to move on. This may take more than one iteration, but you should ensure you do not delay unnecessarily your work for the follow-up stage.

In the next activity, you will use Table 1 to assess whether your report is of good standard.

Table 1 - Criteria to review your report

\begin{table}[htbp]
\begin{minipage}{\linewidth}
\setlength{\tymax}{0.5\linewidth}
\centering
\small
\begin{tabulary}{\textwidth}{@{}ll@{}} \toprule
 \textbf{Criteria} & \textbf{Prompts} \\
\midrule

 \textbf{Completeness} & Are all sections of the suggested structure completed in line with the guidance provided? \\
 \textbf{Good academic writing practices} & Have you applied good academic writing practices throughout? \\
 \textbf{Logical structure and flow} & Have you structured your narrative appropriately to ensure a logical flow of arguments? \\
 \textbf{Supporting references or evidence} & Are your key arguments supported by appropriate references or other evidence? \\
 \textbf{Citation and reference style} & Do all your citations and references comply with the required bibliographical style? \\
 \textbf{Avoiding plagiarism} & Have you acknowledged the work of others and distinguished it from your own appropriately? \\
 \textbf{Standard of English (or any modern language you use)} & Have you proof-read your report carefully to remove all typos and grammatical errors? \\
\bottomrule

\end{tabulary}
\end{minipage}
\end{table}

\begin{question}[subtitle={Activity: Reviewing your report}] Apply the criteria in Table 1 to review your current report and write up a summary of your assessment.

\begin{guidance}For each criteria, consider the related prompts to help you assess your report overall, and write down any further work needed for your next stage.
\end{guidance}\end{question}
%%Hack to correct tcbox behaviour
\color{black}

\paragraph{}
As previously stated, writing up your report is an excellent way to communicate the work you have completed and still planning to do, and something tangible you can share with your supervisor for their formative feedback.

\chapter{Stage 3: Second research increment}

\section{Introducing stage 3}
In Stage 3 you will focus on adding detail to both your aim and objectives, and your research design. Stage 3 assumes that you have completed your Stage 2 work, and possibly discussed it with your supervisor, particularly your research design choices.

With reference to our 5-stage framework, the activities which are in focus in Stage 3 are recalled in Table 1, which also provides some guidelines for your interaction with your supervisor during this stage.

Table 1 - Stage 3 activities

\begin{table}[htbp]
\begin{minipage}{\linewidth}
\setlength{\tymax}{0.5\linewidth}
\centering
\small
\begin{tabulary}{\textwidth}{@{}llll@{}} \toprule
 \textbf{Research activities} & \textbf{Stage 3} \textbf{(20\% of project length)} & \textbf{Effort within stage} & \textbf{Suggested focus of your interaction with your supervisor} \\
\midrule

 \textbf{Identifying the research problem} & Adjust, if needed & 2\% & \\
 \textbf{Reviewing the literature} & Adjust, if needed & 3\% & \\
 \textbf{Setting research aim and objectives} & Finalise aim and objectives, and define tasks and deliverables & 10\% & Suitability of tasks and deliverables from objectives \\
 \textbf{Choosing the research design} & Complete research design, with detailed consideration of data and evidence, research strategy, research methods and procedures & 20\% & Suitability of research procedures \\
 \textbf{Gathering and analysing evidence} & Conduct pilot work to test aspects of your research design & 35\% & Scope of your pilot work \\
 \textbf{Interpreting and evaluating findings} & n\slash a & 0\% & \\
 \textbf{Reporting, critical reflection and conclusions} & Assess research progress and write up Stage 3 report & 25\% & Any further improvements required \\
 \textbf{Work planning and risk management} & At stage start, review work from previous stage and project risk; adjust plan as needed If you have received feedback from supervisor on your previous stage work, adjust plan to include any revision recommended & 5\% & Any major adjustment required \\
\bottomrule

\end{tabulary}
\end{minipage}
\end{table}

\begin{question}[subtitle={Activity: Understanding the effort needed in this stage}] Consider Table 1 carefully, taking notice of the entries in the `Effort within stage' column. Make a note of the activities which are most prominent in this stage and what is expected under each.

\begin{solution}Developing your research design further and conducting your pilot work will constitute your major effort in this stage (55\% of the study time in total): your pilot work will be an initial test of some aspects of your research design, including a proof-of-concept application of some of your chosen methods.
\end{solution}\end{question}
%%Hack to correct tcbox behaviour
\color{black}

\section{Identifying tasks and deliverables}
Recall from Stage 1 that your research aim indicates the specific way in which your research will address the knowledge gap you have identified with your research problem, while your objectives help you break that down into 3 to 4 high-level steps you must take to achieve the aim, as shown in the example below (repeated from Stage 1).

\begin{example}{Example -- Applying Machine Learning}Assume the research aim is stated as:

\emph{to apply Machine Learning (ML) to improve the accuracy of resources and time forecasting in the context of small engineering plants}

Some related objectives could be the following:

1. To identify which ML techniques are applicable to resource and time forecasting in the context of small engineering plants

2. To test the accuracy of forecasting of those techniques

3. To provide recommendations as to how integrate those techniques effectively in engineering practice in order to improve forecasting accuracy

Objective 1 allows us to identify specific ML techniques to be used in the project, to ensure the work is feasible within the time-frame of the project. Objective 2 allows us to investigate and compare how accurate the chosen techniques are in their forecasting application. Objective 3 allows us to draw some conclusions from the research conducted and make recommendations to improve professional practice.

Note how those objectives build on each other and, if successfully completed, they contribute to meet the overall aim.
\end{example}

\paragraph{}
It is now time to break your research objectives down even further, into tasks to help you plan your detailed research work to meet each of them. While a research objective indicates what you need to achieve, \textbf{research tasks} specify the work that you need to complete to get there, so they address the question of what you need to do to meet the objective.

Closely related to tasks are \textbf{deliverables}: these are the tangible things produced by your task work.

\begin{example}{Example -- cont'd}In our example, the first objective is met once we have identified the relevant ML techniques. There are two complementary ways to do this: to look at the literature and to ask practitioners. As a result, we could break this objective down into the tasks, and deliverables, indicated in the following table

\paragraph{Table}
\textbar{} \textbf{Objective} \textbar{} \textbf{Tasks} \textbar{} \textbf{Deliverables} \textbar{}
\textbar{} :----- \textbar{} :----- \textbar{} :----- \textbar{}
\textbar{} \textbf{To identify which ML techniques are applicable to resource and time forecasting in the context of small engineering plants} \textbar{} to investigate the academic literature in order to identify relevant ML techniques \textbar{} a collection of relevant ML techniques reported in the literature \textbar{}
\textbar{} \textbar{} to ask practitioners which techniques they employ \textbar{} a collection of relevant ML techniques used in professional practice \textbar{}\end{example}

\begin{question}[subtitle={ACTIVITY: Establishing tasks and deliverables}] Consider your research objectives. For each, identify related tasks and deliverables.

\begin{guidance}You should draw a table similar to that in our running example. You should ensure that the tasks provide a reasonable break down of your objectives into discrete pieces of work.
\end{guidance}\end{question}
%%Hack to correct tcbox behaviour
\color{black}

\paragraph{}
Your tasks and deliverables don't need to be perfect at this stage, and are likely to be revised as you progress through your project. However, it is important that you have thought about specific work you will need to carry out to meet your objectives.

\section{Relating tasks to research methods}
The way to carry out your tasks and meet your objectives is through the application of research methods.

\begin{example}{EXAMPLE - cont'd }Following on from our previous example, we have extended the table to include an indication and justification of candidate research methods for each task.

\paragraph{TABLE}
\textbar{} \textbf{Objective} \textbar{} \textbf{Tasks} \textbar{} \textbf{Deliverables} \textbar{} \textbf{Relevant research methods} \textbar{} \textbf{Justification and feasibility} \textbar{}
\textbar{} :----- \textbar{} :----- \textbar{} :----- \textbar{} :----- \textbar{} :----- \textbar{}
\textbar{} \textbf{To identify which ML techniques are applicable to resource and time forecasting in the context of small engineering plants} \textbar{} to investigate the academic literature in order to identify relevant ML techniques \textbar{} a collection of relevant ML techniques reported in the literature \textbar{} review of existing literature \textbar{} I can access relevant literature via my university library \textbar{}
\textbar{} \textbar{} to ask practitioners which techniques they employ \textbar{} a collection of relevant ML techniques used in professional practice \textbar{} questionnaire, possibly followed by interviews \textbar{} I have access to professional networks, which I could use to distribute the questionnaire, and possibly to recruit participants for follow-up interviews \textbar{}\end{example}

\paragraph{}
Note that the choice of research methods in relation to your research tasks is an essential part of your research design. In fact, the two influence each other: your objectives and related tasks direct you towards specific research methods, which in turn have to be part of your overall research design.

\begin{question}[subtitle={ACTIVITY: Associating methods to tasks and deliverables}] Extend your tasks and deliverables table with your candidate research methods, including stating why they apply and are feasible for your project. Revise your research design draft from Stage 2 so that is consistent with those choices.

\paragraph{GUIDANCE }
Refresh your understanding of chosen research methods from the study work you carried out in Stage 2. It is important you keep reviewing your choices with your supervisor.\end{question}
%%Hack to correct tcbox behaviour
\color{black}

\section{Detailing research procedures}
Once you have chosen the set of research methods you will apply, you must establish exactly how you will do that, something we refer to as \textbf{research procedures}.

Your research procedures will be method specific, in that each method you choose to apply will come with recommended practices, which you will need to contextualise to your own project needs, including your access to participants, data or other kind of evidence. For instance, there are plenty of guidelines in the literature on how to design questionnaires, including which type of questions to include and how to phrase them. There are also recommendations concerning testing the questionnaire design prior to its use, and of course, there are many ways a questionnaire can be administered. In writing your procedures for this research method, you would have to be specific on how each of the above applies in your project.

It is important, therefore, that you master the research methods of your choice, starting by reviewing once again the related academic literature.

\begin{question}[subtitle={Activity: Sketching research procedures}] Consider the research methods you intend to apply, and the related review you conducted in Stage 2. Reconsider those materials, possibly going back to the literature sources, to learn how to apply the methods effectively within your project.

For each method, sketch possible procedures of application, ensuring you make appropriate reference to the literature you have reviewed and best practice guidelines therein.

\begin{guidance}It is possible that the review you conducted in Stage 2 is not sufficient, in which case you will need to extend it to complete this activity.

You should focus on practical aspects of applying the methods, including specific processes and techniques to gather, summarise and present your evidence in your reports.

Depending on the extent you need to review further academic literature, this activity could be quite substantial, so you should set aside up to 20\% of your study time to complete it.

\end{guidance}\end{question}
%%Hack to correct tcbox behaviour
\color{black}

\section{Assessing validity}
As your intended research design becomes clearer, you will soon be testing some aspects of it in your pilot work. Before you do that, however, you need to consider if the choices you have made will allow you to gather evidence and derive findings in a systematic, rigorous, repeatable and reliable fashion so as to address your research problem. This is referred to as assessing the overall \textbf{validity} of your research design, which is broken down into the following considerations.

\textbf{Construct validity} asks whether you have put your design together logically by focusing on the relationship between evidence and research problem. Here you ask yourself whether the evidence you will generate through your chosen research design will be accurate and relevant to address your research problem. This tests the logical coherence of your aim, objectives, tasks, methods and deliverables in relation to the research problem and the knowledge gap you intend to address. With construct validity, you are asking: \emph{have I designed my research in the right way?}

\textbf{Internal validity} is concerned with the way you gather and analyse evidence. All research strategies and methods come with recommendations of good practice to ensure that your research is both systematic, repeatable and reliable. In your work, you need to ensure that you follow such practices and are aware of possible pitfalls. For instance, in experimental research you need to control all factors which may effect outcomes beyond those under study: failing to exercise such control will lead to observations and measurements which are unreliable. In assessing internal validity, you should also take into account limitations of human perception and cognition, and any potential personal bias. With internal validity, you are asking: \emph{have I executed my research in the right way?}

\textbf{External validity} relates to the extent you will be able to generalise your findings beyond the immediate context of your research. For instance, you may conduct a case study within a specific organisation, so here you are asking whether and how what you have discovered may apply to other organisations. With external validity, you are asking: \emph{will my research lead to findings that apply somewhere else?}

Anything that gets in the way of validity in research is termed a \textbf{threat to validity}. Different research strategies and methods are exposed to different threats, something you should have encountered in your review of the literature on your chosen methods.

\begin{question}[subtitle={Activity: Assessing validity of research design}] Conduct an initial assessment of your chosen research design in relation to the three kind of validity discussed above. Write down a short summary of your thinking in support of each, and of possible threats to validity you envisage.

\begin{guidance}You may need to refer back to the literature you have reviewed to identify specific threats which apply to your chosen research methods and strategies.

You won't be able to address this in full at this point in your project, particularly the internal validity, which refers to the execution of your research design. Nevertheless, it is important for you to consider validity and possible threats from the onset. You will return to this topic at the end of your project, as part of the overall assessment of your research, to reflect on the validity of your completed research.

\end{guidance}\end{question}
%%Hack to correct tcbox behaviour
\color{black}

\section{Conducting your pilot work}

Your \textbf{pilot work} will be a small scale test of some of the methods and procedures you will apply in the next stage of your project. Its main function is to help you assess the feasibility of your research design, or at least some aspects of it, and build your confidence in the approach you have chosen.

As such, your pilot work may not contribute directly to your aim and objectives, but it should help you decide whether you can actually do what you have planned to do, or inform how your research design and project plan should change instead.

There are no constraints on what you can do for your pilot work, other than you should exercise some aspect of your research design. It is therefore essential that you you agree what you are going to do with your supervisor first.

\begin{question}[subtitle={Activity: Planning and executing your pilot work}] Plan your pilot work and discuss your plan with your supervisor.

Once you have agreed the way forward, execute your plan and write a summary of both its execution and outcomes.

\begin{guidance}Your summary should include:

- an indication of which aspects of your research design your pilot work was concerned with

- any methods and procedures applied

- any data or evidence gathered, including possible modelling, artefact design or prototyping, appropriately presented and summarised

- lessons learnt and any resulting revision to your research design and project plan, particularly in relation to construct and internal validity of your research design.

To complete this activity successfully, it is essential that you agree your pilot work plan with your supervisor upfront, and discuss your progress on a regular basis.

This is a substantial activity, which will take you up to 35\% of your study time.

\end{guidance}\end{question}
%%Hack to correct tcbox behaviour
\color{black}

\section{Reporting in Stage 3}
At the end of Stage 3, you should complete a report, extending that of Stage 2 and covering the work you have carried on in this stage. Its recommended structure and content are indicated in Table 1.

Table 1 -- Report structure and content guidance

\begin{table}[htbp]
\begin{minipage}{\linewidth}
\setlength{\tymax}{0.5\linewidth}
\centering
\small
\begin{tabulary}{\textwidth}{@{}ll@{}} \toprule
 \textbf{Structure} & \textbf{Content guidance} \\
\midrule

 Proposed title & Your title should continue to capture succinctly research problem and aim \\
 Sect 1 - Introduction 1.1 Background to the research 1.2 Justification for the research & This section should provide an introduction to your research topic in its wider context (as background) and your justification of why the research is worth pursuing. It should be well articulated and supported by evidence \\
 Sect 2 - Literature review 2.1 Review of existing relevant knowledge 2.2 Critical summary, including knowledge gap to be addressed by the research & Your review should provide a critical account of your in-depth engagement with the academic (and other) relevant literature, including identifying key trends, ideas and possible knowledge gaps. Most of your citations should point to academic articles. Your critical summary should highlight key insights from your review and provide a strong justification for your proposed research. Both coverage and depth of your review matter. You should ensure that your review is well structured, with a logical narrative flow and your arguments are well supported by evidence \\
 Sect 3 - Research definition 3.1 Problem statement 3.2 Aim, objectives, tasks and deliverables 3.3 Knowledge contribution & You should ensure that your research problem is well articulated and appropriate for your course and your personal and professional circumstances, that your aim and objectives are consistent with research problem, that tasks and deliverables break down your objectives appropriately and are clearly related to your chosen research methods, and that the intended knowledge contribution of your research is clearly articulated \\
 Sect 4 - Research design 4.1 Evidence and data 4.2 Research strategy and methods 4.3 Research procedures 4.4 Ethical, legal and EDI considerations & This section should demonstrated your critical engagement with all elements of research design, including a detailed account of the data and evidence needed in your research, the research methods and research strategies you will to apply, and how you will apply them within your project. Your account should be supported by a clear rationale and insights from the related literature, and appropriately justified in relation to your research problem, aim and objectives. It should also demonstrate your careful consideration of ethical and legal matters, and that your research will comply with your course and university requirements \\
 Sect 5 - Analysis and interpretation 5.1 Pilot work & This section should report on a well thought-out pilot work which clearly and competently test some significant aspect of your research design. It should demonstrate good critical reflection on outcomes and highlight any adjustments needed as a result. \\
 Sect 6 - Assessment of your proposed research 6.1 Qualification fit 6.2 Personal and professional fit 6.3 Technical skills and resources needed 6.4 Statement of feasibility 6.5 Personal reflection on research process & In this section you should continue to argue how your research is a good fit across all criteria. You should provide a clear rationale as to why you think what you are proposing is feasible. You should also reflect on your growing understanding of the research process, including key learning and aspects you have found particularly challenging. \\
 Sect 7 - Planning, scheduling and risk assessment 7.1 Statement of progress 7.2 Key priorities in follow-up stage 7.3 Risk assessment & In this section you should reflect on the progress you have made in Stage 2 and establish your priorities for the next stage. You should also review your risk assessment as appropriate. \\
 References & You should keep your growing references in good order and ensure you apply the required bibliographical style consistently. Ideally, you should use a BMT to generate and integrate your references within your report \\
 Appendix - Work schedule & You should include your revised work plan as an appendix \\
 Appendix - Risk assessment table & You should include your updated risk table as an appendix \\
\bottomrule

\end{tabulary}
\end{minipage}
\end{table}

\begin{question}[subtitle={Activity: Putting your report together}] Using your word processor of choice, and starting from your previous report, complete your Stage 3 report by applying the structure and guidance in Table 1, and making good use of your notes and summaries from all related activities you have carried out so far.

\begin{guidance}In this first pass at putting together your report, you should focus primarily on completeness, ensuring that each section includes at least draft content.
\end{guidance}\end{question}
%%Hack to correct tcbox behaviour
\color{black}

\subsubsection{}
As in the previous stages, after you have filled in your report you should review and revise it iteratively until you are happy with your account, and are ready to move on.

Table 1 - Criteria to review your report

\begin{table}[htbp]
\begin{minipage}{\linewidth}
\setlength{\tymax}{0.5\linewidth}
\centering
\small
\begin{tabulary}{\textwidth}{@{}ll@{}} \toprule
 \textbf{Criteria} & \textbf{Prompts} \\
\midrule

 \textbf{Completeness} & Are all sections of the suggested structure completed in line with the guidance provided? \\
 \textbf{Good academic writing practices} & Have you applied good academic writing practices throughout? \\
 \textbf{Logical structure and flow} & Have you structured your narrative appropriately to ensure a logical flow of arguments? \\
 \textbf{Supporting references or evidence} & Are your key arguments supported by appropriate references or other evidence? \\
 \textbf{Citation and reference style} & Do all your citations and references comply with the required bibliographical style? \\
 \textbf{Avoiding plagiarism} & Have you acknowledged the work of others and distinguished it from your own appropriately? \\
 \textbf{Standard of English (or any modern language you use)} & Have you proof-read your report carefully to remove all typos and grammatical errors? \\
\bottomrule

\end{tabulary}
\end{minipage}
\end{table}

\begin{question}[subtitle={Activity: Reviewing your report}] Apply the criteria in Table 1 to review your current report and write up a summary of your assessment.

\begin{guidance}For each criteria, consider the related prompts to help you assess your report overall, and write down any further work needed for your next stage.
\end{guidance}\end{question}
%%Hack to correct tcbox behaviour
\color{black}

\chapter{Study 4: Third research increment}

\section{Introducing stage 4}
Your Stage 4 work will build on your pilot work in Stage 3 and any adjustments to your research design as a result. With reference to our 5-stage framework, the activities which are in focus in Stage 4 are summarised in Table 1, which also provides some guidelines for your interaction with your supervisor during this stage.

Table 1 - Stage 4 activities

\begin{table}[htbp]
\begin{minipage}{\linewidth}
\setlength{\tymax}{0.5\linewidth}
\centering
\small
\begin{tabulary}{\textwidth}{@{}llll@{}} \toprule
 \textbf{Research activities} & \textbf{Stage 4} \textbf{(20\% of project length)} & \textbf{Effort within stage} & \textbf{Suggested focus of your interaction with your supervisor} \\
\midrule

 \textbf{Identifying the research problem} & Adjust, if needed & 1\% & \\
 \textbf{Reviewing the literature} & Adjust, if needed & 1\% & \\
 \textbf{Setting research aim and objectives} & Adjust, if needed & 1\% & \\
 \textbf{Choosing the research design} & Adjust, if needed & 2\% & \\
 \textbf{Gathering and analysing evidence} & Conduct initial data\slash evidence generation and analysis & 50\% & Initial application of collection and analysis methods, and any improvements required \\
 \textbf{Interpreting and evaluating findings} & Critically assess findings up to this point & 15\% & Critical thinking in assessing findings, and any improvements required \\
 \textbf{Reporting, critical reflection and conclusions} & Assess research progress and write up Stage 4 report & 25\% & Any further improvements required \\
 \textbf{Work planning and risk management} & At stage start, review work from previous stage and project risk; adjust plan as needed If you have received feedback from supervisor on your previous stage work, adjust plan to include any revision recommended & 5\% & Any major adjustment required \\
\bottomrule

\end{tabulary}
\end{minipage}
\end{table}

\begin{question}[subtitle={Activity: Understanding the effort needed in this stage}] Consider Table 1 carefully, taking notice of the entries in the `Effort within stage' column. Make a note of the activities which are most prominent in this stage and what is expected under each.

\begin{solution}Gathering and analysing evidence will constitute by far your major effort in this stage (50\% of study time): the expectation is that the main part of your data and evidence collection and analysis should take place in this stage.
\end{solution}\end{question}
%%Hack to correct tcbox behaviour
\color{black}

\section{Gathering, analysing and presenting evidence}
Building on your pilot work, in this stage you will apply your chosen methods and procedures to gather and analyse your data and evidence. Remember that this may well include modelling, design or prototyping work, depending on your chosen research strategy.

You are expected to make significant progress on this aspect of your research during Stage 4, therefore, it is important that you plan your work carefully in consultation with your supervisor.

Depending on your chosen research strategy and methods, you may end up gathering a large quantity of data or evidence, so you will need to think carefully at how you will handle it. In this respect, you will need to consider:

- How to organise and store your `raw' data and evidence. These represent anything you gather directly, say, full transcripts of participants' interviews, survey responses, logs of experimental data, copies of documents, etc. It is essential that you manage them carefully to ensure you don't loose track of important information, and that you can always refer back to them in your subsequent analysis. You may need to include such evidence or a sample of it in an appendix of your dissertation as demonstration of the work you have done.

- How to summarise data and evidence in the body of your stage report, and later in your final dissertation. This will depend on the nature of your data and evidence, and you will need to ensure that your summaries are appropriate to convey the essence of the evidence you have generated and to support your analysis, so that you can build academic arguments which relate evidence to findings, and findings to aim and objectives.

- How to structure your report. Depending on your chosen research strategies and methods, different structures are possible. For instance, you may choose to start with a section which summarises all your evidence followed by one in which you analyse it, which may work well, for instance, for Survey Research. Alternatively, you could have separate sections each including a summary and analysis of a sub-set of your evidence: this may be appropriate for mixed methods research, with each section dealing with a different kind of data, or for Design Science Research, with each section addressing a different design cycle. Whatever you choose, it is important that your report is effective in presenting your evidence and findings in a clear and rigorous manner.

\begin{question}[subtitle={Activity: Gathering and analysing your data and evidence}] Plan and execute your research on gathering and analysing data and evidence, and write up your summary and analysis. Ensure that you plan carefully how to manage your raw data\slash evidence and how to structure your report.

You should aim to complete the bulk of this work by the end of Stage 4.

\begin{guidance}To complete this work successfully, it is essential that you discuss and agree your detailed work plan with your supervisor from the start, and monitor your progress on a regular basis.

This is a substantial activity, which will take up around 50\% of your study time in this stage.
\end{guidance}\end{question}
%%Hack to correct tcbox behaviour
\color{black}

\section{Interpreting and evaluating findings}
Having gathered and analysed your data and evidence, you must now interpret your findings in relation to your aim and objectives, and generally evaluate them in terms of their contribution to knowledge and possible limitations.

By interpretation I mean being able to answer the following questions:

\begin{itemize}
\item What does the analysis of your data\slash evidence you have conducted show? What are your findings?

\item How do your findings relate to your aim and objectives?

\item How do your findings relate to what you know from the literature or from professional practice?

\item Which new knowledge do your findings contribute?

\item What do your findings fail to achieve?

\end{itemize}

\subsubsection{Activity: Interpreting and evaluating your initial findings}
Consider your analysis of data and evidence, and based on it, address each of the above questions. In doing so, you should express well-formed academic arguments, with explicit reference to the data and evidence you have gathered and analysed.

\begin{guidance}You can refresh your understanding of academic arguments by looking back at the related materials for Stage 1.

Your interpretation and evaluation of findings will be, of course, limited by the data\slash evidence you have gathered and analysed up to this point. You will revisit and expand this work in Stage 5 in order to complete your project.

Depending on the extend of your current data\slash evidence gathering and analysis, this activity could take you up to 15\% of your study time.

\end{guidance}

\section{Drafting your abstract and extended abstract}
Within your dissertation you will need to include both an \textbf{abstract} at the beginning. You course may also asked you to produce an \textbf{extended abstract} as an appendix. In this section, we discuss the differences between the two and provide guidance to help your write them.

Both are summaries of your research, but they have a different structure and purpose.

The \textbf{abstract} provides a short summary of your whole research for a specialist audience, that is you can assume the reader has good knowledge of the topic and field of study. It should be a stand-alone item, so that it can be understood without reference to any other part of your dissertation.

Its content should convey succinctly your chosen research problem, how and where it arises and its significance, your research aim and research design, key results obtained by your research, their evaluation and their implications for further research or professional practice. You will have seen many examples of abstracts in reviewing the academic literature.

\subsubsection{Activity: Drafting your abstract}
Write a draft abstract for your project, which should reflect your research progress to date.

\begin{guidance}Typically an abstract should be a piece of text of approximately 200--300 words (once complete), without headings and sub-headings, so only using paragraph breaks, if needed. It should not include any citations or references, and abbreviation and acronyms should be kept to a minimum.

Your course may have different guidelines which you should check and follow to produce your abstract.
\end{guidance}

\paragraph{}
The \textbf{extended abstract} is a précis of your dissertation intended for a more general audience, so that it should be easily read and understood by someone with only a superficial knowledge of your topic. As with the abstract, it should be a stand-alone item without any reference to your full dissertation. However, it is a lengthier piece of academic writing, structured with headings and sub-headings, including citations and references, and possibly tables, figures and diagrams to help you present and summarise your work.

\begin{question}[subtitle={Activity: Drafting your extended abstract}] Write a draft extended abstract for your project, which should reflect your research progress to date.

\begin{guidance}Your extended abstract should be 4 to 6 pages in length (once complete) and a common structure is as follows:

\begin{itemize}
\item Title --- the same as your dissertation

\item Introduction and background --- an outline of your research problem in its context, its significance, and the knowledge gap addressed by your research

\item Aim and objectives --- from your dissertation

\item Research design --- an outline of your research design

\item Results --- a summary of the evidence collected and analysed, and your key findings

\item Discussion --- how significant your findings are in relation to research problem and knowledge gap

\item Conclusion and future work --- your overall conclusions and possible follow-up research

\item References --- selected references cited in the body of your extended abstract

\end{itemize}

Your course may have different guidelines which you should check and follow to produce your extended abstract.

\end{guidance}\end{question}
%%Hack to correct tcbox behaviour
\color{black}

\section{Reporting in Stage 4}
At the end of Stage 4, you should complete a report, extending that of Stage 3 and covering the work you have carried on in this stage. Its recommended structure and content are indicated in Table 1.

Table 1 -- Report structure and content guidance

\begin{table}[htbp]
\begin{minipage}{\linewidth}
\setlength{\tymax}{0.5\linewidth}
\centering
\small
\begin{tabulary}{\textwidth}{@{}ll@{}} \toprule
 \textbf{Structure} & \textbf{Content guidance} \\
\midrule

 Proposed title & Your title should continue to capture succinctly research problem and aim \\
 Abstract & This should provide a succinct summary of your research aimed at a specialised audience \\
 Sect 1 - Introduction 1.1 Background to the research 1.2 Justification for the research & This section should provide an introduction to your research topic in its wider context (as background) and your justification of why the research is worth pursuing. It should be well articulated and supported by evidence \\
 Sect 2 - Literature review 2.1 Review of existing relevant knowledge 2.2 Critical summary, including knowledge gap to be addressed by the research & Your review should provide a critical account of your in-depth engagement with the academic (and other) relevant literature, including identifying key trends, ideas and possible knowledge gaps. Most of your citations should point to academic articles. Your critical summary should highlight key insights from your review and provide a strong justification for your proposed research. Both coverage and depth of your review matter. You should ensure that your review is well structured, with a logical narrative flow and your arguments are well supported by evidence \\
 Sect 3 - Research definition 3.1 Problem statement 3.2 Aim, objectives, tasks and deliverables 3.3 Knowledge contribution & You should ensure that your research problem is well articulated and appropriate for your course and your personal and professional circumstances, that your aim and objectives are consistent with research problem, that tasks and deliverables break down your objectives appropriately and are clearly related to your chosen research methods, and that the intended knowledge contribution of your research is clearly articulated \\
 Sect 4 - Research design 4.1 Evidence and data 4.2 Research strategy and methods 4.3 Research procedures 4.4 Ethical, legal and EDI considerations & This section should demonstrated your critical engagement with all elements of research design, including a detailed account of the data and evidence needed in your research, the research methods and research strategies you will to apply, and how you will apply them within your project. Your account should be supported by a clear rationale and insights from the related literature, and appropriately justified in relation to your research problem, aim and objectives. It should also demonstrate your careful consideration of ethical and legal matters, and that your research will comply with your course and university requirements \\
 Sect 5 - Analysis and interpretation 5.1 Summary and analysis of evidence 5.2 Summary of key findings 5.3 Interpretation in relation to aim and objectives & This section should demonstrate substantial progress towards gathering and analysing your data and evidence, and interpreting them in relation to aim and objectives. It should demonstrate a competent execution of your research design, present appropriate summaries of evidence and data, supported by raw data in an appendix if needed. Key findings should be clearly identified and logically connected to evidence, with good critical reflection on their implications for aim and objectives. \\
 Sect 6 - Assessment of your proposed research 6.1 Qualification fit 6.2 Personal and professional fit 6.3 Technical skills and resources needed 6.4 Statement of feasibility 6.5 Personal reflection on research process & In this section you should continue to argue how your research is a good fit across all criteria. You should provide a clear rationale as to why you think what you are proposing is feasible. You should also reflect on your growing understanding of the research process, including key learning and aspects you have found particularly challenging. \\
 Sect 7 - Planning, scheduling and risk assessment 7.1 Statement of progress 7.2 Key priorities in follow-up stage 7.3 Risk assessment & In this section you should reflect on the progress you have made in Stage 2 and establish your priorities for the next stage. You should also review your risk assessment as appropriate. \\
 References & You should keep your growing references in good order and ensure you apply the required bibliographical style consistently. Ideally, you should use a BMT to generate and integrate your references within your report \\
 Appendix - Extended abstract & If needed, you should include your draft extended abstract as an appendix. This should provide a structured summary of your research aimed at a generalist audience. \\
 Appendix - Raw evidence & If relevant, you should include a sample of your raw data as an appendix \\
 Appendix - Work schedule & You should include your revised work plan as an appendix \\
 Appendix - Risk assessment table & You should include your updated risk table as an appendix \\
\bottomrule

\end{tabulary}
\end{minipage}
\end{table}

\begin{question}[subtitle={Activity: Putting your report together}] Using your word processor of choice, and starting from your previous report, complete your Stage 4 report by applying the structure and guidance in Table 1, and making good use of your notes and summaries from all related activities you have carried out so far.

\begin{guidance}In this first pass at putting together your report, you should focus primarily on completeness, ensuring that each section includes at least draft content.
\end{guidance}\end{question}
%%Hack to correct tcbox behaviour
\color{black}

\paragraph{}
As in the previous stages, after you have filled in your report you should review and revise it iteratively until you are happy with your account, and are ready to move on.

Table 1 - Criteria to review your report

\begin{table}[htbp]
\begin{minipage}{\linewidth}
\setlength{\tymax}{0.5\linewidth}
\centering
\small
\begin{tabulary}{\textwidth}{@{}ll@{}} \toprule
 \textbf{Criteria} & \textbf{Prompts} \\
\midrule

 \textbf{Completeness} & Are all sections of the suggested structure completed in line with the guidance provided? \\
 \textbf{Good academic writing practices} & Have you applied good academic writing practices throughout? \\
 \textbf{Logical structure and flow} & Have you structured your narrative appropriately to ensure a logical flow of arguments? \\
 \textbf{Supporting references or evidence} & Are your key arguments supported by appropriate references or other evidence? \\
 \textbf{Citation and reference style} & Do all your citations and references comply with the required bibliographical style? \\
 \textbf{Avoiding plagiarism} & Have you acknowledged the work of others and distinguished it from your own appropriately? \\
 \textbf{Standard of English (or any modern language you use)} & Have you proof-read your report carefully to remove all typos and grammatical errors? \\
\bottomrule

\end{tabulary}
\end{minipage}
\end{table}

\begin{question}[subtitle={Activity: Reviewing your report}] Apply the criteria in Table 1 to review your current report and write up a summary of your assessment.

\begin{guidance}For each criteria, consider the related prompts to help you assess your report overall, and write down any further work needed for your next stage.
\end{guidance}\end{question}
%%Hack to correct tcbox behaviour
\color{black}

\chapter{Stage 5: Completing your dissertation}

\section{Introducing Stage 5}
Stage 5 concludes your research project, so that by the end of this stage you will have complete your project and written up your full dissertation, ready for submission.

With reference to our 5-stage framework, the activities which are in focus in Stage 5 are summarised in Table 1, which also provides some guidelines for your interaction with your supervisor during this stage.

Table 1 - Stage 5 activities

\begin{table}[htbp]
\begin{minipage}{\linewidth}
\setlength{\tymax}{0.5\linewidth}
\centering
\small
\begin{tabulary}{\textwidth}{@{}llll@{}} \toprule
 \textbf{Research activities} & \textbf{Stage 5} \textbf{(30\% of project length)} & \textbf{Effort within stage} & \textbf{Suggested focus of your interaction with your supervisor} \\
\midrule

 \textbf{Identifying the research problem} & Adjust, if needed & 1\% & \\
 \textbf{Reviewing the literature} & Adjust, if needed & 1\% & \\
 \textbf{Setting research aim and objectives} & Adjust, if needed & 1\% & \\
 \textbf{Choosing the research design} & Adjust, if needed & 1\% & \\
 \textbf{Gathering and analysing evidence} & Complete data\slash evidence generation and analysis & 40\% & Overall quality and quantity of data\slash evidence and their analysis \\
 \textbf{Interpreting and evaluating findings} & Critically assess all findings & 20\% & Overall critical thinking in assessing findings \\
 \textbf{Reporting, critical reflection and conclusions} & Assess entire research and write up dissertation & 35\% & Overall quality of dissertation \\
 \textbf{Work planning and risk management} & At stage start, review work from previous stage and project risk; adjust plan as needed If you have received feedback from supervisor on your previous stage work, adjust plan to include any revision recommended & 1\% & \\
\bottomrule

\end{tabulary}
\end{minipage}
\end{table}

\begin{question}[subtitle={Activity: Understanding the effort needed in this stage}] Consider Table 1 carefully, taking notice of the entries in the `Effort within stage' column. Make a note of the activities which are most prominent in this stage and what is expected under each.

\begin{solution}In this stage too, gathering and analysing evidence will constitute your major effort (40\% of study time), although considerable effort will also be needed in assessing your research overall and completing your dissertation. In particular you shouldn't underestimate the time needed to complete and polish the dissertation so that is ready for submission, which is why the framework assume a significant effort in this stage.
\end{solution}\end{question}
%%Hack to correct tcbox behaviour
\color{black}

\section{Completing your research}
Building on Stage 4, in this stage you will complete your work on gathering and analysing evidence, and presenting and evaluating your findings in your dissertation.

\begin{question}[subtitle={Activity: Completing your data and evidence gathering and analysis, and interpretation of findings}] Complete your research on gathering and analysing data and evidence, and the interpretation of your findings in terms of your aim and objectives. Expand on your analysis and summaries from your Stage 4 report.

\begin{guidance}Ensure you continue to manage your raw data and evidence carefully, and that your report presents all your data\slash evidence, findings and their interpretation in a clear and rigorous manner.

This activity is likely to take up to 40\% of your study time, assuming you were able to complete the bulk of your data and evidence collection and analysis in Stage 4. If that's not the case, you should increase your effort accordingly.
\end{guidance}\end{question}
%%Hack to correct tcbox behaviour
\color{black}

\section{Assessing your research}
Once you have completed your work on gathering and analysing data and evidence, and interpreting your findings, it is time for you to reflect on your whole project and draw some overall conclusions. These will form the body of the concluding chapter of your dissertation, for which you are asked to think critically about each of the following:

\begin{itemize}
\item \textbf{Evaluation against aim and objectives}: you should reflect on the extent your research has met its stated aim and objectives. The interpretation of your findings against aim and objectives is a good starting point to draw these summary conclusions. While your interpretation may be deep and detailed, with reference to specific data and evidence, here you are expected to highlight key conclusions based on such an interpretation. It is not necessary for your research to have met your aim and objectives fully: in this section you need to make a critical assessment of what your research has actually achieved.

\item \textbf{Evaluation against the academic body of knowledge}: this requires you to assess the extent your findings have added to the body of knowledge in your field of study, including whether they support or question findings already known from the literature you have reviewed. You should show awareness of how your own research relates to the wider academic context.

\item \textbf{Implications for practice (if any):} here you should reflect on ways in which your research may be relevant to professional practice, if applicable, including how it could lead to change and improvement. If your research is purely theoretical, then you can skip this section, and focus on the previous two items instead.

\item \textbf{Validity of the research}: this require you to assess your research in terms of construct, internal and external validity. You should refer back to Stage 3 materials to refresh your understanding of validity.

\item \textbf{Further research}: your research may have shed light on aspects of your research problem, or highlighted other related research problems, which you did not have the time to explore in your project. This is the place for you to discuss those of more relevance and to indicate how future research can build on the work you have done.

\item \textbf{Personal reflection on your research experience}: whether or not your research project is your first experience of academic research, you should reflect on what you have learnt from a personal standpoint in relation to thinking and behaving like an academic researcher. You should address how your mindset and skills have changed, or how you would do things differently should you start anew, and any other relevant thoughts you may have.

\end{itemize}

\begin{question}[subtitle={Activity: Assessing your research overall}] Assess your overall research in relation to the above points, and write appropriate summaries of each for inclusion in your dissertation.

\begin{guidance}For each point above, consider the related guidance to help you assess your research overall. Note that this assessment should consider all the work you have conducted in your project.
\end{guidance}\end{question}
%%Hack to correct tcbox behaviour
\color{black}

\section{Finalising and submitting your dissertation}

\subsubsection{Compiling a full draft of your dissertation}
At this point you should have all the materials you need to complete a full draft of your dissertation, which is already a remarkable achievement!

Your dissertation constitution your final report, extending your Stage 4 and covering the work you have carried on in this stage. Its recommended structure and content are indicated in Table 1.

Table 1 -- Report structure and content guidance

\begin{table}[htbp]
\begin{minipage}{\linewidth}
\setlength{\tymax}{0.5\linewidth}
\centering
\small
\begin{tabulary}{\textwidth}{@{}ll@{}} \toprule
 \textbf{Dissertation template} & \textbf{Links to study materials} \\
\midrule

 Proposed title & Your title should continue to capture succinctly research problem and aim \\
 Abstract & This should provide a succinct summary of your research aimed at a specialised audience \\
 Sect 1 - Introduction 1.1 Background to the research 1.2 Justification for the research & This section should provide an introduction to your research topic in its wider context (as background) and your justification of why the research is worth pursuing. It should be well articulated and supported by evidence \\
 Sect 2 - Literature review 2.1 Review of existing relevant knowledge 2.2 Critical summary, including knowledge gap to be addressed by the research & Your review should provide a critical account of your in-depth engagement with the academic (and other) relevant literature, including identifying key trends, ideas and possible knowledge gaps. Most of your citations should point to academic articles. Your critical summary should highlight key insights from your review and provide a strong justification for your proposed research. Both coverage and depth of your review matter. You should ensure that your review is well structured, with a logical narrative flow and your arguments are well supported by evidence \\
 Sect 3 - Research definition 3.1 Problem statement 3.2 Aim, objectives, tasks and deliverables 3.3 Knowledge contribution & You should ensure that your research problem is well articulated and appropriate for your course and your personal and professional circumstances, that your aim and objectives are consistent with research problem, that tasks and deliverables break down your objectives appropriately and are clearly related to your chosen research methods, and that the intended knowledge contribution of your research is clearly articulated \\
 Sect 4 - Research design 4.1 Evidence and data 4.2 Research strategy and methods 4.3 Research procedures 4.4 Ethical, legal and EDI considerations & This section should demonstrate your critical engagement with all elements of research design, including a detailed account of the data and evidence needed in your research, the research methods and research strategies you will to apply, and how you will apply them within your project. Your account should be supported by a clear rationale and insights from the related literature, and appropriately justified in relation to your research problem, aim and objectives. It should also demonstrate your careful consideration of ethical and legal matters, and that your research will comply with your course and university requirements \\
 Sect 5 - Analysis and interpretation 5.1 Summary and analysis of evidence 5.2 Summary of key findings 5.3 Interpretation in relation to aim and objectives & This section should provide a comprehensive account of all the data and evidence you have gathered, appropriately analysed with key findings identified and interpreted in relation to aim and objectives. It should demonstrate that a competent execution of your research design, present appropriate summaries of evidence and data, supported by raw data in an appendix if needed. Key findings should be clearly identified and logically connected to evidence, with good critical reflection on their implications for aim and objectives. \\
 Chapter 6 - Evaluation and conclusion 6.1 Evaluation against aim and objectives 6.2 Evaluation against related work in the literature 6.3 Implication for practice 6.4 Validity of the research 6.5 Further work 6.6 Personal reflection on your experience of & In the section you should reflect on the extent your research has met its stated aim and objectives, bringing together all your findings from both primary and secondary research work. You should also reflect how it has contributed new knowledge in relation to the literature you have reviewed. You should also assess the validity of your research and consider any implication for further research and, if applicable, for professional practice. Considerable insight is evident in the implications of the research identified for relevant stakeholder groups. You should also reflect on what you have learnt from a personal standpoint in relation to thinking and behaving as an academic researcher. \\
 References & You should include all your references and ensure you apply the required bibliographical style consistently. Ideally, you should use a BMT to generate and integrate your references within your report \\
 Appendix - Extended abstract & If needed, you should include your draft extended abstract as an appendix. This should provide a structured summary of your research aimed at a generalist audience. \\
 Appendix - Raw evidence & If relevant, you should include a sample of your raw data as an appendix \\
\bottomrule

\end{tabulary}
\end{minipage}
\end{table}

\begin{question}[subtitle={Activity: Putting your dissertation together}] Using your word processor of choice, and starting from your previous report, complete your dissertation by applying the structure and guidance in Table 1, and making good use of your notes and summaries from all related activities you have carried out.

\begin{guidance}Although the dissertation structure and content we recommend is fairly standard, it is possible that they don't not match exactly the requirements of your own course, which may provide a different template for you to follow. Indeed you should check and apply your course guidance, and map the structure and content in Table 1 to what is required in your course of study.
\end{guidance}\end{question}
%%Hack to correct tcbox behaviour
\color{black}

\subsubsection{Revising your draft for compliance to requirements}

Now that you have a complete draft of your dissertation, you should revise it to ensure it meets your course requirements.

In our experience, a Masters dissertation is usually in the range of 10,000 to 15,000 words. Often, references, abstract, extended abstract and all other appendices are excluded from the word count, but figure and table captions are included.

There is also an expectation that the content of your dissertation is balanced across the different chapters, although it is normal for some chapters to be more substantial than others. The recommended distribution of content across the full body of your dissertation, based on our recommended dissertation structure, is indicated in Table 1, as a percentage of total. This is not a hard and fast constant, but can provide a baseline for you to get an idea of the relative weight of the different chapters of your dissertation. In adapting it to the needs of your own project and course, however, you should ensure you maintain a good balance across the whole piece.

Table 1 --- Word count: recommended breakdown of content

\begin{table}[htbp]
\begin{minipage}{\linewidth}
\setlength{\tymax}{0.5\linewidth}
\centering
\small
\begin{tabulary}{\textwidth}{@{}lllll@{}} \toprule
 \textbf{Element} & \textbf{Breakdown} & \textbf{Recommended word count distribution} & \textbf{Equivalent for 10,000 word dissertation} & \textbf{Equivalent for 15,000 word dissertation} \\
\midrule

 \textbf{Chapter 1} \textbf{Introduction} & Background to the research Justification for the research Definitions (if any) Dissertation outline & 10\% & 1000 & 1500 \\
 \textbf{Chapter 2} \textbf{Literature review} & Review of existing relevant knowledge Critical summary, including knowledge gap & 20\% & 2000 & 3000 \\
 \textbf{Chapter 3 Research definition} & Problem statement Aim, objectives, tasks and deliverables Knowledge contribution & 10\% & 1000 & 1500 \\
 \textbf{Chapter 4 Research design} & Evidence and data Research strategy and methods Procedures Ethical considerations & 15\% & 1500 & 2250 \\
 \textbf{Chapter 5 Analysis and interpretation} & Summary and analysis of evidence Summary of key findings Interpretation in relation to aim and objectives & 30\% & 3000 & 4500 \\
 \textbf{Chapter 6 Evaluation and conclusion} & Evaluation against aim and objectives Evaluation against the academic body of knowledge Implications for practice (if any) Validity of the research Further research Personal reflection on your research experience & 15\% & 1500 & 2250 \\
\bottomrule

\end{tabulary}
\end{minipage}
\end{table}

There is also an expectations that your dissertation conforms to some standard presentation conventions, which we have summarised in Table 2.

Table 2 --- Standard presentation conventions for Masters dissertation

\begin{table}[htbp]
\begin{minipage}{\linewidth}
\setlength{\tymax}{0.5\linewidth}
\centering
\small
\begin{tabulary}{\textwidth}{@{}ll@{}} \toprule
 \textbf{Fonts} & Use a standard font that is easy to read, e.g. Times New Roman or Arial, with font size 11 or 12 \\
\midrule

 \textbf{Margins and spacing} & Leave appropriate margins on both the left and the right of the page, typically around 2 cm. Use 1.5 line spacing \\
 \textbf{You identifiers} & Include your name and student identifier, possibly as a header or as part of the title page \\
 \textbf{Title page} & Include a title page containing your research title. Usually the following statement is also required: ``A dissertation submitted in partial fulfilment of the requirements for the degree of $<$name of degree$>$'', where you should replace $<$name of degree$>$ with your own degree title \\
 \textbf{Table of content} & Include a table of content after the title page \\
 \textbf{Page numbers} & Number all pages, including references and appendices. In particular, use lower-case Roman numerals on the preliminary pages -- iii, iv, v, etc. -- and Arabic numerals starting from page 1 at the beginning of Chapter 1. \\
 \textbf{Chapter and section numbering} & Number chapters sequentially using Arabic numerals starting with 1. Number sections sequentially starting with the chapter number, e.g. 1.1, 1.2, etc. for sections in Chapter 1. Number sub-sections sequentially starting with the section number, e.g. 1.1.1, 1.1.2, etc. for sub-sections in Section 1.1. You should avoid sub-sub-sections, but if needed, number them sequentially starting with the sub-section number, e.g. 1.1.1.1, 1.1.1.2, etc. for sub-sub-sections in Sub-section 1.1.1. \\
 \textbf{Figures and tables} & Number all figures and tables sequentially, starting with their chapter number, e.g. 1.1, 1.2, etc. for figures in Chapter 1. Include appropriate captions positioned after figures and before tables \\
 \textbf{Lists of figures and tables} & List all figures and tables after your table of content. For each include both their number and caption \\
 \textbf{Citations and references} & Apply the required bibliographical style throughout \\
 \textbf{Verb tense} & Your dissertation is an account of what you did in your project, so you should report your work using the past tense throughout \\
\bottomrule

\end{tabulary}
\end{minipage}
\end{table}

\begin{question}[subtitle={Activity: Reviewing your dissertation}] Review you current dissertation draft and make all necessary adjustments to ensure it meets the guidance and requirements above.

\begin{guidance}You should ensure that your dissertation draft fits within the overall word count, and matches the suggested relative weight of each chapter.

You should also ensure you have applied the required presentation conventions.

You should check and apply any further guidance from your course of study.

\end{guidance}\end{question}
%%Hack to correct tcbox behaviour
\color{black}

\subsubsection{Final check and submission}

Before submitting your dissertation, you should perform a final check, focusing on the following aspects:

- \textbf{Logical coherence}: you should ensure that all research elements of your dissertations are coherent and consistent with each other, so that there is a logical progression from research problem, to aim and objectives, to research design and its execution, to findings and conclusions.

- \textbf{Academic writing}: you should ensure that academic arguments are well formed, including being well-supported by secondary and\slash or primary evidence, that the language you use is clear and precise, and there is a good balance between description and critical reflection.

- \textbf{Proof-reading}: you should remove grammatical errors and typos, and ensure that punctuation is correct. You should also check that the narrative makes sense to the reader, for which I strongly advise you ask for help from a friend or family member: even if they are not experts on the topic of your project, they should be able to follow what you have written and get the gist of your work.

- \textbf{Conformance to presentation conventions}: you should ensure that your dissertation conforms to the the requirements of your course, follows its presentation conventions, its length is within the word limit, and its content is well balanced between chapters.

\begin{question}[subtitle={Activity: Performing your final check }] Assess your dissertation draft against each of the points above. Revise and iterate until you are ready to submit.

\begin{guidance}Revising your dissertation for submission is very important as you can lose a substantial proportion of marks should any of these aspects not be addressed carefully and to the expected standards.
\end{guidance}\end{question}
%%Hack to correct tcbox behaviour
\color{black}

\paragraph{}
You should now be ready to submit your dissertation. You should, of course, follow the instructions for your course of study to do so.

\subsubsection{How your dissertation will be assessed}
After submission, your dissertation will go through your university's assessment process, which is designed to ensure that your work is assessed fairly against Masters research benchmarks and your course learning outcomes. The specifics of this process will depend on your own university and course of study, something you should investigate carefully.

You should also investigate the assessment criteria applied to your work. Typically, your Masters dissertation will be assessed from the following perspectives, although the specific marking scheme applied within your course may break each further:

\begin{itemize}
\item \textbf{Research definition and research design}: this refers to an appropriate articulation and justification of the research problem in its wider context, including your critical review of the academic literature to contextualise and justify your research problem and knowledge contribution, a well developed and justified research design, and well constructed academic arguments

\item \textbf{Evidence gathering, analysis, interpretation and conclusion}: this refers to a competent execution of your research design, an adequate amount of evidence gathered and analysed, an appropriate interpretation of your findings, and a critical evaluation of your research overall

\item \textbf{Presentation}: this refers to how your dissertation is put together, its cohesiveness and logical flow, including abstract and extended abstract, and its conformance to conventions, including an appropriate use of tables, figures and diagrams to summarise and present your work

\end{itemize}

The assessment of your work under these perspectives will contribute to your final grade, which will be established by your examiners in relation to Masters level quality benchmarks, like those summarised in Table 1, which are typical in the UK.

Table 1 --- Typical grade benchmarks for Masters dissertations, based on UK quality standards

\begin{table}[htbp]
\begin{minipage}{\linewidth}
\setlength{\tymax}{0.5\linewidth}
\centering
\small
\begin{tabulary}{\textwidth}{@{}ll@{}} \toprule
 \textbf{Grade} & \textbf{Quality descriptor} \\
\midrule

 \textbf{Distinction} & All elements of the dissertation are present, including Abstract and Extended Abstract, and are of a high standard. In particular, the dissertation demonstrates: * advanced, authoritative understanding and analysis of key issues and complex problems * strong evidence of a critical approach to own work and that of others * competent use of a wide range of evidence in support of academic arguments * appropriate and well justified selection of research strategies and methods, applied competently to own research * originality and independence of thought * compelling narrative which is coherently and logically presented * excellent presentation standards * excellent research potential \\
 \textbf{Merit} & All elements of the dissertation are present, including Abstract and Extended Abstract, and are of a good standard. In particular, the dissertation demonstrates: * good understanding and analysis of key issues * good evidence of a critical approach to own work and that of others * good use of evidence in support of academic arguments * appropriate selection of research strategies and methods, applied reasonably well to own research * some originality * coherent and logically presented narrative * good presentation standards * good research potential \\
 \textbf{Pass} & Some elements may be weak or missing, but all thresholds are reached. In particular, the dissertation demonstrates some of: * limited understanding and analysis of key issues * limited evidence of critical approach to own work and that of others * limited use of relevant evidence in support of academic arguments * some appropriate choices of research strategies and methods, with limited application to own research * plausible narrative * adequate standards of presentation \\
 \textbf{Fail with possibility of resubmission} & Many elements are very weak or missing, and not all thresholds are reached. In particular, the dissertation demonstrates many or all of: * superficial understanding and analysis of key issues * weak evidence of critical approach to own work and that of others * gaps in the use of evidence in support of academic arguments * inappropriate choice or application of research strategies and methods * weak narrative * poor standards of presentation \\
 \textbf{Fail} & The dissertation has critical flaws and omissions, so that is not recoverable via a resubmission. In particular, the dissertation demonstrate many or all of: * lack of understanding and analysis of key issues * lack of critical approach to own work and that of others * little or no use of evidence in support of academic arguments * inappropriate choice or application of research strategies and methods * incoherent and confused narrative * inadequate standards of presentation \\
\bottomrule

\end{tabulary}
\end{minipage}
\end{table}

\begin{question}[subtitle={Activity: Assessing your own dissertation}] Apply the three perspectives above together with the benchmarks of Table 1 to your dissertation. Write down your assessment of your work as a result.

\begin{guidance}Your course of study may provide some detailed guidance on how your dissertation will be assessed. If that's the case, you should compare that guidance to the advice in this handbook, and apply it in your own assessment of your dissertation. You should only assess the content of the dissertation as is, disregarding all other knowledge you will have of your research which is not reported.

You should take an objective stance, considering both strengths and weaknesses of your work. You could also ask a friend or a family member to assess your dissertation, then compare their assessment with yours.

\end{guidance}\end{question}
%%Hack to correct tcbox behaviour
\color{black}

\chapter{Closing}

\section{Concluding remarks}
Your dissertation submission concludes your Masters project work. If you have come that far, then you deserve much praise and this is a significant intellectual achievement. A successfully project is a strong indication that you have mastered a wider range of research and transferrable skills, which are of great value to your professional development and provide a strong foundation for any future academic or professional research you may choose to pursue, including doctoral studies.

We hope you have found conducting your own research rewarding, despite, and perhaps because of, the challenges that undoubtably you will have encountered and overcome during your project. We also hope you will have found this handbook valuable in supporting you throughout your project.

We wish you all the best for your future career and studies!

\chapter{Glossary}
\textbf{Academic literature}: the collection of all published research and scholarly work.

\textbf{Active reading}: engaging with written materials in a way which allows you to assimilate the important points in an effective manner

\textbf{Artificial Intelligence (AI)}: a sub-discipline of Computing, aimed at creating software systems able to simulate human intelligence processes.

\textbf{Bibliographical database}: a searchable collection of academic literature.

\textbf{Bibliographic Management Tool (BMT)}: software tool used to collect and save searchable information concerning articles and other literature sources reviewed during research, including digital copies of articles and personal notes, and to generate references, reference lists and bibliographies in a variety of bibliographical styles.

\textbf{Bibliography}: a separate section, usually towards the end of a document, which collects full bibliographical information of sources, whether cited or not, which are relevant to the content of the document.

\textbf{Bibliographical style}: a set of rules which determine what citations and references should look like in academic writing.

\textbf{Categorical (or nominal) data} are qualitative data corresponding to categories that cannot be ordered and on which mathematical operations and function don't apply, e.g., full-time vs part-time study.

\textbf{Citation}: a short-cut that appears in the main body of a written academic piece to refer to a specific source in the academic (or other) literature.

\textbf{Citation searching:} a technique for exploring the literature based on citations in academic articles.

\textbf{Critical writing}: writing displaying a good balance between description, analysis, synthesis and evaluation.

\textbf{Correlation:} statistical relationship among two or more measures, concerning how changes in one measure are reflected in changes in the others.

\textbf{Data analytics tool:} sophisticated digital tools which extend spreadsheet capabilities for collating and visualising data to include some degree of automated analysis, both statistical and based on Machine Learning algorithms.

\textbf{Descriptive statistics:} measures used to provide meaningful summaries of data points within a dataset.

\textbf{Gantt chart}: a scheduling chart used to plan, organise and monitor activities and work over the duration of a project.

\textbf{Google Scholar}: a web search engine specialising in scholarly content.

\textbf{Grey literature}: collection of information produced by organisations whose primary or commercial remit is not publishing, such as as universities, government bodies or businesses (other than publishers). It includes pre-publication and non-peer-reviewed articles, theses and dissertations, research and committee reports, government reports, conference papers, accounts of ongoing research, etc.

\textbf{Interval} \textbf{data} are ordinal data, but for which we can calculate precisely the interval between any two data points. For instance, calendar dates are interval data in the sense that we can calculate precisely the interval between two given dates, e.g., the number of days in between.

\textbf{Kanban board}: an agile project management tool to help individuals or teams organise and track their progress on specific tasks during a project.

\textbf{Machine Learning (ML)}: a branch of Artificial Intelligence aimed at creating software systems able to learn autonomously and improve from experience.

\textbf{Numerical} \textbf{data} are numbers, either discrete or continuous, e.g., the number of students on a module (discrete) or the average temperature in the UK in July 2023 (continuous). Numerical data can be ordered, and mathematical and statistical operations and functions apply.

\textbf{Nominal data}: same as categorical data

\textbf{Ordinal} \textbf{data} are data that can be arranged in an order, but are not necessarily numerical, e.g., a 5-point Likert scale from (1) Strongly disagree to (2) Disagree to (3) Neither agree nor disagree, to (4) Agree to (5) Strongly agree. While these values can be arranged in the order indicated, mathematical and statistical operations and functions don't apply.

\textbf{Plagiarism}: passing off someone else's work, words or ideas as your own, often as a deliberate attempt to deceive.

\textbf{Research asset}: information which is needed, gathered or generated by your research, including articles, data, images, tables, notes, etc, organised and managed in a disciplined and systematic manner.

\textbf{Qualitative data}: descriptive data, like texts, words, images, sounds, etc., including categorical (or nominal) data, e.g., full-time vs part-time study or employed vs unemployed.

\textbf{Quantitative data}: data that can be quantified or measured, and be given numerical values, including numerical, ordinal and interval data.

\textbf{Reference}: the full bibliographic information of a source in the academic (or other) literature which is cited in an academic text.

\textbf{Risk}: the likelihood of something going wrong combined with the impact that may have on a project.

\textbf{Spreadsheet}: a digital tool used to capture, display, analyse and manipulate data arranged in tables.

\textbf{Version control system}: a set of conventions or tools to keep track of different versions of documents and other research assets.

\chapter{References and further reading}

Cottrell, Stella. Critical Thinking Skills : Effective Analysis, Argument and Reflection, Bloomsbury Publishing Plc, 2017. ProQuest Ebook Central, https:\slash \slash ebookcentral.proquest.com\slash lib\slash open\slash detail.action?docID=6234915.

A practical handbook to develop your critical thinking skills, packed with activities and practical advice.

Cryer, Pat. The Research Student's Guide to Success, McGraw-Hill Education, 2006. ProQuest Ebook Central, https:\slash \slash ebookcentral.proquest.com\slash lib\slash open\slash detail.action?docID=316264.

A comprehensive introduction to research skills for post-graduate research students. Some elements are more relevant than others to T802 research, so this is a good reference book to dip in and out.

Potter, Stephen (ed.) (2006) \emph{Doing postgraduate research}, SAGE study skills, 2nd edition., Los Angeles London New Delhi, SAGE.

Another comprehensive introduction to research skills for post-graduate research students, possibly more suited to PhD students than T802 students.

Etzold, Daniel 2020). My Workflow for Reading Scientific Papers. \href{https://betterhumans.pub/my-workflow-for-reading-scientific-papers-d4b27dbb38a6}{https:\slash \slash betterhumans.pub\slash my-workflow-for-reading-scientific-papers-d4b27dbb38a6}

Some practical advice from a practitioner. This is a personal account, rather than a tried-and-tested method. Nevertheless, it contains some good tips that you may find useful.

\textbf{References}

Klopper, Rembrandt, Lubbe, Sam \& Rugbeer, H., 2007. The matrix method of literature review. Alternation, 14(1), pp.262--276.

\textbf{References}

Keshav, S. (2007) `How to read a paper', ACM SIGCOMM Computer Communication Review, 37(3), pp. 83--84.

\textbf{References}

Cottrell, S. (2005) Critical thinking skills. Basingstoke: Palgrave Macmillan

\textbf{References}

Cottrell, Stella (2017). Critical Thinking Skills : Effective Analysis, Argument and Reflection, Bloomsbury Publishing Plc.. ProQuest Ebook Central, https:\slash \slash ebookcentral.proquest.com\slash lib\slash open\slash detail.action?docID=6234915.

\textbf{References}

M. K. S. Sastry and C. Mohammed, ``The summary-comparison matrix: A tool for writing the literature review,'' IEEE International Professional Communication 2013 Conference, 2013, pp. 1--5, doi: 10.1109\slash IPCC.2013.6623891.

\textbf{References}

Booth, W., Colomb, G.G. and Williams, J.M. (1995) \emph{Making good arguments: an overview}. The Craft of Research, London: The University of Chicago Press.

Simon, H. A. (1969). The sciences of the artificial. The MIT Press.

\textbf{References}

Data Protection in the EU, \href{https://commission.europa.eu/law/law-topic/data-protection/data-protection-eu_en}{https:\slash \slash commission.europa.eu\slash law\slash law-topic\slash data-protection\slash data-protection-eu\_en} (Last accessed: February 2023)

\textbf{References}

UKRI Equality, diversity and inclusions key principle. \href{https://www.ukri.org/about-us/policies-standards-and-data/good-research-resource-hub/equality-diversity-and-inclusion/}{https:\slash \slash www.ukri.org\slash about-us\slash policies-standards-and-data\slash good-research-resource-hub\slash equality-diversity-and-inclusion\slash }(last accessed: November 2022)

\end{document}
