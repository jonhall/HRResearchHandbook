\chapter{The 5-stage Masters project framework} \label{ch:framework}

This handbook gives you a 5-stage framework with which to approach your research project. The framework has been refined over many years of working with hundreds of Masters students at the Open University\footnote{If you know anything about the Open University, you'll know that it is a distance--learning university, most of its students work at a distance and so don't attend lectures. If you think about it, the ``at--a--distance'' model is precisely what you'll be doing in your Masters project -- you won't have lectures! This makes the Open University model really appropriate for a Master project.}, UK.

We provide an overview in this chapter --- we will break the framework down into stages in the following chapters, each chapter giving detailed guidance for that stage.

\section{What do we mean by framework?}\label{sect:WhatDoesFrameworkMean}
Writing a Master dissertation is a complex task: the goal is usually a complete 10,000--15,000 word dissertation on the research you have conducted which meets the required academic standards. There are many risks that you will face in writing it, which include:

\begin{itemize}
\item Misunderstanding what is required

\item Running out of time

\item Not having the skills or resources you need

\item Choosing the wrong methods or techniques, or not applying them correctly
\item Not having access to the data and evidence you need
\item Or simply, life getting in the way.
\end{itemize}

Our framework will allow you to manage these risks to give you the best chance of submitting something that will satisfy the examiners and show your Masters research skills off in their best light.

The framework comes with recommended stages and research activities, as well as metrics and guidelines to help you parameterise your project based on your course requirements, develop a work plan that meets your needs, and manage your work and your interaction with your supervisor effectively.

Study within our framework and you'll have your best chance of succeeding.

Because the key research requirement is to make a contribution to knowledge, every Masters student will be working at the leading edge of their discipline. One corollary of this is that every dissertation will be different. Your project will allow you to show how your domain expertise in a focussed topic has developed through your studies and your dissertation will reflect that. That's your goal!

For us in defining our framework, it raises a problem -- we cannot possibly know the details of your final dissertation. We can't know even simple things, like how many pages it will have exactly. We can't know what your arguments will be or what your conclusions or future work will contain. Neither can we know which literature you will consult, or what the seminal paper in it is.

That would be a problem if our only focus was the final form your dissertation will take. But it isn't!

What we teach in this handbook is the process of arriving at your final dissertation. That's why we structure our framework into stages -- each stage building on the previous, each moving you further down the line to your final dissertation. And because we have seen literally hundreds of students follow the framework, we know it can work and work well. 

If this has convinced you that we're onto something, welcome aboard!

Your first step with us will be to learn about the research process that you will follow.

\section{The research process and its key activities}\label{sect:ResearchActivities}
A research process is the sequence of activities you undertake when conducting academic research.

Research is messy --- looking into the unknown means there is no road map to follow and you will often have to retrace your steps and try different paths. So, moderate your expectations -- you shouldn't expect a linear, orderly research process!

Instead, the research process is an iterative, incremental, and adaptive process of knowledge discovery --- it iterates through its research activities in small incremental steps, adapting as your knowledge grows.

The exact path you will follow in your research project will be unique to you. However, you will go through some widely recognised activities, which are common to all research processes.

\subsection{Identifying the research problem}\label{ssect:IdentifyingTheResearch}
A key concept in academic research is that of \emph{research problem}. A research problem captures the knowledge gap to be addressed (\emph{the need}) by the research within a particular field of study (\emph{the context}). A well-defined research problem is the foundation of any research project as it clarifies the purpose of the research and its intended outcomes.

In this handbook, you will learn a practical approach to identifying and articulating your chosen research project.

\subsection{Reviewing the literature}\label{ssect:ReviewingTheLiterature}
One of the major tasks you will have in compiling your dissertation is to give the reader sufficient background that they can check where you're coming from. To do so, you will be referring to the work of other authors and researchers throughout your project; this is collectively known as the academic literature. Depending on your degree, you may already have practise in doing this from your previous studies, for instance in writing academic essays for your assignments. For your Masters, however, the more definitive you can be in referencing the academic literature the better. This means that your review of the academic literature will be a significant part of your research project.

Your review has two key functions.

Firstly, it will help you contextualise your research in what is already known and where the knowledge gaps are, so that you can focus your research more tightly on a research problem which is relevant and significant to your field of study. This includes evaluating your own findings against those of others in your field.

Secondly, it will help you understand how to conduct your research: by reading your field's literature, you'll learn what is academically acceptable in terms of the methods and techniques which you can apply to address your research problem.

Reviewing the literature is both time consuming and demanding, and will require you to apply many skills which this handbook will help you develop.

\subsection{Setting your aim and objectives}\label{ssect:SettingYourAim}
While your research problem helps you establish the contribution to knowledge of your research, your aim and objectives will help you establish the scope and boundaries of your project by stating the specific way in which your research will address the problem.

Scoping your project appropriately is necessary in all research, but is particularly critical at Masters level with its many constraints on the time you have and the methods you can feasibly apply.

\subsection{Developing the research design}\label{ssect:DevelopingTheResearch}
Your research design will summarise, explain and justify how your research is conducted for the benefit of your examiners and other readers\footnote{Although we all want to think differently, it is often the case that the examiners are the only people outside of your supervisory relationship that will read your dissertation. We'll talk later about how you can get your family involved -- which may double or even triple your readership!}. As with your literature review, it will develop during your project: at the start, it will be a collection of your initial ideas; by the end, it will be an account of what you did.

Your research design will depend on many things. There are some obvious ones, including the type of research problem you are trying to address, the intended outcome of your research, the sort of evidence you will need, and the research strategies and methods that are acceptable within your chosen discipline.

There are others that are less obvious: you'll have to tailor your research design to the resources and expertise you easily have access to, and your preferred research style\footnote{You may have heard of a \emph{learning style}. A research style is similar -- you may prefer, for instance, to talk to people about the practical ways they work as opposed to building theories of how they do so.}. As a key participant in your own research, your own personal views and values will also affect your choices while developing your research design.

There are also some esoteric ones: for instance, some philosophical beliefs which are shared by researchers in your field in regard of how knowledge can be generated. Although fascinating, to really understand them well takes decades of study of others' writings, on top of which there will be hours on hours of conversations with other researchers. Clearly, this is not something you will have much time to dwell upon in your Masters project --- you have a supervisor that can advise you, instead.

Research design is a field of study in its own right, one which has grown out of the many different academic traditions and ways of thinking across academic disciplines and subject areas. It's not easy to digest and is far from stable or complete: every so often a new book on research methods will be offered for review by a publisher. Research design is possibly one of the most challenging aspects of doing academic research and can be puzzling\footnote{It can sometimes be puzzling for experienced researchers too!} for those just starting.

Because of this, you should rely on your supervisor for advice, particularly at the beginning, although your understanding will mature during your project. This handbook will help you develop your understanding and assess which choices are most appropriate for your project.

\subsection{Gathering and analysing evidence}\label{ssect:GatheringAndAnalysing}
This is where you use your research design to gather and analyse the data and evidence\footnote{Data is raw information with no interpretation attached, while evidence is information interpreted to support your academic arguments. The two are closely linked, with data forming the basis of evidence, so that they are often used interchangeably. We will return to this topic in~\Cref{sect:evidenceAndData}.} you need for your project. This is possibly the most exciting part of conducting academic research!

You're going to spend a lot of time doing this, and you'll have to circle back\footnote{One PhD student we know got to within 2 months of finishing their PhD when they realised that their research design led them to incorrect data analysis. That was a real critical moment in their studies, but they got it done in the end. Just.} to this activity as you learn more about your field and as your research develops.

\subsection{Interpreting and evaluating findings}\label{ssect:InterpretingAndEvaluating}
This is where you review your findings critically to establish the extent your research has met its stated aim and objectives. Because it relies on the way you interpret your evidence, it is closely related to and typically influences the way evidence is generated and analysed, so that you will iterate between these two research activities for a large proportion of your project.

\subsection{Reporting}\label{ssect:Reporting}
Throughout your project you will need to provide a written account in a critical\footnote{Developing your critical voice is, er, critical to being a successful researcher. On a good research project, an effective critical voice will be highly valued by your examiners.} and rigorous manner on all aspects of your research, including both new insights and the process you followed to arrived at them. Therefore, integral to your writing is a critical appraisal of the strengths and limitations of the research you have conducted, the overall conclusions which can be drawn from it and what the impact on future research might be.

While your course may only require you to submit a final dissertation for assessment, it is essential that you write incremental accounts of your work throughout your project. This will allow you to improve your academic writing skills, share what you have done with your supervisor for feedback and formative assessment, and develop your dissertation incrementally.

\subsection{Reflecting}\label{ssect:Reflecting}
Reflection is what makes your learning more effective, relevant and useful to your own practice. Reflection is important in any kind of learning, but particularly in the experiential learning\footnote{Experiential learning is `learning by doing,' using reflection to think critically about what you have experienced and relate it to knowledge and how it can apply to new situations.} of conducting research.

Our framework therefore emphasises reflection as an essential activity within the research process, to develop and consolidate the knowledge and skills you need to be a competent and confident researchers.

\subsection{Planning work}\label{ssect:PlanningWork}
Some researchers will have worked much of their lives on one problem. You don't have that luxury -- or burden, depending on your perspective!

Time-bound research, like that you are just about to start, must be planned carefully as early mistakes and missteps can be very difficult to correct later.

For this reason, we recommend that you build a \emph{work plan} and we show you how. At its core, a work plan summarises all activities and work to be conducted to complete the research, how to organise and execute them in the time-span of the project and which milestones must be met.

Standard project management techniques apply, but with the caveat that in academic research you are usually planning for the unknown, so not everything can be figured out upfront and some level of adaptation will be needed. Hence, understanding what might change, related risk and how to manage it is an essential part of  planning your research project.

\subsection{Managing risk}\label{ssect:ManagingRisk}
Risk captures the likelihood of something going wrong combined with the impact that will have on your project, on time, resources, and outcome. In theory, both positive and negative impacts can be assessed, but very often the focus is on what can affect your project in a bad way. It is essential that you make an assessment of these risks at the start of your project and identify ways to manage them, then monitor and make adjustments as you go along.

\subsection{How the activities relate to each other}\label{ssect:HowTheActivities}
~\Cref{fig:researchactivities} depicts the relations between activities in the research process: as you can see, these are complex relationships, with much tangling and intertwining. Note also that reflecting, reporting, work planning and risk management apply to both individual activities, and the process overall, which is why in the figure they are separated from the other activities.

\begin{figure}[htbp]
\centering
\includegraphics[width=\textwidth]{researchActivities.pdf}
\caption{Research activities and their relations}
\label{fig:researchactivities}
\end{figure}

The main relations between activities are as follows (with reference to the numbering in the figure):

\begin{enumerate}
\item The research problem helps you identify relevant literature to review
\item The literature you review helps you identify current knowledge and existing gaps and frame the research problem
\item The research problem informs what the aim and objectives should be
\item The research problem identifies the phenomena under study, which inform your choice of research design within your research design
\item The literature also helps you establish which research design to apply and how
\item The objectives inform the research design to discharge them
\item The research designs tells you how to generate and analyse evidence
\item The analysis of evidence feeds your interpretation and evaluation
\item Your interpretation and evaluation may require further evidence generation and analysis
\item Your interpretation and evaluation assess the extent aim and objectives are discharged
\item Aim and objectives gives you criteria for your evaluation and interpretation
\item Activities feed your project work plan
\item Your work plan informs what you have to do and when
\item Activities give rise to risk to be managed
\item Managing risk will constrain the way you carry out activities
\item Activities and their outputs must be documented in your reports
\item Your reflection on all you do will trigger changes and adjustments to all your activities
\end{enumerate}

The figure highlights the interconnected nature of everything you do during a research project, and how the need to revisit some activities, perhaps as the result of reflection, will trigger adjustments to others which are closely related, so that there is no simple linear path through that all projects will follow. The figure also does not assume there is a set starting point, although, as you will see in the next chapter, identifying the research problem from an initial topic is the way we recommend you get your project started. Equally, research can go on forever as reflecting of what you know may trigger new avenues of enquiry to follow. Of course, in the context of a Masters project you only have a set amount of time to complete your work, so that at some point you will need to decide that you have done enough and finalise your project dissertation.

\section{The 5-stage framework for your research}\label{sect:5stageFramework}

This handbook recommends you organise your research project into five major stages. We've developed our 5-stage framework from our experience of working with hundreds of successful\footnote{And a few, a very few, unsuccessful ones{\ldots}} Masters research students, our knowledge of the iterative, incremental and adaptive nature of the research process, and our awareness of the many risks that must be managed to be successful.

Given this, the stages have the following characteristics:

\begin{itemize}
\item Each stage contains many interrelated research activities\footnote{Don`t worry, we're going to go into exhaustive detail about this the remainder of the book!}, so that you will often revisit earlier activities as you learn more about research and move forward in your project. From one stage to the next, however, the balance between those activities will change reflecting your increasing expertise in them and the progress you have made

\item Each stage builds incrementally and adaptively on work from the previous stage(s)

\item Each stage includes critical reflection on what you have achieved and learned, and how this should inform the work ahead

\item Each stage includes some writing so that your dissertation builds with your expertise

\item Each stage includes re-assessing risk and adapting your work plan.

\end{itemize}

The 5-stages framework is summarised in Figure~\ref{fig:slugDiagram}, which indicates the relative length of each stage in the project, and the research activities which are in focus at each stage.

\begin{question}[subtitle={Activity: Investigating stages and activities}] Consider Figure~\ref{fig:slugDiagram} carefully, taking notice of which activities are more prominent in each stage. Write down your observations on how activities are distributed over the length of a project. 
\begin{solution}
Some activities are very prominent in the early stages, specifically identifying the research problem and reviewing the literature, and minimal later on. 

Others start and become prominent in later stages, particularly generating and analysing evidence, and interpreting and evaluating findings. 
The 5 stages are designed to balance the activities you need to carry out in your project; each is allocated a recommended proportion of the overall project time, as outlined in~\Cref{fig:slugDiagram}. Their precise balance is influenced by many factors, including topic, your previous research experience, if any, and your supervisor's advice, so take the figure with a pinch of salt -- it's only meant to show you how the focus on research activities changes in the stages. We will consider more detailed descriptions by stage in the follow-up chapters, including indicative timings, but again, nothing is written in stone.

Others still, feature throughout the project: this is particularly the case for reflecting and reporting. 

Work planning and risk management start early on and are sustained through most of the project.  
\end{solution}\end{question}
%%Hack to correct tcbox behaviour
\color{black}

The 5 stages are designed to balance the activities you need to carry out in your project; each is allocated a recommended proportion of the overall project time, as indicated in Figure~\ref{fig:slugDiagram}, which can help you establish your own project timeline.

\begin{example}{Example of project timeline}
Assuming you have 48 weeks in total for your project, you can use the proportions in Figure~\ref{fig:slugDiagram} to establish an approximate duration for each of your project stages. Specifically, Stage 1 will take 15\% of 48 weeks, or just over 7 weeks, Stage 2 will take the same length, Stage 3 will take 20\% of 48 weeks, or just under 10 weeks, etc.
\end{example}

You should, of course, take the guidelines in the figure with a pinch of salt, as the precise balance between the stages and their activities in an actual project is influenced by many factors, including the topic you have chosen, any previous research experience you may have, your supervisor's advice, requirements established by your course, etc. 

\begin{figure}[htbp]
\centering
\includegraphics[width=\textwidth]{Figures/slugDiagram.pdf}
\caption{Change of focus between research activities across the project stages}
\label{fig:slugDiagram}
\end{figure}

\todo{the following activity is wrong -- it should be about the overall timeline}
%\begin{question}[subtitle={Activity: Considering the stages and activities in the framework }] Consider~\Cref{fig:slugDiagram} carefully, taking notice of which activities are more prominent in each stage, and the relative length of each stage within the framework. Jot down a timeline for your project based on your expected study length in weeks, identifying when each stage starts and ends.

\begin{question}[subtitle={Activity: Jotting down your project timeline}] Consider~\Cref{fig:slugDiagram} carefully, taking notice of the relative length of each stage within the framework. Jot down a timeline for your project based on your expected study length in weeks, identifying when each stage starts and ends.

\begin{solution}
Your timeline will be specific to your project, of course. You can refer to the example above to get you started.
\end{solution}
\end{question}
%%Hack to correct tcbox behaviour
\color{black}

\section{Planning your work}\label{sect:WorkPlan}

\todo{this doesn't work unless we have a table for the whole project or augment the figure with a ballpark figure for each activity}

In the following chapters we will break down the research activities, stage by stage, indicating what is expected at each stage to help you organise and plan your project work in detail. At this point, however, it is worth outlining an initial project workplan which organises the main research activities along your project timeline. 

In this section, you will use the 5-stage framework as a tool to help you put together an initial overarching plan for your project work. At each stage, you will then revise the plan, adding detail as to the activities required, and making adjustments based on the progress you were able to make. 

Planning a research project is no different from planning other kinds of project, so that the following practices are required for you to plan and keep track of your research work:
\begin{itemize}
	\item Identifying key milestones and related deliverable for your project. A milestone is a point on your project timeline where you can assess your progress against your ultimate aim and objectives, while deliverables are concrete artefacts you must have produced at that point, be that a first draft of your literature review, a full characterisation of you research problem, etc.
	\item Identifying the work you will need carry out to reach each milestone
	\item Breaking down that work into discrete tasks, also identifying how they may depend on each other
	\item Scheduling those tasks in the time available, taking any dependency into consideration, and
	\item Making an efficient use of the time you have available.
\end{itemize}

For your project plan you can use of the standard planning charts and tools available; in fact, you may be already be familiar with some of them. Gantt charts are among the most commonly used.

Briefly, \textbf{Gantt charts} are scheduling charts that you can use to plan and organise your project work. Using a Gantt chart will give you an overview of how your work will be broken down and organised over time, including an indication of how much time you will spend on each task, when tasks should start and end, and which tasks might overlap at any point in your project. As such, it is also useful to communicate your project plan to third parties. As a generic project management tool, a Gantt chart can be quite sketchy or very detailed: you should aim for something quite light, but still including all main tasks and deliverables of your project.

\begin{question}[subtitle={Activity: Investigating Gantt charts}] If you are unfamiliar with Gantt charts, conduct a web search on introductory materials and examples. Select and review a small subset of resources you have found. Write down key points about using Gantt charts.

\begin{guidance}In your web search, you'll likely find several links to digital tools supporting Gantt charts, tutorials on how to construct one, or even ready-made templates to fill in. You should focus on materials which can help you construct your own project Gantt chart.
\end{guidance}\end{question}
%%Hack to correct tcbox behaviour
\color{black}

As an important caveat, you should not expect your plan to be cast in stone! There are too many unknowns in a research project for you to be able to predict upfront exactly what is going to happen. Therefore, you shouldn't spend too much time trying to plan every single thing, and you should review your plan often to monitor progress and make adjustments as needed. In summary, your project plan should keep you on track, but should not be a straightjacket.

\subsection{Your project plan}

\todo{to edit once we have the table...}

The 5-stage framework identifies for you key milestones and deliverables at a general level: each stage is broken down into research activities, each with clear deliverables, and a written report on the work completed by the end of each stage.

Within each stage there is much detailed work to be done to reach such milestones. Therefore, at the beginning of each stage, you will be encouraged to plan such work carefully, by matching the generic activities in the framework to specific tasks in your project.

\begin{question}[subtitle={Activity: Identifying your project tasks}] Consider the deliverables in Table~\ref{tab:stage1} and write down how they correspond to tasks in your own project. Allocate time to those tasks based on the study time you estimated in the previous activity.

\begin{guidance}Feel free to break tasks down into sub-tasks, but be wary of your plan becoming too detailed at this point.
\end{guidance}\end{question}
%%Hack to correct tcbox behaviour
\color{black}


\begin{question}[subtitle={Activity: Breaking down your study time}] Consider the timeline for your project that you defined as part of the activity in Section~\ref{sect:5stageFramework}. Based on the number of weeks you allocated to Stage 1 in your timeline and the number of study hours per week at your disposal, use the percentages in Table~\ref{tab:stage1} to calculate the suggested study time for each of its activities.

\begin{solution}Assuming you have allocated 6 weeks to Stage 1, with an average of 20 hours of study per week, then you have a total of 6 x 20 = 120 hours of study for the whole stage. Of these, you will need to spend 18 hours on identifying your research problem (15\%, that is 120 x 0.15), 36 hours on reviewing the literature (30\%), 6 hours on setting your aim and objectives (5\%), etc., including studying the relevant parts of this handbook and carrying out the activities within. While your actual study time may deviate from these calculations, they should still give you a strong indication of how you should allocate proportions of your time among the different research activities.
\end{solution}\end{question}
%%Hack to correct tcbox behaviour
\color{black}


\begin{question}[subtitle={Activity: Constructing your Gantt chart}] Based on the outcome of your previous activities, create a Gantt chart for your own project in relation to Stage 1.

\begin{guidance}If you already use a different project scheduling approach for projects in your own practice, then feel free to use that instead, as long as it can be used to organise your project work effectively and in a way which is easy to communicate with your supervisor, with whom you should share and discuss your plan.

You will augment and adjust your plan in all stages of your project.
\end{guidance}\end{question}
%%Hack to correct tcbox behaviour
\color{black}

\subsection{Key practices for managing your time efficiently} 
Your plan is more likely to work if you make an effective use of your time. Here is some key practices you should keep in mind:

\paragraph{Finding time for your research} You need to make a realistic assessment of how much time you will have for your project, considering other commitments you may have, professional or personal: this is important for all students, but crucial for part-time study. Your course will expect you to spend an average number of hours per week on your project work throughout its duration, so you need to ensure you can dedicate that time to your project on a regular basis. But it is not just about quantity: you should choose your most productive time to dedicate to your research--- in may be that you are a morning person, or you can focus better at night. In either case, it is important you come to your work with a fresh mind, so that you can concentrate on the tasks at hand.

\paragraph{Ensuring continuity of effort} As already noted, research requires continuity of effort, as you need the time to develop your critical thinking and other research skills, and that's a continuous process. If you fall behind, you may find it difficult to catch-up. In your project plan, you must therefore ensure that your study time is arranged evenly and regularly over its duration and that any study break you may take is short, not to affect your continuous progress.

\paragraph{Making your spare time productive} There will be much to read through your project, so you should always keep something to read with you, which you can look at while waiting for something, say a bus, or in a queue or travelling on public transports. You may be surprised of how much reading you can do while waiting.

\paragraph{Avoiding postponing and procrastinating} There will be tasks you may find harder than others, so it would be natural to put them off or engage in displacement activities. You should avoid that and focus on what must be done to meet your project milestones.

\paragraph{Scheduling extra time} Inevitably things don't always go to plan or activities may take longer than estimated, so you should always factor in extra time for the things that may go wrong. If you find you don't need it, you can always allocate the time to other tasks or take a longer break, but at least you won't fall behind in subsequent tasks.

\paragraph{Being adaptive} Research is about looking into the unknown, so you can't expect to be able to plan everything upfront and in great detail. Your planning should keep you on track to reach your major milestones, but should also be flexible. Don't spent too much time trying to plan every single thing to do: keep your planning light and ensure you return to your plan often to make adjustments as your project unfolds.

\paragraph{Investing in the right tools for the job} While you may find it a nuisance to have to spend time learning new tools, such as a Bibliographic Management Tool (BMT) or some advanced word processing features, doing so will save you time in the long run, so make sure you invest in setting up your new systems and learning new skills earlier on in your project.

\begin{question}[subtitle={Activity: Reflecting on your study practices}] Assess your own study habits and practices in light of the above advice. Write down things you think you do well and things you could or should change for your research project.

\begin{guidance}You should take a balance stance and consider both your current strengths and weaknesses: build on the former and put some effort in addressing the latter.
\end{guidance}\end{question}
%%Hack to correct tcbox behaviour
\color{black}


%LR -- commented out for the moment, while we re-design the table and re-consider its inclusion in this chapter
%Table 1 gives you the recommended breakdown of activities between and within stages.
%
%This is designed to help organise and plan your project work to match the requirements of your specific Masters course, something you will consider as part of work planning in the next chapter, which focuses on Stage 1.
%
%Table 1 also provides recommendations on critical aspects of your work you should discuss with your supervisor stage by stage. We have already discussed the importance to your project success of a close working relationship with your supervisor, of engaging in an active dialogue on a regular basis, and your responsibility to drive and manage that interaction. In practice, your supervisor will only have limited time to dedicate to your project, therefore it is essential you make an effective use of that time: the advice in the table will help you guide your interaction with your supervisor, ask key questions and make the time you spend together more productive.
%
%Table 1 --- Recommended breakdown of project work between and within stages, alongside suggested focus of interaction with supervisor
%
%\begin{ltabulary}{1.4\textwidth}{@{}LLLLLLLLLLLLL@{}}
%\textbf{Research activities} & \textbf{Stage 1~} & \textbf{Suggested focus of your interaction with your supervisor} & \textbf{Stage 2~} & \textbf{Suggested focus of your interaction with your supervisor} & \textbf{Stage 3~} & \textbf{Suggested focus of your interaction with your supervisor} & \textbf{Stage 4} & \textbf{Suggested focus of your interaction with your supervisor} & \textbf{Stage 5} & \textbf{Suggested focus of your interaction with your supervisor} \\
% &  & \textbf{Identifying the research problem} & Develop problem statement and intended contribution to knowledge & Suitability of research problem for academic research and to meet the requirements of specific~ course & Adjust, if needed &
%
% & Adjust, if needed &
%
% & Adjust, if needed &
%
% & Adjust, if needed &
%
% \\
% &  & \textbf{Reviewing the literature} & Compile initial draft of literature review and plan remaining review & Scope of literature review and possible gaps & Complete full draft of literature review
%Draft critical summary of key insights from literature review & Suitability of literature review structure and narrative flow
%Appropriate logical argumentation
%Critical thinking in driving insights & Adjust, if needed &
%
% & Adjust, if needed &
%
% & Adjust, if needed &
%
% \\
% &  & \textbf{Setting research aim and objectives} & Define aim and~ objectives & Suitability and feasibility of aim and objectives in relation to research problem and project time & Adjust, if needed &
%
% & Finalise aim and objectives, and define tasks and deliverables & Suitability of tasks and deliverables from objectives & Adjust, if needed &
%
% & Adjust, if needed &
%
% \\
% &  & \textbf{Developing the research design} &
%Consider elements of~ research design, including data and evidence, and types of research methods
%
%Complete ethics assessment, including applying for permission to proceed, if needed
% &
%Consistency of choices in relation to aim and objectives.
%
%Compliance with own university's ethical and legal guidelines
% & Revise research design, with detailed consideration of data and evidence, research strategy and research methods & Any further adjustment needed to the research design & Complete research design, with detailed consideration of data and evidence, research strategy, research methods and procedures & Suitability of research procedures & Adjust, if needed &
%
% & Adjust, if needed &
%
% \\
% &  & \textbf{Gathering and analysing evidence} & n/a &
%
% & n/a &
%
% & Conduct pilot work to test aspects of your research design & Scope of your pilot work & Conduct initial data/evidence generation and analysis & Initial application of collection and analysis methods, and any improvements required~ & Complete data/evidence generation and analysis & Overall quality and quantity of data/evidence and their analysis \\
% &  & \textbf{Interpreting and evaluating findings} & n/a &
%
% & n/a &
%
% & n/a &
%
% & Critically assess findings up to this point & Critical thinking in assessing findings, and any improvements required~ & Critically assess all findings and the overall research conducted & Overall critical thinking in assessing findings \\
% &  & \textbf{Reporting} & Write up Stage 1 report & Demonstration of critical thinking and good academic writing, and any improvements required & Write up Stage 2 report & Any further improvements required & Write up Stage 3 report & Any further improvements required & Write up Stage 4 report & Any further improvements required & Write up dissertation & Overall quality of dissertation \\
% &  & \textbf{Reflecting} & At stage end, think critically about experiential learning in relation to the research process and its activities &
%
% & At stage end, think critically about experiential learning in relation to the research process and its activities &
%
% & At stage end, think critically about experiential learning in relation to the research process and its activities &
%
% & At stage end, think critically about experiential learning in relation to the research process and its activities &
%
% & At stage end, think critically about experiential learning in relation to the research process and its activities & Overall critical thinking in assessing experiential learning \\
% &  & \textbf{Planning work} & Draw your initial project timeline and a detailed plan for Stage 1 & Appropriateness of initial work plan & At stage start, plan Stage 2 work in detail, including any revision or additional work needed from the previous stage. & Any major adjustment required & At stage start, plan Stage 3 work in detail, including any revision or additional work needed from the previous stage. & Any major adjustment required & At stage start, plan Stage 4 work in detail, including any revision or additional work needed from the previous stage. & Any major adjustment required & At stage start, plan Stage 5 work in detail, including any revision or additional work needed from the previous stage. & Any major adjustment required \\
% &  & \textbf{Managing risk} & Assess project risk & Consideration of major risk & Review project risk and make any required adjustments for the next stage & Any major adjustment required & Review project risk and make any required adjustments for the next stage & Any major adjustment required & Review project risk and make any required adjustments for the next stage & Any major adjustment required &  &
%\end{ltabulary}

\section{Critical success factors}\label{sect:successFactors}

You should keep in mind the following critical success factors during your research project:

\paragraph{Making good use of the 5-stage framework}  The framework in this handbook is the result of decades of practice, helping Masters students like you succeeding in their first academic project. However, it is not a straightjacket, and you should adapt it to your own needs. The framework was developed with the novice academic researcher in mind, but it may be that you already have some research experience, in which case some of the the timing suggested by the framework may be too generous. Equally, your own course may require you to submit some interim report as formative assessment, in which case the stage lengths may need altering to match your course requirements. Nevertheless, the framework provides a scaffolding to help you take control when planning and conducting your project.

\paragraph{Making good use of this handbook and its activities}  This handbook is designed to accompany you in your first journey into academic research, so you should follow its stage by stage advice to guide your project work. The handbook is also designed to be very practical, so that there are plenty of activities for you to do. The activities are there to help you make steady progress with essential work for your project and to help you develop your research skills. From them, your dissertation will emerge, hence we strongly advise you complete them systematically.

\paragraph{Your working relation with your supervisor}  Your supervisor is your strongest study ally, a research expert who can guide and advise you on all aspects of your project, with whom you can discuss your research ideas, and who can assess that you are making sufficient progress at each point in your journey. It is essential that you develop a good working relationship with your supervisor, meet regularly and have an open and honest dialogue throughout your project.

\paragraph{Your self-drive and commitment} The course you are studying may provide you with some structure to help you progress your project, including, for instance formal assessment points. However, you will be in the driving seat most of the time. You must choose your topic and research problem, how to investigate it, and how to organise your research time in detail; it is also up to you to understand the academic standards required and to assess which research skills you must develop to meet them. Above all, it is up to you to commit to a sustained effort for the duration of your project, which may be several months.

\paragraph{Continuous effort} Successful research requires continuity, so that you will need to set aside a sufficient and regular time for your research project every week, ensuring you keep making progress as you go along. Long breaks are incompatible with conducting research: while in your previous studies you may have been able to stop and start, and possibly cram lots of work around assignment deadlines, conducting research requires lots of time for reflection and for ideas to develop and mature, something you can't compress close to a deadline. Conducting research requires endurance, perseverance and continuous effort: it is a marathon, not a sprint!

\paragraph{Thinking and writing}  There is a crucial interaction between reading, thinking, and writing in research: reading informs your thinking; your thinking is what you try to express in your writing; your writing helps you make sense of what you have read, and hence of what you think, and informs more reading, thinking and writing. Because of this, while you will spend a lot of time reading and thinking, it is also essential that you write as you go along. The more regularly you write, the easier it will be for you to develop academic writing skills and the more critical your thinking will be.

\paragraph{Your dissertation}  Your research will be assessed on the basis of your dissertation. Your dissertation must therefore demonstrate that you have mastered a wide range of research skills and can communicate your academic research effectively in writing. In fact, this is usually more important than any feature your proposed solution to your research problem might have. At Masters level, that you `solve' your research problem may not even be necessary! What matters most is your scholarly and critical attitude to each element of your research and your grasp of what academic research entails as demonstrated by your written dissertation.

\paragraph{Making good use of your study support}  Last but not least, it is important that you assess your progress as you go along, and make the most of the support which is provided to you by your supervisor, and any other adviser on your course. There will be times when you will find working on your project very challenging and may lose confidence in your ability to complete it. Those are the times when it is going to be particularly important for you to reach out and ask for extra help, even if your first instinct may be to hide. All researchers experience such feelings at one time or another, so do not be discouraged: talk to an adviser and work through the difficulties you are having to overcome them.

\section{Framework Takeaways}\label{sect:FrameworkTakeaways}

\begin{itemize}
\item the research process is the process you undertake when conducting academic research; its many activities are deeply connected, so that the process is adaptive, iterative and incremental, and can be messy
\item the 5-stage framework is designed to help you plan and conduct your project. It is a practical framework, with plenty of activities and guidance to help your develop your research skills and perform essential research tasks to complete your project
\item you can use the guidance provided to map the 5 stages of the framework to your project timeline 
\item following the framework will help you succeed, but other factors matter too (see~\Cref{sect:successFactors}).
\end{itemize}



%We close this chapter with some reflection on what you have learnt.
%
%\begin{question}[subtitle={Activity -- Reflection on learning}] Consider the content of this chapter and write down key things you have learnt or that have surprised you. For each indicate why they are notable or relevant to you, and how you may apply them in new situations or to inform your future learning.

%\begin{guidance}tbd
%\end{guidance}\end{question}
%%Hack to correct tcbox behaviour
\color{black}
\endinput