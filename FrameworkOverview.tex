\chapter{The 5-stage research project framework} \label{ch:5StageFramework}



This book gives you a 5-stage framework with which to approach your research project. The framework has been refined over many years of working with hundreds of Masters students at The Open University\footnote{If you know anything about The Open University, you'll know that it is a distance learning university: most of its students work at a distance and so don't attend lectures. If you think about it, the \enquote{at--a--distance} model is precisely what you'll be doing in your Masters project -- you won't have lectures! This makes The Open University model really appropriate for a Master project.}, UK.



We provide an overview in this chapter --- we will break the framework down into stages in the following parts of this book, each part giving detailed guidance for that stage.

\section{What do we mean by framework?}\label{sect:WhatDoesFrameworkMean}
Writing a research dissertation is a complex task. At Masters level, the goal is usually a complete 15,000 word dissertation\footnote{At Doctoral level, it can be three times as much.} on the research you have conducted which meets the required academic standards. There are many risks that you will face in writing it, which include:

\begin{itemize}
\item misunderstanding what is required
\item running out of time
\item not having the skills or resources you need
\item choosing the wrong methods or techniques, or not applying them correctly
\item not having access to the data and evidence you need
\item or, simply, life getting in the way.
\end{itemize}

Our framework will allow you to manage these risks to give you the best chance of submitting something that will satisfy the examiners and show your research skills off in their best light.

The framework comes with recommended stages and research activities, as well as metrics and guidelines to help you parameterise your project based on your course requirements, develop a work plan that meets your needs, and manage your work and your interaction with your supervisor effectively.

Study within our framework and you'll have your best chance of succeeding.

Because the key research requirement is to make a contribution to knowledge, every research student will be working at the leading edge of their discipline. One corollary of this is that every dissertation will be different. Your project will allow you to show how your domain expertise in a focussed topic has developed through your studies and your dissertation will reflect that. That's your goal!

For us in defining our framework, it raises a problem -- we cannot possibly know the details of your final dissertation. We can't know even simple things, like how many pages it will have exactly. We can't know what your arguments will be or what your conclusions or future work will contain. Neither can we know which literature you will consult, or what the seminal paper in it is.

That would be a problem if our only focus was the final form your dissertation will take. But it isn't!

What we teach in this book is the process of arriving at your final dissertation. That's why we structure our framework into stages -- each stage building on the previous, each moving you further down the line to your final dissertation. And because we have seen literally hundreds of Masters students follow the framework, we know it can work and work well. 

If this has convinced you that we're onto something, welcome aboard!

Your first step with us will be to learn about the research process that you will follow.


\section{The research process and its key activities}\label{sect:ResearchActivities}

A \gls{research process} is the sequence of activities you undertake when conducting academic research.

Research is messy --- looking into the unknown means there is no road map to follow and you will often have to retrace your steps and try different paths. So, moderate your expectations -- you shouldn't expect a linear, orderly research process!

Instead, a research process is an iterative, incremental, and adaptive process of knowledge discovery --- it iterates through its research activities in small incremental steps, adapting as your knowledge grows.

The exact path you will follow in your research project will be unique to you. However, you will go through some widely recognised activities, which are common to all research processes. We summarise them here and return to each of them in much greater detail in the remainder of the book.

\subsection{Identifying the research problem}\label{ssect:IdentifyingTheResearch}
A key concept in academic research is that of \gls{research problem}. A research problem captures the knowledge gap to be addressed (\emph{the need}) by the research within a particular field of study (\emph{the context}). A well-defined research problem is the foundation of any research project as it clarifies the purpose of the research and its intended outcomes.

In this book, you will learn a practical approach to identifying and articulating your chosen research problem.

\subsection{Reviewing the literature}\label{ssect:ReviewingTheLiterature}
One of the major tasks you will have in compiling your dissertation is to give the reader sufficient background that they can check where you're coming from. To do so, you will be referring to the work of other authors and researchers throughout your project; this is collectively known as the \gls{academic literature}. You may already have practise in doing this from your previous studies, for instance in writing academic essays for your assignments. For your research project, however, the more definitive you can be in referencing the \gls{academic literature} the better. This means that your review of the \gls{academic literature} will be a significant part of your  project.

Your review has two key functions.

Firstly, it will help you contextualise your research in what is already known and where the knowledge gaps are, so that you can focus your research more tightly on a research problem which is relevant and significant to your field of study. This includes evaluating your own findings against those of others in your field at the end of your project.

Secondly, it will help you understand how to conduct your research: by reading your field's literature, you'll learn what is academically acceptable in terms of the methods and techniques which you can apply to address your research problem.

Reviewing the literature is both time consuming and demanding, and will require you to apply many skills which this book will help you develop.

\subsection{Setting aim and objectives}\label{ssect:SettingYourAim}
While your research problem helps you establish the contribution to knowledge of your research, your aim and objectives will help you establish the scope and boundaries of your project by stating the specific way in which your research will address the problem.

Scoping your project appropriately is necessary in all research, but is particularly critical at Masters level with its many constraints on the time you have and the methods you can feasibly apply.

\subsection{Developing the research design}\label{ssect:DevelopingTheResearch}
Your \gls{research design} will summarise, explain and justify how your research is conducted for the benefit of your examiners and other readers\footnote{Although we all want to think differently, it is often the case that the examiners are the only people outside of your supervisory relationship that will read your dissertation. We'll talk later about how you can get your friends and family involved -- which may double or even triple your readership!}. As with your literature review, it will develop during your project: at the start, it will be a collection of your initial ideas; by the end, it will be an account of what you did.

Your research design will depend on many things. There are some obvious ones, including the type of research problem you are trying to address, the intended outcome of your research, the sort of evidence you will need, and the research strategies and methods that are acceptable within your chosen discipline.

There are others that are less obvious: you'll have to tailor your research design to the resources and expertise you can access, and your preferred \gls{research style}\footnote{You may have heard of a learning style. A research style is similar -- you may prefer, for instance, to talk to people about the practical ways they work as opposed to building theories of how they do so.}: as a key participant in your own research, your own personal views and values will also affect your choices.

There are also some esoteric ones: for instance, some philosophical beliefs which are shared by researchers in your field in regard of how knowledge can be generated. Although fascinating, to really understand them well takes decades of study of others' writings, on top of which there will be hours on hours of conversations with other researchers. Clearly, this is not something you will have much time to dwell upon as a research student --- you have a supervisor that can advise you, instead.

Research design is a field of study in its own right, one which has grown out of the many different academic traditions and ways of thinking across academic disciplines and subject areas. It's not easy to digest and is far from stable or complete: every so often a new book on research methods will be offered for review by a publisher. Research design is possibly one of the most challenging aspects of doing academic research and can be puzzling for those just starting\footnote{It can sometimes be puzzling for experienced researchers too!}.

Because of this, you should rely on your supervisor for advice, particularly at the beginning, although your understanding will mature during your project. This book will help you develop your understanding and how to assess which choices are most appropriate for your project.

\subsection{Generating and analysing evidence}\label{ssect:GeneratingAndAnalysing}
This is where you use your research design to generate and analyse the data and evidence\footnote{Strictly speaking, \gls{data} is raw information with no interpretation attached, while \gls{evidence} is information interpreted to support your academic arguments. The two are closely linked, with data forming the basis of evidence, so that they are often used interchangeably. We will return to this topic in~\Cref{sect:evidenceAndData}.} you need for your project. This is possibly the most exciting part of conducting academic research!

You're going to spend a lot of time doing this, and you'll have to circle back\footnote{One PhD student we know got to within 2 months of finishing their PhD when they realised that their research design led them to incorrect data analysis. That was a really critical moment in their studies, but they got it done in the end. Just.} to this activity as you learn more about your field and as your research develops.

\subsection{Interpreting and evaluating findings}\label{ssect:InterpretingAndEvaluating}
This is where you review your findings critically to establish the extent your research has met its stated aim and objectives. Because it relies on the way you interpret your evidence, it is closely related to and typically influences the way evidence is generated and analysed, so that you will iterate between these two research activities for a large proportion of your project.

\subsection{Writing up}\label{ssect:WritingUp}
Throughout your project you will need to provide a written account in a critical\footnote{Developing your critical voice is, er, critical to being a successful researcher. On a good research project, an effective critical voice will be highly valued by your examiners.} and rigorous manner on all aspects of your research, including both new insights and the process you followed to arrive at them. Therefore, integral to your writing is a critical appraisal of the strengths and limitations of the research you have conducted, the overall conclusions you can draw from it and what the impact on future research might be.

While you may only be required you to submit your final dissertation for assessment, it is essential that you write incremental accounts of your work throughout your project. This will allow you to improve your academic writing skills, share what you have done with your supervisor for feedback and formative assessment, and develop your dissertation incrementally.

\subsection{Reflection and reflexivity}\label{ssect:ReflectionAndReflexivity}
\Gls{reflection} is about thinking back on what we have done or experienced. It is what makes your learning more effective, relevant and useful to your own practice. Reflection is important in any kind of learning, but particularly in the experiential learning\footnote{\Gls{experiential learning} is `learning by doing,' using reflection to think critically about what you have experienced and to relate it to knowledge and how it can apply to new situations.} of conducting research.

On the other hand, \gls{reflexivity} is about examining your own views, assumptions and beliefs, thinking critically about how they influence your research. Through reflexivity you can bring more objectivity into your research choices.

Our framework therefore emphasises both reflection and reflexivity as essential activities within the research process, to develop and consolidate the knowledge and skills you need to be a competent and confident researchers, and to guard against \gls{bias}.

\subsection{Planning work}\label{ssect:PlanningWork}
Some researchers will have worked much of their lives on one problem. You don't have that luxury -- or burden, depending on your perspective!

Time-bound research, like that you are just about to start, must be planned carefully as early mistakes and missteps can be very difficult to correct later.

For this reason, we recommend that you build a \gls{work plan} and we show you how. At its core, a work plan summarises all activities and work to be conducted to complete the research, how to organise and execute them in the time span of the project and which milestones must be met.

Standard project management techniques apply, but with the caveat that in academic research you are usually planning for the unknown, so not everything can be figured out upfront and some level of adaptation will be needed. Hence, understanding what might change, related risk and how to manage it is an essential part of  planning your research project.

\subsection{Identifying and managing risk}\label{ssect:ManagingRisk}
Risk captures the likelihood of something going wrong combined with the impact that will have on your project, on time, resources, and outcome. In theory, both positive and negative impacts count, but usually the focus is on what can affect your project in a bad way. It is essential that you conduct a risk assessment at the start of your project and identify ways to manage it, then monitor what happens and make adjustments as you go along. This may well be having to change your work plan, so that work planning and risk management go hand in hand.

\subsection{How the activities relate to each other}\label{ssect:HowTheActivities}
~\Cref{fig:researchactivities} depicts the relations between activities in the research process: as you can see, these are complex relationships, with much tangling and intertwining. Note also that reflecting, reporting, work planning and risk management apply to both individual activities, and the process overall, which is why the figure separates them from the other activities.

\begin{figure}[htbp]
\centering
\includegraphics[width=\textwidth]{researchActivities.pdf}
\caption{Research activities and their relations}
\label{fig:researchactivities}
\end{figure}

The main relations between activities are as follows (with reference to the numbering in the figure):

\begin{enumerate}
\item the research problem helps you identify relevant literature to review
\item the literature you review helps you identify current knowledge and existing gaps and frame the research problem
\item the research problem informs what the aim and objectives should be
\item the research problem identifies the phenomena under study, which inform your choice of research design
\item the literature also helps you establish which research design to apply and how
\item aim and objectives inform the research design to discharge them
\item the research designs tells you how to generate and analyse evidence
\item the analysis of evidence feeds your interpretation and evaluation
\item your interpretation and evaluation may require further evidence generation and analysis
\item your interpretation and evaluation assess the extent aim and objectives are discharged
\item aim and objectives gives you criteria for your evaluation and interpretation
\item activities feed your project work plan
\item your work plan informs what you have to do and when
\item activities give rise to risk to be managed
\item managing risk will constrain the way you carry out activities
\item activities and their outputs must be documented in your reports
\item your reflection and reflexivity in relation to all you do will trigger changes and adjustments to all other activities.
\end{enumerate}

Your main takeaway from this figure is the interconnected nature of everything you do during a research project, and how the need to revisit some activities, perhaps as the result of reflection and reflexivity, will trigger adjustments to others which are closely related, so that there is no simple linear path  that all projects will follow. Also, the figure does not assume there is a set starting point, although, as you will see in the next chapter, identifying the research problem from an initial topic is the way we recommend you get your project started. Equally, research can go on forever as reflecting on what you know may trigger new avenues of enquiry to follow. Of course, in the context of a Masters project you only have a set amount of time to complete your work, so that at some point you will need to decide that you have done enough and finalise your project dissertation. Your supervisor will be able to help you assess when you have reached that point.


\section{Framework outline}\label{sect:frameworkOutline}

This book recommends you organise your research project into five major stages. We've developed this 5-stage framework from our experience of working with hundreds of successful\footnote{And a very few, unsuccessful ones{\ldots}} Masters research students, our knowledge of the iterative, incremental and adaptive nature of the research process, and our awareness of the many risks that must be managed to be successful.

Given this, the stages have the following characteristics:

\begin{itemize}
\item each stage contains many interrelated research activities\footnote{Don't worry, we're going to go into exhaustive detail about this the remainder of the book!}, so that you will often revisit earlier activities as you learn more about research and move forward in your project. From one stage to the next, however, the balance between those activities will change reflecting your increasing expertise in them and the progress you have made

\item later stages build incrementally and adaptively on work from previous stages

\item each stage includes critical reflection on what you have achieved and learned, and reflexivity to examine motivations behind research choices you have made. Together they inform the work ahead

\item each stage includes some writing so that your dissertation builds with your expertise

\item each stage includes re-assessing risk and adapting your work plan.

\end{itemize}

The five stages are designed to balance the activities you need to carry out in your project. Each stage is allocated a recommended proportion of the overall project time, and each activity a proportion of each stage time, as summarised in~\Cref{fig:slugDiagram} and~\Cref{t:allStagesActivities}.

\begin{figure}[htbp]
\centering
\includegraphics[width=\textwidth]{Figures/slugDiagram.pdf}
\caption{Project stages and research activities: change of focus over the project duration}
\label{fig:slugDiagram}
\end{figure}


\begin{question}[subtitle={Activity: Investigating stages and activities}] Consider~\Cref{fig:slugDiagram} and~\Cref{t:allStagesActivities} carefully, taking notice of which activities are more prominent in each stage. Write down your observations on how activities are distributed over the length of a project, and how project time is distributed among them. 
\begin{solution}
The figure highlights how the main focus of the project changes over time, but also that many activity are repeated over several stages, reflecting the iterative, adaptive and incremental nature of the research process.

Some activities are prominent in the early stages, like identifying the research problem and reviewing the literature, and minimal later on. In fact, in Stage 1, identifying the research problem and reviewing the literature take up 50\% of the time available (see Table~\Cref{t:allStagesActivities}). This shouldn't surprise you as the early stages are mainly concerned with establishing the scope of, and justifying, your research.

Other activities start or become prominent in later stages, particularly generating and analysing evidence, and interpreting and evaluating findings, which mainly feature in the last two stages. The framework recommends you think deeply about your research design before you start collecting data, to ensure your best chance of generating evidence to address your aim and objectives.

Some activities feature throughout the project, such as reflection and reflexivity and  writing up. For the latter, the framework assumes that 20\% of your time is devoted to writing, which is a quite substantial effort. Similarly, Work planning and risk management start early on and are sustained throughout the project.  
\end{solution}\end{question}
%%Hack to correct tcbox behaviour
\color{black}

You should, of course, take these framework guidelines with a pinch of salt, as the precise balance between the stages and their activities in your own project is influenced by many factors, including the topic you have chosen, any previous research experience you may have, your supervisor's advice, requirements established by your course, etc. 

We will consider more detailed guidance by stage in the follow-up chapters, but again, nothing is written in stone.

\begin{SimpleNColTable}{t:allStagesActivities}{6}{Recommended percentages of project time per research activity and stage}[lccccc]
Research activity&{Stage 1\\ (15\%)}&{Stage 2\\ (15\%)}&{Stage 3\\ (20\%)}&{Stage 4\\ (20\%)}&{Stage 5\\ (30\%)}\\
Identifying the research problem &20\% &10\% &5\% &2\% &0\%\\
Reviewing the literature &30\% &30\% &5\% &2\% &0\%\\

Setting aim and objectives &5\% &5\% &10\% &3\% &0\%\\
Developing the research design &5\% &15\% &40\% &20\% &0\%\\
Generating and analysing evidence &0\% &0\% &0\% &35\% &25\%\\
Interpreting and evaluating findings &0\% &0\% &0\% &0\% &35\%\\
Writing up &20\% &20\% &20\% &20\% &20\%\\
Reflection and reflexivity &10\% &10\% &10\% &10\% &15\%\\
Planning work &5\% &5\% &5\% &5\% &2\%\\
Managing risk &5\% &5\% &5\% &3\% &3\%	
\end{SimpleNColTable}

\section{Planning your project using the framework}\label{sect:yourWorkPlan}


In this section, you will use the 5-stage framework as a tool to help you put together an initial overarching plan for your project work, which organises the main research activities along your project timeline. In the following chapters we will break down those activities further, so that at each stage you will revise your plan, adding detail and making adjustments based on your progress. 


Planning and monitoring your research work is not much different from planning and monitoring other kinds of project, so that the following generic practices apply:
\begin{itemize}
	\item establishing the timeline for your project from start to finish
	\item identifying milestones and deliverables. A \gls{milestone} is a point on your project timeline where you can assess your progress against your ultimate aim and objectives, while a \gls{deliverable} is a concrete artefact you must have produced at a specific point, be that a first draft of your literature review, a full characterisation of your research problem, etc.
	\item identifying the work you will need to complete for each milestone and deliverable
	\item breaking down that work into discrete tasks, also identifying how they may depend on each other
	\item scheduling those tasks in the time available, taking any dependency into consideration, and
	\item monitoring your progress against your plan on a regular basis, makeinh adjustments as needed.
\end{itemize}

For your initial project plan you will focus on your overall project timeline, and how to arrange stages and main activities along it. You should also treat reaching the end of each stage as a milestone. In the following chapters, we will detail stage by stage the work you will be expected to complete under each activity, so that you will be able to identify tasks, milestones and deliverables more precisely and refine your plan accordingly. 

For your project plan you can use any of the standard planning charts and tools available; in fact, you may be already be familiar with some of them. Gantt charts are among the most commonly used. Briefly, a \gls{gantt chart} is a scheduling chart that you can use to plan and organise your project work. Using a Gantt chart will give you an overview of how your work will be broken down and organised over time. As such, it is also be useful to communicate your project plan to third parties. As a generic project management tool, a Gantt chart can be quite sketchy or very detailed: you should aim for something quite light, but still including all main tasks and deliverables of your project.

\begin{question}[subtitle={Activity: Investigating Gantt charts}] If you are unfamiliar with Gantt charts, conduct a web search on introductory materials and examples. Select and review few of the resources you have found. Write down the basic characteristics of a Gantt chart.
\begin{guidance}
In your web search you'll likely find several links to digital tools supporting Gantt charts, tutorials on how to construct one, or even ready-made templates to fill in. You should focus on materials which can help you construct your own project Gantt chart.
\end{guidance}
\begin{solution}
The basic characteristics of a Gantt chart are:
\begin{itemize}
	\item it is used to visualise a project timeline and tasks, and to monitor progress
	\item tasks are listed and also represented visually as bars distributed along the project timeline, indicating when each tasks starts and ends; the length of each bar corresponds to the time allocated to the task
	\item tasks that depend on previous tasks are organised sequentially along the timeline; independent tasks can overlap or occur in parallel
	\item milestones and deliverables are included at specific points on the timeline using special symbols, often diamonds.
\end{itemize}
\end{solution}\end{question}
%%Hack to correct tcbox behaviour
\color{black}

As an important caveat, you should not expect your plan to be cast in stone! There are too many unknowns in a research project for you to be able to predict upfront exactly what is going to happen. Therefore, you shouldn't spend too much time trying to plan every single thing, and you should review your plan often to monitor progress and make adjustments as needed. In summary, your project plan should keep you on track, but should not be a straight-jacket.


The framework can help you establish your project timeline based on the overall project time available to you on your course. Let's look at an example first.

\begin{example}{Example of project timeline}
Let's assume you have 600 hours of project work distributed evenly over 50 weeks of study.
You can use the framework to establish an approximate duration for each of your project stages. Specifically: Stage 1 and Stage 2 will take 15\% of the overall project time each, that is 7.5 weeks or 90 hours; Stage 3 and Stage 4 will take 20\% of the overall project time each, that is 10 weeks or 120 hours; and Stage 5 will take 30\% of the overall project time, that is 15 weeks or 180 hours. The timeline will then span from the project start date to the end of the 50th week after that. Each stage start and end dates can then be arranged along this timeline with each stage following from the previous one: Stage 1 will therefore start at project start while Stage 5 will end at the end of the project.
\end{example}


\begin{question}[subtitle={Activity: Establishing your project timeline}] 
Based on the study time set out by your course of studies, establish the timeline for your project, identifying when each stage starts and ends.
\begin{solution}
Your timeline will be specific to your project. You should refer to the example above to get you started.
\end{solution}
\end{question}
%%Hack to correct tcbox behaviour
\color{black}

Now that you have established your timeline and the start, end and duration of each stage, you can assign time to research activities within.

\begin{example}{Example of breaking down the time for research activities}
Following from our previous example, we can use~\Cref{t:allStagesActivities} to calculate the project time per activity across all stages. The easiest way to do this is to enter the table in a spreadsheet. We have done just that and come up with~\Cref{t:allStagesActivitiesExample}. The calculations are in hours and based on 600 hours for the overall project. To calculate the time of each activity within a stage, we first calculated the time for the stage overall, say 90 hours for Stage 1, then the time of each activity by applying the recommended percentage, say 20\% of 90 for identifying the research problem in Stage 1, corresponding to 18 hours.
\end{example}

\begin{SimpleNColTable}{t:allStagesActivitiesExample}{11}{Example of using the framework to establish effort per research activity and stage}[lcccccccccc][row{Z}={bg=\frameColor,fg=white,font=\bfseries}]
Research activity&{Stage 1\\ (15\%)}&Hours &{Stage 2\\ (15\%)}&Hours &{Stage 3\\ (20\%)}&Hours &{Stage 4\\ (20\%)}&Hours &{Stage 5\\ (30\%)}&Hours \\
Identifying the research problem &20\%&18 &10\%&9 &5\%&6 &2\%&2 &0\%&0\\
Reviewing the literature &30\%&27 &30\%&27 &10\%&12 &2\%&2 &0&0\%\\
Setting aim and objectives &5\%&5 &5\%&5 &5\%&6 &3\%&4 &0\%&0\\
Developing the research design &5\%&5 &15\%&14 &40\%&48 &20\%&34 &0\%&0\\
Generating and analysing evidence &0\%&0 &0\%&0 &0\%&0 &35\%&42 &25\%&45\\
Interpreting and evaluating findings &0\%&0 &0\%&0 &0\%&0 &0\%&0 &35\%&63\\
Writing up &20\%&18 &20\%&18 &20\%&24 &20\%&24 &20\%&36\\
Reflection and reflexivity &10\%&9 &10\%&9 &10\%&12 &10\%&12 &15\%&27\\
Planning work &5\%&5 &5\%&5 &5\%&6 &5\%&6 &2\%&4\\
Managing risk &5\%&5 &5\%&5 &5\%&6 &3\%&4 &3\%&5\\
Total hours & &90 & &90 & &120 & &120 & &180 \\	
\end{SimpleNColTable}


\begin{question}[subtitle={Activity: Calculating activity times by stage}] 
Based on the study time set out by your course, calculate the time to allocate to each research activity by stage using the percentages in~\Cref{t:allStagesActivities}.
\begin{guidance}
In your calculations, ensure you first establish the time per stage, then, based on that, the time for each of the relevant activities in the stage.
\end{guidance}
\end{question}
%%Hack to correct tcbox behaviour
\color{black}

At this point, you have all the elements to define your initial project Gantt chart: this will be an initial outline that you will review, adjust and augment at the start of each project stage.

\begin{question}[subtitle={Activity: Constructing your initial Gantt chart}] Based on the outcome of your previous activities, create an initial Gantt chart for your project.
\begin{guidance}
In allocating time to activities on your chart, you will need to establish possible dependencies between activities and the extent some may overlap within a stage. If in doubt, refer back to the activity descriptions in~\Cref{sect:ResearchActivities} to remind yourself of how research activities affect each other. There may be cases in which you should iterate between two or more activities within a stage and your chart should indicate when that's the case. 
\end{guidance}\end{question}
%%Hack to correct tcbox behaviour
\color{black}

If you are already familiar with a different project scheduling approach for projects, then feel free to use that instead of a Gantt chart. What matters is that it allows you to organise your project work effectively and communicate it easily to others. In particular, you should share and discuss your project plan with your supervisor.

\section{Key practices for managing your time efficiently}\label{sect:timeManagementPractice}
Your plan is more likely to work if you make an effective use of your time. Here are some key practices you should keep in mind:

\paragraph{Finding time for your research} You need to make a realistic assessment of how much time you will have for your project, considering other commitments you may have, professional or personal: this is important for all students, but crucial for part-time study. Your course will expect you to spend an average number of hours per week on your project work throughout its duration, so you need to ensure you can dedicate that time to your project on a regular basis. But it is not just about quantity: you should choose your most productive time to dedicate to your research--- it may be that you are a morning person, or you can focus better at night. In either case, it is important you come to your work with a fresh mind, so that you can concentrate on the tasks at hand.

\paragraph{Ensuring continuity of effort} Research requires continuity of effort, as you need the time to develop your critical thinking and other research skills, and that's a continuous process. If you fall behind, you may find it difficult to catch-up. In your project plan, you must therefore ensure that your research time is arranged evenly and regularly over your project duration and that any study break you may take is short, not to affect your continuous progress.

\paragraph{Making your spare time productive} There will be much to read throughout your project, so you should always keep something to read with you, which you can look at while waiting for something, say a bus, or in a queue or travelling on public transport. You may be surprised of how much reading you can do while waiting!

\paragraph{Avoiding postponing and procrastinating} There will be tasks you may find harder than others, so it would be natural to put them off or engage in displacement activities. You should avoid that and focus on what must be done to meet your project milestones.

\paragraph{Scheduling extra time} Inevitably things don't always go to plan or activities may take longer than estimated, so you should always factor in extra time for the things that may go wrong. If you find you don't need it, you can always allocate the extra time to other tasks or take a longer break, but at least you won't fall behind in subsequent tasks.

\paragraph{Being adaptive} Research is about looking into the unknown, so you can't expect to plan everything upfront and in great detail. Your planning should keep you on track to reach your major milestones, but should also be flexible. Don't spent too much time trying to plan every single thing to do: keep your planning light and ensure you return to your plan often to make adjustments as your project unfolds.

\paragraph{Investing in the right tools for the job} While you may find it a nuisance to have to spend time learning new tools, such as a BMT or some advanced word processing features, doing so will save you time in the long run, so make sure you invest in setting up your new systems and learning new skills earlier on in your project.

\begin{question}[subtitle={Activity: Reflecting on your study practices}] Assess your own study habits and practices in light of the above advice. Write down things you do well and things you could or should change for your research project.

\begin{guidance}
You should take a balanced stance and consider both your current strengths and weaknesses: build on the former and put some effort in addressing the latter.
\end{guidance}\end{question}
%%Hack to correct tcbox behaviour
\color{black}

\section{Critical success factors}\label{sect:successFactors}

You should keep in mind the following critical success factors during your research project:

\paragraph{Making good use of the 5-stage framework}  The framework in this book is the result of decades of practice, helping hundreds of research students succeed in their first academic project. However, it is not a straight-jacket, and you should adapt it to your own needs. The framework was developed with the novice academic researcher in mind, but it may be that you already have some research experience, in which case some of the the timing suggested by the framework may be too generous. Equally, your own course may require you to submit some interim report as formative assessment, in which case the stage lengths may need altering to match your course requirements. Nevertheless, the framework provides a scaffolding to help you take control when planning and conducting your project.

\paragraph{Making good use of this book and its activities}  This book is designed to accompany you in your first journey into academic research, so you should follow its stage by stage advice to guide your project work. The book is also designed to be very practical, so that there are plenty of activities for you to do. The activities are there to help you make steady progress with essential work for your project and to help you develop your research skills. From them, your dissertation will emerge, hence we strongly advise you complete them systematically.

\paragraph{Your working relation with your supervisor}  Your supervisor is your strongest study ally, a research expert who can guide and advise you on all aspects of your project, with whom you can discuss your research ideas, and who can assess that you are making sufficient progress at each point in your journey. It is essential that you develop a good working relationship with your supervisor, meet regularly and have an open and honest dialogue throughout your project.

\paragraph{Your self-drive and commitment} The course you are studying may provide you with some structure to help you progress your project, including, for instance formal assessment points. However, you will be in the driving seat most of the time. You must choose your topic and research problem, how to investigate it, and how to organise your research time in detail; it is also up to you to understand the academic standards required and to assess which research skills you must develop to meet them. Above all, it is up to you to commit to a sustained effort for the duration of your project, which may well be several months.

\paragraph{Continuous effort} Successful research requires continuity, so that you will need to set aside a sufficient and regular time for your research project every week, ensuring you keep making progress as you go along. Long breaks are incompatible with conducting research: while in your previous studies you may have been able to stop and start, and possibly cram lots of work around assignment deadlines, conducting research requires lots of time for reflection and for ideas to develop and mature, something you can't compress close to a deadline. Conducting research requires endurance, perseverance and continuous effort: it is a marathon, not a sprint!

\paragraph{Reading, thinking and writing}  There is a crucial interaction between reading, thinking, and writing in research: reading informs your thinking; your thinking is what you try to express in your writing; your writing helps you make sense of what you have read, and hence of what you think, and informs more reading, thinking and writing. Because of this, while you will spend a lot of time reading and thinking, it is also essential that you write as you go along. The more regularly you write, the easier it will be for you to develop academic writing skills and the more critical your thinking will be.

\paragraph{Your dissertation}  Your research will be assessed on the basis of your written dissertation. Your dissertation must therefore demonstrate that you have mastered a wide range of research skills and can communicate your academic research effectively in writing. In fact, this is usually more important than any feature your proposed solution to your research problem might have. Particularly at Masters level, that you `solve' your research problem may not even be necessary! What matters most is your scholarly and critical attitude to each element of your research and your grasp of what academic research entails as demonstrated by your written dissertation.

\paragraph{Making good use of your study support}  Last but not least, it is important that you make the most of the support which is provided to you by your supervisor, and any other adviser on your course. There will be times when you will find working on your project very challenging and may lose confidence in your ability to complete it. Those are the times when it is going to be particularly important for you to reach out and ask for extra help, even if your first instinct may be to hide. All researchers experience such feelings at one time or another, so do not be discouraged: talk to an adviser and work through the difficulties you are having to overcome them.

\section*{Framework Takeaways}\label{sect:FrameworkTakeaways}

\begin{itemize}
\item the research process is the process you undertake when conducting academic research. Its many activities are deeply connected, so that the process is iterative and incremental, and can be messy. Be flexible and ready to adapt your plans as you go along
\item the 5-stage framework in this book is a practical framework designed to help you plan and execute your project, while developing essential research knowledge and skills
\item you should use the guidance provided by the framework to organise your project work, define your project work plan, ensure you make an effective use of your research time, monitor your progress, and develop work habits that will help you succeed 
\item you should carry out the practical activities included in this book. They are not an overhead! Instead, they are designed to help you develop your research skills and perform essential research and project tasks.
\end{itemize}


\color{black}
\endinput