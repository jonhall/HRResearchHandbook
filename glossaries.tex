\RequirePackage{glossaries}
\RequirePackage{glossaries-extra}
\makenoidxglossaries

%%temp \newglossaryentry
\newcommand{\gloss}[0]{\gls}

\newglossaryentry{abstraction}{ name={abstraction}, description = {deciding whether specific concepts may be instances of more general ones}}
\newglossaryentry{academic literature}{ name={academic literature}, description = {the work of other authors and researchers}}
\newglossaryentry{accidental plagiarism}{ name={accidental plagiarism}, description = {unintended plagiarism that is the result of poor organisation or sloppy review practices}}
\newglossaryentry{active reader}{ name={active reader}, description = {one who engages with the materials and is able to assimilate the important points in an effective manner}}
\newglossaryentry{algorithmic bias}{ name={algorithmic bias}, description = {when an algorithm produces outcomes which systematically and repeatably disadvantage one group of individuals over others}}
\newglossaryentry{alternative}{ name={alternative}, description = {\color{red}???}}
\newglossaryentry{analysis methods}{ name={analysis methods}, description = {methods used to analyse data}}
\newglossaryentry{anti-positivism}{ name={anti-positivism}, description = {that different people experience and understand reality in different ways}}
\newglossaryentry{association}{ name={association}, description = {when one variable can be used to provide information about another}}
\newglossaryentry{beneficiary}{ name={beneficiary}, description = {\color{red}???}}
\newglossaryentry{bell curve}{ name={bell curve}, description = {\color{red}definition needed}}
\newglossaryentry{bias weaknesses}{ name={bias weaknesses}, description = {when the claim you have made to new knowledge has been affected by your implicit or explicit biases, or bias affecting human participants in your study, making the new knowledge invalid}}
\newglossaryentry{bias}{ name={bias}, description = {a tendency to support or oppose particular ideas or things based on personal preferences rather than objective evidence}} 
\newglossaryentry{bibliographic management tool}{ name={bibliographic management tool (bmt)}, description = {a tool for organising and using academic and other references}}
\newglossaryentry{bin size}{ name={bin size}, description = {of a histogram, how measured values are arranged}}
\newglossaryentry{boolean operator}{ name={boolean operator}, description = {used in mathematics and logic to build expressions which can be either true or false}}
\newglossaryentry{bcw model}{ name={Booth, Colomb, and Williams (BCW) model}, description = {\color{red}???}}
\newglossaryentry{box and whiskers plot}{ name={box and whiskers plot}, description = {\color{red}???}}
\newglossaryentry{boxplot}{ name={boxplot}, description = {\color{red}a chart in which a \enquote{box} is drawn around the median of the data, with \enquote{whiskers} on each side of the box}}
\newglossaryentry{bracketing}{ name={bracketing}, description = {\color{red}???}}
\newglossaryentry{capstone project}{ name={capstone}, description = {completes a degree or other programme of study}}
\newglossaryentry{categorical data}{ name={categorical data}, description = {data that denote categories}}
\newglossaryentry{citation searching}{ name={citation searching}, description = {the process of looking backwards and forwards to track citations, starting from an article of interest; backward  searching looks through the articles cited by a paper; forward  searching looks through the articles which cite that paper}}
\newglossaryentry{code categorisation}{ name={code categorisation}, description = {the process of reviewing what you have coded and organise it into categories}}
\newglossaryentry{code}{ name={code}, description = {a label which describes an extract from a qualitative data set}}
\newglossaryentry{codebook}{ name={codebook}, description = {a list of all the codes and their intended meaning}}
\newglossaryentry{coding}{ name={coding}, description = {the process of creating and assigning codes to categorise qualitative data}}
\newglossaryentry{cognitive bias}{ name={cognitive bias}, description = {that which prevents someone from processing information objectively due to limited processing capabilities of our mind or due to emotional responses or social norms and conditioning}}
\newglossaryentry{concept map}{ name={concept map}, description = {a diagram which helps visualises concepts and their relationship}}
\newglossaryentry{concept matrix}{ name={concept matrix}, description = {a table with one concept per row and one article per column, and a tick at their intersection to indicate whether the concept appears in the article}}
\newglossaryentry{confirmation bias}{ name={confirmation bias}, description = {the tendency to favour the selection, analysis and interpretation of data which support the researcher's prior beliefs}} 
\newglossaryentry{confounding factors}{ name={confounding factors}, description = {on a cause-effect relationship, factors not measured in the experiment that may influence cause and effect}}
\newglossaryentry{constructivism}{ name={constructivism}, description = {that reality is a construct of our minds and so is absolutely subjective}}
\newglossaryentry{content analysis}{ name={content analysis}, description = {to identify patterns used for communication}}
\newglossaryentry{contingency table}{ name={contingency table}, description = {an extension of frequency tables to tabulate the value frequencies of two categorical variables}}
\newglossaryentry{correlation}{ name={correlation}, description = {a particular type of association such that the two associated variables always change together}}
\newglossaryentry{critical theory}{ name={critical theory}, description = {that social science can never be 100\% objective or value-free}}
\newglossaryentry{cross-tabulation}{ name={cross-tabulation}, description = {see contingency table}}
\newglossaryentry{data source}{ name={data source}, description = {the location from which your data originates}}
\newglossaryentry{data}{ name={data}, description = {the raw information with no interpretation attached}}
\newglossaryentry{decision}{ name={decision}, description = {determining the status of the null hypothesis based on the p-value in relation to the $\alpha$ value}}
\newglossaryentry{deductive coding}{ name={deductive coding}, description = {in which codes are decided upfront, before looking at the data}}
\newglossaryentry{deductive reasoning}{ name={deductive reasoning}, description = {reasoning that starts from assumptions and arrives at the proposition as a conclusion}}
\newglossaryentry{deliverable}{ name={deliverable}, description = {a concrete artefact you must have produced at a specific point}}
\newglossaryentry{descriptive}{ name={descriptive}, description = {coding which encapsulate a general idea}}
\newglossaryentry{discourse analysis}{ name={discourse analysis}, description = {deconstructing language in conversations in a real-world context}}
\newglossaryentry{diversity}{ name={diversity}, description = {the differences and similarities between individuals}}
\newglossaryentry{empirical data}{ name={empirical data}, description = {data that comes from observation or experience}}
\newglossaryentry{epistemology}{ name={epistemology}, description = {\color{red}???}}
\newglossaryentry{equality}{ name={equality}, description = {treating everybody in the same way}}
\newglossaryentry{equity}{ name={equity}, description = {relates to fairness and justice, in the sense of removing barriers or bias}}
\newglossaryentry{evidence}{ name={evidence}, description = {information interpreted to support your academic arguments}}
\newglossaryentry{experiential learning}{ name={experiential learning}, description = {learning by doing, using reflection to think critically about what you have experienced and relate it to knowledge and how it can apply to new situations}}
\newglossaryentry{falsification}{ name={falsification}, description = {that a  posited theory must make predictions which are testable}}
\newglossaryentry{frequency table}{ name={frequency table}, description = {a table used to summarise the frequency (or count) of values taken by a categorical variables in a data set}}
\newglossaryentry{gantt chart}{ name={Gantt chart}, description = {a scheduling chart that you can use to plan and organise your project work; gives an overview of how your work will be broken down and organised over time; is useful to communicate your project plan to third parties}}
\newglossaryentry{generation methods}{ name={generation methods}, description = {methods used to generate data}}
\newglossaryentry{grey literature}{ name={grey literature}, description = {the collection of information produced by organisations other than publishers, such as as academia, government bodies, or non-publishing businesses and industries; includes pre-publication and non-peer-reviewed articles, theses and dissertations, research and committee reports, government reports, conference papers, accounts of ongoing research, etc.}}
\newglossaryentry{groupthink}{ name={groupthink}, description = {a tendency to conform to majority option to maintain unanimity and avoid confrontation}}
\newglossaryentry{hallucination}{ name={hallucination}, description = {\color{red}???}}
\newglossaryentry{histogram}{ name={histogram}, description = {a chart in which the data on one axis is plotted against the number of times it occurs}}
\newglossaryentry{hypotheses}{ name={hypotheses}, description = {\color{red}???}}
\newglossaryentry{hypothesis}{ name={hypothesis}, description = {a tentative statement about the relationship between the phenomena to be tested in the experiment}}
\newglossaryentry{in vivo}{ name={in vivo}, description = {\color{red}coding that uses the exact language which occurs in the data}}
\newglossaryentry{inclusion}{ name={inclusion}, description = {ensuring that all individuals feel welcome, valued and confident to be treated fairly and respectfully}}
\newglossaryentry{indigenous}{ name={indigenous}, description = {emphasises the connection between people, their culture, and the spiritual and natural worlds, valuing knowledge which is local to communities, and holistic in connecting all beings with nature and spirituality}}
\newglossaryentry{inductive coding}{ name={inductive coding}, description = {in which codes emerge from the data and are not pre-defined}}
\newglossaryentry{inductive reasoning}{ name={inductive reasoning}, description = {reasoning that draws general conclusions from observations}}
\newglossaryentry{intellectual property}{ name={intellectual property}, description = {the owned property created through intellect}}
\newglossaryentry{interval data}{ name={interval data}, description = {data which can be arranged on a scale, so that we can calculate the distance between any two data points}}
\newglossaryentry{learning outcome}{ name={learning outcome}, description = {defines what a student should be able to do after their study; tested in assessment, exams and, in the case of research, the dissertation or thesis}} 
\newglossaryentry{logical fallacies}{ name={logical fallacies}, description = {errors of reasoning that can undermine your argument}}
\newglossaryentry{longitudinal data}{ name={longitudinal data}, description = {data collected over time}}
\newglossaryentry{mean}{ name={mean}, description = {the average value of the data set}}
\newglossaryentry{median}{ name={median}, description = {the midpoint of a data set, sorted by magnitude}}
\newglossaryentry{meta-analysis}{ name={meta-analysis}, description = {statistical techniques to analyse and combine quantitative results from many sources}}
\newglossaryentry{methodology}{ name={methodology}, description = {\color{red}???}} 
\newglossaryentry{milestone}{ name={milestone}, description = {a point on your project timeline where you can assess your progress against your ultimate aim and objectives}}
\newglossaryentry{mode}{ name={mode}, description = {the value in a data set that occurs most frequently}}
\newglossaryentry{model}{ name={model}, description = {a representation of something, be that a system, a structure, a process or a behaviour}}
\newglossaryentry{modelling methods}{ name={modelling methods}, description = {methods used to build models of complex real-world situations, possibly being used to generate further data}}
\newglossaryentry{multi-method research}{ name={multi-method research}, description = {\color{red}the use of many qualitative and/or quantitative methods}}
\newglossaryentry{narrative analysis}{ name={narrative analysis}, description = {focussing on stories made and told by people, to investigate their meaning and how people make sense of reality}}
\newglossaryentry{narrative review}{ name={narrative review}, description = {a narrative synthesis of qualitative results}}
\newglossaryentry{nominal data}{ name={nominal data}, description = {\color{red}data in named categories}}
\newglossaryentry{normal design}{ name={normal design}, description = {re-implementing a solution to a well-known problem through a well-known development process and well-practiced skills}}
\newglossaryentry{normal distribution}{ name={normal distribution}, description = {\color{red}the distribution in which each element has equal probability of occurrence}}
\newglossaryentry{novelty weaknesses}{ name={novelty weaknesses}, description = {when the knowledge you claim to have generated already exists in the academic literature}}
\newglossaryentry{null}{ name={null}, description = {\color{red}???}}
\newglossaryentry{numerical approximation}{ name={numerical approximation}, description = {a numerical value \enquote{close} to the actual value}}
\newglossaryentry{numerical data}{ name={numerical data}, description = {numbers}}
\newglossaryentry{observation bias}{ name={observation bias}, description = {the tendency of participants being observed during a study to change their behaviour}}
\newglossaryentry{ontology}{ name={ontology}, description = {\color{red}???}}
\newglossaryentry{ordinal data}{ name={ordinal data}, description = {non numerical data that can be arranged in an order}} 
\newglossaryentry{p-value}{ name={p-value}, description = {the probability calculated for your test by your statistical package, and which is used to decide the outcome of the test}}
\newglossaryentry{personal data}{ name={personal data}, description = {any information which may identify a living person}}
\newglossaryentry{pivot table}{ name={pivot table}, description = {a table used to summarise, sort, filter, re-organise or group data}}
\newglossaryentry{plagiarism}{ name={plagiarism}, description = {passing off the work of others as if it were your own}}
\newglossaryentry{plan-act-reflect}{ name={plan-act-reflect}, description = {a cognitive cycle which generates both knowledge and action}}
\newglossaryentry{point data}{ name={point data}, description = {data collected at a point in time}}
\newglossaryentry{population}{ name={population}, description = {the entire group you are interested in studying}}
\newglossaryentry{positivism}{ name={positivism}, description = {a philosophy of science that posits a single, objective reality that can be accurately known, described and explained}}
\newglossaryentry{post-positivism}{ name={post-positivism}, description = {\color{red}???}}
\newglossaryentry{primary data}{ name={primary data}, description = {data newly generated or collected during research}}
\newglossaryentry{primary evidence}{ name={primary evidence}, description = {evidence newly generated or collected during research}}
\newglossaryentry{probability sampling}{ name={probability sampling}, description = {see random sampling}}
\newglossaryentry{proof by contradiction}{ name={proof by contradiction}, description = {reasoning that arrives at a contradiction by assuming a proposition to be false, hence demonstrating that it must be true}}
\newglossaryentry{proposition}{ name={proposition}, description = {a statement concerning the population you are studying}}
\newglossaryentry{qualitative data}{ name={qualitative data}, description = {data that are descriptive and defy ordering}}
\newglossaryentry{quantitative data}{ name={quantitative data}, description = {data that can be quantified or measured}} 
\newglossaryentry{random sampling}{ name={random sampling}, description = {an unbiased way of choosing the subset members from the population to inform  sample collection}}
\newglossaryentry{randomisation}{ name={randomisation}, description = {sampling so that all population data or individuals have equal chance to be selected for a study}}
\newglossaryentry{range}{ name={range}, description = {the difference between smallest (minimum) and largest (maximum) numerical values}}
\newglossaryentry{ratio data}{ name={ratio data}, description = {data with an absolute zero considered as a point of origin}}
\newglossaryentry{raw data}{ name={raw data}, description = {data generated and analysed as part of research, upon which evidence and a contribution to knowledge are based}}
\newglossaryentry{recall bias}{ name={recall bias}, description = {the tendency of people to recall certain types of events more vividly than others, and can affect the outcomes of research which relies on participants' memories}} 
\newglossaryentry{reflection}{ name={reflection}, description = {thinking back of what we have done or experienced}}
\newglossaryentry{reflexivity}{ name={reflexivity}, description = {examining your own views, assumptions and beliefs, thinking critically about how they influence your research; through reflexivity you can bring more objectivity into your research choices}}
\newglossaryentry{reliability weaknesses}{ name={reliability weaknesses}, description = {when the steps you have taken to establish your claim of new knowledge are not dependable, cannot be replicated under the same conditions or are not sufficiently repeatable in other contexts, or the descriptions and interpretations you have provided are incoherent or inadequate}}
\newglossaryentry{research assets}{ name={research assets}, description = {information containers needed for your research that will contribute to your dissertation; may include data, images, tables, your own notes, comments from your supervisor, etc}} 
\newglossaryentry{research beneficiary}{ name={research beneficiary}, description = {one who benefits from the research}} 
\newglossaryentry{research design}{ name={research design}, description = {how your research is conducted}} 
\newglossaryentry{research methods}{ name={research methods}, description = {standard ways to collect, analyse, synthesise, present and interpret data to generate evidence and derive findings}}
\newglossaryentry{research paradigms}{ name={research paradigms}, description = {a philosophical way of thinking, a set of shared beliefs which shape a worldview}}
\newglossaryentry{research problem}{ name={research problem}, description = {captures the knowledge gap to be addressed (\emph{the need}) by the researcher within a particular field of study (\emph{the context})}}
\newglossaryentry{research process}{ name={research process}, description={a sequence of activities you undertake when conducting academic research}}
\newglossaryentry{research strategy}{ name={research strategy}, description = {a systematisation of a set of research methods, which can be applied together in order to address research problems of a particular kind}}
\newglossaryentry{research style}{ name={research style}, description = {your preferred style for how you conduct your research}}
\newglossaryentry{review article}{ name={review article}, description = {\color{red}also }}
\newglossaryentry{sample bias}{ name={sample bias}, description = {when some elements of the population are more likely to be selected than others}} 
\newglossaryentry{sample(s)}{ name={sample(s)}, description = {one or more (representative) samples of the population of interest  on which to perform a test}}
\newglossaryentry{sample}{ name={sample}, description = {the portion or subset of that group you have access to in your research}}
\newglossaryentry{sampling frame}{ name={sampling frame}, description = {a sub-set of the collection of the population of interest from which your sample is taken}}
\newglossaryentry{sampling}{ name={sampling}, description = {the process of selecting a subset or a population}}
\newglossaryentry{saturation}{ name={saturation}, description = {when collecting more data would not bring extra relevant information}}
\newglossaryentry{search term}{ name={search term}, description = {a word or a combination of words that you can key into a search engine}}
\newglossaryentry{secondary data}{ name={secondary data}, description = {data available from previous research, and re-used during new research}}
\newglossaryentry{secondary evidence}{ name={secondary evidence}, description = {evidence available from previous research, and re-used during new research}}
\newglossaryentry{selection bias}{ name={selection bias}, description = {when data or participants are selected subjectively, leading to samples which are not representative of the population under study}}
\newglossaryentry{self-plagiarism}{ name={self-plagiarism}, description = {the re-use existing material that you yourself have published, without clear attribution}}
\newglossaryentry{sensitive personal data}{ name={sensitive personal data}, description = {\color{red}data which may reveal...}} 
\newglossaryentry{significance}{ name={significance}, description = {the level of statistical significance for the test}}
\newglossaryentry{skewness}{ name={skewness}, description = {how symmetrically distributed the values in the data set are in relation to the \color{red}centre}}
\newglossaryentry{stakeholder}{ name={stakeholder}, description = {one affected by the research, not necessarily positively}}
\newglossaryentry{standard deviation}{ name={standard deviation}, description = {a measure of the distance of a data set from its mean value}}
\newglossaryentry{statistical significance}{ name={statistical significance}, description = {used to provide evidence (or otherwise) that any pattern or effect observed in your sample is also likely to exist in the population, and is not just the effect of chance}}
\newglossaryentry{summary-comparison matrix}{ name={summary-comparison matrix}, description = {a table which includes a row for each article you have reviewed, and a number of columns corresponding to key aspects of the article you wish to summarise and compare with other articles, such as the research problem or question addressed, contextual facts known at start of the reported study,  key contributions made, research methods applied, etc}}
\newglossaryentry{synthesis}{ name={synthesis}, description = {the act of bringing ideas together into a cogent whole}} 
\newglossaryentry{system}{ name={system}, description = {a set of elements coming together in a complex whole and whose behaviour stems from the interaction of those elements}}
\newglossaryentry{thematic analysis}{ name={thematic analysis}, description = {to identify recurring themes, their definition and relationships}}
\newglossaryentry{theory}{ name={theory}, description = {a system of ideas intended to explain something}}
%\newglossaryentry{triangulation}{ name={triangulation}, description = {to complement your observation with other kinds of data collection and compare the outcomes}}
\newglossaryentry{triangulation}{ name={triangulation}, description = {using multiple data sources and methods, or even multiple researchers, to develop a comprehensive understanding of a phenomena under study and arrive at a particular conclusion about that phenomenon}}
\newglossaryentry{validity weaknesses}{ name={validity weaknesses}, description = {when the claim you have made to new knowledge isn't sufficiently credible, trustworthy, or accurate to be considered knowledge, or can't be generalised or transferred beyond your study}}
\newglossaryentry{version control system}{ name={version control system}, description = {a way of organising the development of creative works}}
\newglossaryentry{work plan}{ name={work plan}, description = {summarises all activities and work to be conducted to complete the research, how to organise and execute them in the time-span of the project and which milestones must be met}}
