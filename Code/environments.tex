% =====================================================
% Activity / discussion boxes and supporting commands
% =====================================================

% Colours (expects xcolor with svgnames already loaded)
\newcommand{\exampleColor}{CornflowerBlue!40!white}
\newcommand{\activityColor}{Pink!70!white}
\newcommand{\discussionColor}{\activityColor}
\newcommand{\tipColor}{Green!20!white}
\newcommand{\quoteColor}{Linen}
\newcommand{\tableColor}{Linen}
\newcommand{\takeawaysColor}{black!10!white}
\newcommand{\frameColor}{black!50!white}

% tcolorbox (load common libraries)
\usepackage[many]{tcolorbox}

%======================================================
% Footnotes (coloured symbol cycle in the margin)
%======================================================
\newcommand{\footnotetextbulletsize}{1.8}

\usepackage[perpage,side,marginal,symbol*]{footmisc}
\DefineFNsymbols{hallrapanotti}%
  {{\color{red}\scaletextbullet{\footnotetextbulletsize}}%
   {\color{green}\scaletextbullet{\footnotetextbulletsize}}%
   {\color{blue}\scaletextbullet{\footnotetextbulletsize}}%
   {\color{orange}\scaletextbullet{\footnotetextbulletsize}}%
   {\color{gray}\scaletextbullet{\footnotetextbulletsize}}%
   {\color{purple}\scaletextbullet{\footnotetextbulletsize}}%
   {\color{black}\scaletextbullet{\footnotetextbulletsize}}}
\setfnsymbol{hallrapanotti}

% --- margin notes that respect page side and ragged-outer formatting ---
\usepackage{marginnote}

\makeatletter
\newcommand{\HRoutermnote}[1]{%
  \begingroup
    \if@twoside
      \ifodd\value{page}\normalmarginpar\raggedright
      \else\reversemarginpar\raggedleft\fi
    \fi
    \marginnote{\footnotesize #1}%
  \endgroup
}
\makeatother

\newcommand{\HRtcboxfootnote}[1]{%
  \refstepcounter{footnote}%
  \textsuperscript{\thefootnote}%
  \HRoutermnote{\thefootnote\quad #1}%
}

\tcbset{
  footnotes-in-margin/.style={
    before upper={\let\HRsavedfootnote\footnote \let\footnote\HRtcboxfootnote},
    after  upper={\let\footnote\HRsavedfootnote},
  },
}

% --- Sensible defaults for all boxes here ---
\tcbset{
  boxbase/.style={
    breakable,
    footnotes-in-margin,
    pad at break=2mm,
    left=2mm,
    right=2mm,
    width=\linewidth,
    frame hidden,
    segmentation hidden,
    after=\par\bigskip,
    colframe=\frameColor,
    boxrule=2pt,
	before upper={
	  \setlength{\parskip}{0.5\baselineskip}
	  \setlength{\parindent}{10pt}
	},
	fonttitle=\bfseries,
    colbacktitle=\frameColor,
    coltitle=white,
    title style={opacity=1}, % ensure title text prints even with light fills
%    title after break,       % keep title visible when a box breaks
    sharp corners
  },
}

% --- Specific boxes -------------------------

% Takeaways
\newtcolorbox{takeaways}[1]{%
  boxbase,
  move upwards=-50pt,%%Arbitrary value
  colback=\takeawaysColor,
  title={\Large #1\space Takeaways%
  \addcontentsline{toc}{chapter}{#1\ Takeaways}},
}

% Activity
\newtcolorbox{activity}[1][No title]{%
  boxbase,
  colback=\activityColor,
  title={Activity: #1},
}

% Discussion
\newenvironment{discussion}{\tcbsubtitle{Discussion}}{}

% Guidance (to be used *inside* a tcolorbox)
\newenvironment{guidance}{\tcbsubtitle{Guidance}}{}

% Guidance (to be used *inside* a tcolorbox)
\newenvironment{solution}{\tcbsubtitle{Solution}}{}

% Tips
\newtcolorbox{tip}{%
  boxbase,
  colback=\tipColor,
  title={Tip},
}

% Example (argument = short title)
\newtcolorbox{example}[1]{%
  boxbase,
  colback=\exampleColor,
  title={\textbf{Example: #1}},
}

% Running (argument = title)
\newtcolorbox{running}[1]{%
  boxbase,
  colback=\exampleColor,
  title={\textbf{#1}},
}

%%% Quote box style (kept commented as in your original)
% \renewtcolorbox{quote}{boxbase,colback=\quoteColor,title=\textbf{Quote}}

% Simple Answer environment (unchanged)
\newenvironment{answer}{\textbf{Answer}\begin{trivlist}\item}{\end{trivlist}}

% =====================================================
% Resource list helpers (avoid name clashes)
% =====================================================

% If the upstream preamble already defined \itemcites, don't redefine.
\makeatletter
\@ifundefined{itemcites}{%
  \DeclareMultiCiteCommand{\itemcites}{\fullcite}{}%
}{}
\makeatother

% Produce an itemised list of full citations for a comma-separated key list.
% Uses existing \itemcites if present; no-op if already defined elsewhere.
\providecommand{\fullcites}[1]{%
  \begingroup
  \def\multicitedelim{\item }%
  \begin{itemize}
    \item \itemcites{#1}%
  \end{itemize}
  \endgroup
}

% =====================================================
% “More details” section and helpers
% =====================================================

\newenvironment{MoreDetails}{%
  \subsection{More details}%
  The following citations provide good starting points for further reading:%
  % (No fixed \label here to avoid duplicate-label warnings on reuse.)
}{}

% Macros that wrap tcolorboxes question with consistent subtitles + resources
\newcommand{\ResGenTechnique}[1]{%
  \begin{activity}[{Do I need to know about #1?}]%
  Check back to your chosen research strategy from~\Cref{stage3}. Does it involve data generation using #1? If so, read through the remainder of this section and complete the activities.%
  \end{activity}%
}

\ExplSyntaxOn
\NewDocumentCommand{\ReadingList}{m m}
 {
  \readingextras:nn { #1 } { #2 }
 }

\cs_new_protected:Npn \readingextras:nn #1 #2
 {
  % Heading / intro
  \begin{activity}[{Deep~dive~into~#1}]%
  To~find~out~more~about~#1,~take~a~look~at~these~references:%
  \begin{itemize}
    % #2 is a comma-separated list of keys
    \clist_map_inline:nn { #2 }
      {
        \item \fullcite{##1}
      }
  \end{itemize}
  \end{activity}
  %%Hack to correct tcbox behaviour
	\color{black}
 }
\ExplSyntaxOff

% =====================================================
% Part title page helper
% =====================================================

\NewDocumentEnvironment{PartTitlePage}{m m +b}{%
%#1: Stage number
%#2: Stage title
%#3: text
  \renewcommand*{\afterpartskip}{\beforepartskip \noindent\Large #3}%
  \cleardoublepage
  \thispagestyle{empty}
  \newgeometry{left=7cm,right=7cm,bottom=4cm}%
  \part{#1}\label{#2}%
  \restoregeometry
  \cleardoublepage
}{}

\NewDocumentEnvironment{Dedication}{+b}{%
%#1: text
  \centering
  \ %to establish an initial line
  \vspace{3cm}
  
  \addcontentsline{toc}{chapter}{Dedication}
  \noindent\LARGE\textbf{Dedication}
  
  \vspace{3cm}  
  
  \Large #1%
  \cleardoublepage
  \newgeometry{margin=7cm}%
  \restoregeometry
  \cleardoublepage
}{}