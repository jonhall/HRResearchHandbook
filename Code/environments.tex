%%% ===============================================================
%%% Activity / discussion boxes and supporting commands
%%% ===============================================================

% Colours (expects xcolor with svgnames already loaded)
\newcommand{\exampleColor}{CornflowerBlue!40!white}
\newcommand{\activityColor}{Pink!70!white}
\newcommand{\discussionColor}{\activityColor}
\newcommand{\tipColor}{Green!20!white}
\newcommand{\quoteColor}{Linen}
\newcommand{\takeawaysColor}{black!10!white}
\newcommand{\frameColor}{black!50!white}

% tcolorbox (load common libraries)
\usepackage[many]{tcolorbox}

%%======================================================================
%% Footnotes (coloured symbol cycle in the margin)
%%======================================================================
\newcommand{\footnotetextbulletsize}{1.8}

\usepackage[perpage,side,marginal,symbol*]{footmisc}
\DefineFNsymbols{hallrapanotti}%
  {{\color{red}\scaletextbullet{\footnotetextbulletsize}}%
   {\color{green}\scaletextbullet{\footnotetextbulletsize}}%
   {\color{blue}\scaletextbullet{\footnotetextbulletsize}}%
   {\color{orange}\scaletextbullet{\footnotetextbulletsize}}%
   {\color{gray}\scaletextbullet{\footnotetextbulletsize}}%
   {\color{purple}\scaletextbullet{\footnotetextbulletsize}}%
   {\color{black}\scaletextbullet{\footnotetextbulletsize}}}
\setfnsymbol{hallrapanotti}

% --- margin notes that respect page side and ragged-outer formatting ---
\usepackage{marginnote}

\makeatletter
\newcommand{\HRoutermnote}[1]{%
  \begingroup
    \if@twoside
      \ifodd\value{page}\normalmarginpar\raggedright
      \else\reversemarginpar\raggedleft\fi
    \fi
    \marginnote{\footnotesize #1}%
  \endgroup
}
\makeatother

\newcommand{\HRtcboxfootnote}[1]{%
  \refstepcounter{footnote}%
  \textsuperscript{\thefootnote}%
  \HRoutermnote{\thefootnote\quad #1}%
}

\tcbset{
  footnotes-in-margin/.style={
    before upper={\let\HRsavedfootnote\footnote \let\footnote\HRtcboxfootnote},
    after  upper={\let\footnote\HRsavedfootnote},
  },
}

%% --- Local style: swap \footnote for a margin note inside box content ---
%% Note the use of '##1' (one level of macro parameterization in key handlers).
%\tcbset{
%  footnotes-in-margin/.style={
%    before upper={
%      \let\realfootnote\footnote
%      \renewcommand{\footnote}[1]{%
%        \refstepcounter{footnote}%
%        \textsuperscript{\thefootnote}%
%        \marginnote{\footnotesize\thefootnote\quad #1}%
%      }%
%    },
%    after upper={\let\footnote\realfootnote},
%  },
%}

% --- Sensible defaults for all boxes here ---
\tcbset{
  boxbase/.style={
%    enhanced jigsaw,
	parbox=false,%%tcolorbox.pdf, 4.18 Text Characteristics, page 111
    breakable,
    footnotes-in-margin,
    pad at break=2mm,
    left=2mm,
    right=2mm,
    width=\linewidth,      % more robust inside lists/floats than \textwidth
    frame hidden,
    segmentation hidden,
    after=\par\bigskip,
    colframe=\frameColor,
    fonttitle=\bfseries,
    colbacktitle=\frameColor,
    coltitle=white,
    title style={opacity=1}, % ensure title text prints even with light fills
    title after break,       % keep title visible when a box breaks
  },
}

% --- Specific boxes -------------------------------------------------

% Takeaways
\newtcolorbox{takeaways}[1]{%
  boxbase,
  move upwards=0pt,
  colback=\takeawaysColor,
  title={\Large #1\space Takeaways},
}

% Activity
\newtcolorbox{activity}{%
  boxbase,
  colback=\activityColor,
  title={\GetQuestionProperty{subtitle}{\CurrentQuestionID}\hfill\#\CurrentQuestionID},
}

% Discussion
\newtcolorbox{discussion}{%
  boxbase,
  colback=\discussionColor,
  title={Discussion\hfill\#\CurrentQuestionID},
}

% Guidance (to be used *inside* a tcolorbox)
\newenvironment{guidance}{\tcbsubtitle{Guidance}}{}

% Tips
\newtcolorbox{tip}{%
  boxbase,
  colback=\tipColor,
  title={Tip},
}

% Example (argument = short title)
\newtcolorbox{example}[1]{%
  boxbase,
  colback=\exampleColor,
  title={\textbf{Example: #1}},
}

% Running (argument = title)
\newtcolorbox{running}[1]{%
  boxbase,
  colback=\exampleColor,
  title={\textbf{#1}},
}

%%% Quote box style (kept commented as in your original)
% \renewtcolorbox{quote}{boxbase,colback=\quoteColor,title=\textbf{Quote}}

% Simple Answer environment (unchanged)
\newenvironment{answer}{\textbf{Answer}\begin{trivlist}\item}{\end{trivlist}}

%%% ===============================================================
%%% exsheets integration: question/solution as boxes
%%% ===============================================================
\usepackage{exsheets}
\SetupExSheets{
  headings=empty,
  question/pre-hook=\begin{activity},
  question/post-hook=\end{activity},
  solution/pre-hook=\IfInsideQuestionTF{\tcbsubtitle{Discussion}}{\begin{discussion}},
  solution/post-hook=\IfInsideQuestionTF{}{\end{discussion}},
  solution/print=true,
}
% To hide solutions in 'final' builds, you can flip this at class level or here:
% \SetupExSheets{solution/print=false}

%%% ===============================================================
%%% Resource list helpers (avoid name clashes)
%%% ===============================================================

% If the upstream preamble already defined \itemcites, don't redefine.
\makeatletter
\@ifundefined{itemcites}{%
  \DeclareMultiCiteCommand{\itemcites}{\fullcite}{}%
}{}
\makeatother

% Produce an itemised list of full citations for a comma-separated key list.
% Uses existing \itemcites if present; no-op if already defined elsewhere.
\providecommand{\fullcites}[1]{%
  \begingroup
  \def\multicitedelim{\item }%
  \begin{itemize}
    \item \itemcites{#1}%
  \end{itemize}
  \endgroup
}

%%% ===============================================================
%%% “More details” section and helpers
%%% ===============================================================

\newenvironment{MoreDetails}{%
  \subsection{More details}%
  The following citations provide good starting points for further reading:%
  % (No fixed \label here to avoid duplicate-label warnings on reuse.)
}{}

% Macros that wrap exsheets' question with consistent subtitles + resources
\newcommand{\ResGenTechnique}[1]{%
  \begin{question}[subtitle={Activity: Do I need to know about #1?}]%
  Check back to your chosen research strategy from~\Cref{stage3}. Does it involve data generation using #1? If so, read through the remainder of this section and complete the activities.%
  \end{question}%
}

\newcommand{\ResGenExtras}[2]{%
  \begin{question}[subtitle={Activity: Deep dive into #1}]%
  To find out more about #1, take a look at these resources:%
  
  \fullcites{#2}%
  \end{question}%
}

\newcommand{\ResModTechnique}[1]{%
  \begin{question}[subtitle={Activity: Do I need to know about #1?}]%
  Check back to your chosen research strategy from~\Cref{stage3}. Does it involve modelling using #1? If so, read through the remainder of this section and complete the activities.%
  \end{question}%
}

\newcommand{\ResModExtras}[2]{%
  \begin{question}[subtitle={Activity: Deep dive into #1}]%
  To find out more about #1, take a look at these resources:%
  
  \fullcites{#2}%
  \end{question}%
}

\newcommand{\ReadingExtras}[2]{%
  \begin{question}[subtitle={Activity: Deep dive into #1}]%
  To find out more about #1, take a look at these resources:%
  
  \fullcites{#2}%
  \end{question}%
}

%%% ===============================================================
%%% Part title page helper
%%% ===============================================================

\NewDocumentEnvironment{PartTitlePage}{m m +b}{%
  \renewcommand*{\afterpartskip}{\beforepartskip \noindent\Large #3}%
  \cleardoublepage
  \newgeometry{margin=7cm}%
  \part{#1}\label{#2}%
  \restoregeometry
  \cleardoublepage
}{}