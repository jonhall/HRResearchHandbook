\chapter{Stage 2: Your first research increment}

\section{Introducing stage 2}
Stage 2 assumes that you have completed your research proposal at Stage 1, and have discussed it with your supervisor who will have helped you assess whether it is appropriate for your course.\footnote{If your proposal still requires some `remedial' work to fully satisfy your course requirements then you should carry that out before moving on.}

We now move onto the next stage, Stage 2, of our 5-stage framework. The activities which are in focus in Stage 2 are shown in Table~\ref{tab:stage2}, which also provides some guidelines for your interaction with your supervisor during this stage.

\bigskip

Stage 2 works from the assumption that you have now a fair idea of how the research process works, having been through one iteration in Stage 1, and that you have gained sufficient basic research skills to be able to carry on your research somewhat more independently, albeit with your supervisor's support.

The good news\footnote{The bad news is that if you haven't picked up everything, you may need to refer back to Stage 1. We will leave breadcrumbs throughout the chapter to help you do this if you need to.} is that this assumption means that there is less to read in Stage 2 and we will not take you through the research process step by detailed step, but focus on \textit{additional} knowledge and skills you'll need to progress.

\begin{table}[htbp]
\caption{Stage 2 activities\label{tab:stage2}}
\begin{tabulary}{\tablewidth}{@{}LLLL@{}} \toprule
 \textbf{Research activities} & \textbf{Stage 2} \textbf{(15\% of project length)} & \textbf{Effort within stage} & \textbf{Suggested focus of your interaction with your supervisor} \\
\midrule

 \textbf{Identifying the research problem} & Adjust, if needed & 3\% & \\
 \textbf{Reviewing the literature} & Complete full draft of literature review Draft critical summary of key insights from literature review & 25\% & Suitability of literature review structure and narrative flow, including appropriate logical argumentation, and demonstrating critical thinking in deriving insights \\
 \textbf{Setting research aim and objectives} & Adjust, if needed & 2\% & \\
 \textbf{Developing the research design} & Develop research design, with detailed consideration of data and evidence, research strategy and research methods, including an brief review of the related academic literature Review ethical assessment & 40\% & Well thought-out, appropriate and coherent research design overall, showing good understanding of chosen research strategies and methods, supported by related citations, and appropriate justification in terms of research aim and objectives \\
 \textbf{Gathering and analysing evidence} & n\slash a & 0\% & \\
 \textbf{Interpreting and evaluating findings} & n\slash a & 0\% & \\
 \textbf{Reporting, critical reflection and conclusions} & Assess research progress and write up Stage 2 report & 25\% & Any further improvements required, particularly in relation to critical thinking and academic writing \\
 \textbf{Work planning and risk management} & At stage start, re-assess risk and work plan and adjust as needed If you have received feedback from supervisor on your previous stage work, adjust plan to include any revision recommended & 5\% & Any major adjustment required to address deficiencies or manage risk \\
\bottomrule

\end{tabulary}
\end{table}

\begin{question}[subtitle={Activity: Understanding the effort needed in this stage}] Consider Table~\ref{tab:stage2} carefully, taking notice of the entries in the `Effort within stage' column. Make a note of the activities which will be the most time consuming in this stage and what is expected – the suggested focus – for each.

\begin{solution}As $40\%$ of your study time, developing the research design will constitute your major effort in this stage. There are two other research activities at $25\%$
%
\begin{itemize}
\item Reviewing the literature, which will get to an almost complete form; and
\item Reporting, critical reflection and conclusions, which will complete Stage 2.
\end{itemize}
%
\end{solution}\end{question}
%%Hack to correct tcbox behaviour
\color{black}

\section{The literature review in stage 2}
In Stage 2, you will continue to read and critically summarise the literature, building on your draft literature review and further literature review plan from Stage 1. At the end of Stage 2, although your literature review won't be in its final form, it will be close to complete – as you work through later stages, you might find a few more papers\footnote{Depending on how thorough you've been, of course.} to consider, but you'll probably be able to count them on one hand.

The work on your literature review in this stage will be similar to what you have done already, and the knowledge, process and techniques you acquired and applied in Stage 1 will still be relevant\footnote{If needed, refresh your understanding by rereading Section~\ref{sect:litrev} in the previous chapter.}.

\subsubsection{Synthesis: consolidating what you have learnt; starting to summarise your understanding based on your ongoing reading}

Given the number of articles you'll have to read, and the complexity of the relationship that exist between them, there's a long (and sometimes winding) road from the literature to your dissertation. You can make best use of the time you spend reading the literature by using the Keshav workflow introduced in Section~\ref{Keshav}, iterating quickly between papers, and keeping notes that refine what you have as your understanding develops. As you notes develop, they will eventually turn into text you can use for your dissertation, too.

This brings us to synthesis – the act of bringing ideas together into a cogent whole.

That cogent whole will eventually be the narrative of your dissertation – the spell-binding story you create to convince your examiner\footnote{Eventually, your examiners, first your supervisor, family, friends, anyone else that will read it{\ldots}} that you have made a contribution to knowledge. Of course, \emph{at this point,} you don't have that much of an actual contribution to knowledge to write about.

However, what you \textit{can} start writing about at this point is the literature you have read, and the knowledge gaps that you have found there. This is the literature review, discussed next.

\subsubsection{Creating a narrative for your literature review}
Once you have selected, read and understood\footnote{At this point, you may have only a rudimentary understanding, but that will develop as you bring your synthesis together.} a good number of relevant articles, you can start develop the narrative to include in your literature review.

There are two key skills you will be able to demonstrate through this task:
%
\begin{itemize}
\item the first is critical thinking, that is, that you have maintained an objective position by weighing up all sides of an argument, evaluating their strengths and weaknesses, and testing how sound the claims made and their supporting evidence are.

At its essence, critical thinking is the skill of systematically asking questions that get you ``under the hood'' of the research -- perhaps into the hidden crevices into which no one has looked before. Part of this is searching for a lack of evidence or poor reasoning behind an argument, experiment or the application of other research tools, instead of accepting what you read `at face value.'

Critical thinking develops with time and practice. The trick is to balance of scepticism: choosing the arguments or experiments to take issue with -- those that are less solid -- and leaving the rest.

%\begin{tip}You may not believe this, but there's a notion in software engineering of ``coding bad smells'' that was introduced around the turn of the century. Equally unbelievable is that coding bad smells arose from the parenthood experiences of one famous software engineer\footnote{Which you can read about here: \href{http://www-public.tem-tsp.eu/~gibson/Teaching/Teaching-ReadingMaterial/BeckFowler99.pdf}{http:\slash \slash www--public.tem--tsp.eu\slash \ensuremath{\sim}gibson\slash Teaching\slash Teaching--ReadingMaterial\slash BeckFowler99.pdf}\footnote{\href{http://www-public.tem-tsp.eu/~gibson/Teaching/Teaching-ReadingMaterial/BeckFowler99.pdf}{http:\slash \slash www-public.tem-tsp.eu\slash \ensuremath{\sim}gibson\slash Teaching\slash Teaching-ReadingMaterial\slash BeckFowler99.pdf}}}, he ``was under the influence of the odors\emph{[sic]} of his newborn daughter at the time{\ldots}''. The point was that an experienced coding specialists just know when something is wrong with code and needs to be looked at. At the beginning of your research career, you're yet to have that experience. But read, anyway, and see if anything `smells' to you. $<$This really isn't very good$>$\end{tip}
%
As an academic author, critical thinking will, later, benefit your ability to build stronger arguments, avoid bias and link your claims to appropriate supporting evidence so it is well worth picking it up as a skill.

\item the second skill you can demonstrate is that of being able to draw links between the articles that you have found, that is, \textit{synthesis}.

Firstly, synthesis is more than summarising each article in turn: you'll be identifying the things that two articles agree on and disagree on. Then you'll do it for three, then four, etc, and begin grouping articles together under various different themes that you have found, those themes being relevant to your research problem.

At some point, you'll find a collection of more or less definitive unifying themes to which you can assign the papers you are reading. At this point, you'll have moved on from an article-based process to a theme-based process and your synthesis will be really coming together. So much so that, perhaps\footnote{We`re getting a little ahead of ourselves here -- you won't have a complete literature review at this point, but you may start to feel that you can see how, eventually, it could come together{\ldots}}, the themes that you have found could  be used as the titles of sections of your literature review. Your thematic critical reflection on the literature will then be the content of those sections, which continue to grow and grow into the completed literature review.

\end{itemize}

Having got to this point, your completed literature review should be a self-contained piece of academic writing, which shows your critical thinking and mastery of academic writing skills, and through which you:

\begin{itemize}
\item demonstrate your understanding of key ideas and their significance to your research, particularly to framing and justifying your research problem

\item relate different ideas to each other, including arguments and counter-arguments,

\item reason through the evidence to argue the possible contribution to knowledge your research can make

\end{itemize}

In this way, a completed literature review demonstrates your ability to synthesise from the academic literature.

\paragraph{More on critical thinking: Descriptive vs critical writing}

Many authors of fiction use descriptive writing to give vivid, detailed descriptions of their characters in the hope you will feel some empathy for them. Being able to write descriptively is a great skill for a fiction author to have.

Descriptive writing is also an essential part of academic writing, too, but its use is very different: in academic writing descriptive writing is used to set the context and to provide any existing evidence behind an argument you are developing. The key word here%\footnote{You might use different words to those used in the original, for instance, to shorten the original, but your intention will be to retain the original meaning.} 
is \emph{existing}: your descriptive writing should not change the sense of what someone else has written. 

This does not mean that you shouldn't interpret or reinterpret what someone else has written if your additions add value to what someone else has written. That's the key to analysis, synthesis, and evaluation, skills in which are good things to demonstrate!

A\footnote{In your literature review, an article summary is likely to be little more than a sentence in length. You'll see examples next{\ldots}} certain amount of description is therefore necessary in any academic writing. But to present new knowledge, your examiner will be looking for something more than description -- they will be looking for a critical approach, by which you will build\footnote{You will be able to add in your generated evidence too, of course.} new arguments using what has gone before by, for instance, analysing, synthesising, and evaluating.

There perfect critique of an article will have the following components:

\begin{itemize}
\item your introduction to the paper, by saying what is involved, where it takes place, or under which circumstances. %\textbf{Description answers the questions: what? who? where? when?}

\begin{example}{The perfect critique}Kirlappos and Sasse (2012) discuss the implications of users visiting fake websites, concluding that trying to get them to stop doing so could prove difficult without appropriate user awareness training. [\ldots]
\end{example}


\item your analysis, which gives your perspective on the paper, perhaps highlighting how the paper comments on your focus in reading, including its strengths and weaknesses from the perspective of the topic. %\textbf{Analysis answers the questions: how? why? what if?}

\begin{example}{[\ldots]}This training should be automated, involve ‘fake’ phishing e-mails being sent to users in order for them to compare the fake e-mail to a legitimate e-mail and understand the differences. [\ldots]
\end{example}


\item your synthesis, which explains how the parts fit into your research context, giving reasons, making comparisons, and highlighting relationships with other papers. %\textbf{Synthesis answers the questions: where else? which relationships?}

\begin{example}{[\ldots]}This can be contrasted with San Nicolas-Rocca. T \& Olfman, L. (2013, p.84) who state that users should be trained on understanding laws and regulations as well as organisational policies and guidelines which define their specific responsibilities. [\ldots]
\end{example}


\item your evaluation of the strengths and weaknesses of the paper from your perspective, the implications that can be made for your purposes, and the impact and value to your research. %\textbf{Evaluation answers the questions: so what? what next?}

\begin{example}{[\ldots]}This false dichotomy is in danger of losing something, however, in that training can be both instructive on the nature of `fake' emails and on the formal understanding of their responsibilities in regards to cyber breaches.	
\end{example}
\end{itemize}

The boundary between these critical review elements can sometimes be fuzzy, of course, in the sense that is that in-depth descriptions may well start to be analytical and synthetic, and some analysis and synthesis may include a level of evaluation. Table~\ref{tab:synthesis} (\textbf{Why are synthesis and analysis together? Is there no difference we can provide?})provides some practical examples which may help you distinguish between description, analysis, synthesis and evaluation in academic writing.

\begin{table}[htbp]
\begin{minipage}{\linewidth}
\setlength{\tymax}{0.5\linewidth}
\centering
\small
\caption{{\color{red}Split A/S up, needs refactoring}Distinguishing between description, analysis, synthesis and evaluation academic writing. (Adapted from \textcite{Cottrell2005}.)\label{tab:synthesis}}
\begin{tabulary}{\textwidth}{@{}LLLL@{}} \toprule
 \textbf{Descriptive} & \textbf{Analytic}&\textbf{Synthetic} & \textbf{Evaluative} \\
\midrule

 What happens or what something is like & Makes comparisons & Identifies significance \\
 Tells a story or the order in which things occur & Explain why things work the way they do & Demonstrate relevance \\
 Explains how to do something & Gives reasons for choices & Draws conclusions \\
 What a theory says or how something works & Structures information based on established criteria & Weighs pieces of information against one another \\
 Lists things, alternatives and options, etc. & Shows relations between pieces of information, parts of a system & Highlights strengths and weaknesses \\
 Describes a system or its components & Demonstrates how a theory works & Considers wider implications \\
\bottomrule

\end{tabulary}
\end{minipage}
\end{table}

\subsection{Core practice for academic writing}

Academic writing has certain core practices that you should be aware of and apply consistently in your project work. They are:

\paragraph{Arguments over descriptions}

Your writing should be based on well-formed arguments, i.e., on claims that are supported by evidence\footnote{Either that of others or that you have discovered yourself.}, on comparing and evaluating alternative arguments, and on forming judgements on the basis of that evidence. Your writing will therefore favour analysis, synthesis and evaluation over descriptions.

In your descriptions you should also focus on essential details and keep general background information to a minimum.

\paragraph{Clarity and precision}

Your writing must make it easy for the reader to follow your arguments and grasp the points you are trying to make. You should:

\begin{itemize}
\item avoid long, over-complicated, poorly punctuated sentences;

\item be accurate in what you report;

\item be precise; avoid ambiguity and vagueness;

\item clearly define terms and concepts that may be open to more than one interpretation to avoid misunderstanding;

\item keep your audience in mind: avoid jargon and, when using technical terms\footnote{If there a lot of technical terms, you might consider keeping a glossary.}, explain any that your readers are unlikely to be familiar with.
\end{itemize}

\paragraph{Topic order}

You need to consider carefully which information your readers need to read first to make best sense of the topics you will introduce\footnote{We get to how to structure an argument later in this stage}. Start with less complex\footnote{You will have experienced which are the simpler and which are the more complex topics in your reading so you will already know in which order to place them.} topics and build on them towards more complex ones. Later topics may depend on earlier ones, and this can give a hint to their presentational order, too.

%%2023-09-28: jgh This will come later
%Often, it's good to present the points that support your own argument first so that you establish your case early in the mind of the reader.
%
%Getting the order right will take time and perhaps many iterations. As you write, you should take time to step back and think about whether your arguments follow logically from each other, or whether moving them around may make your line of reasoning clearer to the reader.

You can make your reader's life easier if similar topics are grouped together in your writing – otherwise, they may get the impression that you are repeating yourself, or they may miss important connections between them. So, group similar topics together.

%, including any analysis and evaluation that relates to them, as well as comparisons and counter arguments.

\paragraph{Signposting}

Another way of making your reader's life easier is to \textit{signpost} your writing, i.e., refer backwards and forwards to avoids the reader getting lost while reading your work. It applies both to the overall structure of your document and to individual arguments within each of its sections.

Common signposting practices at document level are:

\begin{itemize}
\item careful choice of headings and sub-headings

\item `setting the scene'\footnote{Journalists will often structure their narrative in the following way: they will
\begin{itemize}
\item tell you what they're going to tell you;
\item tell you;
\item tell you what they've told you.	
\end{itemize}
%
This is signposting at is best.} at the start of a chapter or major section to provide a roadmap of what comes next

\item summarising key points at the end of a section/chapter.

\end{itemize}

Another form of signalling is at the level of an individual argument, when the use of appropriate `signal words' and `signal phrases' (see Table~\ref{tab:signalwords}) are used to help the reader understand where they are in an argument.

\begin{table}[htbp]
\caption{`Signal words' used in academic writing, inspired by Cottrell (2017)\label{tab:signalwords}}	
\begin{tabulary}{\textwidth}{@{}LL@{}}\toprule
\textbf{Function} & \textbf{Possible word or phrase to use} \\\midrule
\textbf{introducing an argument, a description, a section or a chapter} & first, firstly, first of all, to begin with, initially \\
\textbf{reinforcing similarities/arguments} & similarly, equally, in the same way, also \\
\textbf{adding further evidence/arguments} & furthermore, moreover, in addition \\
\textbf{introducing alternative evidence/arguments} & alternatively, however, on the other hand, differently \\
\textbf{highlighting choices} & either/or, neither/nor \\
\textbf{contrasting ideas/arguments} & instead, by contrast, conversely, on the one hand [...] on the other \\
\textbf{drawing conclusions} & therefore, as a result, as a consequence, in conclusion, consequently, because of this \\
\bottomrule
\end{tabulary}
\end{table}

\paragraph{Write good, grammatical text}

This is an essential characteristic of all written work you are expected to produce. You should therefore proofread carefully all your writing before submission to remove as many grammatical errors and typos as possible.

With modern tools, producing good grammatical text isn't difficult and your examiner will appreciate good grammatical writing. In fact, given the availability of good tools\footnote{Paraphrasing Star Wars: \enquote{Use the tools, Luke, use the tools.}}, they will, most likely, expect you to write perfect natural language prose. 

\paragraph{Use an appropriate format}

You should be sure to understand the required final format for your dissertation. For instance, all your citations and references should comply with the bibliographical style required. Your dissertation pages should be numbered, and all your figures and tables should also be numbered and accompanied by appropriate captions. 

\begin{question}[subtitle={Dissertation format requirements}]
	Check the requirements your module places on the final form of your dissertation then answer these questions\footnote{It may be that you can't answer these questions because they are not part of the requirements. In this case, we've suggested an answer for you.}:
	%
\begin{enumerate}%[start=0,label={(\bfseries R\arabic*):}]
\item How many words should your dissertation be?
\item What font size should you use?
\item What should the line spacing and margins be?
\item Are you allowed to include additional material in appendices?
\item How should the bibliography be formatted?
\item Is there specific information that you should include somewhere in the dissertation?
\item \textbf{More questions here, inspired by OU regs}
\end{enumerate}

\begin{solution}
Our university has no single format. The first that came up from an intranet search had the following guidance:
%
\begin{enumerate}%[start=0,label={(\bfseries R\arabic*):}]
\item The dissertation should be fewer than 12,000 words.
\item The font should be 12-point Times New Roman, with footnotes (if relevant) and indented quotations in 10-point Times New Roman font.
\item The dissertation should be double spaced throughout, ``with he exception of footnotes, bibliography and indented quotations.''
\item Appendices should be avoided unless it is absolutely essential and permission should be sought to be able to include them.
\item The bibliography should use the Harvard style of reference.
\item The title page must give the following information:
%
\begin{itemize}
\item the full title of the dissertation
\item your full name
\item your university identifier
\item the degree for which it is submitted
\item the date (consisting of month and year) of your submission
\item the total word count.
\end{itemize}
In addition, there should be a short statement declaring that no part of the dissertation has been submitted for a degree or other qualification.
\item More here?
\end{enumerate}
\end{solution}
\end{question}


\paragraph{Avoid plagiarism like the plague!}

You can't make a new contribution to knowledge using the work of someone else – using the work of someone else means the contribution won't be new. Duh!

So, in describing your research, you will have to distinguish it from the work of others clearly.

To do that in your writing you will need to acknowledge clearly all sources you have used. This is known as attribution and its role is to give credit where credit is due, avoiding any possible accusation of plagiarism, that is passing off the work of others as if it were your own. In a dissertation, the correct form of attribution is through referencing.

While intentional plagiarism is a deliberate attempt to deceive, something that both universities and publishers take very seriously and may have severe consequences, plagiarism doesn't need to be intentional to be classed as plagiarism.

Even unintentional plagiarism\footnote{Although unlikely at masters level, you can self-plagiarise, meaning that you re-use already existing material that you yourself have published without clear attribution. You can see that academics take this attribution thing very seriously.}, be it through the result of poor practices in academic writing, such as failing to properly delimit and credit text which has been copied \emph{verbatim}\footnote{Even almost verbatim...} from an article, may call the quality of your work and the contribution to knowledge into question. 

To avoid unintentional plagiarism, here are two simple practices you should apply diligently in your work:

\begin{itemize}
\item to include an exact sequence of words from another source – a quote – you must place the quote in quotation marks\footnote{SO that its precise extent is known.} and add both a citation next to it and a full reference in your references section. 

\item Instead of quotations, you should consider rewriting ideas and information from your sources in your own words. In this case, even if you are using your own words, you are still reporting on somebody else's work, so you must acknowledge your source with both a citation and a full reference.
\end{itemize}

If you follow these two simple rules consistently, you will avoid most unintentional plagiarism and your work is more likely to comply with your course requirements and expectations.

%We distinguish between:
%%
%\begin{itemize}
%\item a citation, which is a short-cut that appears in the main body of the text to refer to a specific source
%
%\item a reference, which is the full bibliographic information of a source you cite in your text. References are usually collected in a section at the end of a dissertation, article, report, etc.,
%
%\end{itemize}
%

\begin{question}[subtitle={ACTIVITY: Looking up your University's plagiarism policy}] Look up your university policy on plagiarism and any disciplinary process related to it.

\begin{solution}Our university has strict policy on plagiarism. Intentional plagiarism can lead to severe disciplinary actions, from failing study modules to be expelled from a course. Early on in study, poor academic practices can be addressed by providing extra study support. Repeated offences will incur in disciplinary action, decided on by an Academic Conduct Officer, including failure.
\end{solution}\end{question}
%%Hack to correct tcbox behaviour
\color{black}

\paragraph{Develop your arguments!}
Academic arguments are at the core of any academic writing. But what exactly is an academic argument? One of our favourite models is that proposed by Booth, Colomb and Williams (1995)\textcite{Booth1995} who suggest that an academic argument have 2 crucial parts – the claim and the evidence – and might, in complex arguments, have 3 others – the reason, the warrant and the qualifications. Their model is shown in Figure~\ref{fig:argument}, and the parts are:
%
%
\begin{enumerate}%[label={(\bfseries \arabic*):}]
\item A \textit{claim}, which is a point of view and needs support with... 

\item ... evidence, which provides the grounds on which the claim is made, and ...

\item ...the reason, which is why we believe the claim to be true.

\item The warrant, which explains how the reason is relevant to the claim

\item The qualifications, which are concessions which may limit what is being claimed, for instance by acknowledging objections, alternatives, etc.
\end{enumerate}
%

\begin{figure}[htbp]
\centering
\includegraphics[width=0.8\textwidth]{Screenshot2022-06-20at093914.pdf}
\caption{The five elements of an academic argument, adapted from Booth et al., 1995. \textbf{LR: Needs redrawing as it is currently a merger from a couple of illustrations in that book}
}
%Description
%
%The figure summarises the five elements of an academic argument and their relationship.
%
%End of description
\label{fig:argument}
\end{figure}

%In simple arguments, the warrant – that the reason is relevant to the claim – does not stating if it is obvious. Moreover, for such arguments, the qualification might also be missing.

Booth, Colomb and Williams (1995)\textcite{Booth1995} illustrate their model with the following example.

{\color{red}Needs revisiting}

\begin{example}{A TV watershed}
%
\begin{itemize}
\item Claim: showing violence on TV should only be allowed after the 9pm 

\item Reason: TV violence can have harmful psychological effects on children 

\item Evidence: Smith (1997)\textcite{Smith1997} found that children ages 5–7 who watched more than three hours of violent television each day were 25 percent points more likely to say that what they saw on television was ``really happening''

\item Warrant: If children are protected from watching violence on TV they will be less likely to see violence as a normal part of day-to-day life

\item Qualifications: that a child interprets something on TV as ``really happening'' does not necessarily mean that they will try to emulate it. Not all children are impressionable. Violence is a normal part of day to day life.
\item \end{itemize}
\end{example}

Although qualifications can prove a claim – reducing its scope, for instance, a qualification might require a response and so lead to other arguments. For instance, a rebuttal to \enquote{Violence is a normal part of day to day life} might be the counterclaim that \enquote{To reduce violence in day to day life, we should insulate children from it so that the next generation isn't so likely to see violence as a justified response to problems.} and so to other arguments. These might be structured the same way until all qualifications are discharged to your satisfaction.

Given an argument in this form, the real skill of writing it down is that of producing a narrative that combines all components together in prose that is easily digestible\footnote{Meaning that your reader can easily agree!}.

****

\begin{question}[subtitle={Activity: Differentiating between reason and evidence}] Consider the example above and write down what you think the difference between reason and evidence is.

\begin{solution}According to Booth, Colomb and Williams (1995), reasons are things we think up in our mind, while evidence is somewhat ``out there'' for everybody to see and examine. While in everyday casual conversation, we can often support a claim with just a reason, that should not be the case in academic research where reasons should be backed up by evidence, as your research audience is unlikely to accept your reasons at face value.
\end{solution}\end{question}
%%Hack to correct tcbox behaviour
\color{black}

\begin{question}[subtitle={Acticity: Constructing academic arguments}] Consider your current theme analysis table. Consider some key points you wish to include in your literature review, and try to express them following Booth et al.'s structure.

\begin{guidance}Don't worry if you can't include all the five elements for each argument, but try to capture both your claim and any reasons and evidence in support.

Constructing academic arguments becomes easier with practice and we suggest you spend up to 1 hour on this activity.

\end{guidance}\end{question}
%%Hack to correct tcbox behaviour
\color{black}

\paragraph{Organising your narrative}
Your literature review should tell a story to the reader, supported by evidence from previous research and the work of others. It is therefore useful to start by creating an overview of the main arguments you wish to make, organised in a logical manner.

\begin{question}[subtitle={Activity: Developing your literature review outline}] Based on the arguments you developed in the previous activity, write an outline of your literature review which indicates which arguments to include and in which order. Review and revise your outline, until you are satisfied that all the arguments you wish to make are included and they follow logically from each other.

\begin{guidance}There is no single way to go about doing this. Here are some techniques you may like to try if you find it difficult to get started:

- Draw a mind map of the arguments you have developed and how they relate to each other. Use the map to group and sequence the arguments in your outline

- Write a sentence\slash bullet point for each argument, then re-order them to ensure they follow from one another logically

- Try different ways of grouping your arguments, for instance by theme or chronologically

- Try some free writing, then read, review and re-organise to ensure there is a logical flow

You should spend up to 2 hours on this activity.
\end{guidance}\end{question}
%%Hack to correct tcbox behaviour
\color{black}

\paragraph{}
Once you are happy with your outline, you can start filling in the details. Depending on how much you have read and the arguments you have been able to make, you should end up with a substantial initial draft of you literature review by the end of Stage 1.

As your literature review grows you may find that you will need to structure it further, for instance, by breaking it down into sections and sub-sections. This might be the case if you are covering many different topics, or some related topics should be presented together.

Setting up headings and sub-headings in a table of contents will help you think of the detailed structure of your literature review, even if some of the headings are just placeholders and will be reviewed and finalised later on.

\begin{question}[subtitle={Activity: Choosing headings and sub-headings}] Consider the current draft of your literature review and the further reading you plan to undertake. Consider if breaking it down into sections and sub-sections may help clarity and logical flow, and, in that case, draw a possible table of contents including appropriate headings and sub-headings.

Once you are happy, fit the content of your literature review to that structure.

\begin{guidance}You choice of headings and sub-headings should result in coherent sections and should:
\begin{itemize}
\item indicate, in outline, to the reader how the `story' develops through the document
\item express the purpose of each section
\item accurately reflect the content of each section
\item be concise
\end{itemize}

In other words, your choice of headings and sub-headings should signpost to the reader where they are in your `story', help them follow the thread of your arguments, and allow them to locate efficiently specific content they wish to return to.

Once you fitted your content to the structure, you should check that your arguments continue to follow logically from one another.

Depending on the amount of materials you have, this might be a substantial activity, so you should set 8 hours aside to complete it.
\end{guidance}\end{question}
%%Hack to correct tcbox behaviour
\color{black}

\begin{question}[subtitle={ACTIVITY: Completing your literature review }] Consider your current draft literature review draft and plan for further review from Stage 1, also taking into account any feedback you may have received from your supervisor.

Identify and carry out all remaining work to develop a full draft of your literature review. Conclude your review with a critical summary of the key insights you have gained, which clearly identifies the knowledge gap your project is going to address, so as to justify your research problem, aim and objectives.

\begin{guidance}In this stage, synthesis, critical reflection and academic writing will be more prominent than searching, reading and assimilating content, so that you will need to pay particular attention at how the structure and narrative of your literature review develops as you incorporate more materials.

As you go along, you should continue to take notes of key insights and knowledge gaps you identify, and keep your theme analysis table and summary-comparison matrix up to date.

This is a very substantial activity and you should plan to spend about 25\% of your study time for Stage 2 to complete it.
\end{guidance}\end{question}
%%Hack to correct tcbox behaviour
\color{black}

\subsubsection{Assessing your literature review}
The following activity will help you assess your progress with your literature review and guide any further work required.

\subsubsection{Assessing scope, planning and iterating<Could go later{\ldots}>}
Even if you hit gold first time with your literature search, you are likely to require several iterations of the literature review process in order to develop your full review. In our 5-stage framework, we recommend you develop a comprehensive initial draft of your literature survey by the end of Stage 2, with the bulk of the work on gathering and assimilating articles completed in Stage 1 to allow more time for synthesis in Stage 2. At each iteration\footnote{You'll study how to reflect on things like this later in this stage.}, you should assess the work that still needs to be done and decide what should come next.

So, whether you're trying to find more to read, or updating\footnote{It`s good that you took notes of what you searched for{\ldots} you did, didn't you?} your literature review after some time away, in order to identify further articles to review, you can use different approaches. For instance, you could consider:

\begin{itemize}
\item topics currently under-explored in your theme analysis table $<$WHAT IS THIS, is it the keyword table?$>$

\item interesting ideas or sub-topics you have come across in your previous reading

\item keywords used in relevant articles you have already reviewed

\item articles cited in relevant work you have already reviewed, or that cite that work

\end{itemize}

The last point provides two complementary searching approaches which help you explore the relation of an article to other articles: backwards, to those that came before and it relies upon (its references); and forwards, to those that come after and rely on it (by citing it).

Whichever approach you follow for your next bibliographical search, you need to ensure that you keep the activity focussed on what you are trying to achieve, that is to be able to answer the questions at the start of this section to help you articulate and justify your research problem, and establish your potential contribution to knowledge. There will be opportunities later on in your project to widen your reading of the literature for other purposes.

\begin{question}[subtitle={Activity: Widening your literature review}] Select a number of topics for further exploration and gather a number of articles for further review. Then assimilate and synthesise their content as you did previously, while also updating your theme identification matrix, theme analysis table and summary-comparison matrix as you go along.

\begin{guidance}Apply any of the approaches above to select your new topics. Remember to record all new entries in your BMT, alongside your notes and summaries.

This is a substantial activity which, depending on how much material you have already gathered, may take you several days, if not weeks.

\end{guidance}\end{question}
%%Hack to correct tcbox behaviour
\color{black}

\begin{question}[subtitle={Activity: Assessing your literature review}] Assess your current draft of the literature review by applying the criteria in Table~\ref{tab:litrevcrit}.

\begin{guidance}For each criterium, use the prompts to write down your own assessment and to record what is still missing: the latter will help you identify further work you will need to carry out.
\end{guidance}\end{question}
%%Hack to correct tcbox behaviour
\color{black}

\subsubsection{}
\begin{table}[htbp]
\caption{Criteria for assessing your literature review\label{tab:litrevcrit}}
\begin{minipage}{\linewidth}
\setlength{\tymax}{0.5\linewidth}
\centering
\small
\begin{tabulary}{\tablewidth}{@{}ll@{}} \toprule
 \textbf{Criteria} & \textbf{Prompts} \\
\midrule

 \textbf{Research problem underpinning} & To which extent does it demonstrate your understanding of different facets of your research problem? To which extent does it demonstrate the generality of the research problem? \\
 \textbf{Research problem justification} & To which extent does it argue that the research problem is worth investigating? \\
 \textbf{Potential contribution to knowledge} & How clearly does it articulate what the knowledge gap? How clearly does it articulate why it is significant to address it? \\
 \textbf{Logical progression} & To which extent does it include a logical progression of arguments? \\
 \textbf{Critical writing} & Are connections between ideas appropriately explored? Is there a good balance between description, analysis and evaluation? \\
 \textbf{Supporting references} & To which extent are all key arguments supported by appropriate references? \\
 \textbf{Format and proof reading} & Have you reviewed your writing carefully to remove typos and grammatical errors? Are all citations and references in correct bibliographical style? \\
\bottomrule

\end{tabulary}
\end{minipage}
\end{table}

\section{Research design in stage 2}

In this section we consider some fundamental concepts in research design to help you develop your understanding and inform your project choices. As you learnt in Stage 1, your research design should summarise, explain and justify how your research is conducted, developing into a detailed account of what you have done by the end of your project.

Figure 1 illustrates the relation between the basic building blocks of research design which you will encounter in this section. The \textbf{research} \textbf{methods} are the techniques you will apply in your research to collect, analyse, synthesise and present data and evidence, and to derive findings. Those methods will need to come together within a coherent \textbf{research strategy}, which makes systematic their use to address your research problem, and to meet the aim and objectives of your research. Research strategies are influenced by \textbf{philosophical traditions}, which embody sets of beliefs on the nature of what we can study, how knowledge can be generated, and what is of value in research.

\begin{figure}[htbp]
\centering
\includegraphics[width=288pt,height=333pt]{Screenshot2023-04-20at072215.pdf}
\caption{}
\label{screenshot2023-04-20at072215}
\end{figure}

Figure 1 --- How research design building blocks are related

Description

The figure illustrates how the basic building blocks of research design relate to each other: research methods are techniques applied within a system defined by the chosen research strategy, with such choice informed by underlying sets of of beliefs.

End of description

\subsubsection{Research methods}
\textbf{Research methods} are the means used in research to collect, analyse, synthesise or present data and evidence, and to derive findings from them. It is worth noticing that there is a subtle distinction between data and evidence. Data is raw information with no interpretation attached --- anything you may collect, observe or gather in your research. Evidence is information interpreted to support your academic arguments. Indeed, data form the basis of evidence, so the two concepts are closely linked and often used interchangeably.

The purpose of research methods is to help you conduct your research in a systematic, rigorous, repeatable and reliable fashion. So, research methods are important in research design because they underpin the validity of the research you carry out.

Research methods are many and vary greatly, which can be confusing to the novice researcher --- and even the seasoned one at times! One source of confusion is that the same term is often used to indicate both specific techniques and procedures, and broad \textbf{research strategies} combining many.

In this handbook we will use the term research method as a synonym for research technique\slash procedure, but be aware that you may encounter other meanings while reading the academic literature.

In this section, we recall a wide range of research methods commonly applied in research, particularly at Masters level.

\subsubsection{Data collection methods}

\paragraph{Questionnaires}
A \textbf{questionnaire} is a fixed set of questions organised in a particular order used to gather answers. It can be delivered face-to-face or distributed to respondents to gather their answers. The respondents' answers constitute the generated data that is subsequently analysed by the researcher.

Questionnaires are a data generation technique applicable when:

- you wish to obtain standardised data from many people

- you seek relatively brief information from your respondents

- you expect your respondents to be able to understand and interpret the questions in a straightforward manner.

Questionnaires can be \textbf{self-administered}, in the sense that the respondents complete the questionnaire without the researcher being present, or \textbf{researcher-administered}, in which case the researcher asks the questions and writes down the responses.

\begin{question}[subtitle={Activity: Considering questionnaires}] In relation to your project aim and objectives, reflect on the extent questionnaires may be useful and which form of questionnaire is more likely to fit your needs. Jot down your answer.

\begin{guidance}It may well be that you don't consider this technique useful for your project, in which case you should articulate the reason.
\end{guidance}\end{question}
%%Hack to correct tcbox behaviour
\color{black}

\paragraph{Interviews}
An \textbf{interview} is a form of conversation between the researcher and one or more interviewees, designed by the researcher to gain insights and opinions on a specific topic. The researcher guides and controls the conversation and asks the questions. The interviewees' answers constitute the generated data that is subsequently analysed by the researcher.

An interview is a technique for data generation applicable when you wish to:

- obtain detailed information on a specific issue or topic

- ask open-ended, complex questions, which may be tackled or interpreted differently by different interviewees

- investigate sensitive issues or privileged information that interviewees may not be willing to commit to writing.

Interviews can be one-on-one, between the researcher and one interviewee at a time, or can happen in a group, with several interviewees being interviewed together by the researcher. The latter is referred to as a \textbf{focus groups}.

Interviews can be fully planned or quite open-ended. The former are termed \textbf{structured} and use pre-determined, identical questions with all the interviewees, while the latter are termed \textbf{unstructured} and typically start by introducing a topic but then let the interviewee talk freely around their ideas, experience and beliefs. Somewhere in between are \textbf{semi-structured} interviews, where the researcher selects some themes and related questions upfront, but then adapt them depending on how the conversation with the interviewee develops.

\begin{question}[subtitle={Activity: Considering interviews}] In relation to your project aim and objectives, reflect on the extent interviews may apply and which form of interview is more likely to fit your needs. Jot down your answer.

\begin{guidance}It may well be that you don't consider this technique useful for your project, in which case you should articulate the reason.
\end{guidance}\end{question}
%%Hack to correct tcbox behaviour
\color{black}

\paragraph{Delphi technique}
With the \textbf{Delphi} technique, a group of experts are consulted with a view of obtaining a consensus on a particular issue or topic. It involves an iterative process of collecting, synthesising and circulating anonymous judgements from those experts to eventually arrive at a consensual view. More precisely, each subject expert is initially consulted separately by the researcher, who then anonymises and collates the group responses and circulate them to the same group of experts. The process is repeated until a consensus is reached.

This technique is based on the idea that a group of people are more likely to arrive at an informed and valid position than an individual, with anonymity preventing interpersonal relationships from influencing the outcome. The judgements and consensus gathered constitute the generated data that is subsequently analysed by the researcher.

The Delphi technique is particularly suited to situations in which the researcher wishes to improve their understanding of an under-explored problem or issue in order to inform decision-making.

\begin{question}[subtitle={Activity: Considering the Delphi technique}] In relation to your project aim and objectives, reflect on the extent the Delphi technique may apply. Jot down your answer.

\begin{guidance}It may well be that you don't consider this technique useful for your project, in which case you should articulate the reason.

\end{guidance}\end{question}
%%Hack to correct tcbox behaviour
\color{black}

\paragraph{Observations and measurements}
\textbf{Observations} are used in research to find out what people actually do or what actually happens in a particular context, rather than what has been reported about it. Observations can be of people's behaviour or interactions, e.g., observing a formal meeting in an organisation, or of events and processes, for instance observing a queue at the post-office or a computer-controlled production plant. As such, observations can generate all kinds of data, qualitative and quantitative. Quantitative observations are often referred to as \textbf{measurements}, e.g., the length of time a particular customer has spent waiting in the queue at the post office.

There are two main types of research observation, \textbf{systematic} vs. \textbf{participant}. The former is when the researcher decides in advance what to observe, the schedule of observations and what to record. For example, the observation of a queue at the post office could be planned to take place over a certain week or month, and recordings may include time of arrival and departure of each customer, average and maximum length of the queue, average service time, etc.

In participant observations, the researcher participates directly in the situation under study and produces a rich description of what happens based on what they experience. For instance, in relation to the previous example, the researcher might join the queue at the post office and record their experience in great detail, or even join the staff in the post office to understand why queues are longer or shorter for certain tasks.

\begin{question}[subtitle={Activity: Considering observations and measurements}] In relation to your project aim and objectives, reflect on the extent observations may apply and which type of observations may suit your needs. Jot down your answer.

\begin{guidance}It may well be that you don't consider this technique useful for your project, in which case you should articulate the reason.
\end{guidance}\end{question}
%%Hack to correct tcbox behaviour
\color{black}

\paragraph{Using existing documents or data sets}
The previous techniques can be used to generate primary evidence.

Academic research, however, can also use secondary evidence as its starting point. This can be represented by existing documents of may forms, from academic articles to documents found in organisations, e.g. laws, policies and procedures, reports, formal minutes of meetings, informal communications, etc. Similarly, there are plenty of publicly available data sets that can be used, created by academic communities or private and public organisations, such as business financial data or statistical data from the UK Office for National Statistics or data from social network platforms. As already noted, the academic literature at the core of your literature review is a form of secondary evidence.

\begin{question}[subtitle={Activity: Considering reusing existing evidence}] In relation to your project aim and objectives, reflect on the extent you may be able to reuse secondary evidence, which form this may take and which sources of secondary evidence are available to you. Jot down your answer.

\begin{guidance}It may well be that you won't use secondary evidence (other than the academic literature) for your project, in which case you should articulate the reason.

\end{guidance}\end{question}
%%Hack to correct tcbox behaviour
\color{black}

\subsubsection{Data analysis methods}

\paragraph{Spreadsheets, tables and charts}
Spreadsheets, tables, charts and graphs are the bread and butter of data analysis. They are applicable to all kinds of data, and can be used to summarise and visualise data, and identify interesting patterns. I will assume you are already familiar with this topic, so that this is only a brief overview to refresh your knowledge.

A \textbf{spreadsheet} is a digital tool you can use to capture, display and manipulate data arranged in tables, that is arranged in rows and columns. Common spreadsheets include Microsoft Excel, Apple Numbers and Google Sheets. Spreadsheets are among the most used digital tools, so it is likely you are already familiar with at least their basic functionalities. Spreadsheets have become quite sophisticated tools, including all sort of charts and graphs, as well as programmatic capabilities which allow you to code quite complex data manipulation functions. Some of those advanced functionalities could be advantageous to your research, so it is worth spending some time considering what they can offer to your project. There are plenty of tutorials and other documentation online you can use to learn more.

\begin{question}[subtitle={Activity: Understanding features of spreadsheets}] Conduct a web search on your spreadsheet of choice to find summaries and tutorials on core and advanced functionalities which may be useful in your project.

\begin{guidance}You should make a note of how you could use such functionality in your research, and bookmark web pages of particular relevance to which you may want to return to later on.
\end{guidance}\end{question}
%%Hack to correct tcbox behaviour
\color{black}

\paragraph{}
\textbf{Charts} and \textbf{graphs} are visual representations of data, usually employed to summarise data, and highlight or identify interesting patterns. All modern spreadsheets provide a wide range of charts and graphs which allow you to visualise data from spreadsheet tables.

\begin{question}[subtitle={Activity: Investigating visualisation functionalities}] Investigate the visualisation functionalities of your spreadsheet of choice, identifying those which may be useful in your project.

\begin{guidance}You should make a note of how you may use such visualisations in your research and keep track of useful examples and instructions to which you may return later on.
\end{guidance}\end{question}
%%Hack to correct tcbox behaviour
\color{black}

\paragraph{}
Alongside spreadsheets a growing number of \textbf{data analytics} tools are also available: these are sophisticated digital tools which extend spreadsheet capabilities for collating and visualising data to include some degree of automated analysis, both statistical and based on Machine Learning algorithms. Tools like Tableau and Power BI are notable examples: both are available in free versions for community use and for study.

\begin{question}[subtitle={Activity: Understanding features of data analytics tools}] Conduct a web search on Tableau and Power BI to identify tutorials and other documentation which illustrate their key functionalities. Consider if such tools may be of use in your project, and jot down their potential use.

\begin{guidance}You should identify which of such tools, if any, you may be able to use in your research. You should consider the time which you will require to become proficient in their use, and bookmark web pages of particular relevance to which you may want to return to later on.
\end{guidance}\end{question}
%%Hack to correct tcbox behaviour
\color{black}

\paragraph{Statistical analysis}
\textbf{Statistical analysis} is a broad collection of techniques used to investigate trends, patterns, and relationships in quantitative data. It is a well established field of study with wide application across all kinds of research, and well developed tool support. IN particular, both spreadsheets and data analytics tools include functionalities which allow you to calculate statistical measures on data, check statistical relationships between variables and data sets, and generate basic statistical models. Bespoke statistical tools also exist for more advanced statistical analysis and modelling, like IBM SSPS or Minitab, which are also available in free versions for students.

T802 will not require you to use any advanced statistics, and will assume you are already familiar with some of the basic concepts, should you need to apply them, particularly descriptive statistics and correlation. However, should your project require advanced statistical analysis, then you will need to become proficient in good time to carry out your analysis and interpretation of findings in the second half of your project work.

\begin{question}[subtitle={Activity: Considering statistical analysis}] In the context of your project and in relation to the types of data you will require, reflect on whether you may apply statistical analysis, either basic or advanced techniques. Jot down your answer.

\begin{guidance}It may well be that you don't consider these techniques useful for your project, in which case you should articulate the reason.

\end{guidance}\end{question}
%%Hack to correct tcbox behaviour
\color{black}

\paragraph{Thematic analysis}
\textbf{Thematic analysis} is a way of analysing qualitative data, particularly texts, e.g., transcriptions of interviews or answers to questionnaires or existing text documents, in order to find out something about people's views, opinions, knowledge, etc.

At its core is the identification by the researcher of recurring themes, their definition and relationships: this relies on the researcher's judgement and it is quite subjective.

There are two basic types of thematic analysis, inductive and deductive. In \textbf{inductive thematic} \textbf{analysis} the themes emerge from the data and are not pre-defined, while in \textbf{deductive thematic} \textbf{analysis} some themes are established upfront, possibly based on an existing theory or previous research, which are then used to analyse the qualitative data.

\begin{question}[subtitle={Activity: Considering thematic analysis}] In the context of your project and in relation to the types of data you will require, reflect on whether you may apply thematic analysis. Jot down your answer.

\begin{guidance}It may well be that you don't consider this technique useful for your project, in which case you should articulate the reason.
\end{guidance}\end{question}
%%Hack to correct tcbox behaviour
\color{black}

\paragraph{Content analysis}
\textbf{Content analysis} is a way to investigate certain words, themes, or concepts in qualitative data, whether text, images, videos, or other.

It can be either quantitative, where the focus is on, for instance, counting the occurrence of those words, themes or concepts, or qualitative, where the focus is on interpreting and understanding their meaning and relationships. As such it can be used for many purposes, from discovering and understanding patterns, to looking at intentions behind what is expressed, or to highlight differences of use in different contexts.

\begin{question}[subtitle={Activity: Considering content analysis}] In the context of your project and in relation to the types of data you will require, reflect on whether you may apply content analysis. Jot down your answer.

\begin{guidance}It may well be that you don't consider this technique useful for your project, in which case you should articulate the reason.
\end{guidance}\end{question}
%%Hack to correct tcbox behaviour
\color{black}

\subsubsection{Modelling methods}
At its essence, a \textbf{model} is an abstraction or representation of something, be that a system, a structure or a behaviour. Modelling is used across many disciplines, so a vast repertoire of modelling techniques exist.

Possibly the most important thing you must remember about modelling is expressed by the following oft-cited aphorism:

``All models are wrong, some are useful'' (Box, 1976)

which makes clear that a model should not be regarded as a faithful replication of some reality, but as a tool to investigate some aspects of that reality.

In this section, I will cover a very small set of modelling techniques, which are particularly relevant to T802 projects. A lot more can be found in the academic literature and beyond.

\textbf{Reference}

Box, George E. P. (1976), ``Science and statistics'' (PDF), Journal of the American Statistical Association, 71 (356): 791--799, doi:10.1080\slash 01621459.1976.10480949.

\paragraph{Systems diagrams}
You can use \textbf{systems diagrams} to help you understand the structure of a situation of interest that can be rendered as a system. The term `system' is meant in its widest possible meaning of a set of components interconnected for a purpose. This is a very general and versatile technique that you can apply to all sorts of real-world situations. If you have studied a systems thinking and practice module for your qualification, you will be already familiar with this technique.

There are many different kinds of systems diagrams. For examples, \textbf{systems maps} allow you to sketch the structure of a system by identifying key components and sub-systems. They can be extended to show how those elements influence each other, in which case they are called \textbf{influence diagrams}. On the other hand, \textbf{causal loop diagrams} are used to capture cause-and-effect relations in a system, hence model certain dynamics of that system, particularly underlying feedback structures. They can be turned into \textbf{stock and flow diagrams} by adding quantitative information, so that this type of diagram is useful both for analysis and simulation of systems behaviour.

\textbf{Study Note}: The system thinking diagramming tutorials on Open Learn are a good starting point to look up these techniques. Link: \href{https://www.open.edu/openlearn/science-maths-technology/across-the-sciences/systems-thinking-diagramming-tutorials}{https:\slash \slash www.open.edu\slash openlearn\slash science-maths-technology\slash across-the-sciences\slash systems-thinking-diagramming-tutorials}. For an application of these techniques on a particular case study you can also consider \href{https://www.open.edu/openlearn/science-maths-technology/computing-ict/diagramming-development-1-bounding-realities/content-section-0?active-tab=description-tab}{https:\slash \slash www.open.edu\slash openlearn\slash science-maths-technology\slash computing-ict\slash diagramming-development-1-bounding-realities\slash content-section-0?active-tab=description-tab}

System diagrams have an accepted structure, format and notation but what you choose to describe and include within a system and its components will depend on your own viewpoint. Systems diagrams can be shared with others as learning devices to promote more understanding of a situation.

\begin{question}[subtitle={Activity: Considering systems diagrams}] Reflect on whether you may use systems diagrams in your research, which particular kind may be of particular use and for which purpose. Justify your answer. Jot down your answer.

\begin{guidance}It may well be that you don't consider this technique useful for your project, in which case you should articulate the reason.
\end{guidance}\end{question}
%%Hack to correct tcbox behaviour
\color{black}

\paragraph{UML modelling}
UML (\textbf{Unified Modeling Language}) is a graphical language for visualising, specifying or documenting various artefacts in the process of developing software systems. If you have studied a software engineering module for your qualification, you may be already familiar with UML.

UML can be used as a `sketching' language, to capture elements of systems informally, like you can do with systems diagrams, or as a `blueprinting' language to specify precisely how elements of a software system will be developed. Different kinds of UML diagrams exist, so that you can model elements of software systems both in terms of their structures and behaviours, and their interactions with end-users.

In a T802 project, UML can be used to help you understand existing systems in context, or plan the development of new innovative artefacts.

\begin{question}[subtitle={Activity: Considering ML modelling}] In the context of your project, reflect on whether you may apply UML modelling and for which purpose. Jot down your answer.

\begin{guidance}It may well be that you don't consider this technique useful for your project, in which case you should articulate the reason.
\end{guidance}\end{question}
%%Hack to correct tcbox behaviour
\color{black}

\paragraph{Problem diagrams}
\textbf{Problem diagrams} have their roots in software requirements engineering as a diagrammatic technique to capture requirements in a real-world context to inform the specification of a new software system to satisfy them. They have been subsequently generalised for application to general engineering problems for which some novel solution artefact is to be developed in a real-world context, to guide design and ensure fitness-for-purpose. Problem diagrams span the problem-solution space divide by focusing on those phenomena that characterise a problem and constrain its solution.

In the context of T802 projects, problem diagrams can help you develop a good understanding of real-world requirements in context and explore both constraints and effects of designing new systems for that context.

\begin{question}[subtitle={Activity: Considering problem diagrams}] In the context of your project, reflect on whether you may use problem diagrams and for which purpose. Jot down your answer.

\begin{guidance}It may well be that you don't consider this technique useful for your project, in which case you should articulate the reason.

\end{guidance}\end{question}
%%Hack to correct tcbox behaviour
\color{black}

\paragraph{Statistical modelling}
Many \textbf{statistical modelling techniques} exist. The most commonly applied include those used to model relations between variables, e.g., how crop yields relate to environmental factors, such as soil quality or meteorological conditions, or to model real-world processes, e.g., the spreading of a disease in a population. Once a statistical model is defined, it can then be used to make predictions of what might happen in the real world.

As per advanced statistical analysis, statistical modelling requires well developed statistical knowledge and skills, which you should already possess if you are considering their use in your T802 project.

\begin{question}[subtitle={Activity: Considering statistical modelling}] In the context of your project, reflect on whether you may need statistical modelling and for which purpose. Jot down your answer.

\begin{guidance}It may well be that you don't consider this technique useful for your project, in which case you should articulate the reason.
\end{guidance}\end{question}
%%Hack to correct tcbox behaviour
\color{black}

\subsubsection{Research strategies}

A \textbf{research strategy} is a systematisation of a set of research methods, applied in order to address research problems of a particular kind. The term \emph{methodology} is also sometimes used in the literature with a similar meaning, although methodology also means the study of research methods, so it is an overloaded term this handbook will avoid.

This section provides an overview of some of the best known and most commonly applied research strategies, particularly at Masters level. Note that this not an exhaustive account: for instance it does not cover observational research and ethnography, which are often applied in the social sciences, to observe participants and phenomena in their natural setting usually over long periods of time, so that they are not suited to the time constraints of a Masters project.

\subsubsection{Survey research}
\textbf{Survey research} aims to gain insights which are valid across a target population, by collecting data from a predefined sample in a standardised and systematic way.

A typical application of survey research is to predict the outcome of an upcoming general election by polling data from a representative sample of voters.

For your data collection, you need to identify upfront which data you will collect in a standardised matter, your target population and sample. So questionnaires or structured interviews are usually used for data collection.

In your data analysis, you seek patterns in the sample data collected to arrive at generalisations to the wider population. Statistical analysis is usually applied, possibly complemented by some thematic analysis, if open-ended questions are also included.

For survey research to be successful, you must be able to access an appropriate sample and generate a sufficient volume of data.

The advantages of this strategy are that it can produce a lot of data in a relatively short time, and you can replicate your data collection process on different samples or on the same sample at a later time. However, among its disadvantages are the depth in the data that can sometimes be lacking, its focus on what can be measured, the fact that it cannot reveal cause-and-effect relationships, and can only provide a snapshot at a particular time.

\begin{question}[subtitle={Activity: Considering survey research}] In the context of your project, reflect on whether this strategy might be appropriate and justified, and if so, which data collection and analysis methods you may adopt. Jot down your answer.

\begin{guidance}It may well be that you won't use this strategy for your project, in which case you should articulate the reason.
\end{guidance}\end{question}
%%Hack to correct tcbox behaviour
\color{black}

\subsubsection{Experimental research}
\textbf{Experiments} are used to investigate cause and effect relationships between factors by testing hypothesis or proving\slash disproving causal links.

For instance, you may run an experiment to test ways in which the use of mobile phones just before going to sleep affect people's sleeping patterns.

There are two main kinds of experiments: \textbf{laboratory experiments}, which are carried out in closed environments, such as a laboratory; and \textbf{field experiments}, which are conducted in the `real world'. Laboratory experiments are often applied in engineering and computer science research, while \textbf{field experiments} are usually applied when people are involved.

Possibly the best known kind of field experiment are clinical trials, widely applied in medicine. However, field experiments are also very popular in research which investigates technology in its social context or application of use.

In experiments, for data collection, you would need first to state the \textbf{hypothesis} to be tested: this is a tentative statement about the relationship between phenomena to be tested in the experiment. In the example above, a hypothesis to be tested might be that ``the blue light emitted by a mobile phone reduces the production of melatonin.'' As melatonin is the hormone which controls a person's sleep-wake cycle, its reduction is likely to disrupt a person's sleeping pattern. After formulating the hypothesis, you would then make detailed observations and measurements of outcomes, e.g., the amount of melatonin released by the body, and any changes that take place when particular factors are introduced or removed, e.g., the length of exposure to the blue light.

In analysing your experimental data you seek to explain causal links between factors under study, looking at your observations and measurements under different experimental conditions. Statistical analysis is widely used for data analysis.

For experiments to be successful you must be able to control factors which can affect the outcome. This is possible in laboratory experiments, while the level of control in field experiments is diminished.

Experimental research has well established processes and protocols and is particularly well suited to the consideration of cause-and-effect relations. However, it has its pros and cons. Laboratory experiments are very reliable due to the high level of control, but can be very artificial, with little or no relation to a real-world context. The opposite is true for field experiments.

\begin{question}[subtitle={Activity: Considering experimental research}] In the context of your project, reflect on whether this strategy might be appropriate, and if so, which hypotheses you may wish to test and which type of experiment you may apply. Jot down your answer.

\begin{guidance}It may well be that you won't use this strategy for your project, in which case you should articulate the reason.
\end{guidance}\end{question}
%%Hack to correct tcbox behaviour
\color{black}

\subsubsection{Design science research}
\textbf{Design science} seeks to generate new knowledge about a significant problem or its solution via the design of an artefact. It simultaneously generates knowledge about the problem, the artefact and the method used to design it. Artefact indicates anything made by humans, so this is a very broad definition, encompassing all that does not exist in nature.

Lots of research in Computing is an expression of design science, for instance designing new algorithms able to emulate human cognition.

More than data collection and analysis, in design science you need to follow a process of articulating the problem, and designing, constructing and evaluating a solution artefact. In doing so, you shed new insights on the problem, and argue how the solution and solution process contribute new knowledge. As a result, modelling techniques are widely applied, possibly informed by data collection techniques, like reviewing existing documents or interviews with stakeholders and experts. \textbf{Prototyping} is often used to produce proof-of-concept artefacts to test or demonstrate the design.

For design science research to be successful you must be able to argue that it is not `normal' design, that is you are not simply re-implementing a solution to a well-known problem through a well-known development process.

An advantage of design science research is that it leads to tangible artefacts which fit real-world contexts, and it is particularly suited to emerging and rapidly changing technology-related fields of study, where new problems emerge all the time and known solutions are sparse or become rapidly obsolete. The latter is also a disadvantage, of course, as new solutions may be short lived. Also, it may be difficult to generalise outcomes to different real-world settings. Depending on the nature of the artefact being designed, advanced technical skills may be required.

\begin{question}[subtitle={Activity: Considering design science research}] In the context of your project, reflect on whether this strategy might be appropriate, and if so, in which ways would the artefact design be novel and contribute new knowledge. Jot down your answer.

\begin{guidance}It may well be that you won't use this strategy for your project, in which case you should articulate the reason.
\end{guidance}\end{question}
%%Hack to correct tcbox behaviour
\color{black}

\subsubsection{Case study research}
A \textbf{case study} is used to investigate in great depth a notable instance of what is under study, in its real-world context. Case studies focus on the `how?' and `why?', and what you seek can span from exploring possible questions or hypotheses for follow-up research, to providing a detailed account of a phenomenon in its natural context, to explaining why certain outcomes or phenomena have occurred.

For instance, an example of case study could be a detailed investigation of the US Equifax social security breach of 2017, in which 143 million of their consumer records were stolen by hackers. This may be descriptive of the chain of events that took place or explicative of why things happened the way they did.

Case studies require you to collect data from a great variety of sources, and to focus on depth rather than breadth. Therefore, all data collection techniques which allow you to do so may be used, from interviews to observations to studying existing documents forensically. This will lead to much qualitative data, so that qualitative methods are often needed for the analysis of the evidence.

For a case study to be conducted successfully you must be able to analyse the chosen instance holistically and in its real-world context.

Case studies allow you to study a complex situation where several factors are at play, and to explore alternative meanings and explanations. However, case studies are time-consuming, difficult to perform rigorously and with limited generalisation beyond the particular instance under study.

\begin{question}[subtitle={Activity: Considering case study research}] In the context of your project, reflect on whether this strategy might be appropriate, and which data collection and analysis methods you may adopt. You should also think of whether you will have sufficient time to collect and analyse the amount to evidence required. Jot down your answer.

\begin{guidance}It may well be that you won't use this strategy for your project, in which case you should articulate the reason.
\end{guidance}\end{question}
%%Hack to correct tcbox behaviour
\color{black}

\subsubsection{Systemic inquiry}
\textbf{Systemic inquiry} is used to explore complex, messy problematic situations involving multiple and often contrasting perspectives, with the aim of transforming the situation for social betterment.

Systemic inquiry is based on concepts and principles of systems thinking and systems practice, so that it is the research strategy of choice for T802 projects related to the Systems Thinking in Practice (STiP) Masters degree.

Situations for systemic inquiry can range from local to global. So, it may equally apply to exploring changes in practice within a local organisation, and to international responses to disruptive events such as climate change. Of course, it is highly unlikely that your T802 project will tackle a situation at a global scale!

In systemic inquiry, you must be able to articulate your personal stake in the situation, for example, a deeply felt interest or active involvement , rather than assuming and claiming unbiased passive `neutral' observation. You must also keep your own journal during the course of your research inquiry, tracking changes in your own viewpoint and how you adapted your research as a result. In some sense, a systemic inquiry is a conceptualisation of your own learning system and how it adapts to change during the research. For Masters level research you will not be expected to be fully embedded in the situation, but a successful systemic inquiry should demonstrate \textbf{reflexivity} -- reflecting on your own changing viewpoint and impact on the wider research situation.

To conduct your systemic inquiry successfully you must articulate your research problem with reference to one or more systems of interest pertaining to the problematic situation under study, and frame the research in terms of possible systems change. You must also have access to sources of different perspectives on the situation under study in order to generate your evidence. This may include both primary evidence from people involved and affected by the situation, and secondary evidence from official and grey literature associated with the situation; your own research journal will also be a source of evidence. In terms of methods, a systemic inquiry is essentially a qualitative endeavour in which you can complement traditional research methods with other tools and techniques from your existing repertoire of expertise and professional tradition: this is known as \textbf{bricolage} research (Kincheloe, 2011).

Study Note: Grey literature is the term used for the collection of information produced by organisations whose primary or commercial remit is not publishing, such as as academia, government bodies, or non-publishing businesses and industries. It includes pre-publication and non-peer-reviewed articles, theses and dissertations, research and committee reports, government reports, conference papers, accounts of ongoing research, etc.

Systemic inquiry helps you make sense of complex situations of change and uncertainty, while acknowledging that the research will not deliver `certainty' in terms of `problem-solving' associated with complex situations. Instead, with its STiP principles of humility, empathy, and inevitable fallibility, reflexivity can support development of trust amongst participants, including trust with the researcher in co-exploring the situation.

\begin{question}[subtitle={Activity: Considering systemic inquiry}] In the context of your project, reflect on whether a systemic inquiry might be appropriate, and which data collection\slash analysis\slash modeling methods you may adopt. Jot down your answer.

\begin{guidance}It may well be that you won't use this strategy for your project, in which case you should articulate the reason.
\end{guidance}\end{question}
%%Hack to correct tcbox behaviour
\color{black}

\subsubsection{}
\textbf{References}

Kincheloe, J. L. (2011). Describing the bricolage: Conceptualizing a new rigor in qualitative research. In Key works in critical pedagogy (pp. 177--189). Brill.

\subsubsection{Mixed methods research}
\textbf{Mixed methods research} combines quantitative and qualitative methods to gain different perspectives on phenomena of interest, by exploring connections and contradictions between quantitative and qualitative data.

For instance, in looking at acceptance of a new technology, a mixed methods approach could consider both levels of adoption and demographics , and the reasons behind adoption or otherwise, possibly to inform further development of the technology.

\textbf{Study note}: Mixed-method research should not be confused with multi-method research, which simply indicates the use of many methods, possibly all qualitative or quantitative. It should also not be confused with bricolage -- the adoption and continual adaptation of methods drawing on a practitioner's own experience of using different methods in different situations.

Data collection and analysis will depend on the particular combination of methods selected. An important aspect is the consideration of how connections between findings are established, through comparing and contrasting data from the different methods applied. This is also referred to as \textbf{triangulation}.

The main advantage of mixed methods research is that it can provide a more holistic understanding of the phenomena under study, and facilitate different avenues for exploration. It is particularly suited to situations in which neither quantitative nor qualitative methods alone can provide sufficient insights. However, mixed methods make research design more complex and demanding in terms of execution time, skills required and data variety to handle and analyse.

\begin{question}[subtitle={Activity: Considering mixed methods research}] In the context of your project, reflect on whether this strategy might be appropriate. If so, which methods would you combine in your project? Jot down your answer.

\begin{guidance}It may well be that you won't use this strategy for your project, in which case you should articulate the reason.
\end{guidance}\end{question}
%%Hack to correct tcbox behaviour
\color{black}

\subsubsection{Systematic review research}
A \textbf{systematic review} is a literature review linked to a clearly defined research problem or question. It uses a rigorous set of criteria to identify, select, and critically appraise relevant research from previously published studies in order to generate a scholarly synthesis of the evidence in relation to that problem or question. Such a synthesis is meant to advance a field of research.

For example, a systematic review of randomised controlled trials on the effectiveness of a specific medical treatment could be used to advance evidence-based medicine.

In a systematic review you only use secondary evidence from published studies. You must decide upfront your research problem\slash question and the set of criteria you will use to select, summarise and evaluate those studies. The type of analysis you will conduct will depend on the nature of the evidence you are considering and combining. In \textbf{narrative reviews}, a narrative synthesis is produced, while in \textbf{meta-analysis}, statistical techniques are used to analyse and combine results.

To be successful, a systematic review has to be both systematic and extensive, which requires the researcher to have a very grasp of the subject area, in order to establish appropriate criteria and make a novel contribution to knowledge.

Because of their explicit set of criteria, systematic reviews are considered transparent, reliable, and easy to replicate. However, they can be very time-consuming due to the large body of work to review. Also, in striving to piece together evidence from potentially very different studies, they may obscure important differences. Narrative reviews may also be subject to bias.

\begin{question}[subtitle={Activity: Considering systematic review research}] In the context of your project, reflect on whether this strategy might be appropriate. Jot down your answer.

\begin{guidance}It may well be that you won't use this strategy for your project, in which case you should articulate the reason.
\end{guidance}\end{question}
%%Hack to correct tcbox behaviour
\color{black}

\subsubsection{Philosophical traditions}
Research methods and research strategies are strongly related to \textbf{philosophical traditions}, which are world-views that inform how one should conduct research. Philosophical traditions may sound a bit esoteric, but they matter in that they make explicit assumptions behind research design choices, influencing what a researcher chooses to research and the way they may go about collecting evidence or interpreting findings.

The term \emph{research paradigm} is also used in the literature with a similar meaning.

\paragraph{}
Each philosophical tradition embodies a set of beliefs around three fundamental philosophical issues:

- The nature of our world, which relates to questions such as: What is there? What kind of categories do things belong to? How are those categories related? The part of philosophy dealing with these questions is called \textbf{ontology}. In research design, ontology determines which phenomena are there to be studied as part of the research, and underlies our experience of the world. Hence, ontology is closely connected with the kind of observations we make or evidence we gather.

- How knowledge is acquired, which relates to questions such as: What does it mean to know something? How can one claim to know something? What makes a belief justified? The part of philosophy dealing with these questions is called \textbf{epistemology}. In research design, epistemology is closely related to research methods for knowledge creation and validation.

- What are the values, especially in relation to ethics, which relates to questions such as: What is good or bad? What is right or wrong? Where do values come from? How do we justify our values? The part of philosophy dealing with these questions is called \textbf{axiology}. In research design, axiology is closely related to ethical considerations when planning or executing research.

In what follows, I will provide a brief introduction to some of the better known and most often cited traditions. However, you should be aware that their definitions are not universal, their boundaries not clear-cut, and it is very rarely the case that a research design will fit a specific tradition neatly. You should, instead, consider each of these traditions as a `wrapper' of convenience for a set of beliefs on research practice which have emerged from different disciplines and cultures, and also be aware that such beliefs have changed over time, and continue to do so.

\subsubsection{Positivism}
\textbf{Positivism} is perhaps the oldest tradition, with roots in the natural sciences. It sees the world as ordered and regular, with universal laws governing its functioning, and assumes it can be investigated objectively.

Specifically, positivism encompasses the following set of beliefs:

\begin{itemize}
\item There is a physical world which exists `out there' and can be observed and measured. This also implies that all researchers will observe and measure the same phenomena in exactly the same way.

\item Through observations and measurements, the researcher can produce models of how the world functions, which are `true' explanations of the aspects of the world under study. This also implies that only one true explanation exists.

\item Truths about the world are perfectively objective and independent of the researcher's values or beliefs. This means that all researchers will arrive at the same truth.

\item Research is based on the empirical testing of theories or hypothesis, leading to either confirmation or rejection (a.k.a. `refutation'). As there can only be one truth, either the theory or hypothesis tested is that truth, in which case all subsequent tests will confirm it, or it is not that truth, in which case at some point a test will reject it. The term refutation is used to indicate that a truth, albeit universal, is always tentative: it will be valid until somebody comes up with a test to reject it.

\item Research seeks universal laws and irrefutable facts. This means that re-testing such laws or facts should always confirm them, if they are indeed truths.

\end{itemize}

For instance, starting with the hypothesis that `all swans are white', a positivist researcher would set as a test to look for swans and observe their colour. If all swans are seen white, then the hypothesis would be confirmed, if not, then it would be rejected. If the hypothesis is confirmed, then the truth that `all swans are white' is added to the body of knowledge and will remain so until another test will lead to a rejection --- indeed that's what English people believed until they first spotted a black swan in Australia!

\begin{question}[subtitle={Activity: Summarising positivism}] Given these beliefs, what does positivism assume of the nature of the world (ontology), how knowledge is acquired (epistemology), and what is of value in research (axiology)?

\begin{solution}Ontology: the world exists independently of the researcher, and can be observed and measured objectively.

Epistemology: there are universal truths, which can be acquired by empirical testing of theories and hypothesis. Tests can lead to either confirmation or rejection. Confirmed theories and hypothesis are added to the body of knowledge.

Axiology: positivism values objectivity above all, and dismisses individual's subjective views or experience.
\end{solution}\end{question}
%%Hack to correct tcbox behaviour
\color{black}

\paragraph{}
Positivism has attracted criticism particularly from the social sciences, which consider some of its beliefs untenable, primarily that researchers are totally objective and not influenced by their own values and beliefs, or that knowledge is made of perfectly generalisable truths. This has led to other traditions, which we consider next.

\subsubsection{Interpretivism\slash Constructivism}
With its roots in the social sciences, \textbf{interpretivism} seeks to identify, explore and explain phenomena in social settings, acknowledging that people perceive the world in different ways, mediated by their beliefs, attitudes and values.

Specifically, interpretivism encompasses the following set of beliefs:

\begin{itemize}
\item Different individuals, groups or cultures perceive the world differently and what people consider real is a construction of their mind --- leading to the term \textbf{constructivism} also being used.

\item The researcher is not neutral, and their perceptions of the world are influenced by their values or beliefs. This implies that different researchers can perceive the same phenomena in different ways, and there is no single truth or single explanation of the world.

\item As there are different perceptions of reality, communication among groups of individuals is the only way of constructing some shared meaning or understanding, and this will change over time.

\item As researchers are influenced by their own values and beliefs, they will arrive at different interpretations as a result of their observations. The strengths of their interpretations will depend on the strengths of the evidence and arguments their interpretations are based upon.

\item Research is based on studying people and other phenomena in their `natural' context. Such a context can be unique, so that interpretations based on observations may not be generalisable to other contexts.

\end{itemize}

\begin{question}[subtitle={Activity: Summarising interpretivism/constructivism}] Given these beliefs, what does interpretivism assume of the nature of the world (ontology), how knowledge is acquired (epistemology), and what is of value in research (axiology)?

\begin{solution}Ontology: the researcher acknowledges that they perceive the world based on their belief, values and culture.

Epistemology: the researcher will offer interpretations based on observations in a social context. Different researchers may offer different interpretations. All knowledge is constructed and shared understanding is reached through communication. Interpretations in one context may not be generalisable to other social contexts.

Axiology: The researcher's values and beliefs matter. The strength of their interpretations will depend on the strengths of the evidence and arguments in support.

\end{solution}\end{question}
%%Hack to correct tcbox behaviour
\color{black}

\subsubsection{Critical theory}
Perhaps not as well established as the previous traditions, \textbf{critical theory} originated in the fields of sociology, philosophy and political theory.

Like interpretivism, it assumes multiple interpretations of reality in social contexts. However, it goes a step further by asserting that reality is shaped by those who are powerful, who legitimate particular ways of perceiving the world: `truth' is inherently political, defined by those in charge to the disadvantage of many, and challenged by those who wish to promote equality.

As a result, critical researchers seek to challenge the status quo and perceive research as transformative at a social level, confronting ideology and trying to discover and challenge the mechanisms through which exploitation and disadvantage are perpetuated in society.

\begin{question}[subtitle={Activity: Summarising critical theory}] Given these characteristics, what does critical theory assume of the nature of the world (ontology) , how knowledge is acquired (epistemology), and what is of value in research (axiology)?

\begin{solution}Ontology: reality is the product of power relations, shaped by those who are powerful and there are disadvantages for many.

Epistemology: the researcher confronts ideology and tries to discover the truth of exploitation and the mechanisms by which disadvantage is perpetuated to challenge the status quo and promote social justice and equality.

Axiology: The researcher has the moral responsibility to make things better in society.

\end{solution}\end{question}
%%Hack to correct tcbox behaviour
\color{black}

\subsubsection{Indigeneous}
The traditions I have described so far are attracting increasing criticisms in that they are seen as Western-European centric and often imposed on other indigenous cultures as a result of colonialism.

In counterposition, an indigenous research tradition has emerged with a social and political agenda of decolonisation of indigenous societies. It emphasises the connection between the researcher and their own culture, in the sense that cultural practices and forms of expressions should be reflected in the way the research is conducted, including language, metaphors, oral traditions and knowledge systems. It also advocates an holistic approach which strives to reach a balance between different areas of life, integrating intellectual, social, political, economic, psychological and spiritual dimensions.

\begin{question}[subtitle={Activity: Summarising indigenous traditions}] Given these characteristics, what does the indigenous tradition assume of the nature of the world (ontology), how knowledge is acquired (epistemology), and what is of value in research (axiology)?

\begin{solution}Ontology: reality is determined by the indigenous culture, to which the researcher is strongly connected.

Epistemology: this is determined by indigenous knowledge systems, cultural practices and forms of expressions.

Axiology: The researcher has a social and political agenda of decolonisation of indigenous societies.

\end{solution}\end{question}
%%Hack to correct tcbox behaviour
\color{black}

\subsubsection{Sketching your research design}
It is time for you to put what you have leant into practice to produce a first sketch of the research design for your project. The next set of activities will help you summarise your choices in relation to each of its building blocks.

\subsubsection{Data and evidence}
Table 1 provides a summary of the type of data and evidence that were introduced in Stage 1. In the next activity you will reflect on those which are most relevant to your project.

Table 1 - Common types of data\slash evidence in a research project

\begin{table}[htbp]
\begin{minipage}{\linewidth}
\setlength{\tymax}{0.5\linewidth}
\centering
\small
\begin{tabulary}{\textwidth}{@{}ll@{}} \toprule
 \textbf{Types of data\slash evidence} & \\
\midrule

 \textbf{Quantitative} data that can be quantified or measured, and be given numerical values & \textbf{Numerical} numbers, either discrete or continuous \\
 & \textbf{Ordinal} can be arranged in an order, but are not necessarily numerical \\
 & \textbf{Interval} ordinal data for which we can calculate precisely the interval between any two data points \\
 \textbf{Qualitative} all other data which is descriptive in nature & \textbf{Categorical (or nominal)} correspond to categories which cannot be ordered and on which mathematical operations and function don't apply \\
 & \textbf{Other} e.g., texts, words, images, sounds, etc. \\
\bottomrule

\end{tabulary}
\end{minipage}
\end{table}

\begin{question}[subtitle={ACTIVITY: Detailing data and evidence}] For each type in the table, comment on whether it may be needed in your project.

\begin{guidance}For those needed in your project, you should also give specific examples and indicate the source.
\end{guidance}\end{question}
%%Hack to correct tcbox behaviour
\color{black}

\subsubsection{Research methods}
Table 1 provides a summary of research methods introduced in this section. In the next activity you will reflect on those which are most relevant to your project.

Table 1 - Research methods introduced in this section

\begin{table}[htbp]
\begin{minipage}{\linewidth}
\setlength{\tymax}{0.5\linewidth}
\centering
\small
\begin{tabulary}{\textwidth}{@{}ll@{}} \toprule
 \textbf{Types of method} & \\
\midrule

 \textbf{Data collection} & \textbf{Questionnaires} pre-defined set of questions organised in a particular order, which are distributed to respondents to gather their answers \\
 & \textbf{Interviews} a form of conversation between the researcher and one or more interviewees, designed by the researcher to gain insights and opinions around a specific topic \\
 & \textbf{Delphi} iterative process of collecting and synthesising anonymous judgements from experts to arrive at a consensual view \\
 & \textbf{Observations\slash measurements} direct observation\slash measurement to find out what people actually do or what actually happens in a particular context \\
 & \textbf{Existing documents\slash data sets} reusing secondary data\slash evidence \\
 & \textbf{Other} any other collection method you know which may be applicable \\
 \textbf{Data analysis} & \textbf{Tables and chats} to summarise and visualise data, and identify interesting patterns \\
 & \textbf{Statistical analysis} to investigate trends, patterns, and relationships in quantitative data \\
 & \textbf{Thematic analysis} to identify recurring themes, their definition and relationships \\
 & \textbf{Content analysis} to investigate certain words, themes, or concepts \\
 & \textbf{Other} any other analysis method you know which may be applicable \\
 \textbf{Modelling} & \textbf{Systems diagrams} to help you understand the structure of a situation of interest that can be rendered as a system. Various flavours exist \\
 & \textbf{UML (Unified Modeling Language)} to visualise, specify or document various artefacts in the process of developing software systems \\
 & \textbf{Problem diagrams} to capture requirements in a real-world context to inform the specification of a new software system, or more generally of some novel solution artefact to be designed and develop, and to ensure it fitness-for-purpose \\
 & \textbf{Statistical modelling} to model relations between variables and being able to make predictions \\
 & \textbf{Other} any other modelling method you know which may be applicable \\
\bottomrule

\end{tabulary}
\end{minipage}
\end{table}

\begin{question}[subtitle={ACTIVITY: Detailing research methods}] For each method in the table, comment on whether it is applicable.

\begin{guidance}For those applicable in your project, you should also justify them in terms of the data and evidence needed in your project.
\end{guidance}\end{question}
%%Hack to correct tcbox behaviour
\color{black}

\subsubsection{Research strategies}
Table 1 provides a summary of research strategies introduced in this section. In the next activity you will reflect on those which are most relevant to your project.

Table 1 - Research strategies introduced in this section

\begin{table}[htbp]
\begin{minipage}{\linewidth}
\setlength{\tymax}{0.5\linewidth}
\centering
\small
\begin{tabulary}{\textwidth}{@{}lllllll@{}} \toprule
 \textbf{name} & \textbf{aim} & \textbf{data collection} & \textbf{data analysis} & \textbf{success factors} & \textbf{advantages} & \textbf{disadvantages} \\
\midrule

 \textbf{survey research} & to gain insights which are valid across a target population, by collecting data from a predefined sample in a standardised and systematic way & you need to identify upfront which data you will collect in a standardise matter, your target population and sample & you seek patterns in the sample data collected to devise generalisations to the wider population & you must be able to access an appropriate sample and generate a sufficient volume of data & - can produce a lot of data in a short time - data collection can be replicated on different samples, or the same sample on a later time & * lack of depth * focus on what can be measured * provides a snapshot at a particular time, rather than a longitudinal view * can't reveal cause-and-effect relationships \\
 \textbf{experimental research} & to investigate cause and effect relationships between factors by testing hypothesis or proving\slash disproving causal links & you need first to state the hypothesis to be tested, then make detailed observations and measurements of outcomes and any changes that take place when particular factors are introduced or removed & you seek to explain causal links between factors under study, looking at your observations and measurements under the different experimental conditions & you must be able to control factors which may affect the outcome. This is possible in laboratory experiments, while the level of control in field experiments is diminished. & * there are well established processes and protocols * tailored to the study of causal relations & * Laboratory experiments are very reliable due to the high level of control, but can be very artificial, with little or no relation to a real-world context * The opposite is true for field experiments \\
 \textbf{design science research} & to generate new knowledge about a significant problem or its solution via the design of an artefact. It simultaneously generates knowledge about the problem, the artefact and the method used to design it. By artefact is meant anything made by humans, so this is a very broad definition encompassing all that does not exist in nature & you need to both articulate the problem, and design, construct and evaluate the solution artefact & you need shed new insights on the problem, and argue how solution and solution process contribute new knowledge & you must be able to argue that it is not `normal' design & * leads to tangible artefacts which fit a real-world context * is particularly suited to emerging and rapidly changing technology-related fields of study & * might be difficult to generalise to other real-world settings * may require advanced technical skills * may lead to short shelf-life of the research, particularly in technological volatile fields of study where technology becomes quickly obsolete \\
 \textbf{case study research} & to investigate in great depth a notable instance of what is under study, in its real-world context & you must be able to articulate your personal stake in the situation, for example, a deeply felt interest or active involvement , rather than assuming and claiming unbiased passive `neutral' observation. You must also keep your own journal during the course of your research inquiry, & you may seek to explore questions or hypotheses for follow-up research, or provide a detailed account of a phenomenon in its natural context, or explain why certain outcomes or phenomena have occurred & you must be able to analyse the significant instance holistically and in context & * allows the study of a complex situation where several factors are at play * iallows the researchers to explore alternative meanings and explanations & * can be time-consuming and access may be difficult to obtain * may be perceived as lacking rigour * insights may be difficult to generalise \\
 \textbf{systemic inquiry} & to explore complex, messy situations involving multiple and often contrasting perspectives, with the aim of transforming the situation for social betterment. & - bricolage research applies, in which you can complement traditional research methods with other tools and techniques from your existing repertoire of expertise and professional tradition - your research journal contributes evidence & * mainly a qualitative endeavour * bricolage research applies, in which you can complement traditional research methods with other tools and techniques from your existing repertoire of expertise and professional tradition & * you must articulate your research problem with reference to one of more systems of interest, and frame the research in terms of possible systems change. * you must articulate your personal stake in the situation * You must also have access to sources of different perspectives on the situation under study in order to generate your evidence & * helps you make sense of complex situations of change and uncertainty * support development of trust amongst participants, including trust with the researcher in co-exploring the situation & - acknowledges that the research will not deliver `certainty' in terms of `problem-solving' associated with complex situations \\
 \textbf{mixed methods research} & to gain different perspectives on phenomena of interest, by exploring connections and contradictions between quantitative and qualitative data & will depend on the particular combination of methods selected & a key aspect is consideration of how connections between findings are established, through comparing and contrasting data from the different methods applied & you must be able to apply competently different kinds of methods & * can provide a more holistic understanding of the phenomena under study, and facilitate different avenues for exploration * particularly suited to situations in which neither quantitative nor qualitative methods alone can provide sufficient insights & * add complexity to the research design * can be demanding and time consuming \\
 \textbf{systematic review research} & to generate a scholarly synthesis of evidence in relation to a specific research problem or question & you need to establish and apply a rigorous set of criteria to identify, select, and critically appraise relevant research from previously published studies & you need to generate a critical synthesis of evidence based on the selected set of criteria & - your review must be both systematic and extensive - you need to be a skilled critical thinker and academic writer & - is considered transparent, reliable, and easy to replicate & - can be very time-consuming * is only as reliable as the studies reviewed * can be difficult to synthesise findings from potentially very different studies \\
\bottomrule

\end{tabulary}
\end{minipage}
\end{table}

\begin{question}[subtitle={ACTIVITY: Detailing research strategies }] For each strategy in the table, comment on whether it is applicable to your research.

\begin{guidance}For those applicable to your project, you should say why that's the case, with reference to your research problem, and aim and objectives. You should also comment on which research methods you may apply within, based on your previous research methods analysis, and on how likely you are to be able to apply it successfully.
\end{guidance}\end{question}
%%Hack to correct tcbox behaviour
\color{black}

\subsubsection{Putting it all together}
You should now have enough material to sketch your overall research design.

\begin{question}[subtitle={Activity: Sketching your overall research design}] Based on your judgements, expressed in the previous activities, summarise your research design by addressing each of the following questions:

\begin{itemize}
\item Which evidence and data will you need and why?

\item Where will you source such data\slash evidence from?

\item Which research strategy are you thinking of adopting and why?

\item Which research methods are you thinking of applying for your data collection\slash analysis or modelling within that strategy?

\end{itemize}

Ensure that your answers are justified in term of your research problem, and intended aim and objectives, by indicating explicitly the rationale behind your choices.

\begin{guidance}At this point, your choices will be tentative, but should provide a good starting point for further investigation and a meaningful conversation with your supervisor who will be able to advise you further.

As well as your intended research, you should also keep in mind that you will have limited time and resources to complete your project, so you should limit your choices to be:

\begin{itemize}
\item manageable in terms of their application in the context in which you are going to conduct your research and the time you have available

\item efficient in terms of the data\slash evidence they produce and your ability to process them with the resources and time you have available

\item effective at producing data\slash evidence that you have the skills and expertise to analyse in the time you have available.
\end{itemize}


\end{guidance}\end{question}
%%Hack to correct tcbox behaviour
\color{black}

\subsubsection{Investigating research strategies and methods further}
The overview provided in this section was designed to help you develop a broad understanding of possible choices you can make to help you sketch your initial research design. It is now time to start looking a little deeper into your most likely research strategies and methods.

\begin{question}[subtitle={Activity: Reviewing your chosen research strategies and methods}] Consider the research strategies and methods you have included in your sketched research design. Conduct a small literature review on each of them to help you confirm that they are indeed suitable for your research, and to help you articulate how and why they are suitable for your project.

\begin{guidance}As this is a literature review, your should follow the process and practices you have already learnt and applied, including recording entries and notes in your BMT.

You should focus on materials which will help you understand how they work, and their strength and weaknesses in relation to your project, something you will return to in the next stage of your project, where you will consider specific procedures for applying them.

Your review does not need to be extensive: a couple of references for each strategy\slash method should suffice, as long as they provide the information required. You can use the annotated reading list in the next section as your starting point, but you should also explore the wider literature. Your supervisor should also be able to suggest literature you could start from.
\end{guidance}\end{question}
%%Hack to correct tcbox behaviour
\color{black}

\subsubsection{Annotated reading list}

\begin{itemize}
\item There are many books and web portals which cover a variety of research strategies and methods. You could start from the following:

\begin{itemize}
\item GO-GN (2020), Research Methods handbook, \href{https://go-gn.net/wp-content/uploads/2020/07/GO-GN-Research-Methods.pdf}{https:\slash \slash go-gn.net\slash wp-content\slash uploads\slash 2020\slash 07\slash GO-GN-Research-Methods.pdf} This is a practical introduction to research methods for phd research, written with contributions from doctoral research students.

\item

\item Oates, B.J. (2006) Researching information systems and computing, SAGE. This is very good read for novice research students, particularly in information systems and computing disciplines. It provides a clear, practical and comprehensive introduction to academic research, including key definitions, methods and techniques.

\item

\item The SAGE research portal at \href{https://methods-sagepub-com.libezproxy.open.ac.uk}{https:\slash \slash methods-sagepub-com.libezproxy.open.ac.uk} contains a great variety of resources on both strategies and methods, from articles to video tutorials

\end{itemize}

\item If you are interested in Design Science Research, then you could start from:

\begin{itemize}
\item vom Brocke J., Hevner A., Maedche A. (2020) Introduction to Design Science Research. In: vom Brocke J., Hevner A., Maedche A. (eds) Design Science Research. Cases, Cham. \href{https://doi.org/10.1007/978-3-030-46781-4_1}{https:\slash \slash doi.org\slash 10.1007\slash 978-3-030-46781-4\_1}, which is possibly the most up-to-date introduction to the topic.

\item The International Conference on Design Science Research in Information Systems and Technology (DESRIST) (since 2005) have tracked the development of this research strategy, with many seminal papers published its proceedings. These can be accessed via the OU Library.

\end{itemize}

\item An of-cited reference on case study research is:

\begin{itemize}
\item Yin, R.K., 2009. Case study research: Design and methods. Applied social research methods series, Vol. 5. Fourth Edition, Sage.

\end{itemize}

\item If you are interested in systemic inquiry you could start here:

\begin{itemize}
\item Ison, R. (2017). Systemic inquiry. Ch. 10 in Part 3 Systems Practice: How to Act (pp. 251--274). Springer, London, which is also available as eBook Reading

\item Simon, G and Chard, A. eds. (2014) Systemic Inquiry: Innovations in Reflexive Practice Research. Farnhill: Everything is Connected Press.

\item Ison, R.L., Collins, K.B. and Iaquinto, B.L., 2021. Designing an inquiry-based learning system: Innovating in research praxis to transform science--policy--practice relations for sustainable development. Systems Research and Behavioral Science, 38(5), pp.610--624.

\item McClintock, D., Ison, R. and Armson, R., 2003. Metaphors for reflecting on research practice: researching with people. Journal of Environmental Planning and Management, 46(5), pp.715--731.

\end{itemize}

\item Problem diagrams are part of a wider approach to problem oriented engineering, with roots in software development, but more widely applicable to most forms of design and engineering. The following references are a good starting point:

\begin{itemize}
\item Jackson, M., 2005. Problem frames and software engineering. Information and Software Technology, 47(14), pp.903--912.

\item Hall, J., Rapanotti, L. and Jackson, M., 2008. Problem oriented software engineering: Solving the package router control problem. IEEE Transactions on Software Engineering, 34(2), pp.226--241.

\end{itemize}

\item UML is a modelling language with roots in software engineering. It is now an international standard that can be found at:

\begin{itemize}
\item \href{https://www.iso.org/standard/52854.html}{https:\slash \slash www.iso.org\slash standard\slash 52854.html}

\item Many tutorials are available online, so it should be relatively easy for you to find an introductory one. There are also many UML digital modelling tools, some of which are open source and free to use

\end{itemize}

\item If you wish to find out more about working with table, graphs and charts, you could start with the following free resources from The Open University, UK:

\begin{itemize}
\item `Working with charts, graphs and tables' at:

\end{itemize}

\end{itemize}

\href{https://www.open.edu/openlearn/science-maths-technology/mathematics-statistics/working-charts-graphs-and-tables/content-section-0?active-tab=description-tab}{https:\slash \slash www.open.edu\slash openlearn\slash science-maths-technology\slash mathematics-statistics\slash working-charts-graphs-and-tables\slash content-section-0?active-tab=description-tab}

- `More working with charts, graphs and tables' at:

\href{https://www.open.edu/openlearn/science-maths-technology/mathematics-statistics/more-working-charts-graphs-and-tables/content-section-0?active-tab=content-tab}{https:\slash \slash www.open.edu\slash openlearn\slash science-maths-technology\slash mathematics-statistics\slash more-working-charts-graphs-and-tables\slash content-section-0?active-tab=content-tab}

or the UK BBC Skillswise site at:

\href{https://www.bbc.co.uk/teach/skillswise/graphs/zmkpqp3}{https:\slash \slash www.bbc.co.uk\slash teach\slash skillswise\slash graphs\slash zmkpqp3}

\section{Reporting in Stage 2}
At the end of Stage 2, we recommend you complete a report, extending that of Stage 1, covering the work you have carried on in this stage. As already indicated, writing end-of-stage reports throughout your project will help you consolidate your work, develop your dissertation incrementally, and improve your critical thinking and academic writing skills as you go along.

The recommended structure and content for the Stage 2 report is indicated in Table 1.

Table 1 -- Report structure and content guidance

\begin{table}[htbp]
\begin{minipage}{\linewidth}
\setlength{\tymax}{0.5\linewidth}
\centering
\small
\begin{tabulary}{\textwidth}{@{}ll@{}} \toprule
 \textbf{Structure} & \textbf{Content guidance} \\
\midrule

 Proposed title & Your title should continue to capture succinctly research problem and aim \\
 Sect 1 - Introduction 1.1 Background to the research 1.2 Justification for the research & This section should provide an introduction to your research topic in its wider context (as background) and your justification of why the research is worth pursuing. It should be well articulated and supported by evidence \\
 Sect 2 - Literature review 2.1 Review of existing relevant knowledge 2.2 Critical summary, including knowledge gap to be addressed by the research & Your review should provide a critical account of your in-depth engagement with the academic (and other) relevant literature, including identifying key trends, ideas and possible knowledge gaps. Most of your citations should point to academic articles. Your critical summary should highlight key insights from your review and provide a strong justification for your proposed research. Both coverage and depth of your review matter. You should ensure that your review is well structured, with a logical narrative flow and your arguments are well supported by evidence \\
 Sect 3 - Research definition 3.1 Problem statement 3.2 Aim and objectives 3.3 Knowledge contribution & You should ensure that your research problem is well articulated and appropriate for your course and your personal and professional circumstances, that your aim and objectives are consistent with research problem, and that the intended knowledge contribution of your research is clearly articulated \\
 Sect 4 - Research design 4.1 Evidence and data 4.2 Research strategy and methods 4.3 Ethical, legal and EDI considerations & This section should demonstrated your critical engagement with the key elements of research design, particularly your understanding of the type of evidence, research methods and strategies you will to apply, also supported by your reading of the related literature, and appropriately justified in relation to your research problem, aim and objectives. It should also demonstrate your careful consideration of ethical and legal matters, and that your research will comply with your course and university requirements \\
 Sect 5 - Assessment of your proposed research 5.1 Qualification fit 5.2 Personal and professional fit 5.3 Technical skills and resources needed 5.4 Statement of feasibility 5.5 Personal reflection on research process & In this section you should continue to argue how your research is a good fit across all criteria. You should provide a clear rationale as to why you think what you are proposing is feasible. You should also reflect on your growing understanding of the research process, including key learning and aspects you have found particularly challenging. \\
 Sect 6 - Planning, scheduling and risk assessment 6.1 Statement of progress 6.2 Key priorities in follow-up stage 6.3 Risk assessment & In this section you should reflect on the progress you have made in Stage 2 and establish your priorities for the next stage. You should also review your risk assessment as appropriate. \\
 References & You should keep your growing references in good order and ensure you apply the required bibliographical style consistently. Ideally, you should use a BMT to generate and integrate your references within your report \\
 Appendix - Work schedule & You should include your revised work plan as an appendix \\
 Appendix - Risk assessment table & You should include your updated risk table as an appendix \\
\bottomrule

\end{tabulary}
\end{minipage}
\end{table}

\begin{question}[subtitle={Activity: Putting your report together}] Using your word processor of choice, and starting from your Stage 1 report, create a report with the structured indicated in Table 1, and fill it in by following the guidance provided in the table, and making good use of your notes and summaries from all related activities you have carried out so far.

\begin{guidance}In this first pass at putting together your report, you should focus primarily on completeness, ensuring that each section includes at least draft content.
\end{guidance}\end{question}
%%Hack to correct tcbox behaviour
\color{black}

\paragraph{}
After you have filled in your report with as much material as you can, you should review and revise it until you are happy with your account, and ready to move on. This may take more than one iteration, but you should ensure you do not delay unnecessarily your work for the follow-up stage.

In the next activity, you will use Table 1 to assess whether your report is of good standard.

Table 1 - Criteria to review your report

\begin{table}[htbp]
\begin{minipage}{\linewidth}
\setlength{\tymax}{0.5\linewidth}
\centering
\small
\begin{tabulary}{\textwidth}{@{}ll@{}} \toprule
 \textbf{Criteria} & \textbf{Prompts} \\
\midrule

 \textbf{Completeness} & Are all sections of the suggested structure completed in line with the guidance provided? \\
 \textbf{Good academic writing practices} & Have you applied good academic writing practices throughout? \\
 \textbf{Logical structure and flow} & Have you structured your narrative appropriately to ensure a logical flow of arguments? \\
 \textbf{Supporting references or evidence} & Are your key arguments supported by appropriate references or other evidence? \\
 \textbf{Citation and reference style} & Do all your citations and references comply with the required bibliographical style? \\
 \textbf{Avoiding plagiarism} & Have you acknowledged the work of others and distinguished it from your own appropriately? \\
 \textbf{Standard of English (or any modern language you use)} & Have you proof-read your report carefully to remove all typos and grammatical errors? \\
\bottomrule

\end{tabulary}
\end{minipage}
\end{table}

\begin{question}[subtitle={Activity: Reviewing your report}] Apply the criteria in Table 1 to review your current report and write up a summary of your assessment.

\begin{guidance}For each criteria, consider the related prompts to help you assess your report overall, and write down any further work needed for your next stage.
\end{guidance}\end{question}
%%Hack to correct tcbox behaviour
\color{black}

\paragraph{}
As previously stated, writing up your report is an excellent way to communicate the work you have completed and still planning to do, and something tangible you can share with your supervisor for their formative feedback.
