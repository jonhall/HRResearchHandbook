
\part{Closing}\label{part:closing}

%\section{Concluding remarks}\label{sect:ConcludingRemarks}
Your dissertation submission concludes your Masters project work. If you have come that far, then you deserve much praise and this is a significant intellectual achievement. A successfully project is a strong indication that you have mastered a wider range of research and transferrable skills, which are of great value to your professional development and provide a strong foundation for any future academic or professional research you may choose to pursue, including doctoral studies.

We hope you have found conducting your own research rewarding, despite, and perhaps because of, the challenges that undoubtably you will have encountered and overcome during your project. We also hope you will have found this handbook valuable in supporting you throughout your project.

We wish you all the best for your future career and studies!

\appendix

\chapter{Glossary}\label{ch:Glossary}
\textbf{Academic literature}: the collection of all published research and scholarly work.

\textbf{Active reading}: engaging with written materials in a way which allows you to assimilate the important points in an effective manner

\textbf{Artificial Intelligence (AI)}: a sub-discipline of Computing, aimed at creating software systems able to simulate human intelligence processes.

\textbf{Bibliographical database}: a searchable collection of academic literature.

\textbf{Bibliographic Management Tool (BMT)}: software tool used to collect and save searchable information concerning articles and other literature sources reviewed during research, including digital copies of articles and personal notes, and to generate references, reference lists and bibliographies in a variety of bibliographical styles.

\textbf{Bibliography}: a separate section, usually towards the end of a document, which collects full bibliographical information of sources, whether cited or not, which are relevant to the content of the document.

\textbf{Bibliographical style}: a set of rules which determine what citations and references should look like in academic writing.

\textbf{Categorical (or nominal) data} are qualitative data corresponding to categories that cannot be ordered and on which mathematical operations and function don't apply, e.g., full-time vs part-time study.

\textbf{Citation}: a short-cut that appears in the main body of a written academic piece to refer to a specific source in the academic (or other) literature.

\textbf{Citation searching:} a technique for exploring the literature based on citations in academic articles.

\textbf{Critical writing}: writing displaying a good balance between description, analysis, synthesis and evaluation.

\textbf{Correlation:} statistical relationship among two or more measures, concerning how changes in one measure are reflected in changes in the others.

\textbf{Data analytics tool:} sophisticated digital tools which extend spreadsheet capabilities for collating and visualising data to include some degree of automated analysis, both statistical and based on Machine Learning algorithms.

\textbf{Descriptive statistics:} measures used to provide meaningful summaries of data points within a dataset.

\textbf{Gantt chart}: a scheduling chart used to plan, organise and monitor activities and work over the duration of a project.

\textbf{Google Scholar}: a web search engine specialising in scholarly content.

\textbf{Grey literature}: collection of information produced by organisations whose primary or commercial remit is not publishing, such as as universities, government bodies or businesses (other than publishers). It includes pre-publication and non-peer-reviewed articles, theses and dissertations, research and committee reports, government reports, conference papers, accounts of ongoing research, etc.

\textbf{Interval} \textbf{data} are ordinal data, but for which we can calculate precisely the interval between any two data points. For instance, calendar dates are interval data in the sense that we can calculate precisely the interval between two given dates, e.g., the number of days in between.

\textbf{Kanban board}: an agile project management tool to help individuals or teams organise and track their progress on specific tasks during a project.

\textbf{Machine Learning (ML)}: a branch of Artificial Intelligence aimed at creating software systems able to learn autonomously and improve from experience.

\textbf{Numerical} \textbf{data} are numbers, either discrete or continuous, e.g., the number of students on a module (discrete) or the average temperature in the UK in July 2023 (continuous). Numerical data can be ordered, and mathematical and statistical operations and functions apply.

\textbf{Nominal data}: same as categorical data

\textbf{Ordinal} \textbf{data} are data that can be arranged in an order, but are not necessarily numerical, e.g., a 5-point Likert scale from (1) Strongly disagree to (2) Disagree to (3) Neither agree nor disagree, to (4) Agree to (5) Strongly agree. While these values can be arranged in the order indicated, mathematical and statistical operations and functions don't apply.

\textbf{Plagiarism}: passing off someone else's work, words or ideas as your own, often as a deliberate attempt to deceive.

\textbf{Research asset}: information which is needed, gathered or generated by your research, including articles, data, images, tables, notes, etc, organised and managed in a disciplined and systematic manner.

\textbf{Qualitative data}: descriptive data, like texts, words, images, sounds, etc., including categorical (or nominal) data, e.g., full-time vs part-time study or employed vs unemployed.

\textbf{Quantitative data}: data that can be quantified or measured, and be given numerical values, including numerical, ordinal and interval data.

\textbf{Reference}: the full bibliographic information of a source in the academic (or other) literature which is cited in an academic text.

\textbf{Risk}: the likelihood of something going wrong combined with the impact that may have on a project.

\textbf{Spreadsheet}: a digital tool used to capture, display, analyse and manipulate data arranged in tables.

\textbf{Version control system}: a set of conventions or tools to keep track of different versions of documents and other research assets.

\chapter{References and further reading}\label{ch:ReferencesAndFurther}

Cottrell, Stella. Critical Thinking Skills : Effective Analysis, Argument and Reflection, Bloomsbury Publishing Plc, 2017. ProQuest Ebook Central, https:\slash \slash ebookcentral.proquest.com\slash lib\slash open\slash detail.action?docID=6234915.

A practical handbook to develop your critical thinking skills, packed with activities and practical advice.

Cryer, Pat. The Research Student's Guide to Success, McGraw-Hill Education, 2006. ProQuest Ebook Central, https:\slash \slash ebookcentral.proquest.com\slash lib\slash open\slash detail.action?docID=316264.

A comprehensive introduction to research skills for post-graduate research students. Some elements are more relevant than others to Masters research, so this is a good reference book to dip in and out.

Potter, Stephen (ed.) (2006) \emph{Doing postgraduate research}, SAGE study skills, 2nd edition., Los Angeles London New Delhi, SAGE.

Another comprehensive introduction to research skills for post-graduate research students, possibly more suited to PhD students than Masters students.

Etzold, Daniel 2020). My Workflow for Reading Scientific Papers. \href{https://betterhumans.pub/my-workflow-for-reading-scientific-papers-d4b27dbb38a6}{https:\slash \slash betterhumans.pub\slash my-workflow-for-reading-scientific-papers-d4b27dbb38a6}

Some practical advice from a practitioner. This is a personal account, rather than a tried-and-tested method. Nevertheless, it contains some good tips that you may find useful.

\textbf{References}

Klopper, Rembrandt, Lubbe, Sam \& Rugbeer, H., 2007. The matrix method of literature review. Alternation, 14(1), pp.262--276.

\textbf{References}

Keshav, S. (2007) `How to read a paper', ACM SIGCOMM Computer Communication Review, 37(3), pp. 83--84.

\textbf{References}

Cottrell, S. (2005) Critical thinking skills. Basingstoke: Palgrave Macmillan

\textbf{References}

Cottrell, Stella (2017). Critical Thinking Skills : Effective Analysis, Argument and Reflection, Bloomsbury Publishing Plc.. ProQuest Ebook Central, https:\slash \slash ebookcentral.proquest.com\slash lib\slash open\slash detail.action?docID=6234915.

\textbf{References}

M. K. S. Sastry and C. Mohammed, ``The summary-comparison matrix: A tool for writing the literature review,'' IEEE International Professional Communication 2013 Conference, 2013, pp. 1--5, doi: 10.1109\slash IPCC.2013.6623891.

\textbf{References}

Booth, W., Colomb, G.G. and Williams, J.M. (1995) \emph{Making good arguments: an overview}. The Craft of Research, London: The University of Chicago Press.

Simon, H. A. (1969). The sciences of the artificial. The MIT Press.

\textbf{References}

Data Protection in the EU, \href{https://commission.europa.eu/law/law-topic/data-protection/data-protection-eu_en}{https:\slash \slash commission.europa.eu\slash law\slash law-topic\slash data-protection\slash data-protection-eu\_en} (Last accessed: February 2023)

\textbf{References}

UKRI Equality, diversity and inclusions key principle. \href{https://www.ukri.org/about-us/policies-standards-and-data/good-research-resource-hub/equality-diversity-and-inclusion/}{https:\slash \slash www.ukri.org\slash about-us\slash policies-standards-and-data\slash good-research-resource-hub\slash equality-diversity-and-inclusion\slash }(last accessed: November 2022)

\clearpage
