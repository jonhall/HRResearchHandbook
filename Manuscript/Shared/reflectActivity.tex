%%Repeated Activity for all reflections
\begin{question}[subtitle={Activity}]
$<$Needs assessing for content and structuring into activity + guidance$>$

This activity has four parts: the first is something you should be doing regularly, but won't make you into a disobedient or indocile thinker. The second, third and fourth may help you get started and keep going.

Part 1: Think about your study this far -- using this book and anything you've done for your dissertation in parallel -- as a journey. More from elsewhere, including   !!.

Part 2: think about yourself and the way you think. How does your desk look? Is it messy or tidy? Do the same for your computer desktop. Is it empty or are there hundreds of files strewn across it? Do you think your tidiness or untidiness will affect the way you do your research? How about how you keep your -- critically important -- bibliographic database which may contain up to a hundred academic\footnote{It's not unknown to have more than a hundred.} and other articles by the time you're finished?

Part 3: think about the context of your research. Which professional pressures are there on you to succeed in solving your research problem? Pressures could come in many forms: financial -- there's a promotion for you at the end of it; peer -- your colleagues know that you are studying will have good expectations of your result and you'll want to prove them right\footnote{Or wrong, depending on the colleague!}. Are you sponsored by your employer? Will you be able to report a negative outcomes to your research, for instance, that there is no solution to our problem using the current technology stack? A negative result is a very good research outcome, even if it tends to satisfy fewer non-academics than a positive result.

Which family pressures do you feel? It's' not unusual that you will have given up a paying role to study, moving the responsibility to provide onto another member of your family. What are their expectations?

Part 4: what's that thought nagging at the back of your mind? Is it ``How will I start?'' Or ``Will I be able to dedicate enough time to this?'' Or ``Can I really do this?''. Or ''Is ``shouldn't I be bringing in a wage rather than studying?''

You may be one of the lucky ones that doesn't have such negative thoughts, but negative thoughts are a very natural part of steps into the unknown. And research is precisely that, a step into the unknown. At least if you are aware of the doubts you naturally have, you can manage them. Think about making even the tiniest of steps forward in your research visible and celebrated! Work with Kansan boards where progress is encouragingly visible as you move a task from the inbox to the outbox. If you have concerns about managing your time, start using one of the many tools out there that break time up into manageable units and help manage it for you. If your concerns are about how to organise your thoughts, look into mind maps, lists, todo lists.

Thinking early and often through reflection is a powerful way of doing better. Do it well and your final report will be better than you will have expected.

It's worth saying that, at the end of what could be an exhausting journey, you will not fully appreciate your achievements. That realisation may have to wait until you are rested, graduated, or some distant time later.

But it will come.

\begin{guidance}
%Hack to correct tcbox behaviour
\color{black}
Something here
\end{guidance}\end{question}
%%Hack to correct tcbox behaviour
\color{black}

