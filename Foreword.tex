\chapter*{Foreword}
You have just about embarked in your very first Masters research project. That's both exciting and daunting.

Exciting because you will be able to focus on a topic of your own choice and to investigate in depth an issue or problem which is of particular interest to you, either personally or professionally. In doing so, you will acquire a deep knowledge of that topic, conduct a unique and novel research study, and develop and apply a wide range of research skills.

Daunting because your success will depend on you demonstrating your mastering of the topic and of the research process, that you can exercise competently a wide range of research skills, and can communicate your work effectively through an academic dissertation, possibly the largest and most demanding piece of writing you will ever undertake.

The aim of this handbook is to support you in taking your first steps into academic research at Masters level. It will provide you with a solid scaffolding for you to become a competent and confident researcher. It will demystify the language around academic research and, through practical advice and activities, will help you plan, manage and execute your project work successfully from start to finish.

But the benefits don't stop with the completion of your Masters project. In succeeding in your research project, you will have also gained and demonstrated a wide range of skills which are professionally relevant and valued by employers, from problem solving to effective communication, digital and information literacy, self-management and resilience. These transferable skills will serve you well in your profession and your life, regardless of the path you will take after your Masters course.

\chapter{So, you want to do a research project!}
Well, you've come to the right place{\ldots}

\section{What is academic research?}
In its most general sense, research is investigation, fact-finding, exploration and analysis. However, academic research is first and foremost about the \emph{generation of new knowledge} with respect to what is already known in a field of study. New knowledge is what academic research must deliver.

Academic research is therefore about \emph{the process of knowledge generation} and how it is executed. The expectation is that of a \emph{systematic and rigorous process} of collecting, analysing and interpreting data, of drawing well-founded conclusions from evidence and explicitly stated assumptions, and of presenting findings in a clear and logical manner. There is also an expectation that your work is legal and ethical in all respects, both in the treatment of other participants and of any sensitive information, and in the way you report your work. It is this rigorous process of academic research that you will learn and practise by following the guidance in this handbook.

Academic research is also a \emph{collaborative endeavour} within a field of study: even if you are conducting your project on your own, you will build on what researchers have done before --  standing on the shoulders of giants as one great scientist once said, adding one more piece to the jigsaw puzzle so that future researchers can build on what you have done -- by standing on yours!

With the completion of your Masters, as a new researcher you will join a community of peers.

\begin{question}[subtitle={Activity: What is academic research?}] You will have been involved in a process of personal research already in your life, probably daily, discovering knowledge that is \emph{new to you}. Every time you search the web and find information about something you didn't know, for instance, your own personal knowledge grows.

Take five minutes to think about how this personal research process differs from academic research. Write down your answer\footnote{This book is full of activities like this. Each one is designed to help your personal research process about your Masters project. Each asks you to think about something and then write down your answer, which you should always do. Writing down your answer means you can revisit it later.}.

\begin{solution}If the knowledge you seek is something you can easily find in existing sources, e.g., books, articles, the web\footnote{Of course, we're talking reputable sources. All that is on the web isn't knowledge! Your studies will lead to new \emph{reputable} knowledge.}, etc., then it is already part of the existing body of knowledge, so there is no new knowledge generated in a general sense. It is also unlikely you will have followed a rigorous and systematic process in collecting and analysing data to draw your conclusions, possibly because data are incomplete or of poor quality, or your analysis is not very rigorous. The issue you are investigating may also be of no relevance to a wider community. Rather than academic research, this is a process of personal learning.
\end{solution}\end{question}
%%Hack to correct tcbox behaviour
\color{black}

\subsubsection{Masters level research}
Now that you know what academic research is about, you may be wondering what is special about Masters level research. This is research in the context of a Masters programme: typically such programmes have capstone\footnote{A \emph{capstone} is a large flat stone that completes a tomb. A capstone project completes a degree or other programme of study. The two are not related:)} projects as their final module. 

Masters research is no different from any other kind of academic research in that it you will be expected to generate some original knowledge within your field of study, perhaps by addressing a relevant, non-trivial issue or problem within that field. It will be conducted in a rigorous, systematic and ethical manner, by applying accepted research methods and techniques.

However what makes your project Masters level compared to a Doctorate or post-doctoral research is primarily its scope: you must complete your Masters project within the time constraint of your Masters course, which will be a small fraction of the time you would have to spend on a Doctorate\footnote{A Masters capstone module is usually 600 hours of study. A Doctorate could be as much as 10 times that.}. Therefore, at Masters level you must trade off your ambition as a researcher with what you can realistically achieve within the available time. Often Masters projects take an existing idea or approach and apply it in a novel way, while a Doctorate project may come up with a ground-breaking, completely new approach.

Masters degrees leave very little room for trial and error, hence it is critical to identify a scope appropriate for your intended research very early in your studies. This might well involve choosing a topic proposed by an academic who will become your supervisor. In addition, your choice of topic will be limited by the degree for which you are studying. If you're doing a Masters in Geography, for instance, a capstone project in the area of information security is not going to be relevant.

Lastly, the range of methods and techniques\footnote{You'll hear a lot about methods and techniques that apply at Masters level in Section ? The skills you will build on top of them -- critical thinking, problem solving, reflection, for instance -- are generally useful, particularly in your professional career. So, there are very good reasons to study them well, even if you might get a bit sick of them during your project.} you will apply at Masters level is likely to be narrower than those used by a Doctoral researcher, as the latter may be too time consuming to be feasible. This, too, limits the type of the research problems you will be able to address in your project.

\begin{example}{Example}The recent explosion of AI solutions has been driven by the creation of technology which is able to do human-like things, for instance the creation of images and text that is sometimes indistinguishable from what you or I might write. Some might go as far as to say that AI is creating new knowledge!

A good topic for a Masters' project might be to investigate the extent to which AI technology can be said to be `doing research'.

The originality of such project comes from the novel combination and application of existing techniques, or might build a framework for distinguishing new knowledge from the simpler re-hash of old knowledge.

This project might come up with new Intellectual Property\footnote{Intellectual Property (IP) is the practical part of new knowledge. Many Masters projects have led to IP that has been developed, franchised, and\slash or sold. Many areas, including law, health, software engineering, and others, recognise that IP is an important way of delivering economic benefits to a client base, business, or public body. Your university will have an IP policy that you should know about before signing up if there's any chance your IP will be valuable. There's more -- much more -- about this in Section ?} (IP) capturing ways of distinguishing AI- and human-knowledge generation that leads to practical tools for use by business and education.
\end{example}

\section{What you will have achieved as a Masters' graduate}

While there may be differences in the specifics of each Masters, there are benchmarks which define the knowledge and skills all Masters' graduates must attain. Many countries use standard frameworks to define their academic qualifications. In the UK, for instance, the UK Frameworks for Higher Education Qualifications align to the Framework for Qualifications of the European Higher Education Area, bringing together standards across the European Union to ensure comparability between European degrees.\footnote{The ENIC--NARIC Networks are notable in ensuring the academic recognition of qualifications across 55 countries. See \href{https://www.enic-naric.net/page-framework-qualifications-europe-and-north-america-region}{https:\slash \slash www.enic-naric.net\slash page-framework-qualifications-europe-and-north-america-region}}.

The UK has adopted the following definitions:

\begin{quote}
Master's degrees are awarded to students who have demonstrated:

\begin{enumerate}
\item a systematic understanding of knowledge, and a critical awareness of current problems and\slash or new insights, much of which is at, or informed by, the forefront of their academic discipline, field of study or area of professional practice

\item a comprehensive understanding of techniques applicable to their own research or advanced scholarship

\item originality in the application of knowledge, together with a practical understanding of how established techniques of research and enquiry are used to create and interpret knowledge in the discipline

\item conceptual understanding that enables the student:

\begin{itemize}
\item to evaluate critically current research and advanced scholarship in the discipline

\item to evaluate methodologies and develop critiques of them and, where appropriate, to propose new hypotheses.

\end{itemize}

\end{enumerate}

Typically, holders of the qualification will be able to:

\begin{enumerate}
\item deal with complex issues both systematically and creatively, make sound judgements in the absence of complete data, and communicate their conclusions clearly to specialist and non-specialist audiences

\item demonstrate self-direction and originality in tackling and solving problems, and act autonomously in planning and implementing tasks at a professional or equivalent level

\item continue to advance their knowledge and understanding, and to develop new skills to a high level.

\end{enumerate}

And holders will have:

\begin{enumerate}
\item the qualities and transferable skills necessary for employment requiring:

\begin{itemize}
\item the exercise of initiative and personal responsibility

\item decision-making in complex and unpredictable situations

\item the independent learning ability required for continuing professional development.

\end{itemize}

\end{enumerate}

\end{quote}


\subsubsection{}
This definition establishes key \emph{learning outcomes}\footnote{A \emph{learning outcome} defines what a student should be able to do after their study. Learning outcomes are tested in assessment, exams and, in the case of research, the dissertation or thesis. Although, in this case, they are based on a standard, they may be described differently in for your Masters -- often useful detail is added. Knowing the learning outcomes for the Masters degree you are studying can give you a real advantage when it comes to writing!} for a Masters graduate, primarily:

\begin{itemize}
\item Advanced knowledge at the leading edge of an academic discipline

\item Critical thinking in evaluating existing knowledge, research methods and their application to generate new knowledge, and in making judgements and deriving conclusions

\item Originality in applying knowledge and techniques to address complex problems and generate new knowledge

\item Effective communication to diverse audiences

\item Self direction, autonomy and independent learning

\end{itemize}

\section{The role of your supervisor}
For your Masters, you will be assigned an academic supervisor to support you throughout the project -- a research expert highly knowledgeable in the subject area of your research. It may even be the case that your project topic comes from suggestions from, or previous research conducted by, your supervisor.

There's a hint in the title -- your supervisor is there to supervise your studies: to guide, make suggestions, and engage in discussion on all aspects of your research. They are unlikely to want to micro-manage your project (but it does sometimes happen!). But, whatever, ensuring that you make the required progress and meet the required standards remains your responsibility.

It's sometimes that case that, as you make progress in your research your knowledge and expertise will grow to rival that of your supervisor. However, your supervisor, as an expert in the \emph{process} of research, will continue to provide invaluable guidance and support throughout your project.

Depending on your course regulations, your supervisor may have additional roles. For instance, your supervisor may be required formally to assess whether you are making the necessary progress as you reach each milestone, or may be required to validate that the research carried out is indeed your own work. Missing their assessment and validation is very likely to slow progress, and in the worst case, stop you completing your Masters.

So, to discharge their various roles, your supervisor must have sight of your work and how it develops throughout the project: engaging with your supervisor regularly, discussing your ideas and their development in detail will be a very important part of conducting your project.

It is therefore essential to your success that you develop a close working relationship with your supervisor as early as is possible and engage in an active dialogue on all aspects of your research on a continuing basis.

You supervisor may have more than one student conducting research with them. This means that there will also be an expectation that \emph{you} drive and manage the interaction with your supervisor.

\begin{question}[subtitle={Activity: Managing the interaction with your supervisor }] If you haven't already done so, get in touch with your supervisor to introduce yourself and discuss how they want to work together.

After the meeting, write down what you have agreed and create diary appointments in your favourite calendar app accordingly.

Make sure your supervisor formally agrees -- although, these days, this is usually just a matter of RSVPing to a calendar invitation.

\begin{guidance}Before contacting your supervisor, you should think about \emph{your} availability for a 30 minute meeting every week, for instance, or 60 minutes every fortnight. This will make it easier for you to identify and agree with your supervisor a regular slot to discuss your ideas and work progress. You should also account for periods when you or your supervisor may not be available.

It is important to book regular diary slots for the duration of the project. It may not be that you need every meeting you arrange, especially in the later stages of your research, but it will be good to know that the time is there if you do need it.

This activity is particularly important if you are a part-time student, possibly juggling study, work and family commitments.
\end{guidance}\end{question}
%%Hack to correct tcbox behaviour
\color{black}

\section{What is expected of you}
There are key differences between doing a research project and studying a taught course. Awareness of those differences will help you prepare better for your Masters research.

These differences arise because of the higher demands of an academic research project to demonstrate i) self-direction, ii) critical thinking, iii) time and task management, and iv) information management.

\subsubsection{Self-direction}
A Masters' research project requires you to be in the driving seat\footnote{This includes ``managing'' the interaction with your supervisor, as mentioned earlier.}, including identifying relevant materials you must study and skills you need to develop and apply to carry out your project.

While in your previous studies, most of the materials were likely to be provided for you alongside detailed advice on how and when to study them, on a research project those decisions are left up to you to a large extent. In particular, you will need to decide what to read and what to ignore in relation to your chosen topic, determining its relevance and how to use it in your research.

You will also be responsible for understanding the academic standards required, and for organising your research work. This is part of demonstrating that you can exercise self-direction in planning, executing, and critically reviewing your work and that you are truly an independent learner.

This handbook will provide plenty of advice as to how direct your own research.

\begin{question}[subtitle={Activity -- Self direction}] Write down three areas of your life where you need to be self-directed. For each, write down how you go about being self-directed, from how you start to how you finish. What does it feel like to be self-directed? Is there some emerging pattern?

\begin{guidance}Those areas might be from your work, your previous Masters studies, your personal interests, hobbies, or some other aspects of your life.
\end{guidance}\end{question}
%%Hack to correct tcbox behaviour
\color{black}

\subsubsection{Critical thinking}
Your Masters research will require you to think critically about all aspects of your work. In essence, critical thinking is about asking questions systematically to look for evidence and good reasons in order to form your own judgements, instead of accepting what you read or hear `at face value'.

At the heart of critical thinking is an ability to maintain an `objective' position by weighing up all sides of an argument, evaluating its strengths and weaknesses, and testing how sound the claims made and their supporting evidence are. This will require you to scrutinise arguments in great detail and with some degree of skepticism. Such an objective stance will help you judge good and bad arguments, regardless of whether you agree with them.

In your project, you will be expected to think critically about all aspects of your research and to capture such thinking in your writing. As the author of a dissertation, critical thinking will benefit your ability to build stronger arguments, avoid bias and link your claims to appropriate supporting evidence. As a reader of the academic literature, critical thinking will help you assess the strength of other researchers' arguments and identify unsupported claims and illogical reasoning.

Critical thinking is both an attitude and a skill essential in academic research, which this handbook will help you develop. It is also beneficial to your professional and private life. In fact, it is likely that you already think critically in many aspects of your life.

\begin{question}[subtitle={Activity -- Critical thinking}] Write down three areas of your life where you need to think critically. For each, write down how you go about thinking critically, from how you start to how you finish. What does it feel like to think critically? Is there some emerging pattern?

\begin{guidance}These areas might be the same areas as for the previous activity, or might be different. You might even find that they are parts of previous activities.
\end{guidance}\end{question}
%%Hack to correct tcbox behaviour
\color{black}

\subsubsection{Time and task management}
For your research project, you may be provided with some broad guidelines and deadlines\footnote{Deadlines for research tend to be absolute as they affect the cycle of Masters' assessment and validation.} around which you should conduct your project in general. However, you will be expected to organise your time, and plan and execute your work in detail to meet those deadlines.

Within the set duration of your project, you will need to plan, organise and execute your research and submit your final dissertation. It will be up to you to develop and keep under review your personal research plan, allocate time to tasks, and meet deadlines and milestones\textbf{.}

You will find that one of the challenges of conducting academic research is that it is open ended. Therefore you will need to learn how to set boundaries and use your limited research time effectively to meet the requirements of a Masters.

This handbook will help you plan your project work based on a 5-stage framework we have developed for this purpose.

\begin{question}[subtitle={Activity -- Time and task management}] Write down three areas of your life where you need to manage your time and tasks. For each, write down how you go about managing them, from how you start to how you finish. What does it feel like to manage tasks and time? Is there some emerging pattern?

\begin{guidance}These areas might be the same areas as for the previous activity, or might be different. You might even find that they are parts of previous activities.
\end{guidance}\end{question}
%%Hack to correct tcbox behaviour
\color{black}

\subsubsection{Information management }
The volume of materials you will need to read and study during your project is considerable: you will need to read extensively around the topic of your choice, as well as research approaches to apply in your research. This will result in a large amount of information that you will need to gather, organise, store and make sense of, including making copious notes as you go along.

You will also need to submit a substantial final dissertation. In the UK, for instance, a Masters research dissertation is usually in the range of 10,000 to 15,000 words. This is likely to be much longer than any other written work required by your previous studies. Therefore you will also need to structure your narrative appropriately and present your work in compliance with standards and conventions for academic writing.

All this will require you to develop and apply a range of new skills, which this handbook will help you develop.

\begin{question}[subtitle={Activity -- Information management}] Write down three areas of your life where you need to manage information. For each, write down how you go about managing information, from how you start to how you finish. What does it feel like to manage information? Is there some emerging pattern?

\begin{guidance}Again, these might be the same areas as for the previous activities, or might be different. You might even find that it is part of previous activities.
\end{guidance}\end{question}
%%Hack to correct tcbox behaviour
\color{black}

\section{Key skills}
In this book, there are some basic skills that you will practice over and over again in your project. As they are critical to your success, it's worth considering them upfront. One benefit of thinking about them early is that, if you feel that you're lacking in any of them, you can prioritise their study early on.

\subsubsection{Active reading and note taking}
Throughout your project, and particularly when reviewing the literature, you will read lots of materials, and you are likely to spend more time reading and assimilating new content than you may have experienced in your previous studies.

It is important, therefore, that you become both effective and efficient at reading and note taking. The key points you need to keep in mind are:

\begin{itemize}
\item You take notes as you read.
\item You should be disciplined and systematic in your note taking, so that you can easily locate and review your notes when you need to during your project.
\item You won't need to study in depth all that you read, so you should develop reading techniques both to grasp the essence of an article very quickly, and techniques to dig deeper into content and meaning.
\end{itemize}


These practices will help you become an \textbf{active reader}, that is one who engages with the materials and is able to assimilate the important points in an effective manner.

\begin{question}[subtitle={Activity: Your note taking practice}] Think of how you take notes when reading new materials. Write down key practices you apply and how effective they are in helping you assimilate new materials.

\begin{solution}These are things I usually do. While reading, I highlight definitions and word or phrases of interest, and make annotations in the margin next to paragraphs which make significant points, or provide useful summaries or raise interesting questions. If reading a physical book, I attach post-it notes to pages which are of particular interest, so that I can return to them easily.

After reading a whole article or a chapter in a book, I jot down some bullet points summarising key insights from my reading, my overall judgement of the whole reading, and how it may be useful, or otherwise, for my work.

I usually keep my notes and summaries with the physical or digital copy of what I have read, and I also keep together related articles within my file system --- in the past, I used to have a physical filing cabinet for this, but these days I work almost exclusively with digital content appropriately organised in folders and sub-folders.

You will have come up with a different set of practices. What matters is that you reflect on how effective they are or where improvements may be needed.

\end{solution}\end{question}
%%Hack to correct tcbox behaviour
\color{black}

\subsubsection{Digital literacy and tools}
Unless you're thinking of handwriting your dissertation\footnote{\emph{And it has been done, although perhaps not that recently, check out \textbackslash{}url\{\href{https://www.reddit.com/r/Handwriting/comments/b9vmss/my_mums_handwritten_thesis_from_1982/}{https:\slash \slash www.reddit.com\slash r\slash Handwriting\slash comments\slash b9vmss\slash my\_mums\_handwritten\_thesis\_from\_1982\slash }\}, for instance!}}, you're going to make use of digital tools for text preparation, bibliographic management, note taking, etc.

There is a vast range of tools, some of which you will be already familiar with (but might not be suitable for the very long document you're going to have to write) and some that you are going to have to learn how to use -- and find time for that learning. The most important of this latter group is probably those for \emph{bibliography management}.

So, why don't we start with these?

\subsubsection{Bibliographic management tools}
During your project, it is essential that you keep track of the articles and papers that you read, alongside your notes on their content and relevance, or otherwise, to your project. Some you will access over and over again during your research; others you will only cite as part of your literature review; others still, you may just discard as not relevant. Whatever their final use in your project might be, it is important that you keep track of what you have read and the use you can make of it as you go along.

As the amount of reading grows, keeping track can quickly become overwhelming. This is where a \textbf{Bibliographic Management Tool}\footnote{These are also known as Reference Management Tools,` 'Reference management software` or 'Bibliographic management software.' Wikipedia has a comparison of bibliographic management tools here: url\{\href{https://en.wikipedia.org/wiki/Comparison_of_reference_management_software}{https:\slash \slash en.wikipedia.org\slash wiki\slash Comparison\_of\_reference\_management\_software}\}} \textbf{(BMT)} can help you. If you are not already using a BMT, this is the time for you to pick one and master its basic functionalities.

Briefly, a BMT is a software tool that can help you collect and save details of documents you have used in your research, and from which you can generate easily references, reference lists and bibliographies in a variety of bibliographical styles, to include in your dissertation and other reports you may have to produce during your studies. Hence, its basic functionalities include:

\begin{itemize}
\item a repository into which you can store the details of each document that you have read (authors, title and so on), your notes on the document, and even a copy of the document itself. Where the repository is online, the tool may allow you to share your collections with other users. In addition, in many cases, the reference sharing service will enable you to download many of the necessary citations from the database.

\item an export mechanism to desktop word processors, such as MS Word, Apple Pages or Apache OpenOffice, so that you can include your citations in your text and generate your reference section automatically. The exact mechanism and its accuracy vary between tools, but it is likely to generate a list of references with most (if not all) the required elements of your chosen bibliographical style --- some manual checking and editing may still be needed to ensure that the output conforms to the required style.

\end{itemize}

\begin{question}[subtitle={Activity: Investigating BMTs}] Visit the wikipedia article that compares notable reference management software (the link is in the related note above). Count how many are included, which ones are free or licensed, and which operating systems, devices and work processing software they are compatible with.

Choose a couple which may work with your computing devices and software. Review their key features.

\begin{guidance}If you already know or use a BMT, you may wish to compare its features with those of others in the wikipedia article. If you don't know where to start, you could consider Zotero first.

It is also possible your university has licensed one BMT that you can use for your project, so you should find out whether that is the case and what its key features are.

\end{guidance}\end{question}
%%Hack to correct tcbox behaviour
\color{black}

\subsubsection{Keeping track of your digital assets}
Beside your growing repository of articles from the literature, which you should manage through your chosen BMT, during your project you will be collecting and producing lots of other digital \textbf{research assets} which are needed for your research and will contribute to your dissertation. These may include data, images, tables, your own notes, comments from your supervisor, etc. All these assets will need organising and managing in a disciplined and systematic manner.

The simplest and cheapest way to manage digital assets is to organise your file system appropriately. Here are two basic practices you could apply:

\begin{itemize}
\item Establish a naming convention, so that the content of each file is clear from their name.
\item Establish a coherent folder structure, so that the relation between files within each folder is clear, as is the relation between folders.
\end{itemize}

\begin{question}[subtitle={Activity: Organising your digital assets}] Think of how you currently organise and manage your file system. Write down any rule or convention you usually apply and reflect on the extent it helps you keep track and locate easily the information you need.

\begin{solution}I try to keep my file system well organised and ensure I review it and clean it up on a regular basis. My file system is partitioned between personal and work-related assets. Under each category, I organise assets thematically, using folders and sub-folders structures. For instance, in my work-related folder, I keep a list of my projects organised into current, completed or terminated sub-folders, the latter for things I started but for one reason or other couldn't complete. Under each sub-folder, each project has its own folder, appropriately named with the project name. When folders become too crowded, I tend to refactor them, by creating more sub-folders to group related assets.

You will have come up with a different set of practices. What matters is that you reflect on how effective they are or where improvements may be needed.
\end{solution}\end{question}
%%Hack to correct tcbox behaviour
\color{black}

\subsubsection{Managing document versions}
As you progress in your project, you will generate several versions of parts of your reports, for instance as your literature review grows or your chosen research methods change based on your growing research knowledge and feedback from your supervisor.

All these versions should be kept and carefully managed to ensure you can always access the latest version, but also refer back easily to previous work should you need to.

Your word processor may already include some functionalities to manage versions. If so, you should make sure you can use them effectively.

Equally there are bespoke \textbf{version control systems} you can use for this purpose, like Git\footnote{\href{https://git-scm.com}{https:\slash \slash git-scm.com}}, which is free and open source.

However, you can simply use your file system wisely by applying an appropriate versioning convention to keep successive versions of your files in good order.

\begin{question}[subtitle={Activity: Thinking about how you manage document versions}] Think of how you currently manage different versions of your documents or other digital assets. Summarise your practices and reflect on the extent they help you keep track and locate easily different digital versions.

\begin{solution}Part of keeping my file system well organised is to ensure I keep track of the most recent version of any document, but can also locate previous ones easily. While I don't use a separate version control system, I make sure that I keep previous versions of a document I'm working on in a separate folder within my working folder. I use a simple number system to identify versions, that is 1.0, 2.0, 3.0, etc. I only generate a new version when there is a significant departure from the current document, for instance, if I decide to restructure it significantly, or remove a substantial amount of content. I also generate new versions if I share a document with somebody for feedback, while still carrying on working on it.

This is not a perfect system, but it is simple and works for me. You will have come up with a different set of practices. What matters is that you reflect on how effective they are or where improvements may be needed.

\end{solution}\end{question}
%%Hack to correct tcbox behaviour
\color{black}

\subsubsection{Choosing the right word processor}
If you have not written a substantial document, like a dissertation, before you should put some thought into how you are going to write up your project work. Your final dissertation is likely to be between 10,000 and 15,000 words, plus appendices, which makes it a substantial document. By following the advice in this handbook, you will also write reports at each stage, which will build on each other, so that you will be editing, reshaping, extending and re-versioning content you have written as you progress in your project. Therefore, you will need to choose the right tool for the job.

You could start by considering whether any word processors you may be familiar with and possibly used in your previous studies, is appropriate for these requirements. It is possible that you will need to develop more sophisticated word processing skills or use more advanced functionalities of your word processor of choice. You should also take into consideration the format your course requires you to submit your work in, and ensure your chosen software can generate documents in that format.

Whichever word processor you choose, you should ensure it allows you to:
\begin{itemize}
\item Manage multiple documents
\item Structure and process large documents
\item Back up your work regularly, possibly automatically, and restore documents which may have been corrupted by system errors or crashes
\item Connect to your chosen Bibliographic Management Tool (BMT), so that citations and references can be managed easily and shared automatically between the two
\item Generate documents in the formats required for submitting your dissertation and any other required assignment.
\end{itemize}



\begin{question}[subtitle={Activity: Investigating your word processor's advanced functionalities }] Check the functionalities of your current word processor against the list above. Is it powerful enough to meet the above requirements? Write down the key functionalities you will need to use and make a note of any new skills you will need to develop, then set some time aside to practice them.

\begin{solution}LaTex is my favourite document preparation and typesetting system. It is a very powerful tool, particularly suited to scientific and technical content. It is also free and well supported by a large community of users.

Latex meets all the requirements above. I can use it to structure and process large documents, which I can split up and organise in separate sub-documents: for instance I can have different files for different chapters or different versions of a chapter, which I can then include flexibly in my final document.

The Latex system I use backs up my documents automatically and includes a recovery feature in case of system crash. It also connect with my BMT (I use BibDesk\footnote{\textbackslash{}url\{Insert link\} BibDesk is only available for the Macintosh Operating System}, but could use Zotero), so that all my citations and references are easy to manage and shared automatically between the two systems.

One of Latex's most powerful features is that it separates output format from source text, so that I can change the style of citations and references, and in fact of the the whole document, very easily and generate outputs in different styles from the same source text.
\end{solution}\end{question}
%%Hack to correct tcbox behaviour
\color{black}

\paragraph{}
Although you might not like what we're about to say, the real Masters pros will use LaTeX, especially if you're writing:

\begin{itemize}
\item A mathematical or scientific dissertation

\item A dissertation in which you will have lots of figures or tables

\item A dissertation in which you will be cross-referencing between sections lots

\item A dissertation that has a complex structure

\item A dissertation that has many bullet points or enumerated lists

\item A dissertation with many references and citations

\end{itemize}

You can find out why LaTeX is the document preparation system of choice at many blogs\footnote{Including this one \url{https://blog.orvium.io/latex-over-word/}}.

However, LaTeX has a steep learning curve -- or rather the investment of time that you will use for learning LaTeX will be considerable but it will, if your dissertation has any of the above characteristics, save you time and tears in the end, especially when you're making the final touches.

\begin{figure}[htbp]
\centering
\includegraphics[keepaspectratio,width=\textwidth,height=0.75\textheight]{Screenshot2023-05-10at113717.jpg}
\caption{You've got five minutes to submit your dissertation in Word. You want to move that figure a little. What happens to the rest of your dissertation? Source: \url{https://www.reddit.com/r/funny/comments/2glhbp/moving_a_picture_in_microsoft_word/}}
\label{screenshot2023-05-10at113717}
\end{figure}


\begin{question}[subtitle={Activity: Investigating Latex editors }] LaTex is a mature system and numerous editors are now available, from open source to proprietary, from desktop to web-based. My current favourite is Overleaf\footnote{\href{http://Overleaf.com}{http:\slash \slash Overleaf.com}}. 

Conduct a web search to investigate its key features, and write down those which makes it particularly suited to writing a research dissertation.
\begin{solution}
Overleaf is my favourite LaTeX editor for writing research documents of all size because it:
\begin{itemize}
\item Has all the power of LaTeX
\item Is set up for research work
\item Has a fantastic help system
\item Has many, many templates -- and may even have one customised for your university
\item Can be used to track changes between versions
\item Backs up your work
\item Works through a browser, and so is available on all my smart devices\footnote{It is quite difficult to see on my smartphone, though.}.
\end{itemize}
\end{solution}\end{question}
%%Hack to correct tcbox behaviour
\color{black}

LaTeX also has a Word-like interface, through the LyX tool, which can reduce the learning curve. If you find it easier to use a Word-like interface, but want the benefits of LaTeX, you might like to try LyX.\footnote{LyX is available at \textbackslash{}url\{lyx\}.}

\section{What to take away from this chapter}
After reading this chapter you should be aware of the following:

\begin{itemize}
\item academic research is about knowledge generation
\item at Masters level it is critical you establish the right scope for your research as early as possible 
\item your supervisor is your best ally and you should ensure you establish a good working relationship from the start
\item as a Masters level researcher you are in the driving seat and expected to demonstrate critical thinking, self-direction, and time and information management throughout 
\item there are key skills you must develop and practice from the onset
\item choosing the right digital tools for your project will make your research life a lot easier in the long term, even if they require some investment upfront to learn how to use them effectively.
\end{itemize}


%It is time to reflect on what you have learnt so far. Reflection is what makes your learning more effective, and relevant and useful to your own practice, so you shouldn't be surprised that reflection is a common theme in this handbook. You have already been asked to reflect in the practical activities you have carried out so far. In this closing section, you are asked to reflect of what you have learnt in this chapter as a whole.
%
%\begin{question}[subtitle={Activity -- Reflection on learning}] Consider the content of this chapter and write down key things you have learnt or that have surprised you. For each indicate why they are notable or relevant to you, and how you may apply them in new situations or to inform your future learning.
%
%\begin{guidance}tbd
%\end{guidance}\end{question}
%%Hack to correct tcbox behaviour
\color{black}

\chapter{The 5-stage Masters project framework}
This handbook gives you a 5-stage framework with which to approach your research project. The framework has been refined over many years of working with hundreds of Masters students at the Open University\footnote{If you know anything about the Open University, you'll know that it is a distance--learning university, most of its students work at a distance and so don't attend lectures. If you think about it, the ``at--a--distance'' model is precisely what you'll be doing in your Masters project -- you won't have lectures! This makes the Open University model really appropriate for a Master project.}, UK.

We provide an overview in this chapter --- we will break the framework down into stages in the following chapters, each chapter giving detailed guidance for that stage.

\section{What do we mean by framework?}
Writing a Master dissertation is a complex task: the goal is a complete 10,000--15,000 word document on the research you have conducted which meets the required academic standards. There are many risks that you will face in writing it, which include:

\begin{itemize}
\item Misunderstanding what is required

\item Running out of time

\item Not having the skills or resources you need

\item Choosing the wrong methods or techniques, or not applying them correctly

\item {\ldots}

\end{itemize}

The framework we give you will allow you to manage these risks to give you the best chance of submitting something that will satisfy the examiners and show your Masters research skills off in their best light.

The framework comes with recommended stages and research activities, as well as metrics and guidelines to help you parameterise your project based on your course requirements, develop a work plan that meets your needs, and manage your work and your interaction with your supervisor effectively.

Study within our framework and you'll have your best chance of succeeding.

Because the key research requirement is for new knowledge, every Masters student will be working at the leading edge of their discipline. One corollary of this is that every dissertation will be different. Your project will allow you to show how your domain expertise in a focussed topic has developed through your studies and your dissertation will reflect that. That's your goal!

For us in defining our framework, it raises a problem -- we cannot possibly know the details of your final dissertation. We can't know even simple things, like how many pages it will have exactly. We can't know what your arguments will be or what your conclusions or future work will contain. Neither can we know which literature you will consult, what the seminal paper in it is.

That would be a problem if our focus was only on the final form your dissertation will take. But we don't!

What we teach in this handbook is the process of arriving at your final dissertation. That's why we structure our framework into stages -- each stage building on the previous, each moving you further down the line to your final dissertation. And because we have seen literally hundreds of students follow the framework as we have developed it, we know it can work and work well. 

If this has convinced you that we're onto something, welcome aboard!

Your first step with us will be to learn about the research process that you will follow.

\section{The research process and its key activities}
A research process is the sequence of activities you undertake when conducting academic research.

Research is messy --- looking into the unknown means there is no road map to follow and you will often have to retrace your steps and try different paths. So, moderate your expectations -- you shouldn't expect a linear, orderly research process!

Instead, the research process is an iterative, incremental, and adaptive process of knowledge discovery --- it iterates through its research activities in small incremental steps, adapting as your knowledge grows.

The exact path you will follow in your research project will be unique to you. However, you will go through some widely recognised activities, which are common to all research processes.

\subsubsection{Identifying the research problem}
A key concept in academic research is that of \emph{research problem}. A research problem captures the knowledge gap to be addressed (\emph{the need}) by the research within a particular field of study (\emph{the context}). A well-defined research problem is the foundation of any research project as it clarifies the purpose of the research and its intended outcomes.

In this handbook, you will learn a practical approach to identifying and articulating your chosen research project.

\subsubsection{Reviewing the literature}
One of the major tasks you will have in compiling your dissertation is to give the reader sufficient background that they can check where you're coming from. To do so, you will be referring to the work of other authors and researchers throughout your project: this is collectively known as the academic literature. Depending on your degree, you may already have practise in doing this in your previous studies, for instance in writing academic essays for your assignments. For your Masters, however, the more definitive you can be in referencing the academic literature the better. This means that your review of the academic literature will be a significant part of your research project.

Your review has two key functions.

Firstly, it will help you contextualise your research in what is already known and where the knowledge gaps are, so that you can focus your research more tightly on a research problem which is relevant and significant to your field of study. This includes evaluating your own findings against those of others in your field.

Secondly, it will help understand how to conduct your research: by reading your field's literature, you'll learn what is academically acceptable in terms of the methods and techniques which you can apply to address your research problem.

Reviewing the literature is both time consuming and demanding, and will require you to apply many skills which this handbook will help you develop.

\subsubsection{Setting your aim and objectives}
While your research problem helps you establish the contribution to knowledge of your research, your aim and objectives will help you establish the scope and boundaries of your project by stating the specific way in which your research will address the problem.

Scoping your project appropriately is necessary in all research, but is particularly critical at Masters level with its many constraints on the time you have and the methods you can feasibly apply.

\subsubsection{Developing the research design}
Your research design will summarise, explain and justify how your research is conducted for the benefit of your examiners and other readers\footnote{Although we all want to think differently, it is often the case that the examiners are the only people outside of your supervisory relationship that will read your dissertation. We'll talk later about how you can get your family involved -- which may double your readership!}. As with your literature review, it will develop during your project: at the start, it will be a collection of your initial ideas; by the end, it will be an account of what you did.

Your research design will depend on many things. There are some obvious ones, including the type of research problem you are trying to address, the intended outcome of your research, the sort of evidence you will need, and the research strategies and methods that are acceptable within your chosen discipline.

There are others that are less obvious: you'll have to tailor your research design to the resources and expertise you easily have access to, and your preferred research style\footnote{You may have heard of a \emph{learning style}. A research style is similar -- you may prefer, for instance, to talk to people about the practical ways they work as opposed to building theories of how they do so.}. As a key participant in your own research, your own personal views and values will also affect your choices while developing your research design.

There are also some esoteric ones: some philosophical beliefs which are shared by researchers in your field in regard of how knowledge can be generated. Although fascinating, to really understand them well takes decades of study of others' writings, on top of which there will be hours on hours of conversations with other researchers. Clearly, this is not something you will have much time to dwell upon in your Masters project --- you have a supervisor that can advice you, instead.

Research design is a field of study in its own right, one which has grown out of the many different academic traditions and ways of thinking across academic disciplines and subject areas. It's not easy to digest and is far from stable or complete: every so often a new book on research methods will be offered for review by a publisher. Research design is possibly one of the most challenging aspects of doing academic research and can be puzzling\footnote{It can sometimes be puzzling for experienced researchers too!} for those just starting.

Because of this, you should rely on your supervisor for advice, particularly in your initial research design, although your understanding will mature during your project. This handbook will help you develop such an understanding and assess which choices are most appropriate for your project.

\subsubsection{Gathering and analysing evidence}
This is where you use your research design to gather and analyse the data and evidence\footnote{Data is raw information with no interpretation attached, while evidence is information interpreted to support your academic arguments. The two are closely linked, with data forming the basis of evidence, so that they are often used interchangeably. We will return to this topic in Section ??} you need for your project. This is possibly the most exciting part of conducting academic research!

You're going to spend a lot of time doing this, and you'll have to circle back\footnote{One PhD student we know got to within 2 months of finishing their PhD when they realised that their research design led them to incorrect data analysis. That was a real critical moment in their studies, but they got it done in the end. Just.} to this activity as you learn more about your field and as your research develops.

\subsubsection{Interpreting and evaluating findings}
This is where you review your findings critically to establish the extent your research has met its stated aim and objectives. Because it relies on the way you interpret your evidence, it is closely related to and typically influences evidence gathering and analysis, so that you will iterate between these two research activities for a large proportion of your project.

\subsubsection{Reporting}
Throughout your project you will need to report in a critical\footnote{Developing your critical voice is, er, critical to being a successful researcher. On a good capstone project, an effective critical voice will be highly valued by your examiners.} and rigorous manner on all aspects of your research, including both new insights and the process you followed to arrived at them. Therefore, integral to your reporting is a critical appraisal of the strengths and limitations of the research you have conducted, the overall conclusions which can be drawn from it and what the impact on future research might be.

While your course may only require you to submit a final dissertation for assessment, it is essential that you write up reports throughout your project. This will allow you to improve your academic writing skills, share what you have done with your supervisor for feedback and formative assessment, and develop your dissertation incrementally.

\subsubsection{Reflecting}
Reflection is what makes your learning more effective, and relevant and useful to your own practice. Reflection is important in any kind of learning, but particularly in the experiential learning\footnote{Experiential learning is `learning by doing,' using reflection to think critically about what you have experienced and relate it to knowledge and how it can apply to new situations.} of conducting research.

Our framework therefore emphasises reflection as an essential activity within the research process, to develop and consolidate the knowledge and skills you need to be a competent and confident researchers.

\subsubsection{Planning work}
Some researchers will have worked much of their lives on one problem. You don't have that luxury -- or burden, depending on your perspective!

Time-bound research, like that you are just about to start, must be planned carefully as early mistakes and missteps can be very difficult to correct later.

For this reason, we recommend that you build a \emph{work plan}. At its core, a work plan summarises all activities and work to be conducted to complete the research, how to organise and execute them in the time-span of the project and which milestones must be met.

Standard project management techniques apply, but with the caveat that in academic research you are usually planning for the unknown, so not everything can be figured out upfront and some level of adaptation will be needed. Hence, understanding what might change, related risk and how to manage it is an essential part of  planning your research project.

\subsubsection{Managing risk}
Risk captures the likelihood of something going wrong combined with the impact that will have on your project, both on time, resources and outcome. In theory, both positive and negative impacts can be assessed, but very often the focus is on what can affect your project in a bad way. It is essential that you make an assessment of risk at the start of your project and identify ways to manage it, then monitor and adjust as you go along.

\subsubsection{How the activities relate to each other}
Figure~\ref{pastedgraphic} depicts the relations between activities in the research process. Note that reflecting, reporting, work planning and risk management apply to both individual activities, and the process overall, which is why in the figure they are separated from the other activities.

\begin{figure}[htbp]
\centering
\includegraphics[width=898pt,height=474pt]{PastedGraphic.pdf}
\caption{Research activities and their relations}
\label{pastedgraphic}
\end{figure}

The main relations between activities are as follows (with reference to the numbering in the figure):

1	The research problem helps you identify relevant literature to review

2	The literature you review helps you identify current knowledge and existing gaps and frame the research problem

3	The research problem informs what the aim and objectives should be

4	The research problem identifies the phenomena under study, which inform your choice of the research strategy and methods within your research design

5	The literature also helps you establish which research strategy and methods to apply and how

6	The objectives inform the research strategy and methods to discharge them

7	The research designs tells you how to gather and analyse evidence

8	The analysis of evidence feeds your interpretation and evaluation

9	Your interpretation and evaluation may require further evidence gathering and analysis

10	Your interpretation and evaluation assess the extent aim and objectives are discharged

11	Aim and objectives gives you criteria for your evaluation and interpretation

12	Activities feed your project plan

13 Your work plan inform what you have to do and when

14 Activities give rise to risk to be managed

15 Managing risk will constrain the way you carry out activities

16 Activities and their outputs must be documented in your reports

17 Your reflection on all you do will trigger changes and adjustments to all your activities

The figure highlights the interconnected nature of everything you do during a research project, and how the need to revisit some activities, perhaps as the result of reflection, will trigger adjustments to others which are closely related, so that there is no linear path through that all projects will follow. The figure also does not assume there is a set starting point, although, as you will see in the next chapter, identifying the research problem from an initial topic is the way we recommend you get your project started. Equally, research can go on forever as reflecting of what you know may trigger new avenues of enquiry to follow. Of course, in the context of a Masters project you only have a set amount of time to complete your work, so that at some point you will need to decide that you have done enough and finalise your project dissertation.

\section{The 5-stage framework for your research}
This handbook recommends you organise your research project into five major stages. We've developed our 5-stage framework from our experience of working with hundreds of successful\footnote{And a few, a very few, unsuccessful ones{\ldots}} Masters research students, our knowledge of the iterative, incremental and adaptive nature of the research process, and our awareness of the many risks that must be managed to be successful.

Given this, the stages have the following characteristics:

\begin{itemize}
\item Each stage contains many interrelated research activities\footnote{Don`t worry, we're going to go into exhaustive detail about this the remainder of the book!}, so that you will often revisit earlier activities as you learn more about research and move forward in your project. From one stage to the next, however, the balance between those activities will change reflecting your increasing expertise in them and the progress you have made

\item Each stage builds incrementally and adaptively on work from the previous stage(s)

\item Each stage includes critical reflection on what you have achieved and learned, and how this should inform the work ahead

\item Each stage includes some report writing so that your dissertation builds with your expertise

\item Each stage includes re-assessing risk and adapting your work plan.

\end{itemize}

Because it sets up the whole research project, in Stage 1, alongside developing a deeper knowledge of the research process and its activities, you will do lots of prep, including identifying your project's scope and contribution to knowledge, doing an initial risk assessment, outlining an initial work plan, and setting up the relationship with your supervisor.

Given a successful set up of your research in Stage 1, from Stage 2 onwards you'll spend more time on the specifics of your research and less on the process\footnote{Which will probably come as welcome relief:)}. Even so, in each stage, you will need to review your project risk and progress, adjusting your project plan accordingly.

The 5 stages are designed to balance the activities you need to carry out in your project; each is allocated a recommended proportion of the overall project time. The exact balance is influenced by many factors, including topic, any previous research experience and your supervisor's advice, so take the next figure with a pinch of salt -- it's only meant to show how the focus on research activities changes in the stages. We will consider a more detailed description in the next section, including indicative timings, but again, nothing is written in stone.

\begin{figure}[htbp]
\centering
\includegraphics[width=868pt,height=564pt]{Screenshot2023-06-01at124221.pdf}
\caption{Change of focus between research activities across the project stages}
\label{screenshot2023-06-01at124221}
\end{figure}

\begin{question}[subtitle={Activity: Considering the stages and activities in the framework }] Consider Figure 1 carefully, taking notice of which activities are more prominent in each stage, and the relative length of each stage within the framework. Jot down a timeline for your project based on your expected study length in weeks, identifying when each stage starts and ends.

\begin{solution}Some activities are more prominent in the early stages, specifically identifying the research problem and reviewing the literature. Others become prominent in later stages, particularly gathering and analysing evidence, and interpreting and evaluating findings. Others still feature throughout the project: this is particularly the case for reflecting and reporting.

Your timeline will be specific to your project, of course. Say, you have 48 weeks in total. Then, Stage 1 will take 15\% of 48 weeks, or just over 7 weeks, Stage 2 will take the same length, Stage 3 will take 20\% of 48 weeks, or just under 10 weeks, etc.

\end{solution}\end{question}
%%Hack to correct tcbox behaviour
\color{black}

\subsubsection{Breakdown of research activities by stage}
Table 1 gives you the recommended breakdown of activities between and within stages.

This is designed to help organise and plan your project work to match the requirements of your specific Masters course, something you will consider as part of work planning in the next chapter, which focuses on Stage 1.

Table 1 also provides recommendations on critical aspects of your work you should discuss with your supervisor stage by stage. We have already discussed the importance to your project success of a close working relationship with your supervisor, of engaging in an active dialogue on a regular basis, and your responsibility to drive and manage that interaction. In practice, your supervisor will only have limited time to dedicate to your project, therefore it is essential you make an effective use of that time: the advice in the table will help you guide your interaction with your supervisor, ask key questions and make the time you spend together more productive.

Table 1 --- Recommended breakdown of project work between and within stages, alongside suggested focus of interaction with supervisor

\begin{ltabulary}{1.4\textwidth}{@{}LLLLLLLLLLLLL@{}}
\textbf{Research activities} & \textbf{Stage 1~} & \textbf{Suggested focus of your interaction with your supervisor} & \textbf{Stage 2~} & \textbf{Suggested focus of your interaction with your supervisor} & \textbf{Stage 3~} & \textbf{Suggested focus of your interaction with your supervisor} & \textbf{Stage 4} & \textbf{Suggested focus of your interaction with your supervisor} & \textbf{Stage 5} & \textbf{Suggested focus of your interaction with your supervisor} \\
 &  & \textbf{Identifying the research problem} & Develop problem statement and intended contribution to knowledge & Suitability of research problem for academic research and to meet the requirements of specific~ course & Adjust, if needed &

 & Adjust, if needed &

 & Adjust, if needed &

 & Adjust, if needed &

 \\
 &  & \textbf{Reviewing the literature} & Compile initial draft of literature review and plan remaining review & Scope of literature review and possible gaps & Complete full draft of literature review
Draft critical summary of key insights from literature review & Suitability of literature review structure and narrative flow
Appropriate logical argumentation
Critical thinking in driving insights & Adjust, if needed &

 & Adjust, if needed &

 & Adjust, if needed &

 \\
 &  & \textbf{Setting research aim and objectives} & Define aim and~ objectives & Suitability and feasibility of aim and objectives in relation to research problem and project time & Adjust, if needed &

 & Finalise aim and objectives, and define tasks and deliverables & Suitability of tasks and deliverables from objectives & Adjust, if needed &

 & Adjust, if needed &

 \\
 &  & \textbf{Developing the research design} &
Consider elements of~ research design, including data and evidence, and types of research methods

Complete ethics assessment, including applying for permission to proceed, if needed
 &
Consistency of choices in relation to aim and objectives.

Compliance with own university's ethical and legal guidelines
 & Revise research design, with detailed consideration of data and evidence, research strategy and research methods & Any further adjustment needed to the research design & Complete research design, with detailed consideration of data and evidence, research strategy, research methods and procedures & Suitability of research procedures & Adjust, if needed &

 & Adjust, if needed &

 \\
 &  & \textbf{Gathering and analysing evidence} & n/a &

 & n/a &

 & Conduct pilot work to test aspects of your research design & Scope of your pilot work & Conduct initial data/evidence generation and analysis & Initial application of collection and analysis methods, and any improvements required~ & Complete data/evidence generation and analysis & Overall quality and quantity of data/evidence and their analysis \\
 &  & \textbf{Interpreting and evaluating findings} & n/a &

 & n/a &

 & n/a &

 & Critically assess findings up to this point & Critical thinking in assessing findings, and any improvements required~ & Critically assess all findings and the overall research conducted & Overall critical thinking in assessing findings \\
 &  & \textbf{Reporting} & Write up Stage 1 report & Demonstration of critical thinking and good academic writing, and any improvements required & Write up Stage 2 report & Any further improvements required & Write up Stage 3 report & Any further improvements required & Write up Stage 4 report & Any further improvements required & Write up dissertation & Overall quality of dissertation \\
 &  & \textbf{Reflecting} & At stage end, think critically about experiential learning in relation to the research process and its activities &

 & At stage end, think critically about experiential learning in relation to the research process and its activities &

 & At stage end, think critically about experiential learning in relation to the research process and its activities &

 & At stage end, think critically about experiential learning in relation to the research process and its activities &

 & At stage end, think critically about experiential learning in relation to the research process and its activities & Overall critical thinking in assessing experiential learning \\
 &  & \textbf{Planning work} & Draw your initial project timeline and a detailed plan for Stage 1 & Appropriateness of initial work plan & At stage start, plan Stage 2 work in detail, including any revision or additional work needed from the previous stage. & Any major adjustment required & At stage start, plan Stage 3 work in detail, including any revision or additional work needed from the previous stage. & Any major adjustment required & At stage start, plan Stage 4 work in detail, including any revision or additional work needed from the previous stage. & Any major adjustment required & At stage start, plan Stage 5 work in detail, including any revision or additional work needed from the previous stage. & Any major adjustment required \\
 &  & \textbf{Managing risk} & Assess project risk & Consideration of major risk & Review project risk and make any required adjustments for the next stage & Any major adjustment required & Review project risk and make any required adjustments for the next stage & Any major adjustment required & Review project risk and make any required adjustments for the next stage & Any major adjustment required &  &
\end{ltabulary}

\section{Critical success factors}

We conclude this chapter by summarising critical success factors you should keep in mind during your Masters research:

\subsubsection{Making good use of the 5-stage framework}  The framework in this handbook is the result of decades of practice, helping Masters students like you succeeding in they first academic project. However, it is not a straightjacket, and you should adapt it to your own needs. The framework was developed with the novice academic researcher in mind, but it may be that you already have some research experience, in which case some of the the timing suggested by the framework may be too generous. Equally, your own course may require you to submit some interim report as formative assessment, in which case the stages length may need altering to match your course requirements. Nevertheless, the framework provides a scaffolding to help you take control when planning and conducting your project.

\subsubsection{Making good use of this handbook and its activities}  This handbook is designed to accompany you in your first journey into academic research, so you should follow its stage by stage advice to guide your project work. The handbook is also designed to be very practical, so that there are plenty of activities for you to do. The activities are there to help you make steady progress with essential work for your project and to help you develop your research skills: we strongly advise you complete them systematically.

\subsubsection{Your self-drive and commitment} The course you are studying may provide you with some structure to help your progress your project, including, for instance formal assessment points. However, you will be in the driving seat most of the time. You must choose your topic and research problem, how to investigate it, and how to organise your research time in detail; it is also up to you to understand the academic standards required and to assess which research skills you must develop further to meet them. Above all, it is up to you to commit to a sustained effort for the duration of your project, which may be several months.

\subsubsection{Continuous effort} Successful research requires continuity, so that you will need to set aside a sufficient and regular time for your research project every week, ensuring you keep making progress as you go along. Long breaks are incompatible with conducting research: while in your previous studies you may have been able to stop and start, and possibly cram lots of work around assignment deadlines, conducting research requires lots of time for reflection and for ideas to develop and mature, something you can't compress close to a deadline. Conducting research requires endurance, perseverance and continuous effort: it is a marathon, not a sprint!

\subsubsection{Your working relation with your supervisor}  Your supervisor is your strongest study ally, a research expert who can guide and advise you on all aspects of your project, with whom you can discuss your research ideas, and who can assess that you are making sufficient progress at each point in your journey. It is essential that you develop a good working relationship with your supervisor, meet regularly and have an open and honest dialogue throughout your project.

\subsubsection{Thinking and writing}  There is a crucial interaction between reading, thinking, and writing in research: reading informs your thinking; your thinking is what you try to express in your writing; your writing helps you make sense of what you have read, and hence of what you think, and informs more reading, thinking and writing. Because of this, while you will spend a lot of time reading and thinking, it is also essential that you write as you go along. The more regularly you write, the easier it will be for you to develop academic writing skills and the more critical your thinking will be.

\subsubsection{Your dissertation}  It is very likely that your dissertation will be the only way your research is assessed. To be successful, your dissertation must demonstrate that you have mastered a wide range of research skills and can communicate your academic research effectively in writing. In fact, this is usually more important than any feature your proposed solution to your research problem might have. At Masters level, that you `solve' your research problem may not even be necessary! What matters most is your scholarly and critical attitude to each element of your research and your grasp of what academic research entails as demonstrated by your written dissertation.

\subsubsection{Making good use of your study support}  Last but not least, it is important that you assess your progress as you go along, and make the most of the support which is provided to you by your supervisor, and any other adviser who may be made available to you as part of your course. There will be times when you will find working on your project very challenging and may lose confidence in your ability to complete it. Those are the times when it is going to be particularly important for you to reach out and ask for extra help, even if your first instinct may be to hide. All researchers experience such feelings at one time or another, so do not be discouraged: talk to an adviser and work through the difficulties you are having to overcome them.

\section{What to take away from this chapter}
LR - still to do

%We close this chapter with some reflection on what you have learnt.
%
%\begin{question}[subtitle={Activity -- Reflection on learning}] Consider the content of this chapter and write down key things you have learnt or that have surprised you. For each indicate why they are notable or relevant to you, and how you may apply them in new situations or to inform your future learning.

%\begin{guidance}tbd
%\end{guidance}\end{question}
%%Hack to correct tcbox behaviour
\color{black}
\endinput