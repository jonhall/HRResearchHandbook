\chapter{Foreword}
You have just about embarked in your very first research project, possibly at the last component of your Masters studies. That's both exciting and daunting.

Exciting because you will be able to focus on a topic of your own choice and to investigate in depth an issue or problem which is of particular interest to you, either personally or professionally. In doing so, you will acquire a deep knowledge of that topic, conduct a unique and novel research study, and develop and apply a wide range of research skills.

Daunting because your success will depend on you demonstrating your mastering of the topic and of the research process, that you can exercise competently a wide range of research skills, and can communicate your work effectively through an academic dissertation, possibly the largest and most demanding piece of writing you will ever undertake.

The aim of this book is to support you in taking your first steps into academic research. It will provide you with a solid scaffolding for you to become a competent and confident researcher. It will demystify the language around academic research and, through practical advice and activities, will help you plan, manage and execute your project work successfully from start to finish.

But the benefits don't stop with the completion of your project. In succeeding in your research, you will have also gained and demonstrated a wide range of skills which are professionally relevant and valued by employers, from problem solving to effective communication, digital and information literacy, self-management and resilience. These transferable skills will serve you well in your profession and your life, regardless of the path you will take after you've completed your studies.

This book is written primarily with Masters students in mind: those who are taking their first steps into academic research. However, its content is equally valuable to first-year Doctoral students, particularly those who are required to complete an initial research feasibility study as part of their probationary requirements.