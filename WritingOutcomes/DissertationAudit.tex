\begin{ReportTable}{dissertationTemplate}[Dissertation template and guidance]
%%\ReportTitle1
\ReportTitle*
	&Your title should capture succinctly your research problem and aim &\Cref{sect:ChoosingATitle}
\\

%%\ReportAbstract12345
\ReportAbstract*****
	&Your abstract should provide a succinct account of your research for a specialist audience, covering each of the bullet points indicated	&\Cref{sect:writingAbstract}
\\

%%1-\ReportIntroduction12345
\ReportIntroduction****
	&  This chapter should provide an introduction to your research topic in its wider context (as background) and your justification of why the research is worth pursuing. Its purpose is to introduce and justify your intended research in overview, before entering the detailed work of the subsequent chapters. It should be well argued and supported by appropriate citations and other evidence
	
	\item You can use a separate section for key technical definitions and acronyms used throughout your dissertation, for the reader's benefit	
	&\Cref{sect:stage1ChoosingATopic}\\

%%2-\ReportLitRev---
\ReportLitRev***
	& Your literature review should provide a critical account of your in-depth engagement with the academic (and other) relevant literature, including identifying key trends, ideas and knowledge gaps. Most of your citations should point to academic articles. Both coverage and depth of your review matter. 
	
	\item You should ensure your review is well structured, with a logical narrative flow and with arguments well supported by appropriate citations
	
	\item Your critical summary should highlight key insights from your review and the knowledge gap you research is going to address
	
	  &\Cref{sect:stage2literatureReview}\\

%%3-\ReportResearchDef----
\ReportResearchDef****
	& You should ensure that your research problem is well articulated, your aim  consistent with your research problem and broken down into appropriate objectives, and that the intended knowledge contribution is clearly expressed. You must ensure that these elements of your research definition form a coherent whole and clearly relate to each other &\Cref{sect:stage1ResearchProblem,sect:stage1AimAndObjectives}\\

%%4-\ReportResearchDes-----
\ReportResearchDes*****
	& This chapter should demonstrate your critical engagement with all elements of research design, including a detailed account of the data needed in your research, its sources and use, and the research strategy(ies) and methods you have chosen and applied
	
	\item Your choices should be appropriately justified in relation to your research problem, aim and objectives, and possibly with reference to accepted paradigms and approaches in your field of study
	
	\item You should provide a summary of the  procedures you have followed in the application of your research methods, including measures you have taken to deal with potential research weaknesses
	
	\item You should  demonstrate your careful consideration of ethical and regulatory matters relevant to your project, and that your research complies with your course and university requirements 
	
	&\Cref{ch:ResearchDesignFoundation,ch:yourResearchMethodology,ch:DataGenerationMethods,ch:ModellingMethods,ch:EthicsAndRegulations}
	\\

%%5-\ReportAnalysisInterp----
\ReportAnalysisInterp****
	& This chapter should provide a detailed account of your data generation and analysis, the findings you have derived and their interpretation in relation to your research aim and objectives 
	
	\item It should demonstrate a competent execution of your methodology, including providing appropriate summaries of your data, a clear account of your analysis, and how it led to the key findings in a logical manner 
		
	\item Your interpretation of findings in relation to aim and objectives should demonstrate in-depth critical reflection  
	&\Cref{ch:AnalysisMethods,sect:interpretingFindings}\\

%%6-\ReportEvalConc-------
\ReportEvalConc*******
	& This chapter should demonstrate good critical reflection on the extent your research has met its stated aim and objectives. In a succinct way, your conclusions should bring all your findings together, both from your literature review and your data generation, analysis and interpretation. 
	
	\item You should reflect on how your research has contributed new knowledge in relation to related published work, including highlighting how it is novel 
	
	\item You should evaluate your research in terms of its validity, reliability and lack of bias, highlighting measures you have taken to avoid research weaknesses, but also acknowledging limitations
	
	\item  You should discuss possible implications of your work for further research and, if applicable, for professional practice 
	
	\item Your concluding reflexive statement should highlight what you have learnt during your project from a personal standpoint in relation to thinking and behaving as an academic researcher
 &\Cref{sect:evaluateWholeResearch}\\

%%7-\ReportRefs-
\ReportRefs*
	& You references should be accurate and complete in relation to citations in the body of the dissertation &\Cref{ssect:Processing}\\

%%A-\ReportAppendices-
\ReportAppendices*
	& Appendices can be used to include supplementary material in support of the main body of your dissertation, for instance sample raw data, questionnaires used in data generation, programme code for models or artefacts, detailed calculations, etc. &\\
	
%%\ReportProgressTracking123456789
%\ReportProgressTracking*********
%& You should consider your progress against the checklist of \Cref{Stage5checklist} and record it here \Cref{stage5progress}
%\\
%%
%%%\ReportReflection123
%\ReportReflection***
% & In this section you should reflect on the progress you have made in Stage 5 using the material in \Cref{Stage5workplan}.
%\\
\end{ReportTable}
\endinput