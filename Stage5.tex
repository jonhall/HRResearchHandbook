\chapter*{Stage 5: Completing your dissertation}

\section{Introducing Stage 5}
Stage 5 concludes your research project, so that by the end of this stage you will have complete your project and written up your full dissertation, ready for submission.

With reference to our 5-stage framework, the activities which are in focus in Stage 5 are summarised in Table 1, which also provides some guidelines for your interaction with your supervisor during this stage.

Table 1 - Stage 5 activities

\begin{table}[htbp]
\begin{minipage}{\linewidth}
\setlength{\tymax}{0.5\linewidth}
\centering
\small
\begin{tabulary}{\textwidth}{@{}llll@{}} \toprule
 \textbf{Research activities} & \textbf{Stage 5} \textbf{(30\% of project length)} & \textbf{Effort within stage} & \textbf{Suggested focus of your interaction with your supervisor} \\
\midrule

 \textbf{Identifying the research problem} & Adjust, if needed & 1\% & \\
 \textbf{Reviewing the literature} & Adjust, if needed & 1\% & \\
 \textbf{Setting research aim and objectives} & Adjust, if needed & 1\% & \\
 \textbf{Choosing the research design} & Adjust, if needed & 1\% & \\
 \textbf{Gathering and analysing evidence} & Complete data\slash evidence generation and analysis & 40\% & Overall quality and quantity of data\slash evidence and their analysis \\
 \textbf{Interpreting and evaluating findings} & Critically assess all findings & 20\% & Overall critical thinking in assessing findings \\
 \textbf{Reporting, critical reflection and conclusions} & Assess entire research and write up dissertation & 35\% & Overall quality of dissertation \\
 \textbf{Work planning and risk management} & At stage start, review work from previous stage and project risk; adjust plan as needed If you have received feedback from supervisor on your previous stage work, adjust plan to include any revision recommended & 1\% & \\
\bottomrule

\end{tabulary}
\end{minipage}
\end{table}

\begin{question}[subtitle={Activity: Understanding the effort needed in this stage}] Consider Table 1 carefully, taking notice of the entries in the `Effort within stage' column. Make a note of the activities which are most prominent in this stage and what is expected under each.

\begin{solution}In this stage too, gathering and analysing evidence will constitute your major effort (40\% of study time), although considerable effort will also be needed in assessing your research overall and completing your dissertation. In particular you shouldn't underestimate the time needed to complete and polish the dissertation so that is ready for submission, which is why the framework assume a significant effort in this stage.
\end{solution}\end{question}
%%Hack to correct tcbox behaviour
\color{black}

\section{Completing your research}
Building on Stage 4, in this stage you will complete your work on gathering and analysing evidence, and presenting and evaluating your findings in your dissertation.

\begin{question}[subtitle={Activity: Completing your data and evidence gathering and analysis, and interpretation of findings}] Complete your research on gathering and analysing data and evidence, and the interpretation of your findings in terms of your aim and objectives. Expand on your analysis and summaries from your Stage 4 report.

\begin{guidance}Ensure you continue to manage your raw data and evidence carefully, and that your report presents all your data\slash evidence, findings and their interpretation in a clear and rigorous manner.

This activity is likely to take up to 40\% of your study time, assuming you were able to complete the bulk of your data and evidence collection and analysis in Stage 4. If that's not the case, you should increase your effort accordingly.
\end{guidance}\end{question}
%%Hack to correct tcbox behaviour
\color{black}

\section{Assessing your research}
Once you have completed your work on gathering and analysing data and evidence, and interpreting your findings, it is time for you to reflect on your whole project and draw some overall conclusions. These will form the body of the concluding chapter of your dissertation, for which you are asked to think critically about each of the following:

\begin{itemize}
\item \textbf{Evaluation against aim and objectives}: you should reflect on the extent your research has met its stated aim and objectives. The interpretation of your findings against aim and objectives is a good starting point to draw these summary conclusions. While your interpretation may be deep and detailed, with reference to specific data and evidence, here you are expected to highlight key conclusions based on such an interpretation. It is not necessary for your research to have met your aim and objectives fully: in this section you need to make a critical assessment of what your research has actually achieved.

\item \textbf{Evaluation against the academic body of knowledge}: this requires you to assess the extent your findings have added to the body of knowledge in your field of study, including whether they support or question findings already known from the literature you have reviewed. You should show awareness of how your own research relates to the wider academic context.

\item \textbf{Implications for practice (if any):} here you should reflect on ways in which your research may be relevant to professional practice, if applicable, including how it could lead to change and improvement. If your research is purely theoretical, then you can skip this section, and focus on the previous two items instead.

\item \textbf{Validity of the research}: this require you to assess your research in terms of construct, internal and external validity. You should refer back to Stage 3 materials to refresh your understanding of validity.

\item \textbf{Further research}: your research may have shed light on aspects of your research problem, or highlighted other related research problems, which you did not have the time to explore in your project. This is the place for you to discuss those of more relevance and to indicate how future research can build on the work you have done.

\item \textbf{Personal reflection on your research experience}: whether or not your research project is your first experience of academic research, you should reflect on what you have learnt from a personal standpoint in relation to thinking and behaving like an academic researcher. You should address how your mindset and skills have changed, or how you would do things differently should you start anew, and any other relevant thoughts you may have.

\end{itemize}

\begin{question}[subtitle={Activity: Assessing your research overall}] Assess your overall research in relation to the above points, and write appropriate summaries of each for inclusion in your dissertation.

\begin{guidance}For each point above, consider the related guidance to help you assess your research overall. Note that this assessment should consider all the work you have conducted in your project.
\end{guidance}\end{question}
%%Hack to correct tcbox behaviour
\color{black}

\section{Finalising and submitting your dissertation}

\subsubsection{Compiling a full draft of your dissertation}
At this point you should have all the materials you need to complete a full draft of your dissertation, which is already a remarkable achievement!

Your dissertation constitution your final report, extending your Stage 4 and covering the work you have carried on in this stage. Its recommended structure and content are indicated in Table 1.

Table 1 -- Report structure and content guidance

\begin{table}[htbp]
\begin{minipage}{\linewidth}
\setlength{\tymax}{0.5\linewidth}
\centering
\small
\begin{tabulary}{\textwidth}{@{}ll@{}} \toprule
 \textbf{Dissertation template} & \textbf{Links to study materials} \\
\midrule

 Proposed title & Your title should continue to capture succinctly research problem and aim \\
 Abstract & This should provide a succinct summary of your research aimed at a specialised audience \\
 Sect 1 - Introduction 1.1 Background to the research 1.2 Justification for the research & This section should provide an introduction to your research topic in its wider context (as background) and your justification of why the research is worth pursuing. It should be well articulated and supported by evidence \\
 Sect 2 - Literature review 2.1 Review of existing relevant knowledge 2.2 Critical summary, including knowledge gap to be addressed by the research & Your review should provide a critical account of your in-depth engagement with the academic (and other) relevant literature, including identifying key trends, ideas and possible knowledge gaps. Most of your citations should point to academic articles. Your critical summary should highlight key insights from your review and provide a strong justification for your proposed research. Both coverage and depth of your review matter. You should ensure that your review is well structured, with a logical narrative flow and your arguments are well supported by evidence \\
 Sect 3 - Research definition 3.1 Problem statement 3.2 Aim, objectives, tasks and deliverables 3.3 Knowledge contribution & You should ensure that your research problem is well articulated and appropriate for your course and your personal and professional circumstances, that your aim and objectives are consistent with research problem, that tasks and deliverables break down your objectives appropriately and are clearly related to your chosen research methods, and that the intended knowledge contribution of your research is clearly articulated \\
 Sect 4 - Research design 4.1 Evidence and data 4.2 Research strategy and methods 4.3 Research procedures 4.4 Ethical, legal and EDI considerations & This section should demonstrate your critical engagement with all elements of research design, including a detailed account of the data and evidence needed in your research, the research methods and research strategies you will to apply, and how you will apply them within your project. Your account should be supported by a clear rationale and insights from the related literature, and appropriately justified in relation to your research problem, aim and objectives. It should also demonstrate your careful consideration of ethical and legal matters, and that your research will comply with your course and university requirements \\
 Sect 5 - Analysis and interpretation 5.1 Summary and analysis of evidence 5.2 Summary of key findings 5.3 Interpretation in relation to aim and objectives & This section should provide a comprehensive account of all the data and evidence you have gathered, appropriately analysed with key findings identified and interpreted in relation to aim and objectives. It should demonstrate that a competent execution of your research design, present appropriate summaries of evidence and data, supported by raw data in an appendix if needed. Key findings should be clearly identified and logically connected to evidence, with good critical reflection on their implications for aim and objectives. \\
 Chapter 6 - Evaluation and conclusion 6.1 Evaluation against aim and objectives 6.2 Evaluation against related work in the literature 6.3 Implication for practice 6.4 Validity of the research 6.5 Further work 6.6 Personal reflection on your experience of & In the section you should reflect on the extent your research has met its stated aim and objectives, bringing together all your findings from both primary and secondary research work. You should also reflect how it has contributed new knowledge in relation to the literature you have reviewed. You should also assess the validity of your research and consider any implication for further research and, if applicable, for professional practice. Considerable insight is evident in the implications of the research identified for relevant stakeholder groups. You should also reflect on what you have learnt from a personal standpoint in relation to thinking and behaving as an academic researcher. \\
 References & You should include all your references and ensure you apply the required bibliographical style consistently. Ideally, you should use a BMT to generate and integrate your references within your report \\
 Appendix - Extended abstract & If needed, you should include your draft extended abstract as an appendix. This should provide a structured summary of your research aimed at a generalist audience. \\
 Appendix - Raw evidence & If relevant, you should include a sample of your raw data as an appendix \\
\bottomrule

\end{tabulary}
\end{minipage}
\end{table}

\begin{question}[subtitle={Activity: Putting your dissertation together}] Using your word processor of choice, and starting from your previous report, complete your dissertation by applying the structure and guidance in Table 1, and making good use of your notes and summaries from all related activities you have carried out.

\begin{guidance}Although the dissertation structure and content we recommend is fairly standard, it is possible that they don't not match exactly the requirements of your own course, which may provide a different template for you to follow. Indeed you should check and apply your course guidance, and map the structure and content in Table 1 to what is required in your course of study.
\end{guidance}\end{question}
%%Hack to correct tcbox behaviour
\color{black}

\subsubsection{Revising your draft for compliance to requirements}

Now that you have a complete draft of your dissertation, you should revise it to ensure it meets your course requirements.

In our experience, a Masters dissertation is usually in the range of 10,000 to 15,000 words. Often, references, abstract, extended abstract and all other appendices are excluded from the word count, but figure and table captions are included.

There is also an expectation that the content of your dissertation is balanced across the different chapters, although it is normal for some chapters to be more substantial than others. The recommended distribution of content across the full body of your dissertation, based on our recommended dissertation structure, is indicated in Table 1, as a percentage of total. This is not a hard and fast constant, but can provide a baseline for you to get an idea of the relative weight of the different chapters of your dissertation. In adapting it to the needs of your own project and course, however, you should ensure you maintain a good balance across the whole piece.

Table 1 --- Word count: recommended breakdown of content

\begin{table}[htbp]
\begin{minipage}{\linewidth}
\setlength{\tymax}{0.5\linewidth}
\centering
\small
\begin{tabulary}{\textwidth}{@{}lllll@{}} \toprule
 \textbf{Element} & \textbf{Breakdown} & \textbf{Recommended word count distribution} & \textbf{Equivalent for 10,000 word dissertation} & \textbf{Equivalent for 15,000 word dissertation} \\
\midrule

 \textbf{Chapter 1} \textbf{Introduction} & Background to the research Justification for the research Definitions (if any) Dissertation outline & 10\% & 1000 & 1500 \\
 \textbf{Chapter 2} \textbf{Literature review} & Review of existing relevant knowledge Critical summary, including knowledge gap & 20\% & 2000 & 3000 \\
 \textbf{Chapter 3 Research definition} & Problem statement Aim, objectives, tasks and deliverables Knowledge contribution & 10\% & 1000 & 1500 \\
 \textbf{Chapter 4 Research design} & Evidence and data Research strategy and methods Procedures Ethical considerations & 15\% & 1500 & 2250 \\
 \textbf{Chapter 5 Analysis and interpretation} & Summary and analysis of evidence Summary of key findings Interpretation in relation to aim and objectives & 30\% & 3000 & 4500 \\
 \textbf{Chapter 6 Evaluation and conclusion} & Evaluation against aim and objectives Evaluation against the academic body of knowledge Implications for practice (if any) Validity of the research Further research Personal reflection on your research experience & 15\% & 1500 & 2250 \\
\bottomrule

\end{tabulary}
\end{minipage}
\end{table}

There is also an expectations that your dissertation conforms to some standard presentation conventions, which we have summarised in Table 2.

Table 2 --- Standard presentation conventions for Masters dissertation

\begin{table}[htbp]
\begin{minipage}{\linewidth}
\setlength{\tymax}{0.5\linewidth}
\centering
\small
\begin{tabulary}{\textwidth}{@{}ll@{}} \toprule
 \textbf{Fonts} & Use a standard font that is easy to read, e.g. Times New Roman or Arial, with font size 11 or 12 \\
\midrule

 \textbf{Margins and spacing} & Leave appropriate margins on both the left and the right of the page, typically around 2 cm. Use 1.5 line spacing \\
 \textbf{You identifiers} & Include your name and student identifier, possibly as a header or as part of the title page \\
 \textbf{Title page} & Include a title page containing your research title. Usually the following statement is also required: ``A dissertation submitted in partial fulfilment of the requirements for the degree of $<$name of degree$>$'', where you should replace $<$name of degree$>$ with your own degree title \\
 \textbf{Table of content} & Include a table of content after the title page \\
 \textbf{Page numbers} & Number all pages, including references and appendices. In particular, use lower-case Roman numerals on the preliminary pages -- iii, iv, v, etc. -- and Arabic numerals starting from page 1 at the beginning of Chapter 1. \\
 \textbf{Chapter and section numbering} & Number chapters sequentially using Arabic numerals starting with 1. Number sections sequentially starting with the chapter number, e.g. 1.1, 1.2, etc. for sections in Chapter 1. Number sub-sections sequentially starting with the section number, e.g. 1.1.1, 1.1.2, etc. for sub-sections in Section 1.1. You should avoid sub-sub-sections, but if needed, number them sequentially starting with the sub-section number, e.g. 1.1.1.1, 1.1.1.2, etc. for sub-sub-sections in Sub-section 1.1.1. \\
 \textbf{Figures and tables} & Number all figures and tables sequentially, starting with their chapter number, e.g. 1.1, 1.2, etc. for figures in Chapter 1. Include appropriate captions positioned after figures and before tables \\
 \textbf{Lists of figures and tables} & List all figures and tables after your table of content. For each include both their number and caption \\
 \textbf{Citations and references} & Apply the required bibliographical style throughout \\
 \textbf{Verb tense} & Your dissertation is an account of what you did in your project, so you should report your work using the past tense throughout \\
\bottomrule

\end{tabulary}
\end{minipage}
\end{table}

\begin{question}[subtitle={Activity: Reviewing your dissertation}] Review you current dissertation draft and make all necessary adjustments to ensure it meets the guidance and requirements above.

\begin{guidance}You should ensure that your dissertation draft fits within the overall word count, and matches the suggested relative weight of each chapter.

You should also ensure you have applied the required presentation conventions.

You should check and apply any further guidance from your course of study.

\end{guidance}\end{question}
%%Hack to correct tcbox behaviour
\color{black}

\subsubsection{Final check and submission}

Before submitting your dissertation, you should perform a final check, focusing on the following aspects:

- \textbf{Logical coherence}: you should ensure that all research elements of your dissertations are coherent and consistent with each other, so that there is a logical progression from research problem, to aim and objectives, to research design and its execution, to findings and conclusions.

- \textbf{Academic writing}: you should ensure that academic arguments are well formed, including being well-supported by secondary and\slash or primary evidence, that the language you use is clear and precise, and there is a good balance between description and critical reflection.

- \textbf{Proof-reading}: you should remove grammatical errors and typos, and ensure that punctuation is correct. You should also check that the narrative makes sense to the reader, for which I strongly advise you ask for help from a friend or family member: even if they are not experts on the topic of your project, they should be able to follow what you have written and get the gist of your work.

- \textbf{Conformance to presentation conventions}: you should ensure that your dissertation conforms to the the requirements of your course, follows its presentation conventions, its length is within the word limit, and its content is well balanced between chapters.

\begin{question}[subtitle={Activity: Performing your final check }] Assess your dissertation draft against each of the points above. Revise and iterate until you are ready to submit.

\begin{guidance}Revising your dissertation for submission is very important as you can lose a substantial proportion of marks should any of these aspects not be addressed carefully and to the expected standards.
\end{guidance}\end{question}
%%Hack to correct tcbox behaviour
\color{black}

\paragraph{}
You should now be ready to submit your dissertation. You should, of course, follow the instructions for your course of study to do so.

\subsubsection{How your dissertation will be assessed}
After submission, your dissertation will go through your university's assessment process, which is designed to ensure that your work is assessed fairly against Masters research benchmarks and your course learning outcomes. The specifics of this process will depend on your own university and course of study, something you should investigate carefully.

You should also investigate the assessment criteria applied to your work. Typically, your Masters dissertation will be assessed from the following perspectives, although the specific marking scheme applied within your course may break each further:

\begin{itemize}
\item \textbf{Research definition and research design}: this refers to an appropriate articulation and justification of the research problem in its wider context, including your critical review of the academic literature to contextualise and justify your research problem and knowledge contribution, a well developed and justified research design, and well constructed academic arguments

\item \textbf{Evidence gathering, analysis, interpretation and conclusion}: this refers to a competent execution of your research design, an adequate amount of evidence gathered and analysed, an appropriate interpretation of your findings, and a critical evaluation of your research overall

\item \textbf{Presentation}: this refers to how your dissertation is put together, its cohesiveness and logical flow, including abstract and extended abstract, and its conformance to conventions, including an appropriate use of tables, figures and diagrams to summarise and present your work

\end{itemize}

The assessment of your work under these perspectives will contribute to your final grade, which will be established by your examiners in relation to Masters level quality benchmarks, like those summarised in Table 1, which are typical in the UK.

Table 1 --- Typical grade benchmarks for Masters dissertations, based on UK quality standards

\begin{table}[htbp]
\begin{minipage}{\linewidth}
\setlength{\tymax}{0.5\linewidth}
\centering
\small
\begin{tabulary}{\textwidth}{@{}ll@{}} \toprule
 \textbf{Grade} & \textbf{Quality descriptor} \\
\midrule

 \textbf{Distinction} & All elements of the dissertation are present, including Abstract and Extended Abstract, and are of a high standard. In particular, the dissertation demonstrates: * advanced, authoritative understanding and analysis of key issues and complex problems * strong evidence of a critical approach to own work and that of others * competent use of a wide range of evidence in support of academic arguments * appropriate and well justified selection of research strategies and methods, applied competently to own research * originality and independence of thought * compelling narrative which is coherently and logically presented * excellent presentation standards * excellent research potential \\
 \textbf{Merit} & All elements of the dissertation are present, including Abstract and Extended Abstract, and are of a good standard. In particular, the dissertation demonstrates: * good understanding and analysis of key issues * good evidence of a critical approach to own work and that of others * good use of evidence in support of academic arguments * appropriate selection of research strategies and methods, applied reasonably well to own research * some originality * coherent and logically presented narrative * good presentation standards * good research potential \\
 \textbf{Pass} & Some elements may be weak or missing, but all thresholds are reached. In particular, the dissertation demonstrates some of: * limited understanding and analysis of key issues * limited evidence of critical approach to own work and that of others * limited use of relevant evidence in support of academic arguments * some appropriate choices of research strategies and methods, with limited application to own research * plausible narrative * adequate standards of presentation \\
 \textbf{Fail with possibility of resubmission} & Many elements are very weak or missing, and not all thresholds are reached. In particular, the dissertation demonstrates many or all of: * superficial understanding and analysis of key issues * weak evidence of critical approach to own work and that of others * gaps in the use of evidence in support of academic arguments * inappropriate choice or application of research strategies and methods * weak narrative * poor standards of presentation \\
 \textbf{Fail} & The dissertation has critical flaws and omissions, so that is not recoverable via a resubmission. In particular, the dissertation demonstrate many or all of: * lack of understanding and analysis of key issues * lack of critical approach to own work and that of others * little or no use of evidence in support of academic arguments * inappropriate choice or application of research strategies and methods * incoherent and confused narrative * inadequate standards of presentation \\
\bottomrule

\end{tabulary}
\end{minipage}
\end{table}

\begin{question}[subtitle={Activity: Assessing your own dissertation}] Apply the three perspectives above together with the benchmarks of Table 1 to your dissertation. Write down your assessment of your work as a result.

\begin{guidance}Your course of study may provide some detailed guidance on how your dissertation will be assessed. If that's the case, you should compare that guidance to the advice in this handbook, and apply it in your own assessment of your dissertation. You should only assess the content of the dissertation as is, disregarding all other knowledge you will have of your research which is not reported.

You should take an objective stance, considering both strengths and weaknesses of your work. You could also ask a friend or a family member to assess your dissertation, then compare their assessment with yours.

\end{guidance}\end{question}
%%Hack to correct tcbox behaviour
\color{black}


\section{Reflection: Stage 5}

%%More here

%%Repeated reflection activity
%%Repeated Activity for all reflections
\begin{question}[subtitle={Activity}]
$<$Needs assessing for content and structuring into activity + guidance$>$

This activity has four parts: the first is something you should be doing regularly, but won't make you into a disobedient or indocile thinker. The second, third and fourth may help you get started and keep going.

Part 1: Think about your study this far -- using this book and anything you've done for your dissertation in parallel -- as a journey. More from elsewhere, including   !!.

Part 2: think about yourself and the way you think. How does your desk look? Is it messy or tidy? Do the same for your computer desktop. Is it empty or are there hundreds of files strewn across it? Do you think your tidiness or untidiness will affect the way you do your research? How about how you keep your -- critically important -- bibliographic database which may contain up to a hundred academic\footnote{It's not unknown to have more than a hundred.} and other articles by the time you're finished?

Part 3: think about the context of your research. Which professional pressures are there on you to succeed in solving your research problem? Pressures could come in many forms: financial -- there's a promotion for you at the end of it; peer -- your colleagues know that you are studying will have good expectations of your result and you'll want to prove them right\footnote{Or wrong, depending on the colleague!}. Are you sponsored by your employer? Will you be able to report a negative outcomes to your research, for instance, that there is no solution to our problem using the current technology stack? A negative result is a very good research outcome, even if it tends to satisfy fewer non-academics than a positive result.

Which family pressures do you feel? It's' not unusual that you will have given up a paying role to study, moving the responsibility to provide onto another member of your family. What are their expectations?

Part 4: what's that thought nagging at the back of your mind? Is it ``How will I start?'' Or ``Will I be able to dedicate enough time to this?'' Or ``Can I really do this?''. Or ''Is ``shouldn't I be bringing in a wage rather than studying?''

You may be one of the lucky ones that doesn't have such negative thoughts, but negative thoughts are a very natural part of steps into the unknown. And research is precisely that, a step into the unknown. At least if you are aware of the doubts you naturally have, you can manage them. Think about making even the tiniest of steps forward in your research visible and celebrated! Work with Kansan boards where progress is encouragingly visible as you move a task from the inbox to the outbox. If you have concerns about managing your time, start using one of the many tools out there that break time up into manageable units and help manage it for you. If your concerns are about how to organise your thoughts, look into mind maps, lists, todo lists.

Thinking early and often through reflection is a powerful way of doing better. Do it well and your final report will be better than you will have expected.

It's worth saying that, at the end of what could be an exhausting journey, you will not fully appreciate your achievements. That realisation may have to wait until you are rested, graduated, or some distant time later.

But it will come.

\begin{guidance}
%Hack to correct tcbox behaviour
\color{black}
Something here
\end{guidance}\end{question}
%%Hack to correct tcbox behaviour
\color{black}



