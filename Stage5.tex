\part{The End}\label{part:end}


\todo{Add summary of what has been achieved.}
Stage 5 will see you completing your research project and writing up your full dissertation, ready for submission. It assumes that you have made good progress with your data generation and analysis by applying the methods and procedures in your chosen methodology during Stage 4\footnote{If that's not the case, then, you should go back to Stage 4. You should also discuss your progress with your supervisor, revisiting your project timescale and risk.}.

\chapter{Stage 5 Activities and Outcomes}\label{c:Stage5outcomes}

In Stage 5 you will complete your data generation and analysis, alongside interpreting your findings in relation to your aim and objects, and you will evaluate your research project overall and draw your conclusions. These will help you complete your dissertation ready for submission.

For this stage too, we provide a Research Activity table and a Writing Outcomes table to structure and guide your work.

\section{Your Research Activities for this stage}

The research activities which are in focus in Stage 5 are shown in \Cref{stage5ResearchActivities}, which also provides some prompts for your interaction with your supervisor during this stage. 


Completing your data generation and analysis will be a substantial activity in this stage too (at 25\% of total effort), followed by the interpretation of your findings, these being even more substantial (at 35\% of total effort for this stage). Interpreting your findings should give you a contribution to knowledge that aligns with your aim and objectives, contributing to the most original part of your research. You will spend more time on reflection and reflexivity in this stage (15\% of total effort) compared to the previous stages, as you will need to consider your project work overall, argue the strength of your research and what you have done to deal with possible weaknesses. And, of course, you will need to complete your dissertation (20\% of total effort), by writing a compelling narrative that brings all your research effort together and draw overall conclusions. There is still a small amount of work planning and risk assessment in this stage to ensure you will be able to complete your project and submit your dissertation on time.

\begin{SimpleNColTable}{stage5ResearchActivities}{4}{\RActivitiesTableCaption{5}}[X[4]X[1]X[8]X[8]]
Research activity & Effort & Description &  Supervisor Interaction Focus\\
Identifying the research problem&0\%&n/a&\\
Reviewing the literature&0\%&n/a&\\
Setting research aim and objectives&0\%&n/a&\\
Developing the research design&0\%&n/a&\\
Gathering and analysing evidence&25\%&Apply your chosen methods to complete your data gathering and analysis&Overall quality and quantity of data/evidence and their analysis\\
Interpreting and evaluating findings&35\%&Critically assess findings against aim and objectives& Critical thinking in assessing findings\\
Writing up&20\%&Achieve the writing outcomes of \Cref{stage5WritingOutcomes} & Completeness and good academic writing\\
Reflection and reflexivity &15\% & Apply to overall research & Reflection on research weaknesses and how they have been addressed\\
Planning work &2\% &{Refine your project plan to ensure you can complete your project by the deadline} & What to prioritise for completion\\
Managing risk &3\% &Review remaining risk &Any contingency needed\\	
\end{SimpleNColTable}


\section{Your Writing Outcomes for this stage} 

\Cref{stage5WritingOutcomes} gives you the writing outcomes for this stage: the activities in this part of the book are designed to help you reach them.

Remember that the first column of the table gives you the expected full structure of the dissertation\footnote{If your course assumes a different structure, then use that instead, mapping the writing outcomes accordingly.}. Within that column, the greyed out parts are yet to be written and will be the focus of later stages, while those highlighted with red bullet points are to be written during this stage. The remaining parts are those you wrote in previous stages: depending on your work in this stage, you may need to revise or adjust them.

\todo{edit all the WOs tables at the end}

\begin{ReportTable}{stage5WritingOutcomes}[\WOCaption{5}{20}]
%%\ReportTitle1
\ReportTitle*
	&Finalise your title &
\\

%%\ReportAbstract12345
\ReportAbstract*!!!!
	&Write your abstract	&\Cref{sect:writingAbstract}
\\

%%1-\ReportIntroduction12345
\ReportIntroduction****
	& Finalise your introduction &\\

%%2-\ReportLitRev---
\ReportLitRev***
	& Finalise your literature review &\\

%%3-\ReportResearchDef----
\ReportResearchDef****
	& Finalise your research definition &\\

%%4-\ReportResearchDes-----
\ReportResearchDes*****
	& Finalise your research design, taking into account the execution of your methodology &\\

%%5-\ReportAnalysisInterp----
\ReportAnalysisInterp***!
	& Summarise all data generated and report their completed analysis. Ensure your report is appropriately structured and presented to convey your work concisely, clearly and systematically 

	\item Summarise all key findings from your data analysis

	\item Interpret your finding in relation to your aim and objectives  &\Cref{ch:CompletingYourResearch}\\

%%6-\ReportEvalConc-------
\ReportEvalConc*!!!!!!
	& In this concluding chapter, provide an overall evaluation of your research and draw your conclusions
	
	\item Evaluate all your findings, both from your literature review and your data generation and analysis, against your aim and objectives 

	\item Evaluate your findings comparing and contrasting them to related work in the academic literature to highlight your novel contribution

	\item Argue the validity, reliability and lack of bias of your research, highlighting how you have addressed research weaknesses in your methodology 

	\item  Discuss ways in which your findings may inform future research

	\item If applicable, argue how your findings may influence professional practice 

	\item Include a reflexive account based on your overall project experience
 &\Cref{sect:evaluateWholeResearch}\\

%%7-\ReportRefs-
\ReportRefs*
	& You should complete and check for references for correctness and completeness &\\
	
%%%A-\ReportDissertationAppendices
\ReportDissertationAppendices**
	& If needed, use appendices to include supplementary material in support of the main body of your dissertation, particularly data collection and analysis. For instance, you could include  sample raw data, questionnaires used in the research, programme code, detailed calculations, etc. You should customise this appendix to your own project
	&\\
	
	
%%\ReportProgressTracking123456789
\ReportProgressTracking**???*
& Revise the content of this appendix in view of your increased understanding and the progress you have made in this stage, paying particular attention to feasibility, work plan and risk assessment
&\\
%
%%\ReportReflection123
\ReportReflection*?
 & Update your personal statement based on actions and outcomes in this stage 
 &\\

\end{ReportTable}
\endinput


\section{Planning your work for this stage}\label{sect:stage5WorkPlan}

Before going further, you should refine your project work plan to include more detail of your work in Stage 5.


%For this, you will need to:
%\begin{itemize}
%	\item make sure you have completed all the work for Stage 1 or make the necessary adjustments to your plan
%	\item identify the main tasks under each activity for this stage, allocate them time and include them in your plan. For complex tasks, you may also include some sub-tasks, etc., but you should avoid making your plan too complicated
%	\item establish main milestones and deliverables and include them in your plan
%	\item optimise your plan by considering dependencies and tasks which may overlap. 
%\end{itemize}


\begin{question}[subtitle={Activity: Revising your work pan}] 
Consider the activities in \Cref{stage5ResearchActivities} and the writing outcomes in \Cref{stage5WritingOutcomes}: 

\begin{itemize}
	\item for each activity, identify a number of tasks which capture the work needed, decide how much time to spend on each, and include them in your work plan, also taking into consideration their possible dependencies
	\item for each writing outcome, identify corresponding deliverables and set related milestones in your work plan.
\end{itemize}

At the end, review your overall plan, also considering the progress you made in the previous stages, and make all necessary adjustments to ensure you will be able to complete your project by its deadline.
\begin{guidance}
As in previous stages, you should:
\begin{itemize}
	\item focus on a small number of key tasks for each activity, so to keep your plan light
	\item when allocating time to tasks, ensure that tasks fit within the overall time for their corresponding activity 	
	\item consider task dependencies and things you can progress in parallel, so to optimise your project time 
	\item break down new content you will need to write into deliverables, setting appropriate milestones in your plan.
\end{itemize} 

If you feel you are running out of time, you should focus on the essential activities and tasks needed to complete your project.
\end{guidance}
\end{question}

If after reviewing your progress you find that you are well behind, then you must talk to your supervisor to agree an approach which will allow you to complete your project on time.

\chapter{Completing your research}\label{ch:CompletingYourResearch}

Building on Stage 4, in this stage you will complete your data generation and analysis, then proceed to interpret your findings in relation to your aim and objectives, so to argue your contribution to knowledge. The following activities will help you complete this work.

\section{Completing your data gathering and analysis}\label{sect:completingDataGathering}

Through this activity, you should carry out to completion your work from Stage 4.

\begin{question}[subtitle={Activity: Completing your data generation and analysis}] Complete your data generation and analysis based on your chosen methodology. Expand your Stage 4 report by including summaries of new data and their analysis.

\begin{guidance} Ensure you continue to manage your raw data carefully, and that you summarise and presents your data in a clear and rigorous manner in your report. 

This activity assumes you made some good progress with your data generation and analysis in Stage 4, hence you have sufficient time to complete them in this stage. If that's not the case, you should discuss with your supervisor what you will be able to achieve realistically in the remaining time for your project, for instance whether it would be possible to reduce the scope of your research or apply alternative, more time-efficient methods. Ensure that you account for any methodological change in your work plan and summary of your methodology.
\end{guidance}\end{question}
%%Hack to correct tcbox behaviour
\color{black}

\section{Interpreting your data}\label{sect:interpretingData}

Once your data generation and analysis are complete, you must provide an interpretation in relation to your aim and objectives, to argue your contribution to knowledge. Specifically, in your interpretation you should address the following questions:

\todo{some of this question can move to the overall evaluation}

\begin{itemize}
\item What conclusions can you draw from your data analysis?

\item How do they relate to your aim and objectives? 

\item What do they succeed and/or fail to achieve in relation to them?

\end{itemize}

By answering these questions you can construct a narrative, based on the evidence you have generated, to support your claim that you have contributed new knowledge with your research. The  form of this narrative will depend on the specifics of your objectives, the methods you have applied and the evidence you have generated. It will also depend on what is considered good practice within your discipline or field of study, something you can learn by considering previous successful dissertations.
 
\begin{question}[subtitle={Activity: Looking at published dissertations}] 
Select up to five successful past research dissertations relevant to your own course and field of study. Skim through their content paying particular attention to their methodology section, and how they report and summarise data generation, analysis and interpretation. You should focus on how the information is presented and evidence is used in arguments. Note down practices you could adopt in your own dissertation.
\begin{guidance}
Research dissertations are a matter of public record so that many, if not most, universities make them available through their library or publicly accessible online repositories\footnote{The Open University UK, for instance, makes them accessible through their Open Research Online repository, at https://oro.open.ac.uk/cgi/search/advanced.}. Therefore, you should be able to find relevant past dissertations within your own university. Some programmes of study, particularly at Masters level, also make a selection of past dissertations available as examples to their current students, so you should enquiry whether this is the case for your course.  

Should you find locating relevant dissertations within your institution difficult, then you should talk to your supervisor who will be able to point you to relevant examples, possibly from previous research students they have supervised.

Another option is to look at collections of research dissertations from other universities or consortia. For instance, the British Library manages the EThOS\footnote{It stands for Electronic Thesis Online System.} repository, in partnership with several UK universities, to provide free access to UK research dissertations. 
\end{guidance}
\end{question}

 

%\begin{example}{Example of interpretation against a research objective}
%The following extract is adapted from a Masters dissertation at our institution, provides the student's interpretation of data analysed, conducted with respect to one of the research objectives, that of determining how researchers in a particular field of study select research strategies and methods. Their methodology included questionnaires and interviews with researchers.
%
%\begin{quotation}
%In the questionnaire, the research question was ranked as the most important factor in the selection of strategy and methods, followed by type of data required, then philosophical worldview and skills/experience of the researcher. Specifically:
%\begin{itemize}
%	\item `Research questions comes first' captured the idea that the research question is the dominant factor that determines strategy and method. One interviewee said that “I take the attitude that you set a question and then you try to answer it”, another noted that “You will have to think what is the best way to get the result, and that will be based on the question”.
%	\item ‘Type of data points the way’ is a sub-theme of the above theme. It captured the idea that the research question determines the type of data required and this in turn determines the strategy and methods. One interviewee summed it up as follows: “what you are trying to find out determines what information you need, and what information you need dictates how you can go about collecting it”.
%	\item ‘Philosophy is important too’ captured the idea of research philosophy as a starting point that determines acceptable research questions and strategies/methods. One participant described how “the ontological and epistemological perspectives that you are coming from are always going to shape your research”.
%	\item ‘It must be feasible’ captured the idea that whatever strategy and methods are considered, they also need to be feasible. Feasible in terms of cost: “I wouldn’t be able to do observations and interviews could be inaudible, you know, doing international phone calls”. Feasible in terms of time and access to data/sources: “I tried to understand which of the research questions were most suitable for me in terms of effort in terms of (inaudible) in terms of, you know, sources”.
%	\item ‘Skills and experience matter’ captures idea that personal skills of the researcher can be a deciding factor: “because of my personal skills, I don’t think I am good doing interviews” and “I just don’t go a bundle on quantitative research because my knowledge of statistics is not terribly high”.
%\end{itemize}
%\end{quotation}
%\end{example}
	

It's time for you to have a go at interpreting your data.

\begin{question}[subtitle={Activity: Interpreting your data}] 
Consider your data analysis and based on it, address each of the above questions. Write down your responses, ensuring your arguments are well-formed, with explicit reference to your evidence.
\begin{guidance}
Your interpretation should be specific to your data, in that your arguments should be well-founded in the evidence you have generated in your project through your original research. 

Your aim and objectives should guide your interpretation by helping you focus on the most relevant aspects of your data analysis. 

You should apply some of the practices you have identified through the previous activity to help you develop a narrative that is appropriate for your course and discipline.

It is entirely possible that while interpreting your current findings you realise that further analysis is needed, so that you may have to go back to your raw data and perform further analysis. This is perfectly normal in research, so expect iterations between analysis and interpretation. In some cases, you may find that you haven't generated sufficient data, so that you may also need to generate more data, although this will depend on how much time you have left on your project. If you find that there are substantial omissions in your data, then you should discuss what to do about it with your supervisor.
\end{guidance}
\end{question}


\section{Evaluating your research}\label{sect:evaluateWholeResearch}

A significant and essential activity in Stage 5 is to stand back and evaluate your research as a whole to be able to conclude the extent your project has contributed new knowledge, and to argue its significance and implications for future research and possibly for professional practice. In your overall evaluation, you must also address potential research weaknesses, and whether and how you have dealt with them in your project.

Specifically, you should address each of the following.

\subsection{Meeting your aim and objectives}

You should reflect on the extent your research overall has met its stated aim and objectives. While your interpretation\footnote{See \Cref{sect:interpretingData}.} will be deep and detailed, referring to specific data you have gathered and analysed, here you are expected to highlight key conclusions based on such an interpretation. If, in the cold light of day, your research hasn't fully met your aim and objectives then you will need to establish for the reader what you have actually achieved in relation to them.

\begin{question}[subtitle={Activity: Stating the outcome of your research}]
State the extent your project was able to meet its aim and objectives, and how.
\begin{guidance}
Go back to your interpretation and summarise your main conclusions and how you arrived at them. Indicate where your findings were inconclusive or aspects which have remained unresolved at the end of your project. 
\end{guidance}
\end{question}
%%Hack to correct tcbox behaviour
\color{black}


\subsection{Adding to the academic body of knowledge}

As the primary aim of academic research is to contribute new knowledge, you need to establish the extent your research had added to the body of knowledge you considered in your literature review. For instance, you may have been able to address previously unanswered questions, or to confirm findings from previous studies, but in a different context, or even arrive at different conclusions, perhaps putting into question the validity of previous research. This is where you defend your claim to novelty\footnote{We discussed this topic in \Cref{sect:DealingWithWeaknesses}.} and demonstrate your awareness of how it relates to its wider academic discourse.

\begin{question}[subtitle={Activity: Defending our claim to novelty}]
Indicate which new knowledge you were able to generate in relation to the academic literature.
\begin{guidance}
You should go back to the knowledge gap you identified in your literature review and captured in your research problem and explain how you have addressed it in your research. You should compare and contrast your work with that of related research reported in the literature, highlighted what makes your work novel and distinct.
\end{guidance}
\end{question}
%%Hack to correct tcbox behaviour
\color{black}

\subsection{Your research validity, reliability and lack of bias}

Recall that all research suffer from potential weaknesses\footnote{Recall \Cref{sect:DealingWithWeaknesses}.}, therefore you will need to argue why your research is trustworthy, has not been negatively impacted by your own bias or that of participants in your study, and the extent your findings generalise beyond your project. In doing so, you should appeal to the ways you have dealt with research weaknesses in your methodological choices and procedures, while also acknowledging any remaining ones.   


\begin{question}[subtitle={Activity: Your research validity, reliability and lack of bias}]
Summarise how you have ensured validity and reliability in the application of your methodology and guarded against possible bias related to the methods and strategies you have applied.
\begin{guidance}
You should assess critically the extent the measures you have taken in your project to address these weaknesses have been successful. You should also acknowledge weaknesses you were unable to address and explain why that was the case. 
\end{guidance}
\end{question}
%%Hack to correct tcbox behaviour
\color{black}


\subsection{Implications for future research}

Research is collaborative and incremental: your own research relies on research that came before, while other researchers may pick up from where you have left. In your conclusions, you should highlight possible future research which may follow from your work.

\begin{question}[subtitle={Activity: Future research based on your project outcomes}]
Provide an account of possible avenues for future research which may follow from what you have achieved in your project.
\begin{guidance}
You may discuss aspects of your research problem you didn't have the time to explore during your project, or other emerging related sub-problems which were not within the scope of your research. For each, you should indicate how future research could build on your work. 

If you had identified un-addressed research weaknesses in the previous activity, then you could also outline further research to address them. 
\end{guidance}
\end{question}
%%Hack to correct tcbox behaviour
\color{black}


\subsection{Implications for professional practice} 

Your research may relate to a practical problem in your profession, in which case in your evaluation you should also consider  ways  in which your research may be relevant to professional practice, including how it may lead to change or improvement.

\begin{question}[subtitle={Activity: Influence on professional practice}]
Identify and justify ways in which your research may influence professional practice.
\begin{guidance}
You can skip this activity if your research has no professional relevance beyond academia. You should clarify ways outcomes from your project may apply or be adopted within a professional context, the potential benefits, as well as enablers and barriers.
\end{guidance}
\end{question}
%%Hack to correct tcbox behaviour
\color{black}



\subsection{You as an academic researcher}

Throughout this book, we have highlighted the importance of reflexivity in research as a means for you to examine and question critically your own views, assumptions and beliefs, so to bring objectivity into your work as a researcher. In your concluding reflexive account, you should address what you have learnt from a personal standpoint in relation to thinking and behaving like an academic researcher, and how your mindset and skills may have changed through the process of academic research.

\begin{question}[subtitle={Activity: Your reflexive account}] 
Give an account of how conducting research has changed you, as a person and as an academic researcher, and how you would approach your research now, should you start again. 
\begin{guidance}
This is a very personal account which may or may not be required by your course. We would recommend you do this activity regardless, to continue your process of personal growth through reflexivity.
\end{guidance}
\end{question}
%%Hack to correct tcbox behaviour
\color{black}

\chapter{Completing your dissertation}\label{ch:CompletingYourDissertation}

The end of your project is fast approaching: you can now see the finishing line! What is left is to complete and submit your dissertation. 

You will already have a great proportion of your dissertation written as a result of diligently writing your reports in the previous stages. Therefore, your dissertation should extend your Stage 4 report by covering the work you have carried out in this stage and addressing any remaining gap. The structure and content we recommend are indicated in \Cref{tab:dissertation}, which should not come as a surprise to you. The first thing for you to do is to write your dissertation abstract. 


\begin{SimpleNColTable}{tab:dissertation}{2}{Dissertation structure and guidance}
	
Dissertation template & Guidance \\

Title & Your title should capture succinctly your research problem and aim \\
 Abstract & Your abstract should providing a succinct account of your research \\
 Chapter 1: Introduction 1.1 Background to the research 1.2 Justification for the research 1.3 Fitness of the research & This chapter should provide an introduction to your research topic in its wider context (as background) and your justification of why the research is worth pursuing. Its purpose is to introduce and justify your intended research in overview, before entering the detailed work of the subsequent chapters. It should be well argued and supported by appropriate citations. In this chapter, you should also argue how the research fits within the scope of your qualification, and meets any other personal, professional or organisational criteria. \\
 Chapter 2: Literature review 2.1 Review of existing relevant knowledge 2.2 Critical summary, including knowledge gap to be addressed by the research & Your review should provide a critical account of your in-depth engagement with the academic (and other) relevant literature, including identifying key trends, ideas and possible knowledge gaps. Most of your citations should point to academic articles. Your critical summary should highlight key insights from your review and provide a strong justification for your proposed research. Both coverage and depth of your review matter. You should ensure that your review is well structured, with a logical narrative flow and your arguments are well supported by data \\
 Chapter 3: Research definition 3.1 Problem statement 3.2 Aim, objectives, tasks and deliverables 3.3 Knowledge contribution & You should ensure that your research problem is well articulated and appropriate for your course and your personal and professional circumstances, that your aim and objectives are consistent with research problem, that tasks and deliverables break down your objectives appropriately and are clearly related to your chosen research methods, and that the intended knowledge contribution of your research is clearly articulated \\
  Chapter 4: Research design 4.1 Data 4.2 Research strategy and methods 4.3 Research procedures 4.4 Ethical, legal and EDI considerations & This chapter should demonstrated your critical engagement with all elements of research design, including a detailed account of the data needed in your research, the research methods and research strategies chosen, with justification, and applied within your project. Your account should be supported by a clear rationale and insights from the related literature, and appropriately justified in relation to your research problem, aim and objectives. It should also demonstrate your careful consideration of ethical and legal matters, and that your research complies with your course and university requirements \\
  Chapter 5: Analysis and interpretation 5.1 Summary and analysis of data 5.2 Summary of key findings 5.3 Interpretation in relation to aim and objectives & This chapter should provide a detailed account of your data generating, analysis, the findings you have derived and their interpretation in relation to your research aim and objectives. It should demonstrate a competent execution of your research design, present appropriate summaries of data, supported by raw data in an appendix if needed. Key findings should be clearly identified and logically connected to data, with good critical reflection on their implications for aim and objectives. \\
  Chapter 6: Evaluation and conclusion 6.1 Evaluation against aim and objectives 6.2 Evaluation against related work in the literature 6.3 Implication for practice 6.4 Validity of the research 6.5 Further work 6.6 Personal reflection on your experience of & In this chapter you should reflect on the extent your research has met its stated aim and objectives, bringing together all your findings from both primary and secondary research work. You should also reflect how it has contributed new knowledge in relation to the literature you have reviewed. You should also assess the validity of your research and consider any implication for further research and, if applicable, for professional practice. You should also reflect on what you have learnt from a personal standpoint in relation to thinking and behaving as an academic researcher. \\
 References & You should include all your references and ensure you apply the required bibliographical style consistently.\\
 Appendix - Raw data & If relevant, you should include a sample of your raw data as an appendix \\
\end{SimpleNColTable}


\section{Writing an abstract for your research}\label{sect:abstractdraft}

An {abstract} is a common way to summarise academic research. Abstracts are an integral parts of all published academic articles and you will have encountered many abstracts while reviewing the literature. 

Abstracts are also very common in academic dissertations, therefore it is highly likely you will be asked to include one at the beginning of yours. In all cases, writing an abstract for your research is a good exercise as it gives you an opportunity to write a logical argument that connects all key elements of your research. This can help you check that all the pieces fit together in a coherent manner. It is also something you can share with your supervisor and critical friends to communicate succinctly the essence of what you have done and achieved.  

Your {abstract} should provide a short summary of your whole research written for a specialist audience, that is you should assume that the reader has good knowledge of the topic and field of study. It should also be a stand-alone item, so that your reader should be able to understand its content without any reference to other parts of your dissertation.

The content of your abstract should convey succinctly:
\begin{itemize}
	\item your research problem, how and where it arises and its significance
	\item your research aim
	\item your methodology
	\item your contribution to knowledge and its implications
\end{itemize} 

\begin{question}[subtitle={Activity: Witing your abstract}]
Write an abstract for your project, ensuring it addresses in a succinct manner each of the points above.
\begin{guidance}
You should check whether your course requires a particular structure for your dissertation abstract: while most abstracts are written as a continuous narrative, abstracts which are structured around  each of the points above are becoming more common.

You should also look at how abstracts are written in past dissertations, possibly from your own course of study, as examples of ways you could organise your own narrative. 

You may need to iterate few times to arrive at the final version of your abstract. It will also help you share a complete draft with your supervisor and other critical readers -- family, friends or other students on your course, for feedback on how to improve it further.
\end{guidance}
\end{question}
%%Hack to correct tcbox behaviour
\color{black}


\section{Filling any remaining gaps}\label{ssect:fillingGaps}

\todo{lR: need to sort out all WOs and structure of the dissertation first; this section is yet to be completely edited.}

By now, you should have something to say in each of the chapters and sections suggested in \Cref{tab:dissertation}.

Chapters 1 to 3 should be close to submission standard, while Chapter 4 should be almost complete, perhaps requiring some limited editing, to account for possible changes in methodology while completing your data gathering and analysis. 

Instead, Chapters 5 and 6 are likely to require most of your effort at this point: while you may have developed much of their content as a result of the activities you have carried out in this stage, you will still need to bring it all together into a coherent whole, and even trim it down to make it fit within the expected length of your final dissertation.

 Depending on your chosen methodology, material for certain sections may extend to many pages: for an experiment, it may be that there is an extensive sections on reflexivity, triangulation, and validation. These sections may be much shorter if they appear at all, in the mathematical thinking research strategy.

Irrespective of which research strategy you have chosen, however, some sections will always have content. These include:
%
\begin{itemize}
\item list here
\end{itemize}
%

Some of these may simply not have been written yet\footnote{Or not written to D1 – the first complete draft.} even though you know that they are needed and have things\footnote{Or will have!} to say – for instance, we recommend leaving the Abstract, Introduction and Conclusions until quite late in the writing process. Others you will complete next.


\begin{question}[subtitle={Activity: Putting your dissertation together}] Using your word processor of choice, and starting from your previous report, complete your dissertation by applying the structure and guidance in \Cref{tab:dissertation}, and making good use of your notes and summaries from all related activities you have carried out.

\begin{guidance} 
Although the dissertation structure and guidance we provide is fairly standard, it is possible they don't not match exactly the requirements of your own course, which may provide a different template for you to follow. Indeed you should check and apply your course guidance, and map the structure and guidance in \Cref{tab:dissertation} to what is required in your course of study.
\end{guidance}\end{question}
%%Hack to correct tcbox behaviour
\color{black}

\section{Compliance to presentation requirements}\label{ssect:ReviewingYourDraft}

Now that you have a complete draft of your dissertation, you should review it, and make all necessary adjustments, to ensure it is presented in a way that meets your course requirements.	

\subsection{Word count}

An adjustment which is almost always needed is to trim the dissertation down to fit within the word 
 limit established by your course of study\footnote{In our experience, we all have a tendency to write too much rather than too little!}. For instance, our institution sets word limits for all Masters and Doctoral dissertations, and gives precise rules on what is included or otherwise in that count: say, references, abstract and appendices are excluded, but figure and table captions are not, etc. 

Often, there is also an expectation that the content of your dissertation is balanced across its  chapters\footnote{Although it is normal for some chapters to be more substantial than others.} and your course of study may provide specific recommendations on the expected size of each chapter. As an example, \Cref{tab:contentBreakdown} summarises what is expected in our university for Masters level research dissertations in STEM\footnote{STEM stands for Science, Technology, Engineering and Mathematics} taught programmes, where dissertations are expected to be between 10,000 and 15,000 words in length. Of course, these are just guidelines and should not be applied too rigidly: students can used them as a baseline, then adapt them to the needs of their project.


\begin{SimpleNColTable}{tab:contentBreakdown}{5}{Breakdown of dissertation content, as a percentage of total, as applied in STEM subject in our university}
Element & Breakdown & Recommended word count distribution & Equivalent for 10,000 word dissertation & Equivalent for 15,000 word dissertation \\
Chapter 1 Introduction & Background to the research Justification for the research Definitions (if any) Dissertation outline & 10\% & 1000 & 1500 \\
Chapter 2 Literature review & Review of existing relevant knowledge Critical summary, including knowledge gap & 20\% & 2000 & 3000 \\
Chapter 3 Research definition & Problem statement Aim, objectives, tasks and deliverables Knowledge contribution & 10\% & 1000 & 1500 \\
Chapter 4 Research design & Data Research strategy and methods Procedures Ethical considerations & 15\% & 1500 & 2250 \\
Chapter 5 Analysis and interpretation & Summary and analysis of data Summary of key findings Interpretation in relation to aim and objectives & 30\% & 3000 & 4500 \\
Chapter 6 Evaluation and conclusion & Evaluation against aim and objectives Evaluation against the academic body of knowledge Implications for practice (if any) Validity of the research Further research Personal reflection on your research experience & 15\% & 1500 & 2250 \\
\end{SimpleNColTable}


\begin{question}[subtitle={Activity: Complying with the word limit}] 
Check the regulations and guidelines of your course of studies to establish any word limit for your dissertations overall and its parts. Edit your dissertation to ensure it does not exceed those word limits. 
\begin{guidance} 
You must take any word limit seriously as ignoring them may result in your work being penalised during assessment. There may be some tolerance around such limits, say, you can exceed them by 5\% without penalty, so you should check that too.

If your dissertation is well above the word limit, then it may take quite a bit of effort to reduce its content. You may need to rewrite some sections, provide more succinct summaries throughout or even take something out. You may be able to move some information in appendices at the end of the dissertation, however, you must check the guidelines of your course to make sure you use them appropriately. For instance, in our institution you can use appendices which are not included in the word count, but these are not assessed. As a result, they are usually used to provided additional information on topics which are covered in a more succinct way in the body of the dissertation, but are not a replacement. A typical use is to include sample raw data in an appendix, while their summaries and analysis remain in the dissertation body. In this way your work is still assess, while you are also demonstrating the breadth and depth of the data you have generated.
\end{guidance}\end{question}
%%Hack to correct tcbox behaviour
\color{black}


\subsection{Conventions}

Your dissertation will also need to conform to the presentation conventions established by your course of studies. These usually address typographical aspects of your dissertation, like margins, line spacing, numbering, etc. As an example, those most commonly applied in our institution are summarised in \Cref{tab:presentationConventions}.

\begin{SimpleNColTable}{tab:presentationConventions}{2}{Example of dissertation presentation conventions}[X[2]X[6]]
Feature & Convention\\
Fonts & Use a standard font that is easy to read, e.g. Times New Roman or Arial, with font size 11 or 12 \\
Margins and spacing & Leave appropriate margins on both the left and the right of the page, typically around 2 cm. Use 1.5 line spacing \\
Your identifiers & Include your name and student identifier, possibly as a header or as part of the title page \\
Title page & Include a title page containing your research title. Usually the following statement is also required: \enquote{A dissertation submitted in partial fulfilment of the requirements for the degree of $<$name of degree$>$}, where you should replace $<$name of degree$>$ with your own degree title \\
 Table of content & Include a table of content after the title page \\
Page numbers & Number all pages, including references and appendices. In particular, use lower-case Roman numerals on the preliminary pages -- iii, iv, v, etc. -- and Arabic numerals starting from page 1 at the beginning of Chapter 1. \\
Chapter and section numbering & Number chapters sequentially using Arabic numerals starting with 1. Number sections sequentially starting with the chapter number, e.g. 1.1, 1.2, etc. for sections in Chapter 1. Number sub-sections sequentially starting with the section number, e.g. 1.1.1, 1.1.2, etc. for sub-sections in Section 1.1. You should avoid sub-sub-sections, but if needed, number them sequentially starting with the sub-section number, e.g. 1.1.1.1, 1.1.1.2, etc. for sub-sub-sections in Sub-section 1.1.1. \\
Figures and tables & Number all figures and tables sequentially, starting with their chapter number, e.g. 1.1, 1.2, etc. for figures in Chapter 1. Include appropriate captions positioned after figures and before tables \\
Lists of figures and tables & List all figures and tables after your table of content. For each include both their number and caption \\
Citations and references & Apply the required bibliographical style throughout \\
Verb tense & Your dissertation is an account of what you did in your project, so you should report your work using the past tense throughout \\
\end{SimpleNColTable}

\begin{question}[subtitle={Activity: Complying with presentation conventions}] 
Check the presentation conventions used on your course of studies and edit your dissertation accordingly. 
\begin{guidance} 
If your course does have any specific conventions, then you could use those in the table which are fairly typical in the UK.
\end{guidance}\end{question}
%%Hack to correct tcbox behaviour
\color{black}

\section{Getting ready for submission}\label{sect:finalcheck}



Before submitting your dissertation, you should perform a final check, focusing on the following aspects:

- \textbf{Logical coherence}: you should ensure that all research elements of your dissertations are coherent and consistent with each other, so that there is a logical progression from research problem, to aim and objectives, to research design and its execution, to findings and conclusions.

- \textbf{Academic writing}: you should ensure that academic arguments are well formed, including being well-supported by secondary and/or primary data, that the language you use is clear and precise, and there is a good balance between description and critical reflection.

- \textbf{Proof-reading}: you should remove grammatical errors and typos, and ensure that punctuation is correct. You should also check that the narrative makes sense to the reader, for which we strongly advise you ask for help from a friend or family member: even if they are not experts on the topic of your project, they should be able to follow what you have written and get the gist of your work.

- \textbf{Conformance to presentation conventions}: you should ensure that your dissertation conforms to the the requirements of your course, follows its presentation conventions, its length is within the word limit, and its content is well balanced between chapters.

\begin{question}[subtitle={Activity: Performing your final check }] Assess your dissertation draft against each of the points above. Revise and iterate until you are ready to submit.

\begin{guidance}Revising your dissertation for submission is very important as you can lose a substantial proportion of marks should any of these aspects not be addressed carefully and to the expected standards.
\end{guidance}\end{question}
%%Hack to correct tcbox behaviour
\color{black}

You should now be ready to submit your dissertation. You should, of course, follow the instructions for your course of study to do so.

\chapter{How your dissertation will be assessed}\label{ch:HowYourDissertation}
After submission, your dissertation will go through your university's assessment process, which is designed to ensure that your work is assessed fairly against Masters research benchmarks and your course learning outcomes. The specifics of this process will depend on your own university and course (or programme) of study, something\todo{Make this an activity earlier in the process.} you should investigate carefully.

You should also investigate\todo{And this} the assessment criteria applied to your work. Typically, your Masters dissertation will be assessed from the following perspectives\todo{Is this earlier too?}, although the specific marking scheme applied within your course may break each further:

\begin{itemize}
\item \textbf{Research definition and research design}: this refers to an appropriate articulation and justification of the research problem in its wider context, including your critical review of the academic literature to contextualise and justify your research problem and knowledge contribution, a well developed and justified research design, and well constructed academic arguments

\item \textbf{Data generation, analysis, interpretation, and conclusion}: this refers to a competent execution of your research design, an adequate amount of data gathered and analysed, an appropriate interpretation of your findings, and a critical evaluation of your research overall

\item \textbf{Presentation}: this refers to how your dissertation is put together, its cohesiveness and logical flow, including abstract\footnote{And extended abstract, if needed.}, and its conformance to conventions, including an appropriate use of tables, figures and diagrams to summarise and present your work.
\end{itemize}

The assessment of your work under these criteria will contribute to your final grade, which will be established by your examiners in relation to Masters level quality benchmarks, including those\footnote{These are the typical criteria for the UK. Those in your country may vary.}\todo{Add source} summarised in Cref{tab:gradeBenchmarks}.

\begin{SimpleNColTable}{tab:gradeBenchmarks}{2}{Typical grade benchmarks for Masters dissertations, based on UK quality standards}
	
Grade & Quality descriptor \\

Distinction & All elements of the dissertation are present, including abstract and any required appendix, and are of a high standard. In particular, the dissertation demonstrates: advanced, authoritative understanding and analysis of key issues and complex problems;  strong data of a critical approach to own work and that of others;  competent use of a wide range of data in support of academic arguments;  appropriate and well justified selection of research strategies and methods, applied competently to own research;  originality and independence of thought;  compelling narrative which is coherently and logically presented;  excellent presentation standards;  excellent research potential \\
Merit & All elements of the dissertation are present, including abstract and any required appendix, and are of a good standard. In particular, the dissertation demonstrates:  good understanding and analysis of key issues;  good data of a critical approach to own work and that of others;  good use of data in support of academic arguments;  appropriate selection of research strategies and methods, applied reasonably well to own research;  some originality;  coherent and logically presented narrative;  good presentation standards;  good research potential \\
Pass & Some elements may be weak or missing, but all three perspectives above are sufficiently addressed. In particular, the dissertation may data some of:  limited understanding and analysis of key issues;  limited data of critical approach to own work and that of others;  limited use of relevant data in support of academic arguments;  some appropriate choices of research strategies and methods, but with limited application to own research;  plausible narrative;  adequate standards of presentation \\
Weak fail & Many elements of the dissertations are very weak or missing, and not all three perspectives above are sufficiently addressed. In particular, the dissertation may data many or all of:  superficial understanding and analysis of key issues;  weak data of critical approach to own work and that of others;  gaps in the use of data in support of academic arguments;  inappropriate choice or application of research strategies and methods;  weak narrative;  poor standards of presentation \\
	&(In this case, a course may allow some remedial work and resubmission.)\\
Complete Fail & The dissertation has critical flaws and omissions, so that is not recoverable via a resubmission. In particular, the dissertation demonstrate many or all of:  lack of understanding and analysis of key issues;  lack of critical approach to own work and that of others;  little or no use of data in support of academic arguments;  inappropriate choice or application of research strategies and methods;  incoherent and confused narrative;  inadequate standards of presentation \\
\end{SimpleNColTable}

\begin{question}[subtitle={Activity: Assessing your own dissertation}] Apply the three perspectives above together with the benchmarks of Cref{tab:gradeBenchmarks} to your dissertation. Write down your own assessment of your work as a result.
\begin{guidance}
Your course of study may provide some detailed guidance on how your dissertation will be assessed. If that's the case, you should compare that guidance to the advice in this handbook, and apply it in your own assessment of your dissertation. You should only assess the content of the dissertation as is, disregarding all other knowledge you will have of your research which is not reported.

You should take an objective stance, considering both strengths and weaknesses of your work. You could also ask a friend or a family member to assess your dissertation, then compare their assessment with yours.
\end{guidance}\end{question}
%%Hack to correct tcbox behaviour
\color{black}


\chapter{Stage 5 Takeaways}\label{ch:Stage5Takeaways}
\begin{itemize}
\item Completing your project and finalising your dissertation are substantial tasks, so that you must ensure you have sufficient time in your work plan.
\item Your overall assessment of your project must address several dimensions, including the extent your aim and objectives were met, any new knowledge generated, its wider significance, the validity of your research and its implications for future work.
\item Your dissertation should meet a range of requirements on both coverage, structure, length and presentation convention. You should ensure your work meets the requirements and follows the guidelines provided by your course of study.
\item Your dissertation will be assessed following a process defined by your own course of study and university. Grade benchmarks are likely to apply, which may be based on national, or even international, benchmarks.  
\end{itemize}

%%Sectional bibliography
\printbibliography[segment=\therefsegment,title=Stage 5 \bibname]
