\part{The End}

%\chapter{Stage 5: Completing your dissertation}

\todo{Add summary of what has been achieved.}
Stage 5 will see you completing your research project and writing up your full dissertation, ready for submission.

This stage assumes that you have made good progress with your data generation and analysis and on the interpretation of your findings. This has given you a contribution to knowledge that aligns\footnote{More or less, there'll be opportunities for fine tuning later! If you don't feel this is the case, there'll also be opportunities to return to specific parts of previous stages.}% You should also discuss your progress with your supervisor, revisiting your project timescale and risk.}
, and that you're ready\footnote{You could be almost ready; most of the way towards your goal.} to put it all together into a single narrative which you will complete in this stage.

%Stage 5 assumes that you have made good progress with your data generation and analysis by applying the techniques – Stage 4 – suggested by your chosen research strategy – Stage 3\footnote{If that's not the case, then, you should go back to Stage 4. You should also discuss your progress with your supervisor, revisiting your project timescale and risk.}, which you will complete in this stage.

With reference to our 5-stage framework, the activities which are in focus in Stage 5 are summarised in Table~\ref{tab:stage5}, which also provides some guidance for your interaction with your supervisor during this stage.

\begin{ResActtblr}[caption = {Stage 5 Research Activities (15\% of project length)\label{tab:stage5}}]{}
\tableheader
\reportheader{Identifying the research problem}\\ & Research problem statement, refined as needed & be able to assess and improve your research problem statement & 1\% & \\
\reportheader{Reviewing the literature}\\ & Full literature review &  & 1\% &  \\
\reportheader{Setting your aim and objectives}\\ & Finalised aim and objectives, appropriately broken down into tasks &  & 2\% & \\
\reportheader{Developing the research design}\\ & Complete account of your research design & & 2\% & \\
\reportheader{Generating and analysing data}\\ & Data appropriately presented and analysed, with extracts from raw data in dissertation appendix, if needed; remaining raw data appropriately stored & be able to organise and store your raw data; be able to apply appropriate data analysis methods; be able to present your data in a concise and effective way & 40\% & Appropriateness of data analysis and presentation\\
\reportheader{Interpreting and evaluating findings}\\ & Critical summary and evaluation of findings & be able to derive findings from your data analysis and critically assess them in relation to research aim and objectives & 20\% & Critical and logical thinking\\
\reportheader{Reflecting and reporting}\\ & Full dissertation, including an assessment of the whole project & be able to assess entire research; be able to complete your dissertation to the expected presentation standards & 35\% & Depth of critical thinking, quality of academic writing, and conformance to standards \\
\reportheader{Planning work and managing risk}\\ & Review of work from previous stage and project risk, with adjustment to work plan for Stage 5 & be able to assess risk and revise a work plan & 1\% & Any major adjustment required to complete the project \\
\end{ResActtblr}

\begin{question}[subtitle={Activity: Understanding the effort needed in this stage}] Consider Table~\ref{tab:stage5} carefully, paying particular attention to the entries in the `Effort' column. Make a note of the activities which are most prominent in this stage and what their deliverables and learning outcomes are.

\begin{solution}
In this stage, generating and analysing data and interpreting your findings will constitute your major effort (around 60\% of your study time), although considerable effort (35\%) will also be needed in assessing your research overall and completing your dissertation. You shouldn't underestimate the time needed to complete and polish the dissertation so that is ready for submission, which is why the framework assume a significant effort in this stage.\end{solution}\end{question}
%%Hack to correct tcbox behaviour
\color{black}


\chapter{Completing your research}
Building on Stage 4, in this stage you will complete your work on generating and analysing data, on their interpretation in the context of a contribution to knowledge. This will give you a substantial start of the presentation of your findings in your dissertation.

\begin{question}[subtitle={Activity: Completing your data generating, analysis, and interpretation}] Complete your research on generating and analysing data, and the interpretation of your findings in terms of your aim and objectives. Expand on your analysis and summaries from your Stage 4 report.

\begin{guidance}Ensure you continue to manage your raw data carefully, and that your report presents all your data\slash data, findings and their interpretation in a clear and rigorous manner.

This activity is likely to take up to 40\% of your study time, assuming you were able to make good progress with your data collection, analysis and interpretation in Stage 4. If that's not the case, you should discuss with your supervisor what you will be able to achieve realistically in the remaining time for your project, for instance whether it would be possible to reduce the scope of your research or apply alternative, more time-efficient research strategies and methods. Ensure that any changes are appropriately accounted for in your work plan for this stage.
\end{guidance}\end{question}
%%Hack to correct tcbox behaviour
\color{black}

\section{Interpreting and evaluating data}

Having genearted and analysed a certain amount of data and evidence, it is time for you to start interpreting your findings in relation to your aim and objectives, and generally evaluate them in terms of their contribution to knowledge and possible limitations. This is a process you will repeat and complete in Stage 5, the concluding stage of your project, ending with your dissertation submission.

Interpreting your findings signifies addressing the following questions:

\begin{itemize}
\item What conclusions have you drawn from your data analysis?

\item How do they relate to your aim and objectives?

\item How do they relate to what you know from the literature? 

\item How do they relate to professional practice? (if applicable)

\item Which new knowledge do they contribute?

\item What do they fail to achieve?

\end{itemize}

\begin{question}[subtitle={Activity: Interpreting and evaluating your  findings}] 
Consider your data analysis and based on it, address each of the above questions. Write down your responses, ensuring your arguments are well-formed, with explicit reference to evidence.
\begin{guidance}
Your interpretation and evaluation of findings will be, of course, limited by the data\slash evidence you have generated and analysed up to this point. You will revisit and expand this work in Stage 5 in order to complete your project.
\end{guidance}
\end{question}


\section{Assessing your research}
Once you have completed your work on generating and analysing data, and interpreting your findings, it is time for you to reflect on your whole project, evaluate what you have done and draw some overall conclusions. These will form the body of the concluding chapter of your dissertation, for which you are asked to think critically about each of the following:

\begin{itemize}
\item \textbf{Evaluation against aim and objectives}: you should reflect on the extent your research has met its stated aim and objectives. The interpretation of your findings against aim and objectives is a good starting point to draw these summary conclusions. While your interpretation may be deep and detailed, with reference to specific data, here you are expected to highlight key conclusions based on such an interpretation. If, in the cold light of day, your research hasn't fully met your aim and objectives then you will need to establish for the reader what you have achieved: in this section you need to make a critical assessment of what your research has actually achieved.

\item \textbf{Evaluation against the academic body of knowledge}: this requires you to assess the extent your findings have added to the body of knowledge in your field of study, including whether they support or question findings already known from the literature you have reviewed. You should show awareness of how your own research relates to the wider academic context.

\item \textbf{Implications for practice (if any):} here you should reflect on ways in which your research may be relevant to professional practice, if applicable, including how it could lead to change and improvement. If your research is purely theoretical, then you can skip this section, and focus on the previous two items instead.

\item \textbf{Validity of the research}: this require you to assess your research in terms of construct, internal and external validity. You should refer back to Stage 3 materials to refresh your understanding of validity.

\item \textbf{Further research}: your research may have shed light on aspects of your research problem, or highlighted other related research problems, which you did not have the time to explore in your project. This is the place for you to discuss those of more relevance and to indicate how future research can build on the work you have done.

\item \textbf{Personal reflection on your research experience}: whether or not your research project is your first experience of academic research, you should reflect on what you have learnt from a personal standpoint in relation to thinking and behaving like an academic researcher. You should address how your mindset and skills have changed, or how you would do things differently should you start anew, and any other relevant thoughts you may have.

\end{itemize}

\begin{question}[subtitle={Activity: Assessing your research overall}] Assess your overall research in relation to the above points, and write appropriate summaries of each for inclusion in your dissertation.

\begin{guidance}For each point above, consider the related guidance to help you assess your research overall. Note that this assessment should consider all the work you have conducted in your project.
\end{guidance}\end{question}
%%Hack to correct tcbox behaviour
\color{black}

\chapter{Completing your dissertation}

\section{Drafting an abstract for your project}

An abstract is a common way to summarise academic research. Abstracts are an integral parts of all published academic articles -- you will have encountered many abstracts while reviewing the literature. They are also very common in academic dissertations, therefore it is highly likely you will be required to include one at the beginning of yours.

An {abstract} provides a short summary of the whole research written for a specialist audience, that is you can assume that the reader has good knowledge of the topic and field of study. It should be a stand-alone item, so that it can be understood without reference to any other part of your dissertation.

Its content should convey succinctly the research problem, how and where it arises and its significance, the research aim and research design, key results obtained by the research, their evaluation and their implications for further research or professional practice. 

Writing an abstract for your research is a good exercise, even if one is not needed for your dissertation, as it gives you an opportunity to write a logical argument that connects all key elements of your research. This can help you check that all the pieces fit together in a coherent manner. It is also something you can share with your supervisor and critical friends to communicate succinctly the essence of what you have done and achieved.  

\begin{question}[subtitle={Activity: Drafting your abstract}]
Write a draft abstract for your project, which should reflect your research progress to date. 
\begin{guidance}
You should go back to some of the articles you have reviewed to consider the content and structure of their abstract. Choose a structure which may fit your project and write up your draft abstract accordingly.

As your research is yet to be completed, you will not be able to write up the full abstract, but you should end up with a draft that you can easily complete by the end of your project.
\end{guidance}
\end{question}
%%Hack to correct tcbox behaviour
\color{black}

\section{Finalising and submitting your dissertation}

It's getting exciting – you now have all the data, evidence, and arguments in a form you need to complete your dissertation. You may have nigh-on one hundred pages of carefully written prose that looks very good on your screen. It's now time to finalise your dissertation for submission.

Your dissertation should extend your Stage 4 report by covering the work you have carried on in this stage. The  structure and content we recommend are indicated in Table~\ref{tab:dissertation}.%: note that a dissertation is usually organised in chapters, rather than sections like your previous reports.

\begin{table}[htbp]
\caption{Dissertation structure and guidance\label{tab:dissertation}}
\centering
\begin{tabulary}{\tablewidth}{@{}LL@{}} \toprule
 \textbf{Dissertation template} & \textbf{Guidance} \\
\midrule

Title & Your title should capture succinctly your research problem and aim \\
 Abstract & Your abstract should providing a succinct account of your research \\
 Chapter 1: Introduction 1.1 Background to the research 1.2 Justification for the research 1.3 Fitness of the research & This chapter should provide an introduction to your research topic in its wider context (as background) and your justification of why the research is worth pursuing. Its purpose is to introduce and justify your intended research in overview, before entering the detailed work of the subsequent chapters. It should be well argued and supported by appropriate citations. In this chapter, you should also argue how the research fits within the scope of your qualification, and meets any other personal, professional or organisational criteria. \\
 Chapter 2: Literature review 2.1 Review of existing relevant knowledge 2.2 Critical summary, including knowledge gap to be addressed by the research & Your review should provide a critical account of your in-depth engagement with the academic (and other) relevant literature, including identifying key trends, ideas and possible knowledge gaps. Most of your citations should point to academic articles. Your critical summary should highlight key insights from your review and provide a strong justification for your proposed research. Both coverage and depth of your review matter. You should ensure that your review is well structured, with a logical narrative flow and your arguments are well supported by data \\
 Chapter 3: Research definition 3.1 Problem statement 3.2 Aim, objectives, tasks and deliverables 3.3 Knowledge contribution & You should ensure that your research problem is well articulated and appropriate for your course and your personal and professional circumstances, that your aim and objectives are consistent with research problem, that tasks and deliverables break down your objectives appropriately and are clearly related to your chosen research methods, and that the intended knowledge contribution of your research is clearly articulated \\
  Chapter 4: Research design 4.1 Data 4.2 Research strategy and methods 4.3 Research procedures 4.4 Ethical, legal and EDI considerations & This chapter should demonstrated your critical engagement with all elements of research design, including a detailed account of the data needed in your research, the research methods and research strategies chosen, with justification, and applied within your project. Your account should be supported by a clear rationale and insights from the related literature, and appropriately justified in relation to your research problem, aim and objectives. It should also demonstrate your careful consideration of ethical and legal matters, and that your research complies with your course and university requirements \\
  Chapter 5: Analysis and interpretation 5.1 Summary and analysis of data 5.2 Summary of key findings 5.3 Interpretation in relation to aim and objectives & This chapter should provide a detailed account of your data generating, analysis, the findings you have derived and their interpretation in relation to your research aim and objectives. It should demonstrate a competent execution of your research design, present appropriate summaries of data, supported by raw data in an appendix if needed. Key findings should be clearly identified and logically connected to data, with good critical reflection on their implications for aim and objectives. \\
  Chapter 6: Evaluation and conclusion 6.1 Evaluation against aim and objectives 6.2 Evaluation against related work in the literature 6.3 Implication for practice 6.4 Validity of the research 6.5 Further work 6.6 Personal reflection on your experience of & In this chapter you should reflect on the extent your research has met its stated aim and objectives, bringing together all your findings from both primary and secondary research work. You should also reflect how it has contributed new knowledge in relation to the literature you have reviewed. You should also assess the validity of your research and consider any implication for further research and, if applicable, for professional practice. You should also reflect on what you have learnt from a personal standpoint in relation to thinking and behaving as an academic researcher. \\
 References & You should include all your references and ensure you apply the required bibliographical style consistently.\\
 Appendix - Raw data & If relevant, you should include a sample of your raw data as an appendix \\
\bottomrule
\end{tabulary}
\end{table}

\subsection{Finding and dealing with gaps}

By now, you should have something to say in each of the chapters and sections suggested in Table~\ref{tab:dissertation}. Depending on your chosen research strategy the material for certain sections may extend to many pages: for an experiment, it may be that there is an extensive sections on reflexivity, triangulation, and validation. These sections may be much shorter if they appear at all, in the mathematical thinking research strategy.

Irrespective of which research strategy you have chosen, however, some sections will always have content. These include:
%
\begin{itemize}
\item list here
\end{itemize}
%

Some of these may simply not have been written yet\footnote{Or not written to D1 – the first complete draft.} even though you know that they are needed and have things\footnote{Or will have!} to say – for instance, we recommend leaving the Abstract, Introduction and Conclusions until quite late in the writing process. Others you will complete next.


\begin{question}[subtitle={Activity: Putting your dissertation together}] Using your word processor of choice, and starting from your previous report, complete your dissertation by applying the structure and guidance in Table~\ref{tab:dissertation}, and making good use of your notes and summaries from all related activities you have carried out.

\begin{guidance} 
Although the dissertation structure and guidance we provide is fairly standard, it is possible they don't not match exactly the requirements of your own course, which may provide a different template for you to follow. Indeed you should check and apply your course guidance, and map the structure and guidance in Table~\ref{tab:dissertation} to what is required in your course of study.
\end{guidance}\end{question}
%%Hack to correct tcbox behaviour
\color{black}

\subsection{Revising your draft for compliance to requirements}

Now that you have a complete draft of your dissertation, you should revise it to ensure it meets your course requirements.

In our experience, a Masters dissertation is usually in the range of 10,000 to 15,000 words. Often, references, abstract and appendices are excluded from the word count, but figure and table captions are included. In general, there is an expectation that the content of your dissertation is balanced across the different chapters, although it is normal for some chapters to be more substantial than others. Our recommended distribution of content across the full body of your dissertation, based on our recommended dissertation structure, is indicated in Table~\ref{tab:contentBreakdown}, as a percentage of total. This is not a hard and fast constant, but can provide a baseline for you to get an idea of the relative weight of the different chapters of your dissertation. In adapting it to the needs of your own project and course, however, you should ensure you maintain a good balance across the whole piece.

\begin{table}[htbp]
\caption{Breakdown of dissertation content\label{tab:contentBreakdown}}
\begin{minipage}{\linewidth}
\setlength{\tymax}{0.5\linewidth}
\centering
\small
\begin{tabulary}{\textwidth}{@{}lllll@{}} \toprule
 \textbf{Element} & \textbf{Breakdown} & \textbf{Recommended word count distribution} & \textbf{Equivalent for 10,000 word dissertation} & \textbf{Equivalent for 15,000 word dissertation} \\
\midrule

 \textbf{Chapter 1} \textbf{Introduction} & Background to the research Justification for the research Definitions (if any) Dissertation outline & 10\% & 1000 & 1500 \\
 \textbf{Chapter 2} \textbf{Literature review} & Review of existing relevant knowledge Critical summary, including knowledge gap & 20\% & 2000 & 3000 \\
 \textbf{Chapter 3 Research definition} & Problem statement Aim, objectives, tasks and deliverables Knowledge contribution & 10\% & 1000 & 1500 \\
 \textbf{Chapter 4 Research design} & Data Research strategy and methods Procedures Ethical considerations & 15\% & 1500 & 2250 \\
 \textbf{Chapter 5 Analysis and interpretation} & Summary and analysis of data Summary of key findings Interpretation in relation to aim and objectives & 30\% & 3000 & 4500 \\
 \textbf{Chapter 6 Evaluation and conclusion} & Evaluation against aim and objectives Evaluation against the academic body of knowledge Implications for practice (if any) Validity of the research Further research Personal reflection on your research experience & 15\% & 1500 & 2250 \\
\bottomrule

\end{tabulary}
\end{minipage}
\end{table}

There is also an expectation that your dissertation conforms to some standard presentation conventions, which we have summarised in Table~\ref{tab:presentationConventions}.

\begin{table}[htbp]
\caption{Presentation conventions\label{tab:presentationConventions}}
\begin{minipage}{\linewidth}
\setlength{\tymax}{0.5\linewidth}
\centering
\small
\begin{tabulary}{\textwidth}{@{}ll@{}} \toprule
 \textbf{Fonts} & Use a standard font that is easy to read, e.g. Times New Roman or Arial, with font size 11 or 12 \\
 \textbf{Margins and spacing} & Leave appropriate margins on both the left and the right of the page, typically around 2 cm. Use 1.5 line spacing \\
 \textbf{Your identifiers} & Include your name and student identifier, possibly as a header or as part of the title page \\
 \textbf{Title page} & Include a title page containing your research title. Usually the following statement is also required: ``A dissertation submitted in partial fulfilment of the requirements for the degree of $<$name of degree$>$'', where you should replace $<$name of degree$>$ with your own degree title \\
 \textbf{Table of content} & Include a table of content after the title page \\
 \textbf{Page numbers} & Number all pages, including references and appendices. In particular, use lower-case Roman numerals on the preliminary pages -- iii, iv, v, etc. -- and Arabic numerals starting from page 1 at the beginning of Chapter 1. \\
 \textbf{Chapter and section numbering} & Number chapters sequentially using Arabic numerals starting with 1. Number sections sequentially starting with the chapter number, e.g. 1.1, 1.2, etc. for sections in Chapter 1. Number sub-sections sequentially starting with the section number, e.g. 1.1.1, 1.1.2, etc. for sub-sections in Section 1.1. You should avoid sub-sub-sections, but if needed, number them sequentially starting with the sub-section number, e.g. 1.1.1.1, 1.1.1.2, etc. for sub-sub-sections in Sub-section 1.1.1. \\
 \textbf{Figures and tables} & Number all figures and tables sequentially, starting with their chapter number, e.g. 1.1, 1.2, etc. for figures in Chapter 1. Include appropriate captions positioned after figures and before tables \\
 \textbf{Lists of figures and tables} & List all figures and tables after your table of content. For each include both their number and caption \\
 \textbf{Citations and references} & Apply the required bibliographical style throughout \\
 \textbf{Verb tense} & Your dissertation is an account of what you did in your project, so you should report your work using the past tense throughout \\
\bottomrule
\end{tabulary}
\end{minipage}
\end{table}

\begin{question}[subtitle={Activity: Reviewing your dissertation}] Review you current dissertation draft and make all necessary adjustments to ensure it meets the guidance and requirements above, or similar requirements and guidance from your own course.
\begin{guidance}
While our recommendations are fairly standard, it is essential that you ensure they align with your own course requirements and guidance: if not, you should of course apply the latter.
Whichever guidelines you follow, you should ensure that your dissertation fits within the overall word count, its content is appropriately balanced, and all required presentation conventions apply.
\end{guidance}\end{question}
%%Hack to correct tcbox behaviour
\color{black}

\section{Final check before submission}

Before submitting your dissertation, you should perform a final check, focusing on the following aspects:

- \textbf{Logical coherence}: you should ensure that all research elements of your dissertations are coherent and consistent with each other, so that there is a logical progression from research problem, to aim and objectives, to research design and its execution, to findings and conclusions.

- \textbf{Academic writing}: you should ensure that academic arguments are well formed, including being well-supported by secondary and\slash or primary data, that the language you use is clear and precise, and there is a good balance between description and critical reflection.

- \textbf{Proof-reading}: you should remove grammatical errors and typos, and ensure that punctuation is correct. You should also check that the narrative makes sense to the reader, for which we strongly advise you ask for help from a friend or family member: even if they are not experts on the topic of your project, they should be able to follow what you have written and get the gist of your work.

- \textbf{Conformance to presentation conventions}: you should ensure that your dissertation conforms to the the requirements of your course, follows its presentation conventions, its length is within the word limit, and its content is well balanced between chapters.

\begin{question}[subtitle={Activity: Performing your final check }] Assess your dissertation draft against each of the points above. Revise and iterate until you are ready to submit.

\begin{guidance}Revising your dissertation for submission is very important as you can lose a substantial proportion of marks should any of these aspects not be addressed carefully and to the expected standards.
\end{guidance}\end{question}
%%Hack to correct tcbox behaviour
\color{black}

You should now be ready to submit your dissertation. You should, of course, follow the instructions for your course of study to do so.

\chapter{How your dissertation will be assessed}
After submission, your dissertation will go through your university's assessment process, which is designed to ensure that your work is assessed fairly against Masters research benchmarks and your course learning outcomes. The specifics of this process will depend on your own university and course (or programme) of study, something\todo{Make this an activity earlier in the process.} you should investigate carefully.

You should also investigate\todo{And this} the assessment criteria applied to your work. Typically, your Masters dissertation will be assessed from the following perspectives\todo{Is this earlier too?}, although the specific marking scheme applied within your course may break each further:

\begin{itemize}
\item \textbf{Research definition and research design}: this refers to an appropriate articulation and justification of the research problem in its wider context, including your critical review of the academic literature to contextualise and justify your research problem and knowledge contribution, a well developed and justified research design, and well constructed academic arguments

\item \textbf{Data generation, analysis, interpretation, and conclusion}: this refers to a competent execution of your research design, an adequate amount of data gathered and analysed, an appropriate interpretation of your findings, and a critical evaluation of your research overall

\item \textbf{Presentation}: this refers to how your dissertation is put together, its cohesiveness and logical flow, including abstract\footnote{And extended abstract, if needed.}, and its conformance to conventions, including an appropriate use of tables, figures and diagrams to summarise and present your work.
\end{itemize}

The assessment of your work under these criteria will contribute to your final grade, which will be established by your examiners in relation to Masters level quality benchmarks, including those\footnote{These are the typical criteria for the UK. Those in your country may vary.}\todo{Add source} summarised in Table~\ref{tab:gradeBenchmarks}.

\begin{table}[htbp]
\caption{Typical grade benchmarks for Masters dissertations, based on UK quality standards\label{tab:gradeBenchmarks}}
\begin{minipage}{\linewidth}
\setlength{\tymax}{\linewidth}
\centering
\small
\begin{tabulary}{\textwidth}{@{}LL@{}} \toprule
 \textbf{Grade} & \textbf{Quality descriptor} \\
\midrule

 \textbf{Distinction} & All elements of the dissertation are present, including abstract and any required appendix, and are of a high standard. In particular, the dissertation demonstrates: advanced, authoritative understanding and analysis of key issues and complex problems;  strong data of a critical approach to own work and that of others;  competent use of a wide range of data in support of academic arguments;  appropriate and well justified selection of research strategies and methods, applied competently to own research;  originality and independence of thought;  compelling narrative which is coherently and logically presented;  excellent presentation standards;  excellent research potential \\
 \textbf{Merit} & All elements of the dissertation are present, including abstract and any required appendix, and are of a good standard. In particular, the dissertation demonstrates:  good understanding and analysis of key issues;  good data of a critical approach to own work and that of others;  good use of data in support of academic arguments;  appropriate selection of research strategies and methods, applied reasonably well to own research;  some originality;  coherent and logically presented narrative;  good presentation standards;  good research potential \\
 \textbf{Pass} & Some elements may be weak or missing, but all three perspectives above are sufficiently addressed. In particular, the dissertation may data some of:  limited understanding and analysis of key issues;  limited data of critical approach to own work and that of others;  limited use of relevant data in support of academic arguments;  some appropriate choices of research strategies and methods, but with limited application to own research;  plausible narrative;  adequate standards of presentation \\
 \textbf{Weak fail}\footnote{In this case, a course may allow some remedial work and resubmission.} & Many elements of the dissertations are very weak or missing, and not all three perspectives above are sufficiently addressed. In particular, the dissertation may data many or all of:  superficial understanding and analysis of key issues;  weak data of critical approach to own work and that of others;  gaps in the use of data in support of academic arguments;  inappropriate choice or application of research strategies and methods;  weak narrative;  poor standards of presentation \\
 \textbf{Complete Fail} & The dissertation has critical flaws and omissions, so that is not recoverable via a resubmission. In particular, the dissertation demonstrate many or all of:  lack of understanding and analysis of key issues;  lack of critical approach to own work and that of others;  little or no use of data in support of academic arguments;  inappropriate choice or application of research strategies and methods;  incoherent and confused narrative;  inadequate standards of presentation \\
\bottomrule

\end{tabulary}
\end{minipage}
\end{table}

\begin{question}[subtitle={Activity: Assessing your own dissertation}] Apply the three perspectives above together with the benchmarks of Table~\ref{tab:gradeBenchmarks} to your dissertation. Write down your own assessment of your work as a result.
\begin{guidance}
Your course of study may provide some detailed guidance on how your dissertation will be assessed. If that's the case, you should compare that guidance to the advice in this handbook, and apply it in your own assessment of your dissertation. You should only assess the content of the dissertation as is, disregarding all other knowledge you will have of your research which is not reported.

You should take an objective stance, considering both strengths and weaknesses of your work. You could also ask a friend or a family member to assess your dissertation, then compare their assessment with yours.
\end{guidance}\end{question}
%%Hack to correct tcbox behaviour
\color{black}

\chapter{Managing risk in Stage 5}

Trick chapter!

\chapter{Stage 5 Takeaways}
\begin{itemize}
\item Completing your project and finalising your dissertation are substantial tasks, so that you must ensure you have sufficient time in your work plan.
\item Your overall assessment of your project must address several dimensions, including the extent your aim and objectives were met, any new knowledge generated, its wider significance, the validity of your research and its implications for future work.
\item Your dissertation should meet a range of requirements on both coverage, structure, length and presentation convention. You should ensure your work meets the requirements and follows the guidelines provided by your course of study.
\item Your dissertation will be assessed following a process defined by your own course of study and university. Grade benchmarks are likely to apply, which may be based on national, or even international, benchmarks.  
\end{itemize}
