\part{The End}\label{part:end}

Stage 5 will see you completing your research project and writing up your full dissertation, ready for submission. It assumes that you have made good progress with your data generation and analysis during Stage 4\footnote{If that's not the case, then, you should go back to Stage 4. You should also discuss your progress with your supervisor, revisiting your project timescale and risk.} by applying the methods and procedures in your chosen methodology.

\chapter{Stage 5 Activities and Outcomes}\label{c:Stage5outcomes}

In Stage 5 you will complete your data generation and analysis, alongside interpreting your findings in relation to your aim and objects, and you will evaluate your research project overall and draw your conclusions. These will help you complete your dissertation ready for submission.

For this stage too, we provide a Research Activity table and a Writing Outcomes table to structure and guide your work.

\section{Your Research Activities for this stage}

The research activities which are in focus in Stage 5 are shown in \Cref{stage5ResearchActivities}, which also provides prompts for your interaction with your supervisor during this stage. 


Completing your data generation and analysis will be a substantial activity in this stage too (at 25\% of total effort), followed by the interpretation and evaluation of your findings, these being even more substantial (at 35\% of total effort for this stage). Interpreting your findings should give you a contribution to knowledge that aligns with your aim and objectives, contributing to the most original part of your research. You will spend more time on reflection and reflexivity in this stage (15\% of total effort) compared to the previous stages, as you will need to consider your project work overall, argue the strength of your research and what you have done to deal with possible weaknesses. And, of course, you will need to complete your dissertation (20\% of total effort), by writing a compelling narrative that brings all your research effort together and draw overall conclusions. There is still a small amount of work planning and risk assessment in this stage to ensure you will be able to complete your project and submit your dissertation on time.

\begin{SimpleNColTable}{stage5ResearchActivities}{4}{\RActivitiesTableCaption{5}}[X[4]X[1]X[8]X[8]]
Research activity & Effort & Description &  Supervisor Interaction Focus\\
Identifying the research problem&0\%&n/a&\\
Reviewing the literature&0\%&n/a&\\
Setting research aim and objectives&0\%&n/a&\\
Developing the research design&0\%&n/a&\\
Generating and analysing evidence&25\%&Apply your chosen methods to complete your data generation and analysis&Overall quality and quantity of data/evidence and their analysis\\
Interpreting and evaluating findings&35\%&Critically assess findings against aim and objectives& Critical thinking in assessing findings\\
Writing up&20\%&Achieve the writing outcomes of \Cref{stage5WritingOutcomes} & Completeness and good academic writing\\
Reflection and reflexivity &15\% & Apply to overall research & Reflection on research weaknesses and how they have been addressed\\
Planning work &2\% &{Refine your project plan to ensure you can complete your project by the deadline} & What to prioritise for completion\\
Managing risk &3\% &Review remaining risk &Any contingency needed\\	
\end{SimpleNColTable}


\section{Your Writing Outcomes for this stage} 

\Cref{stage5WritingOutcomes} gives you the writing outcomes for this final stage, which will help you produce your full dissertation. 

The main generative effort in this stage will be to write up your completed data generation and analysis, its main findings and interpretation against aim and objectives, and to report your overall evaluation and conclusions. You will also need to write an abstract to include at the start of your dissertation. 

Finally, by revising all remaining chapters and sections, you will ensure your dissertation is ready for submission.
 
\begin{ReportTable}{stage5WritingOutcomes}[\WOCaption{5}{20}]
%%\ReportTitle1
\ReportTitle*
	&Finalise your title &
\\

%%\ReportAbstract12345
\ReportAbstract*!!!!
	&Write your abstract	&\Cref{sect:writingAbstract}
\\

%%1-\ReportIntroduction12345
\ReportIntroduction****
	& Finalise your introduction &\\

%%2-\ReportLitRev---
\ReportLitRev***
	& Finalise your literature review &\\

%%3-\ReportResearchDef----
\ReportResearchDef****
	& Finalise your research definition &\\

%%4-\ReportResearchDes-----
\ReportResearchDes*****
	& Finalise your research design, taking into account the execution of your methodology &\\

%%5-\ReportAnalysisInterp----
\ReportAnalysisInterp***!
	& Summarise all data generated and report their completed analysis. Ensure your report is appropriately structured and presented to convey your work concisely, clearly and systematically 

	\item Summarise all key findings from your data analysis

	\item Interpret your finding in relation to your aim and objectives  &\Cref{ch:CompletingYourResearch}\\

%%6-\ReportEvalConc-------
\ReportEvalConc*!!!!!!
	& In this concluding chapter, provide an overall evaluation of your research and draw your conclusions
	
	\item Evaluate all your findings, both from your literature review and your data generation and analysis, against your aim and objectives 

	\item Evaluate your findings comparing and contrasting them to related work in the academic literature to highlight your novel contribution

	\item Argue the validity, reliability and lack of bias of your research, highlighting how you have addressed research weaknesses in your methodology 

	\item  Discuss ways in which your findings may inform future research

	\item If applicable, argue how your findings may influence professional practice 

	\item Include a reflexive account based on your overall project experience
 &\Cref{sect:evaluateWholeResearch}\\

%%7-\ReportRefs-
\ReportRefs*
	& You should complete and check for references for correctness and completeness &\\
	
%%%A-\ReportDissertationAppendices
\ReportDissertationAppendices**
	& If needed, use appendices to include supplementary material in support of the main body of your dissertation, particularly data collection and analysis. For instance, you could include  sample raw data, questionnaires used in the research, programme code, detailed calculations, etc. You should customise this appendix to your own project
	&\\
	
	
%%\ReportProgressTracking123456789
\ReportProgressTracking**???*
& Revise the content of this appendix in view of your increased understanding and the progress you have made in this stage, paying particular attention to feasibility, work plan and risk assessment
&\\
%
%%\ReportReflection123
\ReportReflection*?
 & Update your personal statement based on actions and outcomes in this stage 
 &\\

\end{ReportTable}
\endinput

\section{Planning your work for this stage}\label{sect:stage5WorkPlan}

Before going further, you should refine your project work plan to include more detail of your work in Stage 5.


%For this, you will need to:
%\begin{itemize}
%	\item make sure you have completed all the work for Stage 1 or make the necessary adjustments to your plan
%	\item identify the main tasks under each activity for this stage, allocate them time and include them in your plan. For complex tasks, you may also include some sub-tasks, etc., but you should avoid making your plan too complicated
%	\item establish main milestones and deliverables and include them in your plan
%	\item optimise your plan by considering dependencies and tasks which may overlap. 
%\end{itemize}


\begin{question}[subtitle={Activity: Revising your work pan}] 
Consider the activities in \Cref{stage5ResearchActivities} and the writing outcomes in \Cref{stage5WritingOutcomes}: 

\begin{itemize}
	\item for each activity, identify a number of tasks which capture the work needed, decide how much time to spend on each, and include them in your work plan, also taking into consideration their possible dependencies
	\item for each writing outcome, identify corresponding deliverables and set related milestones in your work plan.
\end{itemize}

At the end, review your overall plan, also considering the progress you made in the previous stages, and make all necessary adjustments to ensure you will be able to complete your project by its deadline.
\begin{guidance}
As in previous stages, you should:
\begin{itemize}
	\item focus on a small number of key tasks for each activity, so to keep your plan light
	\item when allocating time to tasks, ensure that tasks fit within the overall time for their corresponding activity 	
	\item consider task dependencies and things you can progress in parallel, so to optimise your project time 
	\item break down new content you will need to write into deliverables, setting appropriate milestones in your plan.
\end{itemize} 

If you feel you are running out of time, you should focus on the essential activities and tasks needed to complete your project.
\end{guidance}
\end{question}

If after reviewing your progress you find that you are well behind, then you must talk to your supervisor to agree an approach which will allow you to complete your project on time.

\chapter{Completing your research}\label{ch:CompletingYourResearch}

Building on Stage 4, in this stage you will complete your data generation and analysis, then proceed to interpret your findings in relation to your aim and objectives, so to argue your contribution to knowledge. The following activities will help you complete this work.

\section{Completing your data generation and analysis}\label{sect:completingDataGeneration}

Through this activity, you should carry out to completion your work from Stage 4.

\begin{question}[subtitle={Activity: Completing your data generation and analysis}] Complete your data generation and analysis based on your chosen methodology. Expand your Stage 4 report by including summaries of new data and their analysis.

\begin{guidance} Ensure you continue to manage your raw data carefully, and that you summarise and presents your data in a clear and rigorous manner in your report. 

This activity assumes you made some good progress with your data generation and analysis in Stage 4, hence you have sufficient time to complete them in this stage. If that's not the case, you should discuss with your supervisor what you will be able to achieve realistically in the remaining time for your project, for instance whether it would be possible to reduce the scope of your research or apply alternative, more time-efficient methods. Ensure that you account for any methodological change in your work plan and summary of your methodology.
\end{guidance}\end{question}
%%Hack to correct tcbox behaviour
\color{black}

\section{Summarising and interpreting findings}\label{sect:interpretingFindings}

Once your data generation and analysis are complete, you must consider which main insights you can derive from them -- your key findings -- and how they may help you claim that you have met your research aim and objectives -- your interpretation of those findings.  

Specifically, you need to answer the following questions:

\begin{itemize}
\item Which results did you obtain through your data analysis?

\item What is their connection to your aim and objectives?

\item How do they provide the evidence you need to claim you have met your aim and objectives?  
 
\item How significant are they in helping you do so? What are their limitations?

\end{itemize}

By answering these questions you can construct a narrative, based on the evidence you have generated, to support your claim that you have contributed new knowledge with your research. The  form of this narrative will depend on the specifics of your objectives, the methods you have applied and the evidence you have generated. It will also depend on what is considered good practice within your discipline or field of study, something you can learn by considering previous successful dissertations.
 
\begin{question}[subtitle={Activity: Looking at published dissertations}] 
Select up to five successful past research dissertations relevant to your own course and field of study. Skim through their content paying particular attention to their methodology section, and how they report and summarise data generation, analysis and interpretation. You should focus on how the information is presented and evidence is used in arguments. Note down practices you could adopt in your own dissertation.
\begin{guidance}
Research dissertations are a matter of public record so that many, if not most, universities make them available through their library or publicly accessible online repositories\footnote{The Open University UK, for instance, makes them accessible through their Open Research Online repository, at https://oro.open.ac.uk/cgi/search/advanced.}. Therefore, you should be able to find relevant past dissertations within your own university. Some programmes of study, particularly at Masters level, also make a selection of past dissertations available as examples to their current students, so you should enquiry whether this is the case for your course.  

Should you find locating relevant dissertations within your institution difficult, then you should talk to your supervisor who will be able to point you to relevant examples, possibly from previous research students they have supervised.

Another option is to look at collections of research dissertations from other universities or consortia. For instance, the British Library manages the EThOS\footnote{It stands for Electronic Thesis Online System.} repository, in partnership with several UK universities, to provide free access to UK research dissertations. 
\end{guidance}
\end{question}

 

%\begin{example}{Example of interpretation against a research objective}
%The following extract is adapted from a Masters dissertation at our institution, provides the student's interpretation of data analysed, conducted with respect to one of the research objectives, that of determining how researchers in a particular field of study select research strategies and methods. Their methodology included questionnaires and interviews with researchers.
%
%\begin{quotation}
%In the questionnaire, the research question was ranked as the most important factor in the selection of strategy and methods, followed by type of data required, then philosophical worldview and skills/experience of the researcher. Specifically:
%\begin{itemize}
%	\item `Research questions comes first' captured the idea that the research question is the dominant factor that determines strategy and method. One interviewee said that “I take the attitude that you set a question and then you try to answer it”, another noted that “You will have to think what is the best way to get the result, and that will be based on the question”.
%	\item ‘Type of data points the way’ is a sub-theme of the above theme. It captured the idea that the research question determines the type of data required and this in turn determines the strategy and methods. One interviewee summed it up as follows: “what you are trying to find out determines what information you need, and what information you need dictates how you can go about collecting it”.
%	\item ‘Philosophy is important too’ captured the idea of research philosophy as a starting point that determines acceptable research questions and strategies/methods. One participant described how “the ontological and epistemological perspectives that you are coming from are always going to shape your research”.
%	\item ‘It must be feasible’ captured the idea that whatever strategy and methods are considered, they also need to be feasible. Feasible in terms of cost: “I wouldn’t be able to do observations and interviews could be inaudible, you know, doing international phone calls”. Feasible in terms of time and access to data/sources: “I tried to understand which of the research questions were most suitable for me in terms of effort in terms of (inaudible) in terms of, you know, sources”.
%	\item ‘Skills and experience matter’ captures idea that personal skills of the researcher can be a deciding factor: “because of my personal skills, I don’t think I am good doing interviews” and “I just don’t go a bundle on quantitative research because my knowledge of statistics is not terribly high”.
%\end{itemize}
%\end{quotation}
%\end{example}
	

It's time for you to have a go at interpreting your data.

\begin{question}[subtitle={Activity: Interpreting your data}] 
Consider your data analysis and based on it, address each of the above questions. Write down your responses, ensuring your arguments are well-formed, with explicit reference to your evidence.
\begin{guidance}
Your interpretation should be specific to your data, in that your arguments should be well-founded in the evidence you have generated in your project through your original research. 

Your aim and objectives should guide your interpretation by helping you focus on the most relevant aspects of your data analysis. 

You should apply some of the practices you have identified through the previous activity to help you develop a narrative that is appropriate for your course and discipline.

It is entirely possible that while interpreting your current findings you realise that further analysis is needed, so that you may have to go back to your raw data and perform further analysis. This is perfectly normal in research, so expect iterations between analysis and interpretation. In some cases, you may find that you haven't generated sufficient data, so that you may also need to generate more data, although this will depend on how much time you have left on your project. If you find that there are substantial omissions in your data, then you should discuss what to do about it with your supervisor.
\end{guidance}
\end{question}


\section{Evaluating your research}\label{sect:evaluateWholeResearch}

A significant and essential activity in Stage 5 is to stand back and evaluate your research as a whole to be able to conclude the extent your project has contributed new knowledge, and to argue its significance and implications for future research and possibly for professional practice. In your overall evaluation, you must also address potential research weaknesses, and whether and how you have dealt with them in your project.

Specifically, you should address each of the following.

\subsection{Meeting your aim and objectives}

You should reflect on the extent your research overall has met its stated aim and objectives. While your interpretation\footnote{See \Cref{sect:interpretingData}.} will be deep and detailed, referring to specific data you have generated and analysed, here you are expected to highlight key conclusions based on such an interpretation. If, in the cold light of day, your research hasn't fully met your aim and objectives then you will need to establish for the reader what you have actually achieved in relation to them.

\begin{question}[subtitle={Activity: Stating the outcome of your research}]
State the extent your project was able to meet its aim and objectives, and how.
\begin{guidance}
Go back to your interpretation and summarise your main conclusions and how you arrived at them. Indicate where your findings were inconclusive or aspects which have remained unresolved at the end of your project. 
\end{guidance}
\end{question}
%%Hack to correct tcbox behaviour
\color{black}


\subsection{Adding to the academic body of knowledge}

As the primary aim of academic research is to contribute new knowledge, you need to establish the extent your research had added to the body of knowledge you considered in your literature review. For instance, you may have been able to address previously unanswered questions, or to confirm findings from previous studies, but in a different context, or even arrive at different conclusions, perhaps putting into question the validity of previous research. This is where you defend your claim to novelty\footnote{We discussed this topic in \Cref{sect:DealingWithWeaknesses}.} and demonstrate your awareness of how it relates to its wider academic discourse.

\begin{question}[subtitle={Activity: Defending our claim to novelty}]
Indicate which new knowledge you were able to generate in relation to the academic literature.
\begin{guidance}
You should go back to the knowledge gap you identified in your literature review and captured in your research problem and explain how you have addressed it in your research. You should compare and contrast your work with that of related research reported in the literature, highlighted what makes your work novel and distinct.
\end{guidance}
\end{question}
%%Hack to correct tcbox behaviour
\color{black}

\subsection{Your research validity, reliability and lack of bias}

Recall that all research suffer from potential weaknesses\footnote{Recall \Cref{sect:DealingWithWeaknesses}.}, therefore you will need to argue why your research is trustworthy, has not been negatively impacted by your own bias or that of participants in your study, and the extent your findings generalise beyond your project. In doing so, you should appeal to the ways you have dealt with research weaknesses in your methodological choices and procedures, while also acknowledging any remaining ones.   


\begin{question}[subtitle={Activity: Your research validity, reliability and lack of bias}]
Summarise how you have ensured validity and reliability in the application of your methodology and guarded against possible bias related to the methods and strategies you have applied.
\begin{guidance}
You should assess critically the extent the measures you have taken in your project to address these weaknesses have been successful. You should also acknowledge weaknesses you were unable to address and explain why that was the case. 
\end{guidance}
\end{question}
%%Hack to correct tcbox behaviour
\color{black}


\subsection{Implications for future research}

Research is collaborative and incremental: your own research relies on research that came before, while other researchers may pick up from where you have left. In your conclusions, you should highlight possible future research which may follow from your work.

\begin{question}[subtitle={Activity: Future research based on your project outcomes}]
Provide an account of possible avenues for future research which may follow from what you have achieved in your project.
\begin{guidance}
You may discuss aspects of your research problem you didn't have the time to explore during your project, or other emerging related sub-problems which were not within the scope of your research. For each, you should indicate how future research could build on your work. 

If you had identified un-addressed research weaknesses in the previous activity, then you could also outline further research to address them. 
\end{guidance}
\end{question}
%%Hack to correct tcbox behaviour
\color{black}


\subsection{Implications for professional practice} 

Your research may relate to a practical problem in your profession, in which case in your evaluation you should also consider  ways  in which your research may be relevant to professional practice, including how it may lead to change or improvement.

\begin{question}[subtitle={Activity: Influence on professional practice}]
Identify and justify ways in which your research may influence professional practice.
\begin{guidance}
You can skip this activity if your research has no professional relevance beyond academia. You should clarify ways outcomes from your project may apply or be adopted within a professional context, the potential benefits, as well as enablers and barriers.
\end{guidance}
\end{question}
%%Hack to correct tcbox behaviour
\color{black}


\subsection{You as an academic researcher}

Throughout this book, we have highlighted the importance of reflexivity in research as a means for you to examine and question critically your own views, assumptions and beliefs, so to bring objectivity into your work as a researcher. In your concluding reflexive account, you should address what you have learnt from a personal standpoint in relation to thinking and behaving like an academic researcher, and how your mindset and skills may have changed through the process of academic research.

\begin{question}[subtitle={Activity: Your reflexive account}] 
Give an account of how conducting research has changed you, as a person and as an academic researcher, and how you would approach your research now, should you start again. 
\begin{guidance}
This is a very personal account which may or may not be required by your course. We would recommend you do this activity regardless, to continue your process of personal growth through reflexivity.
\end{guidance}
\end{question}
%%Hack to correct tcbox behaviour
\color{black}

\chapter{Completing your dissertation}\label{ch:CompletingYourDissertation}

The end of your project is fast approaching: you can see the finishing line! What is left is to complete, review and submit your dissertation. 

You will already have a great proportion of your dissertation written as a result of diligently writing your reports in the previous stages. Therefore, your dissertation should extend your Stage 4 report by covering the work you have carried out in this stage and addressing any remaining gaps. The structure and content we recommend are indicated in \Cref{dissertationTemplate}, which should not come as a surprise to you as it follows the same structure we have used throughout this book. The guidance in the table indicates what we expect the content of each chapter to demonstrate once your dissertation is ready for submission, something you will check towards the end of this chapter.


Before proceeding, however, you should check that our template and guidance meet the expectations of your own course of study\footnote{While this structure is fairly typical, your course of study may have additional requirements, or provide a different template for you to use.}.


\begin{question}[subtitle={Activity: Your dissertation template}] 
Check the regulations of your course of study in relation to the required structure and  content of your dissertation, including whether a dissertation template is provided. 

Compare them with the structure and guidance in \Cref{dissertationTemplate}, make a note of all significant differences, and of any additional content you may need to develop and include.
\begin{guidance}
You should keep this comparison in mind throughout this chapter to ensure you include all required elements in your dissertation before submission. 
\end{guidance}\end{question}
%%Hack to correct tcbox behaviour
\color{black}

\begin{ReportTable}{dissertationTemplate}[Dissertation template and guidance]
%%\ReportTitle1
\ReportTitle*
	&Your title should capture succinctly your research problem and aim &\Cref{sect:ChoosingATitle}
\\

%%\ReportAbstract12345
\ReportAbstract*****
	&Your abstract should provide a succinct account of your research for a specialist audience, covering each of the bullet points indicated	&\Cref{sect:writingAbstract}
\\

%%1-\ReportIntroduction12345
\ReportIntroduction****
	&  This chapter should provide an introduction to your research topic in its wider context (as background) and your justification of why the research is worth pursuing. Its purpose is to introduce and justify your intended research in overview, before entering the detailed work of the subsequent chapters. It should be well argued and supported by appropriate citations and other evidence
	
	\item You can use a separate section for key technical definitions and acronyms used throughout your dissertation, for the reader's benefit	
	&\Cref{sect:stage1ChoosingATopic}\\

%%2-\ReportLitRev---
\ReportLitRev***
	& Your literature review should provide a critical account of your in-depth engagement with the academic (and other) relevant literature, including identifying key trends, ideas and knowledge gaps. Most of your citations should point to academic articles. Both coverage and depth of your review matter. 
	
	\item You should ensure your review is well structured, with a logical narrative flow and with arguments well supported by appropriate citations
	
	\item Your critical summary should highlight key insights from your review and the knowledge gap you research is going to address
	
	  &\Cref{sect:stage2literatureReview}\\

%%3-\ReportResearchDef----
\ReportResearchDef****
	& You should ensure that your research problem is well articulated, your aim  consistent with your research problem and broken down into appropriate objectives, and that the intended knowledge contribution is clearly expressed. You must ensure that these elements of your research definition form a coherent whole and clearly relate to each other &\Cref{sect:stage1ResearchProblem,sect:stage1AimAndObjectives}\\

%%4-\ReportResearchDes-----
\ReportResearchDes*****
	& This chapter should demonstrate your critical engagement with all elements of research design, including a detailed account of the data needed in your research, its sources and use, and the research strategy(ies) and methods you have chosen and applied
	
	\item Your choices should be appropriately justified in relation to your research problem, aim and objectives, and possibly with reference to accepted paradigms and approaches in your field of study
	
	\item You should provide a summary of the  procedures you have followed in the application of your research methods, including measures you have taken to deal with potential research weaknesses
	
	\item You should  demonstrate your careful consideration of ethical and regulatory matters relevant to your project, and that your research complies with your course and university requirements 
	
	&\Cref{ch:ResearchDesignFoundation,ch:yourResearchMethodology,ch:DataGenerationMethods,ch:ModellingMethods,ch:EthicsAndRegulations}
	\\

%%5-\ReportAnalysisInterp----
\ReportAnalysisInterp****
	& This chapter should provide a detailed account of your data generation and analysis, the findings you have derived and their interpretation in relation to your research aim and objectives 
	
	\item It should demonstrate a competent execution of your methodology, including providing appropriate summaries of your data, a clear account of your analysis, and how it led to the key findings in a logical manner 
		
	\item Your interpretation of findings in relation to aim and objectives should demonstrate in-depth critical reflection  
	&\Cref{ch:AnalysisMethods,sect:interpretingFindings}\\

%%6-\ReportEvalConc-------
\ReportEvalConc*******
	& This chapter should demonstrate good critical reflection on the extent your research has met its stated aim and objectives. In a succinct way, your conclusions should bring all your findings together, both from your literature review and your data generation, analysis and interpretation. 
	
	\item You should reflect on how your research has contributed new knowledge in relation to related published work, including highlighting how it is novel 
	
	\item You should evaluate your research in terms of its validity, reliability and lack of bias, highlighting measures you have taken to avoid research weaknesses, but also acknowledging limitations
	
	\item  You should discuss possible implications of your work for further research and, if applicable, for professional practice 
	
	\item Your concluding reflexive statement should highlight what you have learnt during your project from a personal standpoint in relation to thinking and behaving as an academic researcher
 &\Cref{sect:evaluateWholeResearch}\\

%%7-\ReportRefs-
\ReportRefs*
	& You references should be accurate and complete in relation to citations in the body of the dissertation &\Cref{ssect:Processing}\\

%%A-\ReportAppendices-
\ReportAppendices*
	& Appendices can be used to include supplementary material in support of the main body of your dissertation, for instance sample raw data, questionnaires used in data generation, programme code for models or artefacts, detailed calculations, etc. &\\
	
%%\ReportProgressTracking123456789
%\ReportProgressTracking*********
%& You should consider your progress against the checklist of \Cref{Stage5checklist} and record it here \Cref{stage5progress}
%\\
%%
%%%\ReportReflection123
%\ReportReflection***
% & In this section you should reflect on the progress you have made in Stage 5 using the material in \Cref{Stage5workplan}.
%\\
\end{ReportTable}
\endinput

\section{Checking assessment criteria}\label{ch:HowYourDissertation}

While preparing your dissertation for submission, you should be aware of how it's going to be assessed on your course of study, so that while reviewing it for submission you can improve its content against the assessment criteria.

That assessment is designed to ensure that your work is evaluated fairly against established academic benchmarks for research projects and your course learning outcomes\footnote{You looked at benchmarks and learning outcomes for Masters-level research very early on in \Cref{ssect:WhatYouWill}}. For some degrees, there may also be a \emph{viva voce}, an oral examination in which you are asked to defend your research in front of a panel of examiners\footnote{This is usually the case for doctoral degrees.}. 

The specifics of the assessment will depend on your own course of study. However, it will likely cover:

\begin{itemize}
\item \emph{novelty and contribution to knowledge}: these are essential elements of all research, although expectations of their significant are different in different degrees. While novelty is expected from all, at Masters level, a modest contribution to knowledge at the forefront of your field of study is sufficient, while at doctoral level a substantial knowledge increment is expected, which is worthy of publication in the academic literature 
\item \emph{research definition and research design}: this refers to the way you have defined and justified your research problem in its wider context, backed up by your critical review of the academic literature, articulated appropriate aim and objectives, and designed and justified a suitable and coherent research methodology
\item \emph{data generation, analysis, interpretation and conclusions}: this refers to a competent execution of your research methodology, the quality and quantity of your data, their analysis and interpretation, and your overall critical evaluation of the research including your awareness of its weaknesses and limitations
\item \emph{academic writing and presentation}: this refers to how your dissertation is put together, its logical arguments and flow, its cohesiveness and readability, and its conformance to presentation conventions
\item \emph{research skills}: this refers to an assessment of your skills as a researcher in relation to your level of research competence and independence, the latter being of great importance at doctoral level. 
\end{itemize}


\begin{question}[subtitle={Activity: Assessment on your course of study}] 
Check your course of study regulations in relation to the assessment of your dissertation, and map them to the elements listed above. Make a note of any extra assessment criteria.
\begin{guidance}
If you find it difficult to access those regulations, talk to your supervisor or course director who will be able to explain to you how your research dissertation will be assessed. 

It is particularly important that you are aware of any extra criteria which may not be captured in the list above. 
\end{guidance}\end{question}
%%Hack to correct tcbox behaviour
\color{black}


\section{Writing an abstract for your research}\label{sect:writingAbstract}

A research dissertation usually includes an abstract at the beginning. This is a common way to summarise academic research, and an integral part of all published academic articles -- you will have encountered many abstracts while reviewing the literature. 

Regardless of whether one is required on your course of study, writing an abstract for your research is a good exercise as it gives you an opportunity to write a logical argument that connects all key elements of your research. This can help you check that all the pieces fit together in a coherent manner. It is also something you can share with your supervisor and critical friends to communicate succinctly the essence of what you have done and achieved.  

Your {abstract} should provide a short summary of your whole research written for a specialist audience, that is you should assume that the reader has good knowledge of the topic and field of study. It should also be a stand-alone item, so that your reader should be able to understand its content without any reference to other parts of your dissertation.

The content of your abstract should convey succinctly:
\begin{itemize}
	\item your research problem, how and where it arises and its significance
	\item your research aim
	\item your methodology
	\item your contribution to knowledge and its implications
\end{itemize} 

\begin{question}[subtitle={Activity: Writing your abstract}]
Write an abstract for your project, ensuring it addresses in a succinct manner each of the points above.
\begin{guidance}
You should check whether your course requires a particular structure for your dissertation abstract: while most abstracts are written as a continuous narrative, abstracts which are structured around  each of the points above are becoming more common.

You should also look at how abstracts are written in past dissertations, possibly from your own course of study, as examples of ways you could express your own narrative. 

You may need to iterate few times to arrive at the final version of your abstract. It will also help you to share it with your supervisor and other critical readers -- family, friends or other students on your course, for feedback on how to improve it further.
\end{guidance}
\end{question}
%%Hack to correct tcbox behaviour
\color{black}


\section{Filling any remaining gaps}\label{ssect:fillingGaps}

By now, you should have something to say in each of the chapters and sections suggested in \Cref{dissertationTemplatec}.

Chapters 1 to 3 should be close to submission standard, while Chapter 4 should be almost complete, perhaps requiring some limited editing, to account for possible changes in methodology while completing your data generation and analysis. 

Instead, Chapters 5 and 6 are likely to require most of your effort at this point: while you may have developed much of their content as a result of the activities you have carried out in this stage, you will still need to bring it all together into a coherent whole.

\begin{question}[subtitle={Activity: Putting your whole dissertation together}] Starting from your previous report, complete your dissertation by applying the structure and guidance in \Cref{tab:dissertation}, and making good use of your notes and summaries from all related activities you have carried out.

\begin{guidance} 
If your course requires a different structure or template you should use that instead. Equally, if your course requires extra elements to be part of your dissertation,  this is the time to develop and include them in your full dissertation draft.
\end{guidance}
\end{question}
%%Hack to correct tcbox behaviour
\color{black}

\section{Compliance to presentation requirements}\label{ssect:presentationReqs}

Now that you have a complete draft of your dissertation, you should make all necessary adjustments to ensure it is presented in a way that meets your course requirements.	

\subsection{Word count}

An adjustment which is almost always needed is to trim the dissertation down to fit within the word limit established by your course of study\footnote{In our experience, we all have a tendency to write too much rather than too little!}. For instance, our institution sets word limits for all Masters and Doctoral dissertations, and gives precise rules on what is included or otherwise in that count: say, references, abstract and appendices are excluded, but figure and table captions are not, etc. 

Often, there is also an expectation that the content of your dissertation is balanced across its  chapters\footnote{Although it is normal for some chapters to be more substantial than others.} and your course of study may provide specific recommendations on the expected size of each chapter. 

We give an example in \Cref{tab:contentBreakdown} with reference to the dissertation structure of \Cref{dissertationTemplate}. In the example, we have used guidelines from our university for Masters level research dissertations in STEM\footnote{STEM stands for Science, Technology, Engineering and Mathematics.} taught programmes, where dissertations are usually between 10,000 and 15,000 words in length. Of course, these are only guidelines and should not be applied too rigidly: students can used them as a baseline, then adapt them to the needs of their project.


\begin{SimpleNColTable}{tab:contentBreakdown}{4}{Breakdown of dissertation content by chapter, as applied in STEM subjects in our university}[X[3]X[2]X[2]X[2]]
Element & Recommended word count distribution & Equivalent for 10,000 word dissertation & Equivalent for 15,000 word dissertation \\
Chapter 1 Introduction & 10\% & 1000 & 1500 \\
Chapter 2 Literature review & 20\% & 2000 & 3000 \\
Chapter 3 Research definition & 10\% & 1000 & 1500 \\
Chapter 4 Research design & 15\% & 1500 & 2250 \\
Chapter 5 Analysis and interpretation & 30\% & 3000 & 4500 \\
Chapter 6 Evaluation and conclusion & 15\% & 1500 & 2250 \\
\end{SimpleNColTable}


\begin{question}[subtitle={Activity: Complying with word limits on your course of study}] 
Check the regulations and guidelines of your course of study to establish any word limit for your dissertations overall and its parts. Edit your dissertation to ensure it does not exceed those word limits. 
\begin{guidance} 
You must take any word limits seriously as ignoring them may result in your work being penalised during assessment. There may be some tolerance around such limits, say, you can exceed them by 5\% without penalty, so you should check that too.

If your dissertation is well above the word limit, then it may take quite a bit of effort to reduce its content. You may need to rewrite some sections, provide more succinct summaries throughout or even take something out. 

You may be able to move some information in appendices at the end of the dissertation, however, you must check the guidelines of your course to make sure you use them appropriately. For instance, in our institution you can use appendices which are not included in the word count, but these are not assessed. As a result, they are usually used to provided additional information on topics which are covered in a more succinct way in the body of the dissertation, but are not a replacement. A typical use is to include sample raw data in an appendix, while their summaries and analysis remain in the dissertation body. In this way your work is still assessed, while you are also demonstrating the breadth and depth of the data you have generated.
\end{guidance}
\end{question}
%%Hack to correct tcbox behaviour
\color{black}


\subsection{Conventions}

Your dissertation will also need to conform to the presentation conventions established by your course of study. These usually address typographical aspects of your dissertation, like margins, line spacing, numbering, etc. \Cref{tab:presentationConventions} provides an example from our institution.

\begin{SimpleNColTable}{tab:presentationConventions}{2}{Example of dissertation presentation conventions}[X[2]X[6]]
Feature & Convention\\
Fonts & Use a standard font that is easy to read, e.g. Times New Roman or Arial, with font size 11 or 12 \\
Margins and spacing & Leave appropriate margins on both the left and the right of the page, typically around 2 cm. Use 1.5 line spacing \\
Title page & Include a title page containing your research title. Usually the following statement is also required: \enquote{A dissertation submitted in partial fulfilment of the requirements for the degree of $<$name of degree$>$}, where you should replace $<$name of degree$>$ with your own degree title \\
Your identifiers & Include your name and student identifier, possibly as a header or as part of the title page \\
 Table of content & Include a table of content after the title page \\
Page numbers & Number all pages, including references and appendices. In particular, use lower-case Roman numerals on the preliminary pages -- iii, iv, v, etc. -- and Arabic numerals starting from page 1 at the beginning of Chapter 1. \\
Chapter and section numbering & Number chapters sequentially using Arabic numerals starting with 1. Number sections sequentially starting with the chapter number, e.g. 1.1, 1.2, etc. for sections in Chapter 1. Number sub-sections sequentially starting with the section number, e.g. 1.1.1, 1.1.2, etc. for sub-sections in Section 1.1. You should avoid sub-sub-sections, but if needed, number them sequentially starting with the sub-section number, e.g. 1.1.1.1, 1.1.1.2, etc. for sub-sub-sections in Sub-section 1.1.1. \\
Figures and tables & Number all figures and tables sequentially, starting with their chapter number, e.g. 1.1, 1.2, etc. for figures in Chapter 1. Include appropriate captions positioned after figures and before tables \\
Lists of figures and tables & List all figures and tables after your table of content. For each include both their number and caption \\
Citations and references & Apply the required bibliographical style throughout \\
Verb tense & Your dissertation is an account of what you did in your project, so you should report your work using the past tense throughout \\
\end{SimpleNColTable}

\begin{question}[subtitle={Activity: Complying with presentation conventions on your course}] 
Check the presentation conventions used on your course of study and edit your dissertation accordingly. 
\begin{guidance} 
If your course does not have any specific conventions, then you could use those in the table which are fairly typical in the UK.
\end{guidance}
\end{question}
%%Hack to correct tcbox behaviour
\color{black}






%The assessment of your work under these criteria will contribute to your final grade, which will be established by your examiners in relation to Masters level quality benchmarks, including those\footnote{These are the typical criteria for the UK. Those in your country may vary.}\todo{Add source} summarised in Cref{tab:gradeBenchmarks}.

%\begin{SimpleNColTable}{tab:gradeBenchmarks}{2}{Typical grade benchmarks for Masters dissertations, based on UK quality standards}
%	
%Grade & Quality descriptor \\
%
%Distinction & All elements of the dissertation are present, including abstract and any required appendix, and are of a high standard. In particular, the dissertation demonstrates: advanced, authoritative understanding and analysis of key issues and complex problems;  strong data of a critical approach to own work and that of others;  competent use of a wide range of data in support of academic arguments;  appropriate and well justified selection of research strategies and methods, applied competently to own research;  originality and independence of thought;  compelling narrative which is coherently and logically presented;  excellent presentation standards;  excellent research potential \\
%Merit & All elements of the dissertation are present, including abstract and any required appendix, and are of a good standard. In particular, the dissertation demonstrates:  good understanding and analysis of key issues;  good data of a critical approach to own work and that of others;  good use of data in support of academic arguments;  appropriate selection of research strategies and methods, applied reasonably well to own research;  some originality;  coherent and logically presented narrative;  good presentation standards;  good research potential \\
%Pass & Some elements may be weak or missing, but all three perspectives above are sufficiently addressed. In particular, the dissertation may data some of:  limited understanding and analysis of key issues;  limited data of critical approach to own work and that of others;  limited use of relevant data in support of academic arguments;  some appropriate choices of research strategies and methods, but with limited application to own research;  plausible narrative;  adequate standards of presentation \\
%Weak fail & Many elements of the dissertations are very weak or missing, and not all three perspectives above are sufficiently addressed. In particular, the dissertation may data many or all of:  superficial understanding and analysis of key issues;  weak data of critical approach to own work and that of others;  gaps in the use of data in support of academic arguments;  inappropriate choice or application of research strategies and methods;  weak narrative;  poor standards of presentation \\
%	&(In this case, a course may allow some remedial work and resubmission.)\\
%Complete Fail & The dissertation has critical flaws and omissions, so that is not recoverable via a resubmission. In particular, the dissertation demonstrate many or all of:  lack of understanding and analysis of key issues;  lack of critical approach to own work and that of others;  little or no use of data in support of academic arguments;  inappropriate choice or application of research strategies and methods;  incoherent and confused narrative;  inadequate standards of presentation \\
%\end{SimpleNColTable}

\section{Getting ready for submission}\label{sect:finalReview}

In this final step, you will review your completed dissertation draft to improve its quality in view of your submission. Assuming you have filled in all gaps and applied all necessary presentation conventions, in your final effort you should focus on:

\begin{itemize}
	\item \emph{logical coherence}: you should ensure that all research elements of your dissertations are coherent and consistent with each other, so that there is a logical progression from research problem, to aim and objectives, to research design and its execution, to data analysis and interpretation, and to your conclusions
	\item \emph{academic writing}: you should ensure that your academic arguments are well formed, including being well-supported by secondary and/or primary data, that the language you use is concise and precise, that the work overall shows good critical reflection and there is a good balance between description, and analysis and synthesis	
	\item \emph{proof-reading}: you should remove grammatical errors and typos, use punctuation correctly, and ensure your language is clear and relatively easy to follow even by a non-expert reader
	\item \emph{course-specific requirements}: you should take into consideration any requirements established by your course of study in addition to what is recommended in this book.
\end{itemize}

\begin{question}[subtitle={Activity: Your final dissertation revision}] 
Review your dissertation in relation to each of the points above and make all necessary adjustments. You may need to iterate a number of times. 

\begin{guidance}
Revising your dissertation for submission is very important as you can lose a substantial proportion of marks should any of these aspects not be addressed carefully and to the expected standards.

To check logical coherence, you should outline the main research elements and arguments of your dissertation and ensure they are consistent and follow from each other. This is, of course, something you have been doing all along, stage by stage, for instance, ensuring that your aim contributes to your research problem, that your research objectives break down your aim appropriately, that your chosen methods are suitable for your objectives, etc. However, several things will have changed during your research project timespan, often unexpectedly, therefore this final check will give you the opportunity to spot, and deal with, anything which may have become misaligned.

You have learnt the key characteristics of good academic writing and of academic arguments early on in this book in \Cref{sect:CorePracticeFor,sect:DevelopYourArguments}.You should refer back to the content of those section as you complete your final pass at your dissertation.

Removing typos and grammatical errors should be relatively easy if you make good use of the grammar and spell checking features of your word processor. If you don't feel those are sufficient, then you could use one of the many specialised grammar checking tools which are now available.

To improve readability, we strongly advise you ask, once again, your family and friends for help: even if they are not experts on the topic of your project, they will be able to tell you whether they can follow what you have written and get the gist of your work.

Last, but not least, your supervisor will be able to point to any aspect of your dissertation which should be improved before submission.
\end{guidance}
\end{question}
%%Hack to correct tcbox behaviour
\color{black}

Congratulations! By completing this activity, you will have also completed your project work, and all is left for you to do is to submit your dissertation by following the instructions of your course of study.


\chapter*{Stage 5 Takeaways}\label{ch:Stage5Takeaways}
\begin{itemize}
\item Your research findings are the main insights from your data analysis
\item Interpreting findings means arguing how they may help you claim that you have met your research aim and objectives, while acknowledging both their significance and limitations
\item In evaluating your overall research you need to argue if and how it has met its stated research aim and objective, and the extent it is valid, reliable and free of bias
\item In your conclusion, you should demonstrate that you are aware of ways in which your research may inform future research, and any implication for professional practice
\item Your concluding reflexive statement gives you an opportunity to explore how your project has changed you as an academic researcher
\item Your dissertation must meet the assessment and presentations requirements of your course of study
\item  An abstract at the beginning of your dissertation is a common way to summarise and connect all key elements of your research for a specialist audience
\end{itemize}

%%Sectional bibliography
%\printbibliography[segment=\therefsegment,title=Stage 5 \bibname]
