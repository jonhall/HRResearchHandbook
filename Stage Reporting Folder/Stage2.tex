\section*{Stage 2: Research Definition and Design}

\begin{reportblr}{}
Part/Structure&Guidance\\
%%\ReportTitle1
\ReportTitle*
	&At this stage, update your provisional title if necessary. This will now be close to its final version &\cref{stage2title}\\

%%\ReportAbstract12345
\ReportAbstract-----
	&\tablenocontent
\\

%%1-\ReportIntroduction12345
\ReportIntroduction*!!--
	& At this stage you should be able to give an introduction to your research topic and its wider context as your Introduction, but without going into details. This will continue to change as your understanding develops in later stages &\cref{stage2ChoosingATopic}%
\\

%%2-\ReportLitRev---
\ReportLitRev***

	& Your literature review should now be taking good shape as you get through the lion's share of the reading you will complete. Your review should be the outcome of your deep engagement with the academic (and other) literature around your chosen topic, building on the work you started in Stage 1. It should be a structured and include a good critical summary of the key insight from your review of the literature, including articulating and justifying the knowledge gap your project will address. 

	%%\itemize in tblr header
	\item There may be small additions and adjustments required later on in your project, but the main body of your literature review should now be completed. &\cref{stage2litrev}
\\

%%3-\ReportResearchDef----
\ReportResearchDef****
	& At this stage you should be able to describe your candidate research problem in terms of your problem statement, aim and knowledge contribution. This will change as your understanding deepens through the stages &\cref{stage2resdef}\\

%%4-\ReportResearchDes-----
\ReportResearchDes*!!-!
	& At this stage you should be able to give a detailed account of all data needed in the project and its uses, for instance, as evidence in support of hypotheses, etc. 
		
	%%\itemize in tblr header
	\item You should be able to provide a detailed, critical account of your research design choices, including choice of strategy, and reason for that choice, and related methods, showing your engagement with relevant academic literature, and good understanding of pros and cons, and the criteria by which your research will be evaluated
	
	%%\itemize in tblr header
	\item You should provide evidence that ethical and legal considerations for your research have been considered, including a statement showing your organisation's permission to perform the research &\cref{stage2data,stage2resdes,stage2ethics}
\\

%%5-\ReportAnalysisInterp----
\ReportAnalysisInterp----
	& \tablenocontent
\\

%%6-\ReportEvalConc-------
\ReportEvalConc-------
	& \tablenocontent\\

%%7-\ReportRefs-
\ReportRefs*
	& \tablecites{}\\

%%A-\ReportAppendices-
\ReportAppendices*
	& See below\\
	
%%\ReportProgressTracking123456789
\ReportProgressTracking*********
& You should consider your progress against the appropriate checklist and record it here&\cref{Stage2checklist}
\\
%
%%\ReportReflection123
\ReportReflection***
 & In this section you should reflect on the progress you have made in Stage 2 using the appropriate material and set your key priorities for work in Stage 3&\cref{Stage2workplan} 
\\
\end{reportblr}
\endinput
