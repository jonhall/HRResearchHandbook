\section*{Stage 2}

\begin{reportblr}{}
Part/Structure&Guidance\\
%%\ReportTitle1
\ReportTitle*
	&At this stage, update your provisional title if necessary. This will now be close to its final version \cref{stage2title}\\

%%\ReportAbstract12345
\ReportAbstract***-*
	&At this stage, update your provisional abstract to capture your growing understanding of your research. You should now be able to structure the abstract to provide more detail for the reader on the Background, Research Outline and the Knowledge Contributions you intend to make 	\cref{stage2abstract}
\\

%%1-\ReportIntroduction1234
\ReportIntroduction*!!-
	& At this stage you should be able to give an introduction to your research topic and its wider context as your Introduction, but without going into details. This will continue to change as your understanding develops in later stages \cref{stage2ChoosingATopic}%
\\

%%2-\ReportLitRev---
\ReportLitRev***

	& Your literature review should now be taking good shape as you get through the lion's share of the reading you will complete. Your review should be the outcome of your deep engagement with the academic (and other) literature around your chosen topic, building on the work you started in Stage 1. It should be a structured and include a good critical summary of the key insight from your review of the literature, including articulating and justifying the knowledge gap your project will address. 
	
There may be small additions and adjustments required later on in your project, but the main body of your literature review should now be completed. \cref{stage2litrev}
\\

%%3-\ReportResearchDef----
\ReportResearchDef****
	& At this stage you should be able to describe your candidate research problem in terms of your problem statement, aim and knowledge contribution. This will change as your understanding deepens through the stages \cref{stage2resdef}\\

%%4-\ReportResearchDes-----
\ReportResearchDes*!!-!
	& at this stage you should be able to give a detailed account of all data needed in the project and its uses, for instance, as evidence in support of hypotheses, etc. \cref{stage2data}
	
	You should be able to provide a detailed, critical account of your research design choices, including choice of strategy, and reason for that choice, and related methods, showing your engagement with relevant academic literature, and good understanding of pros and cons, and the criteria by which your research will be evaluated \cref{stage2resdes}
	
	You should provide evidence that ethical and legal considerations for your research have been considered, including a statement showing your organisation's permission to perform the research \cref{stage2ethics}
\\

%%5-\ReportAnalysisInterp----
\ReportAnalysisInterp----
	& no content at this stage
\\

%%6-\ReportEvalConc-------
\ReportEvalConc-------
	& no content at this stage\\

%%7-\ReportRefs-
\ReportRefs-
	& You should include those references that you cite in previous chapters.\\

%%A-\ReportAppendices-
\ReportAppendices-
	& See below\\
	
%%\ReportProgressTracking123456789
\ReportProgressTracking*********
& You should consider your progress against the checklist of \cref{Stage0checklist} and record it here.
\\
%
%%\ReportReflection123
\ReportReflection***
 & In this section you should reflect on the progress you have made in Stage 0 using the material in \cref{Stage0workplan} and set your key priorities for wok in Stage 1.
\\
\end{reportblr}
\endinput
