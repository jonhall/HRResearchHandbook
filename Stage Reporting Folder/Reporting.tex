\documentclass[10pt]{article}

\title{ Report content }
\author{Jon G.~Hall}
%%Uncomment for date and version
%\date{\footnotesize\today; Version: \version}

\usepackage[margin=1cm]{geometry}

%%Migrating to tabularray as for tabular
\RequirePackage{tabularray}
\UseTblrLibrary{varwidth}%%For itemize in entries
\UseTblrLibrary{booktabs}%%For lines in entries
%\newcommand{\zerobox}[1]{\setbox0=\vbox{#1}}
%\NewColumnType{Z}{Q[cmd=\zerobox,wd=0pt]}%%Add column type that deletes a column

\usepackage{ninecolors}
\usepackage{tcolorbox}

\NewTblrEnviron{mytblr}
\SetTblrInner[mytblr]{hlines,vlines}
\SetTblrOuter[mytblr]{baseline=B}

\NewTblrEnviron{reportblr}
\SetTblrOuter[reportblr]{long}
\SetTblrInner[reportblr]{measure=vbox, stretch=-1,%Allows lists in body
  hlines = {white}, %column{2} = {co=1}, 
  colsep = 5pt,
  row{odd} = {brown8}, row{even} = {gray8},
  row{1} = {fg=white, bg=purple2, font=\bfseries\sffamily},
  rowhead = 1,
}

%%To define reportenumerate list for report tables
\usepackage{enumitem}
\newlist{reportenum}{enumerate}{2}
\setlist[reportenum]{nosep}
\newlist{reportitem}{itemize}{2}
\setlist[reportitem]{nosep}

\begin{document}
\maketitle

\section{Stage 1}

\begin{reportblr}{X[8,l]X[24]X[1,c,wd=15pt]}%[
Part/Structure&Guidance&Ref\\
\SetCell[c=2]{l}{\textbf{Title}}\\
	&Captures succinctly your research problem and aim.
	&\ref{sect:title}\\
\SetCell[c=2]{l}Introduction\\
\begin{reportenum}
	\item[1.1] Background to the research 
	\item[1.2] Justification for the research 
	\item[1.3] Fitness of the research
\end{reportenum}
 & Provides an introduction to your research topic in its wider context (as background) and your justification of why the research is worth pursuing. Its purpose is to given an overview and justify your intended research  before entering the detailed work of the subsequent sections. This section should: 
 %
\begin{itemize}[nosep]
\item be well argued and supported by appropriate citations, 
\item argue how the research fits within the scope of your qualification, and 
\item describe how it meets any other personal, professional or organisational criteria that might exist.
\end{itemize}
&
\ref{sect:stage1ChoosingATopic}\\
\SetCell[c=2]{l}{\textbf{Literature review}}\\
%
\begin{reportenum}%[start=0,label={(\bfseries R\arabic*):}]
\item [2.1] Draft review of relevant knowledge 
\item [2.2] Planned further literature review
\end{reportenum}
%
 & Based on the theme summaries you constructed towards the end of Section~\ref{sect:stage1LiteratureReview}. You will build on this content in Stage 2 to write an extensive, well-argued literature review to demonstrate your in-depth engagement with the academic (and other) relevant literature. Your planned review should identify further reading you may still have to undertake in Stage 2, although the expectation is that the bulk of your reading (and writing) has taken place in Stage 1, so that you can focus on synthesising your knowledge and understanding in the next stage
 &\ref{sect:stage1LiteratureReview}\\
\SetCell[c=2]{l}{\textbf{Research definition}}\\
\begin{reportenum}
	\item[3.1] Problem statement 
	\item [3.2] Aim and objectives 
	\item [3.3] Knowledge contribution 
	\end{reportenum}
	& You should ensure that your research problem statement is well articulated, that your aim and objectives are consistent with the research problem, and that the intended knowledge contribution of your research is clearly argued. %You should refer back to the activities you conducted in %Sections~\ref{sect:stage1ResearchProblem} and \ref{sect:stage1AimAndObjectives}, but also to the theme summaries you produced at the end of Section~\ref{sect:stage1LiteratureReview}, particularly those which highlight knowledge gaps of interest
	&\ref{sect:stage1ResearchProblem} \ref{sect:stage1AimAndObjectives} \ref{sect:stage1LiteratureReview}\\
\SetCell[c=2]{l}{\textbf{Research design}}\\
%
\begin{reportenum}%[start=0,label={(\bfseries R\arabic*):}]
\item [4.1] Evidence and data 
\item [4.2] Research methods 
\item [4.3] Ethics and regulations
\end{reportenum}
%
  & Demonstrate your initial engagement with research design, particularly that you have thought about the kind of evidence and methods you may need, appropriately justified in relation to your research problem, aim and objectives. It should also demonstrate your careful consideration of ethics and regulations, and that your research will comply with your course and university requirements.
  &\ref{sect:stage1ResearchDesign}\\
\SetCell[c=2]{l}{\textbf{Work planning and risk assessment}}\\
%
\begin{reportenum}%[start=0,label={(\bfseries R\arabic*):}]
\item [5.1] Statement of progress 
\item [5.2] Key priorities in follow-up stage 
\item [5.3] Risk assessment 
\end{reportenum}
%
 & In this section you should reflect on the progress you have made in Stage 1 in relation to your initial work plan and establish your priorities for the next stage. For this, you should revise your initial work plan (\ref{sect:stage1WorkPlan}) accordingly. You should also summarise the outcome of your project risk assessment (\ref{sect:stage1ManagingRisk}).
 &\ref{sect:stage1WorkPlan} \ref{sect:stage1ManagingRisk}\\
\SetCell[c=2]{l}{\textbf{References}}\\
& You should keep your references in good order and ensure you apply the required bibliographical style consistently. Ideally, you should use a BMT to generate and integrate your references within your report \\
 \SetCell[c=2]{l}{\textbf{Appendix - Work schedule}}\\
 & You could include it as an appendix for reference \\
 \SetCell[c=2]{l}{\textbf{Appendix - Risk assessment table}}\\
 & You could include your filled-in risk table as an appendix for reference \\
\end{reportblr}
\end{document}

\begin{longtblr}
	\caption{Recommended structure of your research proposal \label{tab:researchProposalStructure}}
\begin{ltabulary}{\tablewidth} {@{}LL@{}} \toprule
 \textbf{Report template} & \textbf{Guidance} \endhead\midrule
 Proposed title & Your title should capture succinctly your research problem and aim. You should refer back to Section~\ref{sect:title} for more guidance.\\
 Sect 1 - Introduction 1.1 Background to the research 1.2 Justification for the research 1.3 Fitness of the research & This section should provide an introduction to your research topic in its wider context (as background) and your justification of why the research is worth pursuing. Its purpose is to introduce and justify your intended research in overview, before entering the detailed work of the subsequent sections. It should be well argued and supported by appropriate citations. In this section, you should also argue how the research fits within the scope of your qualification, and meets any other personal, professional or organisational criteria. Revising Section~\ref{sect:stage1ChoosingATopic} should help you with this task. \\
 Sect 2 - Literature review 2.1 Draft review of relevant knowledge 2.2 Planned further literature review & This section should be based on the theme summaries you constructed towards the end of Section~\ref{sect:stage1LiteratureReview}. You will build on this content in Stage 2 to write an extensive, well-argued literature review to demonstrate your in-depth engagement with the academic (and other) relevant literature. Your planned review should identify further reading you may still have to undertake in Stage 2, although the expectation is that the bulk of your reading has taken place in Stage 1, so that you can focus on synthesising your knowledge and understanding in the next stage.\\
 Sect 3 - Research definition 3.1 Problem statement 3.2 Aim and objectives 3.3 Knowledge contribution & You should ensure that your research problem is well articulated, that your aim and objectives are consistent with the research problem, and that the intended knowledge contribution of your research is clearly argued. You should refer back to the activities you conducted in Sections~\ref{sect:stage1ResearchProblem} and \ref{sect:stage1AimAndObjectives}, but also to the theme summaries you produced at the end of Section~\ref{sect:stage1LiteratureReview}, particularly those which highlight knowledge gaps of interest. \\
 Sect 4 - Research design 4.1 Evidence and data 4.2 Research methods 4.3 Ethics and regulations & This section should demonstrate your initial engagement with research design, particularly that you have thought about the kind of evidence and methods you may need, appropriately justified in relation to your research problem, aim and objectives. It should also demonstrate your careful consideration of ethics and regulations, and that your research will comply with your course and university requirements. You should refer back to Section~\ref{sect:stage1ResearchDesign} to develop the content of this section. \\
 Sect 5 - Work planning and risk assessment 5.1 Statement of progress 5.2 Key priorities in follow-up stage 5.3 Risk assessment & In this section you should reflect on the progress you have made in Stage 1 in relation to your initial work plan and establish your priorities for the next stage. For this, you should refer back to Section~\ref{sect:stage1WorkPlan} and revise your initial plan accordingly. You should also summarise the outcome of your project risk assessment (see Section~\ref{sect:stage1ManagingRisk}). \\
 References & You should keep your references in good order and ensure you apply the required bibliographical style consistently. Ideally, you should use a BMT to generate and integrate your references within your report \\
 Appendix - Work schedule & You could include it as an appendix for reference \\
 Appendix - Risk assessment table & You could include your filled-in risk table as an appendix for reference \\
\bottomrule
\end{ltabulary}
\end{longtblr}

\end{document}

\section{Stage 2}

\begin{table}[htbp]
\caption{Report structure and guidance guidance\label{tab:S2report}}
\centering
\begin{tabulary}{\tablewidth}{@{}LL@{}} \toprule
 \textbf{Report template} & \textbf{Guidance} \\
\midrule

 Proposed title & Your title should continue to capture succinctly your research problem and aim. \textit{It is likely this is the same as, or very similar to, that in Stage 1}\\
 Sect 1 - Introduction 1.1 Background to the research 1.2 Justification for the research 1.3 Fitness of the research & This section should continue to provide an introduction to your research topic in its wider context (as background) and your justification of why the research is worth pursuing. Its purpose is to introduce and justify your intended research in overview, before entering the detailed work of the subsequent sections. It should be well argued and supported by appropriate citations. In this section, you should also argue how the research fits within the scope of your qualification, and meets any other personal, professional or organisational criteria. \textit{You may review this section from Stage 1 to reflect your growing understanding of the topic in context derived from your literature review.} \\
 Sect 2 - Literature review 2.1 Introduction 2.2 Main body 2.3 Critical summary & \textit{This section should consist of your current literature review, developed in this stage by following the advice in Section~\ref{sect:stage2literatureReview}}. At this point, it should be a substantial, almost complete, draft, well structured and articulated through solid academic arguments. It should demonstrate your understanding of the main literature which relates to your research problem, clearly identifying the knowledge gap your project will address.\\
 Sect 3 - Research definition 3.1 Problem statement 3.2 Aim and objectives 3.3 Knowledge contribution & You should continue to ensure that your research problem is well articulated and appropriate for your course and your personal and professional circumstances, that your aim and objectives are consistent with research problem, and that the intended knowledge contribution of your research is clearly articulated. \textit{You may revise this section from Stage 1 in view of your increasing understanding from your literature review.} \\
 Sect 4 - Research design 4.1 Evidence and data 4.2 Research strategies and methods 4.3 Ethical, legal and EDI considerations & \textit{This section should extend your Stage 1 work with your considerations of candidate research strategies and methods for your project, based on the guidance in Section~\ref{sect:stage2researchDesign}}.  \\
 Sect 5 - Work planning and risk assessment 5.1 Statement of progress 5.2 Key priorities in follow-up stage 5.3 Risk assessment & In this section you should reflect on the progress you have made in Stage 2 and establish your priorities for the next stage. You should also review your risk assessment as appropriate. \\
 References & You should keep your growing references in good order and ensure you apply the required bibliographical style consistently. Ideally, you should use a BMT to generate and integrate your references within your report \\
 Appendix - Work schedule & You should include your revised work plan as an appendix \\
 Appendix - Risk assessment table & You should include your updated risk table as an appendix \\
\bottomrule
\end{tabulary}
\end{table}

\section{Stage 3}

\begin{ReportStructureTable}{tab:reportStructure}
\tabletitle{Title} & Your title should succinctly capture your research problem and aim\\\\
\tabletitle{Section 1: Introduction}\\
\begin{enumerate}[label={1.\arabic*:}]
\item Background to the research 
\item Justification for the research 
\end{enumerate}
& This section should provide an introduction to your research topic in its wider context (as background) and your justification of why the research is worth pursuing. It should be well articulated and supported by evidence \\\\
\tabletitle{Section 2: Literature review}\\
\begin{enumerate}[label={2.\arabic*:}]
\item Review of existing relevant knowledge 
\item Critical summary, including knowledge gap to be addressed by the research 
\end{enumerate}
& Your review should provide a critical account of your in-depth engagement with the academic (and other) relevant literature, including identifying key trends, ideas and possible knowledge gaps. Most of your citations should point to academic articles. Your critical summary should highlight key insights from your review and provide a strong justification for your proposed research. Both coverage and depth of your review matter. You should ensure that your review is well structured, with a logical narrative flow and your arguments are well supported by evidence  \\\\
\tabletitle{Section 3: Research definition}\\
\begin{enumerate}[label={3.\arabic*:}]
\item Problem statement 
\item Aim, objectives, tasks and deliverables
\item Knowledge contribution
\end{enumerate}
& You should ensure that your research problem is well articulated and appropriate for your course and your personal and professional circumstances, that your aim and objectives are consistent with research problem, that tasks and deliverables break down your objectives appropriately and are clearly related to your chosen research methods, and that the intended knowledge contribution of your research is clearly articulated \\
\tabletitle{Section 4: Research design}\\\\
\begin{enumerate}[label={4.\arabic*:}]
\item Evidence and data 
\item Research strategy and methods
\item Research procedures
\item Ethical, legal and EDI considerations
\end{enumerate}
& This section should demonstrated your critical engagement with all elements of research design, including a detailed account of the data and evidence needed in your research, the research methods and research strategies you will to apply, and how you will apply them within your project. Your account should be supported by a clear rationale and insights from the related literature, and appropriately justified in relation to your research problem, aim and objectives. It should also demonstrate your careful consideration of ethical and legal matters, and that your research will comply with your course and university requirements\\\\
\tabletitle{Section 5: Analysis and interpretation}\\
\begin{enumerate}[label={5.\arabic*:}]
\item Pilot work
\end{enumerate}
& This section should report on a well thought-out pilot work which clearly and competently test some significant aspect of your research design. It should demonstrate good critical reflection on outcomes and highlight any adjustments needed as a result. \\\\
\tabletitle{Section 6: Assessment of your proposed research}\\
\begin{enumerate}[label={6.\arabic*:}]
\item Qualification fit
\item Personal and professional fit
\item Technical skills and resources needed
\item Statement of feasibility
\item Personal reflection on research process
\end{enumerate}
& In this section you should continue to argue how your research is a good fit across all criteria. You should provide a clear rationale as to why you think what you are proposing is feasible. You should also reflect on your growing understanding of the research process, including key learning and aspects you have found particularly challenging. \\\\
\tabletitle{Section 7: Planning, scheduling and risk assessment}\\
\begin{enumerate}[label={7.\arabic*:}]
\item Key priorities in follow-up stage
\item Personal and professional fit
\item Risk assessment
\end{enumerate}
& In this section you should reflect on the progress you have made in Stage 2 and establish your priorities for the next stage. You should also review your risk assessment as appropriate.\\\\
\tabletitle{Section 8: References}\\ & You should keep your growing references in good order and ensure you apply the required bibliographical style consistently. Ideally, you should use a BMT to generate and integrate your references within your report\\\\
\textbf{Appendix A: Work schedule}& Your revised work plan\\\\
\textbf{Appendix B: Risk assessment table}& Your revised risk table \\
\bottomrule
\end{ReportStructureTable}

\section{Stage 4}

\begin{table}[htbp]
\caption{Report structure and guidance \label{tab:S4report}}
\centering
\begin{tabulary}{\tablewidth}{@{}LL@{}} \toprule
 \textbf{Report template} & \textbf{Guidance} \\
\midrule

 Proposed title & Your title should continue to capture succinctly your research problem and aim. \textit{It is likely this is the same as, or very similar to, that in Stage 3}\\
 Abstract & You should include your draft abstract providing a succinct account of your research to date \\
 Sect 1 - Introduction 1.1 Background to the research 1.2 Justification for the research 1.3 fitness of the research & This section should continue to provide an introduction to your research topic in its wider context (as background) and your justification of why the research is worth pursuing. Its purpose is to introduce and justify your intended research in overview, before entering the detailed work of the subsequent sections. It should be well argued and supported by appropriate citations. In this section, you should also argue how the research fits within the scope of your qualification, and meets any other personal, professional or organisational criteria. \textit{You may review this section from Stage 1 to reflect your growing understanding of the topic in context derived from your literature review.} \\
 Sect 2 - Literature review 2.1 Review of existing relevant knowledge 2.2 Critical summary, including knowledge gap to be addressed by the research & Your review should provide a critical account of your in-depth engagement with the academic (and other) relevant literature, including identifying key trends, ideas and possible knowledge gaps. Most of your citations should point to academic articles. Your critical summary should highlight key insights from your review and provide a strong justification for your proposed research. Both coverage and depth of your review matter. You should ensure that your review is well structured, with a logical narrative flow and your arguments are well supported by evidence \\
 Sect 3 - Research definition 3.1 Problem statement 3.2 Aim, objectives, tasks and deliverables 3.3 Knowledge contribution & You should ensure that your research problem is well articulated and appropriate for your course and your personal and professional circumstances, that your aim and objectives are consistent with research problem, that tasks and deliverables break down your objectives appropriately and are clearly related to your chosen research methods, and that the intended knowledge contribution of your research is clearly articulated \\
  Sect 4 - Research design 4.1 Evidence and data 4.2 Research strategy and methods 4.3 Research procedures 4.4 Ethical, legal and EDI considerations & This section should demonstrate your critical engagement with all elements of research design, including a detailed account of the data and evidence needed in your research, the research methods and research strategies chosen, with justification, and applied within your project. Your account should be supported by a clear rationale and insights from the related literature, and appropriately justified in relation to your research problem, aim and objectives. It should also demonstrate your careful consideration of ethical and legal matters, and that your research complies with your course and university requirements \\
  Sect 5 - Analysis and interpretation 5.1 Summary and analysis of evidence 5.2 Summary of key findings 5.3 Interpretation in relation to aim and objectives & This section should demonstrate substantial progress towards generating and analysing your data and evidence, and interpreting them in relation to aim and objectives. It should demonstrate a competent execution of your research design, present appropriate summaries of evidence and data, supported by raw data in an appendix if needed. Key findings should be clearly identified and logically connected to evidence, with good critical reflection on their implications for aim and objectives. \\
  Sect 6 - Work planning and risk assessment 6.1 Statement of progress 6.2 Key priorities in follow-up stage 6.3 Risk assessment & In this section you should reflect on the progress you have made in Stage 4 and establish your priorities for the next stage. You should also review your risk assessment as appropriate. \\
 References & You should keep your growing references in good order and ensure you apply the required bibliographical style consistently. \\
 Appendix - Raw evidence & If relevant, you should include a sample of your raw data as an appendix \\
 Appendix - Work schedule & You should include your revised work plan as an appendix \\
 Appendix - Risk assessment table & You should include your updated risk table as an appendix \\
\bottomrule
\end{tabulary}
\end{table}

\section{Stage 5}

\begin{table}[htbp]
\caption{Dissertation structure and guidance\label{tab:dissertation}}
\centering
\begin{tabulary}{\tablewidth}{@{}LL@{}} \toprule
 \textbf{Dissertation template} & \textbf{Guidance} \\
\midrule

Title & Your title should capture succinctly your research problem and aim \\
 Abstract & Your abstract should providing a succinct account of your research \\
 Chapter 1: Introduction 1.1 Background to the research 1.2 Justification for the research 1.3 Fitness of the research & This chapter should provide an introduction to your research topic in its wider context (as background) and your justification of why the research is worth pursuing. Its purpose is to introduce and justify your intended research in overview, before entering the detailed work of the subsequent chapters. It should be well argued and supported by appropriate citations. In this chapter, you should also argue how the research fits within the scope of your qualification, and meets any other personal, professional or organisational criteria. \\
 Chapter 2: Literature review 2.1 Review of existing relevant knowledge 2.2 Critical summary, including knowledge gap to be addressed by the research & Your review should provide a critical account of your in-depth engagement with the academic (and other) relevant literature, including identifying key trends, ideas and possible knowledge gaps. Most of your citations should point to academic articles. Your critical summary should highlight key insights from your review and provide a strong justification for your proposed research. Both coverage and depth of your review matter. You should ensure that your review is well structured, with a logical narrative flow and your arguments are well supported by data \\
 Chapter 3: Research definition 3.1 Problem statement 3.2 Aim, objectives, tasks and deliverables 3.3 Knowledge contribution & You should ensure that your research problem is well articulated and appropriate for your course and your personal and professional circumstances, that your aim and objectives are consistent with research problem, that tasks and deliverables break down your objectives appropriately and are clearly related to your chosen research methods, and that the intended knowledge contribution of your research is clearly articulated \\
  Chapter 4: Research design 4.1 Data 4.2 Research strategy and methods 4.3 Research procedures 4.4 Ethical, legal and EDI considerations & This chapter should demonstrated your critical engagement with all elements of research design, including a detailed account of the data needed in your research, the research methods and research strategies chosen, with justification, and applied within your project. Your account should be supported by a clear rationale and insights from the related literature, and appropriately justified in relation to your research problem, aim and objectives. It should also demonstrate your careful consideration of ethical and legal matters, and that your research complies with your course and university requirements \\
  Chapter 5: Analysis and interpretation 5.1 Summary and analysis of data 5.2 Summary of key findings 5.3 Interpretation in relation to aim and objectives & This chapter should provide a detailed account of your data generating, analysis, the findings you have derived and their interpretation in relation to your research aim and objectives. It should demonstrate a competent execution of your research design, present appropriate summaries of data, supported by raw data in an appendix if needed. Key findings should be clearly identified and logically connected to data, with good critical reflection on their implications for aim and objectives. \\
  Chapter 6: Evaluation and conclusion 6.1 Evaluation against aim and objectives 6.2 Evaluation against related work in the literature 6.3 Implication for practice 6.4 Validity of the research 6.5 Further work 6.6 Personal reflection on your experience of & In this chapter you should reflect on the extent your research has met its stated aim and objectives, bringing together all your findings from both primary and secondary research work. You should also reflect how it has contributed new knowledge in relation to the literature you have reviewed. You should also assess the validity of your research and consider any implication for further research and, if applicable, for professional practice. You should also reflect on what you have learnt from a personal standpoint in relation to thinking and behaving as an academic researcher. \\
 References & You should include all your references and ensure you apply the required bibliographical style consistently.\\
 Appendix - Raw data & If relevant, you should include a sample of your raw data as an appendix \\
\bottomrule
\end{tabulary}
\end{table}

\end{document}