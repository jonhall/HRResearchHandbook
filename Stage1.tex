\chapter{Stage 1: Preparing your research proposal}
At some point -- not too far from starting your project! -- you'll have completed a well articulated research proposal. As you'll see below, a good research proposal has many characteristics and these make it a difficult document to get right.

Starting your proposal and getting it into a workable state is the focus of this chapter: Stage 1 --  Preparing your research proposal.

\section{What is a research proposal?}
A research proposal is simply a detailed outline of your intended research project. It'll be 8 to 10 pages long and its main focus will be to convince the reader -- your supervisor, your examiner, and perhaps any family members or friends that help you -- that a research problem actually exists\footnote{Remember that research is a quest for new knowledge. Your research proposal must be convincing that current knowledge doesn't already have this covered.} and you have some initial ideas of how to approach it.

To be able to do this, your research proposal must be appropriately contextualised in the academic literature and, if relevant, also in professional practice, and should identify the knowledge gap you are going to address\footnote{Together, these give answers to the ``What?'', ``Why'', and ``How?'' questions that motivate your research.}. It should also include some early considerations of the intended research design, together with an initial risk assessment.

The research proposal is \emph{your} initial understanding of where you want your research to go.

\subsubsection{Do I really have to do a research proposal?}
It may be that you have chosen to research a topic proposed by your supervisor(s) which offers a perfectly good research proposal associated with it.

It might even be that you've chosen the topic because you know that there's already a perfectly good research proposal written!

Even if a research proposal is not required in your studies, we strongly recommend you still go through the process of developing one, even if it's just to rewrite a pre-existing one in your own language.

There are some very good reasons to do this:

\begin{itemize}
\item It will provide a foundation for your project that you understand, making it more solid

\item It will help you clarify your initial thinking --- which you're going to have to do anyway

\item It will show your supervisor that you have something distinct to contribute to the research

\item It will force you to think about the risks you face and how you will handle them --  again, you're going to have to do this anyway

\item Most importantly, it will then be yours --  you'll take ownership of your own research.

\end{itemize}

So, our recommendation is that you develop your own research proposal even if one exists as it will help you think deeply about what you are about to embark on.

\subsubsection{Welcome to Stage 1!}
Stage 1 is fundamental to your project as it gives you the solid foundation for the whole of your research. As such it is an intense project stage, where you will be expected to undertake a wide range of research activities. You can have an early peek at the activities that we recommend in Table~\ref{tab:stage1activities}, which also provides some guidelines for your interaction with your supervisor during this stage.


\begin{table}[htbp]
	\caption{Stage 1 activities, including their relative effort\label{tab:stage1activities}}
\begin{ltabulary}{\tablewidth}{LLLLLLLLLLLLL}
\hline
\textbf{Research activities} & \textbf{Stage 1~(15\% of project length)} & \textbf{Effort within stage} & \textbf{Suggested focus of your interaction with your supervisor} &  &  &  &  &  &  &  &  &  \endfirsthead
\hline
\textbf{Identifying the research problem} & Develop problem statement and intended contribution to knowledge & 15\% & Suitability of research problem for academic research and to meet the requirements of your specific course &  &  &  &  &  &  &  &  &  \\
\textbf{Reviewing the literature} & Compile initial draft of literature review and plan remaining review & 30\% & Scope of literature review and possible gaps &  &  &  &  &  &  &  &  &  \\
\textbf{Setting research aim and objectives} & Define aim and~ objectives & 5\% & Suitability and feasibility of aim and objectives in relation to research problem and project timeline &  &  &  &  &  &  &  &  &  \\
\textbf{Developing the research design} &
Consider elements of~ research design, including data and evidence, and types of research methods

Complete ethics assessment, including permission to proceed, if needed
 & 10\% &
Consistency of choices in relation to aim and objectives.

Compliance with own university's ethical and legal guidelines
 &  &  &  &  &  &  &  &  &  \\
\textbf{Gathering and analysing evidence} & n/a & 0\% &

 &  &  &  &  &  &  &  &  &  \\
\textbf{Interpreting and evaluating findings} & n/a & 0\% &

 &  &  &  &  &  &  &  &  &  \\
\textbf{Reporting} & Write up Stage 1 report & 20\% & Demonstration of critical thinking and good academic writing, and any improvements required &  &  &  &  &  &  &  &  &  \\
\textbf{Reflecting} & At stage end, think critically about experiential learning in relation to the research process and its activities & 10\% &

 &  &  &  &  &  &  &  &  &  \\
\textbf{Planning work} & Draw your initial project timeline and a detailed plan for Stage 1 & 5\% & Appropriateness of initial work plan &  &  &  &  &  &  &  &  &  \\
\textbf{Managing risk} & Assess project risk & 5\% & Consideration of major risk &  &  &  &  &  &  &  &  &  \\
\end{ltabulary}
\end{table}

You'll notice that Table~\ref{tab:stage1activities} has entries for each of the research activities that were introduced in Section\ref{sect:researchactivities} (Figure~\ref{fig:researchactivities}) and, from the \textbf{Effort within stage} column, you'll notice that all by two activities are covered in Stage 1. We did say that Stage 1 was intense!

Of the activities covered in this stage, \textbf{Reviewing the literature} to compile an initial draft is the most demanding at 30\%, closely followed by \textbf{Reporting} at 20\% and \textbf{Identifying the research problem} at 15\%, the former consisting in putting your research proposal together after you have developed its constituent parts.
Together, reviewing and literature and identifying the research problem are your first steps to understanding what your contribution to knowledge might be and so it is good to invest a lot of time in these. Setting aim and objectives (5\%) and starting to develop your research design (10\%) follow from them.

We have already mentioned the importance of reflection in research, and how it is essential for you to reflect on your increasing understanding of the research process and your own research practice as you go along. Unsurprisingly, therefore, the framework accounts for 10\% of your time spent doing just that. 

Equally, we have also stressed the importance of considering risk in your research and to manage your time effectively, therefore 5\% of your time should be devoted each to risk management and to work planning.

In this chapter, you will start by thinking about how you will organise your work in this stage, then look at the other activities in turn.

\section{Planning your work for Stage 1}
We have stressed how, as a Masters researcher, you will be expected to manage your project to a large extent, including planning your project work in some detail, monitoring your progress and making appropriate adjustments to your plan as you go along.

Key activities for project planning and to keep on track include:
\begin{itemize}
	\item Identifying key milestones for your project, and what you are required to accomplish or deliver at each milestone
	\item Identifying the work you will need complete to reach each milestone
	\item Breaking down that work into discrete tasks
	\item Scheduling those tasks in the time available, and
	\item Making an efficient use of the time you have available.
\end{itemize}

As an important caveat, you should not expect your plan to be cast in stone! There are too many unknowns in a research project for you to be able to predict upfront exactly what is going to happen. Therefore, you shouldn't spend too much time trying to plan every single thing, and you should review your plan often to monitor progress and make adjustments as needed. In summary, your project plan should keep you on track, but should not be a straightjacket.

In this section, you will use the 5-stage framework as a tool to help you put together an initial plan your project work, focusing on the activities required in Stage 1. You will extend this plan, stage by stage, adding details and making necessary adjustments as you go along.

\begin{question}[subtitle={Activity: Breaking down your study time}] Consider the timeline for your project that you defined as part of Activity XX in the previous chapter. Based on the number of weeks you allocated to Stage 1 in your timeline and the number of study hours per week at your disposal, use the percentages in Table~\ref{tab:stage1activities} to calculate the suggested study time for each of its activities.

\begin{solution}Assuming you have allocated 6 weeks to Stage 1, with an average of 20 hours of study per week, then you have a total of 6 x 20 = 120 hours of study for the whole stage. Of these, you will need to spend 18 hours on identifying your research problem (15\%, that is 120 x 0.15), 36 hours on reviewing the literature (30\%), 6 hours on setting your aim and objectives (5\%), etc., including studying the relevant parts of this handbook and carrying out the activities within. While your actual study time may deviate from these calculations, they should still give you a strong indication of how you should allocate your time among the different research activities.
\end{solution}\end{question}
%%Hack to correct tcbox behaviour
\color{black}

\subsubsection{Milestones, deliverables and tasks}

The 5-stage framework identifies for you key milestones and deliverables at a general level: each stage is broken down into research activities, each with clear deliverables, and a written report on the work completed by the end of each stage.

Within each stage there is much detailed work to be done to reach such milestones. Therefore, at the beginning of each stage, you will be encouraged to plan such work carefully, by matching the generic activities in the framework to specific tasks in your project.

\begin{question}[subtitle={Activity: Identifying your project tasks}] Consider the generic activities in Table~\ref{tab:stage1activities} and write down how they correspond to tasks in your own project. Allocate time to those tasks based on the study time you estimated in the previous activity.

\begin{guidance}Feel free to break tasks down into sub-tasks, but be wary of your plan becoming too detailed at this point.
\end{guidance}\end{question}
%%Hack to correct tcbox behaviour
\color{black}

\subsubsection{Producing a project plan for Stage 1}
You can now produce a project plan based on the milestones, deliverable, tasks and their required time you have identified.

This is no different from any other kind of project, so that many planning charts and tools are available which you can choose; in fact, you may be already familiar with some of them. Gantt charts are one of them, which we recommend you use.

Briefly, \textbf{Gantt charts} are scheduling charts that you can use to plan and organise your project work. A Gantt chart is usefully to get an overview of how your work will be broken down and organised over time, including an indication of how much time you will spend on each task, when tasks should start and end, and which tasks might overlap at any point in your project. As such, it is also useful to communicate your project plan to third parties. As a generic project management tool, a Gantt chart can be quite sketchy or very detailed: you should aim for something quite light, but still including all main tasks and deliverables of your project.

\begin{question}[subtitle={Activity: Investigating Gantt charts}] If you are unfamiliar with Gantt charts, conduct a web search on introductory materials and examples. Select and review a small subset of resources you have found. Write down key points about using Gantt charts.

\begin{guidance}In your web search, you'll likely find several links to digital tools supporting Gantt charts, tutorials on how to construct one, or even ready-made templates to fill in. You should focus on materials which can help you construct your own project Gantt chart.
\end{guidance}\end{question}
%%Hack to correct tcbox behaviour
\color{black}

\begin{question}[subtitle={Activity: Constructing your Gantt chart}] Based on the outcome of your previous activities, create a Gantt chart for your own project in relation to Stage 1.

\begin{guidance}If you already use a different project scheduling approach for projects in your own practice, then feel free to use that instead, as long as it can be used to organise your project work effectively and in a way which is easy to communicate with your supervisor, with whom you should share and discuss your plan.

You will augment and adjust your plan in all stages of your project.
\end{guidance}\end{question}
%%Hack to correct tcbox behaviour
\color{black}

\subsubsection{Key practices for managing your time efficiently} 
Your plan is more likely to work if you make an effective use of your time. Here is some key practices you should keep in mind:

\paragraph{Finding time for your research} You need to make a realistic assessment of how much time you will have for your project, considering other commitments you may have, wether professional and personal: this is crucial for part-time study. You course will expect you to spend an average number of hours per week on your project work throughout its duration, so you need to ensure you can dedicate that time to your project on a regular basis. But it is not just about quantity: you should choose your most productive time to dedicate to your research--- in may be that you are a morning person, or you can focus better at night. In either case, it is important you come to your work with a fresh mind, so that you can concentrate on the tasks at hand.

\paragraph{Ensuring continuity of effort} Research requires continuity of effort. While in previous studies you may have been able to start and stop, and possibly `cram' much of the work around assignment deadlines, in your project you need the time to develop your critical thinking and other research skills, and that's a continuous process. If you fall behind, you may find it difficult to catch-up. For your project, you must therefore ensure that your study time is arranged evenly and regularly over its duration and that any study break you may take is short, not to affect your continuous progress.

\paragraph{Making your spare time productive} There will be much to read through your project, so you should always keep something to read with you, which you can look at while waiting for something, say a bus, or in a queue or travelling on public transports. You may be surprised of how much reading you can do while waiting.

\paragraph{Avoiding postponing and procrastinating} There will be tasks you may find harder than others, so it would be natural to put them off or engage in displacement activities. You should avoid that and focus on what must be done to meet your project milestones.

\paragraph{Scheduling extra time} Inevitably things don't always go to plan or activities may take longer than estimated, so you should always factor in extra time for the things that may go wrong. If you find you don't need it, you can always allocate the time to other tasks or take a longer break, but at least you won't fall behind in subsequent tasks.

\paragraph{Being adaptive} Research is about looking into the unknown, so you can't expect to be able to plan everything upfront and in great detail. Your planning should keep you on track to reach your major milestones, but should not be a straightjacket. Don't spent too much time trying to plan every single thing to do: keep your planning light and ensure you return to your plan often to make adjustments as your project unfolds.

\paragraph{Investing in the right tools for the job} While you may find it a nuisance to have to spend time learning new tools, such as a Bibliographic Management Tool (BMT) or some advanced word processing features, doing so will save you time in the long run, so make sure you invest in setting up your new systems and learning new skills earlier on in your project.

\begin{question}[subtitle={Activity: Reflecting on your study practices}] Assess your own study habits and practices in light of the above advice. Write down things you think you do well and things you could or should change for your research project.

\begin{guidance}You should take a balance stance and consider both your current strengths and weaknesses: build on the former and put some effort in addressing the latter.
\end{guidance}\end{question}
%%Hack to correct tcbox behaviour
\color{black}

\section{Identifying the research problem}
A research project gives you an opportunity to carry out a focused piece of academic research in the subject area of your degree and on a topic of your own choice.

It's a complex thing to do, but we'll walk you through it! You've got this!

\subsection{Choosing a topic for your project}
There is no single way to get started and inspiration can come from many places, from previous studies, any professional experience you may have, articles you have read, or even suggestions from your supervisor.

\begin{question}[subtitle={Activity: Possible topics to investigate}] Write down possible topics to investigate for your research project.

\begin{guidance}Think back\footnote{This might be your opportunity to find out more about them through research.} over your studies to this point. Did anything stick in your mind as very interesting, something you'd like to return to? What study topics did you particularly enjoy?

Next, think about what might be interesting within your industry. What are your industrial colleagues struggling with at the moment?

Next, look in the supporting materials that your course provides for your Masters, particularly those associated with possible Masters supervisors. It may be that you will find topic suggestions for Master projects.

Next, think about what has been happening in your discipline recently. Go to Google Scholar and search for recent topic that might interest you.

You should now have a long list of general possibilities for research.

Unless you already have something very specific in mind, you may start with a broad selection, then narrow down your choice to one or two candidate topics for further investigation.
\end{guidance}\end{question}
%%Hack to correct tcbox behaviour
\color{black}

Before investing lots of your time looking into a particular topic, however, you should assess whether it is appropriate for Masters research. Here are some things to consider before you make your final choice.

\subsubsection{Qualification fit}
If your project is the capstone of a taught degree, you should ensure that the topic is appropriate to the degree you are studying. In many instances, this relevance will be obvious -- you may well be exploring a concept that is a mainstream aspect of your subject and that you have met during previous studies.

However, given the cross-disciplinary nature of many postgraduate programmes, it is equally possible that your topic will span boundaries between disciplines. In such a case, it is important for you to demonstrate that the emphasis of the research will be appropriate for the Masters degree you are aiming for.

\begin{question}[subtitle={ACTIVITY: Considering qualification fit}] For each of the study options you identified in your last activity, write down how it relates to your previous degree studies, and particularly to its core modules. Identify those topics that are most suitable for your degree.

\begin{guidance}You should make a note of key ideas, theories, approaches or principles covered in those modules, which are particularly relevant to the topic you have chosen, and identify specific materials you may like to revise or apply in your project.
\end{guidance}\end{question}
%%Hack to correct tcbox behaviour
\color{black}

\subsubsection{Professional fit}
At Masters level, your research should be of interest -- and of potential value -- to at least one of the following groups of people:

\begin{itemize}
\item Professionals\footnote{This may include both public and private enterprises -- local and national government and agencies -- as well as commercial organisations.} in organisations making up a particular industrial or economic sector that is within the scope of your research. This should be the case even if your research will be focused on a single organisation -- some element of the research you produce must be relevant and applicable to organisations doing broadly similar things.

\item Professionals in other sectors than that on which your research is focused but who experience similar problems or issues. A problem area for one industry may have relevance beyond that industry, or there may be implications for those setting policy or creating legislative frameworks.

\item Professional researchers in the field -- their interests are represented by the publications to be found in academic journals. Will your research contribute value to their future research?

\end{itemize}

You will certainly need to argue the case for the professional fit and relevance of your research in your dissertation\footnote{And so it's really worthwhile thinking through professional fit very early on{\ldots}}: it may also be the case that early and\slash or intermediate reports will require you to do this too. This will be especially important if your primary investigation is exclusively within a single organisation, in which case it may not be obvious that the results are applicable elsewhere. It is not enough just to state that the relevance exists -- you must provide some evidence in the form of logical argument or citation of reputable sources identifying a problem common across a range of organisations or an entire industry. Your review of the academic literature plays an important role: as we will see, a key purpose of such review is to identify gaps in existing knowledge that your research is designed to fill. This alone may demonstrate clearly enough the relevance to the wider academic and professional community.

\begin{question}[subtitle={ACTIVITY: Professional fit }] Consider your chosen topic. Write down who may be interested\slash benefit from research in this topic and why.

\begin{guidance}You should also indicate any evidence you have to support your thinking. Ideally, you should talk to some of those people to explain what you intend to do and gain early feedback on the extent this may be of interest\slash benefit to them.

\end{guidance}\end{question}
%%Hack to correct tcbox behaviour
\color{black}

\subsubsection{Personal fit}
Through your Masters project, you will be living with your chosen research topic for many months\footnote{It may even feel longer:)}, so you \emph{must} choose a topic that will retain your interest over that period. A deep-seated interest in a question\footnote{Great research often arises from the researcher's passion for the topic.} can carry you through the tough bits when your motivation might be flagging.

And, although your topic should interest you, even a passionate interest can only take you so far: you should also avoid topics in which you have little existing knowledge, for instance, or that will require you to learn and master major new skills\footnote{The timescale for a Masters research project is usually relatively short and you will be very busy without having to learn completely new subject areas or master completely new skills.}.

\begin{question}[subtitle={ACTIVITY: Personal fit }] Consider your chosen topic and write down the reasons why are you interested in it.

\begin{guidance}You should also make a list any new skills, if any, you may need to develop in order to research such topic, and indicate how you will develop them and in the time available.

\end{guidance}\end{question}
%%Hack to correct tcbox behaviour
\color{black}

\subsubsection{Organisational fit}
Your Masters course may or may not require that you associate your research with a specific organisation. Particularly in part-time studies, work-inspired research is a good way of getting value out of your research beyond your studies, and to gain support or even sponsorship from your employer.

In case of employer-sponsored projects, however, a note of caution is needed to avoid too narrow a focus for research. A common misconception is that the research is an opportunity to complete a piece of work for the sponsoring organisation. That should not be the case: the organisation may well benefit from your research, but the research itself must address a \emph{bona fide} research problem, i.e., one that has wider appeal and is beyond the needs of any single organisation. So, be careful that your research isn't linked too closely to the fortunes or objectives of a specific company or department.

If you are in the happy position of your employer offering sponsorship, make sure that you discuss with them the outcomes they expect from your research. Tell them about the broad focus that you will need to be successful. And make sure they understand that that broad focus will not stop you contributing to their desired outcomes. If needs be, you could always suggest that the skills and attitudes that you will develop during the course of the research and your findings will be available within your organisation, and that
the Intellectual Property\footnote{Intellectual property (IP) is property of things people create with their own intellect, such as an invention, an artwork, a design, etc. We will return to IP in Section XX} that you create through your research may lead to:

\begin{itemize}
\item Solutions to their specific problems
\item New products and\slash or services that might generate a revenue stream that you can share.
\end{itemize}

If after this conversation, they remain unconvinced, then it may be better to choose a research topic that is not of immediate interest to the sponsor. It is a hard choice to give up sponsorship, but at least it will mean that you won't be constrained in conducting your research.

\begin{question}[subtitle={ACTIVITY: Considering organisational fit }] If you're considering asking your employer to sponsor your Masters research project, or if they're already sponsoring your studies, write down how your chosen topic fits with their expectations, and how these compare to the requirements of your course.

\begin{guidance}You should list possible constraints from your sponsor which may prevent you from conducting your academic research, and how you intend to deal with those constraints.
\end{guidance}\end{question}
%%Hack to correct tcbox behaviour
\color{black}

\subsection{What is a research problem}

Within your chosen topic, you will need to identify a specific research problem that your project is going to address. Your research problem narrows down your focus from a whole topic -- which might be quite broad -- to something very specific: a context, a knowledge gap and a justification.

A well-defined research problem is the foundation of any research project, clarifying the research purpose and intended outcomes, something all good academic research requires. As well as using it to drive much of the process of arriving at your dissertation, you'll also include a description of your research problem in your dissertation to tell a reader what you are trying to achieve.

All research problems you'll find in this handbook stem from the following research problem template:

\emph{In the context C, with phenomena P to address knowledge gap G. This matters to W because R.}

An example of this could be:

\emph{In the context of the food industry, to address ``The replacement of plastic containers and wrappers,'' where} plastic, containers\emph{,} wrappers \emph{and} the process to replace them \emph{are all phenomena of interest. This matters: to the food industry because 10\% of all costs can be attributed to the use of plastic; and to society as plastic is highly polluting.}

As written, this research problem may appear a little contrived, but this is only the starting point to identify its important constituents, that is:

\begin{itemize}
\item C: in which Context will the research take place?

\item P: which are the Phenomena of interest?

\item G: what is the knowledge Gap?

\item W: to Whom does this matter{\ldots}

\item R: {\ldots}and for which Reasons?

\end{itemize}

It may be that in addressing the research problem, there may be other things generated: it might be, for instance, that a new process for manufacturing food containers is invented, which means that intellectual property will have been generated, or that other subsidiary questions are answered too, such as better container design. However, these are not part of the research project based on this research problem. Of course, you may wish to follow them up later, after your project has finished\footnote{And it may be that your project can help you to do this{\ldots}}, but typically you wouldn't consider them otherwise.

Ok, back to the template. Let's unpack each component in turn.

\subsubsection{The context and phenomena of interest}
The context in which the research will take place contribute to constrain the project scope, alongside the phenomena that will be considered, what is of interest about them, what they influence, how they are measured, how they are observed, etc. Different contexts and phenomena therein will typically lead the researcher in different directions: for instance, using wifi in a home will lead to different considerations of its behaviour that using wifi in a hospital, where it could interfere with delicate medical devices.

As we shall see, contexts can be embedded in other contexts or can overlap each other, so that, it's important to identify precisely the context of the research project.

\begin{example}{Clara's research}Clara works for a large multinational engineering company in the production planning department of one of its small plants in Denmark. She finds that her planning department struggles to forecast resources and time accurately, so that production costs often escalate, reducing profit margins and lowering competitiveness. She has been asked to do some research to improve the situation.
\end{example}

\begin{question}[subtitle={Activity: Clara's context}] Identify the context for Clara's research problem.

\begin{solution}Clara's problem is embedded in many contexts, including:

\begin{itemize}
\item Denmark, the country,

\item a large multinational company,

\item a small plant belonging to that company, and

\item its production planning department.

\end{itemize}

If the scope was Denmark, then there would be crazily many phenomena that Clara could think about: choosing Denmark is just too large. The large multinational is beginning to be a more realistic scope, but there are other hints that make even more sense. She's been asked to investigate forecasting resources and time in her plant, so that the small plant and its production planning department are a most suitable context for her research, particularly if they share similarities with other plants within the company or in the industry sector.

\end{solution}\end{question}
%%Hack to correct tcbox behaviour
\color{black}

Let's consider phenomena next. Technically, a phenomenon is an element of the world, the occurrences of which are observable. They can be a material thing, transient, like an event or a situation, or information-based, like a fact or a concept; like we said, anything you can observe.

Some phenomena arise naturally and others artificially: many proteins are phenomena that occurs naturally through protein synthesis, whereas the technologies underpinning social media  are constructed by software engineers. If you look around the room in which you're sitting, every object you perceive -- each light, each fly, your computer and its (virtual) files -- every event that occurs -- that light turning on, the creak of your chair, a bird flying into your window --  is a phenomenon.

Phenomena can be really simple --- a speck of dust, or very complex --- the whole sequence\footnote{Although without specialist equipment, you might not be able to observe it} of steps that your computer goes through to connect to your wifi is a phenomenon. Complex phenomena are usually made of various observable elements, each a phenomenon too.

Phenomena are a very rich source of research problems. They have many and various characteristics, not all of which may be fully known: these call for more research. There are also phenomena which haven't even been identified yet! At one point, although we knew how to send data over radio signals, wifi didn't exist and so the devices and protocols associated with it had to be invented. New phenomena are being created and\slash or discovered all the time. That means new knowledge is needed and so research. The existence of phenomena in one field might also suggest their existence in another -- more research needed.

%\footnote{We don't claim that all research problems relate to phenomena, but suspect that all the good ones do.}

In summary, a phenomenological basis for research problems means there's a very rich seam for researchers to mine.

\begin{question}[subtitle={Activity: Clara's phenomena of interest}] Identify phenomena of interest in Clara's research context.

\begin{solution}From the description above, the \emph{forecasting of resources and time} and its \emph{accuracy}, and the plant's \emph{profit margins} and its \emph{competitiveness} are all phenomena of interest. Also, forecasting is a complex phenomenon, possibly including a process, some techniques and some measurements. As part of Clara's research, each of its constituent parts (other phenomena!) may need investigating, alongside its relation to the other phenomena of interest.
\end{solution}\end{question}
%%Hack to correct tcbox behaviour
\color{black}

\subsubsection{The knowledge gap}
You know by now that academic research aims to generate new knowledge in a field of study. It is therefore essential that your research problem captures the knowledge gap your research intends to address. To help you start thinking about the knowledge gap, let's go back to Clara's scenario once more.

\begin{question}[subtitle={Activity: Clara's candidate knowledge gap}] Given the context and phenomena identified in the previous activities, write down a possible knowledge gap that Clara might address with her research.

\begin{guidance}If you're having difficulty finding the gap, focus on what Clara has been asked to do, that is to investigate forecasting resources and time in her plant, which is found to be inaccurate.

\begin{solution}A possible knowledge gap we've come up with is:

\emph{``How to improve the accuracy of forecasting resources and time in the small plant.''}

Yours is probably similar. If it isn't, look again at the example initial description and focus on what Clara has been asked to do.
\end{solution}
\end{guidance}\end{question}
%%Hack to correct tcbox behaviour
\color{black}

It is important to know that this is only a \emph{candidate knowledge gap}: what we don't know at this point is whether Clara's planning department is just bad at forecasting, possibly not employing appropriate processes and techniques, or whether there is effectively a knowledge gap in the sector in that accurate techniques are unknown. To be able to judge, Clara will need to do more work, including looking at the literature, but we are not there quite yet.

\subsubsection{The justification}
Some research is motivated by the researcher' pure curiosity and desire to advance knowledge in a field of study, which is good justification to conduct research!

However, most research, particularly when located in a real-world context, matters to other people too, usually referred to as the stakeholders or beneficiaries\footnote{We will use the term beneficiary when they benefit from the research; a stakeholder means they will be affected by the research, but not necessarily positively.} (these include the researcher, of course!) and may have measurable real-world impact.

Measuring\footnote{Universities these days are assessed on their real world impact, so it is possible that your project will have some form of impact assessment in it.} the real-world benefit delivered through research is an easy way of showing that it has value. Having beneficiaries also means that you can more easily conclude that the research problem has been solved in context -- as you can ask the beneficiaries. Asking why the research is important to them is another good thing to know as you can then use their criteria to assess the value of your research.

\begin{question}[subtitle={Activity: Who will benefit and why?}] In Clara's case, identify who may benefit and how. Also consider who may be able to judge whether Clara's research has addressed the problem.

\begin{solution}Clara's research should lead to more accurate forecasting, which would save her organisation money and make them more competitive.

Her colleagues in the planning department should be able to help her work out why the current process is less effective than it could be, and assess the extent innovations from her research have made a difference.

\end{solution}\end{question}
%%Hack to correct tcbox behaviour
\color{black}

It is very important to note, however, that it isn't the beneficiaries of the research that determine whether the research is actually research; that's a judgement based on the fact that new knowledge has been generated and this can only be judged by the larger research community.

For a Masters research project, that larger research community might be represented by a very small group of people: your supervisor and your dissertation examiners. In exceptional cases,\footnote{Many of our research students have done this.} a Masters student might go on to report their research findings to a larger community of scholars at an academic conference, for instance, or even through a scientific journal: conference or journal publication is the pinnacle of validation that knowledge is new.

\subsubsection{Problem formulation}

It is now time to write down Clara's problem, by using the template we have provided, and based on what we have found out so far, which is summarised in Table~\ref{tab:claraproblemelements}.

\begin{table}
	\caption{Elements of Clara's problem\label{tab:claraproblemelements}}
\begin{ltabulary}{\textwidth}{ll} \hline
\textbf{Context C} & Clara's engineering plant and planning department \endfirsthead \hline
\textbf{Phenomena P} & Forecasting of time and resources and its accuracy \\ \hline
\textbf{Knowledge gap G} & How to improve the accuracy of forecasting \\ \hline
\textbf{Stakeholders S} & Clara's company \\ \hline
\textbf{Reasons R} & Inaccurate forecasting increases production cost and reduces profit margins and competitiveness \\ \hline
\end{ltabulary}
\end{table}

\begin{question}[subtitle={Activity: Clara's initial problem}] Based on the information in Table~\ref{tab:claraproblemelements} and using the problem template, write down Clara's initial research problem.

\begin{solution}This is what we have come up with:

\emph{To improve the accuracy of resources and time forecasting within Clara's engineering plant and planning department. This matters to Clara's company because inaccurate forecasting increases production cost and reduces profit margins and competitiveness.}

You may have written something similar.

\end{solution}\end{question}
%%Hack to correct tcbox behaviour
\color{black}

As stated, this problem formulation is only the starting point, as Clara has yet to establish that a knowledge gap actually exists. To do so, she will need to look into the literature and revise her problem statement accordingly. This is true in general: your initial problem will help you scope your review of the literature, which in turn will inform which knowledge gaps exist, and will help you reformulate your problem as a proper research problem.

In the next section we will look in detail at how you can do that, but first let's look more closely at diverse types of research problem.

\subsection{Types of research problems}
Research problems originated from our template depend critically on phenomena. And it won't come as any surprise then, that you can classify the types of research problem by the things you can do with those phenomena. And that leads to the different types of research that you can do, which we consider in this section.

\subsubsection{Descriptive problems}
Descriptive problems aim to describe phenomena of which we have little knowledge, accurately and systematically. The goal is to describe phenomena for the first time or to render existing descriptions more detailed or accurate.

\begin{example}{Examples of descriptive problems}To characterise the present state of the European commercial synthetic biology ecosystem. This matters to both producers and consumers in that ecosystem because this knowledge can be used to inform effective routes to commercialisation.

To determine how UK organisations are using the ITIL framework to manage cloud-based IT services. This matters to IT managers within those organisations because it can help them improve their service management practices.
\end{example}

\begin{question}[subtitle={Activity - Descriptive problems}] For each example above, identify its constituent parts with reference to the research problem template we have provided. Jot down any similarity between the problems.

\begin{solution}The table provides our mapping of the problem statements onto the template:

%Table in Latex
\begin{ltabulary}{\textwidth} {@{}LLLL@{}}
\textbf{Context C} & European commercial synthetic biology ecosystem & UK organisations                     & \endfirsthead
\textbf{Phenomena P}                                                & The current state of the ecosystem                                             & Those organisations' management of cloud-based IT services         &   \\
\textbf{Knowledge gap G}                                            & A characterisation of the present state the ecosystem                          & How they use the ITIL framework to manage cloud-based IT services  &   \\
\textbf{Stakeholders S}                                             & Consumers and producers within the ecosystem                                   & Those organisations' managers                                      &   \\
\textbf{Reasons R} & To inform effective routes to commercialisation & To improve service management &
\end{ltabulary}

In both cases, the goal is to provide a description where one is currently lacking: of a state-of-the-art in the first problem, and of the use of a framework in the second problem.
\end{solution}\end{question}
%%Hack to correct tcbox behaviour
\color{black}

\subsubsection{Exploratory problems}
Exploratory problems aim to investigate phenomena of which little is known. The goal is typically to generate new ideas, hypotheses, theories, models or predictions which can be investigated in further research.

Lots of academic research is exploratory as it strives to develop a deep understanding of natural or social phenomena of which little is known. For instance, physicists try to explain the working of our natural world by making empirical observations of natural phenomena and conjecturing cause-and-effect relations, usually expressed as mathematical formulae or theories. This is the way natural sciences have developed over the centuries. This is also true of social scientists conducting observations in social settings in order to develop theories on human behaviour.

\begin{example}{Examples of exploratory problems}Within public service providers, to investigate the relation between demographic and behavioural factors and end-users' awareness of cyber security threats. This matters to those providers as successful cyber security attacks can result in loss of confidential information. This also matters to their customers as such information may include customers' data.

To investigate the possible use of social media in the management of a natural disaster by government organisations. Given the spread of social media, this matters to those organisations, which could integrate them within their critical communication infrastructures.
\end{example}

\begin{question}[subtitle={Activity - Exploratory problems}] For each example above, identify its constituent parts with reference to the research problem template we have provided. Jot down any similarity between the problems.

\begin{solution}The table provides our mapping of the problem statements onto the template:

\begin{tabulary}{\tablewidth}{@{}LLLL@{}}
\textbf{Context C} & Public service providers & Government organisations dealing with natural disasters &  \\
\textbf{Phenomena P} & End-users' demographic and behavioural factors;~ cyber security threats & Social media; the management of natural disasters &  \\
\textbf{Knowledge gap G} & To investigate the relation between demographic and behavioural factors and end-users' awareness of cyber security threats & To investigate the possible use of social media in the management of a natural disaster &  \\
\textbf{Stakeholders S} & The providers and their customers & Those government organisations &  \\
\textbf{Reasons R} & Successful cyber security attacks can result in loss of confidential information & Social media could be integrated within critical communication infrastructures &
\end{tabulary}

In both cases, the goal is to explore a situation for which little understanding is currently available (this being the knowledge gap): in the first problem, the focus are possible effects of demographic and behavioural factors on awareness of cyber security threat; in the second, whether social media could be used effectively for communication in the management of a natural disaster.
\end{solution}\end{question}
%%Hack to correct tcbox behaviour
\color{black}

\subsubsection{Explanatory Problems}
Explanatory problems aim to explain why certain phenomena occur or are related. The goal is often to test a conjecture, hypothesis, theory or model.

\begin{example}{Example of explanatory problems}In the context of software development companies, to investigate how learning strategies adopted by software engineers allow them to develop new skills fast and efficiently. These matters to those companies as software technology changes rapidly, so that software engineers need upskilling and retraining very frequently.

Within the public sector, to investigate how practices adopted by organisations and employees lead to an effective and efficient use of teleconferencing technology. This matters to public sector organisations due to the recent significant increase in flexible and home working, so that much of their business is now conducted online.\\
\end{example}

\begin{question}[subtitle={Activity - Explanatory problems}] For each example above, identify its constituent parts with reference to the research problem template we have provided. Jot down any similarity between the problems.

\begin{solution}The table provides our mapping of the problem statements onto the template:

\begin{tabulary}{\tablewidth}{@{}LLLL@{}}
\textbf{Context C} & Software development companies & The public sector &  \\
\textbf{Phenomena P} & Learning strategies; skills development & Use of teleconferencing technology &  \\
\textbf{Knowledge gap G} & How learning strategies adopted by software engineers allow them to develop new skills fast and efficiently & How practices adopted by organisations and employees lead to an effective and efficient use of teleconferencing technology &  \\
\textbf{Stakeholders S} & Software development companies and software engineer & Public sector organisations &  \\
\textbf{Reasons R} & Software technology changes rapidly, so that software engineers need upskilling and retraining very frequently & More and more business is conducted online &
\end{tabulary}

In both cases, the goal is to explore how and why certain phenomena are related. In the first problem, the focus is why certain learning strategies lead to more effective upskilling; in the second problem, why certain practices lead to a more efficient use of teleconferencing technology.
\end{solution}\end{question}
%%Hack to correct tcbox behaviour
\color{black}

\subsubsection{Predictive Problems }
Predictive problem aim to predict future phenomena. The goal is usually to define a model, extrapolating from current knowledge, that allows predictions to be made together with an assessment of how accurate they might be.

\begin{example}{Examples of predictive problems}To quantify the potential impact on fresh water consumption of recycling domestic bathroom water in the UK within the next decade. This is important both from an economical and an environmental perspective, as fresh water is becoming a scarce resource, so its preservation is imperative.

To use demographic and curriculum data to predict which current students are at risk of dropping out in the context of higher education. This matters to higher education institutions as retaining their students is their statutory duty, and because students not completing their degree results in fewer skilled people entering the job market, which may impact productivity.
\end{example}

\begin{question}[subtitle={Activity - Predictive problems}] For each example above, identify its constituent parts with reference to the research problem template we have provided. Jot down any similarity between the problems.

\begin{solution}The table provides our mapping of the problem statements onto the template:

\begin{tabulary}{\tablewidth}{@{}LLL@{}}
 \textbf{Context C} & UK in the next decade & Higher Education \\
 \textbf{Phenomena P} & Fresh water consumption; recycling domestic bathroom water & Demographic and curriculum data; student drop-outs \\
 \textbf{Knowledge gap G} & To quantify the potential impact on fresh water consumption of recycling domestic bathroom water & To predict which current students are at risk of dropping out  \\
 \textbf{Stakeholders S} & Water management agencies; water suppliers and consumers & Universities; their students \\
 \textbf{Reasons R} & To preserve fresh water which is becoming a scarce resource & To improve study retention; to ensured enough skilled people enter the job market \\
\end{tabulary}
In both cases, the goal is to make predictions. In the first problem, on the impact of recycling water on fresh water availability; in the second problem, on the likelihood of students dropping out.
\end{solution}\end{question}
%%Hack to correct tcbox behaviour
\color{black}

\subsubsection{Evaluative Problems }
Evaluative problems aim to establish whether phenomena that have been introduced have achieved the desired outcome. These might be concepts, theories, products, technology, etc., and the goal is to establish how they have performed.

Evaluative problems may focus on testing the strength of an academic theory or model to ensure its long-lasting validity and scope of applicability. They may place an existing theory or model in a new context, or apply them to new situations, evaluating the effects of a change of conditions, what works, what does not and why, possibly leading to new theories or improved models.

Evaluative research may also take the form of a well-founded critique of diverse explanatory theories or potentially conflicting evidence put forward by previous researchers. A systematic and critical analysis of such evidence may lead to new knowledge on which theories are most reliable, so that future researchers may focus on those.

\begin{example}{Examples of evaluative problems}To evaluate the effectiveness of Artificial Intelligence (AI) to reduce the occurrence and impact of successful cyber-security attacks within the financial sector. This matters to the sector because over 3 billion dollars are lost within the sector every year due to successful attacks, and efforts to prevent them are also very costly.

Within non-clinical laboratories, to determine if lean techniques can improve process flow in the presence of unpredictable demand and supply. Inefficiencies in process flow are costly to non-clinical laboratories, so that improving it is beneficial to them and their customers.
\end{example}

\begin{question}[subtitle={Activity - Evaluative problems}] For each example above, identify its constituent parts with reference to the research problem template we have provided. Jot down any similarity between the problems.

\begin{solution}The table provides our mapping of the problem statements onto the template:

\begin{ltabulary}{\textwidth} {@{}LLL@{}}
 \textbf{Context C} & The financial sector & Non clinical laboratories \\
 \textbf{Phenomena P} & AI techniques and cyber security attacks & Lean techniques; process flow; demand and supply \\
 \textbf{Knowledge gap G} & To evaluate the effectiveness of AI in reducing successful cyber-security attacks & To determine if lean techniques can improve process flow in the presence of unpredictable demand and supply \\
 \textbf{Stakeholders S} & Operators in the sector and their customers & Those laboratories and their customers \\
 \textbf{Reasons R} & Significant financial losses are due to such attacks every year, and the cost of preventing them is high & Inefficiencies in process flow are costly \\
\end{ltabulary}

In both cases, the goal is to evaluate how approaches applied within a certain context perform. In the first problem, the focus is on the performance of AI to counter cyber security attacks; in the second problem, the focus is on the effect of lean techniques on the flow of processes with unpredictable demand and supply.
\end{solution}\end{question}
%%Hack to correct tcbox behaviour
\color{black}

\subsubsection{Design problems}
Design problems aim to create artefacts which embed new knowledge. The term `artefact' is meant in its widest meaning, and includes tangible and intangible products, processes, novel combinations of ideas or technologies, etc. The goal is to embed within the artefact new knowledge which extends human and\slash or organisational capabilities, including improved ways of doing something.

\begin{example}{Examples of design problems}
In the context of data visualisation tools, to devise algorithms able to generate automatically text summaries of line graphs depicting multiple time series for the benefit of sight-impaired users. This matters to both tool providers and sight-impaired users as it would improve the accessibility of such tools.

To define a hybrid project management framework, based on complexity and volatility characteristics, to reduce project failure in the development of embedded software systems. Current project failure across the sector is above 70\%, resulting in a high financial burden.
\end{example}

\begin{question}[subtitle={Activity - Design problems}] For each example above, identify its constituent parts with reference to the research problem template we have provided. Jot down any similarity between the problems.

\begin{solution}The table provides our mapping of the problem statements onto the template:

\begin{ltabulary}{\tablewidth}{@{}LLL@{}} \\
 \textbf{Context C} & Data visualisation tool development industry & Embedded software systems development sector \\
\textbf{Phenomena P} & Text summaries; multiple time series line graphs; algorithms & Complexity and volatility characteristics; project failure; project management \\
 \textbf{Knowledge gap G} & To devise algorithms able to generate automatically text summaries of line graphs depicting multiple time series & To define a hybrid project management framework, based on complexity and volatility characteristics, to reduce project failure \\
 \textbf{Stakeholders S} & Tool providers and sight-impaired end-users & Project managers in the sector \\
 \textbf{Reasons R} & To improve the accessibility of such tools & Project failure rate is high, resulting in a high financial burden for the sector \\
\end{ltabulary}

In both cases, the goal is to design something new. In the first problem, new algorithms for the automated generation of text summaries; in the second problem, a framework for hybrid project management.
\end{solution}\end{question}
%%Hack to correct tcbox behaviour
\color{black}

Note that an essential characteristic of a design research problem is that the to-be-designed artefact is a genuine contribution to knowledge in that it augments what is currently possible. Some of the most exciting research in technology-related subjects is about developing new artefacts. However, for such development to fit academic research it is important to establish what the contribution to knowledge is, and assess its originality and significance. In particular, it is important to distinguish the very first development of an innovative artefact from the routine of implementation of known systems: the former augments knowledge; the latter does not.

\subsection{Masters-appropriate research problems}

Regardless of the type of problem you intend to address with your research, you will only have a limited amount of time to complete your project and write your dissertation, most likely no more than one year. It is therefore important that you think carefully about your research problem from the start as changing direction later on in your project could be very challenging and increase your risk of not completing it successfully.

In particular, you should focus on the following criteria:

\paragraph{Generality} Your chosen research problem should not be so limited in scope to be irrelevant or of no interest to others. While your inspiration may well be something you have observed directly in your personal or professional life, your research problem should address a knowledge gap which is shared within your field of study, is of academic relevance and possibly of interest to professional practice, so that your findings can be generalised to some extent. Reviewing the literature is a way to ensure that what you have in mind meets these requirements.

\paragraph{Complexity } You will only have limited resources for your project, so your problem should not be so complex that it cannot be addressed within those constraints. Some research problems are just too ambitious for Masters research, e.g., ``To combat social inequality in the UK.'' However, there may potentially be suitable sub-problems you could consider, e.g., ``To understand the role of food banks in alleviating poverty in the UK in the last decade.''

\paragraph{Volatility} Your chosen problem should remain current for at least the duration of your project and, ideally, present further opportunities for research in future. It is therefore important that you choose a problem which is not likely to become irrelevant too quickly. This can happen in technology-oriented projects that focus on specific products or tools, and their features. It could be avoided by considering if a more general problem can be found of which the original problem is an instance. For example, instead of ``how to visualise data effectively in tool X'', the problem could be generalised to ``which design principles apply for effective data visualisation'', regardless of the specific tool used.


\begin{running}{Assessing Clara's problem} We left our example with Clara's problem expressed as follows:

\emph{To improve the accuracy of resources and time forecasting within Clara's engineering plant and planning department. This matters to Clara's company because inaccurate forecasting increases production cost and reduces profit margins and competitiveness.}

Let's apply the criteria above to work out whether this is a suitable research problem or more work is needed.

In terms of generality, this problem is too specific to Clara's own company, so may not be indicative of a knowledge gap --- for instance, it is possible that Clara's company applies outdated estimation approaches, and better approaches are known and applied elsewhere. Therefore, for Clara, the next step will be to do some initial reading of the academic literature to establish:

\begin{itemize}
\item What, if anything, is already known about this problem in its wider context, e.g., other engineering companies or industries

\item Which specific aspects of this problem, if any, could be investigated to make a novel contribution to knowledge.
\end{itemize}

In terms of complexity, the problem appears tightly focusses and doable as part of a Masters; similarly, it is unlikely the problem will cease to be relevant in the time span of the project, so neither complexity nor volatility are of concern.
\end{running}

\begin{question}[subtitle={Activity: Appropriateness of research problem statements}] Consider the following research problem statements and discuss the extent you think they are appropriate for Masters research based on the above criteria.
\begin{enumerate}
\item To analyse possible differences in people's web search strategies to inform the design of search algorithms in software. Some individuals are better at conducting online searches than others and this knowledge may lead to more efficient algorithms.
\item To establish which information should be displayed to end users to help them identify energy waste within their heating systems. Energy waste impacts negatively household finances and the environment, so raising end-users' awareness could help reduce waste.
\item To improve coding skills of students at a distance. Coding skills are in demand and there is shortage in job market.
\end{enumerate}

\begin{solution}This is our assessment of each of them. Yours may be different, of course.

\begin{enumerate}
\item Web search algorithms are a very active field of study, so it is likely that a contribution to knowledge can be made which is of wide interest to both academia and industry, and likely to remain so for the foreseeable future. Some thought will be needed to ensure that the work can be conducted within the constraints of the project, for instance by containing the number of participants in the study or the length of observations of their search strategies. Therefore, overall, with appropriate adjustments, this could be suitable for Masters research.
\item This problem is both topical and specific, so it has a good chance to meet the criteria. Some work on contextualising the problem in the literature will be necessary to establish the extent of the potential contribution to knowledge, particularly if design guidelines already exist, in which case it would be important to establish how these are lacking. Overall, this too could lead to a suitable Masters project.
\item This problem is too broad and open-ended for a Masters project --- it is a topic, rather that a specific research problem. A more suitable problem could be defined, for instance, by identifying which specific aspects of improving coding skills at a distance are problematic with reference to the literature, so that a much narrower knowledge gap can be established.
\end{enumerate}
\end{solution}\end{question}
%%Hack to correct tcbox behaviour
\color{black}

\subsection{Formulating your initial research problem}
It is time for you to have a go at formulating your initial research problem, based on what you have learnt in this section.

\begin{question}[subtitle={Activity: Your initial research problem }] Apply the template provided to write down your initial research problem. Discuss what type of problem it is and assess it in terms of generality, complexity and volatility.

\begin{solution} Make sure you include all the elements of the template, i.e., C, P, G, S and R, and explain how your problem aligns with its problem type's goal. Argue concisely why your problem is appropriate for Masters research in terms of generality, complexity and volatility, or indicate further steps you will take to ensure that's the case.

\end{solution}\end{question}
%%Hack to correct tcbox behaviour
\color{black}

\section{Reviewing the literature}
In this section, you will start your literature review, an activity you will continue in Stage 2\footnote{Stage 2 is covered in the next chapter.}. In Stage 1 your focus will be on scoping and gathering what to read, and assimilating and analysing relevant content, keeping track of key themes, ideas and findings, and any emerging knowledge gaps. In Stage 2, your focus will be synthesising what you have learnt from the literature to support your arguments and justify your research. By the end of Stage 2 you will have produced a substantial draft of your literature review.

\subsection{The role of the literature in research}
Academic research does not take place in a vacuum. The academic researcher relies on a body of knowledge in their field of study, the cumulative result of collective research efforts over long periods of time. They then use their creativity to add to it, and collaborate with other researchers to develop new ideas and technologies. The main vehicle for the codification and sharing of that knowledge is the academic literature. So, reviewing the literature and adding to it are intrinsic to academic research.

Your literature review will help you frame, contextualise and justify your chosen topic and research problem and to investigate the methods and modes of research that are considered relevant to your discipline. It will also help you demonstrate\footnote{As always, you're trying to demonstrate this to the examiners and so it's their standards that you're trying to meet.} your understanding of the state-of-the-art, and to highlight knowledge gaps --- the unknowns --- to which your research will contribute.

Towards the end of the project, your literature review will also help you establish precisely which knowledge you have created and so evaluate what you have achieved in relation to other published related work.

Our framework recommends you focus on your literature review in Stages 1 and 2\footnote{Stage 2 is the focus on the next chapter.}, although you will be maintaining your literature review throughout your project -- incorporating newly published papers, adding or removing detail from what you've already written to support your dissertation as you progress.

At Masters level, examiners will be looking for a literature review based on between 30 and 60 academic articles, and in the range of 2,000--3000 words.\footnote{Or something like 8 to 10 pages.}

\subsection{How to access the literature}
As a Masters student, your university library will be your first point of call for the academic literature -- probably online, unless you are able to travel to it every day\footnote{Most university libraries have single occupancy study rooms if you need to get away from the bustle of a home or work. Ask the duty librarian what facilities are available at your library.}. It is highly likely that alongside their collection of printed materials, your library will provide you with online access to a wide range of digital resources and a selection of bibliographical databases. The latter are collections of ejournals, ebooks and articles that can all be accessed online and searched at the same time. Among those resources you will will probably find most, if not all, of what you need for your project.

\begin{question}[subtitle={Activity: Investigating your library resources and services}] Investigate your library resources and services to identify those which are going to be particularly useful for your project.

\begin{guidance}You should pay particular attention to those services which allow you to access resources online at any time and from everywhere.
\end{guidance}\end{question}
%%Hack to correct tcbox behaviour
\color{black}

Another source of articles is Google Scholar\footnote{Google Scholar is available at http:\slash \slash scholar.google.com} which has collected together over 100 million\footnote{This is wikipedia's estimate: \href{https://en.wikipedia.org/wiki/Google_Scholar}{https:\slash \slash en.wikipedia.org\slash wiki\slash Google\_Scholar}} academic articles in English, with direct links to each. Google Scholar also has bibliographic citations for download for each article that you can cut and paste directly into your bibliographic database.

Google Scholar provides access to the full-text of an article, particularly for published articles from commercial publishers, although these are usually behind a pay-wall\footnote{A pay wall is an electronic way of protecting an electronic document.}. In addition, Google Scholar collects together `versions' of the same paper so that, even if you don't have access to a paid service version, there may be a pre-print version of the article also available for free. If there is no version available except a pay-for-access one, you can link Google Scholar to any university library that you have electronic access to: this is an incredibly useful service which gives you direct access to those articles found in Google Scholar that are included in your library subscriptions.

Google Scholar can also help you access the so-called \emph{grey literature}\textbf{,} which is the collection of information produced by organisations whose primary or commercial remit is not publishing, such as as academia, government bodies, or non-publishing businesses and industries. It includes pre-publication and non-peer-reviewed articles, theses and dissertations, research and committee reports, government reports, conference papers, accounts of ongoing research, etc.

Google Scholar is an amazing resource for the researcher.

\begin{question}[subtitle={Activity: Connecting Google Scholar to your library services}] Investigate whether your library provides the facility to integrate Google Scholar alongside their proprietary search engines.

\begin{solution}Our university library provides some instructions on how to set up Google Scholar to connect to the databases the library subscribes to, so that direct links to the full body articles appear as part of the results of a Google Scholar search.
\end{solution}\end{question}
%%Hack to correct tcbox behaviour
\color{black}

A third way to get to specific articles which you can't access from your library or via Google Scholar, is to contact the authors directly: academic authors are usually keen to have their work read and should be able to share pre-publication versions of their articles or even point you to other relevant publications which they have written. Contact details of academic authors can usually be found in the header of the articles they have published, or from their university's web pages. Alternatively, you may be able to contact them via professional networks, such as ResearchGate\footnote{ResearchGate is an online network specifically created to connect researchers around the world. It currently has a community of over 20 million researchers which use ResearchGate to ``connect, collaborate, and share their work.''} or LinkedIn\footnote{LinkedIn is the largest professional network worldwide, used to make professional connections and improve one's career. It is not specific to academia, but many academics are part of that community, so it is a useful place to go to make first contact.}.

\subsection{How to read an article}
In this section, we'll be looking at how to read an academic paper. It may seem strange to think about trying\footnote{Searching comes next, promise!} to read an academic article without having found any first! It's such an important skill, however, that it needs introducing early, and in a way that we can coach you through it the first time.

This is an approach we have used many hundreds of times with our students and is based on an excellent paper ``''How to read a paper'' by Srinivasan Keshav (2007)\footnote{The version we are using was revised by the author in 2013.}. Keshav suggests a practical workflow for reading an academic paper. The workflow has up to three `passes', not all of which need to be used, with each having a specific aim:

\begin{itemize}
\item the first pass gives you a general idea of what the paper is saying. The first pass may take only 5 minutes and will allow you to disregard the paper if you find out it isn't relevant\footnote{You should still be adding it to your BMT, however, who knows when it might become relevant in future -- even a long time after your masters studies are complete.}.

\item the second pass gives you a better grasp of the content in outline of a relevant paper. The second pass may take 20 minutes or so, and gives you the opportunity to annotate the paper\footnote{We mean write your notes on it that capture your growing understanding.} as you begin to understand it. You can stop after the second pass if you need to --  perhaps you have a large pile of papers that you wish to sort for relevance.

\item the third pass deepens your understanding of the paper to that point that you can reproduce its main arguments and conclusions -- Keshav calls this ``Virtually reimplementing the paper''! Clearly, this form of understanding is truly deep and can take many hours, days, months, or even years! You can read a paper over and over again in the third pass, and your annotation at the end may grow to be as big as the paper itself. There will be key papers in your Masters project that have this status but they will be few.

\end{itemize}

Kershav's approach is summarised in Table~\ref{tab:Keshav}.

\begin{table}[htbp]
\begin{minipage}{\linewidth}
\setlength{\tymax}{0.5\linewidth}
\centering
\small
\caption{Summary of Keshav's approach --- adapted from the 2013 revision of Keshav (2007)\label{tab:Keshav}}
\label{summaryofkeshavsapproach---adaptedfromthe2013revisionofkeshav2007}
\begin{tabulary}{\textwidth}{@{}LLLLL@{}} \toprule
 \# & Time & What and how to read or do & Objectives & Suitability \\
\midrule

 1 & 5--10 minutes & * Title, abstract, introduction * Section and sub-heading * Glance at mathematical content (if any) * Conclusions * Glance over references & * Category\slash type: measurement, analysis, description? * Context: related other papers? Theoretical basis? * Correctness: assumptions valid? * Contributions: main contribution? * Clarity: well written & Papers not in your research area but may someday prove relevant \\
 2 & 1 hour & * Jot down key points, comments, questions * Figures\slash diagrams: check axes, etc * Mark relevant unread references & * Grasp the content * Main thrust of the paper with supporting evidence & Papers of interest but not lying in your research speciality. \\
 3 & $>$1hour\footnote{We did day it could take a long time:)} & * Virtually reimplement the paper * Identify\slash challenge assumptions * How should I present it? & * Innovation, hidden failings\slash assumptions * Insight into complex arguments and presentation techniques * Jot down ideas for future work * Reconstruct the structure of the paper from memory & Strong\slash weak points, implicit assumptions, missing citations, potential issues. \\
\bottomrule

\end{tabulary}
\end{minipage}
\end{table}

\begin{question}[subtitle={Activity: Applying Keshav's approach to Keshav's paper!}] The paper you need to investigate is Kershaw's paper itself:

Keshav, S. (2007) `How to read a paper', ACM SIGCOMM Computer Communication Review, 37(3), pp. 83--4.

Use Google Scholar to search the article's title and then download the paper\footnote{Remember to record it in your BMT.}.

Start reading Keshav's paper using the guidance in Table~\ref{tab:Keshav}.

\begin{guidance}You should ensure you make appropriate annotations as you go along, then develop and include appropriate notes and summaries in your BMT.
\end{guidance}
\end{question}
%%Hack to correct tcbox behaviour
\color{black}

Figure~\ref{fig:annotations} shows the annotations from our three readings of Keshav's paper for comparison with yours. LR -- what's the purpose of this? Should it be part of the previous activity ?
\begin{figure}[htbp]
\caption{Our annotations on Keshav's paper\label{fig:annotations}}
%\includegraphics[width=\textwidth]{keshav2007readannotatedpdfpdf.pdf}
\end{figure}

Reading Keshav's paper using Keshav's workflow will give you a head start on the following\footnote{It's also satisfyingly recursive!} --- something we've learned in the time Keshav's paper has been in use by our students. 

Firstly, the time suggested for each pass is only an approximation, and will depend on your own skills and previous experience of reading the academic literature, whether the subject matter or research design is new to you, or even that the article may be poorly written and difficult to follow. Secondly, three passes may not strictly be enough: for some articles, particularly the most critical to your research, the third may actually consist of multiple passes.

\emph{Vice-versa}, not all papers require a third pass and you should be careful in deciding whether a particular paper should have one at any point in time: you should never totally disregard a paper -- unless it is clearly out of scope -- as the importance of understanding a paper may only become apparent later: it may be a highly cited paper that needs to be understood so that other papers become accessible, for instance.

Secondly, we've found the following to be very important: at each pass, Keshav stresses the importance of making annotations as you read. Whether you print the article or work to annotate an electronic copy\footnote{A tablet of some form with appropriate note taking software is a great way of doing this.}, effective annotations will help you identify, and access more easily later on, those elements of the paper which are of particular interest to you. In the second pass, your annotations should include comments or queries on elements you find particularly relevant, interesting, or unclear, while in the third pass, they may include ideas for future work which may provide some inspiration for your own research. Your annotations will then help you write up your own notes and summaries. Electronic copies of notes should, of course, also be stored in your BMT.

Finally, Keshav suggests you should highlight references upon which the arguments of the paper rest and that you may like to read later on, and comment on possible relation to other articles you may have already read. This will help you contextualise the article in the wider literature you are reviewing.

Ok, now it's time to give you something to try Keshav's workflow on!

\subsection{How to review the literature}

Reviewing the literature is a process of knowledge discovery\footnote{To know where there are gaps in knowledge, you need to know what is already known -- hence knowledge discovery. A poor literature review may leave you unable to claim a contribution to knowledge.}, and it is both iterative\footnote{Iterative = it never finishes -- although it's better than it sounds: later iterations will typically make fewer changes than earlier ones} and incremental\footnote{Incremental = you'll add to it over time.}. Its core activities are summarised in Table~\ref{tab:litrevactivities} alongside the key skills required. In Stage 1 you will focus on the first three activities in the table, with synthesising being the focus of Stage 2. 

To conduct your literature review successfully, you will need to be able to select work which is relevant and should be included, assimilate and summarise relevant work of other researchers, synthesise and critically appraise ideas from different sources, establish links between studies and their findings, and draw on strengths and limitations of published research. In summary, in your literature review you will need to make use critically and creatively of the content of articles, books and other literature sources you have reviewed to demonstrate your knowledge of the subject, support your own arguments and justify your research. In this chapter and the next, this handbook will help you develop and apply these skills.

% LR -- I don't think we need this figure
%\includegraphics[width=700pt,height=501pt]{lit rev flow.pdf}

\begin{table}[htbp]
\begin{minipage}{\linewidth}
\setlength{\tymax}{0.5\linewidth}
\centering
\small
\caption{Key activities in reviewing the literature, including key skills required\label{tab:litrevactivities}}
\label{tab:litrevact}
\begin{tabulary}{\tablewidth}{@{}LLLL@{}} \toprule
 \textbf{Activity} & \textbf{Aim} & \textbf{Key skills} & \textbf{Outputs} \\
\midrule

% \textbf{Planning} & To identify topics and sub-topics to investigate & Critical analysis of topic and subtopics & Research Problem & Search keywords & Key skills: ?? Outputs: ?? \\
 \textbf{Searching and gathering} & To identify search terms, conduct bibliographical searches and collect articles for consideration & Applying effective search strategies; systematic searching bibliographical databases & Successfully applied search terms and related electronic (or paper) articles for further consideration \\
 \textbf{Processing} & To establish the relevance of searched articles for follow-up in-depth analysis, and record them in a BMT & Skim reading, note taking, bibliographic management & Populated BMT, article annotations, notes, article ranking for follow-up in-depth analysis  \\
 \textbf{Assimilating and analysing} & To engage with the content of relevant articles, and keep track of them, establishing potential links, common themes and emerging knowledge gaps & Critical reading, including identifying key arguments, findings, strengths and weaknesses; critical thinking, including comparing articles and identifying relationships and gaps; further note taking & Detailed article annotations, notes, short summaries, tables, diagrams, etc. leading to understanding  \\
\midrule
 \textbf{Synthesising} & To consolidate and summarise what you have learnt in well-formed academic arguments and a well structured narrative, able to justify your proposed research on the basis of the identified knowledge gaps & Critical thinking and academic writing, including making academic arguments and organising your summary narrative &  Sound academic arguments and well-structured narrative, constituting  a substantial draft of your literature review \\
\bottomrule

\end{tabulary}
\end{minipage}
\end{table}

You will need to read extensively during your research and you will read a lot\footnote{When you write, you will focus only on the most relevant of the papers your search turned up.} more than your final selection of articles cited in your research proposal and dissertation. So, a natural question is ``How do I know when I have read enough and have the references I need?'' Your final selection, simply, will be sufficient to convince the examiner that you have identified a gap in knowledge. The illustration in Figure\ref{fig:pacman} might help.

\begin{figure}[htbp]
\caption{Pacman figure to add\label{fig:pacman}}
%\includegraphics[width=\textwidth]{keshav2007readannotatedpdfpdf.pdf}
\end{figure}

You might like to think of identifying a knowledge gap as trying to locate a particular rock pool\footnote{Might not be a good metaphor:)} starting from a map of your country. For places extremely far away from the rock pool, you don't need any description at all, other than mentioning the country. But, as you get closer, you might want to pick out a particular town to give an approximate location. Then, you need to give more and more precise descriptions as your reader gets closer and closer to the rock pool, finally giving them a really precise description so they can lock onto it.

In detail, you want your literature to identify the area in which your contribution to knowledge is going to be made, precisely and with support for the significant points in your reasoning.

You'll build up to your full literature review in your dissertation by first building a review for your research proposal, one that supports your arguments in relation to your choice of research problem and the potential contribution to knowledge of your research with reference to the state-of-the-art.

Given this, if there are significant points which are not linked to the relevant literature, and not supported by other evidence otherwise, then you don't have enough references; conversely, if there are references which don't relate to any significant point, then they may well be superfluous. You can apply this rule of thumb at each stage of your research to assess whether you have done enough to progress to the next stage.

There is a caveat. A literature review is never really finished because as as you gain insights from the literature, those insights will in turn point you toward other reading; once you have answered a question, new questions arise, and so on. It is a process you will go through till your final dissertation submission, and even then you will have new insights to follow-up or unanswered questions to address. But that's good, as it simply points to the fact that more can come from your work that other researchers may well pick up and so can contribute to your conclusions and further work, which examiners always like to see.

It's now time for your first literature search!

\subsubsection{Searching and gathering}

At this stage of your project, the main objective of your literature review is to demonstrate sufficient understanding of the topic you have chosen to be able to justify why your chosen research problem is worth investigating, i.e., why it will lead to new knowledge. In particular, your literature review should help you consider its wider significance beyond your personal or professional interest. 

It is not surprising, therefore, that this handbook recommends you start from your research problem to plan how you will go about searching and gathering articles for review. In the course of your project, your research problem may well be refined, so that your literature review will need adjusting accordingly\footnote{This is yet another reason for putting as much effort as you can into your research problem, so that downstream activities, such as the literature review, will change as little as possible as you progress.}. 

\paragraph{What to read}

The gold standard of academic publications are peer-reviewed articles, that is articles which have been rigorously scrutinised by academic experts in the field (the `peers') to ensure high scientific quality. Peer-review is a practice unique\footnote{Well, almost unique.} to academia and it reflects the critical need in the academic community to establish that work contributes new knowledge, that it is the work of those claiming it, and that the paper has been written to be accessible\footnote{This doesn't necessarily mean it will be easy to read!} to the community. Academic journal and conference articles are usually peer-reviewed\footnote{The peer review process may be more or less stringent, so you need to take care when selecting articles and you should treat each publication on its own merits. You can check through your university library if an article is peer--reviewed or not.}.

Most references in your literature review should be peer-reviewed, therefore this handbook recommends you start with those. However, there may be some scope for using other non-academic sources -- the grey literature. For instance, articles that are professionally relevant could be helpful, even if not peer-reviewed. Equally, you may reference government and other official reports: although non peer-reviewed, they may have undergone some level of public scrutiny and still contain reliable information. Books and even websites could also be used sparingly, but they are unlikely to have been peer-reviewed or scrutinised, so you should treat their claims to be new or definitive knowledge with appropriate caution.

\paragraph{Identifying search terms}

The number of peer-reviewed academic articles was estimated to have passed 50 million way back in 2010. So even without the grey literature, focussing on the most relevant articles for your research could be a challenge, if not approached systematically.

A good place to start is to brainstorm a set of search terms\footnote{A search term is a word or a combination of words that you can key into a search engine.} that bear a relationship to each of the elements of your research problem. As we've explained, a research problem has the following elements: a context including some phenomena of interest, a knowledge gap in relation to those phenomena, and a justification in terms of who the stakeholders are and the reasons why the problem matters to them. Your quest for knowledge about each of them will drive your literature search.

Let's see how this could be done on an example.

\begin{running}{Search terms from a research problem} Let us consider the following research problem:

\emph{To evaluate the effectiveness of techniques to test embedded safety-critical software while the hardware is still unavailable. Full software testing can only occur once the software is embedded in its hardware. However, in many safety-critical avionics systems, this can happen very late in the development process, leading to expensive re-design if errors are found. Early software testing while the hardware is still unavailable could reduce such late occurrence of errors and expensive re-design.}

We can take each problem element in turn to identify possible search terms. In doing so, we have come up with those in the table below. This initial set should be sufficient to perform our initial search.

\begin{tabulary}{\tablewidth}{@{}LLL@{}}
 Problem element & From problem statement & Our brainstormed search terms \\
 Context & Avionics systems development & Avionic systems; system development \\
 Phenomena & Embedded software; safety-critical applications; testing approaches & Embedded software; safety-critical systems; software testing \\
 Knowledge gap & To evaluate the effectiveness of testing techniques when the hardware is not available & Testing techniques; effectiveness \\
 Stakeholders & Software developers, safety engineers within the context & Software developers; software engineers \\
 Reasons & Early software testing while the hardware is still unavailable could reduce the late occurrence of errors and expensive re-design. & Late errors; re-design
\end{tabulary}
\end{running}

\begin{question}[subtitle={Activity: Search terms brainstorming}] Using your research problem, fill in the following table with few search terms for each research problem element.

\begin{tabulary}{\tablewidth}{@{}LLL@{}} \toprule
 Problem element & From your problem statement & Your brainstormed search terms \\
\midrule

 Context & & \\
 Knowledge gap & & \\
 Phenomena & & \\
 Stakeholders & & \\
 Reasons & & \\
\bottomrule

\end{tabulary}

\begin{guidance}The aim of this activity is to arrive at an initial set of search terms you can use to start your own literature search. You should start by words in your problem descriptions, but you could also include synonyms, particularly if you are not sure which technical terms are used in the literature.

You shouldn't worry too much if you have come up with too many or too few in relation to each element: you will have an opportunity to review and refine your choices later as this, too, is an iterative and incremental process.
\end{guidance}\end{question}
%%Hack to correct tcbox behaviour
\color{black}

Your brainstormed search terms will provide a starting point for searching the literature. Initially, you should be prepared for the fact that very large numbers of articles might be returned with many of them irrelevant to your final goal of understanding your chosen topic. Therefore, you are likely to need to refine your search terms and also come up with ways to narrow down your search results to manageable set of relevant articles.

Search terms are often used in combination to create more complex search terms which may increase the likelihood of finding articles which are close to your topic of interest. Basic Boolean operators, as those described in Table~\ref{tab:booleanoperators}, are used for this purpose.

\begin{table}[htbp]
\caption{Boolean operators for search terms\label{tab:booleanoperators}}
\begin{minipage}{\linewidth}
\setlength{\tymax}{0.5\linewidth}
\centering
\small
\begin{tabulary}{\tablewidth}{@{}lL@{}} \toprule
 Boolean operator & Effect of the search \\
\midrule

 ``Term 1'' AND ``Term 2'' & Materials containing both Term 1 and 2 will be returned. This is used to narrow down a search. \\
 ``Term 1'' OR ``Term 2'' & Materials containing only Term 1 or only Term 2 or both will be returned. This is used to expand a search. \\
 NOT ``Term 1'' & Materials not containing Term 1 will be returned. This is used to narrow down a search. \\
\bottomrule

\end{tabulary}
\end{minipage}
\end{table}

\begin{question}[subtitle={Activity: Your first literature search}] Conduct a search in Google Scholar from the search terms you brainstormed out of your research problem elements, combining them using the operators above if appropriate.

For each, record both the combination you have used and how many articles Google Scholar returns as hits. Reflect on how using different terms and operators may change the number of hits and how that may inform ways to widen or narrow down your searches.

\begin{guidance}In combining your search terms, you could start by using the AND operator, then repeat the search trying different combinations of search terms, different operators, adding and removing terms, or even using synonyms of your search terms.

You should ensure you record both the combinations you've used and the resulting number of hits.
\end{guidance}

\begin{solution}Using the search terms from our example, we have conducted the following searches.

Firstly, we typed in:

``safety critical systems'' AND ``software testing''

to Google Scholar and this returned 5310 results: that's a lot of papers to read, so we knew that we needed to narrow the search down.

We therefore added another term and typed in:

``safety critical systems'' AND ``software testing'' AND ``avionics systems''

which returned 389 results: more manageable, but still a very large number.

To narrow the search further, we typed in:

``safety critical systems'' AND ``software testing'' AND ``avionics systems'' AND ``testing techniques''

and found 90 results.

We then tried:

``embedded software'' AND ``testing techniques'' OR ``testing methods'' AND ``avionics systems''

Which returned 35 results.

In this example, adding ``avionics systems'' helped us narrow down the search, presumably because only avionics applications were returned, rather than any kind of safety critical system; similarly, adding ``testing techniques'' presumably helped the search engine focus on techniques rather than, say testing processes or other. However, when we added ``testing methods'' using the OR operator, the hit list got even smaller, which was surprising. This demonstrate that search engines do not always behave as we would expect, so that it is important to try different approaches to arrive at a desirable outcome.
\end{solution}\end{question}
%%Hack to correct tcbox behaviour
\color{black}

As you may have gathered from this activity, to a large extent your initial search will be a trial-and-error process\footnote{Be sure to be systematic and record which search terms you have used: this will save a lot of pain later when you're trying to remember which search term you used to pick a particular article!} and you may end up iterating several times. It's not wasted effort, though. You can make the most of the time spent by:

\begin{itemize}
\item Using Boolean combinations of search terms wisely to narrow or widen your searches

\item Particularly if only few hits are returned, replacing your search terms with new terms -- often synonyms for existing ones.
\end{itemize}

Finally, although in this section we have focussed on Google Scholar, the same techniques apply to most bibliographical database search engines, so you could repeat the activities on subscription databases you can access through your Library.

\paragraph{What if there are still too many hits?}
Even if you spend your search time wisely, you may still end up with a large number of hits to consider. An important thing to recognise, is that not all of the hits returned will be relevant, so you will need to make an initial assessment of what to exclude from further consideration. When the number of search hits is high, this will take time, so you will need to select which articles to look at first.

We recommend you start by considering recent review articles. Review articles, also known as surveys, are academic articles which summarise the current understanding of a specific topic or phenomenon. They are usually the result of analysing and synthesising academic literature or bringing together findings from several studies. Any recent review articles that appear in your search offer a very easy way to start your literature review: not only do they collect together recent relevant papers, but they may have an overview or a precis of each.

If no review articles are included in your hits, you could repeat your search by adding AND ``Review'' or AND ``Survey'' to the end of your search term. Alternatively, you could ask your supervisor if they can suggest a review article you can start with. It's a great question to ask at an early meeting with them. Your supervisor may also suggest other seminal papers for you to look at.

By `recent' we mean within 2 to 5 years, particularly for fast changing disciplines, such as Computing. For slow changing disciplines, the time frame of publication may be less important. Reducing the timeframe of publication is an effective way to cut down the returns considerably. Google Scholar allows you to select articles within a specified timeframe, and similar facilities are available in most bibliographical database search engines.

Second thing to recognise is that not all hits may be accurate -- they may have the wrong date on them for instance -- so care is warranted. Repeating the search on different bibliographical databases may help identify erroneous entries.

\begin{question}[subtitle={Activity: Narrowing down your hit list}] Consider the outcome of one of your searches. Identify any recent review paper which may be included. If necessary, revise your search term by including AND ``Review'' or AND ``Survey'' and try again, or set a specific timeframe to reduce the number of outputs.

\begin{solution} We conducted a search for Clara's problem using the search term:

``engineering plant'' AND ``planning'' AND ``forecasting'' AND ``accuracy'' AND ``production cost''

which returned 78 hits.

By considering the last 5 years as timeframe for publication, these were reduced to 17 hits. Among them, we found 3 articles which included some form of review.
\end{solution}\end{question}
%%Hack to correct tcbox behaviour
\color{black}

\subsubsection{Processing}
Once your search has returned a manageable set of articles for further consideration, say up to 20, you should begin processing them to decide whether they could be relevant and, hence, considered further. 

Although you will need to read all of the papers that you find potentially relevant, fortunately you won't need to read all of a paper to find out if that's the case! Sometimes it may sufficient to read what accompanies the paper on a search service\footnote{Google Scholar, for instance, includes an extract from the paper that caused it to match your search term.}. Otherwise, clicking on the link to the paper will reveal its abstract, which you can quickly skim to check whether the paper might be relevant. If so, add it to your BMT as something for further consideration.

Don't be too picky at this point: it's better to include something in your database that you don't use later, than to discard something that you find later might have been relevant.

\paragraph{How will I know something is potentially relevant?}
Understanding the relevance of articles is an important skill that develops with the practice of reading the academic literature. Relevant articles will help you in a number of ways, including: to develop a deeper understanding of the context of your research, its stakeholders and beneficiaries; to raise your awareness of the significance of the contribution to knowledge you intend to make and any inherent difficulties in doing so; to show you different ways of thinking about your problem; to illustrate possible approaches or techniques you could apply to your problem.

During processing you need an efficient way to decide which of the papers in your search hits are potentially relevant, before investing considerable time and effort into assimilating and analysing their content. 
%
%A way to decide whether an article is relevant is to reverse engineer the research problem it addresses and compare it with your own research problem. What do the two problems have in common?
  
  You'll know something is potentially relevant if the phenomena that are mentioned in a paper bear some relation to those of your research problem. Therefore, focusing on phenomena is what we recommend you do at this point. 

It is important to acknowledge that there is no standard way of referring to phenomena, so having an open mind -- at least initially -- as to what could constitute them would be good. Remember, phenomena were defined very generally as observables. In a paper, phenomena, as observables, can be referred to in many ways, including:

{\color{red} LR -- this, till the end of the sub-section, needs work; for Jon to do}
\begin{itemize}
\item Directly: statements such as ``the prices of software services were \emph{observable} in this study'' -- the observables are the prices of software services.

\item Indirectly: statement such as ``the prices of software services were \emph{not directly observable} in this study, so we used the proxy of hours worked accumulated across the team of developers'' -- the observables are the number of hours worked.

\item As measurable: statements such as ``the number of bugs in the software were given through the collection of bug reports'' -- the observables are the bugs in the software.

\item As inferences from other phenomena: statements such as ``the number of bugs in the software were estimated from the number of bug reports'' -- the observables are the number of bugs reported.
\end{itemize}

\begin{example}{Going back to our example of safety-critical development, one of the review articles we found in our search...}Going back to our example, one of the review articles we found in our search is:

Vahid Garousi, Michael Felderer,   a  r   Murat Karap    ak, U  ur Y  lmaz (2018). Testing embedded software: A survey of the literature, Information and Software Technology, Volume 104, Pages 14--45, ISSN 0950--5849, https:\slash \slash doi.org\slash 10.1016\slash j.infsof.2018.06.016.

Its abstract\footnote{Increasingly, and very helpfully, many journals and conferences research articles are beginning to use \emph{structured abstracts}, such as the one in this article. A structured abstract has sections which bring out the context, objectives, results and other information.} reads:

``Context

Embedded systems have overwhelming penetration around the world. Innovations are increasingly triggered by software embedded in automotive, transportation, medical-equipment, communication, energy, and many other types of systems. To test embedded software in an effective and efficient manner, a large number of test techniques, approaches, tools and frameworks have been proposed by both practitioners and researchers in the last several decades.

Objective

However, reviewing and getting an overview of the entire state-of-the-art and the --practice in this area is challenging for a practitioner or a (new) researcher. Also unfortunately, as a result, we often see that many companies reinvent the wheel (by designing a test approach new to them, but existing in the domain) due to not having an adequate overview of what already exists in this area.

Method

To address the above need, we conducted and report in this paper a systematic literature review (SLR) in the form of a systematic literature mapping (SLM) in this area. After compiling an initial pool of 588 papers, a systematic voting about inclusion\slash exclusion of the papers was conducted among the authors, and our final pool included 312 technical papers.

Results

Among the various aspects that we aim at covering, our review covers the types of testing topics studied, types of testing activity, types of test artifacts generated (e.g., test inputs or test code), and the types of industries in which studies have focused on, e.g., automotive and home appliances. Furthermore, we assess the benefits of this review by asking several active test engineers in the Turkish embedded software industry to review its findings and provide feedbacks as to how this review has benefitted them.

Conclusion

The results of this review paper have already benefitted several of our industry partners in choosing the right test techniques \slash  approaches for their embedded software testing challenges. We believe that it will also be useful for the large world-wide community of software engineers and testers in the embedded software industry, by serving as an ``index'' to the vast body of knowledge in this important area. Our results will also benefit researchers in observing the latest trends in this area and for identifying the topics which need further investigations.''

\end{example}

\begin{question}[subtitle={Activity: the relevance of early hits to your search terms}] For five of the hits to the search you conducted in Activity ??, fill in the relevancy matrix below, and score them for relevance.

\begin{ltabulary}{\textwidth} {@{}LLL@{}}
\textbf{Type}&\textbf{Comment}&\textbf{Score}\endhead
 Direct & TBD& TBD\\
 Indirect & & \\
 Measurable & & \\
 Inferences & & \\
Complete this...& &
\end{ltabulary}

\begin{solution}TBD
\end{solution}\end{question}
%%Hack to correct tcbox behaviour
\color{black}

\paragraph{What to do when you find a potentially relevant paper}
First thing you should do is to record it in your BMT. You should record it with the search term that you used to find it so that you can rerun the search to:

\begin{itemize}
\item Find similar papers that might later have added relevance to your work

\item Find new papers that have been added to the literature since you last looked

\end{itemize}

In the latter case, you should also record the date\footnote{Many bibliographic management tools will do this for you. Check whether yours does, perhaps even by reading the manual{\ldots}} that you found it so you can check.

\paragraph{What if none of the hits are relevant?}
This is a point where you need to iterate back to searching bibliographical databases for new hits, using different search terms to widen the scope of your search. Also useful on these occasions is to ask your supervisors for advice: they should be able to point to seminal or review papers you should start from, or indicate  which authors have made core contributions to your topic.

%LR -- stopped here, 10/10, 13.25
\subsubsection{Assimilating and analysing}
Of course not, you need to read the paper too and begin to understand it --  for the first few times, you might like to refer back to the Keshav workflow in Section ??.

\paragraph{Working towards an in-depth understanding of an academic article}
Initially, getting the most from\footnote{Your supervisor may also be able to help with this task.} an academic paper means understanding what it contributes to your literature review. Getting to that understanding quickly is important as your collection of articles grows as then you can determine which are the most important papers to include.

Honestly\footnote{And not just for your masters project. Accessing the academic literature is a professional skill that not many professionals have.}, it's going to take some time to pick up the skills of doing so, but it's an investment worth making. 

As we discussed in Section XXX(Hershaw section), to understand an academic paper is to understand the contribution to knowledge that it makes. While you won't have to understand in depth everything you read, there will be some articles which are so fundamental to your research that you will need to study them carefully. This will require a substantial investment of your time and can be very challenging: even the most experienced researcher is unlikely to be able to read an academic paper once and immediately understand it in its entirety.

In this section we give you a number of techniques to help you engage with the content of the articles you have selected. Your goal here should be to identify each paper's key features and their relationships so to be able to address the following questions and their contributions to your understanding:

\begin{ltabulary}{\textwidth} {@{}lL@{}} \toprule
 Questions & Contribute to \\
\midrule

 * When does the problem occur? * Where does it occur? * How is it caused? & Articulating the problem in its context \\
 * Whom does it affect? * How serious is it? * What are the benefits of addressing it? & Establishing its significance \\
 * What has been done so far to address it? * What else could be done to address it? & Establishing the potential contribution to knowledge \\
\bottomrule
\end{ltabulary}

\paragraph{Tracking key features of academic articles}
As the number of papers you have read grows, it will be increasingly difficult to keep track of what each article is about without continually having to refer back to your annotations. Even more difficult will be to compare quickly the key features of different articles.

To help you keep track of each article, you can use a summary-comparison matrix\footnote{Note that an SCMatrix must adapt to your needs as they change. Table 1 is a good starting point, but you'll probably add further columns as you progress.} (Sastry and Mohammed, 2013 (add \textbackslash{}cite)), SCMatrix, a basic form for which is shown in Table 2. In an SCMatrix you record the key features of each article, such as the research problem addressed, the key contribution made, the research methods applied, etc.

\begin{ltabulary}{\tablewidth} {@{}LLLLLLL@{}} \toprule
 \textbf{Reference (link to your bibliography)} & \textbf{Research problem\slash question\slash hypothesis} & \textbf{Research methods\slash approach} & \textbf{Key findings} & \textbf{Key conclusions} & \textbf{Implications for future research} & \textbf{Implications for practice (if relevant)} \\
\midrule

 & TBD & & & & & \\
 & & & & & & \\
 & & & & & & \\
 & & & & & & \\
 & & & & & & \\
 & & & & & & \\
 & & & & & & \\
\bottomrule
\end{ltabulary}

Example summary-comparison matrix

\begin{question}[subtitle={Activity: Constructing your summary-comparison matrix for Keshav's paper}] Complete the above summary-comparison matrix for Keshav's paper.

\begin{guidance}
$<$TO BE REDONE$>$This activity will require you to go back to your notes on each article and possibly skim through the article again to extract relevant points.

For each paper you read you should return to your SCMatrix and update it, looking for links between the papers you have read. With a few papers, this is simple to do, but don't underestimate the time it will take -- or the value that it will have for your literature review -- as the number of papers grows.

$<$Perhaps examples from students, if we have access?$>$
\end{guidance}
\end{question}
%%Hack to correct tcbox behaviour
\color{black}

\paragraph{Identifying relevant themes}

As the number of articles you read grows, you will find recurrent themes emerging within your chosen discipline. These might be of many types:

\begin{itemize}
\item Subject relevant

\item Questions asked

\item Hypotheses\footnote{We'll say more about hypotheses later, but for now think just think of them as candidate answers to questions.} created

\item Research Methods

\item Ways of thinking through problems

\item Important terms and other nomenclature

\item Add more here

\end{itemize}

That these elements\footnote{Of course, forming a judgement on this is something you will have to do, perhaps with the help of your supervisor.} recur in papers in your chosen area should suggest to you that they are important themes.

You will also notice that, over time, certain of these develop as issues within your discipline as more is understood about it --  some will grow in importance while some reduce, perhaps even disappearing.

\begin{tip}Any new themes that are introduced and recur in very recent papers would be good to make a special note of --  these might reveal current hot research areas.
\end{tip}	

Given that you wish to make a contribution to the knowledge in your discipline, it is important that you're able to track the discipline's development as they will help you in your critical summaries, where you will need to bring together and compare and contrast ideas from different authors and articles, and in your own research, where they will indicate paths that are still to be trodden.

Here, another simple tool becomes important, the theme identification matrix (citation?), or TIMatrix, an example of which is shown in Figure 1 around the topic of safety critical systems --- it is not necessary for you to be familiar with that topic!

The rows represent articles you have read, so that their citations are included in the first column; the remaining columns correspond to the emerging themes --- concepts, questions or ideas --- you have encountered while reading them; the ticks give you an indication of how often and where each of them recurs in the literature you have reviewed.

\begin{table}[htbp]
\begin{minipage}{\linewidth}
\setlength{\tymax}{0.5\linewidth}
\centering
\small
\caption{Actual(?) theme identification matrix}
\label{actualthemeidentificationmatrix}
\begin{tabulary}{\tablewidth}{@{}LLLLLL@{}} \toprule
 Citations vs topics & Definition of safety critical system & Safety analysis & Safety analysis within development & Safety anomalies in development & COTS in safety critical systems \\
\midrule

 Paige, and McDermid, 2010 & x & & & & \\
 IEC, 2022 & & x & x & & \\
 De Lemos, Saeed, and Anderson, 1998 & & & x & & \\
 Martin and Muniak, 2002 & & & x & & \\
 Ellis, 1995 & & & & Xreference 6$<$????meaning? & \\
 Profeta, Adrianos, and Yu, 1996 & & & & & x \\
 Dawkins and Kelly, 1997 & & & & & x \\
\bottomrule

\end{tabulary}
\end{minipage}
\end{table}

$<$Is this from Rich's dissertation? All the better!$>$

\begin{question}[subtitle={Activity: Identifying recurring themes}] Consider the articles your have read so far, and your notes and summaries, to identify key recurrent themes. $<$I don't think I could do this with the reading I would have had to have done to get to this point -- 2023--06--12: might be better now$>$

\begin{guidance}You should construct a TIMatrix similar to the in Figure 1. You should also reflect on the most common themes and make notes on how they are approached by the different articles.
\end{guidance}\end{question}
%%Hack to correct tcbox behaviour
\color{black}

\paragraph{Identifying key ideas, gaps and relationships}
Having identified a number of relevant themes, you should begin to reflect on their relevance to your research and to assess the extent to which they should be addressed in your literature review. In particular, you need to start going deeper into those themes to explore underlying ideas, trends, links to other relevant themes and especially knowledge gaps.

To do so, we suggest you make use of a simple theme analysis table, like the one illustrated in Table 1: you should adapt it in ways which suit your project.

\begin{table}[htbp]
\begin{minipage}{\linewidth}
\setlength{\tymax}{0.5\linewidth}
\centering
\small
\begin{tabulary}{\tablewidth}{@{}LLLLLL@{}} \toprule
 References (link to MTB) & Theme considered & Key ideas\slash contributions\slash arguments & Relevance to my research & Knowledge gaps & Links to other themes \\
\midrule

 Reference 1 & & & & & \\
 Reference 2 & & & & & \\
 Reference 3 & & & & & \\
 Reference 4 & & & & & \\
 Reference 5 & & & & & \\
 {\ldots} & & & & & \\
\bottomrule

\end{tabulary}
\end{minipage}
\end{table}

Example theme analysis matrix or TAMatrix

\begin{question}[subtitle={Activity: Constructing your theme analysis matrix}] Starting from your current TIMatrix, select a theme which you think is relevant to your research and that you wish to investigate further.

For all the articles in your TIMatrix addressing that theme, fill in an entry in the table: this will require you to go back to your notes on each article and possibly skim through the article again to extract relevant points.

Do this activity for all the themes you wish to investigate further.

\begin{guidance}In the first column, you should include a full citation\footnote{Your BMT might allow you to directly link to an article for easy retrieval of your notes, or the whole article.}.

In the second column you should record the theme you are investigating.

The third column gives you an opportunity to summarise the main ideas and contributions of each article in relation to the theme you are investigating.

In the fourth column you should include your assessment of how each article relates to your own research and may inform your work, which also justifies why it should be included in your literature review.

The fifth column should help you explore the unknowns, and the ways in which your research may address existing knowledge gaps.

The sixth final column is for you to establish how ideas may relate to other themes, also helping you select which themes to consider next.

This is a substantial activity which, depending on how much material you have, may take you several hours or a full day of work. It's also an activity that you'll have to loop back on, given that themes in one paper will overlap\footnote{Perhaps not completely{\ldots}} with those in others.
\end{guidance}\end{question}
%%Hack to correct tcbox behaviour
\color{black}

\begin{guidance}There are other tools that will help you, too, such as mind maps. Mind maps are graphical devices for linking ideas together and are very popular with some people. However, there is a learning curve to be climbed and, if you aren't already familiar with Mind Maps, you may like to stick with the simpler matrix version.
\end{guidance}


Once you have completed your table, you will have a number of references associated with each of the themes you have selected, alongside a set of key ideas, contributions and gaps. In doing so, you should have a grasp of existing research around your chosen topic, and this should help shape your own thinking on how your research may contribute new knowledge.

\begin{question}[subtitle={Activity: Considering sources of information for your project}] Other than academic articles , list other sources of information which may be useful to inform your project. $<$ I don't think I'd be able to do this$>$

\begin{guidance}You should also make notes of how you will use them to inform your research.
\end{guidance}\end{question}
%%Hack to correct tcbox behaviour
\color{black}

\paragraph{Identifying further relevant articles}
*** LR - to talk about forward and backward reference searching

\section{Setting research aim and objectives}

Exciting times! Now that you have identified your research problem, and have supported it with an initial review of the literature, you can start defining your research aim and objectives.

\subsection{Articulating your research aim}
Your research aim is a concise statement of the intended outcome of your research. A good way of thinking about it is that it will describe the new knowledge generated by the project. This also tells you how important your research aim is: your overall research goal is to make a contribution to knowledge. That you can describe to some level of detail that contribution means that you will have made some great progress!

That you're at this point means it's the perfect time to think about your research aim.

\begin{example}{Clara's research }In our example, if your recall, Clara had identified the following problem after her initial review of the literature:

\emph{how to apply ML to improve the accuracy of resources and time forecasting in the context of small engineering plants. This matters to engineering companies because conventional methods are often inaccurate, and inaccurate forecasting increases production cost and reduces profit margins and competitiveness.}

Clara's initial research aim formulation is:

\textbf{Aim}: to identify which ML techniques improve the accuracy of resource and time forecasting in the context of small engineering plant.

In which she identifies the new knowledge: which specific ML techniques would be effective in addressing the need of improving forecasting accuracy in her chosen context.

Why is this his new knowledge? It may be -- indeed it is very likely --  that lots of research has been completed and published in the area of suing ML to forecast resources and time for large organisations. It may even be the case that this extends to published knowledge about large or medium sized engineering plant. However, through her initial literature review Clara will have reached the conclusion that this knowledge specifically doesn't apply to small engineering plant.

This way, Clara has identified the knowledge she aims to contribute through her research.
\end{example}

Note that the difference between research problem and research aim can be very subtle. To distinguish them, you should think of a research problem as focusing on the knowledge gap to address, while the aim of the intended way that gap could be addressed.

\begin{question}[subtitle={Activity: Drafting your research aim }] Consider your current problem statement. Draft a possible related research aim.

\begin{guidance}Your research aim formulation doesn't need to be perfect at this stage, and will evolve with your research problem as you progress through your project. However, it is important for you to think about what intended knowledge contribution your research will make, and expressing your research aim is a way to do so.
\end{guidance}\end{question}
%%Hack to correct tcbox behaviour
\color{black}

\subsection{Choosing a title}
With your chosen research problem and aim you should now have a good idea of what you are hoping your research will focus on -- the need you will address in context -- and deliver -- the intended knowledge outcome. From this you can choose a representative title for your project. \footnote{The title is the first contribution you will have made to your dissertation! That first page is no longer blank. Congrats, you're on your way!}

The title provides the first indication to your reader of what you propose to research. It may change as the research progresses, so it is important to review it from time to time to check its current relevance. At this stage, the title can only be your best attempt at anticipating later developments in your research, so don't agonise too much over it.

Here are some guidelines for you to follow:

\begin{itemize}
\item A good title should succinctly convey elements of both research problem and aim, specifically, the focus and intended outcome of your project

\item A good title is around 8--20 words. Here are some examples\footnote{An experienced supervisor may have other examples too.}:

\begin{itemize}
\item Integrated process improvement strategies in small\slash medium-sized manufacturing enterprises in the UK fabricated metals industry

\item Cost-effective greenhouse gas mitigation measures for the UK livestock industry -- a risk assessment of the impact on water footprints

\item Critical Success Factors for enabling Packaged Software to realise the potential Business Benefits

\end{itemize}

\item Avoid titles that are very short and enigmatic, or titles that are long and rambling

\item Do not include acronyms or obscure technical terms, except those which are likely to be widely understood (e.g., UK, USA, IT, or WWW)

\item Do not pose a question in your title

\item A title is not a sentence, so it does not require a full stop at the end.

\end{itemize}

\begin{question}[subtitle={Activity: Choosing a title }] Use the guidelines above together with your research question and aim, to write down an appropriate title for your research proposal.

\begin{guidance}You will have many opportunities to refine your title, so at this point, spend no more than 20 minutes on this.

You should indicate how your chosen title convey elements of both your research problem and aim.

$<$Q: Is this something to do in the title, or in other notes?$>$
\end{guidance}\end{question}
%%Hack to correct tcbox behaviour
\color{black}

\subsection{Articulating your research objectives}
Research objectives help you break down your aim into the steps you should take to achieve it.

While your aim will be expressed in broadish terms, your research objectives must be specific and feasible: specific enough to indicate exactly what you will achieve, so to be able to tell whether you have achieved them by the end of your project; feasible in the sense that you must have the knowledge, skills, time and resources to achieve them.

$<$This might contradict the above statement --  changes to help?$>$Research objectives can still be quite broad, though, so that you will only need, 3 to 5 or them. As they contribute to reaching your aim, usually later research objectives will build on earlier ones.

\begin{example}{Example}Clara's research aim is $<$Check that this is still the case$>$:

\emph{to apply Machine Learning (ML) to improve the accuracy of resources and time forecasting in the context of small engineering plants}

Clara has decided that the `identify-assess-recommend' research objective pattern works for her. Using this pattern she comes up with the following incremental research objectives:

Objective 1. To \emph{identify} ML techniques that are applicable to resource and time forecasting in the context of small engineering plants

Objective 2. To \emph{assess} the accuracy of forecasting of those techniques in the context of small engineering plants

Objective 3. To \emph{recommend} how to integrate the most appropriate techniques effectively in engineering practice in order to improve forecasting accuracy

Objective 1 directs Clare to identify specific ML techniques to be used in the project, to ensure the work is feasible within the time-frame of the project.

Objective 2 directs Clara to assess how accurate the chosen techniques are in the context of her small engineering plants.

Objective 3 directs Clara to draw some conclusions from the research conducted and make recommendations to improve professional practice.

Objective 3 could have been split into two research objectives -- the first being to draw conclusions, the second being to make recommendations. Clara didn't do this as she understands the process of making recommendations will require her to arrive at some conclusions about applicability of what she has found.

Note how those objectives build on each other and, if successfully completed, they contribute to meet Clara's overall research aim.
\end{example}

The `identify-assess-recommend' pattern the Clara used is a very useful pattern to know if the knowledge you create as part of your research will have real world application, which will be the lion's share of masters research projects.

Even if you apply this pattern, identifying the best form for your research objectives will require some creativity on your part, as there is no magic formula to do so. The example breaks down the aim into research objectives following an i\emph{dentify-assess-recommend} pattern, which is quite common, but by no means universal. If your research is more formal, mathematical, clinical, or of some other form, there are other research objective patterns that will be applicable. Examples are:

$<$Insert patterns, perhaps from students' work$>$

You can try to apply this pattern, but are not required to do so. In all cases, your should discuss how to break down your aim into objectives with your supervisor.

At this point in your project your objectives are still speculative, expressing the intention of your research: they are likely to change during your project, so that you you will need to review them at each study stage.

\begin{question}[subtitle={Activity: Articulating your research objectives}] Consider your current research aim. Write down 3 to 5 possible objectives, explaining how they relate to each other and how they contribute, if successfully completed, to meet your research aim.

Also comment on how specific and feasible they are, the latter in relation to your own knowledge, skills and resources, and your project time span.

\begin{guidance}Your aim and objectives don't need to be perfect at this stage and will evolve with your research problem as you progress through your project. However, it is important for you to think about what concrete contribution your research will make: expressing aim and objectives is a way to do this.

\end{guidance}\end{question}
%%Hack to correct tcbox behaviour
\color{black}

\section{Developing the research design}

\subsubsection{Research design}
Although not all aspects of research are manageable, having a design for your research will provide many benefits, including the ability to summarise, explain, and justify it. The design will, like most other aspects of your project, will evolve: at the start, it will be a collection of your initial ideas; by the end, it will be a detailed account of what you have actually done. It will be a touchstone for you to refer to at times of difficulty and allow you to plot your progress against your objectives.

Your research design will depend on many factors, including the type of research problem you are trying to address, the intended outcome of your research, the sort of evidence you will need, the resources and expertise you\footnote{As well as accepted research strategies and methods applied by other researchers in your field to tackle similar problems.} have, and the philosophical beliefs which motivate them. As you are a key participant in your own research, your personal views and values will also affect the choices you make while developing your research design.

Research design is also a field of study in its own right, one which has grown out of many diverse academic traditions and ways of thinking across academic disciplines and subject areas, and which is still evolving\footnote{In this young research area, there is still a lot of post--rationalisation of a particular course of research as authors looks for generalisable themes.}. As such, it is not an easy topic to digest and is one of the most challenging aspects of doing academic research. It can be puzzling for students embarking on academic research for the first time.

For this reason, in Stage 1, we will not consider research design in detail, that will be started in Stage 2 instead. However, so that you can start to think about research design, in this section, we introduce some basic definitions around evidence and research methods, and then focus on ethical and legal matters in research.

\subsubsection{Types of evidence and data}
The phenomena upon which your research will be based must be observed and this gives rise to data. Data can be interpreted to give information and evidence.

Thus, most academic research will be based on data and evidence. Data is the raw observations with no interpretation attached --- anything you may collect, observe or gather in your research. Evidence is information interpreted to support (or otherwise!) your academic arguments. Indeed, data forms the basis of evidence, so the two concepts are closely linked and often used interchangeably. This section recalls briefly the main types of data and evidence used in academic research.

\textbf{Quantitative data} are data that can be quantified or measured, and be given numerical values. They include the following types:

\begin{itemize}
\item \textbf{Numerical} data are numbers\footnote{Yes, they are!}, such as the number of students registered on a module or the temperature in the UK in July (continuous). Simplifying a little, when numerical data has a whole-number values it is called discrete, otherwise it is continuous\footnote{Given the fundamental nature of energy, and the vagaries of quantum physics, it may be that we`re incorrect in stating that real--world temperature is actually a continuous variable. However, even if it isn't, its values lie on a continuous scale.}. In either case, appropriate mathematical and statistical operations can be applied apply.

\item \textbf{Ordinal} data can be arranged in an order, but are not necessarily numerical. An example is the very widely used Likert scale\footnote{Almost certainly, the most recent survey you completed would have used the 5--point Likert scale mentioned here.} often used in questionnaires to elicit opinions. An example of a 5-point Likert scale is that ranging from Strongly disagree to Strongly agree with Disagree, Neither Disagree nor Agree, and Agree in the middle. While these values can be arranged in order\footnote{This might be done by giving Strongly disagree the numerical value 1, Disagree the numerical value 2, and so on.}, mathematical and statistical operations can only be applied with care, for instance, taking the mean (or average) score.

\item \textbf{Interval} data are ordinal data, but for which we can know the distance between any two data points. For instance, calendar dates are interval data in the sense that we can calculate the time interval between two given dates, e.g., the number of days in between.

\end{itemize}

\textbf{Qualitative data, on the other hand,} are descriptive in nature and defy ordering. Sentences, words, images, sounds, etc., are all examples of qualitative data. An important subclass of qualitative data is \textbf{categorical} data\footnote{Categorial data, also called nominal data, because data is categorised or named. An example might be Dog, Cat, Alligator, Rock.} which is data that cannot be ordered and on which mathematical operations and functions don't apply.

Data and evidence are also classed as:

- \textbf{primary,} when newly generated or collected during research; or

- \textbf{secondary}, when already available from previous research, and re-used during new research.

The academic literature that will be at the core of your literature review\footnote{A bit or a hint for the next Activity:)} is secondary evidence, as are all other published academic and non academic documents, e.g., laws, policies and procedures, official reports, etc.

\begin{question}[subtitle={Activity: Considering types of data and evidence}] In the context of your project, consider which primary and secondary data and evidence you may need, where it may come from, and what type it is, either quantitative or qualitative. Write down your answer. $<$Check: I don't know if I could do this, given where we are.$>$

\begin{guidance}If you think you might use surveys in your research, you might like to think about the data that will come from the participants.

If you're designing something, think about the forms in which a description of the problem that your design will solve will be in.

If you're {\ldots}
\end{guidance}

\begin{solution}Taking Clara's example?
\end{solution}\end{question}
%%Hack to correct tcbox behaviour
\color{black}

\subsubsection{Classes of research methods}
\textbf{Research methods} are the means used in research to collect, analyse, synthesise or present data and evidence, and to derive findings from them. Their purpose is to help you conduct your research in a systematic, rigorous, repeatable and reliable fashion.

Research methods can be classes based on their purpose into:

\begin{itemize}
\item \textbf{data} \textbf{collection methods}, used to gather data and evidence

\item \textbf{data analysis methods}, used to analyse data and evidence

\item \textbf{modelling methods}, used to build models of complex real-world situations, where many interrelated phenomena are at play and a holistic understanding is needed.

\end{itemize}

Methods are also classed based on the type of data and evidence they handle into:

\begin{itemize}
\item \textbf{quantitative methods}, which --  unsurprisingly -- are used when dealing with qualitative data;

\item \textbf{qualitative methods}, which -- again unsurprisingly --  are used for qualitative data.

\end{itemize}

Broadly speaking, quantitative methods are widely applied in the natural sciences, with their focus on measurement, natural phenomena and their simpler cause-and-effect relations, while qualitative methods are widely applied in the social sciences, with their focus on understanding human behaviour. In practice, however, this distinction is not as stark and often quantitative and qualitative methods are mixed in research, particularly when research spans several academic disciplines. Instead, modelling methods are often associated with design, computing, engineering and more generally the so-called `sciences of the artificial' (Simon, 1969), which consider technology and its development in its wider social context, focusing on addressing complex, messy socio-technical problems.

Within these broad classes of methods, you will encounter several techniques, some of which are considered in Stage 2.

\begin{question}[subtitle={Activity: Considering research methods}] In the context of your project, write down which classes of methods you are likely to apply.

\begin{guidance}Look back to your answer to the previous activity. You'll be using qualitative methods if your data is qualitative, quantitative methods if your data is qualitative, and mixed methods if there are both. For each, you should indicate their purpose in your research, and relation to the data and evidence you will need in your project.
\end{guidance}

\begin{solution}Taking Clara's example?
\end{solution}\end{question}
%%Hack to correct tcbox behaviour
\color{black}

\subsubsection{Ethical and legal considerations}
All research must be carried out legally and ethically\footnote{People who carry out unethical and illegal research cannot share their results within the academic community without being called out. They cannot, therefore, be called researchers. Examples include those who researched the effects of tobacco and found they caused cancers, but didn`t share their 'research' outside of their sponsoring organisations.$<$Is this ok?$>$}, so that it is essential you consider your proposed research and research design from these standpoints. There are some key considerations in conducting conducting research ethically and legally, which we outline in this section.

\subsubsection{Personal data in research}
In your research, may also wish to collect data about your participants. The collection and use of this kind of data is usually regulated by law, although the specifics may change from country to country, and the university with which you're studying might add specific guidance too.

Within the European Union, the EU General Data Protection Regulation (GDPR) applies.\footnote{At the time of writing{\ldots}} GDPR defines \textbf{personal data} as any information which may identify a living person, be that a name or a personal identification number, or a combination of physical characteristics, or cultural or social identities, and establishes rules for the use of such data.

It also establishes particular legal protection or safeguards for \textbf{sensitive personal data}, that is, data which may reveal:

\begin{itemize}
\item racial or ethnic origin

\item political beliefs

\item religious or philosophical beliefs

\item trade union membership

\item genetic or biometric data

\item physical or mental health

\item sex life or sexual orientation

\item criminal convictions and offences.

\end{itemize}

Depending on your university or course regulation, you may not be allowed to handle sensitive personal data in your research, and you are likely to need to follow strict protocols when handling other personal data. You may even have to apply for permission to conduct the research you wish to do, including allowing an ethical committee to see any surveys that you wish to conduct.

Moreover, in reporting your research, you should \textbf{anonymise} any personal data, that is remove any information which may, directly or indirectly, lead to a living person being identified.

\subsubsection{The rights of research participants<Is it worth asking Clara for examples of ethical research including animals?>}
In your project, you may call on other people to take part in your research, for instance people you intend to interview or observe, or who may complete questionnaires you design, or provide you with documentary evidence you require. These people have a number of rights you must respect, primarily:

\begin{itemize}
\item The right not to participate -- no-one should be pressured to take part in your research

\item The right to withdraw -- they can change their mind at any point

\item The right to give informed consent -- they should be given sufficient information on your research and their role in it for them to decide whether they wish to participate or not

\item The right to anonymity -- their identity should not be disclosed unless they give you explicit permission to do so

\item The right to confidentiality -- the data\slash information you obtain from them should be kept private if they ask you not to disclose it

\item The right to privacy -- you should not intrude unnecessarily into their lives

\item The right to protection from harm, i.e., you must take steps to minimise the risk of harm, either physical or psychological, to all participants.

\end{itemize}

\begin{question}[subtitle={Activity}] By asking your supervisor, searching on your university's intranet, or otherwise, find out what ethical guidelines you will have to conduct your research under. Add any documents that you find to your bibliographic database so that they are easy to find again.

\begin{solution}Our university has the following ethical guidelines:

$<$Add OU guidelines here$>$

\end{solution}\end{question}
%%Hack to correct tcbox behaviour
\color{black}

\subsubsection{Processing personal data}
\textbf{The `processing' of `processing personal data'} refers to any action involving personal data, including obtaining, recording, analysing, and\slash or destroying data from which a living individual can be identified.

The GDPR sets out six principles\footnote{Why not set a reminder in your diary to revisit these principles whenever your research design changes.} that must be observed when processing personal data. Some refer to the collection and intended use of personal data; others to the way personal data are stored. Even if you think GDPR will not apply to your research, the six principles are useful as a reference as your research design develops, as that development might include personal data. Table 1 summarises the six principles, alongside our recommendations on how your can embed them in your project.

Table 1--- Data protection principles and how to apply them in your project$<$are these our creation? Do they have to be as a table?$>$

\begin{tabular}{@{}p{0.5\textwidth}p{0.5\textwidth}@{}} \toprule
 Data Protection Act principles & Recommendations for your project \\
\midrule
\begin{itemize}
\item Personal data processing must be lawful and fair
\item The purposes of personal data processing must be specified, explicit and legitimate
\item Personal data collected must be adequate, relevant and not excessive
\item Personal data must be accurate and kept up to date
\end{itemize} &
\begin{itemize}
\item Introduce yourself to your participants, e.g., ``I'm Karl Poppler, a student with University XXX''
\item Explain the reason why you are collecting the data, e.g., ``as part of my final Masters project I am researching topic YYY''
\item Assure them that the data will only be used for academic research purposes
\item Explain to them that the data will form part of an anonymised research report
\item Assure them that the data will not be passed on to any third party. The only exception is your research supervisor, which is bound by confidentiality
\end{itemize}\\

 \begin{itemize}
 \item Personal data must not be kept for longer than is necessary.
 \item Personal data must must be processed in a secure manner
\end{itemize}&
\begin{itemize}
\item Inform your participants of how long you will retain the data. Although GDPR does not set a limit, you shouldn't retain personal data for much longer than the duration of your project and submission of your dissertation. If you wish to retain some of the data for longer, you should anonymise them, so that personal information cannot be leaked accidentally
\item Inform your participants of how the data will be stored securely, including the use of encryption techniques for digital data, to prevent accidental or unauthorised access to, or destruction, loss, use, modification or disclosure
\item Inform your participants if you use an online data collection service which may store the data outside the EU, possibly in counties with weaker data protection legislation
 \end{itemize}
\\
\bottomrule
\end{tabular}

The six principles apply to personal data from which a living person can be identified. This is a severe hindrance on research that could only be done with live human data. Because of this, many techniques have been created to allow the use of live human data without comprising the six principles. These generally anonymise the data, i.e., the data is processed so that any link to a living person is removed.

If you do need to work with live human data, it is therefore worthwhile to understand which techniques exist for anonymising it. If you dod need to do this, at any point of your research project, the following activity will give you up-to-date information on how to achieve anonymised live human data.

\begin{question}[subtitle={Activity: Anonymising personal data}] Do a web search to identify techniques to anonymise personal data. List and summarise the main techniques you have found.

\begin{solution}Among others, you may have encountered the following common approaches to anonymising personal data:

\begin{itemize}
\item hiding, which refers to removing personal data from a dataset, for instance, the name and address of participants in a study. Those categories are completely removed from the data set

\item masking, which refers to obfuscating personal data by replacing values with certain characters, for instance replacing all names and addresses with asterisks. As are result, the specific values are not visible, but their categories are retained in the data set.

\item pseudonymisation, which refers to replacing identifying data with made-up identification data, for instance replacing names and addressed with faked ones

\item Generalisation, which refers to replacing certain data with more general equivalent, for instance, replacing an exact address with an area code.

\end{itemize}

\end{solution}\end{question}
%%Hack to correct tcbox behaviour
\color{black}

\paragraph{}
These various ways of anonymising data\footnote{A supervisor that has conducted research in the area of your project may have used data anonymising techniques before. It's always worth checking with them what level of anonymisation is needed, and which techniques could be used to achieve this.} have the specific uses depending on how the data will be used, with those later preserving more information than those earlier. For instance, hiding, which is the most destructive method, will stop a participant in data collection being tracked over time. This is good for anonymisation, but in a clinical study, for instance, you might need to identify the outcomes of a series of tests wit a particular individual. If you have removed all identifying information from the personal data, this isn't possible.

In this case, masking --  in which participant ``Cruella Deville'' is given an identifying label ''P1' -- allow tracking over time without revealing identity.

Masking does remove all category information; pseudonymisation -- in which ``Cruella Deville'' is replaced by ``Barry Norman'' allows a named individual to be referred to later, which might help a reader to stay more engaged.

Generalisation, in the other hand, could be used with other ways of identifying a person, so that their postcode could be replaced by the first component only, for instance, MK7 6AA -- which identifies a street location --  could be replaced by the less specific MK7 which is much less specific, but without removing all location information.

\subsubsection{Equity, Diversity and Inclusion in research}
Equity, diversity, and inclusion (EDI for short) are important ethical considerations in research Your university is likely to have policies that guide research towards good EDI. Not only is EDI important for society, but good research EDI leads to better research as your and others' biases can be compensated for.

Such policies may be aligned to national or international guidance. For instance, in the UK, national EDI definitions and principles have been established across the academic sector by UK Research and Innovation, which is a government body that brings together all UK research councils responsible for supporting research and knowledge exchange in UK higher education institutions.

As stated by the UKRI EDI principles, ``research and innovation should be `by everyone, for everyone' -- a dynamic, diverse and inclusive research and innovation system in the UK is an integral part of society and should give everyone the opportunity to participate and to benefit.'' And also that ``By valuing all, we recognise that a diversity of ideas, opinions, knowledge and people enriches our work and enlarges our knowledge economy.'' [Add source]

Equity\footnote{It is worth noting that both equity and equality are often used in the literature as the `E' in EDI. There is, however, a difference: equality refers to treating everybody in the same way, while equity acknowledges specific barriers or obstacles which affect certain individuals or groups of individuals, and seeks to remove them. The latter is now considered the better definition of the two.}, Diversity and Inclusion are as follows:

\begin{itemize}
\item \textbf{Equity,} which relates to fairness and justice, in the sense of removing barriers or bias which may prevent individuals, or groups of individuals, from having equality of access, opportunity or outcomes.

\item \textbf{Diversity}, which refers to the full spectrum of differences and similarities between individuals, whether socio-demographic, such as age, gender, race, ethnicity, etc., or in terms of beliefs and values, life experiences or personal preferences.

\item \textbf{Inclusion}, which concerns ensuring that all individuals feel welcome, valued and confident to be treated fairly and respectfully. Inclusion is often paraphrased as `diversity becoming normal'.

\end{itemize}

In thinking about EDI in your research project, you should consider:

\begin{itemize}
\item Yourself as a researcher: how do your own cultural perspectives and preferences affect your interactions with others or influence the way you consider data and evidence in your project?

\item Your research context: what insights do you have on the diversity of people you will interact with or may be affected by your research?

\item Your research activities: how will you account for equity and diversity in your research?

\end{itemize}

\begin{question}[subtitle={Activity: Considering EDI in your project}] Which EDI considerations apply to each of the following:

\begin{itemize}
\item Inviting human participants to take part in a research project

\item Designing a novel artefact, say a new technology or process

\end{itemize}

Write down your answers.

\begin{solution}Your answer may include some of the following:

\begin{itemize}
\item When inviting research participants, it is essential to consider their diversity to ensure they form a representative sample of the population under study.

\item When designing a novel artefact, it is essential to take the end-users diversity into account, so not to disadvantage some groups of individuals, for instance in terms of accessibility.

\end{itemize}

\end{solution}\end{question}
%%Hack to correct tcbox behaviour
\color{black}

\subsubsection{Intellectual property}
WORK IN PROGRESS

Intellectual property (IP) is the owned property\footnote{In this definition `property 'has the legal meaning of a collection of rights applying to the things which have been created, rather than the things themselves. If you find this confusing, think about a house or a car you may own. In legal terms, you have the property of that house or car, which gives you certain rights in law. However, in common language you may refer to your house or car as your own property. The two meanings are often confused in the wider literature on intellectual property. In this handbook we will always use the term IP to refer to the system of rights.} created through intellect, such as an invention, an artwork, a formula for some chemical compounds, etc.

Some IP gives rise to specific rights which are protected by law. For instance, the UK UK Intellectual Property Act 2014 \footnote{`\textbackslash{}url\{\href{https://www.legislation.gov.uk/ukpga/2014/18/contents/enacted}{https:\slash \slash www.legislation.gov.uk\slash ukpga\slash 2014\slash 18\slash contents\slash enacted}\}`\{=latex\}} defines a number of IP rights, establishing whom they belong to and what they allow the owner to do.

\begin{question}[subtitle={Activity: Types of IP rights}]
Conduct a web search to identify way of protecting intellectual property. For each, write down a definition and comment on whether they may arise from academic research.

\begin{solution}Your answer may include the following:

\begin{itemize}
\item Copyrights --- A copyright is an IP right that applies to any original work that is expressed in a tangible form, say a poem or a painting. It gives legal right to its creator(s) to reproduce, publish and distribute the IP, and to transfer it to a new owner. A creator acquires copyright automatically. A research dissertation, and any part within, gives rise to copyrights.

\item Patents --- A patent is an IP that prevents anybody by the owner from making, using, or selling an invention for a limited period of time. Patents are usually associated with products and processes, e.g., a microchip in a mobile device, or the process to engineer bacteria that digest plastic. For the owner's rights to be protected in law, a patent must be registered with a patent office, and different regulations apply in different counties. Academic research can generate patents, but the protection is not automatic: it requires appropriate registration to take effect.

\item Trademarks --- A trademark applies to something which can be used to distinguish commercial products and services of a trader from all others within the same market, e.g., the `apple' used on all products by Apple inc. A trademark must be registered, but there is no time limit to the protection it provides. Academic research may lead to new products or services which could be protected by registered trademarks.

\end{itemize}

This is not an exhaustive list and different countries' legislation include different forms of IP. For instance, the UK IP Act mentioned above includes `design' as a form of IP, which relates to the appearance, shape or configuration of a product, rather than the product itself (as the case for patents). $<$Check$>$ You may have also encountered `Trade Secrets', which prevents anybody other than the owner to disclose them: the famous CocaCola formula is proceed in this way.\footnote{Merriam Webster has this definition: something (such as a formula) which has economic value to a business because it is not generally known or easily discoverable by observation and for which efforts have been made to maintain secrecy}\end{solution}\end{question}
%%Hack to correct tcbox behaviour
\color{black}

\paragraph{}
Your university is likely to have an IP policy which establishes, among others, which IP applies in the context of academic research and Masters studies, defining both ownership and related rights.

\begin{question}[subtitle={Activity: Understanding your university's IP policy}] Using your university's intranet look up any IP policy you university has. Write a brief summary of what you have found out in relation to possible rights which apply to your Masters project work.

\begin{solution}At The Open University in the UK, the IP policy establishes rights in relation to the ownership, development, protection and exploitation of IP arising from all types of research and scholarship carried out in the institution. Although the policy changes depending on the type of invention, in relation to Masters research projects which are part of a taught course of studies, our university assigns the IP to the student, as long as they have paid their university fees. However, in case of fee waivers or bursary from the university, the IP will belong to the university$<$Check$>$.
\end{solution}\end{question}
%%Hack to correct tcbox behaviour
\color{black}

\subsubsection{Bias in research}
Bias\footnote{Several forms of bias have been recognised in research, with some becoming more and more prominent with the rise of AI research and applications.} can damage research findings, giving outcomes which are unreliable, casting doubt on the claimed knowledge contribution. As well as being experts in their subject, expert examiners are also experts in the forms of bias that can damage research in their field. You should therefore act to remove all bias from your research, and to use every resource at your disposal to identify which forms of bias might be detrimental.

\begin{question}[subtitle={Activity: Types of bias}] You can do some preliminary work on identifying general biases that might affect you by looking on the web. Conduct a web search on types of bias in research in general, and in relation to data in particular. List and summarise the main kinds you find, and measures that can be applied to mitigate them.

\begin{solution}Among others, you may have encountered the following:

\begin{itemize}
\item Confirmation bias, which is the tendency to favour the selection, analysis and interpretation of data which support the researcher's prior beliefs.

\item Observation bias\footnote{Observation bias is also known as the Hawthorne effect. You can read about the original Hawthorne experiment here: \textbackslash{}url\{\href{https://en.wikipedia.org/wiki/Hawthorne_effect}{https:\slash \slash en.wikipedia.org\slash wiki\slash Hawthorne\_effect}\}}, which is the tendency of participants being observed during a study to change their behaviour, possibly to please the researcher or provide the answers they think the research is seeking.

\item Selection bias, which occurs when data is selected subjectively, leading to samples which are not representative of the population under study.

\item Recall bias, which is the tendency of people to recall certain types of events more vividly than others, and can affect the outcomes of research which relies on participants' memories of the past.

\item Algorithmic bias, particularly in AI applications, which occurs when an algorithm produces outcomes systematically and repeatably which disadvantage one group of individuals over others.

\end{itemize}

Bias is subtle. We are all affected by it. It is easy to see in others, but extremely difficult to see in oneself. To be a good researcher you should assume that you are affected by many different forms of bias and work to eliminate them from your work. The upside is that you will be a better researcher for doing this.
\end{solution}\end{question}
%%Hack to correct tcbox behaviour
\color{black}

\paragraph{}
All bias can damage your research. However, confirmation bias is very easy to fall into, can be very disruptive, and can arise when you have too personal a stake in the research beyond the project itself. For example, you may be the manager of a process you are seeking to improve. This means that you will have a stake in the research subject beyond meeting your Masters academic requirements. Although this can be a strength, in that, since you already have an interest in, and knowledge of, the research subject and context, you are likely to have insights into the causes of problems and the factors that have an impact on them, it can also be a weakness in that it may lead you to focus on evidence that confirms any existing beliefs that you might have about how the process should improve, while dismissing evidence that does not support them\footnote{In the worst case, you might find that your current approach to improvement is wrong which might be visible to others within your organisation. If you find yourself in this situation and, in our experience, it does happen, it can be very hard to deal with this sort of bias.}. This may result in a lack of objectivity, and a tendency to make subjective judgements instead of an evaluation based on sound evidence.

Another effect of confirmation bias is that you clearly see a problem where no-one in your readership does. So much so that you don't feel you have to explain the problem or convince others that it exists.

In our experience, both of these can hamper progress. The latter is especially pernicious: that you see a problem is great, but can you find sufficient evidence in the academic literature?

\begin{question}[subtitle={Activity: Potential bias in your research}] Make a note of any potential bias that you may bring to your research. Conduct a web search to identify possible measure you could adopt to prevent them from affecting your objectivity.

\begin{solution}Your answer will depend on the king of bias you have been focusing one. For instance, for confirmation bias you may have found the following measures:

   	having participants, colleagues and\slash or your peers review your arguments and results;

   	making use of diverse data\slash evidence sources;

   	intentionally looking for alternative explanations as part of your data\slash evidence analysis.

LR --- LOOK UP OTHER MEASURES FOR OTHER KINDS OF BIAS
\end{solution}\end{question}
%%Hack to correct tcbox behaviour
\color{black}

\subsubsection{Your responsibility as an ethical researcher
}
To summarise, your responsibilities as a researcher are to:

\begin{itemize}
\item Behave with integrity, i.e., respect the rights of your participants, be open and honest about how you have conducted your research and about your results, including not committing plagiarism, ensure validity and accuracy in the collection and reporting of data, and disclose any conflict of interest, e.g. personal interests or relations with research participants which may compromise your judgement .

\item Comply with ethical standards in research, including EDI, laid down by appropriate bodies, including your university, and possibly to other codes set out by professional bodies in your field of study.

\item Comply with legal requirements in relation to health and safety$<$do we mention these?$>$, and data protection.

\item Guard against all form of bias in your research

\end{itemize}

In addition, there may be further guidelines established by your own university or course of study. For instance, you may be prevented from conducting research in which participants who cannot provide fully informed consent, or which requires you to collect sensitive personal data or commercial purposes. There may be also circumstances in which you will need an explicit permission, for instance if you require your participants to discuss sensitive issues, or be subject to prolonged interviewing, testing or observation.

\begin{question}[subtitle={Activity: Considering your university's ethical and legal guidelines}] Look up your university and course regulations to check any specific ethical and legal guidelines or constraints which apply to your project. Write a brief summary of what you have found out.

\begin{solution}Our own institution, The Open University, UK, has an extensive set of ethics and legal policies and guidelines related to research, alongside processes to gain approval when dealing with human participants and personal data. In addition, our Masters courses put further restrictions on the kind of research which can be conducted, including not allowing research involving minors or vulnerable adults, or the collection of sensitive data.
\end{solution}\end{question}
%%Hack to correct tcbox behaviour
\color{black}

\subsubsection{Sketching your research design}
Let's get a first sketch of the elements of your project research design.

\begin{question}[subtitle={Activity: Summarising elements of your research design}] Based on your answers to the previous activies, address each of the following questions:

\begin{itemize}
\item Which evidence and data you will need in your project and why?

\item Where will such data\slash evidence come from, and how will you ensure access?

\item Which kind of research methods do you intend to apply and why?

\item Which ethical and legal issues are relevant to your project, and how you will address them?

\end{itemize}

\begin{guidance}You should provide an explicit rationale with reference to your research problem, and intended aim and objectives. You should also comment on why you think those choices are feasible within the constraints of your project.

To address feasibility, you should consider the extent your choices are:

\begin{itemize}
\item Effective --- they should produce the data\slash evidence you need.

\item Manageable --- you should be able to apply them within the time available

\item Efficient --- they should produce data\slash evidence that you can process with the skills, resources and time available.

\end{itemize}

\begin{solution}What's our answer for Clara?

\end{solution}
\end{guidance}\end{question}
%%Hack to correct tcbox behaviour
\color{black}

\section{Managing risk}
``What could possibly go wrong in a flourishing concern like the Brixleigh Bank\footnote{Later in the book, Brixleigh's Bank is shown to be a Ponsi scheme -- a gigantic fraud -- which fails.}? My dear, you are too fond of conjuring up imaginary evils.'' -- Matilda Mary Pollard's \emph{Cora: Three Years of a Girl's Life}, 1882

Pollard's imaginary evils are now called risks. They are no longer seen as imaginary evils, but as things that need managing.

\subsubsection{Risk in your project}
Risk captures the likelihood of something going wrong combined with the impact that will have on your project, both on time, resources and outcome. In theory, both positive and negative impacts should be considered, but very often the focus is on what can affect your project in a bad way, letting the good stuff roll.

The management of risk is an important discipline in its own right --  you do not need all of the tools that that discipline offers.

So, in analysing risk for your project, you should focus on the following dimensions:

\begin{itemize}
\item Specific risks: What sort of things can go wrong? For example, you may not be able to recruit sufficient respondents to a survey to gain direct access to key evidence.

\item Impact: What are the consequences if things do go badly? How severe might those be? For instance, not obtaining key evidence will invalidate your all research, so this would be very severe in terms of your project outcome.

\item Likelihood: How likely is it that things will go wrong?

\item Mitigation\slash contingency: What can you do to reduce likelihood or impact? For example, you may have lined up a secondary source of evidence, which may not be as authoritative or useful as what you had in mind originally, but would be easier to access and still allow you to derive some interesting results.

\end{itemize}

In a research project there will be risk which is very specific to what you intend to do, but there are also risk categories which are common to all project, which we consider next.

\subsubsection{Technical skills}
Your intended project may require you to apply expert technical skills, for instance, coding or advanced statistical analysis. Early on in your research project, it may even be possible that the details of precisely which technical skills you will need are not clear -- will conducting a survey, for instance, require sophisticated statistical skills?

Your risk analysis should recognise this:

Specific risks: that your technical skill level isn't sufficient to be able to exercise those skills;

Impact: a lack of appropriate skills might mean that you are not able to analyse your data as well as you would like, losing its value.

Likelihood\footnote{Some approaches to risk ignore likelihood, assuming Murphy's Law, i.e., that anything that can go wrong will go wrong. This makes risk}: how likely is it that skills that you don't possess will be needed?

Mitigation\slash contingency: to reduce the impact (to manage the risk) you can take a course that helps you understand what level of skills you will need, and the love that you actually possess. This is an item that you can including in your project plan.

\begin{question}[subtitle={Activity: Risk in relation to technical skills}] Consider whether there are bespoke technical skills you do not possess, but are essential to your intended research. Perform a risk analysis in relation to their development and write down the outcome.

\begin{guidance}You should record the specific skill development risk, including likelihood, impact, and any mitigation\slash contingency.

On mitigation: if you don't think you will have developed all necessary technical skills in good time, then you should consider whether you have made the right choice of project -- this is also a way of managing risk! At Masters level, it is a lot safer to focus on research which makes good use of the technical skills you already possess.
\end{guidance}\end{question}
%%Hack to correct tcbox behaviour
\color{black}

\subsubsection{Study time}
Consider the time which is required for your project as indicated in your course guidance, and the fact that you should sustain a continuous effort throughout, with little scope for making time up when you are not able to or taking long breaks.

\begin{question}[subtitle={Activity: Risk in relation to study time }] Consider your current personal and professional commitments, and study practices. Perform a risk analysis on whether you will be able to dedicate sufficient time to your project on a regular basis and write down the outcome.

\begin{guidance}Under mitigation\slash contingency, you may include any adjustment to your current study practices and patterns that may be needed. If substantial, that you will need to assess very carefully how feasible it is for you to make such changes. $<$Here and later: should we be adding questions to ask? Do we have them?$>$

\end{guidance}\end{question}
%%Hack to correct tcbox behaviour
\color{black}

\subsubsection{Resources}
Depending on your chosen research problem and aim, you may need access to participants, organisational information, third-party data, industrial case studies, etc., or you may need to acquire specialised software or hardware.

It is important for you to assess how likely it is that you will be able to gain access to or acquire such resources and, should there be any cost involved, whether you can afford it.

If conducting research with your current employer, you should also consider the extent the data you require are confidential and non disclosable, as well as the possibility of changing jobs, and the extent you may be able to retain access or make alternative arrangements in such a case.

\begin{question}[subtitle={Activity: Risk in relation to resources }] Consider resources you are likely to need to conduct your research. Perform a risk assessment in relation to your access to those resources for the duration of your project and write down the outcome.

\begin{guidance}If you don't think you will have access to all necessary resources, then you should consider refocusing your project, so that you can make best use of resources you already have or will find easier to access.

\end{guidance}\end{question}
%%Hack to correct tcbox behaviour
\color{black}

\subsubsection{Ethical and legal issues}
In your initial development of your research design, you should have identified ethical and legal issues which are pertinent to your project. Here you are asked to consider any related risk.

\begin{question}[subtitle={Activity: Risk in relation to ethical and legal issues }] Consider the ethical and legal issues you identified as relevant to your project. For each, perform a risk analysis and write down the outcome.

\begin{guidance}You should pay particular attention to issues of health and safety, data regulations, permissions to access data or documentation from third-party, and the need for explicit permission to proceed from your own university.

It is essential you identify any ethical or legal impediments to your intended research earlier on, as these will prevent you from conducting your chosen research
\end{guidance}\end{question}
%%Hack to correct tcbox behaviour
\color{black}

\subsubsection{Summarising your project risk}
You can use Table 1 to summarise your risk analysis in relation to your project.

Table 1 --- Project risk analysis table

\setlength\tymin{2.5cm}%ensure small columns are not squeezed
\begin{tabulary}{\tablewidth}{@{}LLLLLL@{}} \toprule
 \textbf{Risk category} & \textbf{Specific instance} & \textbf{Likelihood} & \textbf{Impact} & \textbf{Mitigation\slash contingency} & \textbf{Guidance} \\
\midrule

 \textbf{Lack of technical skills for primary research} & & & & & You should consider essential technical skills needed in your research and whether you will be able to develop them in good time to meet your project milestones \\
 \textbf{Lack of study time} & & & & & You should consider your current personal and professional commitments, and study patterns and how these may fit the study assumptions fro your project \\
 \textbf{Lack of access to resources needed, including participants and secondary data} & & & & & You should consider both what you will need for your project and how you will ensure access \\
 \textbf{Ethical and legal constraints} & & & & & You should pay particular attention to health and safety, data regulations, and permissions from third parties \\
 \textbf{Other (specific to your project)} & & & & & You should consider if there is other risk which is specific to your proposed research and not covered by the other categories \\
\bottomrule
\end{tabulary}

\begin{question}[subtitle={Activity: Risk assessment for your project }] Complete your project risk assessment by filling in the entries in Table 1.

\begin{guidance}You should already have most of the required content from your previous activities.

You should ensure you consider carefully any specific risk to you project which is not covered by the generic risk categories we have included in the table.
\end{guidance}\end{question}
%%Hack to correct tcbox behaviour
\color{black}

\section{Reflecting}
[TBC]You might be thinking: ``Why, if I've finished a task, should I cause myself grief by reflecting on it -- in the best case nothing will change. In the worst case, I'll have to do it again.''

We feel exactly the same -- reflecting on what you've written causes a lot of grief, especially when you realise that it's not as good as it could have been\footnote{We`ve reflected on the materials for this book many many times{\ldots} that's meant redrafting, adding extra and -- hurtfully --  having to remove stuff that just wasn`t good enough. It's changed --  for the better we hope -- because of it. You are free to disagree, and we'd welcome your reflection on it too.} and there are very good reasons to improve it.

At the same time, we've learned a lot from that reflection. For instance, we've been able to split up complex issues into more digestible chunks; we've identified links between topics that we didn't know about; we've read more about what the students we have worked with to understand more precisely their contributions; we've taken apart the materials we have written to ensure their validity in the master's research context; and, last but not least, we've been able to generalise much of what we know of this topic to be more applicable across the board, while at the same time realising that there are special topics that will affect only a small part of our readership.

There is a context into which all research fits that colours the knowledge that is its primary contribution making it less valuable, less distinct, less out there. That context is a hegemony of received wisdom, of `common sense thinking', of uncritical investigation, none of which are necessarily knowledge. Reflection --  ''the voluntary disobedience of thought and reasoned undocility'' according to Foucault (1985) -- is your way of breaking that mould of stepping out of conventionalism and of shaking up the world.

Reflection is the thinking of radical thoughts. Improving your research through ever deeper reflection as you progress is the best way to engage (and impress) your reader\footnote{{\ldots}and your examiner, who will want to know that you have reflected on what you have done.} and keep them reading to the end.

\begin{question}[subtitle={Activity}]
$<$Needs assessing for content and structuring into activity + guidance$>$

This activity has four parts: the first is something you should be doing regularly, but won't make you into a disobedient or indocile thinker. The second, third and fourth may help you get started and keep going.

Part 1: Think about your study this far -- using this book and anything you've done for your dissertation in parallel -- as a journey. More from elsewhere, including   !!.

Part 2: think about yourself and the way you think. How does your desk look? Is it messy or tidy? Do the same for your computer desktop. Is it empty or are there hundreds of files strewn across it? Do you think your tidiness or untidiness will affect the way you do your research? How about how you keep your -- critically important -- bibliographic database which may contain up to a hundred academic\footnote{It's not unknown to have more than a hundred.} and other articles by the time you're finished?

Part 3: think about the context of your research. Which professional pressures are there on you to succeed in solving your research problem? Pressures could come in many forms: financial -- there's a promotion for you at the end of it; peer -- your colleagues know that you are studying will have good expectations of your result and you'll want to prove them right\footnote{Or wrong, depending on the colleague!}. Are you sponsored by your employer? Will you be able to report a negative outcomes to your research, for instance, that there is no solution to our problem using the current technology stack? A negative result is a very good research outcome, even if it tends to satisfy fewer non-academics than a positive result.

Which family pressures do you feel? It's' not unusual that you will have given up a paying role to study, moving the responsibility to provide onto another member of your family. What are their expectations?

Part 4: what's that thought nagging at the back of your mind? Is it ``How will I start?'' Or ``Will I be able to dedicate enough time to this?'' Or ``Can I really do this?''. Or ''Is ``shouldn't I be bringing in a wage rather than studying?''

You may be one of the lucky ones that doesn't have such negative thoughts, but negative thoughts are a very natural part of steps into the unknown. And research is precisely that, a step into the unknown. At least if you are aware of the doubts you naturally have, you can manage them. Think about making even the tiniest of steps forward in your research visible and celebrated! Work with Kansan boards where progress is encouragingly visible as you move a task from the inbox to the outbox. If you have concerns about managing your time, start using one of the many tools out there that break time up into manageable units and help manage it for you. If your concerns are about how to organise your thoughts, look into mind maps, lists, todo lists.

Thinking early and often through reflection is a powerful way of doing better. Do it well and your final report will be better than you will have expected.

It's worth saying that, at the end of what could be an exhausting journey, you will not fully appreciate your achievements. That realisation may have to wait until you are rested, graduated, or some distant time later.

But it will come.

\begin{guidance}Something here
\end{guidance}\end{question}
%%Hack to correct tcbox behaviour
\color{black}

\section{Reporting}
It may not feel like it, but you're now ready to write a substantial contribution to your research project: your full research proposal.

We recommend that you write a report at the end of each stage to consolidate the work you have carried out, regardless of whether your course may require to do so. Writing such reports will help you develop your dissertation incrementally, and provide good practice to improve your academic writing skills as you go along.

\subsubsection{Putting your research proposal together}

Here, in Stage 1, your report\footnote{Although not all what you write here will end up in your actual dissertation, substantial parts of it will. You're definitely started now and that has to feel good!} will consist of your full research proposal, which we recommend you structure as indicated in Table 1 --- subsequent reports will build on this structure by adding further elements.

Table 1 -- Research proposal structure and guidance

\begin{ltabulary}{\tablewidth} {@{}LL@{}} \toprule
 \textbf{Report template} & \textbf{Guidance} \endhead\midrule
 Proposed title & Your title should capture succinctly research problem and aim \\
 Sect 1 - Introduction 1.1 Background to the research 1.2 Justification for the research & This section should provide an introduction to your research topic in its wider context (as background) and your justification of why the research is worth pursuing. It should be well articulated and supported by evidence \\
 Sect 2 - Literature review 2.1 Review of existing relevant knowledge 2.2 Planned literature review & Your review should provide a critical summary of your in-depth engagement with the academic (and other) relevant literature to date, including identifying key trends, ideas and possible knowledge gaps. Most of your citations should point to academic articles. Your planned review should identify further reading you will undertake in the next stage. Both coverage and depth of your review matter. You should ensure that your review is well structured, with a logical narrative flow and your arguments are well supported by evidence \\
 Sect 3 - Research definition 3.1 Problem statement 3.2 Aim and objectives 3.3 Knowledge contribution & You should ensure that your research problem is well articulated and appropriate for your course and your personal and professional circumstances, that your and objectives are consistent with research problem, and that the intended knowledge contribution of your research is clearly articulated \\
 Sect 4 - Research design 4.1 Evidence and data 4.2 Research methods 4.3 Ethical and legal considerations & This section should demonstrated your initial engagement with research design, particularly that you have thought about the kind of evidence and methods you may need, appropriately justified in relation to your research problem, aim and objectives. It should also demonstrate your careful consideration of ethical and legal matters, and that your research will comply with your course and university requirements \\
 Sect 5 - Assessment of your proposed research 5.1 Qualification fit 5.2 Personal and professional fit 5.3 Technical skills and resources needed 5.4 Statement of feasibility & In this section you should argue how your research is a good fit across all criteria. You should provide a clear rationale as to why you think what you are proposing is feasible \\
 Sect 6 - Planning, scheduling and risk assessment 6.1 Statement of progress 6.2 Key priorities in follow-up stage 6.3 Risk assessment & In this section you should reflect on the progress you have made in Stage 1 and establish your priorities for the next stage. You should also summarise the outcome of your risk assessment, focusing on your major risk and how you intend to manage it \\
 References & You should keep your references in good order and ensure you apply the required bibliographical style consistently. Ideally, you should use a BMT to generate and integrate your references within your report \\
 Appendix - Work schedule & If you have created a work schedule, you could include it as an appendix for reference \\
 Appendix - Risk assessment table & You could include your filled-in risk table as an appendix for reference \\
\bottomrule
\end{ltabulary}

\begin{question}[subtitle={Activity: Putting your report together}] Using your word processor of choice, create a report with the structured indicated in Table 1, and fill it in by following the guidance provided in the table, making good use of your notes and summaries from, and reflection on, all related activities you have carried out so far.

\begin{guidance}In this first pass at putting together your report you should focus primarily on completeness, ensuring that each section includes at least draft content.
\end{guidance}\end{question}
%%Hack to correct tcbox behaviour
\color{black}

\subsubsection{Assessing, Iterating and finalising}
After you have filled in your report with as much material as you can, you should review and revise it until you are happy with your account, and ready to move on. This may take more than one iteration, but you should ensure you do not delay your work for the follow-up stage.

In the next activity, you will use Table 1 to assess whether your report is of good standard.

Table 1 - Criteria to review your report

\begin{tabulary}{\tablewidth} {@{}LL@{}} \toprule
 \textbf{Criteria} & \textbf{Prompts} \\
\midrule
 \textbf{Completeness} & Are all sections of the suggested structure completed in line with the guidance provided? \\
 \textbf{Good academic writing practices} & Have you applied good academic writing practices throughout? \\
 \textbf{Logical structure and flow} & Have you structured your narrative appropriately to ensure a logical flow of arguments? \\
 \textbf{Supporting references or evidence} & Are your key arguments supported by appropriate references or other evidence? \\
 \textbf{Citation and reference style} & Do all your citations and references comply with the required bibliographical style? \\
 \textbf{Avoiding plagiarism} & Have you acknowledged the work of others and distinguished it from your own appropriately? \\
 \textbf{Standard of English (or any modern language you use)} & Have you proof-read your report carefully to remove all typos and grammatical errors? \\
\bottomrule
\end{tabulary}

\begin{question}[subtitle={Activity: Reviewing your report}] Apply the criteria in Table 1 to review and reflect on your current report and write up a summary of your assessment.

\begin{guidance}For each criteria, consider the related prompts to help you assess your report overall, and write down any further work needed for your next stage.
\end{guidance}\end{question}
%%Hack to correct tcbox behaviour
\color{black}

\subsubsection{}
Writing up your report is an excellent way to communicate the work you have completed and still planning to do, and is something tangible you can share with your supervisor for comment and other formative feedback.

