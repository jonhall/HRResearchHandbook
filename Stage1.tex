\part{The Scope}\label{part:scope}

In all research, but particularly at Masters where time is limited, it is important to establish the scope of your project early on: this is the main focus of Stage 1. 

\todo{Resolve confusion around here}


Stage 1 is fundamental as it gives you the solid foundation for the whole of your research project. As such it is an intense stage, where you will be expected to undertake the wide range of research activities summarised in Cref{tab:stage1}. The table has entries for all but two of the research process activities that were introduced in Cref{sect:ResearchActivities}\todo{Check xref}. We did say that Stage 1 was intense!

\section{Stage 1 Outcomes}\label{s:outcomes}

Each stage has two really important tables that structure and drive the work by which you will see progress in your research and in the creation of your dissertation. 

For the research, each stage contains a Research Activity table\footnote{The first such table is Cref{stage1ResearchActivities}, and we'll get back to it in a minute or so.} and a Writing Outcome table\footnote{The first of which is Cref{stage1WritingOutcomes}.} through which you can understand and track the stage's contribution to your dissertation.

So you get familiar with the tables, they're of the same form and in the same place in the stage – right at the beginning, up front, and centre.

\subsection{Your Research Activities for this stage}

\todo{More here}

\newcommand{\RActivitiesTableCaption}[1]{Stage #1{} Research Activities}

\begin{SimpleNColTable}{stage1ResearchActivities}{3}{\RActivitiesTableCaption{1}}[X[1]X[2]X[2]]
Research activity&Description&Supervisor Interaction Focus\\
Identifying the research problem&Develop problem statement and intended contribution to knowledge&Suitability of research problem for academic research and to meet the requirements of specific  course\\
Reviewing the literature&Compile initial draft of literature review and plan remaining review&Scope of literature review and possible gaps\\
Setting research aim and objectives&Define aim and  objectives&Suitability and feasibility of aim and objectives in relation to research problem and project time\\
Developing the research design&{Consider elements of  research design, including data and evidence, and types of research methods\\
Complete ethics assessment, including applying for permission to proceed, if needed}&{Consistency of choices in relation to aim and objectives.\\
\\
Compliance with own university’s ethical and legal guidelines}\\
Gathering and analysing evidence&n/a&\\
Interpreting and evaluating findings&n/a&\\
Reporting&Write up Stage 1 report&Demonstration of critical thinking and good academic writing, and any improvements required\\
Reflecting&At stage end, think critically about experiential learning in relation to the research process and its activities&\\
Planning work&Draw your initial project timeline and a detailed plan for Stage 1&Appropriateness of initial work plan\\
Managing risk&Assess project risk&Consideration of major risk	
\end{SimpleNColTable}

\subsection{Your Writing Outcomes for this stage} 

\todo{More here}

%%TODO: update all tables to better describe learning outcomes
%Research activities table Stage 1
\begin{ResActtblr}[caption = {Stage 1 Research Activities (15\% of project length)\label{tab:stage1}}]{}
\tableheader
\reportline5{Research problem}\\ & initial research problem statement & be able to express a well-formed research problem & 15\% & suitability in relation to study requirements\\
\reportline5{Literature review}\\  & summaries of literature reviewed, organised by theme & apply Keshav's process to assimilate the content of an academic paper; use a summary-comparison matrix to track the content of articles; use a concept matrix and concept map to identify emerging themes; write up summaries of the literature around emerging themes & 30\% & scope of literature review and remaining gaps \\
\reportline5{Research aim and objectives}\\  & initial aim and objectives statement & be able to write appropriate research aim and objectives  & 5\% &  suitability and feasibility \\
\reportline5{Research design}\\  & kinds of data, evidence and methods your project may require; ethical considerations you will need to take into account & know the kind of data and methods used in research; know which ethical and legal issues may affect research & 10\% & ethical and legal regulations which apply to your studies \\
%\textbf{Gathering and analysing evidence} & n\slash a & & 0\% & \\
% \textbf{Interpreting and evaluating findings} & n\slash a & & 0\% & \\
\reportline5{Reflection}\\  & your collection of thoughts about your own learning and practice  & know the role of reflection in research & 10\% &  \\
\reportline5{Reporting}\\ & Stage 1 report & be able to assess your research progress; be able to apply the given template to write up your Stage 1 report & 20\% &  improvements required \\
\reportline5{Work Plan}\\  & outlined work plan & be able to use Gantt chart (or similar) to produce a work plan & 5\% & suitability and possible adjustments \\
\reportline5{Risk management}\\  & summary risk table & know what risk is; know how risk can affect research; be able to assess risk in your project & 5\% & possible missing entries \\
\end{ResActtblr}

\begin{ReportTable}{stage1WritingOutcomes}[\WOCaption{1}{20}]
%%\ReportTitle1
\ReportTitle*
	&At this stage, give your provisional title 	&\Cref{sect:ChoosingATitle}\\

%%\ReportAbstract12345
\ReportAbstract-----
	&\tablenocontent
\\

%%1-\ReportIntroduction12345
\ReportIntroduction*!!--
	& At this stage you should be able to give an introduction by providing some background to your research topic in its broader context, and some justification for your research in terms of your context of application	&\Cref{sect:stage1ChoosingATopic}
\\

%%2-\ReportLitRev---
\ReportLitRev*!!
	& Create sections for your review of the literature you have read (as analysis), and for your critical summary (as synthesis). Your synthesis will include key trends, ideas and possible knowledge gaps. Most of your citations should typically be academic in nature. This will not be a complete draft as yet – many references could still be missing, to be added in later stages; these references that remain to be read should be captured in your progress tracking appendix (described below) and added to as other papers come to light &\Cref{sect:stage1LiteratureReview}
\\

%%3-\ReportResearchDef----
\ReportResearchDef*!!!
	& Initial formulation of your problem statement, aim and objectives and potential knowledge contribution based on your increased understanding from deeper engagement with the literature
	
	%%\itemize in tblr header
	\item The aim is your reformulation of problem statement based on your increased understanding from deeper engagement with the literature

	\item The aim can be broken down into a few research objectives 

	\item The knowledge contribution is deepened based on your increased engagement with the literature. &\Cref{sect:stage1ResearchProblem,ssect:ProblemFormulation,ch:SettingResearchAim,sect:Articulating,ssect:KnowledgeGap}\\

%%4-\ReportResearchDes-----
\ReportResearchDes*---!
	& Your initial consideration of ethics and regulations relevant to your project, including those of your course and university.\\

%%5-\ReportAnalysisInterp----
\ReportAnalysisInterp----
	& \tablenocontent\\

%%6-\ReportEvalConc-------
\ReportEvalConc-------
	& \tablenocontent\\

%%7-\ReportRefs-
\ReportRefs-
	& You should include those references that you cite in previous chapters &\Cref{stage1refs}\\

%%A-\ReportAppendices-
\ReportAppendices-
	& See below\\
	
%%\ReportProgressTracking123456789
\ReportProgressTracking*********
& You should reconsider the appropriate checklist, adding sections that records your understanding of why your research is feasible against the requirements for your degree and a growing list of references that you have still you read, analyse and add to your synthesis in the literature review&\Cref{stage1progress}
\\
%
%%\ReportReflection123
\ReportReflection*!!
 & You should again reflect on the progress you have made in Stage 1 using the appropriate material and set your key priorities for work in Stage 2&\Cref{stage1workplan}
\\
\end{ReportTable}

\endinput


Looking at the table, of the activities covered in this stage, the \textit{Literature Review} is the most demanding at 30\% of overall effort for this stage, closely followed by \textit{Reporting} at 20\% and \textit{Identifying the research problem} at 15\%, the former consisting in putting a report together after you have developed its constituent parts.
Together, reviewing and literature and identifying the research problem are your first steps to understanding what your project's contribution to knowledge might be and so it is good to invest a lot of time in these. Setting aim and objectives (5\%) and starting to develop your understanding of research design (10\%) follow from them.

We have already mentioned the importance of reflection in research, and how it is essential for you to reflect on your increasing understanding of the research process and your own research practice as you go along. Unsurprisingly, therefore, the framework accounts for 10\% of your time spent doing just that. 

Equally, we have also stressed the importance of considering risk in your research and to manage your time effectively, therefore 5\% of your time should be devoted each to risk management and to work planning.

In this chapter, you will start by thinking about how you will organise your work in this stage, then look at the other activities in turn.

\chapter{Planning your work for Stage 1}\label{sect:stage1WorkPlan}\label{ch:PlanningYourWork}

A research project gives you an opportunity to carry out a focused piece of academic research in the subject area of your course of study and on a topic of your own choice.

\begin{question}[subtitle={Activity: Breaking down your study time}] Consider the timeline for your project that you defined as part of the activity in Cref{sect:5stageFramework}. Based on the number of weeks you allocated to Stage 1 in your timeline and the number of study hours per week at your disposal, use the percentages in Cref{tab:stage1} to calculate the suggested study time for each of its activities.
	
\end{question}

\section{Milestones, deliverables and tasks}\label{sect:Milestones}

\begin{question}[subtitle={Activity: Identifying your project tasks}] Consider the deliverables in Cref{tab:stage1} and write down how they correspond to tasks in your own project. Allocate time to those tasks based on the study time you estimated in the previous activity.
	
\end{question}

\section{A project plan for Stage 1}\label{sect:projectplan}

\subsection{Key practices for managing your time efficiently}\label{ssect:KeyPracticesFor} 

\chapter{Identifying the research problem}\label{ch:IdentifyingTheResearch}

It's a complex thing to do, but we'll walk you through it! You've got this!

\section{Choosing a topic for your project}\label{sect:stage1ChoosingATopic}
There is no single way to get started and inspiration can come from many places, from previous studies, any professional experience you may have, articles you have read, or even suggestions from your supervisor.

\begin{question}[subtitle={Activity: Possible topics to investigate}] Write down possible topics to investigate for your research project.

\begin{guidance}Think back\footnote{This might be your opportunity to find out more about them through research.} over your studies to this point. Did anything stick in your mind as very interesting, something you'd like to return to? What study topics did you particularly enjoy?

Next, think about what might be interesting within your industry. What are your industrial colleagues struggling with at the moment?

Next, look in the supporting materials that your course provides for your Masters, particularly those associated with possible Masters supervisors. It may be that you will find topic suggestions for Master projects.

Next, think about what has been happening in your discipline recently. Conduct a web search on recent topics that might interest you.

You should now have a long list of general possibilities for research.

Unless you already have something very specific in mind, you may start with a broad selection, then narrow down your choice to one or two candidate topics for further investigation.
\end{guidance}\end{question}
%%Hack to correct tcbox behaviour
\color{black}

Before investing lots of your time looking into a particular topic, however, you should assess whether it is appropriate for Masters research. Here are some things to consider before you make your final choice.

\subsection{Qualification fit}\label{ssect:QualificationFit}
If your project is the capstone of a taught course, you should ensure that the topic is appropriate to the course you are studying. In many instances, this relevance will be obvious -- you may well be exploring a topic you have encountered in your studies.

However, given the cross-disciplinary nature of many postgraduate programmes, it is equally possible that your topic will span boundaries between disciplines. In such a case, it is important for you to demonstrate that the emphasis of the research will be appropriate for the Masters qualification you are aiming for.

\begin{question}[subtitle={ACTIVITY: Considering qualification fit}] For each of the study options you identified in your last activity, write down how it relates to materials you have previously studied on your course. Identify those topics that are most suitable for your qualification. 

\begin{guidance}You should make a note of key ideas, theories, approaches or principles covered in those modules, which are particularly relevant to the topic you have chosen, and identify specific materials you may like to revise or apply in your project.
\end{guidance}\end{question}
%%Hack to correct tcbox behaviour
\color{black}

\subsection{Professional fit}\label{ssect:ProfessionalFit}
At Masters level, your research should be of interest to academics in the field, who are conducting related research, as witness, for instance, by their publications in the academic literature: some of your findings may well contribute to their future research.

In addition, your research may be of interest to other professionals in organisations\footnote{This may include both public and private enterprises -- local and national government and agencies -- as well as commercial organisations.} making up a particular industrial or economic sector that is within the scope of your research or experiencing similar problems or issues: some of your research results should be relevant and applicable to them.

You will likely need to argue the case for the professional fit and relevance of your research in your dissertation\footnote{And so it's really worthwhile thinking through professional fit very early on{\ldots}} or early/intermediate reports on your progress. This will be especially important if your primary investigation is exclusively within a single organisation, in which case it may not be obvious that the results are applicable elsewhere. It is not enough just to state that the relevance exists -- you must provide some evidence in the form of a logical argument or citation of reputable sources identifying a problem common across a range of organisations or an entire industry. Your review of the academic literature plays an important role: as we will see, a key purpose is to identify gaps in existing knowledge that your research is designed to fill. This alone may demonstrate clearly enough its relevance to the wider academic and professional community.

\begin{question}[subtitle={ACTIVITY: Professional fit }] Consider your chosen topic. Write down who may be interested/benefit from research in this topic and why. Indicate any evidence in support.

\begin{guidance} 
In addition to citing sources, ideally, you should talk to some of those people to explain what you intend to do and gain early feedback on the extent this may be of interest/benefit to them.
\end{guidance}\end{question}
%%Hack to correct tcbox behaviour
\color{black}

\subsection{Personal fit}\label{ssect:PersonalFit}
Through your Masters project, you will be living with your chosen research topic for many months\footnote{It may even feel longer:)}, so you \emph{must} choose a topic that will retain your interest and remain relevant over that period. 

A deep-seated interest in a question\footnote{Great research often arises from the researcher's passion for the topic.} will help you keep interested. However, even a passionate interest can only take you so far: you should also avoid topics in which you have little existing knowledge, for instance, or that will require you to learn and master major new skills\footnote{The timescale for a Masters research project is usually relatively short and you will be very busy without having to learn completely new subject areas or master completely new skills.}.

\begin{question}[subtitle={ACTIVITY: Personal fit }] Consider your chosen topic and write down the reasons why are you interested in it, and your current knowledge and skills in support.

\begin{guidance}
You should also identify any new knowledge and skills you may need to develop in order to research such topic, and indicate how you will develop them and in the time available.

\end{guidance}\end{question}
%%Hack to correct tcbox behaviour
\color{black}

\subsection{Organisational fit}\label{ssect:OrganisationalFit}
Your Masters course may or may not require that you associate your research with a specific organisation. Particularly in part-time studies, work-inspired research is a good way of getting value out of your research beyond your studies, and to gain support or even sponsorship from your employer.

In case of employer-sponsored projects, however, a note of caution is needed to avoid too narrow a focus for research. A common misconception is that the research is an opportunity to complete a piece of work for the sponsoring organisation. That should not be the case: the organisation may well benefit from your research, but the research itself must address a \emph{bona fide} research problem, i.e., one that has wider appeal and is beyond the needs of any single organisation. So, be careful that your research isn't linked too closely to the fortunes or objectives of a specific company or department.

If you are in the happy position of your employer offering sponsorship, make sure that you discuss with them the outcomes they expect from your research. Tell them about the broad focus that you will need to be successful and make sure they understand that that broad focus will not stop you contributing to their desired outcomes. If needs be, you could always suggest that the skills and attitudes that you will develop during the research and your findings will be available to your organisation, and that
the Intellectual Property\footnote{Intellectual property (IP) is property of things people create with their own intellect, such as an invention, an artwork, a design, etc. We will return to IP in Cref{sect:IntellectualProperty}.} that you create through your research may lead to:

\begin{itemize}
\item Solutions to their specific problems
\item New products and/or services that might generate a revenue stream that you can share.
\end{itemize}

If after this conversation, they remain unconvinced, then it may be better to choose a research topic that is not of immediate interest to the sponsor. It is a hard choice to give up sponsorship, but at least it will mean that you won't be over constrained in conducting your research.

\begin{question}[subtitle={ACTIVITY: Considering organisational fit }] If you're considering asking your employer to sponsor your Masters research project, or if they're already sponsoring your studies, write down how your chosen topic fits with their expectations, and how these compare to the requirements of your course.

\begin{guidance}You should list possible constraints from your sponsor which may prevent you from conducting your academic research, and how you intend to deal with those constraints.
\end{guidance}\end{question}
%%Hack to correct tcbox behaviour
\color{black}

\section{What is a research problem}\label{sect:stage1ResearchProblem}

Within your chosen topic, you will need to identify a specific research problem that your project is going to address. Your research problem narrows down your focus from a whole topic -- which might be quite broad -- to something very specific: a context, a knowledge gap and a justification.

A well-defined research problem is the foundation of any research project, clarifying the research purpose and intended outcomes, something all good academic research requires. As well as using it to drive much of the process of arriving at your dissertation, you'll also include a description of your research problem in your dissertation to tell a reader what you are trying to achieve.

All research problems you'll find in this handbook stem from the following research problem template:


\begin{quotation}
	In the context C, with phenomena of interest P, to address knowledge gap G. This matters to S because R.
\end{quotation}

An example of this could be:

\begin{quotation}
	Within `the food industry (C)', where the use of `plastic containers and wrappers (P)' is widespread, to address `how to reduce their use (G).'  This is important to `the food industry and to society (S)' because `10\% of all costs can be attributed to the use of plastic, and plastic is highly polluting (R).'
\end{quotation}


As written, this research problem may appear a little contrived, but this is only the starting point to identify its important components, that is:

\begin{itemize}
\item C: in which Context will the research take place?

\item P: which are the Phenomena of interest?

\item G: what is the knowledge Gap?

\item S: to which Stakeholders does this matter{\ldots}

\item R: {\ldots}and for which Reasons?

\end{itemize}

Let's unpack these components.

\subsection{The context and phenomena of interest}\label{ssect:ContextAndPhenomena}
The context in which the research will take place constrains the project scope, alongside the phenomena that will be considered, what is of interest about them, what they influence, how they are measured, how they are observed, etc. Different contexts and phenomena therein will typically lead the researcher in different directions: for instance, using wifi in a home will lead to different considerations of its behaviour that using wifi in a hospital, where it could interfere with delicate medical devices.

As we shall see, contexts can be embedded in other contexts or can overlap each other, so that, it's important to identify precisely the context of the research project.

\begin{example}{Anna's research}
Anna works in the portfolio management office (PMO) of a large UK university. The office role is to provide support for pan-university projects, particularly in helping assessing project risk, and produce estimates and forecasting of required resources. Anna has observed that her university's projects often overrun and overspend their budget, and wants to conduct some research on whether project estimates and forecasting can be made more accurate.
\end{example}

\begin{question}[subtitle={Activity: Anna's context}] Identify the context for Anna's research problem.

\begin{solution} Anna's problem is embedded in many contexts, including:

\begin{itemize}
\item her PMO

\item her univerisy

\item the UK higher education sector

\item the UK as a country.

\end{itemize}

If the scope was the UK, then there would be crazily many phenomena that Anna could think about: choosing the UK is just too large. The UK higher eduction sector is beginning to be a more realistic scope, but there are other choices that make even more sense. She aims to investigate project estimates and forecasting in her university, so that her PMO is the most suitable context for her research, particularly if it shares similarities with other PMOs across the sector.

\end{solution}\end{question}
%%Hack to correct tcbox behaviour
\color{black}

Let's consider phenomena next. Technically, a phenomenon is an element of the world, the occurrences of which are observable. They can be a material thing, transient, like an event or a situation, or information-based, like a fact or a concept; like we said, anything you can observe.

Some phenomena arise naturally and others artificially: many proteins are phenomena that occurs naturally through protein synthesis, whereas the technologies underpinning social media  are constructed by software engineers. If you look around the room in which you're sitting, every object you perceive -- each light, each fly, your computer and its (virtual) files -- every event that occurs -- that light turning on, the creak of your chair, a bird flying into your window --  is a phenomenon.

Phenomena can be really simple --- a speck of dust, or very complex --- the whole sequence\footnote{Although without specialist equipment, you might not be able to observe it} of steps that your computer goes through to connect to your wifi is a phenomenon. Complex phenomena are usually made of various observable elements, each a phenomenon too.

Phenomena are a very rich source for research problems. They have many and various characteristics, not all of which may be fully known: these call for more research. There are also phenomena which haven't even been identified yet! At one point, although we knew how to send data over radio signals, wifi didn't exist and so the devices and protocols associated with it had to be invented. New phenomena are being created and/or discovered all the time. That means new knowledge is needed and so research. The existence of phenomena in one field might also suggest their existence in another -- more research needed.

%\footnote{We don't claim that all research problems relate to phenomena, but suspect that all the good ones do.}

In summary, focusing on phenomena to characterise research problems means there's a very rich seam for researchers to mine.

\begin{question}[subtitle={Activity: Anna's phenomena of interest}] Identify phenomena of interest in Anna's research context.

\begin{solution} From the description above, \emph{project estimates and forecasting}, their \emph{accuracy}, and project \emph{overrun} and \emph{overspend} are all phenomena of interest. Also, forecasting is a complex phenomenon, possibly including a process, some techniques and some measurements. As part of Anna's research, each of its constituent parts (other phenomena!) may need investigating, alongside its relation to the other phenomena of interest.
\end{solution}\end{question}
%%Hack to correct tcbox behaviour
\color{black}

\subsection{The knowledge gap}\label{ssect:KnowledgeGap}
You know by now that academic research aims to generate new knowledge in a field of study. It is therefore essential that your research problem captures the knowledge gap your research intends to address. To help you start thinking about the knowledge gap, let's go back to Anna's scenario once more.

\begin{question}[subtitle={Activity: Anna's candidate knowledge gap}] Given the context and phenomena identified in the previous activities, write down a possible knowledge gap that Anna might address with her research.

\begin{guidance}
If you're having difficulty finding the gap, focus on what Anna intends to do, that is to investigate project estimates and forecasts made by her PMO, which often do not correspond to what happens in projects.

\begin{solution}A possible knowledge gap we've come up with is:

\enquote{How to improve the accuracy of project estimates and forecasting made by Anna's PMO.}

Yours is probably similar. If it isn't, look again at the example initial description and focus on what Anna has been asked to do.
\end{solution}
\end{guidance}\end{question}
%%Hack to correct tcbox behaviour
\color{black}

It is important to know that this is only a \emph{candidate knowledge gap}: what we don't know at this point is whether Anna's PMO is just bad at estimates and forecasting, possibly not employing appropriate processes and techniques, or whether there is effectively a knowledge gap in the sector in that accurate techniques are unknown. To be able to judge, Anna will need to do more work, including looking at the literature, but we are not there quite yet.

\subsection{The justification}\label{ssect:Justification}
Some research is motivated by the researcher' pure curiosity and desire to advance knowledge in a field of study, which is good justification to conduct research!

However, most research, particularly when located in a real-world context, matters to other people too, usually referred to as the stakeholders or beneficiaries\footnote{We will use the term beneficiary when they benefit from the research; a stakeholder means they will be affected by the research, but not necessarily positively.} (these include the researcher, of course!) and may have measurable real-world impact.

Measuring\footnote{Universities these days are assessed on their real world impact, so it is possible that your project will have some form of impact assessment in it.} the real-world benefit delivered through research is an easy way of showing that it has value. Having beneficiaries also means that you can more easily conclude that the research problem has been solved in context -- as you can ask the beneficiaries. Asking why the research is important to them is another good thing to know as you can then use their criteria to assess the value of your research.

\begin{question}[subtitle={Activity: Which stakeholders will benefit and why?}] In Anna's case, identify who may have an interest in her research problem and why. Also consider who may be able to judge whether Anna's research has addressed the problem.

\begin{solution} Anna's research should lead to more accurate project estimates and forecasting, which would save her university money and possibly increase project success. So the university as a whole could benefit form her research.

Her colleagues in the PMO should be able to help her work out why their current process is less effective than it could be, and assess the extent findings from her research could make a difference.

\end{solution}\end{question}
%%Hack to correct tcbox behaviour
\color{black}

It is very important to note, however, that it isn't the beneficiaries of the research that determine whether the research is actually research; that's a judgement based on the fact that new knowledge has been generated and this can only be judged by the larger research community.

For a Masters research project, that larger research community might be represented by a very small group of people: your supervisor and your dissertation examiners. In exceptional cases,\footnote{Many of our research students have done this.} a Masters student might go on to report their research findings to a larger community of scholars at an academic conference, for instance, or even through a scientific journal: conference or journal publication is the pinnacle of validation that knowledge is new.

\subsection{Problem formulation}\label{ssect:ProblemFormulation}

It is now time to write down Anna's problem, by using the template we have provided, and based on what we have found out so far, which is summarised in Cref{tab:Annaproblemelements}.

\begin{SimpleNColTable}{tab:Annaproblemelements}{2}[\the\textwidth]{Elements of Anna's problem}
Phenomena P & Forecasting of time and resources and its accuracy \\ 
Context C & Anna's PMO\\
Phenomena P & project estimates and forecasting, and their accuracy \\
Reasons R & Inaccurate forecasting increases production cost and reduces profit margins and competitiveness \\
Stakeholders S & Anna's university \\
Reasons R & Inaccurate estimates and forecasting lead to financial loss related to project overrun and overspend, and decrease the chance of project success \\
\end{SimpleNColTable}

\begin{question}[subtitle={Activity: Anna's initial problem}] Based on the information in Cref{tab:Annaproblemelements} and using the problem template, write down Clara's initial research problem.

\begin{solution}This is what we have come up with:

\blockquote{How to improve the accuracy of project estimates and forecasting within Anna's PMO. This matters to Anna's university because inaccurate project estimates and forecasting lead to financial loss related to project overrun and overspend, and decrease the chance of project success.}

You may have written something similar.

\end{solution}\end{question}
%%Hack to correct tcbox behaviour
\color{black}

As stated, this problem formulation is only the starting point, as Anna has yet to establish that a knowledge gap actually exists. To do so, she will need to look into the literature and revise her problem statement accordingly. This is true in general: your initial problem will help you scope your review of the literature, which in turn will inform which knowledge gaps exist, and will help you reformulate your research problem accordingly.

In the next section we will look in detail at how you can do that, but first let's look more closely at diverse types of research problem.

\section{Types of research problems}\label{sect:problemTypes}

Research problems originating from our template depend critically on phenomena. And it won't come as any surprise then, that you can classify the types of research problem by the things you can do with those phenomena. And that leads to the different types of research that you can do, which we consider in this section.

\subsection{Descriptive problems}\label{ssect:DescriptiveProblems}
Descriptive problems aim to describe phenomena of which we have little knowledge, accurately and systematically. The goal is to describe phenomena for the first time or to render existing descriptions more detailed or accurate.

\begin{example}{Examples of descriptive problems}
\begin{enumerate}
	\item To characterise the present state of the European commercial synthetic biology ecosystem. This matters to both producers and consumers in that ecosystem because this knowledge can be used to inform effective routes to commercialisation.
	\item To determine how UK organisations are using the ITIL framework to manage cloud-based IT services. This matters to IT managers within those organisations because it can help them improve their service management practices.
\end{enumerate}
\end{example}

\begin{question}[subtitle={Activity - Descriptive problems}] For each example above, identify its constituent parts with reference to the research problem template we have provided. Write down any similarity between the problems.

\begin{solution}
Here is our mapping of the problem statements onto the template:

%Table in Latex
\begin{SimpleNColTable}{tab:descriptiveProblemsA}{2}[\the\textwidth]{Description Problems  – Example 1}
Context C & \\
Phenomena P   & The current state of the ecosystem \\
Knowledge gap G  & A characterisation of the present state the ecosystem \\                         
Stakeholders S  & Consumers and producers within the ecosystem  \\
Reasons R & To inform effective routes to commercialisation\\
\end{SimpleNColTable}

\begin{SimpleNColTable}{tab:descriptiveProblemsB}{2}[\the\textwidth]{Description Problems  – Example 2}[X[1]X[2]]
Context C & UK organisations\\
Phenomena P   & Those organisations' management of cloud-based IT services \\
Knowledge gap G  & How they use the ITIL framework to manage cloud-based IT services \\                         
Stakeholders S  & Those organisations' managers  \\
Reasons R & To improve service management\\
\end{SimpleNColTable}

In both cases, the goal is to provide a description where one is currently lacking: of a state-of-the-art in the first problem, and of the use of a framework in the second problem.
\end{solution}\end{question}
%%Hack to correct tcbox behaviour
\color{black}

\subsection{Exploratory problems}\label{ssect:ExploratoryProblems}
Exploratory problems aim to investigate phenomena of which little is known. The goal is typically to generate new ideas, hypotheses, theories, models or predictions which can be investigated in further research.

Lots of academic research is exploratory as it strives to develop a deep understanding of natural or social phenomena. For instance, physicists try to explain the working of the natural world by making empirical observations of natural phenomena and conjecturing cause-and-effect relations, usually expressed as mathematical formulae or theories. This is the way natural sciences have developed over the centuries. This is also true of social scientists conducting observations in social settings in order to develop new theories of human behaviour.

\begin{example}{Examples of exploratory problems}
\begin{enumerate}
	\item Within public service providers, to investigate the relation between demographic and behavioural factors and end-users' awareness of cyber security threats. This matters to those providers as successful cyber security attacks can result in loss of confidential information. This also matters to their customers as such information may include customers' data.
	\item To investigate the possible use of social media in the management of a natural disaster by government organisations. Given the spread of social media, this matters to those organisations, which could integrate them within their critical communication infrastructures.
\end{enumerate}
\end{example}

\begin{question}[subtitle={Activity - Exploratory problems}] For each example above, identify its constituent parts with reference to the research problem template we have provided. Write down any similarity between the problems.

\begin{solution}
Here is our mapping of the problem statements onto the template:

\begin{SimpleNColTable}{tab:exploratoryProblemsA}{2}[\the\textwidth]{Exploratory Problems – Example 1}
Context C & Public service providers\\
Phenomena P   & End-users' demographic and behavioural factors; cyber security threats \\
Knowledge gap G  & To investigate the relation between demographic and behavioural factors and end-users' awareness of cyber security threats \\                         
Stakeholders S  & The providers and their customers \\
Reasons R & Successful cyber security attacks can result in loss of confidential information \\
\end{SimpleNColTable}

\begin{SimpleNColTable}{tab:exploratoryProblemsB}{2}[\the\textwidth]{Exploratory Problems – Example 1}
Context C & Government organisations dealing with natural disasters \\
Phenomena P   & Social media; the management of natural disasters \\
Knowledge gap G  & To investigate the possible use of social media in the management of a natural disaster \\                         
Stakeholders S  & Those government organisations  \\
Reasons R & Social media could be integrated within critical communication infrastructures\\
\end{SimpleNColTable}

In both cases, the goal is to explore a situation for which little understanding is currently available (this being the knowledge gap): in the first problem, the focus are possible effects of demographic and behavioural factors on awareness of cyber security threats; in the second, whether social media could be used effectively for communication in the management of a natural disaster.
\end{solution}\end{question}
%%Hack to correct tcbox behaviour
\color{black}

\subsection{Explanatory Problems}\label{ssect:ExplanatoryProblems}
Explanatory problems aim to explain why certain phenomena occur or are related. The goal is often to test a conjecture, hypothesis, theory or model.

\begin{example}{Example of explanatory problems}
\begin{enumerate}
	\item In the context of software development companies, to investigate how learning strategies adopted by software engineers allow them to develop new skills fast and efficiently. These matters to those companies as software technology changes rapidly, so that software engineers need upskilling and retraining very frequently.
	\item Within the public sector, to investigate how practices adopted by organisations and employees lead to an effective and efficient use of teleconferencing technology. This matters to public sector organisations due to the recent significant increase in flexible and home working, so that much of their business is now conducted online.
\end{enumerate}
\end{example}

\begin{question}[subtitle={Activity - Explanatory problems}] For each example above, identify its constituent parts with reference to the research problem template we have provided. Write down any similarity between the problems.

\begin{solution}
This is our mapping of the problem statements onto the template:

\begin{SimpleNColTable}{tab:explanatoryProblemsA}{2}[\the\textwidth]{Explanatory Problems – Example 1}
Context C & The public sector  \\
Phenomena P   & Use of teleconferencing technology\\
Knowledge gap G  &  How practices adopted by organisations and employees lead to an effective and efficient use of teleconferencing technology\\                         
Stakeholders S  &  Public sector organisations\\
Reasons R & More and more business is conducted online\\
\end{SimpleNColTable}

\begin{SimpleNColTable}{tab:explanatoryProblemsB}{2}[\the\textwidth]{Explanatory Problems – Example 2}
Context C & Software development companies \\
Phenomena P & Learning strategies; skills development   \\
Knowledge gap G & How learning strategies adopted by software engineers allow them to develop new skills fast and efficiently  \\                         
Stakeholders S  &  Software development companies and software engineer\\
Reasons R & Software technology changes rapidly, so that software engineers need upskilling and retraining very frequently\\
\end{SimpleNColTable}

In both cases, the goal is to explore how and why certain phenomena are related. In the first problem, the focus is why certain learning strategies lead to more effective upskilling; in the second problem, why certain practices lead to a more efficient use of teleconferencing technology.
\end{solution}\end{question}
%%Hack to correct tcbox behaviour
\color{black}

\subsection{Predictive Problems}\label{ssect:PredictiveProblems}
Predictive problems aim to predict future phenomena. The goal is often to define a model, extrapolating from current knowledge, that allows predictions to be made together with an assessment of how accurate they might be.

\begin{example}{Examples of predictive problems}
\begin{enumerate}
	\item To quantify the potential impact on fresh water consumption of recycling domestic bathroom water in the UK within the next decade. This is important both from an economical and an environmental perspective, as fresh water is becoming a scarce resource, so its preservation is imperative.
	\item To use demographic and curriculum data to predict which current students are at risk of dropping out in the context of higher education. This matters to higher education institutions as retaining their students is their statutory duty, and because students not completing their course of study results in fewer skilled people entering the job market, which may impact productivity.
\end{enumerate}
\end{example}

\begin{question}[subtitle={Activity - Predictive problems}] For each example above, identify its constituent parts with reference to the research problem template we have provided. Write down any similarity between the problems.

\begin{solution}
Here is our mapping of the problem statements onto the template:

\begin{SimpleNColTable}{tab:PredictiveProblemsA}{2}[\the\textwidth]{Predictive Problems – Example 1}
Context C & UK in the next decade \\
Phenomena P   & Fresh water consumption; recycling domestic bathroom water \\
Knowledge gap G  & To quantify the potential impact on fresh water consumption of recycling domestic bathroom water \\                         
Stakeholders S  &  Water management agencies; water suppliers and consumers\\
Reasons R & Preserving fresh water is important as it is becoming a scarce resource\\
\end{SimpleNColTable}

\begin{SimpleNColTable}{tab:PredictiveProblemsB}{2}[\the\textwidth]{Predictive Problems – Example 1}
Context C &  Higher Education \\
Phenomena P   & Student drop-outs; Demographic and curriculum data\\
Knowledge gap G  & To predict which current students are at risk of dropping out \\                         
Stakeholders S  & Universities; their students  \\
Reasons R & Student retention is a statutory duty; retentions means that enough skilled people enter the job market\\
\end{SimpleNColTable}

In both cases, the goal is to make predictions. In the first problem, on the impact of recycling water on fresh water availability; in the second problem, on the likelihood of students dropping out.
\end{solution}
\end{question}
%%Hack to correct tcbox behaviour
\color{black}

\subsection{Evaluative Problems }\label{ssect:EvaluativeProblems}
Evaluative problems aim to establish whether phenomena that have been introduced have achieved the desired outcome. These might be concepts, theories, products, technology, etc., and the goal is to establish how they have performed.

Evaluative problems may focus on testing the strength of an academic theory or model to ensure its long-lasting validity and scope of applicability. They may place an existing theory or model in a new context, or apply them to new situations, evaluating the effects of a change of conditions, what works, what does not and why, possibly leading to new theories or improved models.

Evaluative research may also take the form of a well-founded critique of diverse explanatory theories or potentially conflicting evidence put forward by previous researchers. A systematic and critical analysis of such evidence may lead to new knowledge on which theories are most reliable, so that future researchers may focus on those.

\begin{example}{Examples of evaluative problems}
\begin{enumerate}
	\item To evaluate the effectiveness of Artificial Intelligence (AI) to reduce the occurrence and impact of successful cyber-security attacks within the financial sector. This matters to the sector because over 3 billion dollars are lost within the sector every year due to successful attacks, and efforts to prevent them are also very costly.
	\item Within non-clinical laboratories, to determine if lean techniques can improve process flow in the presence of unpredictable demand and supply. Inefficiencies in process flow are costly to non-clinical laboratories, so that improving it is beneficial to them and their customers.
\end{enumerate}
\end{example}

\begin{question}[subtitle={Activity - Evaluative problems}] For each example above, identify its constituent parts with reference to the research problem template we have provided. Write down any similarity between the problems.

\begin{solution}
This is our mapping of the problem statements onto the template:

\begin{SimpleNColTable}{tab:EvaluativeProblemsA}{2}[\the\textwidth]{Evaluative Problems – Example 1}
Context C &  The financial sector\\
Phenomena P & AI techniques and cyber security attacks   \\
Knowledge gap G  & To evaluate the effectiveness of AI in reducing successful cyber-security attacks \\                         
Stakeholders S  & Operators in the sector and their customers \\
Reasons R & Significant financial losses are due to such attacks every year, and the cost of preventing them is high\\
\end{SimpleNColTable}

\begin{SimpleNColTable}{tab:EvaluativeProblemsB}{2}[\the\textwidth]{Evaluative Problems – Example 2}
Context C & Non clinical laboratories \\
Phenomena P   & Lean techniques; process flow; demand and supply\\
Knowledge gap G  &  To determine if lean techniques can improve process flow in the presence of unpredictable demand and supply \\                         
Stakeholders S  &  Those laboratories and their customers\\
Reasons R & Inefficiencies in process flow are costly\\
\end{SimpleNColTable}

In both cases, the goal is to evaluate how approaches applied within a certain context perform. In the first problem, the focus is on the performance of AI to counter cyber security attacks; in the second problem, the focus is on the effect of lean techniques on the flow of processes with unpredictable demand and supply.
\end{solution}\end{question}
%%Hack to correct tcbox behaviour
\color{black}

\subsection{Design problems}\label{ssect:DesignProblems}
In design problems the focus is on phenomena that take the form of new artefacts: the aim is to create artefacts which embed new knowledge. The term `artefact' is meant in its widest meaning of something created by people, and includes tangible and intangible products, processes, novel combinations of ideas or technologies, etc. The goal is to embed within the artefact new knowledge which extends human and/or organisational capabilities, including improved ways of doing something.

\begin{example}{Examples of design problems}
\begin{enumerate}
	\item In the context of data visualisation tools, to devise algorithms able to generate automatically text summaries of line graphs depicting multiple time series for the benefit of sight-impaired users. This matters to both tool providers and sight-impaired users as it would improve the accessibility of such tools.
	\item To define a hybrid project management framework, based on complexity and volatility characteristics of a project, to reduce failure in the development of embedded software systems. Current project failure across the sector is very frequent, resulting in a high financial burden.
\end{enumerate}
\end{example}

\begin{question}[subtitle={Activity - Design problems}] For each example above, identify its constituent parts with reference to the research problem template we have provided. Write down any similarity between the problems.

\begin{solution}
Here is our mapping of the problem statements onto the template:

\begin{SimpleNColTable}{tab:DesignProblemsA}{2}[\the\textwidth]{Design Problems – Example 1}
Context C & Data visualisation tool development industry \\
Phenomena P   & Text summaries; multiple time series line graphs; algorithms\\
Knowledge gap G  & To devise algorithms able to generate automatically text summaries of line graphs depicting multiple time series \\                         
Stakeholders S  & Tool providers and sight-impaired end-users \\
Reasons R & To improve the accessibility of such tools\\
\end{SimpleNColTable}

\begin{SimpleNColTable}{tab:DesignProblemsB}{2}[\the\textwidth]{Design Problems – Example 1}
Context C &  Embedded software systems development sector\\
Phenomena P   & Complexity and volatility characteristics; project failure; project management\\
Knowledge gap G  &  To define a hybrid project management framework, based on complexity and volatility characteristics, to reduce project failure\\                         
Stakeholders S  &  Project managers in the sector\\
Reasons R & Project failure rate is high, resulting in a high financial burden for the sector\\
\end{SimpleNColTable}

In both cases, the goal is to design something new. In the first problem, new algorithms for the automated generation of text summaries; in the second problem, a framework for hybrid project management.
\end{solution}\end{question}
%%Hack to correct tcbox behaviour
\color{black}

Note that an essential characteristic of a design research problem is that the to-be-designed artefact is a genuine contribution to knowledge in that it augments what is currently possible. Some of the most exciting research in technology-related subjects is about developing new artefacts. However, for such development to fit academic research it is important to establish what the contribution to knowledge is, and assess its originality and significance. In particular, it is important to distinguish the very first development of an innovative artefact from the routine of implementation of known systems: the former augments knowledge; the latter does not.

\section{Masters-appropriate research problems}\label{sect:Masters-appropriate}

Regardless of the type of problem you intend to address with your research, you will only have a limited amount of time to complete your project and write your dissertation, often no more than one year. It is therefore important that you think carefully about your research problem from the start as changing direction later on in your project could be very challenging and increase your risk of not completing it successfully.

In particular, you should focus on the following criteria:

\paragraph{Generality} Your chosen research problem should not be so limited in scope as to be irrelevant or of no interest to others. While your inspiration may well be something you have observed directly in your personal or professional life, your research problem should address a knowledge gap which is shared within your field of study, is of academic relevance and possibly of interest to professional practice, so that your findings can be generalised to some extent. Reviewing the literature is a way to ensure that what you have in mind meets these requirements.

\paragraph{Complexity } You will only have limited resources for your project, so your problem should not be so complex that it cannot be addressed within those constraints. Some research problems are just too ambitious for Masters research, e.g., \enquote{To reduce social inequality in the UK.} However, there may be potentially suitable sub-problems you could consider, e.g., \enquote{To understand the role of food banks in alleviating poverty in the UK in the last decade.}

\paragraph{Volatility} Your chosen problem should remain current for at least the duration of your project and, ideally, present further opportunities for research afterwards. It is therefore important that you choose a problem which does not loose its relevance too quickly. This can happen in technology-oriented projects that focus on specific products or tools, and their features. It could be avoided by considering if a more general problem can be found of which the original problem is an instance. For example, instead of \enquote{how to visualise data effectively in tool X}, the problem could be generalised to \enquote{which design principles apply for effective data visualisation,} regardless of the specific tool used.


\begin{running}{Assessing Anna's problem} We left our example with Anna's problem expressed as follows:

\emph{How to improve the accuracy of project estimates and forecasting within Anna's PMO. This matters to Anna's university because inaccurate project estimates and forecasting lead to financial loss related to project overrun and overspend, and decrease the chance of project success.}

Let's apply the criteria above to work out whether this is a suitable research problem or more work is needed.

In terms of generality, this problem is too specific to Anna's own university, so may not be indicative of a knowledge gap --- for instance, it is possible that Anna's PMO applies outdated estimation and forecasting approaches, and better approaches are known and applied elsewhere. Therefore, for Anna, the next step will be to do some initial reading of the academic literature to establish:

\begin{itemize}
\item What, if anything, is already known about this problem in its wider context, e.g., other universities in the sector, or even organisations in other sectors\footnote{After all, all large organisations have projects, and possibly a PMO to support them.}
\item Which specific aspects of this problem, if any, could be investigated to make a novel contribution to knowledge.
\end{itemize}

In terms of complexity, the problem appears tightly focusses and doable as part of a Masters; similarly, it is unlikely the problem will cease to be relevant in the time span of the research, so neither complexity nor volatility are of concern.
\end{running}

\begin{question}[subtitle={Activity: Appropriateness of research problem statements}] Consider the following research problem statements and discuss the extent you think they are appropriate for Masters research based on the above criteria.
\begin{enumerate}
\item To analyse possible differences in people's web search strategies to inform the design of search algorithms in software. Some individuals are better at conducting online searches than others and this knowledge may lead to more efficient algorithms.
\item To establish which information should be displayed to end users to help them identify energy waste within their heating systems. Energy waste impacts negatively household finances and the environment, so raising end-users' awareness could help reduce waste.
\item To improve coding skills of students at a distance. Coding skills are in demand and there is shortage in job market.
\end{enumerate}

\begin{solution}This is our assessment of each of them. Yours may be different, of course.

\begin{enumerate}
\item Web search algorithms are currently an active field of study, so it is likely that a contribution to knowledge can be made which is of wide interest to both academia and industry, and likely to remain so for the foreseeable future. Some thought will be needed to ensure that the work can be conducted within the constraints of the project, for instance by limiting the number of participants in the study or the length of observations of their search strategies. Therefore, overall, with appropriate adjustments, this could be suitable for Masters research.
\item This problem is both topical and specific, so it has a good chance to meet the criteria. Some work on contextualising the problem in the literature will be necessary to establish the extent of the potential contribution to knowledge, particularly if design guidelines already exist, in which case it would be important to establish how these are lacking. Overall, this too could lead to a suitable Masters project.
\item This problem is too broad and open-ended for a Masters project --- it is a topic, rather that a specific research problem. A more suitable problem could be defined, for instance, by identifying which specific aspects of improving coding skills at a distance are problematic with reference to the literature, so that a much narrower knowledge gap can be established.
\end{enumerate}
\end{solution}\end{question}
%%Hack to correct tcbox behaviour
\color{black}

\section{Formulating your initial research problem}\label{sect:Formulating}
It is time for you to have a go at formulating your initial research problem, based on what you have learnt in this section.

\begin{question}[subtitle={Activity: Your initial research problem }] Apply the template provided in Cref{sect:stage1ResearchProblem} to write down your initial research problem. Discuss what type of problem it is and assess it in terms of generality, complexity and volatility.

\begin{solution} Make sure you include all the elements of the template, i.e., C, P, G, S and R, and explain how your problem aligns with one of the types in Cref{sect:problemTypes}. Argue concisely why your problem is appropriate for Masters research in terms of generality, complexity and volatility, or indicate further steps you will take to ensure this is the case.

\end{solution}\end{question}
%%Hack to correct tcbox behaviour
\color{black}

\todo{Resolve duplicates}
\chapter{Research aim and objectives, and project title}\label{sect:stage1AimAndObjectives}

\section{Setting your aim}\label{ch:SettingResearchAim}

While your research problem highlights the knowledge gap you intend to address, your research aim should express the specific way that gap will be addressed by your project. The distinction is subtle but important: your aim should provide a clear indication of the focus of your work and the particular form the new knowledge generated will take. That you can provide such a description means that you have made some great progress!

\begin{example}{Anna's project aim} Let's go back to Anna's research problem:

\begin{quotation}
How to improve the accuracy of project estimates and forecasting within Anna's PMO. This matters to Anna's university because inaccurate project estimates and forecasting lead to financial loss related to project overrun and overspend, and decrease the chance of project success.
\end{quotation}	

From this, we need to formulate an aim which provides a more specific indication of the way Anna intends to address the knowledge gap. One such aim may be:

\begin{quotation}
To identify techniques which can produce more accurate project estimates and forecasting within Anna's PMO.
\end{quotation}	

which indicates that Anna's research intends to focus on specific techniques to apply in her own context.

Of course, there may be other ways to address the research gap. For instance, it may be that rather than investigating new techniques, Anna could investigate the skills and competencies of the PMO staff, in which case the aim could be expressed as:

\begin{quotation}
To identify staff's skills and competencies needed to produce more accurate project estimates and forecasting within Anna's PMO.
\end{quotation}	
\end{example}

As per the research problem formulation, your  project aim will also be tentative at this stage in your project: you will need to revise it, alongside your research problem, as your understanding grows through reading the literature and generally thinking critically about your research. Nevertheless, it is important for you to think about your intended knowledge contribution from the start, and expressing your research aim is a way to do so.

\begin{question}[subtitle={Activity: Drafting your research aim }] Consider your current problem statement. Draft a possible related research aim.

\begin{guidance}
Your research aim formulation doesn't need to be perfect -- or even right -- at this stage, and will evolve with your research problem as you progress through your project. 
\end{guidance}\end{question}
%%Hack to correct tcbox behaviour
\color{black}

\section{Articulating your research objectives}\label{sect:Articulating}

Research objectives help you break down your aim into smaller targets that, together, achieve the aim. 

Each objective should be expressed in a way which allows you to tell if you have achieved it by the end of your project. They are generally expressed using action verbs, like to identify, design, compare, explain, describe, evaluate, ..., which point to activities the effect of which can be observed and somehow assessed.
Objectives should also be achievable within your project constraints, that is with the knowledge, skills, time and resources you have or can acquire within your project.
As stepping stones towards reaching your aim, your research objectives should be sequenced in a way that meeting early objectives should enable subsequent ones. 

While more specific than your aim, your research objectives can still be relatively broad at this point, in the sense that they are not meant to enumerate every little thing you are going to do. Usually 3 to 5 research objectives are sufficient: later on you can break them down into several specific tasks, but we are not there yet. 

Let's go back to our example.


\begin{example}{Anna's project objectives} We left Anna's project aim stated as:

\begin{quotation}
To identify techniques which can produce more accurate project estimates and forecasting within Anna's PMO. 
\end{quotation}	

from which we could derive the following research objectives:

\begin{itemize}
	\item Objective 1: To \emph{identify} a set of techniques that can be used to make project estimates and forecasting
	\item Objective 2: To \emph{apply} those techniques in the context of Anna's university to a number of projects
	\item Objective 3: To \emph{evaluate} their accuracy in relation to those project outcomes, particularly cost and duration, and in comparison with current practice
\end{itemize}

Objective 1 requires the identification of specific techniques that could be used in the project. An appropriate selection should take into consideration whether their use on the project is feasible in relation to available knowledge, time, resources.

Objective 2 concerns the application of the techniques selected in context of the target UK university. Feasibility may relate to the kind of data required and their availability with the target university.

Objective 3 concerns the evaluation of the techniques applied in relation to known project outcomes and current practice: this will allow Anna to decide if improvements have been made. Feasibility may concern the availability of such data, say from historic or ongoing projects. 

Note how those objectives are expressed using action verbs and build on each other -- if successfully completed, they will allow Anna to meet her overall research aim.
\end{example}

In this example, we have used an \emph{identify-apply-evaluate} pattern to break down Anna's aim into objectives. This is a useful patten, particularly when you intend to apply and evaluate something -- a theory, a product, a new technology -- within a real-world context, something very common in Masters research projects. If your research is more formal, mathematical, clinical, or of some other form, other patterns may apply. For instance, a mathematical dissertation will require you \textit{to prove} formally that a particular theorem holds, while a clinical trial will require you \textit{to test} some statistical hypothesis\footnote{We haven't discussed hypotheses  yet. In common language a hypothesis is a supposition for which supporting evidence is yet to be found, so it may be the starting point of an investigation. Statistical hypotheses are particularly types of hypothesis that can be tackled with statistical techniques. We will return to this in Stage 4.}

Even if you apply a pattern, identifying the best form for your research objectives will require some creativity on your part, as there is no magic formula to do so.  In all cases, you should discuss how to break down your aim into objectives with your supervisor.

At this point in your project your objectives are still speculative, expressing the intention of your research: they are likely to change during your project, so that you will need to review them at each project stage.

\begin{question}[subtitle={Activity: Articulating your research objectives}] Consider your current research aim. Write down 3 to 5 possible objectives, explaining how they relate to each other and how they contribute, if successfully completed, to meet your research aim.

Also comment on how you can assess if they are met, and how feasible they are in relation to your own knowledge, skills and resources, and your project time span.

\begin{guidance}Your aim and objectives don't need to be perfect at this stage and will evolve with your research problem as you progress through your project. However, it is important for you to think about what concrete contribution your research will make: expressing aim and objectives is a way to do this.

\end{guidance}\end{question}
%%Hack to correct tcbox behaviour
\color{black}

\section{Choosing a title}\label{sect:title}\label{sect:ChoosingATitle}

With your chosen research problem and aim you should now have a good idea of what you are hoping your research will focus on -- the need you will address in context -- and deliver -- the intended knowledge outcome. From this you can choose a representative title for your project\footnote{The title is the first contribution you will have made to your dissertation! That first page is no longer blank. Congrats, you're on your way!}.

The title provides the first indication to your reader of what you propose to research. It may change as the research progresses, so it is important to review it from time to time to check its current relevance. At this stage, the title can only be your best attempt at anticipating later developments in your research, so don't agonise too much over it.

Here are some guidelines for you to follow:

\begin{itemize}
\item A good title should succinctly convey elements of both research problem and aim, specifically, the focus and intended outcome of your project

\item A good title is around 8--20 words. For instance\footnote{An experienced supervisor may have other examples too.}:

\begin{itemize}
\item Integrated process improvement strategies in small/medium-sized manufacturing enterprises in the UK fabricated metals industry

\item Cost-effective greenhouse gas mitigation measures for the UK livestock industry: a risk assessment of the impact on water footprints
\end{itemize}

\item Avoid titles that are very short and enigmatic, or titles that are long and rambling

\item Do not include acronyms or obscure technical terms, except those which are likely to be widely understood, say, WWW or UK.

\item Do not pose a question in your title

\item A title is not a sentence, so it does not require a full stop at the end.

\end{itemize}

\section{Articulating your research objectives}\label{sect:ArticulatingYourResearch}
\todo{Two sections named thus}

\begin{example}{Anna's tentative project title} Let's go back to Anna's research problem:

\begin{quotation}
How to improve the accuracy of project estimates and forecasting within Anna's PMO. This matters to Anna's university because inaccurate project estimates and forecasting lead to financial loss related to project overrun and overspend, and decrease the chance of project success.
\end{quotation}	

and research aim:

\begin{quotation}
To identify techniques which can produce more accurate project estimates and forecasting within Anna's PMO.
\end{quotation}	

A tentative title for her project might be:

\begin{quotation}
Effective techniques for accurate project estimates and forecasting within a UK university
\end{quotation}	
\end{example}

\begin{question}[subtitle={Activity: Choosing a title }] Use the guidelines above together with your current research problem and aim, to write down an appropriate title for your research project.

\begin{guidance}
You will have many opportunities to refine your title, so at this point, spend no more than 20 minutes on this activity.
\end{guidance}\end{question}
%%Hack to correct tcbox behaviour
\color{black}


\chapter{Reviewing the literature}\label{sect:litrev}\label{ch:ReviewingTheLiterature}
In this section, you will start your literature review, an activity you will continue in Stage 2. In Stage 1 your focus will be on scoping and gathering what to read, assimilating and analysing relevant content, keeping track of emerging themes, ideas and findings, and writing up initial summaries. In Stage 2, your focus will be synthesising what you have learnt from the literature to support your arguments and justify your research. By the end of Stage 2 you will have produced a substantial draft of your literature review.

\section{The role of the literature in research}\label{sect:roleLit}
Academic research does not take place in a vacuum. The academic researcher relies on a body of knowledge in their field of study, the cumulative result of collective research efforts over long periods of time. They then use their creativity to add to it, and collaborate with other researchers to develop new ideas and technologies. The main vehicle for the codification and sharing of that knowledge is the academic literature. So, reviewing the literature and adding to it are intrinsic to academic research.

Your literature review will help you frame, contextualise and justify your chosen topic and research problem and to investigate the methods of research that are considered relevant to your discipline. It will help you demonstrate\footnote{As always, you're trying to demonstrate this to the examiners and so it's their standards that you're trying to meet.} your understanding of the state-of-the-art, and to highlight knowledge gaps --- the unknowns --- to which your research will contribute.
Towards the end of your project, your literature review will also help you establish precisely which knowledge you have created and so evaluate what you have achieved in relation to other published related work.

Our framework recommends you focus on your literature review in Stages 1 and 2, although you will be maintaining your literature review throughout your project -- possibly incorporating newly published papers, adding or removing detail from what you've already written to support your dissertation as you progress.

At Masters level, examiners will be looking for a literature review based on between 30 and 60 academic articles, and in the range of 2,000--3,000 words.\footnote{Or something like 8 to 10 pages.}

\section{How to access the literature}\label{sect:howToAccessTheLiterature}
As a Masters student, your university library will be your first point of call for the academic literature -- probably online, unless you are able to travel to it every day\footnote{Most university libraries have single occupancy study rooms if you need to get away from the bustle of a home or work. Ask the duty librarian what facilities are available at your library.}. It is highly likely that alongside their collection of printed materials, your library will provide you with online access to a wide range of digital resources and a selection of bibliographical databases. The latter are collections of ejournals, ebooks and articles that can all be accessed online and searched at the same time. Among those resources you will will probably find most, if not all, of what you need for your project.

\begin{question}[subtitle={Activity: Investigating your library resources and services}] Investigate your library resources and services to identify those which are going to be particularly useful for your project.

\begin{guidance}You should pay particular attention to those services which allow you to access resources online at any time and from everywhere.
\end{guidance}\end{question}
%%Hack to correct tcbox behaviour
\color{black}

Another source of articles is Google Scholar\footnote{Google Scholar is available at http://scholar.google.com} which has collected together over 100 million\footnote{This is wikipedia's estimate: \href{https://en.wikipedia.org/wiki/Google_Scholar}{https://en.wikipedia.org/wiki/Google\_Scholar}} academic articles in English, with direct links to each. Google Scholar also has bibliographic citations for download for each article that you can cut and paste directly into your bibliographic database.

Google Scholar provides links to access to the full-text of an article, particularly for published articles from commercial publishers, although these are usually behind a pay-wall\footnote{A pay wall is an electronic way of protecting an electronic document.}. In addition, Google Scholar collects together `versions' of the same paper so that, even if you don't have access to a paid service version, there may be a pre-print version of the article also available for free. If there is no version available except a pay-for-access one, you may be able to link Google Scholar to a university library that you have electronic access to: this is an incredibly useful service which gives you direct access to those articles found in Google Scholar that are included in your library subscriptions.

Google Scholar can also help you access the so-called \emph{grey literature}, which is the collection of information produced by organisations other than publishers, such as as academia, government bodies, or non-publishing businesses and industries. It includes pre-publication and non-peer-reviewed articles, theses and dissertations, research and committee reports, government reports, conference papers, accounts of ongoing research, etc.

Google Scholar is an amazing resource for the researcher.

\begin{question}[subtitle={Activity: Connecting Google Scholar to your library services}] Investigate whether your library provides the facility to integrate Google Scholar alongside their proprietary search engines.

\begin{solution}Our university library provides some instructions on how to set up Google Scholar to connect to the databases the library subscribes to, so that direct links to full body articles appear as part of the results of a Google Scholar search.
\end{solution}\end{question}
%%Hack to correct tcbox behaviour
\color{black}

A third way to get to specific articles which you can't access from your library or via Google Scholar, is to contact the authors directly: academic authors are usually keen to have their work read and should be able to share pre-publication versions of their articles or even point you to other relevant publications which they have written. Contact details of academic authors can usually be found in the header of the articles they have published, or from their university's web pages. Alternatively, you may be able to contact them via professional networks, such as ResearchGate\footnote{ResearchGate is an online network specifically created to connect researchers around the world. It currently has a community of over 20 million researchers which use ResearchGate to \enquote{connect, collaborate, and share their work.}} or LinkedIn\footnote{LinkedIn is the largest professional network worldwide, used to make professional connections and improve one's career. It is not specific to academia, but many academics are part of that community, so it is a useful place to go to make first contact.}.

\section{How to read an article}\label{sect:Keshav}
In this section, we'll be looking at how to read an academic paper. It may seem strange to think about trying\footnote{Searching comes next, promise!} to read an academic article without having found any first! It's such an important skill, however, that it needs introducing early, and in a way that we can coach you through it the first time. Honestly\footnote{And not just for your masters project. Reading the academic literature is a professional skill that not many professionals have.}, it's going to take some time to pick up this skill, but it's an investment worth making. 

The approach we introduce in this section is one that we have used many hundreds of times with our students and is based on an excellent paper \citetitle{keshav2007read} by \citeauthor{keshav2007read}\footnote{The version we are using was revised by the author in 2013. \color{red} Why are we using this version, should we not cite it?}. Keshav suggests a practical workflow for reading an academic paper. The workflow has up to three `passes', not all of which need to be used, with each having a specific aim:

\begin{itemize}
\item the first pass gives you a general idea of what the paper is saying. It may take only 5 minutes and will allow you to disregard the paper if you find out it isn't relevant\footnote{You should still be adding it to your BMT, however, who knows when it might become relevant in future -- even a long time after your masters studies are complete.}.

\item the second pass gives you a better grasp of the content in outline of a relevant paper. It may take 20 minutes or so, and gives you the opportunity to annotate the paper\footnote{We mean write your notes on it that capture your growing understanding.} as you begin to understand it. You can stop after the second pass if you need to --  perhaps you have a large pile of papers that you wish to sort for relevance.

\item the third pass deepens your understanding of the paper to that point that you can reproduce its main arguments and conclusions -- \citeauthor{keshav2007read} calls this \enquote{Virtually reimplementing the paper.} Clearly, this form of understanding is truly deep and can take many hours, days, months, or even years! You can read a paper over and over again in the third pass, and your annotations at the end may grow to be as big as the paper itself. There will be key papers in your Masters project that have this status but they will be few.

\end{itemize}

Keshav's approach is summarised in Cref{tab:Keshav}.


\begin{SimpleNColTable}{tab:Keshav}{5}{Summary of Keshav's approach --- adapted from \textcite{Keshav2007read}}
number & Time & What to read or do & What you should know after the pass & Criteria for stopping at current pass \\
 1 & 5--10 minutes & Title, abstract, introduction; Section and sub-heading; Glance at mathematical content (if any); Conclusions; Glance over references & Main contribution; Relation to other papers & Paper not in research area \\
 2 & 1 hour & Write down key points, comments, questions; Mark relevant unread references & Main thrust of the paper with supporting evidence & Paper of interest but not key to your research \\
 3 & $>$1hour & Virtually reimplement the paper; Identify/challenge assumptions  & New knowledge contributed; Strong/weak points, implicit assumptions, missing citations, potential issues; Insight into complex arguments and presentation techniques; Jot down ideas for future work  &  \\
\end{SimpleNColTable}

Ok, now it's time to give you something to try Keshav's workflow on!

\begin{question}[subtitle={Activity: Applying Keshav's approach to Keshav's paper!}] The paper you need to investigate is Kershav's paper itself:

Keshav, S. (2007) `How to read a paper', ACM SIGCOMM Computer Communication Review, 37(3), pp. 83--4.

Use Google Scholar to search the article's title and then download the paper\footnote{Remember to record it in your BMT.}.

Start reading Keshav's paper using the guidance in Cref{tab:Keshav}.

\begin{guidance}You should ensure you make appropriate annotations as you go along, then develop and include appropriate notes and summaries in your BMT.
\end{guidance}
\end{question}
%%Hack to correct tcbox behaviour
\color{black}

% Cref{fig:annotations} shows the annotations from our three readings of Keshav's paper for comparison with yours. LR -- what's the purpose of this? Should it be part of the previous activity ?
%\begin{figure}[htbp]
%\caption{Our annotations on Keshav's paper\label{fig:annotations}}
%%\includegraphics[width=\textwidth]{keshav2007readannotatedpdfpdf.pdf}
%\end{figure}

Reading Keshav's paper using Keshav's workflow will give you a headstart on the following\footnote{It's also satisfyingly recursive!} --- something we've learned in the time Keshav's paper has been in use by our students. 

Firstly, the time suggested for each pass is only an approximation, and will depend on your own skills and previous experience of reading the academic literature, whether the subject matter is new to you, or even that the article may be poorly written and difficult to follow. Secondly, three passes may not strictly be enough: for some articles, particularly the most critical to your research, the third may actually consist of multiple passes.

\emph{Vice-versa}, not all papers require a third pass and you should be careful in deciding whether a particular paper should have one at any point in time: you should never totally disregard a paper -- unless it is clearly out of scope -- as the importance of understanding a paper may only become apparent later: it may be a highly cited paper that needs to be understood so that other papers become accessible, for instance.

Secondly, we've found the following to be very important: at each pass, Keshav stresses the importance of making annotations as you read. Whether you print the article or work to annotate an electronic copy\footnote{A tablet of some form with appropriate note taking software is a great way of doing this.}, effective annotations will help you identify, and access more easily later on, those elements of the paper which are of particular interest to you. In the second pass, your annotations should include comments or queries on elements you find particularly relevant, interesting, or unclear, while in the third pass, they may include ideas for future work which may provide some inspiration for your own research. Your annotations will then help you write up your own notes and summaries. Electronic copies of notes should, of course, also be stored in your BMT.

Finally, Keshav suggests you should highlight references upon which the arguments of the paper rest and that you may like to read later on, and comment on possible relation to other articles you may have already read. This will help you contextualise the article in the wider literature you are reviewing.

We recommend you apply Keshav's workflow when reading the academic literature. But first, let's consider the literature review process as a whole.

\section{How to review the literature}\label{sect:stage1LiteratureReview}

Reviewing the literature is a process of knowledge discovery\footnote{To know where there are gaps in knowledge, you need to know what is already known -- hence knowledge discovery. A poor literature review may leave you unable to claim a contribution to knowledge.}, and it is both iterative\footnote{Iterative = it never finishes -- although it's better than it sounds: later iterations will typically make fewer changes than earlier ones} and incremental\footnote{Incremental = you'll add to it over time.}. Its core activities are summarised in Cref{tab:litrevactivities} alongside the key skills required. 

%In Stage 1 you will focus on the first three activities in the table, with synthesising being the focus of Stage 2. 

To conduct your literature review successfully, you will need to be able to select work which is relevant and should be included, assimilate and summarise relevant work of other researchers, synthesise and critically appraise ideas from different sources, establish links between studies and their findings, and draw on strengths and limitations of published research. In summary, in your literature review you will need to make use critically and creatively of the content of articles, books and other literature sources you have reviewed to demonstrate your knowledge of the subject, support your own arguments and justify your research. In this chapter and the next, this handbook will help you develop and apply these skills.

% LR -- I don't think we need this figure
%\includegraphics[width=700pt,height=501pt]{lit rev flow.pdf}

\begin{SimpleNColTable}{tab:litrevactivities}{4}{Key activities in reviewing the literature, including key skills required}
Activity & Aim & Key skills & Outputs \\

%Planning & To identify topics and sub-topics to investigate & Critical analysis of topic and subtopics & Research Problem & Search keywords & Key skills: ?? Outputs: ?? \\
Searching and gathering & To identify search terms, conduct bibliographical searches and collect articles for consideration & Applying effective search strategies; systematic searching bibliographical databases & Successfully applied search terms and related electronic (or paper) articles for further consideration \\
Processing & To establish the relevance of searched articles for follow-up in-depth analysis, and record them in a BMT & Skim reading, note taking, bibliographic management & Populated BMT, article annotations, notes, article ranking for follow-up in-depth analysis  \\
Assimilating and analysing & To engage with the content of relevant articles, and keep track of them, establishing potential links, common themes and emerging knowledge gaps & Critical reading, including identifying key arguments, findings, strengths and weaknesses; critical thinking, including comparing articles and identifying relationships and gaps; further note taking & Detailed article annotations, notes, short summaries, tables, diagrams, etc. leading to understanding  \\
Synthesising & To consolidate and summarise what you have learnt in well-formed academic arguments and a well structured narrative, able to justify your proposed research on the basis of the identified knowledge gaps & Critical thinking and academic writing, including making academic arguments and organising your summary narrative &  Sound academic arguments and well-structured narrative, constituting  a substantial draft of your literature review \\
\end{SimpleNColTable}

You will need to read extensively during your research and you will read a lot\footnote{When you write, you will focus only on the most relevant of the papers your search turned up.} more than your final selection of articles cited in your dissertation. So, a natural question is \enquote{How do I know when I have read enough and have the references I need?} Your final selection, simply, will be sufficient to convince the examiner that you have identified a gap in knowledge. 


%The illustration in Figure\Cref{fig:pacman} might help.
%
%\begin{figure}[htbp]
%\caption{Pacman figure to add\label{fig:pacman}}
%%\includegraphics[width=\textwidth]{keshav2007readannotatedpdfpdf.pdf}
%\end{figure}

You might like to think of identifying a knowledge gap as trying to locate a particular rock pool\footnote{Might not be a good metaphor:)} starting from a map of your country. For places extremely far away from the rock pool, you don't need any description at all, other than mentioning the country. But, as you get closer, you might want to pick out a particular town to give an approximate location. Then, you need to give more and more precise descriptions as your reader gets closer and closer to the rock pool, finally giving them a really precise description so they can lock onto it.

In detail, you want your literature to identify the area in which your contribution to knowledge is going to be made, and to support your arguments in relation to your choice of research problem and the potential contribution to knowledge of your research with reference to the state-of-the-art.

Given this, if there are significant points which are not linked to the relevant literature, and not supported by other evidence otherwise, then you don't have enough references; conversely, if there are references which don't relate to any significant point, then they may well be superfluous. You can apply this rule of thumb at each stage of your research to assess whether you have done enough to progress to the next stage.

There is a caveat. A literature review is never really finished because as you gain insight from the literature, those insights will in turn point you toward other reading; once you have answered a question, new questions arise, and so on. It is a process you will go through till your final dissertation submission, and even then you will have new insights to follow-up or unanswered questions to address. But that's good, as it simply points to the fact that more can come from your work that other researchers may well pick up and so can contribute to your conclusions and further work, which examiners always like to see.

It's now time for your first literature search!

\subsection{Searching and gathering}\label{ssect:SearchingAndGathering}

At this stage of your project, the main objective of your literature review is to demonstrate sufficient understanding of the topic you have chosen to be able to justify why your chosen research problem is worth investigating, i.e., why it will lead to new knowledge. In particular, your literature review should help you consider its wider significance beyond your personal or professional interest. 

It is not surprising, therefore, that we recommend you start from your research problem to plan how you will go about searching and gathering articles for review. Throughout your project, your research problem may well be refined, so that your literature review will need adjusting accordingly\footnote{This is yet another reason for putting as much effort as you can into your research problem, so that downstream activities, such as the literature review, will change as little as possible as you progress.}. 

\paragraph{What to read}

The gold standard of academic publications are peer-reviewed articles, that is articles which have been rigorously scrutinised by academic experts in the field (the `peers') to ensure high scientific quality. Peer-review is a practice unique to academia and it reflects the critical need in the academic community to establish that work contributes new knowledge, that it is the work of those claiming it, and that the paper has been written to be accessible\footnote{This doesn't necessarily mean it will be easy to read!} to the community. Academic journal and conference articles are usually peer-reviewed\footnote{The peer review process may be more or less stringent, so you need to take care when selecting articles and you should treat each publication on its own merits. You can check through your university library if an article is peer--reviewed or not.}.

Most references in your literature review should be peer-reviewed, therefore this handbook recommends you start with those. However, there may be some scope for using other non-academic sources -- the grey literature. For instance, articles that are professionally relevant could be helpful, even if not peer-reviewed. Equally, you may reference government and other official reports: although non peer-reviewed, they may have undergone some level of public scrutiny and still contain reliable information. Books and even websites could also be used sparingly, but they are unlikely to have been peer-reviewed or scrutinised, so you should treat their claims to be new knowledge with appropriate caution.

\paragraph{Identifying search terms}

The number of peer-reviewed academic articles is very large and growing exponentially\footnote{2023 estimates put the number of published academic articles to between 40 and 50 millions, with 5 millions new articles published in 2022!}. So even without the grey literature, focussing on the most relevant articles for your research could be a challenge if not approached systematically.

A good place to start is to brainstorm a set of search terms\footnote{A search term is a word or a combination of words that you can key into a search engine.} that bear a relationship to each of the elements of your research problem. As we've explained, a research problem has the following elements: a context including some phenomena of interest, a knowledge gap in relation to those phenomena, and a justification in terms of whom the stakeholders are and the reasons why the problem matters to them. Your quest for knowledge about each of them will drive your literature search.

Let's see how this could be done on an example.

\begin{running}{Search terms from a research problem} Let us consider the following research problem:

\blockquote{To evaluate the effectiveness of techniques to test embedded safety-critical software while the hardware is still unavailable. Full software testing can only occur once the software is embedded in its hardware. However, in many safety-critical avionics systems, this can happen very late in the development process, leading to expensive re-design if errors are found. Early software testing while the hardware is still unavailable could reduce such late occurrence of errors and expensive re-design.}

We can take each problem element in turn to identify possible search terms. In doing so, we have come up with those in the table below. This initial set should be sufficient to perform our initial search.

\begin{SimpleNColTable}{tab:searchTerms}{3}[\textwidth]{Brainstormed search term}[X[2]X[6]X[6]]
 Problem element & From problem statement & Our brainstormed search terms \\
 Context & Avionics systems development & Avionic systems; system development \\
 Phenomena & Embedded software; safety-critical applications; testing approaches & Embedded software; safety-critical systems; software testing \\
 Knowledge gap & To evaluate the effectiveness of testing techniques when the hardware is not available & Testing techniques; effectiveness \\
 Stakeholders & Software developers, safety engineers within the context & Software developers; software engineers \\
 Reasons & Early software testing while the hardware is still unavailable could reduce the late occurrence of errors and expensive redesign. & Late errors; redesign
\end{SimpleNColTable}
\end{running}

\begin{question}[subtitle={Activity: Search terms brainstorming}] Using your research problem, fill in a table similar to that in the example, indicating a few search terms you have brainstormed for each research problem element.

\begin{guidance}The aim of this activity is to arrive at an initial set of search terms you can use to start your own literature search. You should start by words in your problem descriptions, but you could also include synonyms, particularly if you are not sure which technical terms are used in the literature.

You shouldn't worry too much if you have come up with too many or too few in relation to each element: you will have an opportunity to review and refine your choices later as this, too, is an iterative and incremental process.
\end{guidance}\end{question}
%%Hack to correct tcbox behaviour
\color{black}

Your brainstormed search terms will provide a starting point for searching the literature. Initially, you should be prepared for the fact that very large numbers of articles might be returned with many of them irrelevant to your final goal of understanding your chosen topic. Therefore, you are likely to need to refine your search terms and also come up with ways to narrow down your search results to manageable set of relevant articles.

Search terms are often used in combination to create more complex search terms which may increase the likelihood of finding articles which are close to your topic of interest. Basic Boolean operators, as those described in Cref{tab:booleanoperators}, are used for this purpose.

\begin{SimpleNColTable}{tab:booleanoperators}{2}{Boolean operators for search terms}[X[1]X[3]]
 Boolean operator & Effect of the search \\
\fakeverb{Term 1 AND Term 2} & Materials containing both \fakeverb{Term 1} and \fakeverb{Term 2} will be returned. This is used to narrow down a search. \\
\fakeverb{Term 1 OR Term 2} & Materials containing only \fakeverb{Term 1} or only \fakeverb{Term 2} or both will be returned. This is used to expand a search. \\
 \fakeverb{NOT Term 1} & Materials \emph{not} containing \fakeverb{Term 1} will be returned. This is used to narrow down a search. \\
\end{SimpleNColTable}

\begin{example}{Using Boolean operators}
Let's return to our example. We have conducted the following searches.

\begin{enumerate}
	\item Firstly, we typed in:

	\fakeverb{safety critical systems AND software testing}

	to Google Scholar and this returned 5310 results: that's a lot of papers to read, so we knew that we needed to narrow the search down.
	
	\item We therefore added another term and typed in:

	\fakeverb{safety critical systems AND software testing AND avionics systems}

	which returned 389 results: more manageable, but still a very large number.

	\item To narrow the search further, we typed in:

	\fakeverb{safety critical systems AND software testing AND avionics systems AND testing techniques}

	and found 90 results.

	\item Finally we tried:

	\fakeverb{embedded software AND testing techniques OR testing methods AND avionics systems}

	which returned 35 results.
\end{enumerate}

In this case, adding \fakeverb{avionics systems} helped us narrow down the search, presumably because only avionics applications were returned, rather than any kind of safety critical system; similarly, adding \fakeverb{testing techniques} presumably helped the search engine focus on techniques rather than, say testing processes or other. However, when we added \fakeverb{testing methods} using the \fakeverb{OR} operator, the hit list got even smaller, which was surprising. This demonstrate that search engines do not always behave as we would expect, so that it is important to try different approaches to arrive at a desirable outcome.
\end{example}


\begin{question}[subtitle={Activity: Your first literature search}] Conduct a search in Google Scholar from the search terms you brainstormed out of your research problem elements, combining them using the operators above if appropriate.

For each, record both the combination you have used and how many articles Google Scholar returns as hits. Reflect on how using different terms and operators may change the number of hits and how that may inform ways to widen or narrow down your searches.

\begin{guidance}
In combining your search terms, you could start by using the \fakeverb{AND} operator, then repeat the search trying different combinations of search terms, different operators, adding and removing terms, or even using synonyms of your search terms.

You should ensure you record both the combinations you've used and the resulting number of hits.
\end{guidance}
\end{question}
%%Hack to correct tcbox behaviour
\color{black}

As you may have gathered from this activity, to a large extent your initial search will be a trial-and-error process\footnote{Be sure to be systematic and record which search terms you have used: this will save a lot of pain later when you're trying to remember which search term you used to pick a particular article!} and you may end up iterating several times. It's not wasted effort, though. You can make the most of the time spent by:

\begin{itemize}
\item Using Boolean combinations of search terms wisely to narrow or widen your searches

\item Particularly if only few hits are returned, replacing your search terms with new terms -- often synonyms for existing ones.
\end{itemize}

Finally, although in this section we have focussed on Google Scholar, the same techniques apply to most bibliographical database search engines, so you could repeat the activities on subscription databases you can access through your library.

\paragraph{What if there are still too many hits?}
Even if you spend your search time wisely, you may still end up with a large number of hits to consider. An important thing to recognise is that not all of the hits returned will be relevant, so you will need to make an initial assessment of what to exclude from further consideration. When the number of search hits is high this will take time, so you will need to select which articles to look at first.

We recommend you start by considering recent review articles. Review articles, also known as surveys, are academic articles which summarise the current understanding of a specific topic or phenomenon. They are usually the result of analysing and synthesising academic literature or bringing together findings from several studies. Any recent review articles that appear in your search offer a very easy way to start your literature review: not only do they collect together recent relevant papers, but they may have an overview or a précis of each.

If no review articles are included in your hits, you could repeat your search by adding \fakeverb{AND Review} or \fakeverb{AND Survey} to the end of your search term. Alternatively, you could ask your supervisor if they can suggest a review article you can start with. It's a great question to ask at an early meeting with them. Your supervisor may also suggest other seminal papers for you to look at.

By `recent' we mean within 2 to 5 years, particularly for fast changing disciplines, such as Computing. For slow changing disciplines, the time frame of publication may be less important. Reducing the timeframe of publication is an effective way to cut down the returns considerably. Google Scholar allows you to select articles within a specified timeframe, and similar facilities are available in most bibliographical database search engines.

Second thing to recognise is that not all hits may be accurate -- they may have the wrong date on them for instance -- so care is warranted. Repeating the search on different bibliographical databases may help you identify erroneous entries.

\begin{question}[subtitle={Activity: Narrowing down your hit list}] Consider the outcome of one of your searches. Identify any recent review paper which may be included. If necessary, revise your search term by including \fakeverb{AND Review} or \fakeverb{AND Survey} and try again, or set a specific timeframe to reduce the number of outputs.

\begin{solution} Going back to our embedded software testing example, we added:

\fakeverb{review OR survey}

to our last search term, and only considered publications since 2020. The resulting list only included 9 hits, of which 4 were literature reviews.
\end{solution}\end{question}
%%Hack to correct tcbox behaviour
\color{black}

\subsection{Processing}\label{ssect:Processing}
Once your search has returned a manageable set of articles for further consideration, say up to 20, you should begin processing them to decide whether they could be relevant and, hence, considered further. 

Although you will need to read all of the papers that you find potentially relevant, fortunately you won't need to read all of a paper to find out if that's the case! Sometimes it may be sufficient to read what accompanies the paper on a search service\footnote{Google Scholar, for instance, includes an extract from the paper that caused it to match your search term.}. Otherwise, clicking on the link to the paper will reveal its abstract, which you can quickly skim to check whether the paper might be relevant. If so, add it to your BMT as something for further consideration.

Don't be too picky at this point: it's better to include something in your database that you don't use later, than to discard something that you find later might have been relevant.

\paragraph{How will I know something is potentially relevant?}
Understanding the relevance of articles is an important skill that develops with the practice of reading the academic literature. Relevant articles will help you in a number of ways, including: to develop a deeper understanding of the context of your research, its stakeholders and beneficiaries; to raise your awareness of the significance of the contribution to knowledge you intend to make and any inherent difficulties in doing so; to show you different ways of thinking about your problem; to illustrate possible approaches or techniques you could apply to your problem.

During processing you need an efficient way to decide which of the papers in your search hits are potentially relevant, before investing considerable time and effort into assimilating and analysing their content. 
%
%A way to decide whether an article is relevant is to reverse engineer the research problem it addresses and compare it with your own research problem. What do the two problems have in common?
  
  You'll know a paper is potentially relevant if the phenomena that are mentioned in its abstract bear some relation to those of your research problem. Therefore, focusing on such phenomena is what we recommend you do at this point. 
  
  You should keep an open mind at this point, as it is likely the articles you are considering will use a different terminology from yours. Nevertheless, you should be able to judge whether they are close enough to elements of your own research problem, particularly the knowledge gap and its context.  
  

It is important to acknowledge that there is no standard way of referring to phenomena, so having an open mind -- at least initially -- as to what could constitute them would be good. Remember, phenomena were defined very generally as observables. In a paper, phenomena, as observables, can be referred to in many ways, including:

\todo{LR -- this, till the end of the sub-section, needs work; for Jon to do}

\begin{itemize}
\item Directly: statements such as \enquote{the prices of software services were \emph{observable} in this study} -- the observables are the prices of software services.

\item Indirectly: statement such as \enquote{the prices of software services were \emph{not directly observable} in this study, so we used the proxy of hours worked accumulated across the team of developers} -- the observables are the number of hours worked.

\item As measurable: statements such as \enquote{the number of bugs in the software were given through the collection of bug reports} -- the observables are the bugs in the software.

\item As inferences from other phenomena: statements such as \enquote{the number of bugs in the software were estimated from the number of bug reports} -- the observables are the number of bugs reported.
\end{itemize}

\begin{example}{Going back to our example of safety-critical development...}...one of the review articles we found in our search is:

\fullcite{garousi2018testing}

Its abstract reads\footnote{Increasingly, and very helpfully, many journals and conferences research articles are beginning to use \emph{structured abstracts}, such as the one in this article. A structured abstract has sections which bring out the context, objectives, results and other information.}:

\newcommand{\firstquotedparagraph}[1]{\textbf{#1}\quad}%%So that textcquote places quotes in correct place

\textcquote{garousi2018testing}{\firstquotedparagraph{Context}\quad Embedded systems have overwhelming penetration around the world. Innovations are increasingly triggered by software embedded in automotive, transportation, medical-equipment, communication, energy, and many other types of systems. To test embedded software in an effective and efficient manner, a large number of test techniques, approaches, tools and frameworks have been proposed by both practitioners and researchers in the last several decades.

\paragraph{Objective} However, reviewing and getting an overview of the entire state-of-the-art and the – practice in this area is challenging for a practitioner or a (new) researcher. Also unfortunately, as a result, we often see that many companies reinvent the wheel (by designing a test approach new to them, but existing in the domain) due to not having an adequate overview of what already exists in this area.

\paragraph{Method} To address the above need, we conducted and report in this paper a systematic literature review (SLR) in the form of a systematic literature mapping (SLM) in this area. After compiling an initial pool of 588 papers, a systematic voting about inclusion/exclusion of the papers was conducted among the authors, and our final pool included 312 technical papers.

\paragraph{Results} Among the various aspects that we aim at covering, our review covers the types of testing topics studied, types of testing activity, types of test artifacts generated (e.g., test inputs or test code), and the types of industries in which studies have focused on, e.g., automotive and home appliances. Furthermore, we assess the benefits of this review by asking several active test engineers in the Turkish embedded software industry to review its findings and provide feedbacks as to how this review has benefitted them.

\paragraph{Conclusion} The results of this review paper have already benefitted several of our industry partners in choosing the right test techniques/approaches for their embedded software testing challenges. We believe that it will also be useful for the large world-wide community of software engineers and testers in the embedded software industry, by serving as an \enquote{index} to the vast body of knowledge in this important area. Our results will also benefit researchers in observing the latest trends in this area and for identifying the topics which need further investigations.}

\end{example}

\begin{question}[subtitle={Activity: the relevance of early hits to your search terms}] For five of the hits to the search you conducted in Activity ??, fill in the relevancy matrix below, and score them for relevance.

\begin{SimpleNColTable}{tab:earlyHits}{3}{Early Hit Relevance}
Type&Comment&Score\\
 Direct & TBD& TBD\\
 Indirect & & \\
 Measurable & & \\
 Inferences & & \\
Complete this...& &
\end{SimpleNColTable}

\begin{solution}TBD
\end{solution}\end{question}
%%Hack to correct tcbox behaviour
\color{black}

\paragraph{What to do when you find a potentially relevant paper}
First thing you should do is to record it in your BMT. You should record it with the search term that you used to find it so that you can rerun the search to:

\begin{itemize}
\item Find similar papers that might later have added relevance to your work

\item Find new papers that have been added to the literature since you last looked

\end{itemize}

In the latter case, you should also record the date\footnote{Many bibliographic management tools will do this for you. Check whether yours does.} that you found it so you can check.

\paragraph{What if none of the hits are relevant?}
This is a point where you need to iterate back to searching bibliographical databases for new hits, using different search terms to widen the scope of your search. Also useful on these occasions is to ask your supervisor for advice: they should be able to point to seminal or review papers you should start from, or indicate  which authors have made core contributions to your topic.

\paragraph{What if too few hits appear relevant?}
In this case too you will need to iterate back to searching bibliographical databases for new hits. In such cases, however, the few hits you have identified can also help you find other potentially relevant work. This is because articles cite other related articles, and in turn they are cited by other others in their papers. 

\textit{Citation searching} is the process of looking backwards and forwards to track citations, starting from an article of interest. Specifically, backward  searching looks through the articles cited by a paper, while forward  searching looks through the articles which cite that paper.

Looking backward is straightforward: all you have to do is go through the reference list at the end of the article. Looking forward, instead, will require you to make use of Google Scholar which, for every hit, provides a list of the articles that cite that paper\footnote{Similar features are available in many other bibliographical databases, like the Web of Science.}. 

\begin{question}[subtitle={Activity: Using Google Scholar `Cited by...' feature}] Conduct a Google Scholar search for Keshav's paper, and examine the information returned. At the bottom, you should see a number of links, included a `Cited by...' link which indicates how many articles cite the paper. Click on the link and examine the related list of articles.
\begin{solution}
At the time of writing, the `Cited by...' list included 236 articles: they were all the papers citing Keshav's work at that time. That number may be greater when you do this activity.
\end{solution}
\end{question}
%%Hack to correct tcbox behaviour
\color{black}

Alongside, the `Cited by...' link, Google Scholar also provides a `Related articles' link, where you can find yet another list of articles, some of which may be relevant to your work.

\begin{question}[subtitle={Activity: Using Google Scholar `Related articles' feature}] Return to your Google Scholar search for Keshav's paper and click on the `Related articles' link. Compare this list of articles to those at the 'Cited by...' link.  What's their main difference?
\begin{solution}
The `Cited by...' list includes articles which cite Keshav's work. As such they may be addressing very different topics and may belong to different disciplines. On the other hand, the `Related articles' list includes papers which address the same topic as Keshav, including various approaches to reviewing the literature.
\end{solution}
\end{question}
%%Hack to correct tcbox behaviour
\color{black}

You should make good use of such facilities as they can speed up considerably your  search for relevant work to review. They are equally valuable at the beginning of your literature review process, while you are trying to identify an initial set of articles to read, and later on, when you wish to expand your literature review on specific topics.

\subsection{Assimilating and analysing}\label{sect:Stage1AssimilatingAndAnalysing}\label{ssect:AssimilatingAndAnalysing}
It's time to engage with the content of the articles you've identified as potentially relevant during processing. For this, you need to refer back to the Keshav's workflow you practiced in Cref{sect:Keshav}.

To understand an academic paper is to understand the contribution to knowledge that it makes. While you won't have to understand in depth everything you read, there will be some articles which are so fundamental to your research that you will need to study them carefully. This will require a substantial investment of your time and can be very challenging: even the most experienced researcher is unlikely to be able to read an academic paper once and immediately understand it in its entirety. 

 As your collection of articles grows, Keshav's approach will help you separate efficiently which are the most important papers, deserving several passes, from those where only a first pass may be sufficient, perhaps because they are not all that relevant ofter all.
 
 As the number of papers you read grows, it will also be increasingly difficult for you to keep track of what each article is about without having to refer back to your annotations. Even more difficult will be to compare and contrast quickly key concepts you have encountered in different articles.
 
 Therefore, alongside Keshav's approach, you will also need techniques, introduced in this section, to help you track and compare content, establish relationships and identify emerging themes. 
 
\paragraph{Tracking content of academic articles}

A summary-comparison matrix\footnote{This term is used in \textcite{sastry2013summary}, although another term used online is `literature matrix.'}is a useful tool to track the content of academic articles in a way which also makes it easier to compare the articles you have read. 

A summary-comparison matrix is a table which includes a row for each article you have reviewed, and a number of columns corresponding to key aspects of the article you wish to summarise and compare with other articles, such as the research problem or question addressed, contextual facts known at start of the reported study,  key contributions made, research methods applied, etc. The exact form of the matrix can vary, but we recommend you include at least the following columns in your matrix\footnote{Feel free to adapt it to your own needs, for instance by changing the columns or adding new ones.}:

\begin{itemize}
	\item Reference: either the full reference or a link to your BMT, to help you locate the article quickly.
	\item Research problem/question: your understanding of the research problem or question(s) the reported research aimed to address.
	\item Research methods/approach: how the authors conducted their study. Don't worry if you don't understand all the details at this point: we will look at research methods in detail in Stages 3 and 4.
	\item Known facts/assumptions/definitions/gaps at start: these are the premises of the study, that is arguments or evidence on which the study relies upon.
	\item Key findings/contribution made: what the study has contributed in terms of advancing knowledge in the field of study.
	\item Notes in relation to own research: these are your own notes on how the content of the article is relevant to, or could be used in your own project.
\end{itemize}


\begin{question}[subtitle={Activity: Using a summary-comparison matrix for the Keshav's paper}] Create a summary-comparison matrix with the columns recommended above, and fill in its first entry in relation to the Keshav's paper you considered in Cref{sect:Keshav}.

\begin{guidance}
You can use a spreadsheet to create the matrix. To fill in the entry, you will need to go back to your notes on the article, and possibly skim through the article again to extract the relevant points.
\end{guidance}
\begin{solution}
	This is our attempt. Yours may differ.
	
\begin{SimpleNColTable}{tab:keshavBibdata}{2}[\narrowtablewidth]{Bibliographic database entry for \citeauthor{keshav2007read}, \citeyear{keshav2007read}}[{X[1]X[3]}]
\relax\\
%\SetCell[c=2]{l}\fullcite{keshav2007read}\\
Research problem/question & to define an approach to reading and assimilating academic articles \\
Research methods/approach & the approach is explained and exemplified; no formal evaluation is presented \\
Known facts/assumptions/definitions/gaps at start & research requires spending a significant amount of time and effort reading academic papers; particularly novice researchers lack the skills to do so effectively, leading to wasted effort; such skills are rarely taught  \\
Key findings/contribution made & a practical approach to help researchers engage with the content of an academic articles effectively and efficiently; an outlined process for reviewing the literature based on the approach \\
Notes in relation to own research & it seems a simple, yet effective approach, well worth applying when working on my literature\\
\end{SimpleNColTable}
\end{solution}
\end{question}
%%Hack to correct tcbox behaviour
\color{black}

What to include in your summary-comparison matrix is, of course, a matter of judgement, so there is an element of subjective interpretation of each article content. This is expected, so that the column headings we have chosen should help you be consistent and systematic across different articles. It is also up to you to choose the level of detail for each entry: while you should try to be succinct, it is important that you include enough information to use the matrix without having to continually refer back to your notes and articles. These may result in quite a bit of text included in the matrix, as shown in the next example, which we will use to illustrate the techniques introduced in this section.

\begin{example}{An example of summary-comparison matrix}
We are interested in reviewing the literature in relation to the following research problem: 

\textit{How to leverage curriculum data effectively in Higher Education Institutions (HEIs) to augment curricular decision making capability. This matters to HEIs because, while they collect vast amount of curriculum data, how to harness effectively remains an open question. An effective use of such data may lead to better curriculum benefiting both HEIs and their students. 
}

%\begin{itemize}
%	\item[] \textbf{Context:}  Higher Education Institutions (HEIs) collect vast amount of curriculum data which could be harnessed to support curriculum decision making. While data collection, and data technology and know-how are growing across all sectors, embedding data in HEIs’ curricular decision making remains problematic
%	\item[] \textbf{Phenomena:} curriculum data; curriculum decision making processes; data technology and know-how; enablers and barriers to embedding data in curricular decision making
%	\item[] \textbf{Knowledge gap:} how to leverage curriculum data effectively in HEIs to augment curricular decision making capability
%	\item[] \textbf{Stakeholders:} HEIs, students, society
%	\item[] \textbf{Reasons:} HEIs' key role is to develop and deliver curriculum to educate and exchange knowledge. It’s vital to HEIs that their curriculum is fit-for-purpose and continuously adapts to the changing needs of students and society. Data can be use as an effective tool for measuring and predicting, in support of curricular decision making 
%\end{itemize}
We conducted a literature search, out of which we identified eleven potentially relevant articles. We constructed a summary-comparison matrix, whose first three entries are illustrated in Cref{fig:exampleSCMatrix}. It is not necessary for you to read it through, but you may notices that our entries are quite detailed.
\end{example}

\begin{figure}[htbp]
\centering
\includegraphics[width=\textwidth]{Figures/exampleSCMatrix.pdf}
\caption{Example of summary-comparison matrix}
\label{fig:exampleSCMatrix}
\end{figure}


\begin{question}[subtitle={Activity: Building your own summary-comparison matrix}] Create a summary-comparison matrix for your own project, based on your reading of the articles you have identified as potentially relevant during processing.
\begin{guidance}
You should apply the Keshav's approach to read and assimilate the content of each article. It is worth including in the matrix only articles which you deem as definitely relevant after a first pass, as those are the articles you're likely to return to over and over again in writing up your literature review.
\end{guidance}
\end{question}
%%Hack to correct tcbox behaviour
\color{black}

\paragraph{Identifying concepts of interest}

Once you have a number of entries in your summary-comparison matrix, you can start analysing them in order to identify concepts --- facts, ideas, questions, etc. --- relevant to your research problem. You are looking for concepts that can help you address the following questions in relation to your research problem:

\begin{SimpleNColTable}{tab:conceptsOfInterest}{2}[\narrowtablewidth]{Concepts of Interest}[X[4]X[4]]
 Questions & Contribute to \\
 {\textbullet~When, where and why does the problem occur?} & Articulating the problem in its context \\
 {\textbullet~Whom does it affect?\\ \textbullet~How serious is it?\\ \textbullet~What are the benefits of addressing it?} & Establishing its significance, stakeholders and beneficiaries \\
{ \textbullet~What has been done so far to address it?\\ \textbullet~What else could be done to address it? }& Establishing knowledge gaps and the potential contribution to knowledge \\
\end{SimpleNColTable}

To this end, we recommend you highlight concepts of interest within your summary-comparison matrix, then collect them in a \textit{concept matrix}. This is yet another table, with one concept per row and one article per column, and a tick at their intersection to indicate whether the concept appears in the article. Some concepts will recur across several articles, and the concept matrix is a useful visual tool to identify those recurring concepts. 

That some concepts recur in papers in your chosen area might suggest to you that they are important concepts, which have attracted the attention of several researchers\footnote{Of course, forming a judgement on this is something you will have to do, perhaps with the help of your supervisor.}. On the other hand, new concepts may are introduced in very recent work, which are yet to be established across the literature: it may be good to make a special note of those, as they might reveal current hot research areas or research paths that are still to be trodden.

Recurring concepts may also point to relationships and parallels which may be drawn between the work of diverse authors. This will help you in your critical summaries, where you will need to bring together and compare and contrast ideas from different authors and articles.

\begin{example}{An example of concept matrix}
Carrying on with our example, we have highlighted a number of concepts in our summary-comparison matrix, as shown in Cref{fig:highlightedConcepts}, from which we have constructed a concept matrix, a small extract of which is given in Cref{fig:exampleCMatrix}

In our example, we have applied the following heuristics to each article, to decide which concepts to include:
\begin{itemize}
	\item an indication of the research problem/question
	\item facts contributing to a characterisation of the problem context 
	\item key contributions made
	\item any specific tool/technique/technology developed or adopted to address the problem
\end{itemize}
We have, however, avoided details about research methods at this point, as our focus is to characterise and justify our research problem. Research methods are, of course, very important, and something we will consider in detail later, in Stages 3 and 4.
\end{example}


\begin{figure}[htbp]
\centering
\includegraphics[width=\textwidth]{Figures/exampleHighlightedConcepts.pdf}
\caption{Example of highlighted concepts}
\label{fig:highlightedConcepts}
\end{figure}

\begin{figure}[htbp]
\centering
\includegraphics[width=\textwidth]{Figures/exampleConceptMatrix.pdf}
\caption{Example of concept matrix}
\label{fig:exampleCMatrix}
\end{figure}

\begin{question}[subtitle={Activity: Building your own concept matrix}] Create a concept matrix for your own project, derived from your summary-comparison matrix through the identification of concepts of interest.
\begin{guidance}
You could start by applying the heuristics from the example to identify and highlight concepts in your summary-comparison matrix, but do not be constrained by them, and feel free to come out with other heuristics to guide your choices. Expect to go over each entry a number of times before reaching the set of concepts you wish to focus on.
\end{guidance}
\end{question}
%%Hack to correct tcbox behaviour
\color{black}

\subsection{Synthesising}\label{sect:stage1Synthesising}\label{ssect:Synthesising}

The next step is to analyse both recurrent concepts and concept relationships to work out whether they contribute to wider common themes. This is a step of synthesis and abstraction: synthesis in the sense of bringing together different ideas; abstraction in the sense of deciding whether specific concepts may be instances of more abstract ones. Let's look at what we mean through our example.

\begin{example}{What we mean by common themes}
In our review, we found that many concepts we have identify point to some sort of modelling of students' study pathways through study programmes. While both modelling techniques and the purpose of it may vary from article to article, they all shared this common theme. By recognising `modelling study pathway' as an emerging theme, we can both abstract and bring together research from different authors, making it easier for us to write a summary which compares and contrasts their work.  
\end{example}

\paragraph{Developing a concept map}
To find common themes you will need to exercise both judgement and creativity, and the tool we recommend you use is a \textit{concept map}: this is a diagram where you can represent visually concepts and their relationship. To produce your map you could use a drawing package, or a physical board. What matters is that the tool you use should allow you to `move things around' to explore different ways to relate and group your concepts. 

In constructing your concept map, we recommend you follow the following process:
\begin{description}
	\item[Step 1] reproduce on the map all the concepts from your concept matrix, linking them to the articles they came from. For a recurring concept, only include the concept once, but clearly link it to all the articles in which it appears. 
	\item[Step 2] colour-code concepts\footnote{If you have problems using colours, then you could use symbols to distinguish themes instead.} so that concepts you think are related share the same colour; name the themes corresponding to each colour
	\item[Step 3] re-arrange the concepts, grouping them by colour; make sure the link between concepts and articles are transferred to the groups
\end{description}
%
Note that it is unlikely you will reach your final grouping in one pass: this is yet another iterative process, so you should expect to go back and forth between steps!

 Cref{fig:exampleConceptMap} illustrates the outcomes of applying the three steps in our example: in the map, concepts are in plain text, and references are within yellow stickers; a legend was used to relate the identified themes to their colours. You can, of course, choose different visual representations to achieve similar outcomes.

Although we only illustrate the final outcome at each step, we should stress that it did take us several attempts to identify a set of themes we were happy with. Note how some groups are larger than others: the group `size' provides an indication of which themes feature more prominently in the articles reviewed. For instance, in our example, lots of content is dedicated to `modelling student trajectories' and the `benefits of CA tools', while only one article deals with the `student voice'. This is also useful as a measure of which themes you have already addressed comprehensively and which may deserve further consideration through further literature search, reading and assimilating.

\begin{figure}[htbp]
\centering
\includegraphics[width=\textwidth]{Figures/exampleConceptMap.pdf}
\caption{Example of constructing a concept map}
\label{fig:exampleConceptMap}
\end{figure}


\begin{question}[subtitle={Activity: Building your own concept map}] Create a concept map for your own project, in order to identify emerging themes.
\begin{guidance}
You should start from your concept matrix and apply the process outlined. There may be different themes, and groupings, emerging from your analysis, which you should explore in relation to the extent they help you characterise and justify your research problem.
\end{guidance}
\end{question}
%%Hack to correct tcbox behaviour
\color{black}

\paragraph{Writing summaries}
It is now time to start writing! Each theme you have identified should include enough concepts and references for you to start summarising what you have found in the literature under that theme. Later on, you will be using these summaries as building blocks towards writing up your literature review draft.
  
\begin{example}{Writing a summary for an identified theme}
Among the themes of our example, `modelling study pathways' is one of the most prominent. By considering the information we have collected in our summary-comparison matrix and concept map, we have produced the following summary:

\begin{quotation}
	Several authors have considered modelling and analysing students' learning trajectory through a programme of study, in order to understand how students progress or otherwise through their study and the learning outcomes they achieve in doing so. Such an understanding can then be used to inform scholarship and reflection around curriculum and its design, and inform possible changes. 
	
For instance, Dawson and Hubball (2014) deploy social network analysis techniques to identify and visualise the most common learning pathways followed by students within complex curriculum structures, in which many links may exist between the various curriculum components. They suggest that their proposed tool could  be used by curriculum practitioners to study student progression and completion across different pathways, and the extent students acquired the expected learning outcomes, although their study does not evaluate the extent that might be the case. 

Similarly, Salazar-Fernandez et al. (2021) use process mining to extract students' educational trajectories from historic data: in this case, their aim is to understand which trajectories are more likely to result in late dropouts. In their proof-of-concept tool evaluation over a specific data set, they achieve some positive results with the tool providing a strong indication that students taking a study break before resitting a failed module are most likely to drop out. Further research is needed to apply and evaluate their tool in other settings. 

Somewhat distinct from these studies is the work of McEneaney and Morsink (2022), who propose a simulation tool, based on Coloured Petri Nets, to be used as a design tool to help curriculum practitioners to test the possible effect on learning of envisaged curriculum changes, such as including or removing modules or study pathways in an existing programme. As another proof-of-concept tool, the work still requires wider application and evaluation. 

Finally, both Greer et al (2016) and Molinaro et al. (2016) focus on potentially useful visualisations for curriculum practitioners. In particular, Greer et al (2016) introduces the \enquote{Ribbon tool}, based on Sankey diagrams, for visualising student flows through academic programmes, with interactive capabilities which allow practitioners to study and compare specific student demographics.  The same tool is recommended by  Molinaro et al. (2016), alongside other visualisation tools, based on both students and course data, to allow practitioners explore both curriculum features and students' attainment. Both articles are part of the proceedings of the very first Curriculum Analytics workshop, in 2016, which also explain why they contain primarily poof-of-concept work and suggestions for future research.

Overall, this collection of articles contains some interesting ideas as to how curriculum and student data could be combined and analysed through the application and development of bespoke Curriculum Analytics tools. They all appear, however, quite preliminary studies, which is also an indication that this remains a young field of study where much research is still needed.
\end{quotation}

This is already a sizeable summary, which makes good use of the articles we have reviewed, appropriately cited in the text. Note how we have structured the narrative by including: an opening paragraph introducing the theme; separate paragraphs for the relevant articles we have reviewed, highlighted the specific problem and contribution made in each; a concluding paragraph with an overall assessment of the work reviewed. We have also presented together articles with similar contributions, and highlighted those which are quite distinct from the others.
\end{example}


\begin{question}[subtitle={Activity: Writing you own theme summaries}] Write up summaries for each of the common themes you have identified, making good use of the content of your summary-comparison matrix and concept map.
\begin{guidance}
As long as the entries in your summary-comparison matrix are sufficiently detailed and informative, you should be able to write your summaries without having to go back to the body of the articles, although be ready to do so on occasion. The concept map should tell you which articles to focus on for each of the themes.
\end{guidance}
\end{question}
%%Hack to correct tcbox behaviour
\color{black}

%\begin{question}[subtitle={Activity: Constructing your theme analysis matrix}] Starting from your current TIMatrix, select a theme which you think is relevant to your research and that you wish to investigate further.
%
%For all the articles in your TIMatrix addressing that theme, fill in an entry in the table: this will require you to go back to your notes on each article and possibly skim through the article again to extract relevant points.
%
%Do this activity for all the themes you wish to investigate further.
%
%\begin{guidance}In the first column, you should include a full citation\footnote{Your BMT might allow you to directly link to an article for easy retrieval of your notes, or the whole article.}.
%
%In the second column you should record the theme you are investigating.
%
%The third column gives you an opportunity to summarise the main ideas and contributions of each article in relation to the theme you are investigating.
%
%In the fourth column you should include your assessment of how each article relates to your own research and may inform your work, which also justifies why it should be included in your literature review.
%
%The fifth column should help you explore the unknowns, and the ways in which your research may address existing knowledge gaps.
%
%The sixth final column is for you to establish how ideas may relate to other themes, also helping you select which themes to consider next.
%
%This is a substantial activity which, depending on how much material you have, may take you several hours or a full day of work. It's also an activity that you'll have to loop back on, given that themes in one paper will overlap\footnote{Perhaps not completely{\ldots}} with those in others.
%\end{guidance}\end{question}
%%%Hack to correct tcbox behaviour
%\color{black}
%
%\begin{guidance}There are other tools that will help you, too, such as mind maps. Mind maps are graphical devices for linking ideas together and are very popular with some people. However, there is a learning curve to be climbed and, if you aren't already familiar with Mind Maps, you may like to stick with the simpler matrix version.
%\end{guidance}
%
%
%Once you have completed your table, you will have a number of references associated with each of the themes you have selected, alongside a set of key ideas, contributions and gaps. In doing so, you should have a grasp of existing research around your chosen topic, and this should help shape your own thinking on how your research may contribute new knowledge.
%
%\begin{question}[subtitle={Activity: Considering sources of information for your project}] Other than academic articles , list other sources of information which may be useful to inform your project. $<$ I don't think I'd be able to do this$>$
%
%\begin{guidance}You should also make notes of how you will use them to inform your research.
%\end{guidance}\end{question}
%%%Hack to correct tcbox behaviour
%\color{black}



\chapter{Ethics and regulations}\label{ch:EthicsAndRegulations}

All research must be carried out legally and ethically\footnote{People who carry out unethical and illegal research cannot share their results within the academic community without being called out. They cannot, therefore, be called researchers.}, so that it is essential you consider your proposed research and research design from these standpoints. 

Your responsibilities as a researcher, many of which we will consider in this section, are to:

\begin{itemize}
\item Behave with integrity, i.e., respect the rights of all participants in your research, be open and honest about how you have conducted your research and about your results, including not committing plagiarism\footnote{Briefly, plagiarism is the action of passing somebody else's work as your own. We will cover plagiarism in detail in Stage 2.}, ensure validity and accuracy in the collection and reporting of data, and disclose any conflict of interest, e.g. personal interests or relations with research participants which may compromise your judgement.

\item Comply with ethical codes and standards for research, laid down by your own university, and possibly professional bodies in your field of study.

\item Comply with legal requirements in relation to health and safety and the protection of personal data.

\item Guard against all forms of bias in your research.

\end{itemize}

%LR -- to see where this fits
%In addition, there may be further guidelines established by your own university or course of study. For instance, you may be prevented from conducting research in which participants who cannot provide fully informed consent, or which requires you to collect sensitive personal data or commercial purposes. There may be also circumstances in which you will need an explicit permission, for instance if you require your participants to discuss sensitive issues, or be subject to prolonged interviewing, testing or observation.
%
%\begin{question}[subtitle={Activity: Considering your university's ethical and legal guidelines}] Look up your university and course regulations to check any specific ethical and legal guidelines or constraints which apply to your project. Write a brief summary of what you have found out.
%
%\begin{solution}Our own institution, The Open University, UK, has an extensive set of ethics and legal policies and guidelines related to research, alongside processes to gain approval when dealing with human participants and personal data. In addition, our Masters courses put further restrictions on the kind of research which can be conducted, including not allowing research involving minors or vulnerable adults, or the collection of sensitive data.
%\end{solution}\end{question}
%%%Hack to correct tcbox behaviour
%\color{black}

\section{The rights of human participants in your research}\label{sect:rights}
In your project, you may call on other people to take part in your research, for instance people you intend to interview or observe, or who may complete questionnaires you design, or provide you with documentary evidence you require. These people have a number of rights you must respect, primarily:

\begin{itemize}
\item The right not to participate -- no-one should be pressured to take part in your research

\item The right to withdraw -- they can change their mind at any point

\item The right to give informed consent -- they should be given sufficient information on your research and their role in it for them to decide whether they wish to participate or not

\item The right to anonymity -- their identity should not be disclosed unless they give you explicit permission to do so

\item The right to confidentiality -- the data/information you obtain from them should be kept private if they ask you not to disclose it

\item The right to privacy -- you should not intrude unnecessarily into their lives

\item The right to protection from harm, i.e., you must take steps to minimise the risk of harm, either physical or psychological, to all participants.

\end{itemize}


\begin{question}[subtitle={Your university's guidelines for research with human participants}] By asking your supervisor, searching on your university's intranet, or otherwise, find out which guidelines apply in relation to human participants in your research.
\begin{solution}
Our university has a very comprehensive set of ethical guidelines and codes for research with human participants, covering expected behaviours and protocols, including the need for explicit approval from the university's ethical research committee to conduct the research, how to handle personal data and how to safeguard the health and safety of all participants.
\end{solution}\end{question}
%%Hack to correct tcbox behaviour
\color{black}


\section{Personal data in research}\label{sect:personalData}
In your research, you may wish to collect data about your human participants. The collection and use of this kind of data is usually regulated by law, although the specifics may change from country to country, and the university with which you're studying might add specific guidance too.

Within the European Union, the EU General Data Protection Regulation (GDPR) applies. %\footnote{At the time of writing{\ldots}} 
GDPR defines \textbf{personal data} as any information which may identify a living person, be that a name or a personal identification number, or a combination of physical characteristics, or cultural or social identities, and establishes rules for the use of such data.

It also establishes particular legal protection or safeguards for \textbf{sensitive personal data}, that is, data which may reveal:

\begin{itemize}
\item racial or ethnic origin

\item political, religious or philosophical beliefs

\item trade union membership

\item genetic or biometric data

\item physical or mental health

\item sex life or sexual orientation

\item criminal convictions and offences.

\end{itemize}

Depending on your university or course regulations, you may not be allowed to handle sensitive personal data in your research, and you are likely to need to follow strict protocols when handling other personal data. You may even have to apply for permission from an ethical committee to conduct the research you wish to do.

\begin{question}[subtitle={Activity}] By asking your supervisor, searching on your university's intranet, or otherwise, find out which guidelines you should follow in relation to personal data in research and whether you are allowed to handle sensitive personal data in your project.
\begin{solution}
Our university upholds GDPR regulations, with which all research must comply. The use of personal data in a project must be declared and permission to proceed obtained from and academic board. At Masters level, research students are not allowed to make use of sensitive personal data in their projects.
\end{solution}\end{question}
%%Hack to correct tcbox behaviour
\color{black}

\paragraph{Processing personal data}
This refers to any action involving personal data, including obtaining, recording, analysing, and/or destroying data from which a living individual can be identified.

The GDPR sets out six principles\footnote{Why not set a reminder in your diary to revisit these principles whenever your research design changes.\color{red} difficult to know how to do this} that must be observed when processing personal data. 

The first four principles refer to the collection and intended use of personal data. They are:

\begin{itemize}
\item Personal data processing must be lawful and fair
\item The purposes of personal data processing must be specified, explicit and legitimate
\item Personal data collected must be adequate, relevant and not excessive
\item Personal data must be accurate and kept up to date
\end{itemize}

\begin{question}[subtitle={GDPR principles for processing personal data}] 
Consider these four principles. Write down how you would ensure these principles apply when collecting personal data from your participants.
\begin{solution}
This is what we recommend our students do:
\begin{itemize}
\item Explain to your participants the reason why you are collecting the data as part of your Masters project, and the use you will make of them, including giving assurance that the data  will only be used for academic research and be kept confidential otherwise
\item be specific as to which data you will collect and for which purpose
\item explain how you will ensure the data collected will remain accurate
\end{itemize}
\end{solution}
\end{question}
%%Hack to correct tcbox behaviour
\color{black}

The two remaining principles concern the way personal data are stored and processed. They are:

\begin{itemize}
\item Personal data must not be kept for longer than is necessary
\item Personal data must be processed in a secure manner
\end{itemize}

\begin{question}[subtitle={GDPR principles for processing personal data, cont'd}] 
Consider the two remaining principles. Write down how you would ensure these principles apply when storing and processing personal data you have collected from your participants.
\begin{solution}
This is what we recommend our students do:
\begin{itemize}
\item Although GDPR does not set an explicit limit, personal data should not be kept much longer than the duration of your project, and submission and examination of your dissertation. If you wish to retain some of the data for longer, you should anonymise\footnote{To anonymise personal data means to remove any information which may lead to identify a living person. We will cover this topic in Stage 4.} them, so that personal information cannot be leaked accidentally
\item Store your data securely to prevent unauthorised access, accidental disclosure or corruption or loss of data. For physical data, this may require locking them in a secure, locked cabinet. For digital data, you could make use of encryption and secure repositories, ensuring appropriate back-ups of all digital files.
\item If using online storage services for digital data, you must also ensure they comply with GDPR: you should avoid repositories kept in countries outside the EU which have weaker regulations and legal protection.
\end{itemize}
\end{solution}
\end{question}
%%Hack to correct tcbox behaviour
\color{black} 

Even if you think GDPR will not apply to your research, we would still encourage you to apply those principles in your project, to guard against an improper use of personal data.


\section{Equity, Diversity and Inclusion in research}\label{sect:Equity}
Equity, diversity, and inclusion (EDI for short) are important ethical considerations in research. Not only is EDI important for society, but good research EDI leads to better research as your and others' biases can be compensated for.

In the UK, national EDI definitions and principles have been established across the academic sector by UK Research and Innovation (UKRI), which is a government body that brings together all UK research councils responsible for supporting research and knowledge exchange in UK higher education institutions. In your own country, similar guidance may also exist.

As stated by the UKRI EDI guidance\footnote{https://www.ukri.org/manage-your-award/good-research-resource-hub/guidance-for-equality-diversity-and-inclusion/}, ``research and innovation should be `by everyone, for everyone' -- a dynamic, diverse and inclusive research and innovation system in the UK is an integral part of society and should give everyone the opportunity to participate and to benefit.'' And also that \enquote{By valuing all, we recognise that a diversity of ideas, opinions, knowledge and people enriches our work and enlarges our knowledge economy.}

Equity\footnote{It is worth noting that both equity and equality are often used in the literature as the `E' in EDI. There is, however, a difference: equality refers to treating everybody in the same way, while equity acknowledges specific barriers or obstacles which affect certain individuals or groups of individuals, and seeks to remove them. The latter is now considered the better definition of the two.}, Diversity and Inclusion are as follows:

\begin{itemize}
\item Equity relates to fairness and justice, in the sense of removing barriers or bias\footnote{We introduce bias in Cref{sect:biasInResearch}.} which may prevent individuals, or groups of individuals, from having equality of access, opportunity or outcomes.

\item Diversity refers to the full spectrum of differences and similarities between individuals, whether socio-demographic, such as age, gender, race, ethnicity, etc., or in terms of beliefs and values, life experiences or personal preferences.

\item Inclusion concerns ensuring that all individuals feel welcome, valued and confident to be treated fairly and respectfully. Inclusion is often paraphrased as `diversity becoming normal'.
\end{itemize}

Your university is likely to have policies that guide research towards good EDI, with which you should become familiar.

\begin{question}[subtitle={Your university's EDI guidelines}] By asking your supervisor, searching on your university's intranet, or otherwise, find out which EDI guidelines you should follow in your project.
\begin{solution}
Our university complies with the UKRI EDI guidance mentioned above. In fact, it has a broad approach to EDI, which requires the embedding of EDI principles within its whole operation, including research, teaching, and the hiring and professional development of all staff.
%LR: I couldn't find anything specific to research at the OU. I've asked around, but this should be ok if nothing else comes up.
\end{solution}\end{question}
%%Hack to correct tcbox behaviour
\color{black}

It is important to embed EDI principles within all research projects. Failing to do so may lead to studies whose results are not applicable, or are even detrimental, to particular groups of people. For instance, medical trials in which some groups are under-represented may lead to treatments which are not effective for those groups. Clearly, a medical trial is beyond the scope of Masters research! However, taking an EDI stand can also inform your project, as we discuss in the next activity.

\begin{question}[subtitle={Activity: Considering EDI in research activities}] Write down how taking an EDI lens may influence the following research activities:

\begin{itemize}
\item Framing your research problem
\item Inviting people to take part in your research project
\item Collecting data to use in your research
\item Designing a novel artefact as a main deliverable of your research
\end{itemize}

Write down your answers.

\begin{solution}
Your answer may include some of the following:
\begin{itemize}
\item When framing your research problem, you should consult with a diversity of stakeholders and ensure you take differing views into consideration. This will help you validate that the problem is felt across a wider community of stakeholders who my be affected by your research, and will also mitigate any personal bias or pre-conception you may have.
\item When inviting research participants, you should consider their diversity to ensure they form a representative sample of the population under study, so that your results are more likely to be generalisable to that population.
\item When collecting data to use in your research you should ensure they are not biased against specific groups\footnote{This is currently a hot topic given the recent rise of Artificial Intelligence and Machine Learning algorithms, which have started to replace human decision making: bias in the data often leads to unfavourable outcomes for those groups.}, for instance, by gender or race, which could lead to results which are not applicable or even unfair for those groups. 
\item When designing a novel artefact, you should take end-users diversity into account, so not to disadvantage some groups of individuals, for instance in terms of accessibility, which could lead to less equity.
\end{itemize}
\end{solution}
\end{question}
%%Hack to correct tcbox behaviour
\color{black}



\section{Research involving animals}\label{sect:AnimalResearch}
%to do
Although it is rarely the case at Masters level, research may involve animals, something which is regulated in many countries. For instance, the EU has issued a Directive `On the Protection of Animals Used for Scientific Purposes\footnote{https://eur-lex.europa.eu/legal-content/EN/TXT/?uri=celex:32010L0063}', while in the UK all research involving animals must comply with the Animal Welfare Act 2006\footnote{www.legislation.gov.uk/ukpga/2006/45} and the Wildlife and Countryside Act 1981\footnote{www.legislation.gov.uk/ukpga/1981/69}.

As argued by Mancini and Nannoni (2022), %proper citation needed
the ethical dilemma about involving animals in research stems from \enquote{a recognition that they are capable of suffering while being incapable of consenting to procedures that can harm them,} which motivated two researchers, Russell and Burch, back in 1959, to propose three core principles for humane animal research. They are known as the 3R principles and concern:

\begin{itemize}
	\item the \textit{replacement} of the use of animals with alternative techniques, possibly avoiding using animals altogether
	\item the \textit{reduction} of the number of animals used to a minimum, by deploying ways to obtain the same or more information from fewer animals
	\item the \textit{refinement} of experimental procedures to limit animal suffering, including improving animal welfare and minimising pain.
\end{itemize}

Your university is likely to have specific guidelines on conducting research involving animals, which you should take into consideration.

\begin{question}[subtitle={Your university's guidelines on research involving animals}] By asking your supervisor, searching on your university's intranet, or otherwise, find out which guidelines apply for research involving animals. In particular, you should check whether you are allowed to conduct this kind of research for your Masters project.
\begin{solution}
At our university, all research involving animals is tightly regulated: all proposals are subject to rigorous ethical scrutiny and it necessary to obtain a special licence to practice it. Masters research students are not allowed to conduct research involving animals.
\end{solution}\end{question}
%%Hack to correct tcbox behaviour
\color{black}


\section{Intellectual property}\label{sect:IntellectualProperty}


Intellectual property (IP) is the owned property created through intellect, such as an invention, an artwork, a formula for some chemical compounds, etc. It is worth noticing that in this definition\footnote{If you find this confusing, think about a house or a car you may own. In legal terms, you have the property of that house or car, which gives you certain rights in law. However, in common language you may refer to your house or car as your own property}. \enquote{property} has the legal meaning of a collection of rights applying to the things which have been created, rather than the things themselves The two meanings are often confused in the wider literature on intellectual property: in this book we will use the term IP to refer to the system of rights.

Some IP gives rise to specific rights which are protected by law\footnote{For instance, the UK Intellectual Property Act 2014 (\url{https://www.legislation.gov.uk/ukpga/2014/18/contents/enacted}) defines a number of IP rights, establishing whom they belong to and what they allow the owner to do.}.

\begin{question}[subtitle={Activity: Types of IP rights}]
Conduct a web search to identify way of protecting intellectual property. For each, write down a definition and comment on whether they may arise from academic research.

\begin{solution}
Your answer may include the following:

\begin{itemize}
\item Copyrights --- A copyright is an IP right that applies to any original work that is expressed in a tangible form, say a poem or a painting. It gives legal right to its creator(s) to reproduce, publish and distribute the IP, and to transfer it to a new owner. A creator acquires copyright automatically. A research dissertation, and any part within, gives rise to copyrights.

\item Patents --- A patent is an IP that prevents anybody by the owner from making, using, or selling an invention for a limited period of time. Patents are usually associated with products and processes, e.g., a microchip in a mobile device, or the process to engineer bacteria that digest plastic. For the owner's rights to be protected in law, a patent must be registered with a patent office, and different regulations apply in different counties. Academic research can generate patents, but the protection is not automatic: it requires appropriate registration to take effect.

\item Trademarks --- A trademark applies to something which can be used to distinguish commercial products and services of a trader from all others within the same market, e.g., the `apple' used on all products by Apple inc. A trademark must be registered, but there is no time limit to the protection it provides. Academic research may lead to new products or services which could be protected by registered trademarks.

\item Trade secrets --- A trade secret is something which has economic value to a business because it is not generally known or easily discoverable by observation and for which efforts have been made to maintain secrecy. In other worlds, a trade secret is confidential business information which provides some competitive edge: the famous CocaCola formula is a trade secret. Trade secrets are used instead of patents when the owner doesn't wish to disclose the information publicly, something required by a patent. Academic research may lead to trade secrets, under certain circumstances, for instance if the research is entirely financed by industry and protected by a non-disclosure agreement.
\end{itemize}

This is not an exhaustive list and different countries' legislation may include different forms of IP. For instance, the UK IP Act includes `design' as a form of IP, which relates to the appearance, shape or configuration of a product, rather than the product itself (as the case for patents). 
\end{solution}\end{question}
%%Hack to correct tcbox behaviour
\color{black}

Your university is likely to have an IP policy which establishes, among others, which IP applies in the context of academic research and Masters studies, defining both ownership and related rights.

\begin{question}[subtitle={Activity: Understanding your university's IP policy}] Using your university's intranet look up any IP policy your university has. Write a brief summary of what you have found out in relation to possible rights which apply to your Masters project work.

\begin{solution} At our university, the IP policy establishes rights in relation to the ownership, development, protection and exploitation of IP arising from all types of research and scholarship carried out in the institution. Although the policy changes depending on the type of invention, in relation to Masters research projects which are part of a taught course of studies, our university assigns the IP to the student, as long as they have paid their university fees. However, in case of fee waivers or bursary from the university, the IP will belong to the university.
\end{solution}\end{question}
%%Hack to correct tcbox behaviour
\color{black}

\section{Use of generative AI in research}\label{sect:GenAI}

Generative Artificial Intelligence (GenAI) is an emerging digital technology able to generate new content automatically in response to prompts from the user. Its current capabilities include the automatic generation of text, images, videos, music or computer code.

It relies on algorithms trained on vast amounts of digital data, from which they learn recurrent patterns and structures, which they then apply to generate original content. For instance, a GenAI system trained on English text will learn patterns related to the English grammar alongside structures and expressions commonly used in writing, and frequent combinations and sequences of words and phrases. Such patterns and structures are then used to generate original English text in response to a user's query: often such responses are indistinguishable from those of a human English writer. 

Learning and originality have a specific meaning in GenAI. Learning is the process followed by the algorithm to build an abstract model from the data used in training, which can then be used to generate content similar to those data, be that text, images, sounds, or other. Originality refers to the fact that the particular content generated is not copied from an existing repository, but constructed by the algorithm based on its abstract model combined with the vast repository of information it has access to: as a result no two answers to the same user's query are exactly the same.

\begin{question}[subtitle={Activity: Investigating GenAI systems}] 
Conduct a web search on GenAI systems. List the ones you have found and indicate their use.
\begin{solution} 
At the time of writing, these are among the most used GenAI systems:
\begin{itemize}
	\item ChatGPT and Bard: these are conversational systems in that they allow users to conduct conversations in written natural language. They can be used to answer specific questions, for language translation and even creative writing.
	\item DALL E: this is an image and art generation system. It can be used to produce images from a text description, or to edit or enhance existing images.
	\item AIVA: this a system to generate music. It can generate new songs in a great variety of styles and genres.
	\item  Copilot: this is a system to generate code in a variety of programming languages. It can assist a software developer by suggesting new code, completing code being written or generating documentation. 
\end{itemize}
\end{solution}\end{question}
%%Hack to correct tcbox behaviour
\color{black}

Impressive as it may be, GenAI has a number of pitfalls you should be aware of. Firstly, the content generated depends heavily on the data set used to train the algorithm, so that any intrinsic bias in the data set will also be present in the new content generated. For instance, if discriminatory language is present in the training data, the same will be the case in the content generated by the algorithm.

Secondly, the technology is designed to fill in any gap there may be in their knowledge base with plausible, but made-up information, so that what is generated may not be true or reliable. The term `hallucination' is used to denote such behaviour, although fabrication would probably be a more accurate term! So, for instance, if you ask the AI to provide a summary of a particular topic including a list of references on which the summary is based,  more often than not, those references will include completely made-up entries, although constructed in a way which makes them look perfectly legitimate (by applying the pattern learnt for references from academic papers used in the training process). Hallucination means that you can't necessarily rely on the veracity of the content generated, and any significant use of it in your research is only possibly after you have double checked both accuracy and authenticity, and, ideally where it comes from. 

Another important consideration is that not being copied from somewhere else doesn't mean novelty: although originally expressed, the content is still pieced together from what is already known, so there is no new knowledge contribution or creative invention. Equally, the algorithm has no understanding or critical insight of what is generated: it is just very clever at putting the puzzle together from existing pieces and rules learnt. As a result, you should not expect deep or critical accounts of a topic of interest, but you may still obtain good factual summaries or descriptions.

Even with these drawbacks, GenAI can be a useful tool in research, particular to get you started on specific tasks, like gathering some initial information on a particular topic, identifying some initial articles to read, or even summarising certain literature, as long as you then put your own effort and work to generate your own content which is both authentic, unbiased and creative. 

A final consideration is that the content generated by the AI is not your own work, so it should be treated as any other secondary source you use, and clearly identified and distinguished from your work, to avoid any accusation of plagiarism. In fact, you should be open and transparent in your use of GenAI in all aspects of your research.

GenAI is a fascinating subject, which is generating, as it may be expected, much controversy. At the extremes are those who focus solely on the risks or the opportunities it affords, often overestimating what the technology can actually do. The state-of-the-art is somewhere in the middle: yes, there are both risks and opportunities, but as often with technology, the issue is what people choose to do with it. 

Because of this, organisations, and particularly academic institutions, are developing their own rules and regulations on the use of GenAI by staff and students. If you are considering using GenAI in your project, you should become familiar with your university' position on what you can and can't do

\begin{question}[subtitle={Activity: Understanding your university's guidelines on GenAI}] Using your university's intranet look up any policy or guideline related to the use of generative AI. Write a brief summary of what you have found out which may be relevant to your Masters project work.
\begin{solution} 
Our university explicitly states some admissible and not admissible uses of GenAI. In particular, it considers inadmissible, and a form of plagiarism, to copy and paste content generated by an AI system without proper attribution. On the other hand, admissible uses include:
\begin{itemize}
	\item to make materials more accessible, for instance by auto-generating transcripts of audio or image captions,
	\item to help with spelling and grammar in written text, and generally to help you improve your writing skills,
	\item to prompt you to develop new ideas,
	\item to revise basic knowledge on a particular topic.
\end{itemize}
\end{solution}\end{question}
%%Hack to correct tcbox behaviour
\color{black}




\section{Bias in research}\label{sect:biasInResearch}

Bias is a tendency to support or oppose particular ideas or things based on personal preferences rather than objective evidence. Cognitive bias, in particular, prevents people from processing information objectively due to limited processing capabilities of our mind or due to emotional responses or social norms and conditioning. 

Bias is problematic when it leads to the unfair treatment of others, particularly as a result of pre-conceptions or prejudice.  Bias is subtle. We are all affected by it. It is easy to see in others, but extremely difficult to see in oneself.


 Bias can damage research findings, leading to outcomes which are unreliable, and casting doubt on the claimed knowledge contribution. To be a good researcher you should assume that you are affected by many different forms of bias and work to eliminate them from your work. The upside is that you will be a better researcher for doing this.
 
 Several forms of bias have been recognised in research: you can do some preliminary work on identifying general biases that might affect you by looking on the web.

\begin{question}[subtitle={Activity: Bias in research}]  Conduct a web search on types of bias affecting research and ways to mitigate them. List and summarise the main kinds you find, and their counter-measures. Make sure you include bias which relate to the use of data\footnote{Data bias is becoming more and more topical with the rise of AI algorithms which may replace human decision makers.}.

\begin{solution}Among others, you may have encountered the following:

\begin{itemize}
\item Confirmation bias, which is the tendency to favour the selection, analysis and interpretation of data which support the researcher's prior beliefs. A way to counteract it is to consider your own beliefs and actively look for reasons why you might be wrong. Another way is to ask colleagues and/or your peers to review your work, including your arguments and results. 

\item Observation bias\footnote{Observation bias is also known as the Hawthorne effect. You can read about the original Hawthorne experiment here: \textbackslash{}url\{\href{https://en.wikipedia.org/wiki/Hawthorne_effect}{https://en.wikipedia.org/wiki/Hawthorne\_effect}\}}, which is the tendency of participants being observed during a study to change their behaviour, possibly to please the researcher or provide the answers they think the research is seeking. A way to counteract it is to complement your observation with other kinds of data collection and compare the outcomes -- this is called triangulation.

\item Selection bias, which occurs when data is selected subjectively, leading to samples which are not representative of the population under study. A common way to avoid selection bias is randomisation.

\item Recall bias, which is the tendency of people to recall certain types of events more vividly than others, and can affect the outcomes of research which relies on participants' memories of the past. Reducing the recall period, for instance by interviewing participant soon after the fact, or  asking then to keep detailed diaries of their experience are ways to mitigating it. 

\item Algorithmic bias, particularly in relation to AI applications, which occurs when an algorithm produces outcomes systematically and repeatably which disadvantage one group of individuals over others. Counteracting algorithmic bias effectively is a hot research topic, and much remains unknown. Involving domain experts alongside data scientists in the detailed consideration of both the data used by an algorithm and its outputs is one approach, alongside ensuring that the working of the algorithm are transparent to its users, and its behaviour can be explained.
\end{itemize}
\end{solution}\end{question}
%%Hack to correct tcbox behaviour
\color{black}

All bias can damage your research. However, confirmation bias is very easy to fall into, can be very disruptive, and can arise when you have too personal a stake in the research beyond the project itself. For example, you may be the manager of a process you are seeking to improve. This means that you will have a stake in the research subject beyond meeting your Masters academic requirements. Although this can be a strength, in that, since you already have an interest in, and knowledge of, the research subject and context, you are likely to have insights into the causes of problems and the factors that have an impact on them, it can also be a weakness in that it may lead you to focus on evidence that confirms any existing beliefs that you might have about how the process should improve, while dismissing evidence that does not support them\footnote{In the worst case, you might find that your current approach to improvement is wrong which might be visible to others within your organisation. If you find yourself in this situation and, in our experience, it does happen, it can be very hard to deal with this sort of bias.}. This may result in a lack of objectivity, and a tendency to make subjective judgements instead of an evaluation based on sound evidence.

Another effect of confirmation bias is that you clearly see a problem where no-one in your readership does. So much so that you don't feel you have to explain the problem or convince others that it exists.

In our experience, both of these can hamper progress. The latter is especially pernicious: that you see a problem is great, but can you find sufficient evidence in the academic literature?

We will return to bias and how to mitigate it in the context of research design in Stages 3 and 4.

\chapter{Managing risk in Stage 1}\label{sect:stage1ManagingRisk}\label{ch:ManagingRiskInStage1}
``What could possibly go wrong in a flourishing concern like the Brixleigh Bank\footnote{Later in the book, Brixleigh's Bank is shown to be a Ponsi scheme -- a gigantic fraud -- which fails.}? My dear, you are too fond of conjuring up imaginary evils.'' -- Matilda Mary Pollard's \emph{Cora: Three Years of a Girl's Life}, 1882

Pollard's imaginary evils are now called risks. They are no longer seen as imaginary evils, but as things that need managing.

Risk captures the likelihood of something going wrong combined with the impact that will have on your project, both on time, resources and outcome. In theory, both positive and negative impacts should be considered, but very often the focus is on what can affect your project in a bad way, letting the good stuff roll.

The management of risk is an important discipline in its own right --  you do not need all of the tools that that discipline offers.

So, in analysing risk for your project, you should focus on the following dimensions:

\begin{itemize}
\item Specific risks: What sort of things can go wrong? For example, you may not be able to recruit sufficient respondents to a survey to gain direct access to key evidence.

\item Impact: What are the consequences if things do go badly? How severe might those be? For instance, not obtaining key evidence will invalidate your whole research, so this would be very severe in terms of your project outcome.

\item Likelihood\footnote{Some approaches to risk ignore likelihood, assuming Murphy's Law, i.e., that anything that can go wrong will go wrong.}: How likely is it that things will go wrong?

\item Mitigation/contingency: What can you do to reduce likelihood or impact? For example, you may have lined up a secondary source of evidence, which may not be as authoritative or useful as what you had in mind originally, but would be easier to access and still allow you to derive some interesting results.

\end{itemize}


\section{Research project risk}\label{sect:ResearchProjectRisk}
In a research project there will be risk which is very specific to what you intend to do, but there are also risk categories which are common to all projects, which we consider in this section.

\subsection{Technical skills}\label{ssect:TechnicalSkills}
Your intended project may require you to apply expert technical skills, for instance, coding or advanced statistical analysis. Early on in your research project, it may  be possible that the details of precisely which technical skills you will need are unclear. Nevertheless, you should start thinking about technical skills you may need and what might happen if they are not sufficiently developed.

For instance, will conducting a survey require sophisticated statistical skills? Your risk analysis should recognise this by considering:

\begin{itemize}
	\item Specific risks: whether your current statistical skills level may be insufficient;
	\item Impact: whether a lack of appropriate skills might mean that you are not able to analyse your data as well as you would like, losing their value;
	\item Likelihood: how likely it is that skills that you don't possess will be needed;
	\item Mitigation/contingency: to reduce the impact (to manage the risk), which course of action you could take to help you enhance your skills to the level you will need. This is an item that you can include in your project plan.
\end{itemize}


\begin{question}[subtitle={Activity: Risk in relation to technical skills}] Consider whether there are bespoke technical skills which are essential to your intended research. Perform a risk analysis in relation to whether your possess those skills and write down the outcome.

\begin{guidance}
As shown in the example, you should record the specific risks, their likelihood, impact, and any mitigation/contingency.

On mitigation: if you don't think you can develop all necessary technical skills in good time, then you should consider whether you have made the right choice of project -- this is also a way of managing risk! At Masters level, it is a lot safer to focus on research which makes good use of  technical skills you already possess and employ your limited time wisely to conduct your research.
\end{guidance}\end{question}
%%Hack to correct tcbox behaviour
\color{black}

\subsection{Study time}\label{ssect:StudyTime}
Consider the time which is required for your project --- you may have to go back to your estimates from Cref{ch:framework}. To succeed in your Masters project it is essential you sustain a continuous effort throughout, with little scope for making time up when you are not able to, or for taking long breaks. The key question, therefore is whether any other commitments, professional or personal, may get in the way and what you could do about it.

\begin{question}[subtitle={Activity: Risk in relation to study time }] Consider your current personal and professional commitments, and study practices. Perform a risk analysis on whether you will be able to dedicate sufficient time to your project on a regular basis and write down the outcome.

\begin{guidance}
You should focus on the extent your current study practices are appropriate for your project, or whether there may me changes in your professional or personal circumstances which may have a negative impact on the time you will have for your project. 

Under mitigation/contingency, you should include any adjustments you may need to make in your life and your studies. If these are substantial, however,  you will need to assess very carefully how feasible it is for you to make them. 
\end{guidance}\end{question}
%%Hack to correct tcbox behaviour
\color{black}

\subsection{Resources}\label{ssect:Resources}
Depending on your chosen research problem and aim, you may need access to participants, organisational information, third-party data, industrial case studies, etc., or you may need to acquire specialised software or hardware.

It is important for you to assess how likely it is that you will be able to gain access to or acquire such resources and, should there be any cost involved, whether you can afford it.

If conducting research with your current employer, you should also consider the extent the data or information you require from them are confidential and non disclosable, as well as the possibility of changing jobs, and the extent you may be able to retain access or make alternative arrangements if that's the case.

\begin{question}[subtitle={Activity: Risk in relation to resources }] Consider resources you are likely to need to conduct your research. Perform a risk assessment in relation to your access to those resources for the duration of your project and write down the outcome.

\begin{guidance}
You consider both access to those resources, the extent circumstances may change which might prevent you from accessing them at any point in your project, and possible alternatives. 

If you don't think you will be able to guarantee access to all necessary resources or appropriate alternatives, then you should consider refocusing your project, so that you can make best use of resources you already have or will find easier to access.
\end{guidance}\end{question}
%%Hack to correct tcbox behaviour
\color{black}

\subsection{Ethics and regulations}\todo{Check overlap with previous chapter}\label{ssect:EthicsAndRegulations}
Under research design in Cref{sect:stage1ResearchDesign}, you have encountered a wide range of ethical and regulatory issues which may be pertinent to your research project. Here you are asked to consider any related risk.

\begin{question}[subtitle={Activity: Risk in relation to ethics and regulations}] Consider the ethical and regulatory issues from Cref{sect:stage1ResearchDesign}, and write down those which are most pertinent to your project. For each, perform a risk analysis and write down the outcome.


\begin{guidance} You should pay particular attention to your university's regulations in regard to those topics, in particular if your project is likely to involve human participants and the handling of personal data, and to assess risk around possible bias.

It is also essential you identify any regulations which may limit what you can do in your project, for instance working with animals, and you should double-check with your supervisor that what you are proposing is acceptable.
\end{guidance}\end{question}
%%Hack to correct tcbox behaviour
\color{black}

\section{Summarising your project risk}\label{sect:SummarisingYourProject}
It is time to summarise the outcome of your risk analysis. To do so, you could use a risk table similar to Cref{tab:riskTable}. Note that the table encourages you to consider also risk specific to your project which may not fall within the categories we have already considered.

\begin{SimpleNColTable}{tab:riskTable}{5}{Risk table}[X[3]X[1]X[1]X[1]X[1]]
\\
 Technical skills & & & &  \\
 Study time & & & &  \\
 Resources & & & & \\
 Ethics and regulations & & & &  \\
 Other (specific to your project) & & & & \\
\end{SimpleNColTable}


\begin{question}[subtitle={Activity: Risk assessment for your project }] Complete your project risk assessment by filling in the entries in Cref{tab:riskTable}. 

\begin{guidance}
You should already have most of the required content from your previous activities.

You should ensure you consider carefully any specific risk to your project which is not covered by the generic risk categories we have included in the table.
\end{guidance}\end{question}
%%Hack to correct tcbox behaviour
\color{black}

\chapter{Reflecting}\label{sect:stage1Reflection}\label{ch:Reflecting}
You might be thinking: ``Why, if I've finished a task, should I cause myself grief by reflecting on it -- in the best case nothing will change. In the worst case, I'll have to do it again.''

We feel exactly the same -- reflecting on what you've written causes a lot of grief, especially when you realise that it's not as good as it could have been\footnote{We`ve reflected on the materials for this book many many times{\ldots} That's meant redrafting, adding extra and -- hurtfully --  having to remove stuff that just wasn`t good enough. It's changed --  for the better we hope -- because of it. You are free to disagree, and we'd welcome your reflection on it too.} and there are very good reasons to improve it.

At the same time, we've learned a lot from that reflection. For instance, we've been able to split up complex issues into more digestible chunks; we've identified new links between topics that we hadn't thought about before; we've read more about what the students we have worked with to understand more precisely their contributions; we've taken apart the materials we have written to ensure their validity in the Master's research context; and, last but not least, we've been able to generalise much of what we know of this topic to be more applicable across the board, while at the same time realising that there are special topics that will affect only a small part of our readership.

We feel exactly the same -- reflecting on what you've written causes a lot of grief, especially when you realise that it's not as good as it could have been\footnote{We've reflected on the materials for this book many many times{\ldots} that's meant redrafting, adding extra and -- hurtfully --  having to remove stuff that just wasn't good enough. It's changed --  for the better we hope -- because of it. You are free to disagree, and we'd welcome your reflection on it too.} and there are very good reasons to improve it.

Reflection is also the thinking of radical thoughts. There is a context into which all research fits that colours the knowledge that is its primary contribution, making it less valuable, less distinct, less out there. That context is a hegemony of received wisdom, of `common sense thinking', of uncritical investigation, none of which are necessarily knowledge. Reflection --  ''the voluntary disobedience of thought and reasoned undocility'' according to Foucault (1985) -- is your way of breaking that mould, of stepping out of conventionalism and of shaking up the world.


\begin{question}[subtitle={Reflecting on your learning and practice}]
In this activity you are asked to stand back and reflect deeply on what you have leant and done, the wider context of your work and your own attitude to it. Specifically, you are asked to think deeply about each of the following:

\begin{itemize}
	\item Think about your study this far -- using this book and anything you've done for your project in parallel -- as a journey. How has your view of academic research changed from when you started? In which ways has it help you think as a researcher? Which things have surprised you or challenged your initial beliefs? How have your original research goals changed as a result? 
	\item Think about the way you work. Are you tidy and systematic, or let things happen organically? For instance, how does your desktop (physical and digital) look? Think about how your way of working affects the way you do your research. Are you systematically keeping your references, notes and summaries in your BMT? How successful were you at reviewing the literature systematically and extracting relevant themes for your research? Has your approach to work changed since you started your project?
	\item Think about the context of your research. What motivated you to do your project? Are there pressure, professional or otherwise, on you to succeed? Professional pressures could come in many forms: financial -- there's a promotion for you at the end of it; peer -- your colleagues know that you are studying will have good expectations of your result and you'll want to prove them right\footnote{Or wrong, depending on the colleague!}. Are you sponsored by your employer? Will you be able to report a negative outcomes to your research, for instance, that there is no solution to your problem using the current technology stack? A negative result is a very good research outcome, even if it tends to satisfy fewer non-academics than a positive result. Which family pressures do you feel? It's not unusual that you will have given up a paying role to study, moving the responsibility to provide onto another member of your family, or given up aspects of your family life to make space for your study. What are their expectations? How are you standing up to those pressure?
	\item Think about your feelings about your project. What's that thought nagging at the back of your mind? Is it \enquote{How will I start?} Or \enquote{Will I be able to dedicate enough time to this?} Or \enquote{Can I really do this?}. Or ''Is ``shouldn't I be bringing in a wage rather than studying?'' You may be one of the lucky ones that doesn't have such negative thoughts, but negative thoughts are a very natural part of steps into the unknown. And research is precisely that, a step into the unknown. Being aware of the doubts you naturally have, will help you manage them. Think of how to make visible and celebrate even the tiniest of steps forward in your research. Think of tools that may help you. For instance, if you have concerns about managing your time, start using one of the many tools out there that break time up into manageable units and help manage it for you. If your concerns are about how to organise your thoughts, look into mind maps, lists, todo lists.
\end{itemize}

\begin{guidance}
This activity has four parts, each prompting you to reflect on different aspects of your thinking and practice: you should give proper consideration to each of them, as together they will help you reflect deeply on your experience overall.

Thinking early and often through reflection is a powerful way of doing better. Do it well and you will be a better researcher, and your final report will be better than you will have expected.

It's worth saying that, at the end of what could be an exhausting journey, you will not fully appreciate your achievements. That realisation may have to wait until you are rested, graduated, or some distant time later. But it will come.
\end{guidance}\end{question}
%%Hack to correct tcbox behaviour
\color{black}

\chapter{Reporting}\label{ch:Reporting}

It may not feel like it, but you're now ready to write a substantial contribution to your research project: your full research proposal.

We recommend that you write a report at the end of each stage of our framework to consolidate the work you have carried out, regardless of whether your course may require to do so. Writing such reports will help you develop your dissertation incrementally, and provide good practice to improve your academic writing skills as you go along.

\section{Putting your research proposal together}\label{sect:PuttingYourResearch}

Here, in Stage 1, your report will consist of your full research proposal, which we recommend you structure as indicated in Cref{tab:researchProposalStructure}\footnote{Unless your course requires a different structure, which you should apply instead.} --- subsequent reports will build on this structure by adding further elements. Although not all what you write here will end up in your actual dissertation, substantial parts of it will. You're definitely started now and that has to feel good!

\begin{SimpleNColTable}{tab:researchProposalStructure}{2}{Recommended structure of your research proposal}[X[-1]X[4]]
Report template & Guidance\\
 Proposed title & Your title should capture succinctly your research problem and aim. You should refer back to Cref{sect:title} for more guidance.\\
 {Sect 1 – Introduction \\1.1 Background to the research \\1.2 Justification for the research \\1.3 Fitness of the research} & This section should provide an introduction to your research topic in its wider context (as background) and your justification of why the research is worth pursuing. Its purpose is to introduce and justify your intended research in overview, before entering the detailed work of the subsequent sections. It should be well argued and supported by appropriate citations. In this section, you should also argue how the research fits within the scope of your qualification, and meets any other personal, professional or organisational criteria. Revising Cref{sect:stage1ChoosingATopic} should help you with this task. \\
 {Sect 2 – Literature review \\2.1 Draft review of relevant knowledge \\2.2 Planned further literature review }& This section should be based on the theme summaries you constructed towards the end of Cref{sect:stage1LiteratureReview}. You will build on this content in Stage 2 to write an extensive, well-argued literature review to demonstrate your in-depth engagement with the academic (and other) relevant literature. Your planned review should identify further reading you may still have to undertake in Stage 2, although the expectation is that the bulk of your reading has taken place in Stage 1, so that you can focus on synthesising your knowledge and understanding in the next stage.\\
 {Sect 3 – Research definition \\3.1 Problem statement \\3.2 Aim and objectives \\3.3 Knowledge contribution} & You should ensure that your research problem is well articulated, that your aim and objectives are consistent with the research problem, and that the intended knowledge contribution of your research is clearly argued. You should refer back to the activities you conducted in Sections Cref{sect:stage1ResearchProblem} and \Cref{sect:stage1AimAndObjectives}, but also to the theme summaries you produced at the end of Cref{sect:stage1LiteratureReview}, particularly those which highlight knowledge gaps of interest. \\
 {Sect 4 – Research design \\4.1 Evidence and data \\4.2 Research methods \\4.3 Ethics and regulations} & This section should demonstrate your initial engagement with research design, particularly that you have thought about the kind of evidence and methods you may need, appropriately justified in relation to your research problem, aim and objectives. It should also demonstrate your careful consideration of ethics and regulations, and that your research will comply with your course and university requirements. You should refer back to Cref{sect:stage1ResearchDesign} to develop the content of this section. \\
 {Sect 5 – Work planning and risk assessment \\5.1 Statement of progress \\5.2 Key priorities in follow-up stage \\5.3 Risk assessment }& In this section you should reflect on the progress you have made in Stage 1 in relation to your initial work plan and establish your priorities for the next stage. For this, you should refer back to Cref{sect:stage1WorkPlan} and revise your initial plan accordingly. You should also summarise the outcome of your project risk assessment (see Cref{sect:stage1ManagingRisk}). \\
 References & You should keep your references in good order and ensure you apply the required bibliographical style consistently. Ideally, you should use a BMT to generate and integrate your references within your report \\
 {Appendix – Work schedule} & You could include it as an appendix for reference \\
 {Appendix – Risk assessment} table & You could include your filled-in risk table as an appendix for reference \\
\end{SimpleNColTable}

\begin{question}[subtitle={Activity: Putting your report together}] Using your word processor of choice, create a report with the structured indicated in Cref{tab:researchProposalStructure}, and fill it in by following the guidance provided in the table, making good use of your notes and summaries from, and reflection on, all related activities you have carried out so far.

\begin{guidance}
In this first pass at putting together your report you should focus primarily on completeness, ensuring that each section includes at least a draft of the main points you wish to make.
\end{guidance}\end{question}
%%Hack to correct tcbox behaviour
\color{black}

\section{Assessing and Iterating}\label{sect:AssessingAndIterating}
After you have filled in your report with as much material as you can, you should review and revise it until you are happy with your account, and ready to move on. This may take more than one iteration, but you should ensure you do not delay your work for the follow-up stage.

In the next activity, you will use Cref{tab:criteriaForResearchProposal} to assess whether your report is of good standard.

\begin{SimpleNColTable}{tab:criteriaForResearchProposal}{2}{Criteria for reviewing your research proposal}[X[1]X[5]]
Criteria & Prompts \\
Completeness & Are all sections included and their content complete? What is missing?\\
Academic writing & Have you applied good academic writing practices throughout? Which main issues do you still have to address? \\
Logical structure and flow & Have you structured your writing appropriately to ensure a logical flow of arguments? Which restructuring may be needed?\\
Supporting evidence & Are your key arguments supported by appropriate references or other evidence? Which further evidence is needed?\\
Citation and reference style & Do all your citations and references comply with the required bibliographical style? \\
Avoiding plagiarism & Have you acknowledged the work of others and distinguished it from your own appropriately? \\
Grammar and spelling & Have you proof-read your report carefully to remove all typos and grammatical errors? \\
\end{SimpleNColTable}

\begin{question}[subtitle={Activity: Reviewing your report}] Apply the criteria in Cref{tab:criteriaForResearchProposal} to review and reflect on your current report, what is good, what is weak and what is yet to do, and write up a summary of your assessment. For work still do do, decide whether to complete it within Stage 1 or carry it over to Stage 2.
\begin{guidance}
For each criteria, consider the related prompts to help you assess your report overall, and write down any further work needed: some will be minor and may be done at another pass at editing your report; other may point to substantial work you will need to do in Stage 2, and should be included in your work plan.
\end{guidance}
\end{question}
%%Hack to correct tcbox behaviour
\color{black}

Writing up your report is an excellent way to communicate to consolidated the work you have completed and are still planning to do, and is something tangible you can share with your supervisor for comment and other formative feedback.

\chapter{Stage 1 Takeaways}\label{ch:Stage1Takeaways}
\begin{itemize}
\item a research problem can be framed by identifying its key elements, that is, context, knowledge gap, phenomena of interest, stakeholders and their reasons
\item there are different kinds of research problem, including descriptive, exploratory, explanatory, predictive, evaluative and design problems
\item candidate research problems for Masters research should be assessed in terms of their generality, complexity and volatility
\item bibliographical databases and Google Scholar are key online tools to access the academic literature
\item Keshav's approach is an effective way to assimilate the content of an academic article
\item a summary-comparison matrix can be used to keep track of the content of articles reviewed, also making it easier to compare them
\item a concept matrix and a concept map can be used to identify themes emerging from the literature reviewed
\item while a research problem highlights a knowledge gap, a research aim indicates a particular way the knowledge gap is to be addressed
\item several kinds of data, evidence and methods are used in research
\item research is subject to a wide range of ethical and regulatory frameworks
\item project risk in research arises due to many factors, including lack of technical skills, time or resources, as well as ethics and regulations 
\item Gannt charts can be used in combination with the 5-stage framework guidance to outline your project work plan
\item reflection can help you become a better researcher
\item the template provided can help you structure your Stage 1 report
\end{itemize}




