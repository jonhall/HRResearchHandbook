\begin{PartTitlePage}{The Methodology}{stage3}
In~\Cref{stage3} you will focus on developing your \glsf{research-methodology}. As you learned in~\Cref{stage2}, this will require you to choose an appropriate \glsf{research-strategy} and methods.	
\end{PartTitlePage}


%\section{Developing your research design}\label{sect:DevelopingYourResearch}\label{sect:DevelopingYourResearch}

It's time to move to~\Cref{stage3} of the 5-stage framework. This assumes you have completed your work at the previous stages and have discussed it with your supervisor, who will have helped you assess whether it is appropriate for your course of study. 

\Cref{stage3} also assumes you have a good grasp of the relevant literature and a clear idea of the knowledge contribution you intend to make with your project, appropriately captured by your aim and objectives. 

This stage will build on your understanding of \glsf{research-design} from~\Cref{stage2} to help you define an appropriate methodology for your project, which you will exercise in~\Cref{stage4} to generate, analyse and interpret your \glsplf{datum} and \glsf{evidence}.

As per~\Cref{stage2},  you won't go through the \glsf{research-process} step by detailed step, but will focus instead on \textit{additional} knowledge and skills you'll need to make progress.


\chapter[Stage III Activities and Outcomes]{\Cref{stage3} Activities and Outcomes}\label{c:Stage3outcomes}

You should use the Research Activity table and the Writing Outcomes table for this stage to structure and guide your work.

\section{Your Research Activities for this stage}

The research activities which are in focus in~\Cref{stage3} are shown in~\Cref{stage3ResearchActivities}, which also provides some prompts for your interaction with your supervisor during this stage. 

Looking at the table, developing your \glsf{research-design} is by far the most demanding activity, at 40\% of the overall stage effort: you will consider several research strategies, and related methods, to arrive at an appropriate choice for you project, and this will also require you to access the literature beyond the content of this book. As a result of your choice, you will break down your objectives into specific tasks which relate to work you must complete in order to apply your methodology in~\Cref{stage4}. This explains the 10\% effort in relation to setting \glsf{research-aim} and objectives in this stage. 

All other research activities are similar in effort to previous stages. You should ensure you complete each of them, even if they are not covered explicitly in this stage. In particular, you should engage in \glsf{reflection} and \glsf{reflexivity} as you go along, and revisit your risk assessment to ensure all major risk remains accounted for.


\begin{SimpleNColTable}{stage3ResearchActivities}{4}{\RActivitiesTableCaption{3}}[R[4]cR[8]R[8]]
Research Activity & Effort & Description &  Supervisor Interaction Focus\\
Identifying the \glsf{research-problem}&5\%&Adjust, if needed&\\
Reviewing the literature&5\%&Adjust, if needed&\\
Setting \glsf{research-aim} and objectives&10\%&Finalise aim and objectives, and break them down into tasks and \glsplf{deliverable} in line with your chosen methodology&Suitability of tasks and \glsplf{deliverable} from objectives\\
Developing the \glsf{research-design}&40\%&{Establish your \glsf{research-methodology}, including your detailed consideration of \glsplf{datum} and \glsf{evidence}, \glsf{research-strategy}, research methods and how to apply them}&Suitability of methodological choices\\
Generating and analysing \glsf{evidence}&0\%&n/a&\\
Interpreting and evaluating findings&0\%&n/a&\\
Writing up&20\%&Achieve the writing outcomes of~\Cref{stage3WritingOutcomes} & Demonstration of good academic writing and identification of any improvements required\\
\Glsf{reflection} and \glsf{reflexivity} &10\% & Apply to~\Cref{stage3} work and experience as you go along &\\
Planning work &5\% &{Refine your project plan by detailing tasks, \glsplf{milestone} and \glsplf{deliverable} for this stage\\ Review progress} & Appropriateness of work plan and progress\\
Managing risk &5\% &Review and adjust project risk &Any major adjustment required\\	
\end{SimpleNColTable}


\section{Your Writing Outcomes for this stage} 

\Cref{stage3WritingOutcomes} gives you the writing outcomes for this stage: the activities in this part of the book are designed to help you reach them.

Remember that the first column of the table gives you the expected full structure of the dissertation\footnote{If your course assumes a different structure, then use that instead, mapping the writing outcomes accordingly.}. Within that column, the greyed out parts are yet to be written and will be the focus of later stages. The remaining parts are those you started to write in previous stages: some will need substantial revision based on your work in this stage, while others may require minor changes. Note that in this stage you are not expected to start new sections of your dissertation from scratch: instead, your work will focus on deepening your knowledge and understanding of \glsf{research-design}, leading to your justified methodological choices.   

\todo{ensure all references defined}\begin{ReportTable}{stage3WritingOutcomes}[\WOCaption{3}{20}]
%%\ReportTitle1
\ReportTitle*
	& Update your title if necessary
	 &\\

%%\ReportAbstract12345
\ReportAbstract-----
	&\tablenocontent
\\

%%1-\ReportIntroduction12345
\ReportIntroduction****
	& Revise introduction and justification, also adding any new definitions, if necessary 
	&\\

%%2-\ReportLitRev---
\ReportLitRev***
	& Continue to develop your literature review by adding late-found papers &\\

%%3-\ReportResearchDef----
\ReportResearchDef****
	& Revise if necessary &\\

%%4-\ReportResearchDes-----
\ReportResearchDes*??-?
	& Provide a detailed account of all data needed in the project, its uses and sources	
	%%\itemize in tblr header
	\item Indicate and justify the research strategy(ies) for your project 

	\item Revise your account of relevant ethics and regulations in view of your methodological choices 
	
	&\Cref{ch:ResearchStrategyCandidates,ch:yourResearchMethodology}
\\

%%5-\ReportAnalysisInterp----
\ReportAnalysisInterp----
	& \tablenocontent
\\

%%6-\ReportEvalConc-------
\ReportEvalConc-------
	& \tablenocontent &\\

%%7-\ReportRefs-
\ReportRefs*
	& \tablecites{} &\\

%%%A-\ReportDissertationAppendices
\ReportDissertationAppendices--
	& \tablenocontent\\
		
%%\ReportProgressTracking123456789
\ReportProgressTracking***?*??
& Revise the content of this appendix in view of your increased understanding and the progress you have made in this stage, paying particular attention to feasibility, work plan and risk assessment
&\\
%
%%\ReportReflection123
\ReportReflection*?
 & Update your personal statement based on actions and outcomes in this stage 
 &\\

\end{ReportTable}
\endinput


\section{Planning your work for this stage}\label{sect:stage3WorkPlan}

Before going further, you should refine your project work plan to include more detail on your work for~\Cref{stage3}. 


%For this, you will need to:
%\begin{itemize}
%	\item make sure you have completed all the work for~\Cref{stage1} or make the necessary adjustments to your plan
%	\item identify the main tasks under each activity for this stage, allocate them time and include them in your plan. For complex tasks, you may also include some sub-tasks,~\etc{}, but you should avoid making your plan too complicated
%	\item establish main \glsplf{milestone} and \glsplf{deliverable} and include them in your plan
%	\item optimise your plan by considering dependencies and tasks which may overlap. 
%\end{itemize}


\begin{activity}[{Revising your work pan}] 
Consider the activities in~\Cref{stage3ResearchActivities} and the writing outcomes in~\Cref{stage3WritingOutcomes}: 

\begin{itemize}
	\item for each activity, identify a number of tasks which capture the work needed, decide how much time to spend on each, and include them in your work plan, also taking into consideration their possible dependencies
	\item for each \glsf{writing-outcome}, identify corresponding \glsplf{deliverable} and set related \glsplf{milestone} in your work plan.
\end{itemize}

At the end, review your overall plan, also considering the progress you made in the previous stages, and make all necessary adjustments. 
\begin{guidance}
%Hack to correct tcbox behaviour
\color{black}


%Hack to correct tcbox behaviour
\color{black}
Make sure you:
\begin{itemize}
	\item focus on a small number of key tasks for each activity, so as to keep your plan light
	\item when allocating time to tasks, ensure that tasks fit within the overall time for their corresponding activity 	
	\item consider task dependencies and things you can progress in parallel, so to optimise your project time 
	\item break down new content you will need to write into \glsplf{deliverable}, setting appropriate \glsplf{milestone} in your plan.
\end{itemize} 
\end{guidance}
\end{activity}
%%Hack to correct tcbox behaviour
\color{black}

If after reviewing your progress you find that you are well behind, then talk to your supervisor who will be able to advise you on how to bring your project back on track and improve your planning.


\chapter{Defending your claim of new knowledge}\label{ch:DefendingYourClaim}

Being able to assert that you have made a contribution to knowledge is the point of a well thought-out \glsf{research-methodology}  -- hence, its importance in research. 

Choosing a good methodology is only the starting point, however. Having made your \glsf{claim} to knowledge at the end of your project, you still need to defend it in your dissertation. In essence, that means considering everything that could have gone wrong -- any weakness -- with the execution of your \glsf{research-methodology}, and explaining how you've dealt with it. 

The purpose of this chapter is to introduce potential research weaknesses upfront and ways to deal with them: with this information, you can then be more mindful in the choice and execution of your own \glsf{research-strategy} and methods.

 \Cref{fig:RVs} illustrates the key points of this chapter. At its core is your claimed knowledge contribution at the end of your project. Its defence is what you need to argue in your dissertation. Such a defence has to withstand external scrutiny, say that of your examiner or the wider community of scholars, researchers or practitioners your work is intended for. Your \glsf{claim} to knowledge is subject to a number of weaknesses: we consider four main types, illustrated as potential \enquote{cracks} in your defence.  Should you recognise any of them in your research, then your defence should explain how they've been dealt with, illustrated as bandaids over the cracks. The kind of \enquote{bandaid} will depend on what you decided to do. Some specific kinds of bandaid are available to you: the ones considered in this chapter are \glsf{triangulation}, \glsf{reflexivity}, critical review and appealing to the literature.

\begin{figure}[hbtp]
\centering{
  \includegraphics[width=\textwidth]{Figures/ResearchWeaknesses}
  \caption[]{Addressing research weaknesses: cracks and bandaids  \label{fig:RVs}
  	}  
  }
\end{figure}

\section{Weaknesses and actions to take}\label{sect:DealingWithWeaknesses}

This book classes weaknesses in claimed knowledge contribution as follows:
%
\begin{itemize}
\item \glsplf{validity-weakness}, i.e., the \glsf{claim} you have made to new knowledge isn't sufficiently credible, trustworthy, or accurate to be considered knowledge, or can't be generalised or transferred beyond your study

\item \glsplf{reliability-weakness}, i.e., the steps you have taken to establish your \glsf{claim} of new knowledge are not dependable, cannot be replicated under the same conditions or are not sufficiently repeatable in other contexts, or the descriptions and interpretations you have provided are incoherent or inadequate

\item \glsplf{bias-weakness}, i.e., the \glsf{claim} you have made to new knowledge has been affected by your implicit or explicit \glsplf{bias}, or \glsf{bias} affecting human participants in your study, making the new knowledge invalid

\item \glsplf{novelty-weakness}, i.e., the hole in the literature that you claimed existed doesn't actually exist.  If there is no hole, then you cannot have contributed new knowledge --- perhaps you missed some key papers in your literature review or, perhaps in the time that you've taken to complete your research, someone else has made a similar contribution to knowledge as that you \glsf{claim}. Alternatively, while a hole may exist, the novelty of your \glsf{claim} may be in doubt --- perhaps your research was not able to achieve all that you were hoping for. 
\end{itemize}

There are, of course, connections between these types of weakness. In particular, if your methodology is not reliable, then any resulting \glsf{claim} to knowledge is unlikely to be valid.

\begin{example}{Validity and reliability} 
 If the scale you use to measure the weight of an object returns different values every time, then it is unreliable. As result, you can't draw a valid conclusion on the weight of that object. 
 
 However, reliability is not sufficient for \glsf{validity}. For instance, your scale may reliably return the same weight every time you weigh the same object, but may overestimates it. In this case, while your scale is reliable, it is inaccurate, so that you still can't draw a valid conclusion on the weight of the object\footnote{Unless you know precisely by how much your scale overestimates weights.}.
\end{example}

 \Glsf{bias} also affects \glsf{validity}. 
 
 \begin{activity}[{\Glsf{bias} and \glsf{validity}}]
Consider how preconceptions you may have could impact the \glsf{validity} of your research. Write down your thoughts.
\begin{solution}
You preconceptions may lead you to discard any \glsf{evidence} to the contrary and only retain \glsf{evidence} that confirms your \glsf{bias}. In this case, your conclusions would be untrustworthy, hence invalid. 
\end{solution}
\end{activity}
%%Hack to correct tcbox behaviour
\color{black}
 
As you will see in the following chapters, different research strategies and methods are affected differently by these weaknesses. For instance, research based on quantitative objective measurements will focus more on ensuring reliability than research based on subjective interpretations of qualitative information, where the researcher's \glsf{bias} is more likely to have a negative impact. In it important for you to consider all weaknesses that may affect your research and take action to ensure they will not impact the \glsf{validity} of your \glsf{claim}.

 In general, the possible actions you can take to deal with potential weaknesses fall into three options\footnote{There is actually a fourth way, which is to be aware of the weakness but to ignore it. This is not recommended as your dissertation examiner is likely to have detailed understanding of the \glsf{research-strategy} and methods you have chosen, including their potential weaknesses, and is likely to pick on any of them.}:
%
\begin{itemize}
\item avoid the weakness, i.e., choose a strategy or methods which is not troubled by the weakness. Part of the justification for your choice can then be a discussion, if necessary, that the weakness doesn't arise. This would be reported in your dissertation as part of the \glsf{research-design}

\item address the weakness, i.e., be aware of the weakness during the research and put in place further strengthening research. This might be, for instance, a second or further iteration of a \glsf{research-strategy} or method which addresses discovered weaknesses in earlier research. This also would be reported in your dissertation as part of the \glsf{research-design}

\item acknowledge and defer\footnote{Although it may seem to have similar outcomes, this is a much better strategy than simply ignoring the weakness as, although you don't address it, you make the examiner aware that you are aware of it. It can also give you a very neat way of filling out your future work.}, i.e., write a \glsf{reflection} on the effect the weakness had on the outcomes and commit to addressing that weakness in future research. This would be most usually done at the end of the research period when the research is complete, and reported towards the end of your dissertation as part of your discussion and conclusions.
\end{itemize}

If you can't avoid a weakness and you can't defer it, you have to address it. Addressing it will make your research more credible\footnote{And will also help you prepare to answer all the questions an expert examiner will have about the weaknesses they know occur in the type of research you're doing.}. 

\section{Addressing weaknesses}\label{sect:ApproachesToAddress}

This section disscusses common approaches used to address weaknesses in research.

\subsection{Triangulation}\label{ssect:triangulation}


\Glsf{triangulation} \parencite{mathison1988triangulate} consists of using multiple \glsplf{datum} sources and methods, or even multiple researchers, to develop a comprehensive understanding of a \glsplf{phenomenon} under study and arrive at a particular conclusion about that \glsf{phenomenon}. \Glsf{triangulation} was introduced in the social sciences in the mid 1950s \parencite{campbell1959convergent}, and since has become an accepted approach across all disciplines and research mindsets.

The core idea behind \glsf{triangulation} is that if different \glsplf{datum} and methods converge towards the same conclusion, then it is more likely that such a conclusion is valid, that rival explanations can be dismissed, that the steps followed to conduct the research are reliable, and that the effect of any \glsf{bias} is mitigated. In this way, \glsf{triangulation} makes your research more credible, and your \glsf{claim} more defensible. 

However, because \glsf{triangulation} applies many techniques or derives conclusions from many sources, it can result in inconsistent or contradictory findings. So, it is important to understand that \glsf{triangulation} does not necessarily guarantee convergence on a single \glsf{proposition} about a \glsf{phenomenon}. Instead, it provides a rich, often  complex, picture that requires careful interpretation and explanation by the researcher. As a result, you should use \glsf{triangulation} cautiously in your research, and be prepared to explain and make sense of the various outcomes it may produce. \Glsf{triangulation} also adds complexity and requires more time and effort that you must account for in our work plan.

Main kinds of \glsf{triangulation} include \parencite{denzin1978research,patton1999enhancing}:

\paragraph{\Glsf{data-source-triangulation}} refers to using several \glsplf{datum} sources. These may be the \glsf{inclusion} of multiple participants to \glsf{interview}, or the consideration of a particular \glsf{phenomenon} under different conditions in space and time. For example, in an educational setting, you may wish to measure the efficacy of an educational programme on different student cohorts, possibly over different academic years, or delivered by different educators. With \glsf{data-source-triangulation} you increase the \glsf{validity} of your \glsf{claim} across different contexts, so that your results are more generalisable.

\paragraph{\Glsf{investigator-triangulation}} involves several researchers collecting and analysing \glsplf{datum}. For instance, you may have different researchers repeating measurements using the same lab equipment and procedures. The involvement of different researchers who independently apply the same techniques to arrive at the same conclusions, increases both reliability and \glsf{validity} of those outcomes, and mitigates against each researcher's \glsf{bias}.  This is particularly important in qualitative research where \glsplf{datum} are often interpreted rather than measured precisely. As there is more than one researcher involved, it is unlikely that you  will be required to perform this form of \glsf{triangulation} in a master's or a doctoral project. You may, however, be a researcher in the \glsf{triangulation} of another's researcher – your supervisor, for instance\footnote{In which case, be sure to schedule some time with your supervisor to discuss their needs.}.

\paragraph{\Glsf{methodological-triangulation}} refers to the use of multiple methods in the examination of a \glsf{phenomenon}. For instance, a neuropsychologist may combine direct \glsf{observation} of human behaviour with neurological \glsplf{datum} from brain scans to obtain a comprehensive picture of what motivates people to make certain choices. \Glsf{methodological-triangulation} allows strengths and weaknesses of different methods to compensate for each other, increasing both reliability and \glsf{validity}. However, it may be difficult for you to combine results from different methods because of their differing ontological and epistemic stance.

\paragraph{\Glsf{theory-triangulation}} refers to the use of different theories or \glsplf{hypothesis} to analyse \glsplf{datum} and interpret \glsplf{phenomenon}. For instance, you could apply different motivation theories to study resistance to change in organisations. By employing several theories, findings can be considered from different angles, compensating for possible limitations or \glsplf{bias} of each individual \glsf{theory}. 

\begin{activity}[{Distinguishing different kinds of triangulation}]
Consider each of the following examples and indicate which kind of \glsf{triangulation} they represent:
\begin{itemize}
	\item research on student experience in a university looking at student survey \glsplf{datum} and students' study results
	\item research on study practice and academic performance, combining an online survey and interviews with a selected number of participants
	\item research on sleeping patterns of the elderly, using \glsplf{datum} from care homes in the UK
	\item research on volcanos asking vulcanologists around the world to contribute seismological measurements over a period of time. 
\end{itemize}
\begin{solution}
These are example of, respectively:
	\begin{itemize}
	\item \glsf{data-source-triangulation}, in which two different kinds of \glsplf{datum} are considered  
	\item \glsf{methodological-triangulation}, in which two different methods are applied 
	\item \glsf{data-source-triangulation}, in which similar \glsplf{datum} from different locations are considered
	\item \glsf{investigator-triangulation}, in which several researchers are invited to collect and contribute \glsplf{datum}. Presumably, this also includes some \glsplf{datum} \glsf{triangulation} as similar \glsplf{datum} from different locations around the world are collected and analysed.
\end{itemize}
\end{solution}
\end{activity}
%%Hack to correct tcbox behaviour
\color{black}



%Researchers use various methods, such as pilot studies and comparison to other validated measures, to establish reliability and \glsf{validity}. Computerised \glsplf{datum} analysis packages can also enhance reliability. It is important to balance standardisation with maintaining the context and meaning of the \glsplf{datum}.

%\subsubsection{Evaluation of Reliability}\label{sssect:EvaluationOfReliability} ...
%
%% Messick's model introduced the idea of social consequences of measurement outcomes as an important aspect of \glsf{validity}. The introduction of social consequences raised debates about the feasibility and \glsf{inclusion} of consequences in the \glsf{validity} framework. Kane's argumentative approach to \glsf{validity} emphasises the practical feasibility and provides guidance on how to allocate research efforts and gauge the validation process. The integration of \glsf{validity} \glsf{evidence} and the need for guidance on prioritising \glsf{validity} questions are also discussed.
%%

%\begin{activity}[{More on \glsf{triangulation}}]
%Read these sources on \glsf{triangulation}
%\end{activity}
%%Hack to correct tcbox behaviour
%\color{black}

\subsection{Documenting procedures}\label{ssect:Procedures}

You should document your research procedures by maintaining a clear and detailed record of how you have conducted your study, including which \glsplf{datum} you collected and from which sources, and how you applied your chosen \glsplf{datum} generation and analysis methods, the process you followed and any adjustment you may have made. You should develop such record as you go along in your research, and include a summary in your dissertation.

The kind of detail to include in your record will depend on your chosen methods to a large extent. So, for instance, if you were to conduct some experiments in a laboratory, you would have to describe any apparatus used, how your measurements were obtained, their margin of error,~\etc{}. Other methods will require other kinds of record: part of developing your understanding of the methods you are going to apply will include understanding what kinds of record you will be expected to keep.

Such records are important as they make the work you have done transparent to others, like your examiners or other researchers who may be interested in replicating your work. Through your records you will be able to demonstrate that you have applied your methods correctly and minimised the chance of errors, making your research outcomes more reliable.

\subsection{Reflexivity}\label{ssect:reflexivity}

\Glsf{reflexivity} should not be a new concept to you! As you learnt in~\Cref{ssect:ReflectionAndReflexivity,ch:ReflectingReflexivity}, \glsf{reflexivity} should be an integral part of doing research and a way to help you become a better researcher. Reflexivity is also an effective way to deal with potential methodological weaknesses.

According to \cite{jamieson2023reflexivity}:

\begin{quotation}
	\Glsf{reflexivity} is the act of examining one's own assumption, belief, and judgement systems, and thinking carefully and critically about how these influence the \glsf{research-process}. The practice of \glsf{reflexivity} confronts and questions who we are as researchers and how this guides our work.
\end{quotation}

So, \glsf{reflexivity} admits that the researcher isn't an objective, unbiased observer of truth, but someone whose worldviews and subjectivity influence every step of the \glsf{research-process}. Through reflexive practice, the researcher can then engage in a more honest and transparent \glsf{research-process}, increasing research reliability and mitigating \glsf{bias}.  

\Glsf{reflexivity} is relevant and applicable to all types of research. Qualitative research has the longest tradition of \glsf{reflexivity}, with qualitative researchers encouraged to examine and acknowledge openly their own beliefs and \glsplf{bias}, and their impact on the research. In quantitative research, the acceptance of the importance of \glsf{reflexivity} is growing, and goes alongside an acknowledgement that there are limitations and \glsplf{bias} also in methodologies based on a positivist worldview, hence favouring measurable and objective observations and empirical \glsf{evidence}. This is the case for the \glsf{scientific-method}, often seen as the \enquote{gold standard} of such kind of research, developed and widely applied in the natural sciences. 

In relation to your methodology, \glsf{reflexivity} can help you in a number of ways. In \glsplf{datum} generation, it can expose \glsplf{bias} and unchecked assumptions which may affect how you select samples and \glsplf{datum} sources or recruit participants. In \glsplf{datum} analysis and interpretation, \glsf{reflexivity} may lead you to uncover reasons why you may give more weight or meaning to certain \glsf{evidence}, while discarding other, for instance due to \glsf{confirmation-bias}. In formulating conclusions, \glsf{reflexivity} can support you in \enquote{thinking about thinking}\footnote{So-called \enquote{meta-cognition}.}: the process of questioning the way you think to assess how valid and reliable your conclusions are. This is particularly important because while the human brain has the potential for logic and \glsf{critical-thinking}, these are not innate skills: rather they need developing, akin to the skills that one must develop to become, say, a proficient musician or mathematician. Psychologists have uncovered that left untrained, our brain tends to make mistakes, which stem from a variety of factors\footnote{\color{black}A fascinating series of lectures on this topic is \enquote{Your Deceptive Mind: A Scientific Guide to Critical Thinking Skills} by Steven Novella.}, including errors in perceptions, flawed memories, heuristic thinking, \glsplf{logical-fallacy} and \glsf{cognitive-bias}. \Glsf{reflexivity} can help us become aware of these tendencies. 

 \begin{activity}[{Reflexivity practices}]
Conduct a web search on \glsf{reflexivity} practices adopted by researchers. Briefly summarise what they are, and how they may be useful, particularly in relation to methodology. Comment on which of such practices you could adopt in your work.
\begin{solution}
You may have found some or all of the following:
\begin{itemize}
	\item \glsf{reflexive-writing}, such as research journals, diaries, fields notes and memos. These are common tools used by the reflexive researcher at any point in the \glsf{research-process} to record assumptions, experiences, observations, perceptions, procedures, and decision points. They are used to bring into focus the researcher's intention and gaps in their knowledge or thinking, as well as interpersonal dynamics, including power ones 
	\item \glsf{positionality-statement}. This is a kind of \glsf{reflexive-writing} aimed at describing a researcher's characteristics (such as age, social class, race,~\etc{}) and beliefs (such as political, philosophical,~\etc{}) which may influence the research
	\item \glsf{narrative-autobiography}. This is also a kind of \glsf{reflexive-writing} focussed on the researcher's life experiences and motivations which may influence the research, particularly the researcher's interaction with participants and understanding of participants' accounts. The aim is to prepare better the researcher for their interaction with participants, so it is best conducted when planning \glsplf{datum} collection/generation
	\item \glsf{reader-response-exercise}. This addresses how the researcher's own assumptions and experiences may affect their interpretation of participants' accounts. It consists of adding information to indicate how the researcher reacts to and interprets participants' accounts in relation to their own background and personal history. As such, this practice is useful during \glsplf{datum} analysis and interpretation
	\item \glsf{collaborative-reflexivity}. This entails engaging in \glsf{reflexivity} as part of a research team, with collaborators questioning assumptions and decisions. It assumes mutual trust, and a commitment to \glsf{ethics} and rigorous research, regardless of seniority or status. It applies to all stages of the \glsf{research-process}.
\end{itemize}
\end{solution}
\end{activity}
%%Hack to correct tcbox behaviour
\color{black}


\subsection{Returning to the literature}\label{ssect:ReturningToThe}

Addressing novelty weaknesses means returning to your literature review as your research progresses to cast an increasingly critical eye over it, and possibly widen its scope to further related work, perhaps some which was published more recently than your original review. 

Each source should be reconsidered for what you thought it originally said and what you now think it says, using any difference to drive further \glsf{reflection} on your findings, methods, \glsplf{datum} generation, or even the \glsf{research-problem}. This process will help you both ensure there is still a gap your research can contribute to, and assess the extent of such a contribution.

While defending your \glsf{claim} or explaining your \glsf{research-design} in your dissertation, you can made your reader aware of this process and how it has altered your research. Deepening the critical nature of your literature review allows your reader to understand that you are a reflective researcher and can turn any \glsf{novelty-weakness} into a research strength.

%LR -- section commented out (too much of a collection of lists at present); to  consider whether content can be factored into the research strategies instead}
%\subsection{Weakness types  --- this section may be omitted; to decide}\label{ssect:WeaknessTypes}
%\todo{this section is either to develop or to remove; to rethink after the \glsf{research-strategy} bit is done} In this section we recall many of the common weaknesses under each of the categories we have introduced. All research projects are subject to some or even most of them, so awareness of them will help you better inform your choice of research strategies and the step to take to address them. 
%
%\subsubsection{Novelty weakness}\label{sssect:NoveltyWeakness}
%
%To identify your \glsf{research-problem}, you will have found a hole in current knowledge through the literature you will have surveyed and reflected critically upon. During subsequent research you may find that you were:
%%
%\begin{itemize}
%\item unable to contribute knowledge in that area, or not able to contribute as much as you had initially hoped;
%\item found further sources that had already made a contribution.
%\end{itemize}
%%
%If you do encounter this weakness, you will not be the first – virtually all researchers find that their initial aspirations for a knowledge contribution has to be reduced or altered as their research – and understanding – progresses.
%
%
%
%\subsubsection{\Glsf{validity} weaknesses}\label{sssect:ValidityWeaknesses}
%
%%\subsection{Dealing with \glsf{validity} weaknesses}\label{ssect:DealingWithValidity}
%%%[Distinguish and add Validation of evaluation...]
%
%There are a number of recognised \glsf{validity} weaknesses\footnote{These are often called \enquote{threats to \glsf{validity}} in the literature. We prefer \emph{weakness} as it suggests that there is an issue caused by the \glsf{research-strategy} application rather than by the environment.} upon which an evaluation of research \glsf{validity} is made. \todo{why this list? where does it come from?}
%%
%\begin{itemize}
%\item mismatch between quantitative and qualitative samples
%\item imbalance between an insider's and outsider's views
%\item insufficient knowledge of research question, \glsf{theory}, \glsplf{hypothesis}, statistical tests, and analysis
%\item occurrence of unrelated events or conditions during \glsplf{datum} collection
%\item insufficient or biased knowledge of earlier studies and theories
%\item lack of descriptive \glsf{validity} of settings and events during \glsplf{datum} analysis and interpretation
%\item \glsf{population}, time, and environmental \glsf{validity} in quantitative research
%\item lack of cognitive and empathy training of researchers
%\item value or ideologically based conflicts in collaboration between quantitative and qualitative researchers
%\item difficulty in persuading consumers to value the meta-inferences from both qualitative and quantitative findings.
%\end{itemize}
%%
%
%\subsubsection{Reliability weaknesses}\label{sssect:ReliabilityWeaknesses} refers to the trustworthiness and consistency of the procedures and \glsplf{datum} generated in research. It is concerned with the extent to which the results of a study or measure are repeatable in different circumstances. Reliability can be demonstrated through methods such as inter-rater reliability, test-retest reliability, and internal consistency.
%
%\subsubsection{Bias}\label{sssect:bias}
%
%[Adapted from \cite{simundic2013bias}]
%
%\textcquote{simundic2013bias}{%
%\Glsf{bias} is any trend or deviation from the truth in \glsplf{datum} collection, \glsplf{datum} analysis, interpretation and publication which can cause false conclusions. \Glsf{bias} can occur either intentionally or unintentionally (1). Intention to introduce \glsf{bias} into someone’s research is immoral. Nevertheless, considering the possible consequences of a biased research, it is almost equally irresponsible to conduct and publish a biased research unintentionally.
%
%It is worth pointing out that every study has its confounding variables and limitations. Confounding effect cannot be completely avoided. Every scientist should therefore be aware of all potential sources of \glsf{bias} and undertake all possible actions to reduce and minimise the deviation from the truth. If deviation is still present, authors should confess it in their articles by declaring the known limitations of their work.
%%
%%It is also the responsibility of editors and reviewers to detect any potential \glsf{bias}. If such \glsf{bias} exists, it is up to the editor to decide whether the \glsf{bias} has an important effect on the study conclusions. If that is the case, such articles need to be rejected for publication, because its conclusions are not valid.
%}
%
%\textcite{simundic2013bias} then goes onto detailed four forms of \glsf{bias}: 
%%
%\begin{itemize}
%\item \glsplf{datum} collection \glsf{bias}: including \glsf{selection-bias}, volunteer \glsf{bias}, admission \glsf{bias}, survivor \glsf{bias}, and misclassification \glsf{bias}
%
%\item \glsplf{datum} analysis \glsf{bias}: including%
%	\begin{itemize}
%	\item \glsplf{datum} fabrication: reporting non-existing \glsplf{datum} from experiments that were never done;
%	\item \glsplf{datum} elimination: eliminating \glsplf{datum} which do not support a \glsf{hypothesis} (outliers, or even whole subgroups);
%	\item using inappropriate statistical tests to test your \glsplf{datum};
%	\item performing multiple testing (\enquote{fishing for P}) by pair-wise comparisons, testing multiple end-points and performing secondary or subgroup analyses, which were not part of the original plan in order \enquote{to find} statistically significant differences regardless of \glsf{hypothesis}.
%	\end{itemize}
%%
%
%\item \glsplf{datum} interpretation \glsf{bias}: including
%\begin{itemize}
%\item discussing observed differences and \glsplf{association} even if they are not statistically significant (the often used expression is \enquote{borderline \glsf{significance}});
%\item discussing differences which are statistically significant but are not otherwise meaningful;
%\item drawing conclusions about causality, even if the study was not designed as an experiment;
%\item drawing conclusions about values outside the \glsf{range} of observed \glsplf{datum} (extrapolation);
%\item over-generalisation of the study conclusions to the entire general \glsf{population}, even if a study was confined to the \glsf{population} subset;
%\item type I (the expected effect is found significant, when actually there is none) and type II (the expected effect is not found significant, when it is actually present) errors.
%\end{itemize}
%
%\item publication \glsf{bias}: including
%%
%\begin{itemize}
%\item funding \glsf{bias}: due to the prevailing number of studies funded by the same company, related to the same scientific question and supporting the interests of the sponsoring company
%\item anti-negative \glsf{bias}: scientific journals are much more likely to accept for publication a study which reports some positive than a study with negative findings. Such behaviour creates false impression in the literature and may cause long-term consequences to the entire scientific community. Also, if negative results would not have so many difficulties to get published, other scientists would not unnecessarily waste their time and financial re- sources by re-running the same experiments.
%\end{itemize}
%\end{itemize}
% LR -- end of commented out section

\section{Where to defend your claim}\label{ssect:defendingYourClaim}

In your defence of claimed knowledge contribution, you should consider all potential weaknesses in turn – ignoring them leaves you open to a negative outcome of expert scrutiny. For each, you should make arguments as to why your \glsf{claim} doesn't suffer from it, or if it does to some extent, that you have dealt with it in a way that ensures there is still a contribution to knowledge arising from your research.

Typically, there are two places at which weaknesses in your claimed knowledge contribution should be discussed:
%
\begin{itemize}
\item in your dissertation, in all cases
\item in any \glsf{viva-voce} associated with your  course of study\footnote{Not all research has an associated \glsf{viva-voce}, particularly at master's level.}% Therefore, weaknesses should always be addressed in the dissertation itself. Even if your course does have a \glsf{viva-voce}, it can be a nerve-racking experience to be confronted by an examiner asking questions to which you have no answer because you haven't thought about it!}.
\end{itemize}

In general, an examiner will explore such weaknesses through a number of questions they ask of your dissertation. For each \glsf{research-strategy} and method, many of these questions\footnote{If not all; although examiners will have their own way of asking them!} can be predicted with reference to the types of weaknesses we have discussed above. Somewhere in your dissertation, then, you will need to expose those weaknesses and argue how your research has addressed them.

Here is an example paragraph taken from an actual dissertation~\parencite{miles2019dispelling} with our commentary on specific points to the right alongside:

%\bigskip

\begin{longtblr}[
	label=none, %no Table 1:
	caption=Commentary on~\parencite{miles2019dispelling},
	entry=none, %no entry in lot
	]{
	column{2}={font=\itshape},
	row{1}={font=\bfseries},
	rowhead=1,
	width=0.9\linewidth,
	colspec={X[2,j]X[2,l]},
	}
Text&Commentary\\
My observational study focuses solely on the external elements of the embouchure and what can be seen in real time with the human eye,...&Being specific on which {phenomenon} are studied and on the observations made of them... \\

...through the recording of video images. My analysis, and the conclusions that come from it, has been made from a purely visual perspective, captured by combinations of camera angles, without needing the use of any complex and expensive technologies.&...thus correcting any expectations of what might have been achieved...\\
In embarking on this research project, the initial intention was to measure facial muscle activity using Electromyography. This method proved to be too costly...&...contextual factors prevented more sophisticated observations...\\
...and the heavily mathematic \emph{(sic)} and science based analysis process, out of the current skill set of this researcher.&...and initial investigations reveals how difficult this would be\\ 
Furthermore, due to the significant {evidence} found in the literature regarding the internal embouchure, the concept of the tongue being a pivotal element in facilitating pitch change has been accepted as fact and deemed unnecessary for further study in this project.&There was no knowledge contribution to be made in this particular area...\\
Therefore the ultimate goal of my research is to inform the teaching and learning of brass wind performance, with particular reference to the role of the embouchure.&...and so the knowledge contribution was ...\\ 
With this in mind, it is therefore important that the {datum} obtained through this study be identifiable through the simplest means possible, so that it can be of the most benefit to the brass-playing community.&...and our research goals were set accordingly.
\end{longtblr}

\begin{activity}[{Which weaknesses are discussed?}]
	Consider the extract above alongside the commentry. Which kinds of weakness does it refer to? How were they dealt with in that research? Which other weaknesses could have been discussed?
\begin{solution}
Two potential weaknesses were considered and addressed:
\begin{itemize}
	\item novelty: by being specific on the \glsplf{phenomenon} studied (the external elements of the embouchure), the text clarifies where the claimed novelty of the research lies. This makes it easy to check against related work in the literature, something the text could have mentioned explicitly
	\item \glsf{validity}: the \glsf{observation} of such \glsplf{phenomenon} through video images is defended as a valid method in relation to the aim of devising a practical approach to inform teaching and learning. This is in contrast to more sophisticated, but costly, approaches that would have been possible, but deemed unnecessary for the aim of the research.
\end{itemize}	
Other potential weaknesses not discussed are:
\begin{itemize}
	\item reliability: how reliable were the observations? Would another researcher have reached similar conclusions?
	\item \glsf{validity}: the study assumes the embouchure is a key factor in the teaching and learning of a brass instrument. Where does this assumption come from? 
\end{itemize} 	
As this is only a brief extract, it is, of course, possible that these weaknesses were considered and dealt somewhere else in the dissertation. 
\end{solution}
\end{activity}
%%Hack to correct tcbox behaviour
\color{black}

%We go into more detail of the forms of weakness and how they can be treated %– through \glsf{triangulation}, \glsf{reflexivity} and critical literature – 
%below.
%
%
%There are three ways of dealing with the vulnerabilities in your research. You can:
%%
%\begin{itemize}
%\item patch: 
%\end{itemize}
%%
%

%\subsection{Introduction}\footnote{Probably not here.}\label{ssect:Introduction}
%
%There's still a great debate raging about what knowledge actually is. 
%
%The analysis of knowledge involves identifying the components that make up knowledge and understanding the structure of the concept of knowledge. The traditional definition of knowledge is as \emph{justified true belief}. This combines three conditions, usually presented out of the order they written in:
%%
%\begin{itemize}
%\item the \emph{belief} condition: we must \emph{believe} that something is true for it to be knowledge; we cannot know something if we don't believe it to be true, even if it is, in actuality, false;
%\item the \emph{justification} condition: a belief is justified if we have a reason to believing it, i.e., in the words of Wikipedia, \enquote{if another mental state supports it}~\parencite{enwiki:1204227228}, such as \enquote{a sensory experience, a memory, or a second belief}. This says that there is a reason or \glsf{evidence} for believing something;
%\item the \emph{truth} condition: that something \emph{believed} is actually true. Although this condition doesn't prevent from \emph{believing} something that is, in actuality, false, it prevents us \emph{knowing} something that is false. To know something means that that something is true;
%\end{itemize}
%
%\begin{activity}[{Understanding the effort needed in this stage}]
%Do we want to explore this concept more deeply?
%\end{activity}
%
%It's a pretty neat definition that stems from ancient Greece. Unfortunately, it has more recently been argued that it's incorrect. 
%
%The Gettier problem~\parencite{gettier1963is-justified}, for instance, challenges this traditional definition of knowledge. The Gettier problem shows that even if a belief is justified and true, it can still fail to qualify as knowledge. This problem has led to debates about whether the justification condition is sufficient for knowledge, and some have proposed adding additional conditions to the analysis. However, there is also skepticism about whether any analysis of knowledge can fully capture its complexity and nuances. 
%
%This raises real problems for the naive researcher and justifies the need for a lot of additional argument to establish that we have new knowledge.
%
%\begin{activity}[{Optional activity: The Gettier Problem}]
%\textcite{gettier1963is-justified} is only 3 pages long, give it a read and try to understand the examples Gettier uses to question the traditional definition of knowledge.
%\end{activity}
%
%The result is that we must be careful to ensure that what we \glsf{claim} to be new knowledge is actually both \emph{new} and \emph{knowledge}.
%
%Against this background of the meaning of knowledge, part of your mission\footnote{...should you choose to accept it.} is to justify your \glsf{claim} to have made a contribution to knowledge.  Given the traditional view of knowledge, for new knowledge you must convince your reader that:
%%
%\begin{enumerate}
%\item the thing you \glsf{claim} to be knowledge is actually true;
%\item that you believe it to be true,
%\item you have a solid justification for believing it,
%\item it hasn't been claimed to be true before.
%\end{enumerate}
%%
%
%Working backwards, the literature review
%
%Let's look at the belief condition first: do you believe your contribution to knowledge to be true. You've expended a great deal of effort in learning, understanding, and writing in convincing yourself that what you have created is believable, and your investment is testimony to the fact that you believe it to be true. Therefore your dissertation is \glsf{evidence} that your contribution to knowledge is something you believe in\todo{Needs to work in \glsf{validity}, reliability, \glsf{bias} and the other elements of this part}.
%
%The thing you ...
%
%You are justified in believing it ...
%
%\subsection{Literature review}\label{ssect:LiteratureReview}
%
%As\todo{refer reader to previous section on this topic.} 
%
%\subsection{In detail: dealing with \glsf{validity} weaknesses}\label{ssect:InDetail}

%Reliability and \glsf{validity}\footnote{Adapted AI summary of \textcite{roberts2006reliability}.} are important – linked – concepts in research that serve to demonstrate the rigour and trustworthiness of both quantitative and qualitative research. Thus, researcher's planning of and \glsf{reflection} on reliability and \glsf{validity} and its reporting should be a part of all research strategies.
%
%Reliability and \glsf{validity} – and the effort needed by you to address them – varies greatly depending on the \glsf{research-strategy} chosen. For logical proof, for instance, reliability and \glsf{validity} may be as simple as having a local community of logicians check your work, or even through the use of a computerised proof checker. At another extreme, in \glsf{grounded-theory}, for instance, the treatment of reliability and \glsf{validity} will typically form large component of the reported research.
%
%\begin{tblr}{
%colspec={lXX},
%row{1}={font=\bfseries},
%}
%Threat&Meaning&Example\\
%History&An unrelated event influences the outcomes.&A week before the end of the study, all employees are told that there will be layoffs. The participants are stressed on the date of the post-test, and performance may suffer.\\
%Maturation&The outcomes of the study vary as a natural result of time.&Most participants are new to the job at the time of the pre-test. A month later, their productivity has improved as a result of time spent working in the position.\\
%Instrumentation&Different measures are used in pre-test and post-test phases.	&In the pre-test, productivity was measured for 15 minutes, while the post-test was over 30 minutes long.\\
%Testing&The pre-test influences the outcomes of the post-test.	&Participants showed higher productivity at the end of the study because the same test was administered. Due to familiarity, or awareness of the study’s purpose, many participants achieved high results.\\
%\end{tblr}


%\paragraph{\Glsf{validity}} refers to the extent to which a measure accurately represents the concept it claims to measure. It is concerned with the relevance and representativeness of the items or questions in a study. There are different levels of \glsf{validity}, including content \glsf{validity}, criterion-related \glsf{validity}, and construct \glsf{validity}. Content \glsf{validity} focuses on the relevance and representativeness of individual items, while criterion-related \glsf{validity} involves comparing a measure to other similar validated measures. Construct \glsf{validity} examines the underlying theoretical concepts and constructs being measured. \Glsf{validity} is important in ensuring that research findings are accurate and trustworthy.

%\Glsf{validity}\footnote{Adapted AI summary of \textcite{ihantola2011threats}.}, as already mentioned, refers to the extent to which a research study measures what it intends to measure and accurately reflects the concept or \glsf{phenomenon} under investigation. It is a crucial aspect of research quality and involves establishing the credibility, trustworthiness, and accuracy of the findings. 
%
%%Research is vulnerable to claims that it is not valid, and invalid research cannot be a contribution to knowledge. A good researcher will therefore consider each \textit{potential} vulnerability in turn and design their research so that it is robust should that potential vulnerability doesn't threaten their claimed knowledge contribution.
%
%The two main forms of \glsf{validity} are: \glsf{internal-validity} and \glsf{external-validity}, with each having a number of weaknesses.
%
%\subsubsection{Confusion reigns: validity} Although, or perhaps because, \glsf{validity} is an extremely important concept, its use as an analytic framework has spawned many different forms. \textcite{wortman1983evaluation}\label{sssect:ConfusionReigns}, for instance, say:
%%
%\blockquote{At the level of specific threats to \glsf{validity}, however, the sheer number is both overwhelming and somewhat confusing. Some threats seem a bit esoteric, especially for evaluators (e.g. \enquote{ambiguity about the direction of casual influence} and \enquote{hypothesis-guessing within experimental conditions,} for example); some seem to differ only in small degree (\enquote{compensatory rivalry by respondents receiving less desirable treatments} and \enquote{resentful demoral­ization of respondents receiving less desirable treatments,} for example); and still others (noted above) seem to be miscategorized, thereby blurring the differences among the major \glsf{validity} types. In addition, some of these threats are relevant during the design and planning for an evaluative study while others are more appropriate to the management and conduct of the study (e.g. multiple statistical testing, program implementation, diffusion, \etc{}). Perhaps some consolidation is needed to make the whole structure less cumbersome and \enquote{threatening.}}
%
%Because of this, the treatment of \glsf{validity} below is partial, focussing on three main areas: experiment, observations, and instruments. It would be wise to discuss which expectations your supervisor has of \glsf{validity}\footnote{Schedule a meeting with your supervisor, with the subject \enquote{\Glsf{validity} discussion}.}.
%%
%
%%The \glsf{validity} of a piece of research can be evaluated in many different ways. Those we consider here cover the most often found, and include\footnote{Adapted from \textcite[and enclosing material]{bhandari2023construct}.} \glsf{internal-validity} (including), external \glsf{validity} (including), construct \glsf{validity}, content \glsf{validity}, criterion \glsf{validity}, concurrent \glsf{validity}, discriminant \glsf{validity}, face \glsf{validity}, convergent \glsf{validity}, \glsf{population} \glsf{validity}, and predictive \glsf{validity}. 
%
%\paragraph{\Glsf{internal-validity}} refers to the extent to which the \glsf{research-design} and methodology accurately measure what they are intended to measure. For a conclusion to be internally valid, you need to be able to rule out other explanations (including control, extraneous, and confounding variables) for your results. 
%
%\Glsf{internal-validity} applies across research strategies and research instruments, each involving the analysis of a \glsf{phenomenon}, having different interpretations according to each. The various areas of \glsf{validity} are illustrated in Cref{fig:validity}.
%
%\begin{figure}[hbtp]
%\centering{
%  \includegraphics[width=\textwidth]{figures/InternalValidityTypes}
%  \caption{Types of \glsf{validity}
%  \label{fig:validity}
%  	}  
%  }
%\end{figure}
%
%\paragraph{Experiment} The forms of \glsf{validity} for experiments\footnote{The acronym for experimental \glsf{internal-validity} is \emph{THIS MESS} \parencite{wortman1983evaluation}, hence the addition of \emph{statistical} to \emph{regression toward \glsf{mean}}. Subsequently, other authors have added further \glsf{validity} types; \textcite{cook1979quasi}, for instance, lists 33 potential weaknesses.}, i.e., research designed to establish a cause and effect relationhip, are:
%%
%\begin{itemize}
%\item testing: [if helpful, explanations needed throughout]
%\item history:
%\item instrument change:
%\item (statistical) regression towards \glsf{mean}:
%\item maturation:
%\item experimental mortality:
%\item selection:
%\item selection interaction:
%\end{itemize}
%%
%
%\paragraph{Observations} The forms of \glsf{validity} for observations are:
%%
%\begin{itemize}
%\item selective participation: [explanations needed throughout]
%\item selective recall: 
%\item accentuated perception:
%\end{itemize}
%%
%
%\paragraph{Instruments} The forms of \glsf{validity} for \glsplf{datum} generation instruments are \parencite[, adapted]{middleton2022the4types}:
%%
%\begin{itemize}
%\item construct \glsf{validity}: Does the instrument measure the concept that it’s intended to measure?
%\item content \glsf{validity}: Is the instrument fully representative of what it aims to measure?
%\item face \glsf{validity}: Does the content of the instument appear to be suitable to its aims?
%\item criterion \glsf{validity}: Do the results accurately measure the concrete outcome they are designed to measure?
%\end{itemize}
%%
%
%\paragraph{External \glsf{validity}} is the extent to which you can generalise the findings of a study to other situations, people, settings, and conditions \parencite{wortman1983evaluation}. In other words, external \glsf{validity} means the extent to which you can apply your knowledge contribution in a broader context\footnote{In qualitative studies, external \glsf{validity} is also referred to as \emph{transferability}, i.e., how transferable are results.}. 
%
%Although poor external \glsf{validity} may not disqualify your knowledge contribution as novel, it will require you to be extremely careful about its context of applications. Instead, for instance, of being able to say:
%%
%\blockquote{This result applies to all children of school age studying computing}
%%
%you may need to say
%%
%\blockquote{The result applies to all children between 12 and 13 years old, studying computing at the such-and-such a school with teacher A.}
%
%Clearly, the restricted nature of the new knowledge reduces its applicability and use as predictive, for instance.

%External \glsf{validity} breaks down further into \glsf{population} \glsf{validity} and ecological \glsf{validity}.

%%
%\begin{itemize}
%\item \glsf{population} \glsf{validity} refers to whether you can reasonably generalise the findings from your \glsf{sample} to a larger group of people (the \glsf{population}). \Glsf{population} \glsf{validity} depends on the choice of \glsf{population} and on the extent to which the study \glsf{sample} mirrors that \glsf{population}. \Glsf{population} \glsf{validity} is established by showing that the \glsf{sample} and the \glsf{population} are similar.
%
%\item ecological \glsf{validity} refers to the extent to which the measures measures how generalisable experimental findings are to the real world, such as situations or settings typical of everyday life. Ecological \glsf{validity} is demonstrated by showing that the restricted research context sufficiently \enquote{mimicked} the real world.
%\end{itemize}
%

%
%\begin{itemize}
%\item construct \glsf{validity} refers to the extent to which the measures used in the study accurately capture the underlying constructs or concepts of interest.
%
%\item content \glsf{validity} refers to the extent to which the measures used in the study accurately capture the underlying constructs or concepts of interest.
%
%\item criterion \glsf{validity} refers to the extent to which the measures used in the study accurately capture the underlying constructs or concepts of interest.
%
%\item concurrent \glsf{validity} refers to the extent to which the measures used in the study accurately capture the underlying constructs or concepts of interest.
%
%\item discriminant \glsf{validity} refers to the extent to which the measures used in the study accurately capture the underlying constructs or concepts of interest.
%
%\item face \glsf{validity} refers to the extent to which the measures used in the study accurately capture the underlying constructs or concepts of interest.
%
%\item convergent \glsf{validity} refers to the extent to which the measures used in the study accurately capture the underlying constructs or concepts of interest.
%
%\item predictive \glsf{validity} refers to the extent to which the measures used in the study accurately capture the underlying constructs or concepts of interest.
%\end{itemize}
%%
%



%\Glsf{validity}\footnote{Adapted AI summary of \textcite{wolming2010the-concept}.} is a concept that has evolved over time and has become more complex. Initially, \glsf{validity} was considered to be a fixed property of a test, with adequate correlations between test scores and an external criterion being seen as \glsf{evidence} of \glsf{validity}. However, the definition of \glsf{validity} has now expanded to include different types of \glsf{validity} related to the purpose of the test. These types include \emph{content \glsf{validity}}, \emph{criterion-related \glsf{validity}}, and \emph{construct \glsf{validity}}. 
%
%\paragraph{Content \glsf{validity}} is used for tests that describe an individual researcher's performance on a defined subject and is, for instance, concerned with the relevance and representativeness of items in a questionnaire.
%
%\paragraph{Criterion-related \glsf{validity}} is used for tests that predict future performance. It is established when a research tool can be compared to other similar validated measures for instance, when the results of two \glsplf{questionnaire} deliver by two independent samples agree\footnote{If not exactly, then at least within statistical bounds.}.
% 
%\paragraph{Construct \glsf{validity}} is used to make inferences about psychological traits.

\chapter{Your research strategy candidate list}\label{ch:ResearchStrategyCandidates}

While your own \glsf{research-strategy} will be specific and unique to your project in the way it informs the research you will conduct, standard research strategies have emerged over time, influenced by research mindsets and practice within specific disciplines. Each of them can be seen as a sort of \enquote{recipe} which summarises common ways to conduct \glsf{academic-research}: by adopting or combining some of these strategies, you can come up with your own specific instance for your project. 

There are many standard research strategies in the literature, often with many variants: this chapter discusses the 12 most widely applied strategies. The outcome of working through this chapter should be your choice of a candidate \glsf{research-strategy} that:
%
\begin{itemize}
\item is a good fit for your \glsf{research-problem}, i.e., that will allow you to develop a contribution to knowledge arising from your \glsf{research-problem}
\item makes the most of your current research skills and resources, i.e., the background knowledge and skills you bring to the research, the time that is available to you, and the context of your research
\item can be assessed by answering a set of evaluative questions.
\end{itemize}

From the first two, you will gain an understanding of which steps you will need to take to generate, analyse and interpret research \glsplf{datum} for your contribution to knowledge. 

The evaluative questions are those that could be asked by a knowledgeable evaluator, such as an examiner: your answers to those questions will help you describe and assess your strategy in your dissertation, particularly in relation to potential research weaknesses and how you dealt with them in your project.


\section{The process of choosing your candidate research strategy}

The 12 candidate research strategies considered in this book are listed in~\Cref{tab:ResearchStrategyChoice} and detailed in the following sections. For each, after a brief description, you will find:

%
\begin{itemize}
\item a discussion of what kind of knowledge contribution can be made through it, so that you can decide if it is a good fit for your \glsf{research-problem}
%\item describes its focus\footnote{meaning?}
%\item describe any variants that exist and the choices that constitute them
\item a description of the ways in which \glsplf{datum} is generated and analysed within the strategy, informing your choice of methods to use within the strategy
\item an indication of how a contribution to knowledge using the strategy will be evaluated, with a set of questions that you can also use to structure your dissertation narrative
\item the question \enquote{Is this strategy right for me?}, to help you establish how feasible might be for your to adopt the strategy in your project
\item a number of references that give more detail and that you should read\footnote{Your supervisor will also be able to suggest further reading.} if you are seriously considering the strategy.
\end{itemize}
%

%%Intentionally \ref, not~\Cref
\newcommand{\tick}{$\Box$}
\begin{SimpleNColTable}{tab:ResearchStrategyChoice}{5}{\Glsf{research-strategy} choice}[X[3,r]X[1,c]ccX[6]][row{2-Z}={ht=10ex},vline{5}={fg=\frameColor,wd=2pt},hlines={fg=\frameColor,wd=2pt}]
	\Glsf{research-strategy} candidate
		&Section
		&Considered
		&Excluded
		&Reason excluded\\
	\Glsf{survey-research}&\ref{sect:SurveyResearch}&\tick&\tick\\
	\Glsf{design-science-research}&\ref{sect:DesignScienceResearch}&\tick&\tick\\
	\Glsf{experimental-research}&\ref{sect:ExperimentalResearch}&\tick&\tick\\
	Case study&\ref{sect:CaseStudyResearch}&\tick&\tick\\
	\Glsf{action-research}&\ref{sect:ActionResearch}&\tick&\tick\\
	\Glsf{ethnography}&\ref{sect:ethnography}&\tick&\tick\\
	\Glsf{systematic-review} &\ref{sect:SystematicResearchReviews}&\tick&\tick\\
	\Glsf{grounded-theory}&\ref{sect:GroundedTheory}&\tick&\tick\\
	\Glsf{phenomenology}&\ref{sect:phenomenology}&\tick&\tick\\
	\Glsf{simulation}&\ref{sect:simulation}&\tick&\tick\\
	\Glsf{mathematical-logical-proof}&\ref{sect:MathematicalAndLogical}&\tick&\tick\\
	Mixed methods&\ref{sect:MixedMethodsResearch}&\tick&\tick
\end{SimpleNColTable}


This is a lot to digest! Therefore, rather than going through all the information about each strategy in turn, you should follow the process below and make good use of~\Cref{tab:ResearchStrategyChoice} to first reduce your list of candidate strategies from which to arrive at your chosen one. Here are the steps you should take:

\paragraph{Step 1} Consider a \glsf{research-strategy}, and read its description and type of knowledge contribution that can be made through it. Compare these with your \glsf{research-problem} to check whether that \glsf{research-strategy} should be a candidate for your project. When you have done this, you should check its tickbox in the first column of~\Cref{tab:ResearchStrategyChoice} – I've considered the strategy. If there's a clear mismatch with your \glsf{research-problem}, you should check the tickbox in the second column – that the \glsf{research-strategy} has been excluded – and give a reason why you have excluded it – say, the knowledge contribution it makes is not of the correct form – and you can move onto the next \glsf{research-strategy} and repeat this step. You will use the \enquote{\Glsf{reason} excluded} column in the your dissertation to justify your choice of \glsf{research-strategy}, so think deeply about what you write here – you can use the text of the knowledge contribution and subsequent subsections to frame your reason for excluding it. Whatever you do, don't leave it blank!

\paragraph{Step 2} If you have not excluded the \glsf{research-strategy}, then you should read further – next come the methods you would use to generate and analyse \glsplf{datum}. This gives you another reason to exclude a \glsf{research-strategy} – that the \glsplf{datum} or participants the strategy needs are not accessible or the methods are not feasible within your project\footnote{Of course, you will need to choose \emph{one} \glsf{research-strategy}, so be careful not to exclude something that, perhaps with some adjustment, can be made to work.}. If this analysis leads you to exclude the \glsf{research-strategy}, complete the second tickbox column and record the reason\footnote{This time, the reason will be something to do with \glsplf{datum} not be accessible or methods not feasible.}, then  move on to the next \glsf{research-strategy} going back to Step 1.

\paragraph{Step 3} If you have not excluded the \glsf{research-strategy}, then you should read the \enquote{Evaluation} section, which provides questions you should be able to address once your research is completed, but which you should keep in mind from the very start. While going through those questions, you should take notes on things you haven't immediately understood, to which you will return later on, should you include the strategy in your candidates list.  If you feel you are unlikely to be able to address those questions in your research, then this gives you yet another reason to reject the strategy. Once again, if you have excluded the strategy, tick the box in column 2, record the reason in column 3, then move on to the next strategy and restart the process from Step 1.

\paragraph{Step 4}  If you have not excluded the \glsf{research-strategy}, it's time to look at \enquote{Is this strategy right for me} section. This lists a number of other things you should consider that might lead you to exclude it, particularly in relation to skills you may need, or other features of the strategy which may not align with what you can achieve within your project. If you came to reject the strategy, as before, tick the box in column 2, record the reason in column 3, then move on to the next strategy and restart the process. This section may suggest alternative strategies you could consider next, otherwise, just proceed through the list.

\paragraph{Step 5} If you have not excluded the \glsf{research-strategy}, then look at the \enquote{Further reading} section and record the suggested references in your bibliographical database. You will access these articles later on, to gain a deeper understanding of your candidate research strategies. You can now move on to the next strategy and restart the process.

This process is wrapped up in the following activity, which constitutes the most substantial practical work for you to carry on in this chapter:

\begin{activity}[{Arriving at your candidate strategy list}]
Copy~\Cref{tab:ResearchStrategyChoice} to your favourite word processor or spreadsheet application. Apply the process above until you have considered all the strategies, updating your table as you go along, and recording related references in your BMT. 
\begin{guidance}
%Hack to correct tcbox behaviour
\color{black}


%Hack to correct tcbox behaviour
\color{black}
The aim of this activity is to help you narrow down the possible choices of candidate strategy for your project, without having to dive deep into the detail of all 12 strategies presented. A deep dive will come after you have completed this activity: the references recorded in your BMT will then provide a starting point for your review of methodology-related literature.
\end{guidance}
\end{activity}
%%Hack to correct tcbox behaviour
\color{black}

Once you have exhausted all the strategies, there are three possible outcomes:
%
\begin{itemize}
\item you find yourself with a single candidate \glsf{research-strategy}, in which case you should go for it!

\item you find yourself with a number of candidate research strategies, in which case you need to study more in order to make a choice. You may also like to think about mixing up bits of each to give you your own \glsf{mixed-methods-research} strategy.

\item you find yourself without a choice, in which case you've probably been too picky... and you should try again – you can't do research without a \glsf{research-strategy} and you're unlikely to come up with one not on this list.
\end{itemize}
%

In all cases, you should discuss the outcome with your supervisor.

\begin{activity}[{Discussing your choice with your supervisor}]
Arrange a time to talk to your supervisor about the process you have followed to identify possible choices of \glsf{research-strategy} for your project, and what the outcome was.	
\begin{guidance}
%Hack to correct tcbox behaviour
\color{black}


%Hack to correct tcbox behaviour
\color{black}
As an expert in the \glsf{research-process} and in your field of study, your supervisor will be able to advise on whether the choices you have made are appropriate, or even recommend which strategies you should consider in details.
\end{guidance}
\end{activity}
%%Hack to correct tcbox behaviour
\color{black}

\section{Survey research}\label{sect:SurveyResearch}

\Glsf{survey-research} focuses on collecting, in a standardised and systematic fashion, up-to-date, real-world \glsplf{datum} from a \glsf{sample}\footnote{A \glsf{sample} is a subset of \glsplf{datum} from a \glsf{population} of interest. You will return to sampling in~\Cref{stage4}.} of the \glsf{population} which is the focus of your research. Depending on the \glsf{population} and selected \glsf{sample}, you may need to collect large amounts of \glsplf{datum}.

\subsection{Knowledge contribution}\label{ssect:SurveyResearchKnowledgeContribution}

The contribution to knowledge of \glsf{survey-research} is to uncover patterns that can be generalised from the \glsf{sample} to the target \glsf{population}.

A typical application of \glsf{survey-research} is to predict the outcome of an upcoming election by polling \glsplf{datum} from a \glsf{sample} of voters.

%\subsection{Variants}\label{ssect:SurveyResearchVariants}

\subsection{Data generation and analysis}\label{ssect:SurveyResearchDataGenerationAnalysis} 

For your \glsplf{datum} generation, you need to identify upfront your target \glsf{population} and \glsf{sample}. The \glsf{sample} must be a group of participants representative of the \glsf{population}, in the sense that it should reflect accurately the \glsf{population} characteristics.

To collect \glsplf{datum} in a standardise and systematic manner, a survey uses a set of questions for the sampled participants in order to gather information about the \glsf{population}. This can be done through different methods, from interviews to \glsplf{questionnaire} to focus groups\footnote{Be aware that academic authors don't all agree on what is appropriate to use in a survey. When considering which methods to use in your survey, you must ensure that you meet your supervisor's (or other's) expectations of what is appropriate in your field of study.}, which can be applied face-to-face, or via the internet, or over the phone or even mail. You can also combine these options into a mixed-mode survey: for instance, you could use a broader but simpler questionnaire online to identify potential participants for a deeper face-to-face \glsf{interview} to follow.

In your \glsplf{datum} analysis, you seek patterns in the \glsplf{datum} collected from your \glsf{sample} to arrive at generalisations to the wider \glsf{population}. \Glsf{statistical-analysis} is usually applied, possibly complemented by some \glsf{thematic-analysis}, if open-ended questions are also included to elicit \glsplf{quantitative-datum}.


\subsection{Evaluation}\label{ssect:SurveyResearchEvaluation} 

The following questions are typically asked of \glsf{survey-research}:

	\begin{enumerate}
	\item reliability: 
		\begin{itemize}
		\item are \glsf{sampling-frame}\footnote{The \glsf{sampling-frame} is the set of individual units of the \glsf{population} from which the \glsf{sample} is drawn. Such individual units may be participants or \glsplf{datum} points in a \glsplf{datum} set.} and sampling techniques\footnote{We will look at sampling in~\Cref{stage4}.} adequately explained?
		\item are the \glsplf{datum} generation and \glsplf{analysis-method} adequately described?
		\item do the survey questions allow for consistent and dependable measures by different respondents? 
		\item are significant differences between respondents and non-respondents discussed?
		\end{itemize}
	\item \glsf{validity}: 
		\begin{itemize} 
		\item is the \glsf{sampling-frame} appropriate? Does it provide sufficient coverage of the target \glsf{population} in terms of its characteristics of interest?
		\item is the \glsf{sample} representative\footnote{This relates to the question of whether the \glsf{sample} is sufficiently large and/or as diverse as the \glsf{population}.}?
		\item is the response rate adequate? How were non-respondents handled?
		\item do the survey questions allow to measure or assess all that is needed? 
		\item has statistical, or other, analysis been appropriately applied?
		\item are generalisations made about the target \glsf{population} appropriate? What reasoning chains have led to such generalisations?
		\end{itemize}
	\item \glsf{bias}: 
		\begin{itemize} 
		\item is the questionnaire designed to avoid leading questions, which may unduly influence the respondents? 	
		\end{itemize}

	%\item are limitations or omissions in relation to any of the points above identified? Have their effect on the research and its outcomes been discussed?
	%\item has the \glsf{research-strategy} been successful in relation to the \glsf{research-aim} and objectives	
	\end{enumerate}

\subsection{Is this strategy right for me?}\label{ssect:SurveyResearchIsThisStrategy}

%\endinput

This strategy sets certain requirements of the researcher for them to be successful. These include that:
	\begin{enumerate}
	\item you must have access to an appropriate \glsf{population} \glsf{sample}, so that a sufficient volume of \glsplf{datum} can be collected and deep analysis performed. If this is not possible, for instance, because you have limited access to the \glsf{population}, you might like to consider \glsf{case-study-research} instead
	\item the \glsplf{phenomenon} and characteristics of the \glsf{population} which are of interest should be measurable through questions asked through a survey. If this is not the case that then you're not going to be able to make a contribution to knowledge about those \glsplf{phenomenon} or characteristics, and you might like to consider \glsplf{phenomenon} that can be measured, or a different \glsf{population} for which those \glsplf{phenomenon} can be measured
	\item while this strategy may produce lots of \glsplf{datum} in a relatively short time, the depth in the \glsplf{datum} can sometimes be lacking, given the focus on what can be measured. If deeper or more nuanced \glsplf{datum} is needed, then you may like to consider \glsf{case-study-research} instead
	\item conducting a survey means that you'll be analysing \glsplf{phenomenon} using \glsplf{point-datum}, i.e., \glsplf{datum} that were collected at a point in time – the time at which the survey was conducted. If your research requires \glsplf{longitudinal-datum}, i.e., \glsplf{datum} that could change over time, then \glsf{survey-research} becomes more difficult as you might need two or more surveys at different times to collect the changing \glsplf{datum}. While that's not impossible, it adds many complications: earlier participants might not be available for later surveys, their mindsets might have changed in the intervening period,~\etc{}. If this is the case, then you should consider whether your choice of \glsplf{phenomenon} is appropriate. Alternatively, you might like to consider \glsf{experimental-research}
	\item surveys are not suitable to investigate the mechanisms behind cause and effect relationships, for which you should use an \glsf{experimental-research} strategy instead 
	\item conclusions from \glsf{survey-research} rely on the veracity of the responses received, something you can't necessarily take for granted. Even when there is no intention to deceive, people's answers may be inaccurate due to many factors, including a tendency to wishing to provide the \enquote{right answers}, that is what they may believe is expected, or poor recall of past events or of detailed observations they have made\footnote{You encourntered \glsf{observation-bias} and \glsf{recall-bias} in \Cref{sect:biasInResearch}.}, or even lack of trust which may influence what they are willing to disclose. \Glsf{triangulation}, therefore, may be required to increase \glsf{validity}, but this will add complexity to the strategy. If this is not possible, then other strategies may be advisable, for instance, participant \glsf{observation} through ethnographic research.
	\end{enumerate}
	
\subsection{Further reading}\label{ssect:SurveyResearchFurtherReading}	

\ReadingList{survey research}{dillman2014internet,oates2008researching,johannesson2014research,kalaian2008encyclopedia}


\newcommand{\RSActivity}[2]{\begin{activity}[{Considering #1}]
	Having read the above section, do you consider #1 to be a serious candidate for your \glsf{research-strategy}?
%	
\begin{guidance}
%Hack to correct tcbox behaviour
\color{black}


%Hack to correct tcbox behaviour
\color{black}
	If so, add the following references to your list of reading:
\resourcelistcite{#2}
	\end{guidance}
	\end{activity}
%%Hack to correct tcbox behaviour
\color{black}}

%\RSActivity{Survey research}{dillman2014internet,
%	%oates2008researching,
%	%johannesson2014research,
%	kalaian2008encyclopedia}

\section{Design science research}\label{sect:DesignScienceResearch}

The \glsf{design-science-research} strategy\footnote{AKA {design-creation-strategy}.} focuses on developing novel solutions to problems, a problem being a need in context. The solution should be an artefact, by which is meant anything designed and constructed by humans: this is a very broad definition, encompassing all that does not exist in nature, including any artificial object, construct, process, policy, \glsf{model}, method,~\etc{}.

\subsection{Knowledge contribution}\label{ssect:DesignScienceResearchKnowledgeContribution}

The contribution to knowledge is that which can be learned from the design and creation of the artefact as the novel solution to a problem. Knowledge contributions therefore come from an exploration of the problem, of the artefact itself, and its design, development, use, or other characteristics of the real-world problem solving process -- for instance, whether it is linear or iterative,  or the ways in which problem and solution understanding and validation are conducted.

This strategy leads to tangible artefacts which fit real-world contexts, and is particularly suited to emerging and rapidly changing technology-related fields of study, where new problems emerge all the time and known solutions are sparse or become rapidly obsolete, hence necessitating continuous innovation. Lots of research in technology-related disciplines is an expression of this strategy, for instance designing computational systems able to emulate human cognition, as is the case of \glsf{artificial-intelligence}.

%\subsection{Focus}\label{ssect:DesignScienceResearchFocus}

%\subsection{Variants}\label{ssect:DesignScienceResearchVariants}

\subsection{Data generation and analysis}\label{ssect:DesignScienceResearchDataGeneration}

\Glsplf{datum} generation is through the problem-solving process of articulating the problem, and designing and constructing the solution artefact, with the interactions between actors (customers, clients, designers, \ldots), technologies and/or knowledge as the source of \glsplf{datum}. Modelling methods are widely applied, possibly informed by \glsplf{datum} generation methods, like reviewing existing \glsplf{document} or interviewing stakeholders and experts, or observing people's behaviour. Prototyping is often used to produce proof-of-concept artefacts to test, demonstrate and improve the artefact design. 

\Glsplf{datum} analysis focuses on knowledge generated in the evaluation of both problem and artefact, including solution characteristics in relation to the extent they address the  problem. Specific evaluation techniques will depend on the nature of the artefact, and may include problem owner's validation\footnote{A problem owner is the person or group of people who have expressed the need to be addressed and are able to establish whether the solution has met it.}, various forms of testing, or end-users' evaluation and feedback.

\subsection{Evaluation}\label{ssect:DesignScienceResearchEvaluation}

Evaluation of \glsf{design-science-research} typically consists of the following questions:
					%
\begin{enumerate}
	\item \glsf{novelty}: 
		\begin{itemize}
			\item what is the novelty in the artefact, its design, development, and/or creation?
			\item to which extent does the artefact address the problem? Have its efficacy and utility been demonstrated? What \glsf{evidence} is provided?
		\end{itemize}
	\item reliability: 
		\begin{itemize}
			\item are all stages of the problem solving process discussed, including interactions with stakeholders?
			\item are the ways \glsplf{datum} are generated and analysed, both in problem and solution space, adequately described?
		\end{itemize}
	\item \glsf{validity}: 
		\begin{itemize}
			\item are appropriate approaches applied in the design and creation of the artefact?
			\item how is the artefact assessed in its real-world context? Are the assessment criteria appropriate and documented? How were they determined?
			\item which generalisations are made from the design and creation of the artefact? Are they appropriate?
		\end{itemize}
%	\item are limitations or omissions in relation to any of the points above identified? Have their effect on the research and its outcomes been discussed?
%
%	\item has the \glsf{research-strategy} been successful in relation to the \glsf{research-aim} and objectives?
	\end{enumerate}
	

\subsection{Is this strategy right for me?}\label{ssect:DesignScienceResearchIsThisStrategy}

%\endinput
For this strategy to be successful:

\begin{enumerate}
\item there must be demonstrable novelty. You must be able to argue that your research does not focus on \glsf{normal-design}, that is you are not simply re-implementing a solution to a well-known problem through a well-known development process and well-practiced skills\footnote{Learning new skills may be valuable from a personal perspective, but will not, by itself, make a contribution to knowledge – learning them means that they exist already!}. If you cannot clearly identify that novelty, then you will not be able to \glsf{claim} a contribution to knowledge
\item there should be a problem owner which is separate from the researcher, and who sets the need and context for the artefact, with the researcher working on its development for that context to meet that need. If you do not have access to a real-world problem owner then this strategy is not applicable
\item if the problem owner is, say, your employer or a business you are collaborating with, and for which addressing the problem is a matter of urgency, then you must establish whether it is feasible for you to deliver a novel solution in a timely fashion. Research always brings a level of uncertainly so that estimating time to success, or if success is even possible may not be easy. If you can't ensure feasibility within the timescale of your project, then you may need to rethink the problem to address.
\end{enumerate} 


\subsection{Further reading}\label{ssect:DesignScienceResearchFurtherReading}

\ReadingList{design science research}{oates2008researching,brocke2020introduction}

%\RSActivity{Design and creation \glsf{research-strategy}}{oates2008researching,brocke2020introduction}

\section{Experimental research}\label{sect:ExperimentalResearch}

\Glsf{experimental-research} provides a controlled environment in which a cause and effect relationship can be investigated, expressed as a \glsf{hypothesis}, which is a tentative statement about the causal relationship to be tested in the experiment. 

Experiments should be designed to control the influence of \glsplf{confounding-factor} on a cause-effect relationship, these being some other factors not measured in the experiment that may influence cause and effect.
%From wikipedia:

%\textcquote{enwiki:1195800578}{An experiment is a procedure carried out to support or refute a \glsf{hypothesis}, or determine the efficacy or likelihood of something previously untried. Experiments provide insight into cause-and-effect by demonstrating what outcome occurs when a particular factor is manipulated. Experiments vary greatly in goal and scale but always rely on repeatable procedure and logical analysis of the results.}

The potential scope of application of the \glsf{experimental-research} strategy is wide, ranging from scientific experiments under laboratory conditions controlled by the researcher to field experiments involving people in a real-world setting in which some factors may be outside the control of the researcher. 

There are pros and cons of each. While laboratory experiments are very reliable due to the high level of control, they can be very artificial, with little or no relation to a real-world context. The opposite is true for field experiments.


\subsection{Knowledge contribution}\label{ssect:ExperimentalResearchKnowledgeContribution}

The \glsf{experimental-research} strategy contributes to knowledge by establishing cause and effect relationships between real-world \glsplf{phenomenon}. 

For instance, you may run an experiment to test whether the use of mobile phones just before going to sleep disrupts people's sleeping patterns.

%\subsection{Focus}\label{ssect:ExperimentalResearchFocus}
%
%\subsection{Variants}\label{ssect:ExperimentalResearchVariants}
%
%There are many variants of the \glsf{experimental-research} strategy, including\footnote{Add context of application for each}:
%%
%	\begin{description}
% 	\item [True] see~\textcite[p.126]{oates2008researching}
%	\item [Quasi] see~\textcite[p.133]{oates2008researching}
%	\item [Uncontrolled] see~\textcite[p.134]{oates2008researching}
%	\end{description}
%	%
%\noindent and, even more specialised, for social science applications
%	%
%	\begin{description}
%	\item [One group, pre-test and post-test] see~\textcite[p.135]{oates2008researching}
%	\item [Static group comparison] see~\textcite[p.135]{oates2008researching}
%	\item [Pre-test/post-test control group] see~\textcite[p.135]{oates2008researching}
%	\item [Solomon four-group design] see~\textcite[p.126]{oates2008researching}
%	\item see \textcite{field2002design}
%	\end{description}
%	
%%\endinput
		

\subsection{Data generation and analysis}\label{ssect:ExperimentalResearchDataGeneration} 

\Glsf{experimental-research}  revolves around making an intervention within tightly controlled parameters. You need to make observations and measurements before and after the intervention and then compare them: any difference is then assumed associated with the intervention made. 

For instance, in establishing a causal relation between the use of mobile phones and sleeping patterns, your could investigate the effect of the blue light emitted by a mobile phone on reducing the production of melatonin: this is the hormone which controls a person's sleep-wake cycle, so that its reduction is likely to disrupt a person's sleeping pattern. You would then measure the amount of melatonin produced by the body (these are your measurements) with and without exposure to the blue light of a mobile phone (this is the intervention), then analyse any difference to establish whether a causal relation exists.

So, you generate \glsplf{datum} through observations and measurements under different experimental conditions, and analyse your experimental \glsplf{datum} to explain causal relationships between the \glsplf{phenomenon} under study. 

Depending on the complexity of the relationship between cause and effect, you will need more or less complex experimental designs. Those involving an inaccessibly large \glsf{population} of individuals, as might be the case for a medical drug trial, use sophisticated techniques to choose representative samples, as well as sophisticated \glsf{statistical-analysis} to test \glsplf{hypothesis}.

However, even simpler cause-effect \glsplf{hypothesis} may rely on the availability of a fully equipped scientific laboratory to work.


\subsection{Evaluation}\label{ssect:ExperimentalResearchEvaluation}

Typical questions in the evaluation of \glsf{experimental-research} include:
	
	\begin{enumerate}
	\item \glsf{reliability}: 
		\begin{itemize}
			\item are the experimental variables manipulated or measured adequately described?
			\item is there a clear account of what is controlled?
			\item what are the experimental procedures? Are they sufficiently detailed so that the experiment can be repeated by an independent third party?
			\item in a social setting, what information is given about  participants and how they were found?
			\item what information is given about the apparatus and the process used for measurements?
		\end{itemize}
	\item \glsf{validity}: 
		\begin{itemize}
			\item was a \glsf{hypothesis} or predicted outcome of the experiment clearly stated?
			\item if a \glsf{population} \glsf{sample} was selected for the experiment, how representative is it? How was it selected? Which measures were taken to avoid \glsf{sample-bias}\footnote{\Glsf{sample-bias} occurs when some elements of the \glsf{population} are more likely to be selected than others.}?
			\item if \glsf{statistical-analysis} is applied, how adequate is it? Have appropriate statistical tools been used and their use justified?
			\item are confounding factors or outliers\footnote{In simple terms, an outlier is a \glsplf{datum} point that sticks out by being very different from the others. You will learn about outliers in the context of \glsf{statistical-analysis} in~\Cref{stage4}.} identified and discussed?
			\item are the statistical and other analyses convincing of the conclusions?
			\item has the experiment being replicated?
		\end{itemize}
	%	\item are limitations or omissions in relation to any of the points above identified? Have their effect on the research and its outcomes been discussed?
%	\item has the \glsf{research-strategy} been successful in relation to the \glsf{research-aim} and objectives?
	\end{enumerate}

\subsection{Is this strategy right for me?}\label{ssect:ExperimentalResearchIsThisStrategy}

Although widely applicable, \glsf{experimental-research} has some counter-indications:
%
	\begin{enumerate}
	\item when a testable \glsf{hypothesis} cannot be formulated, concerning the cause-and-effect relationship of interest
	\item when the cause/effect relationship is very complex, for instance, depending on many factors, which cannot be accounted for in an experiment 
	\item when confounding factors and variables cannot be isolated, or no level of control is possible
	\item when the experiment is a one-off and cannot be repeated
	\item if you don't have access to specialised equipment required
	\item if you don't have (or can't develop) \glsf{statistical-analysis} skills required.
	\end{enumerate}


\subsection{Further reading}\label{ssect:ExperimentalResearchFurtherReading}

\ReadingList{experimental research}{oates2008researching,johannesson2014research,field2002design}


\section{Case study research}\label{sect:CaseStudyResearch}

\Glsf{case-study-research} proceeds through the in-depth study of a notable instance of a \glsf{phenomenon} in its real-world context, particularly when it not possible to separate the \glsf{phenomenon} from that context. The study of a single \glsf{phenomenon} requires the researcher to delve deeply into the context of that \glsf{phenomenon}, whether that be a project, an organisation, an engineered \glsf{system}, a policy, an economic or historical setting, or other. Case studies allow you to study complex \glsplf{phenomenon} where several factors are at play, and to explore alternative meanings and explanations.  

\subsection{Knowledge contribution}\label{ssect:CaseStudyKnowledgeContribution}

Case studies focus on the \enquote{how?} and \enquote{why?}, so that the knowledge contribution is a detailed insightful account of the \glsf{phenomenon} in its context, including its relationships with other relevant \glsplf{phenomenon}, and any causal chain. 

What you seek with a case study can span from exploring possible questions or \glsplf{hypothesis} for follow-up research, to explaining why certain outcomes have occurred, to investigate changes over time. For instance, an example of case study could be a detailed investigation of the US Equifax social security breach of 2017, in which 143 million of their consumer records were stolen by hackers. This may be descriptive of the chain of events that took place or explicative of why things happened the way they did, or both.

Therefore, case studies come in many forms, including:

	\begin{itemize}
	\item exploratory: in which the researcher explores a \glsf{research-problem} sufficiently to be able to conduct a further study. If you are conducting master's research and are considering a Doctorate afterwards, then this might provide a head-start for your future research
	\item multiple: in which two or more instances of the \glsf{phenomenon} are chosen, which present both similarities and differences, to provide an even richer analysis of the \glsf{phenomenon} in its context 
	\item longitudinal: in which the researcher considers the state of a \glsf{phenomenon} over time. This offers a natural storytelling context for you to analyse changes in the \glsf{phenomenon} and/or its context.
\end{itemize}

Combinations of the above are also common, allowing even deeper exploration of both relationships between \glsplf{phenomenon} and how they develop over time or in response to contextual factors.
		
%and have many possible foci \parencite[p.44, adapted]{johannesson2014research}:
%	\begin{itemize} 
%	\item focus on One Instance: in which a single  The idea is \enquote{To see a World in a Grain of Sand, And a Heaven in a Wild flower, Hold Infinity in the palm of your hand, And Eternity in an hour} as expressed by William Blake (2012).
%	
%	\item focus on Depth. As much information as possible about the instance studied should be obtained, without shying away from any details.
%	
%	\item natural Setting. The instance exists before and independently of the research project, and it should be studied in its ordinary context; it should not be moved to, or created in, a laboratory.
%	
%	\item relationships and Processes. The instance should be studied in a holistic way, taking into account all the relationships and processes within the instance as well as in its environment.
%	
%	\item multiple Sources and Methods. Multiple information sources should be consulted in order to obtain rich, many-faceted knowledge about the instance; when doing this, different \glsplf{datum} collection methods could be used, such as interviews and \glsf{observation}.
%	\end{itemize}
					
\subsection{Data collection and analysis}\label{ssect:CaseStudyDataCollection}

Case studies require you to collect \glsplf{empirical-datum}\footnote{\Glsplf{empirical-datum} comes from \glsf{observation} or experience.} from a great variety of sources, and to focus on depth rather than breadth. Therefore, all \glsplf{datum} collection techniques which allow you to do so may be used, from participant \glsf{observation}, to surveys of those who experience the \glsf{phenomenon} in context (through interviews, \glsplf{questionnaire}, \etc{}.), allowing for multiple \glsf{stakeholder} views to be taken into account, to studying forensically existing \glsplf{document} that directly or indirectly describe the \glsf{phenomenon}. This will lead to much \glsplf{datum} to be collected --- mainly qualitative, but also quantitative to some extent, so that their analysis can be very rich and complex.

\subsection{Evaluation}\label{ssect:CaseStudyEvaluation}

An experienced researcher evaluating \glsf{case-study-research} will ask the following questions:

	\begin{enumerate}
	\item \glsf{reliability}: 
		\begin{itemize}
			\item has the type of case study conducted been clearly described and justified?
			\item how was a detailed investigation of the \glsplf{phenomenon} conducted? Was the researcher able to work within the case study context?
			\item how were ethical considerations taken into account, particularly in relation to participants and confidential information handled?
			\item were the \glsplf{datum} generation and \glsplf{analysis-method} adequately described?
			\item are the procedures followed appropriately documented?
		\end{itemize}
	\item \glsf{validity}: 
		\begin{itemize}	
			\item have the criteria for choosing the particular case study been described and justified? Is the choice appropriate for the \glsf{phenomenon} studied?	
			\item did the  methods generate the right type of \glsplf{datum} about the \glsf{phenomenon} in sufficient depth and quantity?	
			\item does the research adequately describe the relationships between \glsplf{phenomenon} and the processes in which the \glsplf{phenomenon} participate?
			\item is the \glsplf{datum} analysis systematic and transparent? Are the steps taken to arrive at  conclusions clearly explained? 
			\item what generalisations were made from the \glsf{case-study-research}? Are they appropriate for the \glsf{phenomenon} and its context?
		\end{itemize}
	%\item what use of \glsf{theory} of the \glsf{phenomenon} is made in the case study? IS the \glsf{theory} chosen appropriate? If no \glsf{theory} was used, how is the theoretical basis of the research covered?
	
%	\item are limitations or omissions in relation to any of the points above identified? Have their effect on the research and its outcomes been discussed?
%
%	\item has the \glsf{research-strategy} been successful in relation to the \glsf{research-aim} and objectives?
	\end{enumerate}
	
%\endinput
					
\subsection{Is this strategy right for me?}\label{ssect:CaseStudyIsThisStrategy}

There are conditions for this strategy to be successful:

\begin{enumerate}
\item \glsf{case-study-research} requires you to have access the \glsf{phenomenon} in its context to be able study it holistically and generate rich, detailed descriptions. As an example, if you're not a teacher, it might be difficult to gain access to a classroom to study student/teacher interactions. If access is an issue, then you should consider a different strategy, like systematic reviews, which work from secondary sources
\item access to \glsplf{datum} sources, such as policy, processes or procedures within an organisation, may rely upon interaction with others. Even if you already have a good relationship with them they might not have the time to assist you sufficiently for your \glsplf{datum} generation to be successful within the timeline of your project. If time is an issue you should consider alternative sources, or even a different \glsf{research-strategy} 
\item being embedded within the context of the \glsf{phenomenon}, as might be the case, for instance, of an employee of an organisation, facilitates the investigation of the \glsf{phenomenon}. In this case, however, alternative research strategies are also applicable, such as \glsf{ethnography} or \glsf{action-research}
\item you must have the required knowledge to understand the \glsplf{phenomenon} under study. For instance studying the processes by which an engine controller in an aircraft is designed may require a detailed understanding of technical documentation, language and even mathematical or computational theories. Acquiring this knowledge from zero as part of your research may not be possible or may consume too much time\footnote{The success of your research will depend critically on climbing any learning curve quickly and successfully, even if that learning curve looks like El Capitan!}. In such cases you should reconsider the \glsf{phenomenon} to study
\item you need to make a judicious choice of case study to be able to make any generalisations about the \glsf{phenomenon} beyond the particular instance. If you don't have access to a significant instance of the \glsf{phenomenon} and \glsf{generalisation} is an important consideration, then you should consider a different \glsf{research-strategy}.
\end{enumerate}


%%
%\begin{enumerate}%[start=0,label={(\bfseries R\arabic*):}]
%\item large amounts of \glsplf{datum} are needed: use ??? instead
%\item ???
%\end{enumerate}
%			

\subsection{Further reading}\label{ssect:CaseStudyFurtherReading}

\ReadingList{case study research}{yin2009case,oates2008researching}

%\RSActivity{Case study research}{oates2008researching}

\section{Action research}\label{sect:ActionResearch}
  
\Glsf{action-research} focuses on real-world situations for which improvement is sought through participatory and collaborative research. Its focus is on practice change, and continuous learning and improvement via an iterative \glsf{plan-act-reflect-cycle} cycle which generates both knowledge and action: this is an iterative process in which you plan what to do to improve the situation, then enact the plan and reflect on the outcome to inform the next iteration.

\subsection{Knowledge contribution}\label{ssect:ActionResearchKnowledgeContribution}

\Glsf{action-research} should make both a contribution to knowledge \textit{and} to practice: an action researcher strives to generate knowledge and action to address important problems that people experience in their practices, so that the knowledge contributed originates in real-world needs. The researcher is an active participant in the research, rather than solely an observer, alongside other collaborating practitioners: in fact, collaboration and \glsf{reflexivity} are essential elements of this strategy.

The outcomes of \glsf{action-research} may be new theories or methods alongside their direct implementation to improve practice within a specific professional or social setting. For example, in an educational setting, where this strategy is widely applied, a group of teachers may come together to study the composition and effectiveness of homework at each school grade, with a view to improve the balance between knowledge-based and practice-based learning.

%\subsection{Focus}\label{ssect:Focus}

%\subsection{Variants}\label{ssect:Variants}

\subsection{Data generation and analysis}\label{ssect:ActionResearchDataGeneration}

Similar to case studies, \glsf{action-research} requires you, and your collaborators, to collect \glsplf{empirical-datum} from a variety of sources to gain a deep understanding of the current practical situation to be improved. Common methods which allow you to do so include observations, questionnaires, focus groups and document reviews, and you will collect and analyse both qualitative and quantitative \glsplf{datum}. You may also need to design your intervention, so that you could use modelling methods this purpose.


\subsection{Evaluation}\label{ssect:ActionResearchEvaluation}

The evaluation of \glsf{action-research} will include the following questions:

	\begin{enumerate}
	\item \glsf{reliability}: 
		\begin{itemize}
			\item did the work used a \glsf{plan-act-reflect-cycle} cycle? How many iterations were conducted?
			\item was the research collaborative? Is the level of collaboration achieved appropriate?		
			\item were the \glsplf{datum} generation and \glsplf{analysis-method} appropriately described?
		\end{itemize}	
	\item \glsf{bias}: 
		\begin{itemize}
			\item have the researcher's personal stake and potential \glsplf{bias} been discussed? Was a \glsf{reflexivity} account included?
			\item is there a \glsf{reflection} on self-delusion and \glsf{groupthink}\footnote{\Glsf{groupthink} is a tendency to conform to majority option to maintain unanimity and avoid confrontation.}? How were these successfully mitigated?
		\end{itemize}	
	\item \glsf{validity}: 
		\begin{itemize}
			\item was the learning from the \glsf{plan-act-reflect-cycle} cycle clearly identified and discussed?
			\item were the \glsplf{datum} generation methods appropriate? Was enough \glsplf{datum} generated?
			\item were detailed descriptions and accounts of findings provided?
			\item has the research generated both knowledge and action leading to change? How useful or impactful on practice are they?
			\item were generalisations made and appropriately supported by \glsf{evidence}, including \glsf{triangulation}? 
		\end{itemize}
%	\item are limitations or omissions in relation to any of the points above identified? Have their effect on the research and its outcomes been discussed?
%
%	\item has the \glsf{research-strategy} been successful in relation to the \glsf{research-aim} and objectives?
	\end{enumerate}
					
\subsection{Is this strategy right for me?}\label{ssect:ActionResearchIsThisStrategy}

There are conditions for this strategy to be successful:

\begin{enumerate}
	\item \glsf{action-research} focuses on action aimed at solving real-world problems in professional and other social contexts. If that's not the case for your research, then you should consider a different strategy
					
	\item the action researcher is expected to be embedded in the context in which the research takes place, and have a professional stake in addressing the problem beyond the research itself, for instance, as an employee of an organisation. If this is not the case for your project, then you should consider case studies instead
				
	\item \glsf{action-research} requires you to involve other practitioners as collaborators in the research. This goes beyond being merely participants in surveys or observations: instead it requires a much higher commitment and continuous involvement in the study. If this is not possible, you should consider \glsf{case-study-research} instead
									
	\item \glsf{action-research} works through \glsf{reflection}, \glsf{reflexivity}, and continuous learning and improvement. As such, it does not exhibit the same level of scientific rigour as, for instance, an experiment. If scientific rigour is needed in your research, then you should consider a different strategy 
					
	\item \glsf{action-research} may not be suitable to study complex causal-effect relationships. If you need to establish one such relationship, then you should consider \glsf{experimental-research} instead
					
	\item \glsf{generalisation} can be difficult to achieve with \glsf{action-research}. If you need to be able to generalise your research widely, then consider case studies instead
							
	\item while \glsf{action-research} is accepted and commonly applied in some social sciences, like education and healthcare, this may not be the case in your discipline. You should therefore check with your supervisor whether this strategy is acceptable or you should consider a different strategy which is more widely applied in your discipline. 
					
	%\item will the organisation in which you are embedded require you to work for them, rather than to conduct research? If so, ensure they are clear that you are not a consultant.
				
	%\item do envisage issues working with others in complex, problematic and unpredictable real-world situations? If so, consider ???
\end{enumerate}

\subsection{Further reading}\label{ssect:ActionResearchFurtherReading}

\ReadingList{action research}{johannesson2014research,oates2008researching}


\section{Ethnography}\label{sect:ethnography}

\Glsf{ethnography} aims to study the culture of a group of people in their natural setting. 

\subsection{Knowledge contribution}\label{ssect:EthnographyKnowledgeContribution}

\Glsf{ethnography} contributes to knowledge by providing a cultural characterisation of the group under study. Such characterisation should be one that the group members recognise and find familiar, and should be inclusive of various cultural facets, both social and economical, rather than focusing solely on one specific aspect.

While \glsf{ethnography} was originally developed within the discipline of anthropology, particularly for the study of \glsf{indigenous} populations, it can be applied widely in social settings, for instance to study the work culture within a particular profession or organisation, or the culture of online communities within social media. 

%\subsection{Focus}\label{ssect:EthnographyFocus}

%\subsection{Variants}\label{ssect:EthnographyVariants}

\subsection{Data generation and analysis}\label{ssect:EthnographyDataGeneration}

The researcher is expected to join the group and share what the group members' experience in their natural social setting in order to gain an insider's perspective and arrive at a rich characterisation. This requires the researcher to make detailed participant observations, appropriately recorded in field notes, accompanied by gathering detailed \glsplf{datum} through interviews and document reviews, linking them to the existing literature and reflecting on what they have learnt from their own experience, including using \glsf{reflexivity} to examine their state of mind and emotional reactions. \Glsplf{datum} generation and analysis are predominantly qualitative.

\subsection{Evaluation}\label{ssect:EthnographyEvaluation}

Evaluating \glsf{ethnography} may involve asking the following questions:

	\begin{enumerate}
	\item \glsf{reliability}: 
		\begin{itemize}
			\item are field notes sufficiently rich and detailed? Do they capture people’s actions and behaviours, and their motivating reasons?
		\end{itemize}	
	\item \glsf{validity}:
		\begin{itemize}
			\item is the cultural characterisation obtained sufficiently rich to account adequately for the group's beliefs, customs, behaviours and interpersonal relations?
			\item was adequate time spent with the group in their natural setting? What \glsf{reflection} has been done on such time?
			\item are \glsplf{datum} appropriately interpreted through a cultural lens?
			\item has the ethnographic characterisation been linked to \glsf{theory}, other ethnographic studies or issues in other cultures?
		\end{itemize}
	\item \glsf{bias}: 
		\begin{itemize}
			\item does the research include a reflexive account?
		\end{itemize}		
	
	%\item to which extent is the research presented as an ethnographic construction rather than as a literal description?	
%	\item are limitations or omissions in relation to any of the points above identified? Have their effect on the research and its outcomes been discussed?
%
%	\item has the \glsf{research-strategy} been successful in relation to the \glsf{research-aim} and objectives?
	\end{enumerate}					

\subsection{Is this strategy right for me?}\label{ssect:EthnographyIsThisStrategy}

There are conditions for this strategy to be successful:

	\begin{enumerate}
	\item \glsf{ethnography} requires you to be a researcher located within the context of your situated research, for instance your employer organisation, where the research is likely to require an extensive amounts of time. If you have yet to  identify the context, or are unlikely to obtain permission to proceed, then ethnographic research may not be feasible, in which case you may wish to consider \glsf{case-study-research} instead
	\item even if you are already located within the context of your ethnographic research, the context must be accepting of an ethnographic approach for your research to be successful. An organisation, for instance, in which there is a culture of strict compartmentalisation may not provide sufficient opportunities for ethnographic research. In such cases, you should consider case studies instead 
	\item in ethnographic research you allow the culture under study to determine the outcomes of the research, so you should approach it without any preconception or \glsf{bias}. If there is any possibility that you could be biased to a particular outcome – as might happen if you feel you already know the outcome and are simply trying to confirm this – then \glsf{ethnography} is unlikely to lead to a successful outcome for your research. Any competent ethnographer will be particularly sensitive to expressions of such \glsf{confirmation-bias}
	\item \glsf{ethnography} can lead to rich descriptions of complex social settings, and the characterisation produced may be very deep in representing a particular group culture. However, this may be difficult to generalise to other social groups or settings. If \glsf{generalisation} is an important aspect of your research, then you should consider case studies instead.
	%\item \glsf{ethnography} is analytical in the extreme. Should you not have an analytical mindset, then \glsf{ethnography} should be avoided.
	\end{enumerate}

\subsection{Further reading}\label{ssect:EthnographyFurtherReading}

\ReadingList{ethnography}{johannesson2014research,oates2008researching}

\section{Systematic review}\label{sect:SystematicResearchReviews}

A \glsf{systematic-review} is used to generate new insights from published work, linked to a clearly defined \glsf{research-problem} or question.

\subsection{Knowledge contribution}\label{ssect:SystematicResearchReviewsKnowledgeContribution}

A \glsf{systematic-review} is meant to advance a field of study by providing insights from across the literature not contained in individual research papers. You need to use a rigorous set of criteria to identify, select, and critically appraise relevant research from previously published studies in order to generate a scholarly \glsf{synthesis} of the \glsf{evidence} in relation to an explicit \glsf{research-problem} or question. 

For example, a \glsf{systematic-review} could be conducted to generate new insights on the effectiveness of a specific medical treatment, in order to advance evidence-based medicine: this would require you to review published articles on randomised controlled trials for that treatment and form a judgement  based on a \glsf{synthesis} of the results from the accumulated body of work.

				
%\subsection{Focus}\label{ssect:SystematicResearchReviewsFocus}
%\subsection{Variants}\label{ssect:SystematicResearchReviewsVariants}

\subsection{Data collection and analysis}\label{ssect:SystematicResearchReviewsDataCollection}

In a \glsf{systematic-review} you only use \glsf{evidence} from published studies and rely on explicit, reproducible techniques to identify the relevant research to review. Specifically, you must decide upfront your \glsf{research-problem}/question and the set of criteria you will use to select, compare and evaluate those studies, and combine their results. 

The type of analysis you will conduct will depend on the nature of the \glsf{evidence} you are considering and combining. Specifically, a \glsf{narrative-review} produces a narrative \glsf{synthesis} of qualitative results, while \glsf{meta-analysis} uses statistical techniques to analyse and combine quantitative results. Combinations of the two are also possible. 
				
\subsection{Evaluation}\label{ssect:SystematicResearchReviewsEvaluation}

Evaluation of a \glsf{systematic-review} will involve answers to the following:

	\begin{enumerate}
	\item \glsf{reliability}: 
		\begin{itemize}
			\item are the criteria used to select, exclude, evaluate and combine the published research explicit and reproducible? Were there any deviations from this protocol and, if so, are they documented, explained and justified?
		\end{itemize}
	\item \glsf{validity}: 
		\begin{itemize}
			\item has the researcher accessed all relevant published research in the area?
			\item have the relative strengths and weaknesses of the research reviewed been described? To which extent have conflicts between sources been identified and, when appropriate, resolved?
			\item in combining results from different studies, are significant differences between those studies appropriately acknowledged?
			\item to which extent has a definitive \glsf{synthesis} from the literature been achieved? To which extent are the limits of current knowledge described?
			\item to which extent has precision and/or generalisability been improved through the \glsf{systematic-review}? 
			\item in \glsf{meta-analysis}, to which extent have statistics been used to produce overarching conclusions? Were the studies sufficiently homogeneous for \glsf{meta-analysis} to be feasible?
		\end{itemize}
	\item \glsf{bias}: 
		\begin{itemize}
		\item in narrative reviews, to which extent potential \glsf{bias} has been acknowledged and mitigation measures applied?
		\end{itemize}
%	\item are limitations or omissions in relation to any of the points above identified? Have their effect on the research and its outcomes been discussed?
%
%	\item has the \glsf{research-strategy} been successful in relation to the \glsf{research-aim} and objectives?

	%\item given the research question, to which extent is the review type the most appropriate ~\parencite[p.142]{pollock2018systematic}?
	
	%\item have implications for future research and practice been discussed both for the research question and in other areas that were raised by the review?
	
	%\item do you discuss the updating of the review~\parencite[p.142]{pollock2018systematic}?
		\end{enumerate}

%{\parencite[adapted]{wright2007write}: 
%%
%	\begin{enumerate}%[start=0,label={(\bfseries R\arabic*):}]
%%	\item clarifying the relative strengths and weaknesses of the literature on the question, 
%%	\item summarising a large amount of literature, 
%%	\item resolving literature conflicts, 
%%	\item evaluating the need for large interventions, 
%%	\item avoiding a redundant unnecessary interventions, 
%%	\item increasing the statistical power of smaller studies, 
%	\item improving the precision or identify of smaller interventions, and 
%	\item improving the generalisability of intervention outcomes.
%	\end{enumerate}
%	%
					
\subsection{Is this strategy right for me?}\label{ssect:SystematicResearchReviewsIsThisStrategy}

You should consider the following points when choosing this strategy:

\begin{enumerate}
	\item a \glsf{systematic-review} is both systematic and extensive in its coverage of the topic of interest. This requires you to have a very good grasp of the subject area in order to establish appropriate criteria for the selection of all relevant published work, and this may lead to a large body of work to review. If you lack such knowledge of the filed of study or the time to conduct an extensive review of the literature, then you should consider a different strategy
	\item a systematic research review assumes that there is a substantial body of knowledge already published from which new insights can be generated. In emerging research areas, this may not be the case, so that a systemic research review is unlikely to reach any meaningful \glsf{synthesis}. If there is paucity of literature on your chosen topic, then this strategy is not for you
	\item you need relatively easy access to the \glsf{academic-literature} to be able to select the body of work to review, for instance, through a university library with a large research collection in your chosen discipline. If not, you will need to devise alternative ways to access the relevant literature, like contacting the author(s) directly. Although most authors will be happy to send their published research to you, the round trip time can introduce lengthy delays in the \glsf{research-process} as you wait for the requested research to arrive. You may also need to be persistent to ensure that a busy author is aware of your research needs. If you don't think you will be able to access easily a large proportion of the published work you need, you should rethink your \glsf{research-strategy}
	\item systematic research reviews are required to be transparent, reliable, and easy to replicate. You will be expected to have stated explicit \glsf{inclusion} and exclusion criteria so that another researcher shouuld be able to arrive at the same collection of published work, and ensure that no \glsf{bias} has influenced such choice. Choosing appropriate criteria may be difficult and may require you to iterate, starting perhaps from a broader focus, then narrowing it down as your research progresses. This, too, can be time consuming, so if your time is limited, you should consider a more time-efficient strategy for your research, perhaps one which allows you to generate your own \glsf{primary-evidence}.
	\end{enumerate}


%	\begin{itemize}
%%	\item unable to consult all relevant literature; if so perhaps you're only trying to complete a preliminary literature review? \parencite[p.414]{andrews2005place}
%%	\item identified focus is too narrow to provide generalisable results~\parencite{wright2007write}
%%	\item difficult to construct strict \glsf{inclusion} and exclusion criteria for the survey
%%	\item no meaningful conclusion reached due to paucity of literature
%%	\item study heterogeneity precludes \glsf{meta-analysis}, the authors of the \glsf{systematic-review} need to summarise the findings based on the strength of the individual studies and reach conclusions if indicated~\parencite[p.27]{wright2007write}
%%	\item won't be more comprehensive than what exists~\parencite[p.405]{andrews2005place}
%%	\item there is some latent \glsf{bias} possible~\parencite[p.405]{andrews2005place}
%%	\item it will not transparent or nor replicable~\parencite[p.405]{andrews2005place}
%	\end{itemize}
%	%
								
\subsection{Further reading}\label{ssect:SystematicResearchReviewsFurtherReading}

\ReadingList{systematic review}{wright2007write,moher2009preferred,pollock2018systematic}
 
In addition, you should consider the \glsf{PRISMA} statement\footnote{http://prisma-statement.org/prismastatement/checklist.aspx}, a 27-item checklist to help authors improve the reporting of narrative reviews and meta-analyses.
					
%\RSActivity{systematic research survey}{wright2007write,moher2009preferred,pollock2018systematic}


%\endinput

\section{Grounded theory}\label{sect:GroundedTheory}

\Glsf{grounded-theory} aims to define theories\footnote{In simple terms, a \glsf{theory} is a \glsf{system} of ideas intended to explain something.} on social \glsplf{phenomenon} based\footnote{I.e., \enquote{grounded}, hence the name!} on \glsplf{empirical-datum}. The intention is for the \glsf{theory} to emerge from the collection and analysis of the \glsplf{datum}, rather than using the \glsplf{datum} to confirm or disprove a previously formulated \glsf{theory}, or test a previously formulated \glsf{hypothesis}. 

%\textcquote[3.1.5]{johannesson2014research}{\Glsf{grounded-theory} is a \glsf{research-strategy} that strives to develop theories through the analysis of \glsplf{empirical-datum}. In contrast to experiments, \glsf{grounded-theory} does not start with a \glsf{hypothesis} to be tested but instead with \glsplf{datum} from which a \glsf{theory} can be generated. \Glsf{grounded-theory} also differs from research strategies, such as \glsf{ethnography}, which are content to provide rich descriptions of particular situations, but no theories. \Glsf{grounded-theory} challenges a top-down theorising approach, in which the researcher first develops a \glsf{theory} and then checks whether it conforms to \glsplf{empirical-datum}. Instead, \glsf{grounded-theory} insists that \glsplf{empirical-datum} is the starting point, upon which theories are to be built. \Glsf{theory} emerges through analysis and is grounded in the \glsplf{datum}.}

%\endinput

\subsection{Knowledge contribution}\label{ssect:GroundedTheoryKnowledgeContribution}

\Glsf{grounded-theory} contributes knowledge in the form of theories concerning complex social \glsplf{phenomenon}, striving to provide explanations of people’s choices and actions grounded in those people's own accounts and interpretations.

For example, \glsf{grounded-theory} could be used to formulate theories on what motivates people to join or leave a particular sector, or why employees may feel fulfilled or frustrated in their workplace.  

%\subsection{Focus}\label{ssect:GroundedTheoryFocus}

%\subsection{Variants}\label{ssect:GroundedTheoryVariants}

\subsection{Data collection and analysis}\label{ssect:GroundedTheoryDataCollection}

\Glsf{grounded-theory} requires the systematic collection and analysis of \glsplf{datum} without any preconceived belief or theoretical framework. The \glsplf{datum} are collected, coded and analysed to identify emerging concepts, categories and relationships. You iterate this process with new \glsplf{datum} which you use to review and revise those concepts, categories and relationships until no more can be gained from further \glsplf{datum} collection and analysis. At that point you can put forward a \glsf{theory} based on what was derived from the \glsplf{datum}. In this process, it is essential for you to be open to multiple explanations, and to explore the \glsplf{datum} from all angles in order to gain a fresh perspective.


%\textcquote{drew202310-grounded}{\Glsf{grounded-theory} aims to understand social \glsplf{phenomenon} by systematically collecting and analyzing \glsplf{datum} without preconceived notions or theoretical frameworks. The process involves iterative \glsf{coding} and constant comparison of \glsplf{datum} to generate concepts, categories, and relationships.}


%\endinput

%\Glsplf{empirical-datum} is extant;~\textcquote[p.13]{strauss1998basics}{In fact, Patton (1990), a qualitative evaluation researcher, made the comment, \enquote{Qualitative evaluation inquiry draws on both critical and creative thinking – both the science and the art of analysis} (p. 434). He went on to provide a list of behaviors that he found useful for promoting creative thinking, something every analyst should keep in mind. These include (a) being open to multiple possibilities; (b) generating a list of options; (c) exploring various possibilities before choosing any one; (d) making use of multiple avenues of expression such as art, music, and metaphors to stimulate thinking; (e) using nonlinear forms of thinking such as going back and forth and circumventing around a subject to get a fresh perspective; (f) diverging from one's usual ways of thinking and working, again to get a fresh perspective; (g) trusting the process and not holding back; (h) not taking shortcuts but rather putting energy and effort into the work; and (i) having fun while doing it (pp. 434–435).~\parencite{patton1990qualitative}}

%\endinput



%\endinput

%\textcquote{corbin1990grounded}{The success of a research project is judged by its products. Except in unusual instances when these are only orally presented, the study design and methods, findings, theoretical formulations, and conclusions are judged through publication. Yet, how are these to be evaluated and by what criteria? When judging qualitative research it is not appropriate, we have asserted, to use criteria ordinarily used to judge the procedures and canons of quantitative studies. It has been one of the aims of this paper to show how the \glsf{grounded-theory} approach accepts the usual scientific canons but redefines them carefully to make them appropriate to its specific procedures. In the instance of any \glsf{grounded-theory} study, the specific procedures and canons as described above should be part of its evaluation.}

%\endinput

%\textcite{smith1997understanding}
%					\begin{enumerate}%[start=0,label={(\bfseries R\arabic*):}]
%					\item are all processes~\parencite[remove when checked]{strauss1998basics} made explicit? 
%					\item have a \glsf{range} of \glsplf{datum} sources been used?
%					\item {}[add as general initial item]Was the research question which originated the study included in the final report? 
%					\item were the \glsplf{phenomenon} to be studied observable in social interaction?
%					\item {}[general]does the literature review allow the reader to identify the issues that the researcher found interesting initially? ~\parencite{smith1997understanding}
%					\item {}[general?]does the literature review provide a backdrop against which the new findings can be evaluated?~\parencite{smith1997understanding}
%					\item have the outcomes been triangulated - has more than one way of arriving at the \glsf{theory} been used?~\parencite{smith1997understanding}
%					\item has a wide \glsf{range} of participants been used? Did the importance of an issue come through repeatedly? Did the participants agree with the analysis process?~\parencite{smith1997understanding}
%					\item has the extent to which the \glsf{theory} is supported by the extant \glsplf{datum} been described?~\parencite{smith1997understanding} 
%					\item what new insight do the developed theories provide?~\parencite{smith1997understanding} 
%					\item {}[general and specific]have ideas for further study and implication for practice been discussed~\parencite{smith1997understanding}?
%					\end{enumerate} 
%					%
%					and from \textcite[p.425]{corbin1990grounded}
%					%
%					%
%					\begin{itemize}
%					\item [The \Glsf{research-process}:]
%					\begin{enumerate}[label={Criterion \arabic*:}]
%						\item how was the original \glsf{sample} selected? What grounds (selective sampling)?
%						\item what major categories emerged? 
%						\item what were some of the events, incidents, actions, and so on that (as indicators) pointed to some of these major categories?
%						\item on the basis of what categories did theoretical sampling proceed? That is, how did theoretical formulations guide some of the \glsplf{datum} collection? After the theoretical sampling was done, how representative did these categories prove to be?
%						\item what were some of the \glsplf{hypothesis} pertaining to conceptual relations (that is, among categories), and on what grounds were they formulated and tested?
%						\item were there instances when \glsplf{hypothesis} did not hold up against what was actually seen? How were these discrepancies accounted for? How did they affect the \glsplf{hypothesis}?
%						\item how and why was the core category selected? Was this selection sudden or gradual, difficult or easy? On what grounds were the final analytic decisions made?
%					\end{enumerate}
%
%					\item [Empirical Grounding of findings]
%\begin{enumerate}[label={Criterion \arabic*:}]


%Strauss & Corbin state that there are four primary requirements for judging a good \glsf{grounded-theory}: 1) It should fit the \glsf{phenomenon}, provided it has been carefully derived from diverse \glsplf{datum} and is adherent to the common reality of the area; 2) It should provide understanding, and be understandable; 3) Because the \glsplf{datum} is comprehensive, it should provide generality, in that the \glsf{theory} includes extensive variation and is abstract enough to be applicable to a wide variety of contexts; and 4) It should provide control, in the sense of stating the conditions under which the \glsf{theory} applies and describing a reasonable basis for action.

\subsection{Evaluation}\label{ssect:GroundedTheoryEvaluation}

You should address the following questions in evaluating \glsf{grounded-theory} research:

\begin{enumerate}
	\item \glsf{reliability}: 
		\begin{itemize}
			\item was the process followed to arrive at the \glsf{theory} appropriately described? Was it systematic and iterative?
		\end{itemize}
	\item \glsf{validity}: 
		\begin{itemize}
			\item were sufficient \glsplf{datum} collected and described? How relevant were they to the \glsf{phenomenon} under study?	
			\item which concepts, categories and relationships were generated by the research? How were they grounded in the \glsplf{datum}? How do they contribute to the \glsf{theory}?
			\item has the \glsf{phenomenon} been examined under a broad \glsf{range} of conditions and from a variety of perspectives?
 			\item is the \glsf{theory} plausible in relation to the \glsplf{datum}? Does it provide sufficient explanation of the \glsf{phenomenon} under study? Is it general enough to account for variation in conditions and context of application?
			\item can the \glsf{theory} be easily understood by its intended users? How useful is it in helping them understand their social reality and be the basis for action?
		\end{itemize} 
	\item \glsf{bias}: 
		\begin{itemize}
			\item is there a \glsf{reflexivity} account of the researcher to guard against possible \glsf{bias}?
		\end{itemize}
%	\item are limitations or omissions in relation to any of the points above identified? Have their effect on the research and its outcomes been discussed?
%	\item has the \glsf{research-strategy} been successful in relation to the \glsf{research-aim} and objectives?
\end{enumerate}

%\textcquote{corbin1990grounded}{Since the basic building blocks of any \glsf{grounded-theory} is a set of concepts grounded in the \glsplf{datum}, the first question to be asked of any publication is: Does it generate (via coding-categorizing activity) or at least use concepts, and what is or are their source or sources? If concepts are drawn from common usage (such as, “uncertainty”) but not put to technical use, then these are not concepts in the sense of being part of a \glsf{grounded-theory}, for they are not actually grounded in the \glsplf{datum} themselves.}
					
					
%\textcquote{corbin1990grounded}{The name of the scientific game is systematic conceptualization through conceptual linkages. So, the questions to ask here of a \glsf{grounded-theory} publication are whether such linkages have been made and do they seem to be grounded in the \glsplf{datum}? Furthermore, are the linkages systematically carried out? As in other qualitative writing, the linkages are unlikely to be presented as a listing of \glsplf{hypothesis} or in propositional or other formal terms but will be woven throughout the text of the publication.}
					
									
%\textcquote{corbin1990grounded}{If there are only a few specified conceptual relationships, even if grounded and identified systematically, this leaves something to be desired in terms of the overall grounding of the \glsf{theory}. A \glsf{grounded-theory} should be tightly linked, both in terms of categories to their subcategories and between categories in the final integration in terms of the paradigm features conditions, context, action/interaction (strategies) and consequences. Also categories, as mentioned in the body of the paper, should be theoretically dense (have many properties that are dimensionalized). It is the tight linkages, in terms of the paradigm features and density of the categories, that give a \glsf{theory} its explanatory power. Without these, the \glsf{theory} is less than satisfactory.}
					
					
%\textcquote{corbin1990grounded}{Some qualitative studies report only about a single \glsf{phenomenon} and establish only a very few conditions under which it appears, and specify only a few actions/interactions that characterize it, and a limited number or \glsf{range} of consequences. By contrast, a \glsf{grounded-theory} monograph should be judged in terms of the \glsf{range} of its variations and the specificity with which these are spelled out in relation to the \glsplf{datum} that are their source. In a published paper, the \glsf{range} of variations touched upon may be more limited, but the author should at least suggest that the fully study included their specification.}
					
					
%\textcquote{corbin1990grounded}{The \glsf{grounded-theory} mode of research requires that the explanatory conditions brought into analysis are not restricted to those that seem to have immediate bearing on the \glsf{phenomenon} under study. That is, the analysis should not be so “microscopic” as to disregard conditions that derive from more “macroscopic” sources: for instance, those such as economic conditions, social movements, trends, cultural values, and so forth.
					
%These also must not simply be listed as background material but directly linked to \glsplf{phenomenon} through their effect on action/interaction, and through these latter to consequences. Therefore, any \glsf{grounded-theory} publication that either omits these broader conditions or fails to explicate their specific connections to the \glsf{phenomenon}(a) under investigation, falls short in its empirical grounding.}

					
%\textcquote{corbin1990grounded}{Identifying and specifying change or movement in the form of process is an important part of \glsf{grounded-theory} research. Any change must be linked to the conditions that gave rise to it. Process may be described as stages or phases and also as fluidity or movement of action/interaction over the passage of time in response to prevailing conditions.}
					
										
%\textcquote{corbin1990grounded}{The question of \glsf{significance} is generally thought of in terms of the relative importance of a \glsf{theory} for stimulating further studies and for giving useful explanations of a \glsf{range} of \glsplf{phenomenon}. We have in mind here, however, the adequacy of a study’s empirical grounding in relation to its actual analysis insofar as this combination of activities succeeds or fails, in some degree, at producing useful theoretical findings. If the researcher simply follows the \glsf{grounded-theory} procedures/canons without any imagination or insight into what the \glsplf{datum} are reflecting - because he or she fails to see what they are really saying except in terms of trivial or well known \glsplf{phenomenon} - then the published findings can be judged as failing on this criterion. Recollect that there is an interplay between the researcher and the \glsplf{datum}, and no method, certainly not the \glsf{grounded-theory} one, can insure that the interplay will be creative. This depends on three characteristics of the researcher: analytic ability, theoretical sensitivity, and sensitivity to the subtleties of the action/interaction (plus sufficient writing ability to convey the findings). Of course, a creative interplay also depends on the other pole of the re­searcher-data equation: the quality of the \glsplf{datum} collected or utilized. An unimaginative analysis may in a technical sense be adequately grounded in the \glsplf{datum}, but actually it is insufficiently grounded for the researcher’s theoretical purposes. This is because the researcher either does not draw on the fuller resources of \glsplf{datum} or fails to push \glsplf{datum} collection far enough.}

%\endinput

%\textcquote{corbin1990grounded}{This double set of criteria, for the \glsf{research-process} and for the empirical grounding of the theoretical findings, bear directly on the issues of how verified any given \glsf{grounded-theory} study is and how this is to be ascertained. When the study is published,, if components of the \glsf{research-process} are clearly laid out and if there are sufficient cues in the publication itself, then the presented \glsf{theory} or theoretical formulations can be assessed in terms of degrees of plausibility. We can judge under what conditions the \glsf{theory} might fit with “reality”, give understanding, and be useful (practically and in theoretical terms). Researchers themselves can be rendered more aware of precisely what their operations have been and the possible inadequacies of these operations. In other words, they would be able to identify and convey what were the limitations of their study.}

%\endinput

%\textcquote{drew202310-grounded}{\Glsf{grounded-theory} is an innovative way to gather \glsplf{quantitative-datum} that can help introduce new thoughts, theories, and ideas into \glsf{academic-literature}. While it has its strength in allowing the “\glsplf{datum} to do the talking” , it also has some key limitations – namely, often, it leads to results that have already been found in the \glsf{academic-literature}. Studies that try to build upon current knowledge by testing new \glsplf{hypothesis} are, in general, more laser-focused on ensuring we push current knowledge forward. Nevertheless, a \glsf{grounded-theory} approach is very useful in many circumstances, revealing important new information that may not be generated through other approaches. So, overall, this methodology has great value for qualitative researchers, and can be extremely useful, especially when exploring specific case study projects. I also find it to synthesize well with \glsf{action-research} projects.}
%

\subsection{Is this strategy right for me?}\label{ssect:GroundedTheoryIsThisStrategy}

In choosing this strategy, you should consider the following:
\begin{enumerate}
\item \glsf{grounded-theory} is about letting the \enquote{\glsplf{datum} to do the talking} \parencite{drew2023grounded}, so you should not have any prior belief, \glsf{theory} or \glsf{hypothesis} you wish to put to test. If that's not the case, other strategies are more appropriate, like case studies or experiments
\item \glsf{grounded-theory} requires you to gather a significant amount of \glsplf{empirical-datum}, making sure you examine a social \glsf{phenomenon} under various conditions and from many perspectives. If you do not have access to such \glsplf{datum}, then \glsf{grounded-theory} cannot get started, and you should consider other strategies
\item \glsf{grounded-theory} is generally time consuming, given the iterative nature of the process of generating and analysing \glsplf{datum}. If time is an issue in your project, then you should choose a more time-efficient strategy, like case studies or experiments 
\item \glsf{grounded-theory} aims at generating theories concerning social \glsplf{phenomenon}. If a new \glsf{theory} is not the aim of your work, then you should choose a different strategy, like \glsf{ethnography} or case studies
\item \glsf{grounded-theory} is particularly useful when there is a paucity of theories in relation to the \glsplf{phenomenon} of interest. If there are already several theories available, it is less likely \glsf{grounded-theory} will be able to contribute something new. In such cases, you should rethink whether a new \glsf{theory} is actually needed or choose a different aim and strategy for your project.
\end{enumerate}
%

								
\subsection{Further reading}\label{ssect:GroundedTheoryFurtherReading}

\ReadingList{grounded theory}{drew2023grounded,smith1997understanding,corbin1990grounded,strauss1998basics,gibson2013rediscovering,charmaz2014constructing}


\section{Phenomenology}\label{sect:phenomenology}

\Glsf{phenomenology} focuses on people's conscious experience of a \glsf{phenomenon}, that is how people perceive and give meaning to it, including any feeling and emotions it evokes. 

\subsection{Knowledge contribution}\label{ssect:PhenomenologyKnowledgeContribution}

\Glsf{phenomenology} contributes knowledge by providing insights into people's lived experience, seeking to describe or interpret the essence of a \glsf{phenomenon} from the perspective of the people who have experienced it.

For instance, a phenomenological study of hospital emergency care could focus on the experience of nurses and doctors in emergency departments.

%\subsection{Focus}\label{ssect:PhenomenologyFocus}
%
%\subsection{Variants}\label{ssect:PhenomenologyVariants}

\subsection{Data generation and analysis}\label{ssect:PhenomenologyDataGeneration}

Data generation in \glsf{phenomenology} is primarily through in-depth, unstructured interviews and focus groups, which should allow participants to give their own account of their experience and surface key issues, without being influenced by the researcher. These are often complemented by participant \glsf{observation}, in which the researcher is immersed in the day-to-day activities of the study participants, hence sharing their experience of the \glsf{phenomenon} of interest. Audio and video recording, alongside field notes and journals are used to record \glsplf{datum}.

This typically results in a large quantity of \glsplf{quantitative-datum}, which are both detailed and unstructured, so that qualitative methods are then needed for their analysis. The analysis process requires you to set aside\footnote{Known as \glsf{bracketing}.} any preconception, assumption or judgement, and focus solely on the \glsplf{datum}, considering every participant's statement or expression as equally important and relevant.

\subsection{Evaluation}\label{ssect:PhenomenologyEvaluation}

The following questions should be asked of phenomenological research:

\begin{enumerate}
	\item \glsf{reliability}: 
		\begin{itemize}
			\item are the criteria for selecting the study participants properly explained and justified?
			\item are participant observations appropriately documented?
		\end{itemize}
	\item \glsf{validity}: 
		\begin{itemize}
			\item is the \glsf{phenomenon} accurately and objectively described?  Is the account provided one that can be recognised by anyone who has experienced that \glsf{phenomenon}?
			\item how have similarities and differences in the participants' experience of the \glsf{phenomenon} accounted for in the study? How are they dealt with in the \glsplf{datum} analysis, particularly in the \glsf{coding}\footnote{You will study coding in~\Cref{stage4}.} and categorisation process?
		\end{itemize}
	\item \glsf{bias}: 
		\begin{itemize}
			\item is there a reflexive account of how judgments were suspended to focus on the analysis of experience?
		\end{itemize}
%	\item are limitations or omissions in relation to any of the points above identified? Have their effect on the research and its outcomes been discussed?
%	\item has the \glsf{research-strategy} been successful in relation to the \glsf{research-aim} and objectives?
\end{enumerate}

%			\begin{enumerate}
%					\item will the experience as described be understandable to any reader and can be identified by anyone who has had that particular experience?
%					\item is the description of the \glsf{phenomenon} clearly presented so that experience differs from other experiences that are similar? {\color{red}what does this \glsf{mean}?}
%					\item are quotations from the \glsplf{datum} used to demonstrate the emergence of themes?
%					\item is there a discussion of discrepancies among participants and how those discrepancies were factored in \glsplf{datum} analysis?
%					\item have meaning units, themes, and summaries been described?
%					\item are meaning units grouped together to form themes?
%					\item are themes combined to form a composite summary of the \glsf{phenomenon}?
%					\item are quotes used to support the findings?
%					\item research participants will have their individual ways of experiencing a certain \glsf{phenomenon}. Have you looked for these common to all or most of the participants and not clustered meaning units together where significant differences exist?
%				\end{enumerate}
%					%
%					and the \glsf{research-process}
%					%
%				\begin{enumerate}
%					\item \glsf{bracketing}/Epoché/Phenomenological reduction - have you discussed how judgments were suspended to focus on analysis of experience. How did you use suspend your judgments to focus on the analysis of participants' experiences?
%					\item horizon - during \glsplf{datum} analysis, what was your present experience, your horizon? The horizon cannot be bracketed so you will need to discuss that not everything could have been realized by you, the researcher. This discussion might also lead into a discussion about future research implications in Chapter 5.
%					\item intentionality - discuss your level of scrutiny of the \glsplf{datum} you analyzed. How did you keep your focus on the topic you were studying? Perhaps you slowed down and dwelled on each narrative and did not pass over the details of the account as if you understood it already.
%					\item dasein - How has your Dasein (being-there) affected the research? How did the research affect your Dasein?
%					\item fore-sight/Fore-conception - What was your preconceived knowledge about the \glsf{phenomenon} you were studying?
%					\item hermeneutic Circle - How did were your understandings revised as you analyzed the \glsplf{datum}?
%			\end{enumerate}
%	
					%


%				\Glsf{phenomenology} is sometimes presented as having similarities with \glsf{ethnography} \parencite[p.52]{johannesson2014research}, and this leads us to the following evaluation criteria{\color{red} sense?}:
%					\parencite[p.180, adapted for phenomology]{oates2008researching}%
%					\begin{itemize}
%					\item does the research focus on lifestyle, meaning, and belief?
%					\item are the \glsplf{datum} generation methods that were used described? Did they lead to sufficient \glsplf{datum} having been collected?
%					\item how long did you spend in the field? Do you think this was long enough?
%					\item do you describe your approach holistic, semiotic or critical?
%					\item is the \glsf{phenomenology} a standalone description, or is it linked to \glsf{theory}, other \glsf{phenomenology} or issues in your culture?
%					\item have you included an account of you as researcher?
%					\item have you presented an phenomenological construction rather than a literal description?
%					\item what limitations in the \glsf{phenomenology} have you recognised?
%					\item which other flaws and/or omissions in your reporting of the \glsf{phenomenology} do you admit?
%					\item overall, how effectively has your phenomenological strategy been?
%					\end{itemize}
%					and \parencite[29m \textit{ff}]{office2020the-phenomenological}
%					%
%						\begin{itemize}
%						\item have you described the \enquote{what, when, where, and how} of the study? What has be done? When the steps were sequenced? Where each step happened? How each step happened?
%						\item have you described where will \glsplf{datum} be collected? Who collected the \glsplf{datum}? How often and how much \glsplf{datum} was collected? How long it took to collect the \glsplf{datum}? How the \glsplf{datum} was recorded? (ex: transcriptions, video recordings, audio recordings) Were there follow ups to interviews?
%						\end{itemize}
%						%
%				

\subsection{Is this strategy right for me?}\label{ssect:PhenomenologyIsThisStrategy}

In choosing this strategy, you should consider the following:

\begin{enumerate}
	\item the focus on \glsf{phenomenology} is lived experience, to uncover what is really like to experience a \glsf{phenomenon} from the perspective of those who have lived through it. If you do not have access to participants who can share their experience, then you should consider a different strategy
	\item \glsf{phenomenology} asks you to suspend any prior belief on the \glsf{phenomenon} and only focus on the participants' experience. If you have a \glsf{theory} or \glsf{hypothesis} you want to test, then you should choose a different strategy, for instance experiments
	\item the amount of \glsplf{quantitative-datum} to generate and analyse is considerable, and this can be very time consuming. If time is an issue in your research, then you will need to choose a more time-efficient \glsf{research-strategy}
	\item \glsf{phenomenology} is about going deep into the experience of a \glsf{phenomenon}, and this constrains the number of participants in your study. If you are more interested in  gaining consensus from a large number of participants, or making general predictions from your \glsf{sample} \glsplf{datum}, then you should choose something else, like \glsf{survey-research} or \glsf{grounded-theory}.
	\end{enumerate}

%	\item is your audience expecting scientific rigour? If so, choose ???
%\item does your \glsplf{datum} (source) allow analysis above the \glsplf{datum}, or will the outputs be mostly descriptive?
%\item are you short of time for \glsplf{datum} collection? If so, consider ???
%\item do you appreciate the value of deep philosophical discourse?
%\item ...



\subsection{Further reading}\label{ssect:PhenomenologyFurtherReading}


\ReadingList{phenomenology}{merleau-ponty1956what,anderson1991qualitative,smith2018phenomenology,shudak2018phenomenology,academic-educational-materials2019understanding,office2020phenomenological,groenewald2004phenomenological,hycner1985some}

\section{Simulation}\label{sect:simulation}

A \glsf{simulation} uses an explicative mechanism to imitate or reproduce the behaviour of a real-world artefact or \glsf{system}.

\subsection{Knowledge contribution}\label{ssect:SimulationKnowledgeContribution}

\Glsf{simulation} contributes knowledge by allowing the study of the simulated artefact or \glsf{system} under different conditions, in order to answer \enquote{What if?} questions, make predictions or gain insights on behaviour or properties, particularly when this can't be easily achieved directly on the real-world artefact or \glsf{system}.

Simulations are used in all disciplines and vary greatly in their purpose, nature and design. For instance: financial simulations are used to study the behaviour of the global stock market; climate simulations to study possible effects of climate change; engineering simulations, to test the properties of materials under different stress conditions; social simulations to study human behaviour in social settings, to name just few examples.  

%\subsection{Focus}\label{ssect:SimulationFocus}
%
%\subsection{Variants}\label{ssect:SimulationVariants}


\subsection{Data generation and analysis}\label{ssect:SimulationDataGeneration}

\Glsplf{datum} are needed to inform the \glsf{simulation} design. Their kind and how to obtain them will depend on the nature of the artefact or \glsf{system} under study and on your \glsf{research-aim}, so that all known methods for \glsplf{datum} generation apply. For instance, to simulate a new aircraft design under different wind conditions, you may need to gather \glsplf{datum} on the physical characteristics of the aircraft and the materials to be used to build it, alongside meteorological \glsplf{datum} which can be used to perform tests under different simulated conditions. On the other hand, to simulate how size and age of a \glsf{population} may change in future decades within a particular geographical region, you may need to gather \glsplf{datum} on current \glsf{population} size and age, birth and mortality rates, migration rates in and out of the region, and conditions which may affect them over time. 

\Glsplf{datum} are also needed to establish measures and criteria to evaluate whether the \glsf{simulation} is sufficiently representative of the artefact or \glsf{system} being simulated. This may involve comparing \glsf{simulation} outputs with \glsplf{empirical-datum} or theoretical predictions, or gathering expert opinions on such outputs, with the aim of establishing the extent expectations are met or significant discrepancies exist. 

Whichever methods you use to generate \glsplf{datum} for your \glsf{simulation} design, once constructed, the \glsf{simulation} should allow you to generate observations of the simulated artefact or \glsf{system}, both past, present and future, the latter being a unique characteristic of this \glsf{research-strategy}\footnote{All other strategies can only look at the past or the present.}. Such observations can then be analysed in order to address the research probelem, as well as to evaluate the \glsf{simulation} against the established measures or criteria. 

%According to \parencite{dooley2017simulation}, the most common types\footnote{Although this list is not exhaustive} include:
%		%
%	\begin{itemize}
%	\item discrete event \glsf{simulation}, which involves modeling the organizational \glsf{system} as a set of entities evolving over time according to the availability of resources and the triggering of events.
%	\item \glsf{system} dynamics, which involves identifying the key “state” variables that define the behavior of the \glsf{system}, and then relating those variables to one another through coupled, differential equations.
%	\item agent-based \glsf{simulation}, which involves agents that attempt to maximize their fitness (utility) functions by interacting with other agents and resources; agent behavior is determined by embedded schema which are both interpretive and action-oriented in nature.
%	\end{itemize}
%	%

%Depending on the context in which the \glsf{simulation} is studied or used. May include:
%	%
%	\begin{itemize}
%	\item observations, interviews, \glsplf{questionnaire}, \glsplf{document}, in-depth description when supporting social objectives
%	\item accuracy, predicative capability, when used to \glsf{model} real-world processes – think weather forecasting
%	\item ...
%	\end{itemize}
%	%
				
\subsection{Evaluation}\label{ssect:SimulationEvaluation}

Evaluation questions specific to this strategy include:

\begin{enumerate}
	\item \glsf{reliability}: 
		\begin{itemize}
			\item how was the \glsf{simulation} tested and improved during its development? Were different testing methods applied and the results documented?
			\item were \glsf{simulation} performance measures or criteria clearly established? How were they chosen and why?
			\item how was the \glsf{simulation} constructed? Were appropriate computational/mathematical/statistical techniques applied?
		\end{itemize}
	\item \glsf{validity}: 
		\begin{itemize}
			\item is the \glsf{simulation} design appropriate to address the \glsf{research-problem}/answer the research question?
			\item were appropriate \glsplf{datum} generated to inform the \glsf{simulation} design? How were they chosen and why?
			\item how close are the \glsf{simulation}'s outputs or behaviours to the real-world \glsplf{datum}? How was this established? Do the results make sense?
		\end{itemize}
\end{enumerate}

%For computer simulations:
%
%\begin{itemize}
%\item good development: has a documented set of requirements been maintained? Has a change control process been implemented?Is there a corresponding document (or version) control process.
%
%\item has the architecture been documented? What is its relationship to the \glsf{model}?
%
%\item have a variety of testing methods, including \glsf{code} walk-throughs, scenario testing, and user testing been used to establish \glsf{code} quality?
%
%\item was a project plan for \glsf{coding} and testing developed?
%
%\item how close is the \glsf{simulation}'s behaviour to the \enquote{real} answer? Do the results make sense?
%
%\item has the \glsf{simulation} been compared to any extant quantitative behaviour available? Does it match exactly, distributionally (a \glsf{variable} of interest has statistically similar characteristics), or pattern-wise (variables are generally related to one another in a valid manner, but perhaps differ from reality)?
%
%\item which experimental set-up was used? Was it appropriate?
%
%\item have observations from analysis been noted, and results discussed in order to sense-make? Has over-interpretation of the results been avoided so that retrofitting to theories is avoided?
%\end{itemize}
%
%For \glsf{simulation}:
%
%%
%\begin{itemize}
%\item ...
%\end{itemize}
%%

\subsection{Is this strategy right for me?}\label{ssect:SimulationIsThisStrategy}

If you are considering this \glsf{research-strategy}, then you should consider the following:
\begin{enumerate}
	\item the design and construction of a \glsf{simulation} may require advanced computational skills, often alongside mathematical and statistical skills. Do you already have such skills? If not, you need to ensure that you will be able to develop them in the time of your project, otherwise you should consider other research strategies
	\item you must have access to the \glsplf{datum} and stakeholders needed for the design and evaluation of the \glsf{simulation}, otherwise it is unlikely you will be able to build a representative \glsf{simulation}, and generate valid and reliable results. If access is a problem, you should consider other strategies.
\end{enumerate}

\subsection{Further reading}\label{ssect:SimulationFurtherReading}

\ReadingList{simulation}{dooley2017simulation}

\section{Mathematical and logical proof}\label{sect:MathematicalAndLogical}

A mathematical proof is a rigorous argument that demonstrates the truth of a certain \glsf{proposition}. When the argument is made within a fully formal logical system, then we have a logical proof.

\subsection{Knowledge contribution}\label{ssect:MathematicalAndLogicalKnowledgeContribution}

Mathematical and logical proofs contribute knowledge in the form of true propositions, which are the means by which Mathematics functions and grows its scope and applicability.  
Knowledge is accumulated by relying on sets of assumptions and previously proven propositions which are taken as the starting point to generate, through proofs, new propositions, that is new truths.

A famous example is the proof of Fermat's last theorem. This was formulated by Fermat in 1637, but remained a conjecture\footnote{A conjecture is an unproven \glsf{proposition}.} till 1994, when it was finally proved by a mathematician called Andrew Wiles. The proof relied on a number of other theorems proved during those three centuries by other mathematicians, and which provided the building blocks for Wiles' final proof. 

It is worth noting that within a mathematical or logical \glsf{system}, such truths are absolute, something which does not hold in any other scientific discipline: even in the natural sciences and by taking a (post-)positivist stance, truths are always only tentative and falsifiable, in that they hold only until new \glsf{evidence} emerges to contradict them.

%\subsection{Focus}\label{ssect:MathematicalAndLogicalFocus}
%
%\subsection{Variants}\label{ssect:MathematicalAndLogicalVariants}

\subsection{Data generation and analysis}\label{ssect:MathematicalAndLogicalDataGeneration}
In this strategy, rather than \glsplf{datum} generation and analysis, it makes sense to talk about  kinds of reasoning. Specifically, you can apply: \glsf{deductive-reasoning}, which starts from (true) assumptions and arrives at the \glsf{proposition} as a conclusion; \glsf{inductive-reasoning}, which draws general conclusions from particular observations; and \glsf{proof-by-contradiction}, which arrives at a contradiction by assuming the \glsf{proposition} to be false, hence demonstrating that it must be true. This was the case for Wiles' proof of Fermat's last theorem.

\subsection{Evaluation}\label{ssect:MathematicalAndLogicalEvaluation}

A new proof is subject to the scrutiny of the community of mathematicians, which employs various means to check both assumptions and reasoning, so as to reach a verdict on the reliability and \glsf{validity} of the proof. 

Such means may include using alternative kinds of reasoning to check they can reach the same conclusion, using examples in support of the reasoning, or even recreating a mathematical proof within a fully formal \glsf{system} or using a computer-based automated checker, when applicable. 

\subsection{Is this strategy right for me?}\label{ssect:MathematicalAndLogicalIsThisStrategy}

If you are considering this strategy, you should ask yourself:

\begin{enumerate}
	\item mathematical and logical proofs only make sense within research which is amenable to formalisation. Is that the case for your project? If not, then the notion of proof may not apply and you should instead consider a \glsf{research-strategy} which deals with \glsplf{empirical-datum}
	\item you will need to be a skilled mathematician or logician to come up with a proof that can withstand the scrutiny of the mathematical community. Do you possess such skills? If not, then this strategy may not be for you\footnote{One way to determine whether your background is suitable would be to read the first few pages of \textcite{lakatos2015proofs} (up to page 9 is available through Google Books).} 
	\item ideally, you should seek formative feedback from an experienced mathematician as you develop your proof, to reduce the chance of mistakes or reasoning pitfalls. Do you have access to such an expert advice? If not, you should discuss with your supervisor to ensure they do have the skills to take up that role.
\end{enumerate}


%Am I trying to predict future behaviours of a \glsf{system}? If so, mathematical, statistical or computational modelling might be a better option.


\subsection{Further reading}\label{ssect:MathematicalAndLogicalFurtherReading}

\ReadingList{mathematical and logical proof}{kleene1964introduction,lakatos2015proofs,antonini2011examples,johannesson2014research}

\section{Mixed methods research}\label{sect:MixedMethodsResearch}

The \glsf{mixed-methods-research} strategy\footnote{Mixed-method research should not be confused with \glsf{multi-method-research}, which simply indicates the use of many methods, possibly all qualitative or quantitative.} combines elements of both qualitative and quantitative research, with the aim to increase both breadth and depth of understanding of the \glsf{phenomenon} under study, and corroboration of results, giving more confidence in the conclusions reached. As a result, \glsf{triangulation} is in-built within the strategy.


%\textcquote{johnson2007toward}{\Glsf{mixed-methods-research} is the type of research in which a researcher or team of researchers combines elements of qualitative and quantitative research approaches (e.g., use of qualitative and quantitative viewpoints, \glsplf{datum} collection, analysis, inference techniques) for the broad purposes of breadth and depth of understanding and corroboration.}

%\textcquote[p.119, quoting \emph{Valerie Caracelli}]{johnson2007toward}{[P]lanfully [combining] methods of different types (qualitative and quantitative) to provide a more elaborated understanding of the \glsf{phenomenon} of interest (including its context) and, as well, to gain greater confidence in the conclusions generated by the evaluation study.}

\subsection{Knowledge contribution}\label{ssect:MixedMethodsResearchKnowledgeContribution}

This is the combination of the knowledge contributed by each of the methods applied, appropriately synthesised by considering connections and contradictions between qualitative and quantitative data. 

Mixed method research is particularly suited to interdisciplinary research and to the study of complex situations or social settings, particularly when one kind of method alone would not deliver the desired depth of understanding or richness of results. 

For example, within urban planning, you may be interested in improving pedestrians' safety through a mixed methods study in which you could consider both quantitative data on pedestrian accidents and qualitative data on pedestrians' experiences and perceptions, in order to identify both safe and dangerous areas: this would allow you to both learn lessons and plan remedial actions.

%In addition, \textcquote[p.119, quoting \emph{Huey Chen}]{johnson2007toward}{[the methods] can be adapted, altered, or synthesized to fit the research and cost situations of the study (modified form mixed methods).}


%\subsection{Focus}\label{ssect:MixedMethodsResearchFocus}
%
%\subsection{Variants}\label{ssect:MixedMethodsResearchVariants}
%
\subsection{Data generation and analysis}\label{ssect:MixedMethodsResearchDataGeneration}

How \glsplf{datum} are generated and analysed will depend on how the different methods are combined. Typical combinations include:
\begin{itemize}
	\item parallel, in which separate qualitative and quantitative methods are applied to gather different sets of \glsplf{datum}. For instance, your collection of pedestrians' accident \glsplf{datum}, and pedestrian's opinions may occur in parallel, independently of one another, then the results may be analysed and compared
	\item sequential, in which the methods are applied one after the other, with outcomes from the first used to inform the second. For instance, you could start with the pedestrians' experience, then collect accident \glsplf{datum} on areas which are perceived as particularly safe or dangerous
	\item nested (or embedded), in which a quantitative method is applied within a wider qualitative method (or vice versa). For instance, your focus may be primarily on the qualitative pedestrians' experience, within which you could apply some \glsf{statistical-analysis} to look for correlations between experience and accident \glsplf{datum}.
\end{itemize}

%Those of the individual methods, combined with those of the methods mixed. The latter form provides for \glsf{triangulation}: \textcquote[p.291]{denzin1978research}{the combination of methodologies in the study of the same \glsf{phenomenon}}: \textcquote[p.3]{webb2000unobtrusive}{If a \glsf{proposition} can survive the onslaught of a series of imperfect measures, with all their irrelevant error, confidence should be placed in it. Of course, this confidence is increased by minimizing error in each instrument and by a reasonable belief in the different and divergent effects of the sources of error.}

\subsection{Evaluation}\label{ssect:MixedMethodsResearchEvaluation}

Evaluation questions for this strategy will include questions on the specific methods which are combined, alongside the following questions on their combination:

\begin{enumerate}
	\item \glsf{reliability}: 
		\begin{itemize}
			\item how was the use of mixed methods justified in relation to the \glsplf{phenomenon} of interest? How has the study benefitted from their combination?
			\item is the way the methods are combined appropriately described?
		\end{itemize}
	\item \glsf{validity}: 
		\begin{itemize}
			\item how were connections between qualitative and quantitative findings established?
			\item how were conflicting or mismatched results from the different methods handled?
		\end{itemize}
\end{enumerate}

\subsection{Is this strategy right for me?}\label{ssect:MixedMethodsResearchIsThisStrategy}

If you are considering \glsf{mixed-methods-research}, you should take the following into account:

\begin{enumerate}
	\item \glsf{mixed-methods-research} requires competence in more than a single \glsf{research-method}, which takes time to develop. If your project is your first research project, it is unlikely you will have a developed understanding sufficient to apply the \glsf{mixed-methods-research} strategy, and you should consider a simpler approach. However, if your work is part of broader \glsf{mixed-methods-research}, perhaps led by your supervisor, then you may be able to contribute by focusing on the particular method you are being asked to work with
	\item collecting and analysing both qualitative and quantitative \glsplf{datum} requires substantial time and resources. If this is going to be an issue, then a strategy with a single method focus might be a better choice.
\end{enumerate}

\subsection{Further reading}\label{ssect:MixedMethodsFurtherReading}

\ReadingList{mixed methods research}{johnson2007toward,denzin1978research,webb1999unobtrusive}

%\subsection{More details}\label{ssect:MoreDetails}
%
%[Add summary of decision process as a diagram]
%
\chapter{Establishing your research methodology}\label{ch:yourResearchMethodology}

By now you should have narrowed down your list of candidate strategies to few choices, and possibly discussed them with your supervisor. It is now time to learn more about them in order to make a final choice for your project. This, in turn, will allow you to establish the overall methodology for your project.

While the content of this chapter is brief, the two activities it contains are going to be demanding. It is important, however, that you don't skip them as they provide the foundation for your work in~\Cref{stage4}.


\section{Choosing your research strategy}\label{sect:yourResearchStrategy}

This section is made of just one activity, at the end of which you should have established which \glsf{research-strategy} you will adopt in your research. 

\begin{activity}[{Choosing your \glsf{research-strategy}}]
For each candidate strategy in your list, consider the references you have recorded in your BMT, and review the related literature in order to learn more about the strategy, and how it may apply practically to your project, including specific methods you could use to generate and analyse your \glsplf{datum}. 

At the end of the activity reach a decision on which \glsf{research-strategy}, or combination of strategies, to adopt.
\begin{guidance}
%Hack to correct tcbox behaviour
\color{black}


%Hack to correct tcbox behaviour
\color{black}
	The references provided are only a starting point, and you should also explore other literature for each strategy. There is a vast literature on \glsf{research-methodology}, so it is easy to get lost. You should look at introductory materials first to gain a broad understanding, then delve into more specialised literature for some of the details. Your supervisor will also be able to suggest appropriate reading.
	
	As you read the literature, you should take notes to augment the summaries provided in the previous sections, to capture your deeper engagement and growing understanding of each \glsf{research-strategy}. By the end of this activity you should have gained a good general understanding of how each strategy you have reviewed may suit your research.

	 In your review, you should pay particular attention to possible \glsplf{datum} generation and \glsplf{analysis-method} under each strategy, reflecting on those which may be most applicable to your project and how, alongside any risk or other factors which may affect their successful application. 
	
	It is not necessary for you to learn the fine details of each method at this point. However, by the end of this activity, you should have a clear idea of which methods you will be focusing on in~\Cref{stage4}, in which you will engage with the specific procedures to apply those methods in your own project. 
\end{guidance} 
\end{activity}
%%Hack to correct tcbox behaviour
\color{black}


\section{Drafting your research methodology}\label{sect:draftingMethodology}

It's now time to draft your chosen methodology, as a starting point for the narrative you will develop during the remainder of your project and eventually include in your dissertation. 

As your learnt in~\Cref{ch:ResearchDesignFoundation}, your methodology determines \textit{how} you are going to conduct your research, particularly how to go about generating, analysing and interpreting \glsplf{datum} and \glsf{evidence} to meet your \glsf{research-aim} and objectives. It is therefore the result of your choice of \glsf{research-strategy} and methods therein, and how you are going to apply them within your project. Your methodology should allow you to make your contribution to knowledge in a way which meets the expectations of researchers in your field of study, therefore it is important that you also reflect on your motivating mindset\footnote{Refer back to~\Cref{sect:ResearcherMindsets}, if necessary.} in your choice.

%Of course, the notion of correctness here is a loose one: innovation in \glsf{research-design} in a particular domains happens more or less often. Indeed, it is often an innovation in the design of research by which new knowledge is contributed. And it's not unknown for a researcher new to the domain to bring fresh thinking – it may even be you. There are risks of innovating, however, including finding it difficult to publish – not a concern for you – having difficulty convincing a reader that's not ready to accept innovation – essentially, being accepted. Risk management is often a Good Thing.

To structure your account of your methodology, you may find the guidance in~\Cref{tab:researchMethodologyReport} useful.

\begin{SimpleNColTable}{tab:researchMethodologyReport}{2}{Elements to include in your \glsf{research-methodology} account}[X[-1,r]X[l]]
Section & Content \\
\Glsf{research-strategy} & Indicate which strategy, or combination of strategies, you have chosen and justify your choice by considering how it aligns with the aim of your research in relation to the way it will contribute new knowledge\\

Methods & Indicate the selection of methods you intend to apply within your \glsf{research-strategy}, including possible generation, modelling and \glsplf{analysis-method}. Explain why they are consistent with your choice of \glsf{research-strategy}, and suitable and feasible for your own project\\

\Glsplf{datum} and sources & Indicate which \glsplf{datum} you will need in your project, their possible \glsplf{datum} sources and how you will gain access \\

Mindset(s) & Discuss how your chosen strategy is consistent with prevalent mindset(s) in your own field of study, and your own \\

Reflexive statement & Summarise your own standpoint as a researcher, including specific assumptions, beliefs and potential \glsf{bias} you bring to the research, and the steps you will take ensure they will not weaken your research \\

Ethical issues & Indicate all ethical issues relevant to your chosen \glsf{research-strategy}, and how you will ensure compliance with regulations and explicit permission to proceed with your research  \\

Research evaluation & Reflect on the potential weaknesses which may affect your chosen strategy, and consider the related evaluation questions from the previous sections, alongside other evaluation criteria you may have found in the literature. Highlights those which are most relevant to your project and indicate which actions you will take to address them\\
\end{SimpleNColTable}

\begin{activity}[{Drafting your {research-methodology}}]
Make use of~\Cref{tab:researchMethodologyReport} to write a first account of your \glsf{research-methodology}.
\begin{guidance}
%Hack to correct tcbox behaviour
\color{black}


%Hack to correct tcbox behaviour
\color{black}
	This first account will be tentative, and you will develop it further in the next stage.
		
	It is important that you engage with all the elements in the table, including the evaluation section: although your full evaluation will only be completed at the end of your project, it is essential that you start thinking about the questions you will need to address. This, in turn, will help you ensure that the steps you take in your methodology application are likely to provide satisfactory answers to those questions.
\end{guidance} 
\end{activity}
%%Hack to correct tcbox behaviour
\color{black}



\chapter{Writing up}\label{ch:writingUpInStage3}

As in previous stages, your end-of-stage report will help you consolidate your work so far and provide some more content for your dissertation. Its recommended structure and content are indicated in~\Cref{stage3WritingOutcomes}. 

\begin{activity}[{Writing and assessing your report for~\Cref{stage3}}] Using your word processor of choice, revise and expand your~\Cref{stage2} report by applying the structure and guidance in~\Cref{stage3WritingOutcomes}, making good use of your notes and summaries from all related activities you have carried out.

Assess your report by applying the criteria in~\Cref{tab:criteriaForReport3}. Revise and iterate until you are ready to move on. 
\begin{guidance}
%Hack to correct tcbox behaviour
\color{black}


%Hack to correct tcbox behaviour
\color{black}
There are no completely new sections to fill in for~\Cref{stage3}; instead, you will need to revise and expand the content of your previous report, particularly in relation to your \glsf{research-design} which is the main focus of this stage. In writing your report:
\begin{itemize}
	\item you should review and revise all content from~\Cref{stage2} in light of your increasing understanding of \glsf{research-design} and methodological choices you will have made, which may have caused you to re-think your aim and objectives to some extent, and highlighted additional papers to include in your literature review. As you revise previous content, you should continue to ensure that all related research elements remain coherent
	\item you should revise and expand your \glsf{research-design} chapter substantially by adding content you generated as part of your activities on selecting and documenting your chosen \glsf{research-strategy}. Make sure your summary includes specific methods you intend to apply, and shows awareness of potential research weaknesses and how you may overcome them. If appropriate, your justification may include consideration of prevalent mindset(s) in your field of study
	\item you should continue to monitor your progress and ensure your project remains viable by revisiting your project risk and work plan
	\item equally you should continue to practise \glsf{reflection} and \glsf{reflexivity} to further develop your skills and attitude to research.
\end{itemize}
In evaluating your report, for each criteria in the table, you should consider the related prompts, write down any further work needed for your next stage, and update your work plan and risk assessment accordingly.
\end{guidance}\end{activity}
%%Hack to correct tcbox behaviour
\color{black}

\begin{SimpleNColTable}{tab:criteriaForReport3}{2}{Criteria for reviewing your report}[X[-1,r]X[l]]
Criteria & Prompts \\
Completeness & Are all sections included and their content complete? What is missing?\\
Academic writing & Is your writing clear, concise and precise? Should you improve it further? \\
Logical structure and flow & Have you structured your writing so that your narrative follows a logical flow? Which restructuring may be needed?\\
Supporting \glsf{evidence} & Are all your claims supported by appropriate references or other \glsf{evidence}? Which further \glsf{evidence} do you still need?\\
Citation and reference style & Do all your citations and references comply with the bibliographical style required by your course of study? \\
Avoiding \glsf{plagiarism} & Have you acknowledged the work of others? Is it clearly distinguished from your own? \\
Grammar and spelling & Have you proof-read your report carefully to remove typos and grammatical errors? \\
\end{SimpleNColTable}


\begin{takeaways}{Stage III}\label{ch:Stage3Takeaways}
\Cref{stage3} focuses on methodology, developing your understanding of research strategies, their potential weaknesses and how to overcome them. Here are takeaways of \Cref{stage3}:
\begin{itemize}
	\item to make a contribution to knowledge through your research you need to apply an appropriate methodology and defend your \glsf{claim} to knowledge at the end of your project
	\item defending your \glsf{claim} to knowledge means that you have considered and dealt with potential weaknesses in the execution of your methodology
	\item research weaknesses include \glsf{validity}, \glsf{reliability}, \glsf{novelty} and \glsplf{bias-weakness}
	\item \glsf{triangulation}, \glsf{reflexivity}, documenting your procedures and appealing to the literature are common ways to address those research weaknesses 
	\item standard research strategies have emerged over time in different disciplines, 12 of which are discussed in this book
	\item you should apply the process suggested in this book to consider the 12 strategies systematically and arrive at your final choice	
	\item for your chosen strategy, you should know how it contributes knowledge, its potential weaknesses, which methods are admissible and where to go in the literature to find out more
	\item by the end of~\Cref{stage3}, you should know enough about your chosen strategy to be able to decide how you will apply it within your project
	\item in writing up your project methodology you should include your chosen \glsf{research-strategy} and related methods, which \glsplf{datum} you will need and where from, and any ethical issues and potential research weaknesses you will need to deal with and how
	\item in the account of your methodology you should also include a reflexive account of how your own research mindset has motivated your methodological choices. 
	\end{itemize}
\end{takeaways}


%\begin{figure}[hbtp]
%\centering{
%  \includegraphics[width=0.7\textwidth]{Figures/researchStrategies}
%  \caption{\Glsf{research-strategy} choices
%  \label{fig:researchStrategies}
%  	}  
%  }
%\end{figure}

%\section{Research skills audit}\label{sect:ResearchSkillsAudit}
%
%\begin{activity}[{Skills audit}]
%Something here
%\end{activity}
%%Hack to correct tcbox behaviour
%\color{black}

%%Sectional bibliography
\printbibliography[segment=\therefsegment,title=Stage III \bibname]
	
\endinput

%\section{Materials not used}\label{sect:MaterialsNotUsed}
%
%\subsection{For your chosen \glsf{research-strategy}}\label{ssect:ForYourChosen}
%
%\begin{activity}[{Dissertation structure}]
%Whichever tool you've chosen in which to write your dissertation, create chapters entitled \enquote{\glsf{research-strategy}}, \enquote{method}, and \enquote{Evaluation}. 
%
%\begin{guidance}
%	For the \glsf{research-strategy} chapter, make notes from the paper you've read on the general form of the \glsf{research-strategy}. 
%
%For the method chapter, add details on the methods that are used in the \glsf{research-strategy}. For a complex strategy such as \glsf{ethnography}, you may not use all of them, but you will need to be explicit – when you come to complete it – as to which you have excluded and the reasons for their exclusion. 
%
%For the evaluation chapter, create subsections for each of the questions of your chosen \glsf{research-strategy} from the lists above.
%\end{guidance}
%	
%\end{activity}
%\color{black}
%
%
%
%\subsection{Structuring research}\label{ssect:StructuringResearch}
%
%Like a recipe, research needs to be structured. 
%
%To a large extent, the structure you use depends on your resources: massive research labs with many hundreds, even thousands, of researchers may need to run many different strands of research at the same time\footnote{According to \url{https://www.nature.com/articles/nature.2015.17567}, there are 5,154 authors on the paper \fullcite{aad2015combined}, in a collaboration between ATLAS and CMS, \blockquote{two massive detectors at the Large Hadron Collider (LHC) at CERN, Europe’s particle-physics lab near Geneva, Switzerland}. \textit{George Aad}, the first author, has the perfect surname for an academic.}, each contributing a small part towards an overall research goal. 
%
%Imagine having to manage that collaboration: 5154 researchers working in parallel.
%
%You aren't likely to have access to such resources. That's good, in a way, because you can keep the research simple and your research can be linear: one step after another. 
%
%Given that you're resource limited means we can plot your research as a single line
%%
%\newcommand{\taskline}[0]{\ \rule[0.5ex]{0.8cm}{1pt}\ }
%\[T\taskline T\taskline\cdots\taskline T\taskline T	
%\]
%%
%each node $T$ being one task, taken from one of the entries in Cref{tab:researchTasks}. The design is simple because it's linear, and so there's not much to think about:
%%
%\begin{itemize}
%\item how long the line should be;
%\item what each $T$ will be.
%\end{itemize}
%%
%\todo{Need to talk about relationship between methods, design, tasks, \etc{}}.
%More complex research designs, those involving multiple researchers, for instance, will require some amount of sophisticated project management to ensure that the sequencing of parallel research tasks is done correctly.
%
%The table below introduces more than 30 research tasks – the possible $T$s above – and for each gives a brief introduction and a key reference from which you can find out more\todo{Paraphrase descriptions in the table; more to come here.}.
%
%Your literature review may have thrown up papers with an explicit \enquote{methods} section that describes the \glsf{research-design} – how you performed the research\footnote{There are examples from the APA here: \url{https://www.scribbr.com/apa-style/methods-section/}.}\todo{Work this in more}. 
%
%The point of a methods section is to report 
%\begin{stolen}{https://www.scribbr.com/apa-style/methods-section/}
%enough information to understand and replicate your study, including detailed information on the \glsf{sample}, measures, and procedures used.	
%\end{stolen}
%
%You may have come across\todo{complete.}.
%
%\subsection{A simple \glsf{research-design}: \Glsf{experimental-research}}\label{ssect:SimpleResearchDesign}
%
%As an example of a simple \glsf{research-design}, the shortest \glsf{research-design} possible is
%%
%\[R\taskline O\taskline X \taskline O\]
%%
%which means (from the table\footnote{...where you can read more now, or wait until later in this chapter.}):
%%
%\begin{itemize}
%\item [R] Randomise \glsf{sample} from \glsf{population}, then
%\item [O] Observe, then
%\item [X] Change experimental \glsf{variable}, then
%\item [O] Observe, again.
%\end{itemize}
%%
%This form of \glsf{research-design} is called \enquote{\Glsf{experimental-research}}\footnote{There are simpler research designs} and is what you might think of as the quintessential scientific research – doing an experiment ($X$) on a random \glsf{sample} ($R$) \glsf{population}, a drug trial, for instance and observing the consequences ($O$).
%
%Even though we have called this a simple \glsf{research-design}, it doesn't \glsf{mean} that the results that you can obtain by using it will be simple, it could be that the drug you're testing will make amazing strides in curing some illness, improving the lives of millions of people. What we \glsf{mean} by \emph{simple} is simply that there are few steps in the research. Simple doesn't \glsf{mean}, either, that the work that will need to go onto each step is simple, quick, or trivial. \Glsf{observation}, for instance, is an immensely difficult thing to do correctly; in the worst case, it may take many months of work to get to the point where you \enquote{change the experimental \glsf{variable}} – administer the drug, for instance – and so have something to observe.
%
%The text we recommend for experimental design is \textcite{marczyk2005essentials}\footnote{Page~124 is a good place to start in that edition.}, who speak from a social science background. .\todo{Do we need to write more here about the source?} 
%
%\todo{more here, using the above as a template, and the table below as source...}
%
%\subsection{Another \glsf{research-design}: Quasi-experimental Research}\label{ssect:AnotherResearchDesign}
%
%More here from \textcite{marczyk2005essentials}.
%
%\subsection{Another \glsf{research-design}: Non-experimental Research}\label{ssect:AnotherResearchDesign}
%
%More here from \textcite{marczyk2005essentials}.
%
%\subsubsection{Single-study or Case-study research}\label{sssect:Single-studyOrCase-study}
%
%More here from \textcite{yin2009case}.
%
%\subsection{Design science research}\label{ssect:DesignScienceResearch}
%
%More here from \textcite{oates2007researching}.
%
%\subsubsection{Completing your \glsf{research-design}}\label{sssect:CompletingYourResearch}
%
%Add \Glsf{validity}, \Glsf{bias}, Reporting.
%
%\newcommand{\midtitle}[2]{\SetCell[c=3]{l,clear=preto}{\textbf{\Glsf{research-strategy}: #1}~\parencite{#2}}\\*%
%	\Glsf{code}&Description&Comments\\*%
%	}
%\SetCiteCommand{\parencite}
%\begin{longtblr}[%
%	expand=\midtitle,%\midtitle needs to be expanded so that pattern matching from LaTeX3 can work
%	caption={Research: tasks, codes and descriptions},%caption is an outer key
%	label={tab:researchTasks},%we can add a label
%]{%
%	width=\tablewidth,
%	colspec = {>{\arabic{rownum}–}r|X[2,l]X[7]},%first column right aligned, then 2/7 of remaining width
%%	column{1}={preto={\qquad}},%this doesn't seem to work
%%	row{1} = {font=\bfseries},%first row is bold, but don't need it because of \midtitle
%	measure=vbox,%needed to allow lists, \UseTblrLibrary{varwidth} added above
%	}
%%	\Glsf{code}&Description&Comments\\
%\midtitle{Experimental}{marczyk2005essentials,oates2007researching}
%	EXP&Experiment&\textcquote[p.35]{oates2007researching}{\textbf{Experiment}: focuses on investigating cause and effect relationships, testing \glsplf{hypothesis} and seeking to prove or disprove a causal link between a factor and an observed outcome. There is \enquote{before} and \enquote{after} measurement, and all factors that might affect the results are carefully excluded from the study, other than the one factor that is thought will cause the \enquote{after} result. (See Chapter 9.)}\\
%	 R&	Randomise \glsf{sample} from \glsf{population}& \textcquote[p.124]{marczyk2005essentials}{A true experimental design is one in which study participants are randomly assigned to experimental and control groups. We have discussed randomization in previous chapters, so this chapter will simply highlight the importance of randomization in terms of the strength of a \glsf{research-design}. Although randomization is typically described using examples such as rolling dice, flipping a coin, or picking a number out of a hat, most studies now rely on the use of random numbers tables to help them assign their research participants (as discussed in Chapters 2 and 3).}\\
%	 O&	Observe \glsf{phenomenon}&\textcquote[p.119]{marczyk2005essentials}{\Glsf{observation} is another versatile approach to \glsplf{datum} collection. This approach relies on the direct \glsf{observation} of the construct of interest, which is often some type of behavior. In essence, if you can observe it, you can find some way of measuring it. The use of this approach is widespread in a variety of research, educational, and treatment settings.}\\
%	 X&	Change experimental \glsf{variable}&\textcquote[p.127]{marczyk2005essentials}{experimental manipulation (independent \glsf{variable})}\\
%	 Y&	Change other \glsf{variable}&\textcquote[p.127]{marczyk2005essentials}{experimental manipulation (other \glsf{variable})}\\
%	\\
%\midtitle{Quasi-experimental}{marczyk2005essentials}
%	 NR&\Glsf{non-random-sampling}&\textcquote[p.138]{marczyk2005essentials}{when randomized designs are not feasible, researchers must often make use of quasi-experimental designs. A good rule of thumb is that researchers should attempt to use the most rigorous \glsf{research-design} possible, striving to use a randomized experimental design whenever possible (Campbell, 1969).
%	 
%	 Cook and Campbell (1979) present a variety of quasi-experimental designs, which can be divided into two main categories: nonequivalent comparison-group designs and interrupted time-series designs. In this section, we will discuss these two major groups of quasi-experimental designs, followed by a brief overview of single-subjects designs.}\\
%	 REV&	Before the intervention, then after&\textcquote[p.142]{marczyk2005essentials}{\textbf{Reversal Time-Series Design} Also known as an ABA design (detailed on page 145), the reversal time-series design is basically a multi-subject variation of the single-subject reversal design, which will be discussed later in this chapter. The basic goal of this design is to establish causality by presenting and withdrawing an intervention, or independent \glsf{variable}, one to several times while concurrently measuring change in the dependent \glsf{variable} (as depicted in the following). As in the simple time-series design, this design begins with a series of pretests to observe normal fluctuations in baseline. The name “reversal” refers to the idea that causality can be inferred if changes that occur following the presentation of an intervention diminish or “reverse” when the independent \glsf{variable} is withdrawn.}\\
%	 ABA&	Before, Intervention, After&See REV.\\
%	 ABABA&	Iterated ABA&See REV.\\
%	 ABABA...&Further Iterated ABA	&See REV.\\
%	 EC&	Establish control&\textcquote[p.144]{marczyk2005essentials}{As with time-series designs, single-subject designs typically begin by establishing a stable baseline. Establishing a stable baseline involves taking repeated measures of a participant’s behavior (dependent \glsf{variable}) prior to the administration of any intervention to make certain that the participant’s behavior is occurring at a consistent rate. To obtain a stable baseline, the researcher must make special efforts to control all relevant environmental variables that otherwise might affect the participant’s responses. If the researcher does not know, or is uncertain, about which variables are relevant, the researcher must attempt to keep the participant’s environment as constant as possible by maintaining highly controlled conditions.}\\
%	 1P&	Single participant&\textcquote[p.144]{marczyk2005essentials}{Not to be confused with non-experimental single-subject case studies, which are covered later in this chapter, the single-subject experimental design has a long and respected tradition in empirical research. According to Kazdin (2003c), single-subject experiments might be seen as true experiments because they “can demonstrate causal relationships and can rule out or make implausible threats to \glsf{validity} with the same elegance of group research” (p. 273). Similar to other experimental designs, the single subject design seeks to (1) establish that changes in the dependent \glsf{variable} occur following introduction of the independent \glsf{variable} (temporal precedence) and (2) identify differences between study conditions.
%	 
%	 The one way that single-subject designs differ from other experimental designs is in how they establish control, and thereby demonstrate that changes in a dependent \glsf{variable} are not due to extraneous variables. For example, experimental designs rely on randomization to equally distribute extraneous variables and on statistical techniques to control for such factors if they are found. Alternatively, single-subject designs eliminate between-subject variables by using only one participant, and they control for relevant environmental factors by establishing a stable baseline of the dependent \glsf{variable}. If change occurs following the introduction of the intervention, or independent \glsf{variable}, the researcher can reasonably assume that the change was due to the intervention and not to extraneous factors.}\\
%	 SB&	Stable Baseline&See 1P\\
%	 RC&	Retain control of Env&See 1P\\
%	 \\
%\midtitle{Non-experimental}{yin2009case,oates2007researching}
%	CASE&Case Study&\textcquote[p.35]{oates2007researching}{\textbf{Case study}: focuses on one instance of the \enquote{thing} that is to be investigated: an organization, a department, an information \glsf{system}, a discussion forum, a systems developer, a development project, a decision and so on. The aim is to obtain a rich, detailed insight into the \enquote{life} of that case and its complex relationships and processes. (See Chapter 10.)}\\
%	AR&\Glsf{action-research}&\textcquote[p.35]{oates2007researching}{\textbf{\Glsf{action-research}}: focuses on research into action. The researchers plan to do something in a real-world situation, do it, and then reflect on what happened or was learnt, and then begin another cycle of \glsf{plan-act-reflect-cycle}. (See Chapter 11.)}\\
%	ETH&Ethanography&\textcquote[p.35]{oates2007researching}{\Glsf{ethnography}: focuses on understanding the culture and ways of seeing of a particular group of people. The researcher spends time in the field, taking part in the life of the people there, rather than being a detached observer. (See Chapter 12.)}\\
%	CS& 	Choose subject&\textcquote[p.144]{marczyk2005essentials}{single-subject designs eliminate between-subject variables by using only one participant, and they control for relevant environmental factors by establishing a stable baseline of the dependent \glsf{variable}. If change occurs following the introduction of the intervention, or independent \glsf{variable}, the researcher can reasonably assume that the change was due to the intervention and not to extraneous factors.
%	
%	As with time-series designs, single-subject designs typically begin by establishing a stable baseline. Establishing a stable baseline involves taking repeated measures of a participant’s behavior (dependent \glsf{variable}) prior to the administration of any intervention to make certain that the participant’s behavior is occurring at a consistent rate. To obtain a stable baseline, the researcher must make special efforts to control all relevant environmental variables that otherwise might affect the participant’s responses. If the researcher does not know, or is uncertain, about which variables are relevant, the researcher must attempt to keep the participant’s environment as constant as possible by maintaining highly controlled conditions.}\\
%	COMP&	Comprehensive description&\textcquote[p.148]{marczyk2005essentials}{the focus of the case-study approach is on individuality and describing the individual as comprehensively as possible. The case study requires a considerable amount of information, and therefore conclusions are based on a much more detailed and comprehensive set of information than is typically collected by experimental and quasi-experimental studies.}\\
%	IDIP&	In-depth interviews with participants&\textcquote[p.148]{marczyk2005essentials}{Case studies of individual participants often include in-depth interviews with participants ...}\\
%	IDIC&	In-depth interviews with collaterals&\textcquote[p.148]{marczyk2005essentials}{...and collaterals (e.g., friends, family members, colleagues), review of medical records, \glsf{observation}, and excerpts from participants’ personal writings and diaries}\\
%	SUR& Surveys&\textcquote[p.33]{oates2007researching}{\textbf{Survey}: focuses on obtaining the same kinds of \glsplf{datum} from a large group of people (or events), in a standardized and systematic way. You then look for patterns in the \glsplf{datum} using statistics so that you can generalize to a larger \glsf{population} than the group you targeted. (See Chapter 7.)}\\
%	RA&	Review of artefacts&\textcquote[p.148]{marczyk2005essentials}{According to Kazdin (1982), the major characteristics of case studies are the following:
%	\begin{itemize}
%		\item they involve the intensive study of an individual, family, group, institution, or other level that can be conceived of as a single unit.
%		\item the information is highly detailed, comprehensive, and typically reported in narrative form as opposed to the quantified scores on a dependent measure.
%		\item they attempt to convey the nuances of the case, including specific contexts, extraneous influences, and special idiosyncratic details.
%		\item the information they examine may be retrospective or archival.
%	\end{itemize}}\\
%%	(RQ)&	Research question??\\
%	PROPS&	Identify propositions&\textcquote[p.28]{yin2009case}{\textbf{Study propositions} [...] each \glsf{proposition} directs attention to something that should be examined within the scope of study.}\\
%	UNITS&	Identify units&\textcquote[p.29]{yin2009case}{\textbf{Unit of analysis} [...] related to the fundamental problem of defining what the \enquote{case} is [... what the primary unit of analysis is].
%	
%Without such questions and propositions, you might be tempted to cover \enquote{everything} about the individual(s), which is impossible to do.}\\
%	LINKS&	Identify how is \glsplf{datum} linked to propositions&\textcquote[p.34ff]{yin2009case}{be aware of the main choices and how they might suit your case study]}\\
%	CRITS&Which are criteria to interpret findings&\textcquote[p.34]{yin2009case}{Criteria for interpreting a study's findings}\\
%	THD&\Glsf{theory} Development&\textcquote[p.35]{yin2009case}{[including types on p.37]}\\
%	GEN&\Glsf{generalisation}&\textcquote[p.38]{yin2009case}{[including fig 2.2]}\\
%	NAR&Narrative&\textcquote[p.121]{yin2009case}{Certain types of narrative, produced by a case study investigator upon completion of all \glsplf{datum} collection, also may be considered a formal part of the database and not part of the final case study report. The narrative reflects a special practice that should be used more frequently: to have case study investigators compose open-ended answers to the questions in the case study protocol. This practice has been used on several occasions in multiple-case studies designed by the author (see BOX 24). 
%	
%	[Box 24]
%	
%		In such a situation, each answer represents your attempt to integrate the available \glsf{evidence} and to converge upon the facts of the matter or their tentative interpretation. The process is actually an analytic one and is the start of the case study analysis. }\\
%	{NSC\\NEI\\NID}&Nuance from the specific context/extraneous influences/idiosyncratic details&\textcquote{kazdin1982single}{According to Kazdin(1982), the major characteristics of case studies are the following:
%		\begin{itemize}
%			\item  They involve the intensive study of an individual, family, group, institution, or other level that can be conceived of as a single unit.
%			\item the information is highly detailed, comprehensive, and typically reported in narrative form as opposed to the quantified scores on a dependent measure.
%			\item they attempt to convey the nuances of the case, including specific contexts, extraneous influences, and special idiosyncratic details.
%			\item the information they examine may be retrospective or archival.
%		\end{itemize}}\\ 
%%	NEI&Nuance from extraneous influences\\ 
%%	NID&Nuance from idiosyncratic details\\
%	\\
%\midtitle{\Glsf{design-science-research}}{oates2007researching}
%	D\&C&Design and creation&\textcquote{oates2007researching}{\textbf{Design and creation}: focuses on developing new IT products, or artefacts. Often the new IT product is a computer-based \glsf{system}, but it can also be some element of the development process such as a new construct, \glsf{model} or method. (See Chapter 8.)}\\
%%	PSA&Problem solving awareness&\textcquote[p.111]{oates2007researching}{Awareness is the recognition and articulation of a problem, which can come from studying the literature where authors identify areas for further research, or reading, about new findings in another discipline, or from practitioners or clients expressing the need for something, or from field research or from new developments in technology.}\\
%%	PSS&Problem solving suggestion&\textcquote[p.112]{oates2007researching}{Suggestion involves a creative leap from curiosity about the problem to offering a very tentative idea of how the problem might be addressed}\\
%%	PSD&Problem solving development&\textcquote[p.112]{oates2007researching}{Development is where the tentative design idea is implemented. How this is done depends on the kind of IT artefact being proposed. For example, an algorithm might need the construction of a formal proof. A new user interface embodying novel theories about human cognition will require software development. A systems development method will need to be captured in a manual that can then be followed in a systems development project. A new approach in digital art might require the development of an art portfolio tracing the development of the artist's creative ideas.}\\
%%	PSE&Problem solving evaluation&\textcquote[p.112]{oates2007researching}{Evaluation examines the developed artefact and looks for an assessment of its worth and deviations from expectations.}\\
%%	PSC&Problem solving conclusion&\textcquote[p.112]{oates2007researching}{Conclusion is where the results from the design process are consolidated and written up, and the knowledge gained is identified, together with any loose ends - unexpected or anomalous results that cannot yet be explained and could be the subject of further research.}\\
%	PRU&Problem Understanding&\textcquote[p.49]{hall2017a-design}{gaining an understanding of the real-world environment in which the problem is located, and of the problem owner’s identified need}\\
%	PRV&Problem Validation&\textcquote[p.49]{hall2017a-design}{agreeing with the problem owner that the problem is representative, a form of validation}\\
%	SOU&Solution Understanding&\textcquote[p.49]{hall2017a-design}{producing the solution}\\
%	SOV&Solution Validation&\textcquote[p.49]{hall2017a-design}{convincing the problem owner that the solution meets the agreed recognised need in the agreed real-world environment to their satisfaction, another form of validation}\\
%\midtitle{General}{}
%	LITREV& Literature Review&\textcquote[p.33]{oates2007researching}{literature review in figure 3.1}\\
%	VALID&	Threats to \glsf{validity}&\textcquote[p.40]{yin2009case}{fours (general) tests for \glsf{validity}}\\
%	\Glsf{bias}&	\Glsf{reflection} on \glsf{bias}&\textcquote[p.72]{yin2009case}{[Avoiding \glsf{bias} for case studies]}\\ 
%	REP&	Reporting&Something here\\
%	TRI&	\Glsf{triangulation}&\textcquote[p.37]{oates2007researching}{The use of more than one \glsf{data-generation-method} to corroborate findings and enhance their \glsf{validity} is called method \glsf{triangulation}. Many types of \glsf{triangulation} are possible in a research project:
%	%
%\begin{itemize}
%\item method \glsf{triangulation}: the study uses two or more \glsplf{datum} generation methods.
%\item strategy \glsf{triangulation}: the study uses two or more research strategies.
%\item time \glsf{triangulation}: the study takes place at two or more different points in time.
%\item space \glsf{triangulation}: the study takes place in two or more different countries or cultures to overcome the parochialism of a study based in just one country or culture.
%\item \glsf{investigator-triangulation}: the study is carried out by two or more researchers who then compare their accounts.
%\item theoretical \glsf{triangulation}: the study draws on two or more theories rather than one theoretical perspective only.
%\end{itemize}
%%
%
%
%Researching Information Systems and Computing
%\Glsf{triangulation} gives researchers multiple modes of \enquote{attack} on their research question.
%However, researchers differ over whether they should expect \glsf{triangulation} of method or time or space, and so on, to lead to consistency of findings. It depends on their underlying research philosophy (see Chapters 19 and 20 for a detailed explanation). \enquote{Positivists} subscribe to the idea of a single \enquote{truth} or \enquote{reality} and would expect the multiple lines of attack to lead to a consistent set of findings. Interpretivists', on the other hand, do not subscribe to the idea of a single reality, believing any notion of \enquote{reality} to be constructed by individuals and groups, so there are multiple realities for people in our world, and different research approaches are likely to lead to different findings. \Glsf{interview} \glsplf{datum} about recollections of a meeting and company minutes of the same meeting, for example, are two different \enquote{stories}, created by different people for different audiences. Interpretivists would not always expect to see convergence in the \glsplf{datum} they generate using \glsf{triangulation}.}\\
%\midtitle{\Glsplf{datum} Generation}{oates2007researching}
%	INT&\Glsf{interview}&\textcquote[p.36]{oates2007researching}{\textbf{\Glsf{interview}}: a particular kind of conversation between people where, at least at the beginning of the \glsf{interview} if not all the way through, the researcher controls both the agenda and the proceedings and will ask most of the questions. One-to-one and group interviews are possible. (See Chapter 13.)}\\
%	OBS&\Glsf{observation}&\textcquote[p.36]{oates2007researching}{\textbf{Observations}: watching and paying attention to what people actually do, rather than what they report they do. Often involves looking, but it can involve the other senses too: hearing, smelling, touching and tasting. (See Chapter 14.)}\\
%	QUES&Questionnaire&\textcquote[p.36]{oates2007researching}{\textbf{Questionnaire}: a pre-defined set of questions assembled in a pre-determined order. Respondents are asked to answer the questions, often via multiple choice options, thus providing the researcher with \glsplf{datum} that can be analysed and interpreted. (See Chapter 15.)}\\
%	DOC&\Glsplf{document}&\textcquote[p.36]{oates2007researching}{\textbf{\Glsplf{document}}: \glsplf{document} that already exist prior to the research (for example, policy \glsplf{document}, minutes of meetings and job descriptions) and \glsplf{document} that are made solely for the purposes of the research task (for example, a researcher's logbook or design models). Also includes \enquote{multimedia \glsplf{document}}: visual \glsplf{datum} sources (for example, photographs, diagrams, videos and animations), aural sources (for example, sounds and music) and electronic sources (for example, websites, computer games and electronic bulletin boards). (See Taspic
%	Chapter 16.)}\\
%	EVAL&Evaluate&\textcquote[p.40]{oates2007researching}{\textbf{Evaluating the \Glsf{research-process}}
%		Now that you know something of the \glsf{research-process}, you can start to analyse and evaluate how well other researchers have described their process. Use the Evaluation Guide' below to help you.
%
%EVALUATION GUIDE: \Glsf{research-process}
%%
%\begin{enumerate}%[start=0,label={(\bfseries R\arabic*):}]
%\item do the researchers make clear their research question(s)?
%\item do the researchers explain the \glsf{theory} (ies) they use to conceptualize the research topic?
%\item do the researchers make clear both their strategy and the \glsf{data-generation-method}(s) within that strategy?
%\item do the researchers indicate their criteria for judging the success or usefulness of their work?
%\item is there a clear process summarized, from the original motivation and literature review through to final outcome(s)? If not, how does that affect your confidence in the research and its reporting?
%\end{enumerate}}\\
%\end{longtblr}
%
%\subsection{Alternative models of the \glsf{research-process}}\label{ssect:AlternativeModelsOf}
%
%See \parencite[p.39]{oates2007researching} for two more models for the \glsf{research-process} (in the context of Information Systems and Computing)
%%
%\begin{itemize}
%\item conceptualise, operationalise, generalise;
%\item the SLDC (Software Development Life Cycle) Analogy.
%\end{itemize}
%%
%
%
%
%%As a running example, we'll be working with the following research objectives, which you saw in Example Cref{ex:machinelearning} in~\Cref{stage1}\footnote{See page~\pageref{ex:machinelearning}.}\todo{What was the \glsf{research-problem}?}:
%
%\subsection{Decomposing objectives into tasks}\label{ssect:DecomposingObjectivesInto}
%
%You've chosen your \glsf{research-design} based on area, and you've got your research objectives from Cref{sect:???}. How do you go about mapping one into the other?
%
%Our suggested template for creating objectives had three components: identify, assess, and recommend.
%%
%%
%\begin{description}
%\item [identify:] literature review; questionnaire, interviews; problem solving awareness, ...
%\item [assess:]	what goes here? problem solving suggestions; interviews; problem solving development; ...
%\item [recommend:] what goes here? problem solving evaluation; problem solving conclusion; \glsf{validity}; \glsf{bias}; ...
%\end{description}
%%
%
%\todo{Turn this example into identifying which \glsf{research-design}.}
%
%\begin{example}{Recap: Applying \Glsf{machine-learning}}
%In~\Cref{stage2}, we refined Clara's \glsf{research-aim}, which was:
%%
%\blockquote{to apply \Glsf{machine-learning} (ML) to improve the accuracy of resources and time forecasting in the context of small engineering plants}
%%
%to three following three objectives:\todo[inline]{JGH: needs doing if not already done}
%
%\begin{description}%[start=0,label={(\bfseries R\arabic*):}]
%\item [Objective 1] to identify which ML techniques are applicable to resource and time forecasting in the context of small engineering plants, which will allow us to identify specific ML techniques to be used in the project, to ensure the work is feasible within the time-frame of the project. 
%
%\item [Objective 2] to test the accuracy of forecasting of those techniques which will allow us to investigate and compare how accurate the chosen techniques are in their forecasting application. 
%
%\item [Objective 3] to provide recommendations as to how integrate those techniques effectively in engineering practice in order to improve forecasting accuracy which will allows us to draw some conclusions from the research conducted and make recommendations to improve professional practice.
%\end{description}
%
%Note how those objectives were designed to build on each other and, when successfully completed, they'd contribute to meet the overall aim.
%\end{example}
%
%\begin{example}{Example – cont'd}In our example, the first objective is met once we have identified the relevant ML techniques. There are two complementary ways to do this: to look at the literature and to ask practitioners. As a result, we could break this objective down into the tasks, and \glsplf{deliverable}, indicated in the following table
%
%\begin{longtable}{@{}p{0.1\textwidth}@{}p{0.9\textwidth}@{}}
%\caption{Objective 1: to identify which ML techniques...}\\
%\toprule
%\textbf{Task} & \textbf{Deliverable} \\\midrule
%\tabletitle{to identify relevant ML techniques in the \glsf{academic-literature}} & a collection of relevant ML techniques reported in the literature \\\\
%\tabletitle{to ask practitioners which techniques they employ} & a collection of relevant ML techniques used in professional practice \\
%\bottomrule
%\end{longtable}
%\end{example}
%
%\begin{activity}[{Establishing tasks and deliverables}] Consider your research objectives. For each, identify related tasks and \glsplf{deliverable}.\todo[inline]{I don't think I could do this at this point.}
%
%\begin{guidance}You should draw a table similar to that in our running example. You should ensure that the tasks provide a reasonable break down of your objectives into discrete pieces of work.
%\end{guidance}\end{activity}
%%%Hack to correct tcbox behaviour
%\color{black}
%
%Your tasks and \glsplf{deliverable} don't need to be perfect in~\Cref{stage3} – there are two more stages to perfect them after all – and are likely to be revised as you progress through your project. However, it is important that you have thought about specific work you will need to carry out to meet your objectives.
%
%\subsubsection{Relating tasks to research methods}\label{sssect:RelatingTasksTo}
%The way to carry out your tasks and meet your objectives is through the application of research methods.
%
%\begin{example}{EXAMPLE - cont'd }Following on from our previous example, we have extended the table to include an indication and justification of candidate research methods for each task.
%
%\begin{tblr}{colspec={XXXX},
%row{1}={font=\bfseries},
%}
%%\caption{\textbf{Objective 1: to identify which ML techniques...}
%%\hline[1pt]
%Task&\Glsf{deliverable}&Relevant research methods&Justification and feasibility\\
%%\hline[0.5pt]
%\SetCell[c=3]{l}{to investigate the \glsf{academic-literature} in order to identify relevant ML techniques}\\ 
%&a collection of relevant ML techniques reported in the literature & review of existing literature & I can access relevant literature via my university library\\
%\SetCell[c=3]{l}{to ask practitioners which techniques they employ}\\ 
%& a collection of relevant ML techniques used in professional practice & questionnaire, possibly followed by interviews & I have access to professional networks, which I could use to distribute the questionnaire, and possibly to recruit participants for follow-up interviews \\
%%\hline[1pt]
%\end{tblr}
%\end{example}
%
%Note that the choice of research methods in relation to your research tasks is an essential part of your \glsf{research-design}. In fact, the two influence each other: your objectives and related tasks direct you towards specific research methods, which in turn have to be part of your overall \glsf{research-design}.
%
%\begin{activity}[{Associating methods to tasks and deliverables}] Extend your tasks and \glsplf{deliverable} table with your candidate research methods, including stating why they apply and are feasible for your project. Revise your \glsf{research-design} draft from~\Cref{stage2} so that is consistent with those choices.
%
%\begin{guidance}
%Refresh your understanding of chosen research methods from the study work you carried out in~\Cref{stage2}. It is important you keep reviewing your choices with your supervisor.
%\end{guidance}
%\end{activity}
%%%Hack to correct tcbox behaviour
%\color{black}
%
%\subsubsection{Research task deliverables}\label{sssect:ResearchTaskDeliverables}
%
%\todo[inline]{Add something here}
%
%\subsection{Research procedures}\label{ssect:ResearchProcedures}
%\todo{What's the relationship to objectives and tasks?}
%
%Once you have chosen the set of research methods you will apply, you must establish exactly how you will do that, something we refer to as \textbf{research procedures}.
%
%Your research procedures will be method specific, in that each method you choose to apply will come with recommended practices, which you will need to contextualise to your own project needs, including your access to participants, \glsplf{datum} or other kind of \glsf{evidence}. For instance, there are plenty of guidelines in the literature on how to design \glsplf{questionnaire}, including which type of questions to include and how to phrase them. There are also recommendations concerning testing the questionnaire design prior to its use, and of course, there are many ways a questionnaire can be administered. In writing your procedures for this \glsf{research-method}, you would have to be specific on how each of the above applies in your project.
%
%It is important, therefore, that you master the research methods of your choice, starting by reviewing once again the related \glsf{academic-literature}.
%
%\begin{activity}[{Sketching research procedures}] Consider the research methods you intend to apply, and the related review you conducted in~\Cref{stage2}. Reconsider those materials, possibly going back to the literature sources, to learn how to apply the methods effectively within your project.
%
%For each method, sketch possible procedures of application, ensuring you make appropriate reference to the literature you have reviewed and best practice guidelines therein.
%
%\begin{guidance}It is possible that the review you conducted in~\Cref{stage2} is not sufficient, in which case you will need to extend it to complete this activity.
%
%You should focus on practical aspects of applying the methods, including specific processes and techniques to gather, summarise and present your \glsf{evidence} in your reports.
%
%Depending on the extent you need to review further \glsf{academic-literature}, this activity could be quite substantial, so you should set aside up to 20\% of your study time to complete it.
%
%\end{guidance}\end{activity}
%%%Hack to correct tcbox behaviour
%\color{black}
%
%\subsection{Assessing \glsf{validity}}\label{ssect:AssessingValidity}
%As your intended \glsf{research-design} becomes clearer, you will soon be testing some aspects of it in your pilot work. Before you do that, however, you need to consider if the choices you have made will allow you to gather \glsf{evidence} and derive findings in a systematic, rigorous, repeatable and reliable fashion so as to address your \glsf{research-problem}. This is referred to as assessing the overall \textbf{\glsf{validity}} of your \glsf{research-design}, which is broken down into the following considerations.
%
%\textbf{Construct \glsf{validity}} asks whether you have put your design together logically by focusing on the relationship between \glsf{evidence} and \glsf{research-problem}. Here you ask yourself whether the \glsf{evidence} you will generate through your chosen \glsf{research-design} will be accurate and relevant to address your \glsf{research-problem}. This tests the logical coherence of your aim, objectives, tasks, methods and \glsplf{deliverable} in relation to the \glsf{research-problem} and the knowledge gap you intend to address. With construct \glsf{validity}, you are asking: \emph{have I designed my research in the right way?}
%
%\textbf{\Glsf{internal-validity}} is concerned with the way you gather and analyse \glsf{evidence}. All research strategies and methods come with recommendations of good practice to ensure that your research is both systematic, repeatable and reliable. In your work, you need to ensure that you follow such practices and are aware of possible pitfalls. For instance, in \glsf{experimental-research} you need to control all factors which may effect outcomes beyond those under study: failing to exercise such control will lead to observations and measurements which are unreliable. In assessing \glsf{internal-validity}, you should also take into account limitations of human perception and cognition, and any potential personal \glsf{bias}. With \glsf{internal-validity}, you are asking: \emph{have I executed my research in the right way?}
%
%\textbf{External \glsf{validity}} relates to the extent you will be able to generalise your findings beyond the immediate context of your research. For instance, you may conduct a case study within a specific organisation, so here you are asking whether and how what you have discovered may apply to other organisations. With external \glsf{validity}, you are asking: \emph{will my research lead to findings that apply somewhere else?}
%
%Anything that gets in the way of \glsf{validity} in research is termed a \textbf{threat to \glsf{validity}}. Different research strategies and methods are exposed to different threats, something you should have encountered in your review of the literature on your chosen methods.
%
%\begin{activity}[{Assessing \glsf{validity} of \glsf{research-design}}] Conduct an initial assessment of your chosen \glsf{research-design} in relation to the three kind of \glsf{validity} discussed above. Write down a short summary of your thinking in support of each, and of possible threats to \glsf{validity} you envisage.
%
%\begin{guidance}You may need to refer back to the literature you have reviewed to identify specific threats which apply to your chosen research methods and strategies.
%
%You won't be able to address this in full at this point in your project, particularly the \glsf{internal-validity}, which refers to the execution of your \glsf{research-design}. Nevertheless, it is important for you to consider \glsf{validity} and possible threats from the onset. You will return to this topic at the end of your project, as part of the overall assessment of your research, to reflect on the \glsf{validity} of your completed research.
%
%\end{guidance}\end{activity}
%%%Hack to correct tcbox behaviour
%\color{black}
%
%\subsection{Conducting your pilot work}\label{ssect:ConductingYourPilot}
%
%Your \textbf{pilot work} will be a small scale test of some of the methods and procedures you will apply in the next stage of your project. Its main function is to help you assess the feasibility of your \glsf{research-design}, or at least some aspects of it, and build your confidence in the approach you have chosen.
%
%As such, your pilot work may not contribute directly to your aim and objectives, but it should help you decide whether you can actually do what you have planned to do, or inform how your \glsf{research-design} and project plan should change instead.
%
%There are no constraints on what you can do for your pilot work, other than you should exercise some aspect of your \glsf{research-design}. It is therefore essential that you agree what you are going to do with your supervisor first.
%
%\begin{activity}[{Planning and executing your pilot work}] Plan your pilot work and discuss your plan with your supervisor.
%
%Once you have agreed the way forward, execute your plan and write a summary of both its execution and outcomes.
%
%\begin{guidance}
%This is a substantial activity, which will take you up to 35\% of your study time.
%
%Your summary should include:
%%
%\begin{itemize}
%\item an indication of which aspects of your \glsf{research-design} your pilot work was concerned with
%\item any methods and procedures applied
%\item any \glsplf{datum} or \glsf{evidence} gathered, including possible modelling, artefact design or prototyping, appropriately presented and summarised
%\item lessons learnt and any resulting revision to your \glsf{research-design} and project plan, particularly in relation to construct and \glsf{internal-validity} of your \glsf{research-design}.
%\end{itemize}
%%
%To complete this activity successfully, it is essential that you agree your pilot \glsf{work-plan} with your supervisor upfront, and discuss your progress on a regular basis.
%\end{guidance}\end{activity}
%%%Hack to correct tcbox behaviour
%\color{black}
%
%\subsection{Reporting in~\Cref{stage3}}\label{ssect:ReportingInStage}
%At the end of~\Cref{stage3}, you should complete a report, extending that of~\Cref{stage2} and covering the work you have carried on in this stage. The structure we suggest and an indication of the contents are shown in Cref{tab:reportStructure}.
%
%%%Report Structure Table is repeated throughout the thesis. This is the template
%%%Format is:
%%%\begin{ReportStructureTable}
%%%	\tabletitle{Section 1: Introduction}\\
%%%	\begin{enumerate}[label={1.\arabic*:}]
%%%	\item background to the research 
%%%	\item justification for the research 
%%%	\end{enumerate}
%%%	& This section should provide an introduction to your research topic in its wider context (as background) and your justification of why the research is worth pursuing. It should be well articulated and supported by \glsf{evidence} \\
%%%\end{ReportStructureTable}
%%%Still to do: remove space from above enumerate environment
%%%Sets the chapter across two columns in bold
%\begin{ReportStructureTable}{tab:reportStructure}
%\tabletitle{Title} & Your title should succinctly capture your \glsf{research-problem} and aim\\\\
%\tabletitle{Section 1: Introduction}\\
%\begin{enumerate}[label={1.\arabic*:}]
%\item background to the research 
%\item justification for the research 
%\end{enumerate}
%& This section should provide an introduction to your research topic in its wider context (as background) and your justification of why the research is worth pursuing. It should be well articulated and supported by \glsf{evidence} \\\\
%\tabletitle{Section 2: Literature review}\\
%\begin{enumerate}[label={2.\arabic*:}]
%\item review of existing relevant knowledge 
%\item critical summary, including knowledge gap to be addressed by the research 
%\end{enumerate}
%& Your review should provide a critical account of your in-depth engagement with the academic (and other) relevant literature, including identifying key trends, ideas and possible knowledge gaps. Most of your citations should point to academic articles. Your critical summary should highlight key insights from your review and provide a strong justification for your proposed research. Both coverage and depth of your review matter. You should ensure that your review is well structured, with a logical narrative flow and your arguments are well supported by \glsf{evidence}  \\\\
%\tabletitle{Section 3: Research definition}\\
%\begin{enumerate}[label={3.\arabic*:}]
%\item problem statement 
%\item aim, objectives, tasks and \glsplf{deliverable}
%\item knowledge contribution
%\end{enumerate}
%& You should ensure that your \glsf{research-problem} is well articulated and appropriate for your course and your personal and professional circumstances, that your aim and objectives are consistent with \glsf{research-problem}, that tasks and \glsplf{deliverable} break down your objectives appropriately and are clearly related to your chosen research methods, and that the intended knowledge contribution of your research is clearly articulated \\
%\tabletitle{Section 4: \Glsf{research-design}}\\\\
%\begin{enumerate}[label={4.\arabic*:}]
%\item \glsf{evidence} and \glsplf{datum} 
%\item \glsf{research-strategy} and methods
%\item research procedures
%\item ethical, legal and EDI considerations
%\end{enumerate}
%& This section should demonstrated your critical engagement with all elements of \glsf{research-design}, including a detailed account of the \glsplf{datum} and \glsf{evidence} needed in your research, the research methods and research strategies you will to apply, and how you will apply them within your project. Your account should be supported by a clear rationale and insights from the related literature, and appropriately justified in relation to your \glsf{research-problem}, aim and objectives. It should also demonstrate your careful consideration of ethical and legal matters, and that your research will comply with your course and university requirements\\\\
%\tabletitle{Section 5: Analysis and interpretation}\\
%\begin{enumerate}[label={5.\arabic*:}]
%\item pilot work
%\end{enumerate}
%& This section should report on a well thought-out pilot work which clearly and competently test some significant aspect of your \glsf{research-design}. It should demonstrate good critical \glsf{reflection} on outcomes and highlight any adjustments needed as a result. \\\\
%\tabletitle{Section 6: Assessment of your proposed research}\\
%\begin{enumerate}[label={6.\arabic*:}]
%\item \glsf{qualification} fit
%\item personal and professional fit
%\item technical skills and resources needed
%\item statement of feasibility
%\item personal \glsf{reflection} on \glsf{research-process}
%\end{enumerate}
%& In this section you should continue to argue how your research is a good fit across all criteria. You should provide a clear rationale as to why you think what you are proposing is feasible. You should also reflect on your growing understanding of the \glsf{research-process}, including key learning and aspects you have found particularly challenging. \\\\
%\tabletitle{Section 7: Planning, scheduling and risk assessment}\\
%\begin{enumerate}[label={7.\arabic*:}]
%\item key priorities in follow-up stage
%\item personal and professional fit
%\item risk assessment
%\end{enumerate}
%& In this section you should reflect on the progress you have made in~\Cref{stage2} and establish your priorities for the next stage. You should also review your risk assessment as appropriate.\\\\
%\tabletitle{Section 8: References}\\ & You should keep your growing references in good order and ensure you apply the required bibliographical style consistently. Ideally, you should use a BMT to generate and integrate your references within your report\\\\
%\textbf{Appendix A: Work schedule}& Your revised \glsf{work-plan}\\\\
%\textbf{Appendix B: Risk assessment table}& Your revised risk table \\
%\bottomrule
%\end{ReportStructureTable}
%
%\endinput
%
%\begin{activity}[{Putting your report together}] Using your word processor of choice, and starting from your previous report, complete your~\Cref{stage3} report by applying the structure and guidance in Cref{tab:ReviewCrit}, and making good use of your notes and summaries from all related activities you have carried out so far.
%
%\begin{guidance}In this first pass at putting together your report, you should focus primarily on completeness, ensuring that each section includes at least draft content.
%\end{guidance}\end{activity}
%%%Hack to correct tcbox behaviour
%\color{black}
%
%As in the previous stages, after you have filled in your report you should review and revise it iteratively until you are happy with your account, and are ready to move on. 
%
%\begin{table}[htbp]
%\begin{minipage}{\linewidth}
%\setlength{\tymax}{0.5\linewidth}
%\centering
%\caption{Criteria to review your report\label{tab:ReviewCrit}}
%\small
%\begin{tabulary}{\tablewidth}{@{}LL@{}} \toprule
% \textbf{Criteria} & \textbf{Prompts} \\
%\midrule
%
% \tabletitle{Completeness} & Are all sections of the suggested structure completed in line with the guidance provided? \\
% \tabletitle{Good academic writing practices} & Have you applied good academic writing practices throughout? \\
% \tabletitle{Logical structure and flow} & Have you structured your narrative appropriately to ensure a logical flow of arguments? \\
% \tabletitle{Supporting references or \glsf{evidence}} & Are your key arguments supported by appropriate references or other \glsf{evidence}? \\
% \tabletitle{Citation and reference style} & Do all your citations and references comply with the required bibliographical style? \\
% \tabletitle{Avoiding \glsf{plagiarism}} & Have you acknowledged the work of others and distinguished it from your own appropriately? \\
% \tabletitle{Standard of English (or any modern language you use)} & Have you proof-read your report carefully to remove all typos and grammatical errors? \\
%\bottomrule
%
%\end{tabulary}
%\end{minipage}
%\end{table}
%
%\begin{activity}[{Reviewing your report}] Apply the criteria in Table 1 to review your current report and write up a summary of your assessment.
%
%\begin{guidance}For each criteria, consider the related prompts to help you assess your report overall, and write down any further work needed for your next stage.
%\end{guidance}\end{activity}
%%%Hack to correct tcbox behaviour
%\color{black}
%
%\subsection{\Glsf{reflection}:~\Cref{stage3}}\label{ssect:reflection}
%
%%%More here
%
%%%Repeated \glsf{reflection} activity
%%%Repeated Activity for all reflections
\begin{question}[subtitle={Activity}]
$<$Needs assessing for content and structuring into activity + guidance$>$

This activity has four parts: the first is something you should be doing regularly, but won't make you into a disobedient or indocile thinker. The second, third and fourth may help you get started and keep going.

Part 1: Think about your study this far -- using this book and anything you've done for your dissertation in parallel -- as a journey. More from elsewhere, including   !!.

Part 2: think about yourself and the way you think. How does your desk look? Is it messy or tidy? Do the same for your computer desktop. Is it empty or are there hundreds of files strewn across it? Do you think your tidiness or untidiness will affect the way you do your research? How about how you keep your -- critically important -- bibliographic database which may contain up to a hundred academic\footnote{It's not unknown to have more than a hundred.} and other articles by the time you're finished?

Part 3: think about the context of your research. Which professional pressures are there on you to succeed in solving your research problem? Pressures could come in many forms: financial -- there's a promotion for you at the end of it; peer -- your colleagues know that you are studying will have good expectations of your result and you'll want to prove them right\footnote{Or wrong, depending on the colleague!}. Are you sponsored by your employer? Will you be able to report a negative outcomes to your research, for instance, that there is no solution to our problem using the current technology stack? A negative result is a very good research outcome, even if it tends to satisfy fewer non-academics than a positive result.

Which family pressures do you feel? It's' not unusual that you will have given up a paying role to study, moving the responsibility to provide onto another member of your family. What are their expectations?

Part 4: what's that thought nagging at the back of your mind? Is it ``How will I start?'' Or ``Will I be able to dedicate enough time to this?'' Or ``Can I really do this?''. Or ''Is ``shouldn't I be bringing in a wage rather than studying?''

You may be one of the lucky ones that doesn't have such negative thoughts, but negative thoughts are a very natural part of steps into the unknown. And research is precisely that, a step into the unknown. At least if you are aware of the doubts you naturally have, you can manage them. Think about making even the tiniest of steps forward in your research visible and celebrated! Work with Kansan boards where progress is encouragingly visible as you move a task from the inbox to the outbox. If you have concerns about managing your time, start using one of the many tools out there that break time up into manageable units and help manage it for you. If your concerns are about how to organise your thoughts, look into mind maps, lists, todo lists.

Thinking early and often through reflection is a powerful way of doing better. Do it well and your final report will be better than you will have expected.

It's worth saying that, at the end of what could be an exhausting journey, you will not fully appreciate your achievements. That realisation may have to wait until you are rested, graduated, or some distant time later.

But it will come.

\begin{guidance}
%Hack to correct tcbox behaviour
\color{black}
Something here
\end{guidance}\end{question}
%%Hack to correct tcbox behaviour
\color{black}



