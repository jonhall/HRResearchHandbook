\begin{PartTitlePage}{The \Glsplf{datum}}{stage4}
In~\Cref{stage4} you will focus on your research methods and their application to generate and analyse the \glsplf{datum} you need in your project to make your original contribution to knowledge.	
\end{PartTitlePage}


You've now reached~\Cref{stage4}, which means you are half way from the completion of your project. In this stage you will be in the midst of your \glsplf{datum} generation and analysis, which is possibly the most exciting, yet demanding, part of your research: this is where you get your opportunity to make that contribution to knowledge.

This stage assumes that you have chosen your research strategies and methods and are now in a position to deepen your understanding of how to apply them for your \glsplf{datum} generation and analysis\footnote{If that's not the case then you should go back to~\Cref{stage3}. You should also discuss your progress with your supervisor, revisiting your project timescale and risk.}.


\chapter[Stage IV Activities and Outcomes]{\Cref{stage4} Activities and Outcomes}\label{c:Stage4outcomes}

As usual, a Research Activity table and a Writing Outcomes table will help you structure and guide your work.

\section{Your Research Activities for this stage}

The research activities which are in focus in~\Cref{stage4} are shown in~\Cref{stage4ResearchActivities}, which also provides prompts for your interaction with your supervisor during this stage. 

Generating and analysing \glsplf{datum} will constitute by far your major effort in this stage (35\% of total stage effort). However, before you do that you will need to put some significant effort (20\% of total effort for this stage) into refining your methodology by adding detail to the methods you will apply and, particularly, the procedures you will follow to do so. Related to this, you should ensure that your methods execution is consistent with your objectives (5\% of total effort for this stage) and reflected in the tasks you will need to complete, which should be captured in your work plan (5\% of total effort for this stage).

\begin{SimpleNColTable}{stage4ResearchActivities}{4}{\RActivitiesTableCaption{4}}[R[4]cR[8]R[8]]
Research activity & Effort & Description &  Supervisor Interaction Focus\\
Identifying the \glsf{research-problem}&2\%&Adjust, if needed&\\
Reviewing the literature&2\%&Adjust, if needed&\\
Setting \glsf{research-aim} and objectives&3\%&Ensure that appropriate methods are mapped to each objective, and break down tasks further in line with your method procedures &Suitability of mapping of methods to objectives, and of tasks to method procedures\\
Developing the \glsf{research-design}&20\%&{Augment your \glsf{research-methodology} with detail of your research methods and the procedures to apply them, and how to deal with their potential weaknesses}&Suitability of procedures and mitigation approaches\\
Generating and analysing \glsf{evidence}&35\%&Apply your chosen methods to start generating and analysing your \glsplf{datum}&Initial application of methods of choice and any improvements required \\
Interpreting and evaluating findings&0\%&n/a&\\
Writing up&20\%&Achieve the writing outcomes of~\Cref{stage4WritingOutcomes} & Demonstration of good academic writing and any improvements required\\
\Glsf{reflection} and \glsf{reflexivity} &10\% & Apply to~\Cref{stage4} work and experience as you go along & \Glsf{reflection} on research weaknesses and how to address them\\
Planning work &5\% &{Refine your project plan by detailing tasks, \glsplf{milestone} and \glsplf{deliverable} for this stage\\ Review progress} & Appropriateness of work plan and progress\\
Managing risk &5\% &Review and adjust project risk, particularly in relation to the application of your methods & Any major adjustment required\\	
\end{SimpleNColTable}


\section{Your Writing Outcomes for this stage} 

\Cref{stage4WritingOutcomes} gives you the writing outcomes for this stage: the activities in this part of the book are designed to help you reach them.

Remember that the first column of the table gives you the expected full structure of the dissertation\footnote{If your course of study assumes a different structure, then use that instead, mapping the writing outcomes accordingly.}. Within that column, the greyed out parts are yet to be written and will be the focus of the next final stage, while those highlighted in red are to be written from scratch during this stage. The remaining parts are those you wrote in previous stages: depending on your work in this stage, you may need to revise or adjust them.

\begin{ReportTable}{stage4WritingOutcomes}[\WOCaption{4}{20}]
%%\ReportTitle1
\ReportTitle*
	& Update if necessary. At this stage your title should be close to its final form &\Cref{stage4title}\\

%%\ReportAbstract12345
\ReportAbstract-----
	&\tablenocontent
\\

%%1-\ReportIntroduction12345
\ReportIntroduction****
	& Continue to add new definitions, if necessary. Background and justification should be close to their final form  
	&
\\

%%2-\ReportLitRev---
\ReportLitRev***
	& Continue to add late-found papers  &\Cref{stage4litrev}\\

%%3-\ReportResearchDef----
\ReportResearchDef****
	& Revise if necessary &\\

%%4-\ReportResearchDes-----
\ReportResearchDes***!*
	& Finalise and justify your methodological choices and detail your procedures for generating and analysing data
	\item Revise your account of relevant ethics and regulations in view of your methodological choices
	&\Cref{ch:DataGenerationMethods,ch:ModellingMethods,ch:AnalysisMethods}\\

%%5-\ReportAnalysisInterp----
\ReportAnalysisInterp*!!-
	& Summarise the data generated in this stage, and report their analysis. Ensure your report is appropriately structured and presented to convey your work concisely, clearly and systematically 

	\item Summarise the key findings from your data analysis in this stage
	&\Cref{ch:writingUpInStage4}\\

%%6-\ReportEvalConc-------
\ReportEvalConc-------
	& \tablenocontent &\\

%%7-\ReportRefs-
\ReportRefs*
	& \tablecites{} &\\

%%%A-\ReportDissertationAppendices
\ReportDissertationAppendices*!
	& If needed, use appendices to include supplementary material in support of the main body of your dissertation, particularly data collection and analysis. For instance, you could include  sample raw data, questionnaires used in the research, programme code, detailed calculations, etc. You should customise this appendix to your own project
	&\\
	
%%\ReportProgressTracking123456789
\ReportProgressTracking**???*
& Revise the content of this appendix in view of your increased understanding and the progress you have made in this stage, paying particular attention to feasibility, work plan and risk assessment
&\\

%
%%\ReportReflection123
\ReportReflection*?
 & Update your personal statement based on actions and outcomes in this stage 
 &\\

 

\end{ReportTable}
\endinput


\section{Planning your work for this stage}\label{sect:stage4WorkPlan}

Before going further, you should refine your project work plan to include more detail on your work for~\Cref{stage4}.


%For this, you will need to:
%\begin{itemize}
%	\item make sure you have completed all the work for~\Cref{stage1} or make the necessary adjustments to your plan
%	\item identify the main tasks under each activity for this stage, allocate them time and include them in your plan. For complex tasks, you may also include some sub-tasks,~\etc{}, but you should avoid making your plan too complicated
%	\item establish main \glsplf{milestone} and \glsplf{deliverable} and include them in your plan
%	\item optimise your plan by considering dependencies and tasks which may overlap. 
%\end{itemize}


\begin{activity}[{Revising your project work plan}] 
Consider the activities in~\Cref{stage4ResearchActivities} and the writing outcomes in~\Cref{stage4WritingOutcomes}: 

\begin{itemize}
	\item for each activity, identify a number of tasks which capture the work needed, decide how much time to spend on each, and include them in your plan, also taking into consideration their possible dependencies
	\item for each \glsf{writing-outcome}, identify corresponding \glsplf{deliverable} and set related \glsplf{milestone} in your work plan.
\end{itemize}

At the end, review your overall plan, also considering the progress you made in the previous stages, and make all necessary adjustments. 
\begin{guidance}
%Hack to correct tcbox behaviour
\color{black}


%Hack to correct tcbox behaviour
\color{black}
Make sure you:
\begin{itemize}
	\item focus on a small number of key tasks for each activity, so as to keep your work plan light
	\item when allocating time to tasks, ensure that tasks fit within the overall time for their corresponding activity 	
	\item consider task dependencies and things you can progress in parallel, so to optimise your project time 
	\item break down new content you will need to write into \glsplf{deliverable}, setting appropriate \glsplf{milestone} in your plan.
\end{itemize} 
\end{guidance}
\end{activity}
%%Hack to correct tcbox behaviour
\color{black}

If after reviewing your progress you find that you are well behind, then talk to your supervisor who will be able to advise you on how to bring your project back on track and improve your planning.


%\section{Managing \glsplf{raw-datum}}\label{sect:rawdata}
%Your \textit{\glsplf{raw-datum}} are the \glsplf{datum} you generate\footnote{\enquote{\Glsplf{datum} you generate} includes \glsplf{datum} \emph{gathering} – if the \glsplf{datum} already exists, for instance, as might be the case with \glsf{evidence}. We use the more active term generate to cover all cases.} as part of your research. Which \glsplf{datum} you generate and how is determined by the choices you have made in your \glsf{research-design}, informed by your \glsf{research-aim} and objectives. In this section, we touch on key topics in \glsplf{datum} generating which are relevant to many projects, although they may not all apply to yours. This section is only meant as an introduction to those topics, to raise your awareness and point you towards specialised literature, should you need it, to deepen your understanding.
%By the end of~\Cref{stage4} your \glsplf{datum} generation, analysis and interpretation should be on a solid ground, and consistent with your aim and objectives. Your \glsf{research-design} description should also be close to its final form. Given the criticality of this stage, it is essential that you work very closely with your supervisor throughout.

\chapter{Data}\label{ch:RawData}

Your \glsplf{raw-datum} represent any \glsplf{datum} you generate and analyse as part of your research, and upon which your \glsf{evidence} and contribution to knowledge are based. All research deals with \glsplf{datum}, irrespective of the specific methodology, so that there are general practices for dealing with \glsplf{raw-datum} which are common to most research. These are reviewed in this chapter.

\section{Sampling: what, who (and how) to choose}\label{sect:sampling}

\Glsf{sampling} is the process of selecting a subset %\footnote{We could have said \enquote{\glsf{sample}} but that would have been circular.} 
for further analysis from a \glsf{population} of interest. You should use it when you wish to study a \glsf{population} which is infeasibly large or inaccessible for you to be able to study every single member of it. In such cases, you should choose a \glsf{sample} which is somewhat representative of the \glsf{population} characteristics, hoping that by studying the \glsf{sample} you can establish some properties or patterns of interest which can be assumed true of the \glsf{population} as a whole. 

\Glsf{sampling} assumes that there is a \glsf{sampling-frame} as \glsf{data-source}: a sub-set of the collection of the \glsf{population} of interest from which your \glsf{sample} is taken. In some sense, you have already experienced \glsf{sampling} as part of your literature review\footnote{You might remember the relatively complex procedures for recording search terms, discovered papers, their relationships, and your growing collection of notes on them.}. Unless you had infinite amounts of time – which you didn't – and infinite patience – which you might have – you could never be 100\% certain that your literature search collected \emph{all} relevant papers:  the search space may be infeasibly large (and not indexed particularly well). But you were systematic and achieved a practically good\footnote{By \emph{practically good}, we mean you found the most of the most important papers, some other papers, and didn't have to read \emph{every single paper}. I.e., you found a \emph{representative \glsf{sample}}.} coverage because of that.

Broadly speaking, \glsf{sampling} can either be random or non-random. 

In \glsf{random-sampling} (also called \glsf{probability-sampling}) you must establish upfront some unbiased way of choosing the members from the \glsf{population} to inform your \glsf{sample} collection, which is usually completed prior to any analysis. This is used particularly in quantitative research, and when \glsf{generalisation} of the results to the \glsf{population} is of primary importance, something enabled by the lack of \glsf{bias} in the \glsf{sample} selection process. 

\Glsf{random-sampling} techniques include:
\begin{description}
	\item \Glsf{simple-random-sampling}, where each member of the \glsf{population} has exactly the same chance of being selected. It has the advantage that it is easy to implement, and given the complete randomness of the \glsf{sample}, \glsf{generalisation} is fairly reliable. However, it can be time consuming if the \glsf{population} if very large, and may not lead to a representative \glsf{sample} if the \glsf{population} has large sub-groups, which may be over-represented in the \glsf{sample}, with minority groups being under-represented
	
	\item \Glsf{stratified-random-sampling}, where sub-groups of the \glsf{population} are identified based on common characteristics, the \textit{strata}, and \glsf{sampling} is random across those strata. The strata need not be mutually exclusive: for instance, the \glsf{population} may have sub-groups defined by gender, ethnicity and level of education, which may overlap. This approach overcomes the over/under representation problem of \glsf{simple-random-sampling}; however, deciding on the strata may be difficult and will also complicate \glsplf{datum} analysis 
	
	\item \Glsf{cluster-sampling}, where the \glsf{population} is divided up in naturally occurring separate clusters, and the \glsf{sample} is obtained by randomly selecting some clusters and then randomly selecting members of those clusters. It is more cost-efficient than the other two approaches, but can introduce \glsf{bias} if the selected clusters are not representative of the whole \glsf{population}, so that the over/under representation problem remains.
\end{description}

In \glsf{non-random-sampling}\footnote{\Glsf{non-random-sampling} is also called \glsf{non-probability-sampling}.}, you choose your \glsf{sample} based on your own judgement and discretion as a researcher, so that some element of \glsf{bias} may exist. You can also add members to the \glsf{sample} as your research progresses, interleaving \glsplf{datum} collection and analysis, until no more collection is possible or \glsf{saturation} is reached, that is collecting more \glsplf{datum} would not bring extra relevant information. This kind of \glsf{sampling} is used particularly in qualitative research, where depth and richness of results are more important than the ability to generalise.


\Glsf{non-random-sampling} techniques include:
\begin{description}
	\item \Glsf{purposive-sampling}, where participants are selected by the researcher based on particular characteristics, knowledge, or expertise they have. It is often used for small, rare or unique populations, and is particularly suited to studies which intend to be deep and narrow, and for which \glsf{generalisation} to the \glsf{population} is not the main concern. As the \glsf{sample} choice is made by the researcher, it is prone to \glsf{bias}. However, it also allows the researcher to involve participants who are likely to provide insights into such rare or unique groups

	\item \Glsf{convenience-sampling}, where participants are selected based on their availability or accessibility. This is quick and easy, but unlikely to produce a representative \glsf{sample}, so, once again, \glsf{bias} is an issue
	
	\item \Glsf{snowball-sampling}, which relies on referral from previous participants to recruit new ones. This is an effective approach when a \glsf{population} is difficult to access or when the topic is sensitive or tabu. This too is unlikely to generate a representative \glsf{sample}, and is prone to \glsf{bias}. However, it is a way to gain access to members of a \glsf{population} which may be otherwise inaccessible.
\end{description}

In summary, when choosing a \glsf{sample}, you need to consider various factors, including the aim of your study, the kind of methods you are applying, and the level of access you may have. Trade-offs are likely involved and you may not be able to obtain an ideal \glsf{sample}. Nevertheless, your \glsf{sample} will still be useful to your research, as long as you clearly explain and justify how it was obtained and what its limitations are.

\begin{activity}[{Deep reading on \glsf{sampling}}]
Go back to your choice of \glsf{research-strategy} and methods from~\Cref{stage3}. If you've chosen one which may require \glsf{sampling}, then you should go deeper into this topic to ensure you select the right kind of \glsf{sampling} for your study. You could start by reading the following: 

\fullcites{chance2024statistical}{plummer-damato2008focus}{wretman2010reflections}

\begin{guidance}
%Hack to correct tcbox behaviour
\color{black}


%Hack to correct tcbox behaviour
\color{black}
You can skip this activity if \glsf{sampling} is not required by your choice of \glsf{research-strategy}.

The suggested reading is only a starting point. You should go deeper into the specific kind of \glsf{sampling} you are most likely to apply.
\end{guidance}
\end{activity}
%%Hack to correct tcbox behaviour
\color{black}

\begin{activity}[{Your chosen \glsf{sampling} approach}]
Assuming your study requires you to perform some \glsf{sampling}, write down the \glsf{sampling} approach you are going to apply, with its justification in terms of your aim and objectives, and any trade-offs due to the practicality of accessing the \glsf{sample}. Record any possible weakness or limitation of your chosen approach, and how you will address them in your project.
\begin{guidance}
%Hack to correct tcbox behaviour
\color{black}


%Hack to correct tcbox behaviour
\color{black}
You can skip this activity if \glsf{sampling} is not required by your choice of \glsf{research-strategy}.
\end{guidance}
\end{activity}
%%Hack to correct tcbox behaviour
\color{black}



\section{Sharing and anonymising data}\label{sect:sharingAnonymising}

Modern standards of research often require that your \glsplf{datum} be made available to other researchers so that your research can be verified or even rerun. In fact, it is increasingly the case that \glsplf{datum} sets are published and shared by entire research communities, often used as testbeds or benchmarks for new knowledge contributions. For instance, in medical applications of \glsforce{machine-learning}{Machine-Learning}, in which new knowledge can be the fractional improvement of the performance of an AI algorithm to, say, diagnose a medical condition from images, not being able to share the \glsplf{datum} set you have used in your research can negate your knowledge contribution. Therefore, you should consider whether your \glsplf{datum} (or a \glsf{sample} of it) should appear as an appendix to your dissertation, or whether they should be made available in their entirety to your examiners, or even to the wider research community, and how.

Modern standards of research also require you to comply with regulations on \glsplf{datum} privacy and protection. These put many constraints on the use of \glsplf{personal-datum} as they are designed to protect the identity of the people they refer to\footnote{You encountered GDPR in~\Cref{sect:personalData}.}. 

Sometimes this can be an hindrance, particularly in reporting and sharing your research where there may be a need to also share the \glsplf{datum} used in your study -- like in your dissertation. As a result, you will need to apply techniques that have been developed to allow the use of such \glsplf{datum} without comprising their protection. These generally anonymise the \glsplf{datum}, i.e., the \glsplf{datum} are processed so that any link, direct or indirect, to a living person is removed.

You should therefore consider the need to anonymise \glsplf{datum} in your project and familiarise yourself with techniques you can apply to do so.

\begin{activity}[{Anonymising \glsplf{personal-datum}}] Do a web search to identify techniques to anonymise \glsplf{personal-datum}. List and summarise the main techniques you have found.

\begin{solution}Among others, you may have encountered the following common approaches to anonymising \glsplf{personal-datum}:

\begin{itemize}
\item \glsf{hiding}, which refers to removing \glsplf{personal-datum} from a \glsplf{datum} set, for instance, the name and address of participants in a study. Those categories are completely removed from the \glsplf{datum} set

\item \glsf{masking}, which refers to obfuscating \glsplf{personal-datum} by replacing values with certain characters, for instance replacing all names and addresses with asterisks. As a result, the specific values are not visible, but their categories are retained in the \glsplf{datum} set

\item \glsf{pseudonymisation}, which refers to replacing identifying \glsplf{datum} with made-up identification \glsplf{datum}, for instance replacing names or addresses with fake ones

\item \glsf{generalisation}, which refers to replacing certain \glsplf{datum} with more general equivalent, for instance, replacing an exact address with an area code.

\end{itemize}
\end{solution}\end{activity}
%%Hack to correct tcbox behaviour
\color{black}

Choosing a particular way of anonymising \glsplf{personal-datum} will depend on how the \glsplf{datum} will be used and the level of information that needs preserving\footnote{A supervisor who has conducted research in the area of your project may have used \glsplf{datum} anonymising techniques before. It's always worth checking with them what level of anonymisation is needed, and which techniques should be used to achieve this.}.

Among the techniques mentioned in the activity, \glsf{hiding} is the most destructive: by removing \glsplf{personal-datum} from the \glsplf{datum} set you won't be able track a participant in \glsplf{datum} collection over time. While this is good for anonymisation, it may be inappropriate for your research. In a clinical study, for instance, you might need to identify the outcomes of a series of tests on a particular individual. If you  remove all identifying information, that is not possible.

If you need to track participants through their \glsplf{datum} over time, \glsf{pseudonymisation} -- in which, say,  \enquote{Cruella Deville} is replaced by the made up identifier \enquote{Mario Rossi} is better, as it allows you to track a named individual without giving any \glsplf{personal-datum} away.

\Glsf{generalisation}, on the other hand, can be used when more specific \glsplf{datum} can be replaced by more general one without loss of accuracy. For instance, if specific addresses are not needed in your study, then replacing them with area codes would not compromise your findings.

It is therefore important that you consider carefully how \glsplf{personal-datum} are to be used in your project when choosing an anonymisation technique. 

Commercially sensitive information also needs protecting from disclosure, so that anonymisation may also be needed in this case and some of the techniques you have reviewed still apply.

\begin{activity}[{Your anonymisation approach}]
Assuming your study requires you to anonymise \glsplf{personal-datum}, write down which technique(s) you may apply, with justification in relation to the aim and objectives of your research. 
\begin{guidance}
%Hack to correct tcbox behaviour
\color{black}


%Hack to correct tcbox behaviour
\color{black}
You can skip this activity if you don't need to anonymise \glsplf{personal-datum} in your research.
\end{guidance}
\end{activity}
%%Hack to correct tcbox behaviour
\color{black}


\section{Managing raw data}\label{sect:managingRawData}

%If, for instance, you conduct \glsf{experimental-research}, you will have \glsplf{datum} about your subjects, perhaps collected through \glsplf{questionnaire}, and variables of interest, with their measurements. If you use interviews or focus groups, you may have audio or video recordings and their transcripts, or notes you have taken. For \glsf{survey-research}, you will have a large number of responses to the set questions. If you use existing \glsf{secondary-evidence}, for instance as part of \glsf{case-study-research}, then you will have all sort of \glsplf{document}, from reports to images, diagrams,~\etc{}: from such \glsplf{document}, you will need to extract relevant \glsplf{raw-datum} for your analysis.

Before proceeding with your \glsplf{datum} analysis, you must ensure your \glsplf{raw-datum} are properly organised and stored, so that you don't loose track of important information, and you can easily locate and refer back to appropriate \glsplf{datum} during your analysis and while writing up your research.

It is highly likely that your \glsplf{raw-datum} will be in some digital form, in which case you need to worry about possible \glsplf{datum} loss, which may occur due to technical issues, such as computer failure, or due to a \glsplf{datum} breach, such as through cyber security attacks. Therefore, you must ensure that your digital \glsplf{datum} storage is secure\footnote{How to do this is beyond the scope of this book. If you don't feel sufficiently IT savvy, you should seek advice from your university's IT support team.} at least to the standards required by law, any additional requirements made by your university, those of any participants, their organisations, and any other stakeholders\footnote{It used to be that \glsplf{datum} generation and storage was a \emph{laissez-faire} thing. Today, your university or employer can be fined for \glsplf{datum} misuse. Do take \glsplf{datum} use seriously.}.

%is regularly backed up, not to risk loosing it, and, if you collect any personal information, your \glsplf{datum}, whether physical or digital, must be stored securely to comply with \glsplf{datum} protection\footnote{See Cref{sect:personalData}.} regulations.
%
It is also important that you put your \glsplf{raw-datum} in a form which is useable for analysis. Spreadsheets are particularly useful for this purpose, especially if your \glsplf{datum} are quantitative, so that this is a common way to organise and store \glsplf{raw-datum}. In fact, most publicly available \glsplf{datum} sets used in research and beyond are stored as spreadsheet files: if you are going to use one such \glsplf{datum} set, then your \glsplf{raw-datum} are likely already organised for you!  

Spreadsheets organise \glsplf{datum} in rows and columns, so that you can easily enter your \glsplf{raw-datum} using rows for your \glsf{observations-measurements} and columns for your variables. As you will see later on\footnote{See~\Cref{ssect:UsingTables}.}, spreadsheets come a wide range of functionalities for \glsplf{datum} manipulation and for some level of \glsplf{datum} analysis. They are also easily extensible, so that you can grow your \glsplf{datum} sets incrementally.

\begin{activity}[{Managing your \glsplf{raw-datum}}]
Consider the \glsplf{datum} you will need for your project. Write down the actions you will take in relation to:
\begin{itemize}
	\item organising your \glsplf{raw-datum}
	\item storing and backing up your \glsplf{datum}
	\item protecting \glsplf{personal-datum}.
\end{itemize}
Make sure you complete those actions as you generate your \glsplf{datum}, and before performing any \glsplf{datum} analysis.
\end{activity}
%%Hack to correct tcbox behaviour
\color{black}

%\endinput

\chapter{Data generation methods}\label{ch:DataGenerationMethods}

All research needs \glsplf{datum} and this chapter steps through the most used \glsplf{data-generation-method}. Most of them concern {\glsplf{empirical-datum}}, that is \glsplf{datum} that is obtained through our five senses or from experience, yours or that of your study participants. 

 The term \glsplf{datum} generation doesn't necessarily mean that you will create new-to-the-world \glsplf{datum}\footnote{Although this might indeed be the outcome of your \glsplf{datum} generation.}. It simply means new-to-your-project \glsplf{datum}, which covers a multitude of sins, including:

\begin{itemize}
\item the creation of a brand new \glsplf{datum} set, which did not exist before your research project. This may be the result of \glsplf{datum} collected through new observations or measurements, a new survey, questionnaire or \glsf{focus-group}, through selecting passages from \glsplf{document},~\etc{}.
\item the extension of a previously collected \glsplf{datum} set with new elements, derived from those that already exist, for instance adding the \glsf{mean} value of a collection of \glsplf{numerical-datum}, or grouping together specific distinct \glsplf{datum} into new categories that you have created
\item the collection of previous \glsplf{datum} for re-interpretation, for instance if you are re-running a previous experiment in order to confirm its results, or doing a \glsf{meta-analysis} of the literature in a particular area.
\end{itemize}

Of course, the \glsplf{datum} you generate as part of your research must allow you to make a contribution to knowledge, that is to conclude something new, even if you use an existing \glsplf{datum} set.

Related to \glsplf{datum} generation is the concept of \glsf{data-source}, which is the location from which your \glsplf{datum} originates. If you are re-using existing \glsplf{datum} sets, this may well be an archive or a digital repository. For new \glsplf{datum} sets you generate, this may be the experimental or real-world setting of your own observations and measurements, or a \glsf{population} of interest from which you will derive a \glsf{sample} for further analysis. 

For each \glsf{data-generation-method} in this chapter\footnote{And also for each modelling and \glsf{analysis-method} in the next two chapters.}, you will find:
%
\begin{itemize}
\item a brief description of the method
\item key procedural considerations you should take into account
\item other important issues, particularly in relation to potential research weaknesses or feasibility within your project
\item further sources to consult for more detail.
\end{itemize}

You concluded~\Cref{stage3} with an initial draft of your methodology which included both your chosen \glsf{research-strategy} and possible research methods within. Now it's the time to consider those methods in detail to ensure they are the right ones for your project, to plan how to apply them, and to start generating your \glsplf{datum}. 

\subsection{Choosing and applying your data generation methods}\label{sect:choosing-data-generation}

The next two activities will help you do that for your chosen \glsplf{datum} generation methods. These are very substantial activities which will allow you to take a significant step in your research.

\begin{activity}[{Finalising your \glsplf{datum} generation methods}]
 Consider your chosen \glsplf{datum} generation methods. For each, read through the related section in this chapter (see~\Cref{tab:generationMethodsChoice}), and write a summary including the following points:
 \begin{itemize}
 	\item explain why it is an appropriate method for your project, in relation to your aim and objectives, your chosen \glsf{research-strategy} and the \glsplf{datum} sources at your disposal
 	\item if there are variants of the method, indicate the one you will apply and why 
 	\item if \glsf{sampling} is needed, indicate the kind of \glsf{sampling} you will perform and why
 	\item if \glsplf{datum} anonymisation is needed, indicate which approach you will take
 	\item detail your procedures in applying the method within your project
 	\item indicate how you will guard against potential research weakness intrinsic to the method.  
 \end{itemize}
 Discuss your choices with your supervisor to ensure they are appropriate for your project. 
\begin{guidance}
%Hack to correct tcbox behaviour
\color{black}


%Hack to correct tcbox behaviour
\color{black}
This activity assumes that you have completed a draft of your methodology which includes your chosen \glsf{research-strategy} and possible research methods within. If that's not the case, then you should return to~\Cref{stage3} and complete the activity in~\Cref{ch:yourResearchMethodology}.

For each method, conduct a first pass at addressing each of the points above based on the content of this chapter, then follow some of the references provided to read more about the method and improve your summary. Your supervisor should also be able to point to relevant literature or provide expert advice on the application of the method.

A procedure indicates the specific steps you will take to ensure the method is applied correctly and that you have guarded against its potential weaknesses. In your dissertation, your procedures should be sufficiently detailed for other researchers to follow what you have done or even replicate your work independently. 

You may have to iterate with your supervisor to refine your choices and procedures.
\end{guidance}
\end{activity}
%%Hack to correct tcbox behaviour
\color{black}

%%%Intentionally \ref, not~\Cref
%\newcommand{\tick}{$\Box$}
\begin{SimpleNColTable}{tab:generationMethodsChoice}{2}[\narrowtablewidth]{\Glsplf{datum} generation methods}[rX[c, wd=20ex]]
	Method & Section\\
	\Glsplf{observation}&\ref{sect:observations}\\
	\Glsplf{questionnaire}&\ref{sect:questionnaire}\\
	\Glsplf{interview}&\ref{sect:interviews}\\
	\Glsplf{focus-group}&\ref{sect:focusGroups}\\
	\Glsf{delphi}&\ref{sect:delphi}\\
	\Glsf{journaling}&\ref{sect:journaling}\\
	\Glsf{fieldwork} &\ref{sect:fieldwork}\\
	\Glsplf{document}&\ref{sect:documents}\\
\end{SimpleNColTable}

After completing the previous activity, you should be ready to start generating your \glsplf{datum}. 

\begin{activity}[{Applying your \glsplf{datum} generation methods}]
 Apply your \glsplf{datum} generation methods by following the procedures you have outlined and agreed with your supervisor. You should:  
  \begin{itemize}
 	\item test your procedures before their wider application, if possible
 	\item document any deviation or adjustment you may have to make in executing your methods
 	\item ensure you manage your collected \glsplf{datum} appropriately
 	\item establish regular checkpoints with your supervisor to review your \glsplf{datum} generation process and outcomes, and to agree any required adjustment. 
 \end{itemize}
\begin{guidance}
%Hack to correct tcbox behaviour
\color{black}


%Hack to correct tcbox behaviour
\color{black}
The kind of testing you can do will depend on the method you will apply. For instance, you may test a questionnaire with a family member to ensure its design is appropriate, before using it with your research participants.

Regular monitoring and adjustments are essential: it is highly unlikely that things will go exactly as you originally planned them!
\end{guidance}
\end{activity}
%%Hack to correct tcbox behaviour
\color{black}


\section{Observations}\label{sect:observations}

 Observations constitute one of the main ways in which you can generate \glsplf{empirical-datum}. In fact, according to \cite{marvasti2014analysing}, \enquote{\Glsf{observation} is the foundation of science}\footnote{Whole books have been written on observations as a \glsf{research-method}. This chapter can only give a shallow introduction. Further sources you can use are included at the end of this section.}. 
 
  This method requires the researcher to make their own observations\footnote{Hence the name...} of \glsplf{phenomenon} of interest. What you will observe are core characteristics of the \glsplf{phenomenon} that you have identified as part of defining your \glsf{research-problem}. 
  
  %\ResGenTechnique{observations}
 
 Observations can be made directly, through your naked senses, or through instruments which enhance your sensory capabilities, such as a telescope, a microscope or the \enquote{myriad of other ingenious inventions designed to make the invisible visible, the evanescent permanent, and the abstract concrete} \parencite{daston2011introduction}. Quantitative observations, say, the size or weight of an object, are usually referred to as measurements.
 
 
   
 %\blockcquote{daston2011introduction}{\Glsf{observation}[s...] instruments include not only the naked senses, but also tools such as the telescope and the microscope, the questionnaire, the photographic plate, the glassed-in beehive, the Geiger counter, and a myriad of other ingenious inventions designed to make the invisible visible, the evanescent permanent, and the abstract concrete. Where is society? How blue is the sky? Which ways do X-rays scatter? Over the course of centuries, scientific observers have devised ways to answer these and many other riddles.}

 %\todo{Ensure that\footcite{driscoll2011introduction} \emph{Voluntary participation}, \emph{Confidentiality and anonymity}, and \emph{Researcher \glsf{bias}} are covered elsewhere, if not here.}

%\paragraph{What to observe} What you will observe are core characteristics of \glsplf{phenomenon} of interest that you have identified as part of your \glsf{research-problem}, whether they are interactions between atoms, between school children, or even your own actions, thoughts, and – perhaps even – \glsplf{bias}. It may be that, in the course of your research, you will observe related or proxy \glsplf{phenomenon}, as needs must. Part of your \glsf{observation} skills will be to identify those related to and proxies for, to record those observations and to iterate those records into structures that capture the characteristics that are necessary to do the research on their back. 

\paragraph{What to observe} Observations are versatile tools for almost any research domain, and your own domain will determine what sort of observations you will make. Observations range across natural \glsplf{phenomenon} – such as the different proportions of plant species that populate a wilderness garden – through to artificial \glsplf{phenomenon} – the way that buses drop off and pick up their passengers at a train station – and to social \glsplf{phenomenon} – the different ways in which a train station is used by commuters in the morning and the evening – as well as more complex combinations of each. Each will use different observations techniques and tools, and each with different constraints. 

\paragraph{\Glsf{observation} types} Observations can be \emph{naturalistic}, when \glsplf{phenomenon} are observed as they happen in their natural setting -- for instance, observing the behaviour of animal species in their habitat, or \emph{structured}, when \glsplf{phenomenon} are observed in a somewhat artificial environment, such as during an experiment  -- for instance, giving people a specific task to perform and observing how they carry it out. In this case, often the aim is to collect quantitative \glsplf{datum}, say the speed at which the people can complete the task. 


\paragraph{More on social observations} As an observer of people, you can act as either a \emph{participant} or a \emph{non-participant} observer. The former is a researcher that interacts as a member of a community under \glsf{observation}, becoming an active participant in the group or situation under study. In effect, as a participant observer, you would be \enquote{living} alongside those you observe – you might be a commuter that uses the train station and so share the experience of the other commuters you observe. Instead, in non-participant \glsf{observation}, you would remain separate from the group or situation being observed -- you may observe commuters using the train station, but not actively become one of them. 

Depending on whether people are or not aware of being observed, observations can be \textit{covert} or \textit{overt}\footnote{The terms \textit{disguised} and \textit{undisguised} are also used in the literature.}. Covert observations have the advantage that people's behaviour is not affected by their awareness of being observed, but, of course, they raise some important ethical and legal issues in relation to informed consent, and privacy and anonymity. If the judgement is that the \glsplf{phenomenon} observed do require privacy – perhaps you wish to observe commuters' use of restrooms in the train station or customers in a betting shop – then you must ask explicitly  for permission – or change your \glsf{research-problem}! Otherwise, in a public space, there may be no overriding expectation of privacy and observations can be done without explicit consent. Your university is likely to have strict regulations on the matter, or even prevent you from conducting covert observations as part of your research. 
%Alternatively, you may be an \emph{unobtrusive observer}, a researcher that works outside of the \enquote{frame of reference}\footnote{The frame of reference is the relation between the observer and what is under \glsf{observation}.} for the observations you will make, perhaps standing at the entrance/exit to the train station counting how many commuters pass.

\subsection{Procedural considerations}\label{ssect:DG:observationsProceduralConsiderations}

In order to apply this method, some preparation is needed for you to decide what you will observe as how. Specifically:

\paragraph{\Glsplf{phenomenon}}  You will need to decide which \glsplf{phenomenon} to observe, whether natural, artificial, social, or their combination: this choice will depend on your \glsf{research-problem}, and aim and objectives. 

\paragraph{Kind of observations} Depending on the \glsplf{phenomenon}, you will need to decide whether you will perform naturalistic or structured observations. In addition, for social \glsplf{phenomenon}, you will need to decide which \glsf{mode}, whether participant, non participant, overt or covert. For the latter, you must also identify the steps you will take you ensure compliance with ethical and legal guidelines.

\paragraph{Time and place} For all kinds of \glsf{observation}, you must determine the time and place at which those observations will be made. For participant observations, this choice will be determined by your participation in the group or activity being observed, which may or may not be under your control. For instance, if you are a participant observer of train station usage, then you can determine when to use the station and make your observations. However, if you are a participant observer in a change project within your organisation, then the timeline of the project will determine when your observations can take place. In all cases, you should draw a schedule which establishes the timing and frequency of your observations, and which provides an efficient way for you to conduct your observations. For instance, it may be that you make exaggerated use of the train station\footnote{You could eat there, for instance, or use any shops that are colocated.} to condense many months of participant \glsf{observation} into weeks or days – after all, your research project is time bounded – perhaps visiting ten times per day rather than just two. 

\paragraph{Use of instruments} You may be able to make your observations using only your senses. However, many \glsplf{phenomenon} do not permit \glsf{observation} through the senses unassisted – for instance, the search for exoplanets\footnote{For instance, \textcite{jones2008exoplanets}. But don't let this exciting mission to \enquote{explore strange new worlds; to seek out new life and new civilizations; to boldly go where no one has gone before} distract you. Too much.} requires complex and delicate instruments which will be located on mountain tops. In this case, the availability of the equipment you will use will determine when and how you make your assisted observations  --- you will also need to gain access to the equipment, and this will constrain your schedule and may alter your research plans\footnote{Remember, plans never survive first contact with reality, so also plan to have a backup plan that you can use should your first plan to make your observations fail!}.

\paragraph{How to record observations} It is important your record what you directly observe, separate by any added interpretation\footnote{That should happen later on, as part of your \glsplf{datum} analysis.} to avoid possible \glsf{bias} affecting recorded observations. To this end, observations are typically recorded in notebooks with a double entry, which separates pure observations from possible value judgements made by the observer – for instance, observing someone \enquote{waiting impatiently for the train door to open} is ascribing feelings to the person observed which, by their nature, are hidden from the observer, but may be inferred from the observed's body language. By separating them out, another researcher reading your notes can clearly differentiate direct observations from such inferences. In some cases, you may be able to use audio and video recording, say using your smart phone, to capture your observations for follow-up analysis. In this case, ethical issues in relation to privacy and informed consent also apply.

\subsection{Other things to think about}\label{ssect:DG:observationsOtherThingsTo}

Observations are by no means an easy way of generating research \glsplf{datum}, and there are many issues that can arise.
%
\paragraph{\Glsf{hawthorne-effect} (\glsf{observation-bias})} Overt \glsf{observation} can lead to the so-called \Glsf{hawthorne-effect}\footnote{Which takes its name from a 1950's productivity study carried out at the \enquote{Hawthorne Works}, an electric plant in Chicago.}, in which those observed change their behaviour due to being observed. 

The \Glsf{hawthorne-effect} can be mitigated by:
\begin{itemize}
	\item building rapport with those observed, for instance by spending time with them
	\item observing them for longer periods of time
	\item in case of structured observations, by ensuring that the tasks participants are asked to do come naturally to them.
\end{itemize}
	
\paragraph{\Glsf{observer-bias}} All observations can be influenced by the observer's own \glsf{bias}, whether implicit or explicit, or the result of overfamiliarity with the \glsplf{phenomenon} of interest. To guard against it, \glsf{triangulation} should apply, including using different \glsplf{datum} sources and collection methods, or having multiple observers all following a standardised procedure. The use of double-entry notebooks, as described above, can also help, as they separate pure observations from interpretation and inferences made by the observer, with the latter being the subject of scrutiny for potential \glsf{bias}.

\paragraph{Volume of observations} \Glsplf{observation} can lead to vast amounts of \glsplf{datum} to analyse. Different analysts, or different stages of analysis as your research progresses, may focus on different aspects of the same \glsplf{datum}. This can be a good thing should your analysis deepen due to understanding more about your observations, but may also lead to analysis drift or even \enquote{paralysis through analysis} in which no progress is made due to too much depth. To avoid this, keep a clear eye on the prize: your research goal, and set regular times at which you can reflect on progress.

%\item the analysis of observations depends on the researcher's chosen focus and philosophical and analytical framework. This is a natural dependency and can give great richness to observations, even those that are taken from the record. However, they can also lead to overthinking and, like the previous item, paralysis, as well as to prose that is obtuse and disengaging for the reader\footnote{And examiner! Consider, also, that the examiner might not share your focus and that this difference might be sufficient for them to discount your work as without value. Not likely, but care is needed.}.


\subsection{Further reading}\label{ssect:DG:observationsFurtherReading}

\ReadingList{observation}{daston2011introduction,marvasti2014analysing,simpson2003using,driscoll2011introduction,angrosino2003observations,sapsford1996data}

\section{Questionnaires}\label{sect:questionnaire}

\Glsplf{questionnaire}\footnote{\Glsplf{questionnaire} are just one in a rich collection of \emph{survey tools}, others of which are described below.} are versatile tools for generating \glsplf{datum} from participants by asking questions\footnote{There's a hint in the name – \emph{question}naire – although why two \enquote{n}s; does no millionaire, billionaire, or debonaire use them?}. They allow a researcher to collect participants' answers about their attitudes, preferences, opinions, behaviours,~\etc{}. You might use a questionnaire as a way of collecting statistically significant responses from a \glsf{population} \glsf{sample}, but there are other uses as well, for instance as the basis of interviews\footnote{Which you will encounter in the next section.}.

%\ResGenTechnique{\glsplf{questionnaire}}

If you do use a questionnaire, its thoughtful design\footnote{\emph{Questionnaire design} may conjure up glossy format and whizzy web-pages, which is of secondary importance. Unless your questionnaire is about the design of \glsplf{questionnaire}, of course.} is of critical importance. Otherwise, you might be asking your (willing) respondents to spend a considerable amount of their valuable time answering questions the content of which is not helpful for your research. As they might not be so willing to help a second time, getting the questions right the first time is important. 

Administering \glsplf{questionnaire} is nowhere near as difficult as it used to be as the number of online resources for doing so continues to increase. Because of this, there are plenty of resources to help you design your questions. Their descriptions can be a little technical, so the following definitions might help you engage with them better.

\paragraph{Essential questions} %You should identify 
This the smallest possible set of questions you absolutely need to ask to address your \glsf{research-aim} and objectives. While using several questions will give you richer \glsplf{datum} sets, long \glsplf{questionnaire} tend to put people off, so that fewer people may be willing to participate or complete the full questionnaire. 

\paragraph{Profiling questions} %These are 
These questions ensure that your respondents match specific characteristics you are interested in: say, you are studying the usability of a new product, then you will need to know the extent your respondents have engaged with that product. This is particularly the case if you are running a large survey and don't know who is going to respond.  

\paragraph{Demographic questions} %These are 
These are often used so that you can then compare answers across different sub-groups, say, based on gender, age, ethnicity,~\etc{}.

\paragraph{Response options} Questions are broadly divided into \emph{closed} and \emph{open}-ended. Close questions restrict the possible responses to a set of given choices, while open questions allow respondents to use their own words freely to answer each question. 

\subsection{Procedural considerations}\label{ssect:questionnaireProceduralConsiderations} 

In applying \glsplf{questionnaire} you should consider:

\paragraph{Using tools} While you can design your \glsplf{questionnaire} from scratch using your word processor, there are plenty of specialised digital tools, many of which are free, that can make it a lot easier\footnote{Such as Google Forms or SurveyMonkey, but many more are available at the time or writing.}. They usually come with: templates and pre-defined question types that you can customise for your study; \glsf{statistical-analysis} and \glsplf{datum} visualisation features that you can apply to the \glsplf{datum} you have collected; and export functions that allow you to save the \glsplf{datum} to a spreadsheet for further analysis. Overall, if you need to develop \glsplf{questionnaire} for your research, they can really help you speed up the process, so that it's well-worth the investment of time in climbing their learning curve. 

\paragraph{Drafting, testing and piloting} Unless you have many years of experience in questionnaire design, your first questionnaire draft will be far from suitable. Indeed, releasing your first draft without further thought may lead to you not generating useful \glsplf{datum} from it, and also putting off your audience sufficiently that they may not be willing even to look at your second version. So, once you have a first draft of your questionnaire, you should test it and refine it. 

Early testing can be done by asking a family member, friend, or colleague to work through the questions, provide their answers and any other feedback they might have. This will give you early indications of problems with your questionnaire\footnote{Although it's sometimes difficult, you'll make more progress and more quickly if you think of the questionnaire as imperfect, rather than you. You can then apply comments – even if they are negative – to the questionnaire rather than having a personal emotional reaction to them. For each comment, make sure you understand how it can be addressed in your questionnaire. This last tip also means that you can welcome (but ignore) comments that can't be addressed.}, which you can spot, for instance, if the respondents are confused – which may point to a lack of clarity in the questions – or hesitant – which may point to a poor choice of response options or to inappropriate scales – or disengage before completion – which may point to too many questions being asked. Be sure to loop back to those who have helped you to check that you have addressed their comments. 

Later in the process of designing your questionnaire, however, you should also take expert advice including, of course, that of your supervisor, to get to the final agreed form. In addition, you could pilot your questionnaire on a small number of respondents first, then revise it as necessary before using it more widely.  

\subsection{Other things to think about}\label{ssect:questionnaireOtherThingsTo}

In designing your questionnaire you should pay particular attention to the following practices.

\paragraph{Plain language} Your questions should be clear and plain, and you should avoid jargon and idioms, to ensure your participants understand what you are asking, particularly if not native speakers. 

\paragraph{Unbiased language} Your questions should be objectives, that is you should avoid any judgemental term or tone which may reveal your own opinions or believes, or influence participants to answer in a particular way. You should also avoid questions which make assumptions about your respondents' habits or behaviours: for instance, asking participants what they eat for breakfast, assumes they all take breakfast, which may not be the case.

\paragraph{Double-barrelled questions} Also termed \enquote{compound}, these are questions that ask more than one thing, while only allowing one answer. These should be avoided as it would be difficult, if not impossible, to establish in your analysis which part of the question each participant has answered. Instead, you should split the question into separate questions each addressing a specific thing.

\paragraph{Closed and open questions} For your closed questions you should ensure that the answers provided cover all possible options\footnote{Or, at least those you're interested in.} and do not overlap, that is they are mutually exclusive. Instead, for your open questions you must ensure they are sufficiently constrained so that your participants' answers don't end up being too vague or off topic, hence not providing much value to your research.

\paragraph{Scales} If your questions require participants to estimate or measure something, you need to worry about the scales they should use for their answers. Those scales should allow respondents to: measure something accurately, that is to provide valid measurements; and, under the same conditions, to provide consistent, hence reliable, measurements.

\paragraph{Question grouping, ordering and flow} You should group related questions\footnote{For instance, those intended to establish a demographic of respondents.} together, and establish a logical flow in sequencing groups of questions, so that topics follow naturally from one another. Question order in each group also matters: as a rule of thumb, simpler questions should precede more complex ones.

\subsection{Further reading}\label{ssect:questionnaireFurtherReading}

\ReadingList{questionnaire}{oates2008researching,hays2003case,burns2009action,mcclure2002common,najafi2016observation,robertson2002automated,kielmann2012introduction}

\section{Interviews}\label{sect:interviews}

Interviews are a method for generating \glsplf{datum} from participants by asking questions and recording detailed answers. They are a form of conversation between the researcher and one or more interviewees, designed by the researcher to gain insights and opinions on a specific topic. The researcher guides and controls the conversation and asks the questions.

%\ResGenTechnique{interviews}

The personal approach that is a characteristics of interviews means that they are a great way of accessing a (group of) individuals' feelings, thoughts, ideas, and/or experiences, \glsplf{datum} that can be difficult to generate in other ways. Interviews can help you obtain detailed information on a specific issue or topic, asking open-ended questions which may be tackled or interpreted differently by different interviewees. They are also an effective way to investigate sensitive issues or privileged information that interviewees may not be willing to commit to writing.

Interviews can also provide direction for new research by giving expert indications of where problems lie in a particular domain. As such, interviews can often be used as a way into an emerging topic or field of study, filling in useful background through personal experiences, and providing access to otherwise difficult to access information. 

There are three main kinds of research interviews:

\paragraph{The \glsf{structured-interview}} serves as a repeatable protocol by which each participant is asked the same questions in the same way. There is no scope for deviation from the structure, so that auxiliary questions and follow-ups are not used.

Structured interviews are the closest to being time bounded and predictable; if you only have 8 hours to conduct 24 interviews for instance, a \glsf{structured-interview} would be the best way to achieve this. 

Your skills as an interviewer will be tested by structured interviews: it is often difficult not to stray outside of the structure when an interesting answer is given, and you may have to cut a participant short if their answers overrun or diverge from the structure\footnote{In our experience, this is a perennial problem, so don't underestimate the difficulties you will face as an interviewer.}.

\paragraph{The \glsf{semi-structured-interview}} serves to identify areas of interest to the researcher, with interesting responses being welcomed and followed up if appropriate. Interviewing a domain expert on your chosen topic would be well-served by a \glsf{semi-structured-interview} as their expert knowledge could be probed with follow-up questions.

The \glsf{semi-structured-interview} does not naturally time-bound the interaction, and so – if you don't have unbounded amounts of time – you will have to balance the breadth of questions with the depth of responses.


\paragraph{The \glsf{unstructured-interview}} has no structure to constrain the route through the \glsplf{datum} that the interviewee wishes to take, although the \glsf{interview} may begin with the same question each time. As the direction may wholly be decided by the participant, you challenge may be retaining forward motion and focus during the \glsf{interview}.

\subsection{Procedural considerations}\label{ssect:InterviewsProceduralConsiderations}

Procedural considerations for interviews include:

\paragraph{\Glsf{interview} type} In being standardised, structured interviews make it easier to compare your interviewees' answers objectively. However, interviewees are limited to the set questions, so there is no scope for digging deeper into their answers. If you need deep insights, you should use the less \glsf{structured-interview} forms which allow you to probe your participants responses.

Also, if you have a good level of domain knowledge, you can use the less \glsf{structured-interview} models, as you will be able to follow your participants' answers more easily and direct their comments towards your research interests\footnote{Of course, you can always use structured interviews even if you do have domain knowledge – there's nothing to stop you.}. If you don't already have a good level of domain knowledge, then you will be putting more effort into the design of the \glsf{interview}, so that you can simply capture your participants' responses to your – well-designed – questions. 
%
%\endinput
\paragraph{Choice of interviewees} You will need to choose your interviewees carefully. In the perfect case, you should continue to \glsf{interview} until the responses you receive from new interviewees are \enquote{guessable}\footnote{I.e., you no longer get novel answers to your questions, indicating that the topic has been covered.}. Practically, you will have limited time and resources, and limited access to interviewees, so that \glsf{sampling} may be required: in such case, you should follow the advice on \glsf{sampling} in~\Cref{sect:sampling}. 
%
\paragraph{Ethical and legal matters}  Your university will have strict guidelines on how to approach and work with human research participants, which you should investigate before you contact your potential interviewees. At a minimum, those guidelines will cover informed consent, handling \glsplf{personal-datum}, and health and safely, but they may also prevent you from interviewing certain groups of people, for instance minors or vulnerable adults. You should go back to the advice in~\Cref{ch:EthicsAndRegulations} to refresh your understanding of ethical and legal issues in relation to human research participants.

\paragraph{Testing and reviewing} You can apply some of the advice in~\Cref{sect:questionnaire} to help you design your \glsf{interview} questions. Once you've drafted your \glsf{interview} questions, you should do a dummy run of your \glsf{interview} with a willing friend, family member or colleague, and use their feedback to improve your questions\footnote{Repeat this with as many willing participants as you can until you're happy with the \glsf{interview} format or until you run out of willing participants, or time! Make sure that you do not dip into your target \glsf{population} for these preliminaries.}. In particular, you should consider:
%
\begin{itemize}
\item whether you were able to put the participants at ease during the \glsf{interview}. If not their nervousness might influence their ability to contribute, and you should consider what to do differently
\item which questions worked well and led to useful responses, and which were confusing or led to unhelpful answers. For the latter, you should consider how to reword them – perhaps with the help of your participant\footnote{\enquote{I would have asked it this way}...}, by having alternative versions of the questions, or by removing or replacing the questions
\item (if time constrained) which questions overran, and whether you can rephrase them to be less \enquote{open}
\item (if structured) whether were you able to keep to your \enquote{script}. You should reflect on how will you resist the temptation to probe more deeply, or whether you should consider moving to a semi-structured or unstructured form
\item whether the questions were in a logical order, ideally grouped by topic. If not, consider re-arranging them to build responses in the most productive way
\item whether you have sufficient questions to elicit the \glsplf{datum} you need, or there are other questions you should ask.
\end{itemize}
Once you are satisfied with your questions, you should run them past your supervisor – they will have comments for sure.

\paragraph{Recording answers} You need to decide how you will capture your interviewees's answers. If you are planning to use audio or video recordings, you will need to make your participants aware and elicit their explicit consent. If not, you will need to take notes manually. In this case, you should also test your note taking, to check that you are able to capture everything of interest\footnote{Longhand notes can be taken at 35 words per minute; spoken text is often as fast as 120 words per minute.}.

\paragraph{Where to hold the \glsf{interview}} With the advances in video conferencing technology, interviews can be conducted effectively online, with the added bonus that they can be easily recorded, often with transcripts automatically generated. However, interviews in a physical space where you are colocated with your interviewee, remain common. For these, you will need to ensure that you have an appropriately comfortable venue for your \glsf{interview}, including access to comfort facilities. Public spaces – where you could share a coffee, for instance – may create a more immediate feeling of intimacy, and so deeper responses, but they may not be suitable for discussing sensitive issues or if background noise might interfere with your record keeping. Therefore, make sure to check the venue out at the right time of day to ensure it is appropriate for the \glsf{interview} and your recording device can handle any difficulties.
%
\paragraph{Opening and closing interviews} As part of giving their inform consent, interviewees should be fully aware of what you are trying to achieve in your research and what the purpose of the \glsf{interview} is within it. They will also be interested in how you will use their answers, and should be reassured as to any use of confidential information or \glsplf{personal-datum}. It is therefore good practice to provide this information at the start of your \glsf{interview}, or even as part of inviting them to participate: perhaps share a sheet which includes this information and describes the research you are doing. At the end of the \glsf{interview} you should also thank them and explain what will happen next, including how they can get in touch if they have any further concerns and follow-up questions.

\subsection{Other things to think about}\label{ssect:InterviewsOtherThingsTo}

To be successful at interviews you should also consider:

\paragraph{Making a checklist} There is lots to think about when preparing and conducting interviews. Have you made all necessary arrangements to conduct the interviews? Did you obtain all necessary permissions, including informed consent? Do you know how you will record the answers? What would happen if your audio recording device went wrong? Would you have a backup for the \glsf{interview}? What if you forgot to turn it on?\footnote{Oh so easy to do...} Write a checklist of instructions for yourself to follow before and after each \glsf{interview}, so that you can be sure not to miss anything important. 

\paragraph{Being a good interviewer} Try, to the extent possible, given the format, to allow your participants to govern the speed and direction of the \glsf{interview}. Allow them to talk in complete sentences without interruption, or have a good reason to interrupt. If you need to interrupt, apologise for doing so and tell them the reason why you have done so\footnote{\enquote{I'm sorry to have to interrupt, but we only have 5 minutes left and ...}.}. Be polite and encouraging, as your participants might be nervous.


\subsection{Further reading}\label{ssect:InterviewsFurtherReading}

\ReadingList{interview}{oates2008researching,johannesson2014research,secor2010social,hays2003case,mcclure2002common,peoples2020write,jorgensen2001grounded,hycner1985some,englander2012interview,ramsook2018methodological,robertson2002automated,kielmann2012introduction}

%\endinput

\section{Focus groups}\label{sect:focusGroups}

Focus groups engage participants in interactive discussions to develop an understanding of complex \glsplf{phenomenon} and generate new \glsplf{hypothesis} for further research or practice. They are effective at surfacing a full range of perspectives held by the participants and, through their interaction, expand on their individual contributions. They are particularly useful to uncover \glsplf{datum} and ideas that may not come up in one-on-one interviews, and are generally a more efficient way of collecting \glsplf{datum} than multiple interviews.

%\ResGenTechnique{focus groups}

Focus groups include participants who share some common characteristics or interests. They are moderated, often by the researcher, so that they combine elements of interviews and observations alongside the group discussion. The moderator plays a crucial role in facilitating group processes, maintaining focus, and controlling participant interactions. Depending on your \glsf{research-aim}, it might be necessary for you to run a series of focus groups so that you can identify trends across different groups.

There are different flavours of \glsf{focus-group}, including:
\begin{itemize}
\item having two moderators with separate roles, say one looking after the procedures, the other focusing on the discussion, or both contributing to the discussion, but taking opposite sides or with one playing devil's advocate\footnote{The term \enquote{duelling moderators} is used in this case.}
\item a \glsf{two-way-focus-group}, in which there are actually two moderated groups each listening to each other's discussion, with a view to stimulate richer insights through rebuttal or further elaboration of ideas
\item a \enquote{mini} focus, with a reduced number of participants to create a more intimate atmosphere for the discussion.
\end{itemize}

\subsection{Procedural considerations}\label{ssect:FocusGroupsProceduralConsiderations}

To optimise your focus groups, you should pay careful attention to the following issues.

\paragraph{Group type} Depending on your \glsf{research-aim}, you should decide which kind of focus groups you will need, including whether more than one moderator is required, or if a series of focus groups would be desirable.

\paragraph{Group size} The number of participants in a \glsf{focus-group} is usually between 8 and 12\footnote{Is a 12-person jury simply a \glsf{focus-group}?}, although other sizes work too; the smallest useful size is considered to be between 4 and 8 for \enquote{mini} focus groups. Anticipating subject loss, you should over-recruit participants by approximately 25\%.

\paragraph{Participants} The purpose of a \glsf{focus-group} is to obtain \glsplf{datum} regarding ideas, attitudes, understanding, and perceptions on a specific topic, and choosing participants that can contribute to this purpose is an important part of identifying the right participants. You should therefore select participants based on their experience and interest in the topic, rather than through random selection. Although the potential range of participants might be limited by context – say, they come from a small organisational group – you should aim for as a diverse range of backgrounds, views, and experiences as possible.

\paragraph{Moderation} You will need a skilled moderator to guide the discussion. Moderators' key qualities include empathy, positive regard\footnote{This denotes a general affirming, caring, and supportive attitude.}, being able to use pauses and probes effectively in the group discussion, and exercising control in an unobtrusive manner. If you do not have access to a skilled moderator, then accessing moderator training for yourself might be desirable\footnote{There are free videos designed to guide moderators on YouTube.}.

\paragraph{Location} You can run \glsplf{focus-group} online or participants can be physically co-located. The latter requires a comfortable environment for your participants, with appropriate consideration of  comfort breaks as part of your schedule.

\paragraph{Choice of questions} Focus groups are sometime referred to as group interviews, in the sense that the moderator seeds and controls the discussion by asking questions. It is important therefore that you consider which questions to ask, including opening questions to get the discussion going, or questions to probe further and to ensure all participants get involved. Open-ended questions are the norm in focus groups as the intention is to elicit insights, attitudes, opinions and perceptions. 

\paragraph{Discussion etiquette} You will need to establish an etiquette for the group discussion, including expected participants' behaviour, for instance in addressing each other, taking turns when speaking, whether mobile devices should be switched off,~\etc{}. 

\paragraph{Recording the discussion} Usually video or audio recordings are used to record the group discussion, so you will need explicit consent from the participants. Note taking is possible but only if there are separate moderators and recorders.

\subsection{Other things to think about}\label{ssect:FocusGroupsOtherThingsTo}

To conduct successful focus groups you must also be aware of:

\paragraph{\Glsf{groupthink}} This a tendency to conform to majority opinion to maintain unanimity and avoid confrontations. \Glsf{groupthink} may inhibit discussion and the expression of diverging views, which rather defeats the point of a focus groups where you aim to elicit diverse views from participants. As a moderator you can mitigate against \glsf{groupthink} by asking probing questions, ensuring that a plurality of views are expressed, or playing devil's advocate in relation to prevailing ideas.

\paragraph{\Glsf{social-desirability-bias}} This is the tendency of participants to express opinions which they think are more likeable or acceptable by the group, even if they are not honest accounts of their views or experiences. As a moderator, you can mitigate against this \glsf{bias} by framing a question in an hypothetical or indirect manner, to distance it from the participant's personal experience, the latter being something they may be reluctant to share. Establishing an atmosphere of trust, anonymity and confidentiality can also help participants be more open and honest.

\paragraph{\Glsf{group-dynamics}} Group culture and power relations, and participants' personality may also introduce \glsf{bias} and affect the end result. For instance, shy participants or introverts may feel overpowered and intimidated by assertive participants, whose views may then become prevalent. You, as a moderator, have the task to ensure that all voices are heard, possibly by calling out shy participants individually, or time-limiting contributions to prevent the most talkative participants from taking over. Larger groups may be more difficult to manage and control, so choose the size of your focus group wisely.


\subsection{Further reading}\label{ssect:FocusGroupsFurtherReading}

\ReadingList{focus group}{powell1996focus,smithson2000using,plummer-damato2008focus}

%https://www.eiu.edu/ihec/Krueger-FocusGroupInterviews.pdf

\section{Delphi}\label{sect:delphi}

With the {\Glsf{delphi}} method\footnote{Also called \Glsf{delphi} technique in the literature. Its name is a reference to the ancient Greek temple that hosted the Oracle of \Glsf{delphi}, famous for her prophecies.}, you consult a group of experts individually with a view to reach a consensus on a particular issue, problem or topic. 

This method assumes that a group of experts are more likely to arrive at an informed and valid position than an individual, due to their \glsf{diversity} of knowledge and experience. It involves an iterative process of collecting, synthesising and circulating anonymous judgements among those experts to arrive eventually at a consensual view. At each iteration the experts can revise their opinion in light of what has emerged in the previous iteration. Anonymity is used to ensure that no individual expert exercises undue influence on the other experts, hence  mitigating against \glsf{groupthink}, and that all participants feel free to express their opinions without any fear of judgement or criticism, hence mitigating against \glsf{social-desirability-bias}. 

%\ResGenTechnique{the \Glsf{delphi} method}

The \Glsf{delphi} method is suited to situations where it important to access collective expertise due to paucity of relevant published knowledge, particularly to inform \glsf{decision} making, policy creation, risk management or forecasting.

\subsection{Operational considerations}\label{ssect:DelphiOperationalConsiderations}

To apply this method you need to consider:

\paragraph{Selecting experts} You should select participants based on their knowledge and experience in relation to the issue, topic or problem under study --- this is purposive rather than \glsf{random-sampling}. You should also ensure a \glsf{diversity} of experts to capture the breadth of expertise, hence to generate valid outcomes. Between 10 and 50 experts are usually selected to participate in a \Glsf{delphi} study, although both smaller and bigger numbers have been used in studies reported in the literature. The more participants, the more resource-intensive the process is going to be to collect, analyse and combine feedback at each iteration.

\paragraph{Process} Maintaining anonymity is essential throughout the process. Initially, you should consult each expert separately, then anonymises and aggregates the group responses and circulate them to the same group of experts to seed the next round of consultation. Providing feedback at each round is the essential mechanism to foster convergence of opinions, as such feedback is used by each expert to review and refine their own opinions or judgements. Ideally, you should repeat this process over multiple rounds until a consensus is reached. Practically, there will only be a limited research time over which you can iterate the process, so you must decide how many iterations you can realistically have. For this, you should also take the experts' availability into account, and their possible fatigue resulting from too many iterations.

\paragraph{Consensus criteria} You must establish explicit criteria to decide when consensus is reached. This may be as simple as establishing a threshold, say when a certain percentage of the experts agree\footnote{A 75\% threshold is often used in the literature.}.

\paragraph{Location} The method is usually executed remotely as there is no direct interaction between the experts.

\subsection{Other things to think about}\label{ssect:DelphiOtherThingsTo}

To be successful, you should also consider the following:

\paragraph{Participants' commitment} This method is resource intensive and you must ensure your participants' commitment for the whole duration of your study. If this is not possible, then a \glsf{focus-group} may be a better option.

\paragraph{Lack of discussion} There is no direct interaction or discussion among the experts, and you, as the researcher, controls and mediates the feedback at each iteration. If your research requires a deeper investigation of ideas, then you should consider alternative methods, like interviews or focus groups.

\paragraph{Difficulty in reaching consensus} Consensus may be difficult to reach, particularly, if you are investigating a very complex or contentious issue, or you are hoping for predictions concerning highly uncertain or volatile contexts. In such cases, you should reflect of the extent consensus is needed in your research and, if not, then consider alternative methods which may allow you to explore alternative or contrasting views and positions.

\subsection{Further reading}\label{ssect:DelphiFurtherReading}


\ReadingList{delphi}{skulmoski2007delphi,skinner2015delphi}



\section{Journaling}\label{sect:journaling}

\Glsf{journaling} is a method requiring participants in a research study to keep regular written personal records in the form of a diary\footnote{Or journal, hence the name.} of their experiences and observations during the study. Participants are encouraged to engage in self-reflection and \glsf{reflexivity} in order to surface their inner thoughts, feelings, motivations and perceptions.

%\ResGenTechnique{\glsf{journaling}}

This method generates rich and detailed \glsplf{quantitative-datum} on the participants' subjective experience. It allows you to collect \glsplf{datum} on everyday experience in naturalistic settings and can provide deep insights into complex \glsplf{phenomenon}, including how things change over time. Because of these characteristics, it is often applied in ethnographic and \glsf{grounded-theory} research.

\subsection{Procedural considerations}\label{ssect:journalingProceduralConsiderations}

%The two uses of \glsf{journaling} lead to two related but different workflows.
%
%\paragraph{For research participants}
%
In applying this method you should take the following into consideration.

\paragraph{Diary form} \Glsf{journaling} can make use of either hand-written or digital diaries\footnote{Of course, accessing hand-written notes will be more labour intensive than electronic ones.}. One or more diaries can be used for different aspects of the \glsf{journaling} process. 

\paragraph{Prompts and guidance} As a researcher, you should establish the goal of the self-reflection and \glsf{reflexivity} at the beginning and explain it clearly to the participants. This is because a lack of guidance or of clear instructions are likely to lead to irrelevant, inconsistent or incomplete \glsplf{datum}. You can use \glsf{journaling} prompts to ensure that diary entries align with your research objectives, guiding participants towards entries that will be useful to the research. Prompts can take the form of questions or comments on such entries. 

\paragraph{Participants} Journalling is demanding for participants as maintaining regular diary entries over lengthy periods can be challenging. Therefore, it is essential that you recruit participants willing to commit to \glsf{journaling} for the duration of the study. In addition, you should monitor their engagement with \glsf{journaling}, and perhaps re-stating research goals and prompts from time to time to help them refocus their effort as necessary.

\paragraph{Managing \glsplf{datum}} \Glsf{journaling} can generate large volumes of \glsplf{datum} from multiple participants, which must be managed carefully. This also raises issues of confidentiality, privacy and more generally \glsplf{datum} protection. As a result it is essential that you establish a systematic approach to storing and backing up the \glsplf{datum}, whether physical or digital. 


\subsection{Other things to think about}\label{ssect:journalingOtherThingsTo}

For a successful execution of the method you should also be aware of the following issues.

\paragraph{Subjectivity and \glsf{bias}} Participants are in control of their diary entries, which, as a result are influenced by their emotions, beliefs, preconceptions and cognitive limitations. These, in turn, may lead to a number of well recognised \glsplf{bias} such as \glsf{confirmation-bias} -- focusing on \glsf{evidence} in support of prior beliefs, memory and recall \glsplf{bias} -- difficulties in remembering or reporting accurately past events, or \glsf{social-desirability-bias} -- only making entries deemed socially acceptable or desirable. Awareness of such \glsplf{bias} and the use of \glsf{triangulation} may help guard against those weaknesses.

\paragraph{Limited generalisability} Diary entries are personal, subjective and usually specific to a particular context or setting. As a results, there may be limited scope for generalising findings based on \glsf{journaling}. If \glsf{generalisation} is an important goal of your research, then you should consider different \glsplf{datum} generation methods, like \glsplf{questionnaire} involving \glsf{random-sampling}.

\newcommand{\ed}[0]{Ed}

\subsection{Further reading}\label{ssect:journalingFurtherReading}

\ReadingList{journaling}{kadarisman2017classroom,burns2009action,james2005journaling,peoples2020write,feinblum2016journaling,hayman2012journaling,mcgrath2021journalling,ovens2020weaving,giguere2012self-reflective,bacon2014journaling}

%\endinput

\section{Fieldwork}\label{sect:fieldwork}

\Glsf{fieldwork} is a \glsf{data-generation-method} which requires the researcher to collect \glsplf{datum} directly in a natural setting\footnote{The \enquote{field}, hence the name.}. The goal is to gain first-hand knowledge of the \glsplf{phenomenon} under study, and the method is widely applied across many disciplines, including anthropology, sociology, archaeology, geography or environmental science. 

Wolcott \cite{wolcott2005fieldwork} offers this salutary advice about \glsf{fieldwork}: 
%
\blockquote{There may be discomfort and hardship aplenty connected with the experience, ranging from the distractions of diarrhoea or lost luggage to the despair of personal failure or lost hope, but the extent of one's suffering and sacrifice are not factored into judgments about the worth of the \glsf{fieldwork} as \glsf{fieldwork}.}


That said, \glsf{fieldwork} is not confined to exotic, distant locations\footnote{Although \glsf{fieldwork} may have this characteristic for some lucky researchers.}! Instead, you can apply it in all natural settings, including, say, your own university or workplace.
%\ResGenTechnique{\glsf{fieldwork}}

Through \glsf{fieldwork} you can generate rich, contextualised \glsplf{datum} to provide deep insights into the \glsplf{phenomenon} under study. It is particularly suited to situations in which such \glsplf{datum} cannot be obtained in any other way, and may also lead to unexpected discoveries. 

The way such \glsplf{datum} are generated will depend on the nature and context of the \glsplf{phenomenon} under study, so that many choices are possible. For instance, you may collect samples and specimens, or make observations and measurements, or provide detailed descriptions of your direct experience in the field, including those from participant observations of social \glsplf{phenomenon}\footnote{This also means that \glsf{fieldwork} may rely on other \glsplf{datum} generation methods.}.

\subsection{Operational considerations}\label{ssect:FieldworkOperationalConsiderations}

Particularly for distant or exotic location, you will need to consider:

\paragraph{Logistics, equipment and budget} Depending on where the field is located, \glsf{fieldwork} may require some detailed planning addressing travel arrangements and accommodation, as well as access to the research site. If in a foreign country, all sort of factors must be considered, including the local transport networks, administrative processes you may need to go through, the climate,~\etc{}. You may also need specialised equipment on site, say field equipment and tools for \glsplf{datum} collection, alongside your personal protection. This can all be quite expensive, so that careful budgeting and securing the required funding in advance are essential.

\paragraph{Permissions and ethical issues} If access to the field of interest is restricted, then you will need to gain appropriate permissions to proceed from the relevant authorities. This covers anything from obtaining permits and licences to access, say, an archeological site, to permission from your employer to perform participant observations in your workplace. You must consider the time and effort to obtain such permissions upfront, and factor in all ethical and legal implications. In addition, your study must be conducted in full respect of local culture and norms.

\paragraph{Health and Safety} In working in the field, you, and anybody else participating in the research, may be exposed to all sort of hazards, so that assessing and mitigating health and safety risk is paramount. This may involve the introduction of safety protocols, of suitable training, and appropriate contingencies in the case of an emergency.

\paragraph{Managing \glsplf{datum}} \Glsf{fieldwork} can generate large volumes of \glsplf{datum}, and may also include precious samples and specimens. All that is collected must be managed carefully, so that alongside issues of confidentiality, privacy and more generally \glsplf{datum} protection, you may also have to worry about the physical security of those samples and specimens. For this, you will need a systematic approach to storing, protecting and backing up your \glsplf{datum}, whether physical or digital. 

\subsection{Other things to think about}\label{ssect:FieldworkOtherThingsTo}

To be successful with \glsf{fieldwork}, you should also consider the following issues.

\paragraph{Quality of results} The quality of results obtained from \glsf{fieldwork} depends on the \glsplf{datum} generated in the field, which, in turn, depends upon your skills as a field worker in relation to the specific \glsplf{datum} generation methods you applied. For instance: using standardised measuring tools will increase the reliability and accuracy of measurements; a reflexive approach will help mitigate against your own personal \glsf{bias}, including \glsf{confirmation-bias}; \glsf{triangulation} may guard against observer and \glsf{social-desirability-bias} in participant observations,~\etc{}. Whichever method you choose to apply in your \glsf{fieldwork}, you should ensure you are aware of their potential weaknesses and adopt appropriate strategies to mitigate their effect on the outcomes of your research.

\paragraph{Logistical challenges} The logistical challenges to organise \glsf{fieldwork} may well be beyond what you can address in the limited time of your research project, unless you are able to contribute to a wider research effort, perhaps led by your supervisor, where all logistical issues have already been dealt with.

\paragraph{Time and cost} \Glsf{fieldwork} can be time consuming both for \glsplf{datum} collection and analysis, and expensive if travel is required. If time or cost are an issue in your project, then you should consider more time efficient or cheaper alternatives.

\subsection{Further reading}\label{ssect:FieldworkFurtherReading}

\ReadingList{fieldwork}{wolcott2005fieldwork,randall2007fieldwork}

\citeauthor{wolcott2005fieldwork}'s book is both detailed and entertaining. Reading it gives on a feeling that spending time in \glsf{fieldwork} with \enquote{Harry} would have been an education in itself. The book includes the importance of laundry to \glsf{fieldwork}, for instance, with experiences of \glsf{fieldwork} in a Canadian Indian reserve to illustrate.

\section{Documents}\label{sect:documents}

Existing \glsplf{document} can be used as \glsplf{datum} sources in order to develop new insights or answer research questions.

%\ResGenTechnique{\glsplf{document}}

The term \glsf{document} is used in a broad sense to refer both to all text-based \glsplf{document} and to audio-visual materials -- such paintings, maps or photographs, video and audio recordings, both physical or digitally stored\footnote{As a researcher, \glsplf{document} – in the form of academic articles – will already occupy a large proportion of your time/brain/computer. Your collection of academic papers could currently be as many as 50 or more. You will, therefore, already have good experience of interacting with \glsplf{document} and those interactions may well have already helped you gain valuable insights for your project.}. 

Researchers examine and interpret \glsplf{document} systematically to extract meaningful information. The \glsplf{document} may be of interest because of their content, or their relation to other \glsplf{document}, or could be studied to discover what they may reveal about their authors, or the historical or cultural context in which they were created. Therefore, a researcher's may have a direct interest in the factual content of a document, or be interested in what that content may indirectly say about some other \glsplf{phenomenon} of interest. An example may help clarify the difference between these two modes. 


\begin{example}{Extracting different meanings from \glsplf{document}}
This is a copy of a passage from \textcite{fynes1873miners}:
%
\quotation{Miner :– I believe you have something like 150 collieries to inspect?

Mr.~Dunn :– Yes.

Miner :– Twenty-eight in Cumberland?

Mr.~Dunn :– Yes.

Miner :– Do you think you are able to inspect all these? 

Mr.~Dunn :– Well, the Government thinks I am able, you know.

Another Miner :– Were you satisfied with the one shaft at this colliery, if so there is an end to the matter; if not, what steps did you take to remedy the defect? Did you apply to the Secretary of State, showing him that it was defective?

Mr.~Dunn :– At this very moment there are three of the largest collieries in Northumberland – Seaton Delaval, North Seaton, and Newsham – managed by the most talented men in Northumberland, all with single shafts. Now, what would you have me to do? Do you think it is my duty to call in question the management of these pits?

Miner :– Am I to understand this is an answer to my question?

Mr.~Dunn :– Well, I am not so well satisfied as if they had two, but I have not the power to alter it.}
\\

Such a passage could be considered in two ways. A direct reading could be to identify collieries in which a single shaft existed at that time. An indirect reading could be to explore social relationships within a mining community in 18th century England.
\end{example}

Document-based research provides the researcher with \glsf{evidence} of historical events or social \glsplf{phenomenon}, including nuanced details and perspectives that may not be available through other means. In particular, \glsplf{document} allow you to access \glsplf{datum} pertaining to different time periods, locations or cultural contexts. 

\subsection{Procedural considerations}\label{ssect:DocumentsProceduralConsiderations}

To apply this method you should consider the following issues.

\paragraph{Accessing \glsplf{document}} You need to ensure you have access to the \glsplf{document} you need for your research. While \glsplf{document} are increasingly digitised and easily accessible online, it is also the case that access for many may be restricted by either policies or physical restrictions -- say you wish to study restricted confidential \glsplf{document} in an organisation, or access rare or ancient manuscripts kept in a museum. Ensuring you have the right access at the right time in your project is essential, but can be time consuming, particularly if there are bureaucratic processes you need to go through. Related to access are issues of translation if the \glsplf{document} are in a language you are not familiar with, or transcription, if your sources are audio or video recordings. As well as being time consuming, these processes may introduce errors which may be difficult to spot. Lastly, it is essential you ensure that your \glsplf{document} are authentic: only using trusted sources is a way to do so.

\paragraph{Selection criteria} You must develop clear and explicit selection criteria to decide which \glsplf{document} to include in your study, based on your \glsf{research-problem}, and aim and objectives. Such criteria should help you collect an appropriate and representative selection of \glsplf{document} for your research, guiding you in what to include and what to leave out. 

\paragraph{\Glsplf{datum} management} Alongside generic issues of \glsplf{datum} storage, protection and privacy, you also need to ensure that your source \glsplf{document} remain easily accessible and that their integrity is maintained: this is both to allow you to revisit those \glsplf{document} repeatedly during your study, and  to allow other researchers to check your sources to verify and validate your findings. Whenever possible, you should digitise your source \glsplf{document} to enhance their accessibility and preservation.

%LR -- the rest is commented out as it is about analysis, which is covered more generally in under \glsplf{analysis-method}

%Analytical Framework: Developing an appropriate analytical framework is essential for guiding the analysis of \glsplf{document}. Researchers may use thematic \glsf{coding}, \glsf{content-analysis}, \glsf{discourse-analysis}, or other methods to identify patterns, themes, and relationships within the \glsplf{document} and generate meaningful insights.
%
%In more detail, indirect and direct reading use the following techniques to classify and conceptualise \glsplf{document} for qualitative research:
%\begin{itemize}
%\item studying the form, content, and function of \glsplf{document}
%\item analysing the ways in which \glsplf{document} justify decisions, display hierarchies, and exercise agency
%\item distinguishing between primary sources (\glsplf{document} providing \glsplf{raw-datum}) and secondary sources (\glsplf{document} providing information about primary sources)
%\end{itemize}
%

%As we have already mentioned, direct interaction with \glsplf{document} is already something with which you are\footnote{Or, should be!} familiar. However, according to \textcite[p.~370]{coffey2014analysing}\footnote{Freely available from \citeurl{coffey2014analysing}.}, a good place to start in analysing \glsplf{document} is with the realisation that \glsplf{document} are \enquote{socially defined, produced, and consumed} so that, alongside the content, the processes by which a document is produced and is intended to be consumed, i.e., by which it becomes an \enquote{accomplishment} for some individual or organisation, often contain useful \glsplf{datum}.
%
%In this case, there are many approaches possible from the simplest – counting instances, for instance, of words, phrases, or other elements\footnote{\textcite{mckenzie1959shakespearian}, for instance, counts commas – as well as other punctuation – in Shakespear's \emph{first Folio} to help identify which compositors were responsible for which parts of it.} – and indexing and \glsf{coding} document elements to identify thematic content and identify patterns, ala \textcite{schreier2014qualitative}. 
%
%\citeauthor{schreier2014qualitative}'s approach, a type of the more formal \emph{qualitative \glsf{content-analysis}}, involves the selection of material from a wide range of source with adequate coverage for the topic of interest. As such it can result ing large amounts of material to analyse and so an element of formal in the process is mandated. An important step in this is the building of a \emph{\glsf{coding} frame}, i.e., a structure consisting of \emph{main category} and two \emph{sub-categories}~\parencite{schreier2014qualitative}:
%%
%\begin{itemize}
%\item a main category\footnote{\Glsf{coding} frames with more than 40 main categories are known to be difficult for one person to handle at one time.} being those aspects of the material that interest the researcher, 
%\item a subcategory being what is said in the material with respect to a main categories.
%\end{itemize}

%\paragraph{\Glsf{coding} for keeps} once you're convinced that your \glsf{coding} frame is stable, you then apply it to \emph{all} \glsplf{document}. This will involve:
%%
%\begin{itemize}
%\item  paying attention to how \glsplf{document} are constructed as distinctive artefacts, their structure, vocabulary, level of formality,~\etc{}. \parencite[p.~5]{coffey2014analysing}
%\item analysing their form and content, considering how they were produced and how they were – and how they were intended to be – consumed
%\item considering the relationships \emph{between} \glsplf{document}, highlighting dimensions of similarity, comparison, contrast, and difference by tracing how texts refer to other texts, sharing conventional formats, and constructing a uniform style\footnote{This scaled \emph{Intertextuality} in the literature~\parencite[p.~8]{coffey2014analysing}; it's akin to how you went about analysing the \glsf{academic-literature} as part of your literature survey.}
%\item exploring the ways in which they function and are used in everyday life and social context
%\item examine their formal properties, including their linguistic registers, and rhetorical features.
%\end{itemize}

%\paragraph{Presenting your findings} can be as simple as delivering the statistics that you have collected as the basis for further analysis. Alternatively, a completed \glsf{coding} frame  with an accompanying narrative and example quotes as needed could be the output. Like other \glsplf{datum}, your \glsf{coding} frame can also be the starting point of further \glsplf{datum} generation and analysis, \textcquote{schreier2014qualitative}{examining the results [...] for patterns and co-occurrences of selected categories.}

\subsection{Other things to think about}\label{ssect:DocumentsOtherThingsTo}

To be successful in document-based research you must also consider:

\paragraph{\Glsf{bias} in \glsplf{document}} \Glsplf{document} are created by people, who necessarily inject their own personal \glsf{bias} into their content\footnote{So called \enquote{creator} \glsf{bias}.}, which in turn is the result of their historical, cultural and social contexts\footnote{Another \glsf{bias}, called \enquote{contextual} \glsf{bias}.}. In addition, particularly in the case of ancient manuscripts, the \glsplf{document} that have survived may only provide a partial historical account\footnote{You'll have guessed there is a name for this too, which is \enquote{survivorship} \glsf{bias}.}. Therefore, you must choose your selection criteria wisely to ensure a \glsf{diversity} of sources and guard against \glsf{selection-bias}, which may lead to certain positions, perspectives, or types of \glsplf{document} to be either overrepresented or underrepresented. Being aware of all these \glsplf{bias} is essential to the interpretation of \glsplf{document} content and how they may skew or limit your research results. 

\paragraph{Interpretation challenges} \Glsplf{document} are unlikely to provide a complete picture of the \glsplf{phenomenon} under study, partially due to their inherent \glsplf{bias}, but also because they may be incomplete or lack crucial details or contextual information may not be available to you. Also certain \glsplf{phenomenon} may be more documented than others, so that the availability and quality of \glsplf{document} can vary widely across topics, history or geography. All these factors affect your ability as a researcher to interpret their content and draw robust conclusions.

\paragraph{Time and effort} Document-based research may require large volumes of materials to be selected, collected and analysed, so that it can be very time-consuming. If time is an issue in your project, then you should consider alternative \glsplf{datum} generation methods.

\subsection{Further reading}\label{ssect:DocumentsFurtherReading}

\ReadingList{document-based research}{coffey2014analysing,schreier2014qualitative}

\chapter{Modelling methods}\label{ch:ModellingMethods}

So far you have considered methods which can help you generate \glsplf{datum} from either direct observations or experience (whether yours or of other research participants), or from secondary sources. Once generated, you can organise and analyse such \glsplf{datum} by applying \glsplf{analysis-method}, which is the subject of the next chapter.

Somewhat in between \glsplf{datum} generation and analysis are modelling methods, whose aim is to help you build models of natural, social or artificial \glsplf{phenomenon}, that you can then use for analysis, prediction or \glsf{decision} making, including informing the design and engineering of new artefacts. Such models need \glsplf{datum} to inform their development and, in turn, generate new \glsplf{datum} for analysis. Modelling methods support a variety of research strategies including \glsf{simulation} and \glsf{design-science-research}, but also case studies in which models of socio-technical systems may be useful for investigation.

At its essence, a \glsf{model} is a representation of something, be that a \glsf{system}, a structure, a process or a behaviour. Possibly the most important thing to remember about modelling is expressed by the following oft-cited aphorism:
%
\begin{displayquote}
	All models are wrong, some are useful.~\parencite{box1976science} 
\end{displayquote}
%
\noindent which makes clear that a \glsf{model} should not be regarded as a faithful replication of some  reality, but as a tool to investigate some aspects of that reality.

At the core of modelling is a process of \glsf{abstraction} which starts from an understanding of what is to be modelled and ends with the definition of the desired \glsf{model}. The nature of both determines the kind of thinking required in the \glsf{abstraction} process. 

This chapter covers the modelling methods summarised in~\Cref{tab:modellingMethodsChoice} and provides similar descriptions to the \glsplf{datum} generation methods of~\Cref{ch:DataGenerationMethods}.

If your chosen \glsf{research-methodology} does not include any modelling methods, then you can skip this chapter altogether and move on to the \glsplf{analysis-method} in~\Cref{ch:AnalysisMethods}.

Otherwise it's the time to consider your modelling methods in detail, to ensure they are the right ones for your project, and to apply them to your research. The next two activities will help you do that: they are very substantial activities which will allow you to take a significant step in your research.

\begin{activity}[{Finalising your modelling methods}]
 Consider your chosen modelling methods. For each, read through the related section in this chapter (see~\Cref{tab:modellingMethodsChoice}), and write a summary including the following points:
 \begin{itemize}
 	\item explain why it is an appropriate method for your project, in relation to your aim and objectives, your chosen \glsf{research-strategy}, the \glsplf{datum} sources at your disposal, and your relevant technical skills 
 	\item if there are variants of the method, indicate the one you will apply and why 
 	\item detail your procedures in applying the method within your project
 	\item indicate how you will guard against potential research weaknesses intrinsic in the method
 	\item discuss your choices with your supervisor to ensure they are appropriate for your project.   
 \end{itemize}
\begin{guidance}
%Hack to correct tcbox behaviour
\color{black}


%Hack to correct tcbox behaviour
\color{black}
This activity assumes that you have completed a draft of your methodology which includes your chosen \glsf{research-strategy} and possible research methods within. If that's not the case, then you should return to~\Cref{stage3} and complete the activity in~\Cref{ch:yourResearchMethodology}.

For each method, conduct a first pass at addressing each of the points above based on the content of this chapter, then follow some of the references provided to read more about the method and improve your summary. Your supervisor should also be able to point to relevant literature or provide expert advice on the application of the method.

A procedure indicates the specific steps you will take to ensure the method is applied correctly and that you have guarded against potential methodological weaknesses. In your dissertation, your procedures should be sufficiently detailed for other research to follow what you have done or even replicate your work independently. 

You may have to iterate with your supervisor to refine your choices and procedures.
\end{guidance}
\end{activity}
%%Hack to correct tcbox behaviour
\color{black}

%%%Intentionally \ref, not~\Cref
%\newcommand{\tick}{$\Box$}\todo{Check computational (and other) )thinking for glossary entry}
\begin{SimpleNColTable}{tab:modellingMethodsChoice}{2}{Modelling methods discussed}[rX[c, wd=20ex]]
	Method & Section\\
	\Glsf{computational-thinking}&\ref{sect:ComputationalThinking}\\
	\Glsf{mathematical-thinking}&\ref{sect:MathematicalThinking}\\
	\Glsf{statistical-thinking}&\ref{sect:StatisticalThinking}\\
	\Glsf{systemic-thinking}&\ref{sect:SystemicThinking}\\
\end{SimpleNColTable}

At the end of this activity you should be ready to start modelling. 

\begin{activity}[{Applying your modelling methods}]
 Apply your modelling methods by following the procedures you have outlined and agreed with your supervisor. You should:  
  \begin{itemize}
 	\item proceed incrementally, checking at each increment that your modelling choices remain appropriate
 	\item document any deviation or adjustment you may have to make in executing your methods
 	\item ensure you manage different \glsf{model} increments and versions appropriately
 	\item establish regular checkpoints with your supervisor to review your modelling process and outcomes, and to agree any required adjustment. 
 \end{itemize}
\begin{guidance}
%Hack to correct tcbox behaviour
\color{black}


%Hack to correct tcbox behaviour
\color{black}
The kind of incremental process you can follow will depend on the method you will apply. For instance, you may be able to split a computation \glsf{model} into different functionalities, which you can then \glsf{code} and integrate in subsequent increments.

Regular monitoring and adjustments are essential: it is highly unlikely that things will go exactly as you originally planned them!
\end{guidance}
\end{activity}
%%Hack to correct tcbox behaviour
\color{black}



\section{Computational thinking}\label{sect:ComputationalThinking}

\Glsf{computational-thinking} is needed when the end point is a \glsf{model} in the form of a computational artefact, that is something that a digital computer can execute\footnote{\Glsf{computational-thinking} has a much broader scope than what is reported here. For instance, it underpins learning and curriculum in Computing-related disciplines, as well as professional skills in related industries.}.

%\ResModTechnique{computation thinking}

\Glsf{computational-thinking} is a problem solving approach in which problems are explored with a view to identify and implement computational solutions in the form of computer programmes and systems. In addition to writing code that a computer can execute, \glsf{computational-thinking} involves a wide range of cognitive processes such as being able to think at different levels of \glsplf{abstraction}, to decompose problems into sub-problems, to identify useful patterns and structures in \glsplf{datum}, to conceptualise logical steps the computer should take alongside how people may interact with those programmes and systems.  

Computational artefacts are becoming more and more prominent in \glsf{academic-research}, which both makes use of existing ones and develops new, bespoke ones to advance knowledge. 



%The techniques generally associated with \glsf{computational-thinking} are:
%%
%\begin{itemize}
%\item decomposition: breaking a problem into parts that are easier to solve;
%\item pattern recognition \& \Glsf{generalisation}: seeking \glsf{significance} from repeated structures within \glsplf{datum}
%\item \glsf{abstraction}: building higher level representations of things
%\item algorithms: associating behaviours with computational structures
%\item evaluation: called testing for software, but also has wider applicability when an algorithm is socialised, i.e., used within its final social context.
%\end{itemize}


%Clearly, part of computation thinking is programming a computer to achieve a goal. 
%
%However, \citeauthor{wing2006computational}\footnote{\citeauthor{wing2006computational} is American, and so uses American spelling – -i\textit{z}i-, art\textit{i}fact,~\etc{}}, in her \citeyear{wing2006computational} article that popularised \Glsf{computational-thinking}~\cite{wing2006computational} as a candidate curriculum entry, places programming in the broader context of \glsf{computational-thinking}. If you are aiming at \glsf{computational-thinking} as a way of focussing on your programming skills, it may be that you have misjudged this wider context. 
%
%\citeauthor{wing2006computational} says:
%
%\noindent\emph{\Glsf{computational-thinking}} [...] has the following characteristics\footnote{\citeauthor{wing2006computational} includes other characteristics, of relevance to curricula but not so much to research.}:
%
%\begin{description}
%\item [conceptualising] Computer science is not [just] computer programming. Thinking like a computer scientist means more than being able to program a computer. It requires thinking at multiple levels of \glsf{abstraction}; 
%%
%%\item[Fundamental, not rote skill.] A fundamental skill is something every human being must know to function in modern society. Rote means a mechanical routine. Ironically, not until computer science solves the AI Grand Challenge of making computers think like humans will thinking be rote; 
%%
%\item [A way that humans, not computers, think.] \Glsf{computational-thinking} is a way humans solve problems; it is not trying to get humans to think like computers. %Computers are dull and boring; humans are clever and imaginative. We humans make computers exciting. Equipped with computing devices, we use our cleverness to tackle problems we would not dare take on before the age of computing and build systems with functionality limited only by our imaginations; 
%%
%\item [Complements and combines mathematical and engineering thinking.] Computer science inherently draws on \glsf{mathematical-thinking}, given that, like all sciences, its formal foundations rest on mathematics. Computer science inherently draws on engineering thinking, given that we build systems that interact with the real world.% The constraints of the underlying computing device force computer scientists to think computationally, not just mathematically. Being free to build virtual worlds enables us to engineer systems beyond the physical world; 
%%
%\item [Ideas, not artefacts.] It's not just the software and hardware artefacts we produce that will be physically present everywhere and touch our lives all the time, it will be the computational concepts we use to approach and solve problems, manage our daily lives, and communicate and interact with other people.%; and
%%
%%\item [For everyone, everywhere.] \Glsf{computational-thinking} will be a reality when it is so integral to human endeavors it disappears as an explicit philosophy.
%\end{description}
%
%Recognising the social context of \glsf{computational-thinking} is, thus, critical to research in the area.

\subsection{Procedural considerations}\label{ssect:ComputationalThinkingProceduralConsiderations}

Given the explosive growth in the use of computers over the past half century, you may not be surprised to hear that there are thousands of useful\footnote{As well as some that are less than useful!} tools to support \glsf{computational-thinking}. They vary in many of their characteristics, so that you will need to make some judicious choices for your project. In particular you will need to consider:

\paragraph{Programming language and paradigm} This concerns the language you will use to write your \glsf{code}, and its underlying philosophy, i.e., the assumed building blocks of the language and how they fit together in programs.

\paragraph{\Glsf{computational-model}} This refers to the way computations take place in the implemented artefact, one of sequential, concurrent, distributed or agent-based. The latter is particularly suited to the \glsf{simulation} of complex systems made of many interacting, independent agents. While all programming languages allow you to develop sequential computations, you may need specialised languages\footnote{You can look, for instance, at Petri Nets or NetLogo to get some ideas.} for the other modes. 

\paragraph{Delivery platform} This refers to where your computational artefact will be made available for use, be that the web, a mobile device, or some other bespoke hardware.

\paragraph{\Glsf{integrated-development-environment}} This the combination of tools to help you develop and keep track of your \glsf{code}, including how it changes over time, and to perform tests to check its intended behaviour and to correct errors and mistakes. 	

\paragraph{Stakeholders and participants} These are all the people you may have to involve to tease out requirements\footnote{Which need will your artefact meet? Which characteristics should it have?}, validate your artefact or generate \glsplf{datum} by interacting with it.

\paragraph{Development process} This is the process\footnote{Several schools of thought exist as to what constitute a good process for developing computational artefacts. You can look up Agile and Plan-driven development processes to get some ideas.} you will follow to determine what your \glsf{code} should do, and to design, implement, test and release it for use.

%The main idea behind \glsf{computational-thinking} is to link a wished-for behaviour to a computational structure. Here, we're thinking of the wished for behaviours of a client in context being translated to an algorithm, \glsplf{datum} \glsf{model}, or other computational element. 
%
%The ways in which the wished-for behaviour of a client in context are understood is through the discipline of requirements analysis. Although naive requirements analysis is possible for simple projects – asking a client what functional behaviours they need – it is more often driven by the explicit or implicit management or developmental risk, the risks being those of 
%

\subsection{Other things to think about}\label{ssect:ComputationalThinkingOtherThingsTo}

You should take the following considerations into account.

\paragraph{Foundational knowledge} If you don't have any experience of \glsf{computational-thinking} or writing \glsf{code}, then you can learn, but the learning curve is going to be very steep. Unless you have direct access to experts to guide you, you will need to consider very carefully whether you will be able to achieve the proficiency you will need within the timeline of your project. 

%\footnote{This video will give you some idea of what will be required: \url{https://www.youtube.com/watch?v=fLMZAHyrpyo}. Stephen Wolfram is a particularly interesting, if restless, speaker and a leader in this area.}\todo{But this isn't a great video, nor is any other that I've looked at, as they're created by computer scientists... They're also heavily advertise laden.}. 

\paragraph{\Glsf{model} \glsf{validity}} You need to worry about two key aspects of \glsf{validity} when developing \glsplf{computational-model}. One is internal, and concerns the issue of whether you have made mistakes in your code: appropriate code review and testing techniques can help you take care of this. The other is external, and concerns the relation between the \glsf{model} itself and the reality it means to \glsf{model}. In order to establish this, you will need to consider:
		\begin{itemize}
			\item how well it fits the context in which it is eventually installed	
			\item how well it addresses the problem(s) it is meant to solve, and
			\item how well it meets stakeholders' expectations, including existing professional quality standards.
		\end{itemize}

\paragraph{Timing issues} Developing a \glsf{computational-model} can be very time consuming, particularly when you need to interact with many stakeholders as part of its development, which may then require you to iterate between \glsf{coding} and validation several times. If you are not confident you can accommodate such a development effort within your project, then you should consider other methods or reduce the scope of your \glsf{model}. 

\subsection{Further reading}\label{ssect:ComputationalThinkingFurtherReading}

\ReadingList{computational thinking}{angevine2017computational,figueiredo2017improve,lyon2020computational}

\section{Mathematical thinking}\label{sect:MathematicalThinking}

\Glsf{mathematical-thinking} is problem solving with mathematics. It has had many centuries more than \glsf{computational-thinking} to develop and the tools that exist as part of it are very stable. They are also much better explored due to the efforts of many great mathematicians. However, they do require a high level of skills and sophistication in their application to achieve their full potential.

%\ResModTechnique{\glsf{mathematical-thinking}}

Although mathematical techniques can apply to real-world problems, they tend to create \glsplf{closed-form-solution}, that is solutions which can be calculated exactly from mathematical expressions. For instance, systems of differential equations are widely used in Finance to model fluctuations on stock or investment markets. As long as a real-world problem can be captured in this way, then a mathematical model is feasible. However, many real-world problems do not admit such characterisations, so that there are limitations as to what you can treat mathematically. 

Note that there is a strong connection between mathematical and \glsf{computational-thinking} in that lots of mathematical models are now implemented as computational algorithms executed by computers. These, however, require some \glsf{numerical-approximation} as computers cannot calculate exact values\footnote{Due to such approximations, there may be accumulated errors which also need to be taken into account.}.

%Similarly to \glsf{computational-thinking}, \glsf{mathematical-thinking} involves various cognitive processes:
%
%\begin{description}
%\item [Specialising] exploring a problem through examples. Each example provides the opportunity for manipulating elements that are concrete, whether they are physical manifestations or ideas.
%\item [Conjecturing] when enough such examples have been examined, you can conjecture about the relationships that connect them. Through conjecturing, underlying patterns are explored, expressed, and then substantiated.
%\item [Generalising] if you are lucky enough to have found a pattern, then you might try to generalise it to creating order and meaning out of a – potentially, overwhelming – amount of \glsplf{datum}.
%\item [Convincing] a \glsf{generalisation} must be tested until it is convincing to the reader – this is the basis of the knowledge contribution from \glsf{mathematical-thinking}.
%\end{description}



%Mathematics has limits of applicability, some of which are extremely subtle, even in relatively simple situations: complete closed form solutions in radicals do not exist for problems as simple as finding the roots to polynomial equations of order 5\footnote{So called \emph{quintic equations}~\parencite{enwiki:1209364767}. A radical is a mathematical expression involving only the coefficients of the equation, and the basic arithmetic operations (addition, subtraction, multiplication, division, and taking the n\fmtord{th}-root).} or above; the general three body problem, for instance, being resistant to differential equation analysis; and others. 

%Alternatively, so called \emph{numerical approximations} can be made to most problems, but these are often related to computational solutions and so might be better approached through \glsf{computational-thinking}.
%
%Another alternative that again takes us back to \glsf{computational-thinking}, is the use of agent based simulations in which concurrently acting independent agents are used.\todo{More here.}
%
%Differential equations; matrices; algebra; numbers \glsf{theory}; sets; functions; relations; logics\footnote{\emph{Logics} is plural as there are many, depending on which area of mathematics you are using.}; topology; geometry; calculus; algebra; analysis; \todo{More here?}
%
%According to \textcite[, adapted]{burton1984mathematical}, there are four cognitive processes that are central to \glsf{mathematical-thinking}: 

\subsection{Operational considerations}\label{ssect:MathematicalThinkingOperationalConsiderations}

To apply \glsf{mathematical-thinking} you should consider:

\paragraph{Mathematical tools} This concerns the choice of the kind of Mathematics to apply, including notation, and symbolic and diagrammatic representations appropriate to the problem you are trying to address. 

\paragraph{Computational tools} Should you wish to use a computer to run your mathematical models, then you will also need to make many of the choices related to \glsf{computational-thinking}\footnote{See~\Cref{ssect:ComputationalThinkingProceduralConsiderations}.}. Note that many modern programming languages and environments include a wide range of mathematical libraries which you can use directly in your code, so that you don't have to start your code from scratch, reducing substantially the time and effort required. Such libraries are also likely to have been tested extensively, hence their code should be error-free and highly reliable.

\paragraph{Relevant examples} Although you will need to be creative in applying mathematics to your own \glsf{research-problem}, there may be relevant examples in the literature which can provide a good starting point. Working from simple examples to more complex ones may help you establish an appropriate mathematical approach.


\subsection{Other things to think about}\label{ssect:MathematicalThinkingOtherThingsTo}
%\endinput
For a successful application of \glsf{mathematical-thinking} you must also consider:

\paragraph{Foundational knowledge} \Glsf{mathematical-thinking} is arrived at through creative thinking and deep study of mathematical tools and techniques. The sophistication of mathematics often means that, either:
%
\begin{itemize}
\item a particular area of research has already been taken past the abilities of an entry-research-level mathematician;
\item it is not amenable to (current) mathematical tools and techniques, and further creative\footnote{And, most likely, deep and advanced, out of the box, out of this world, and further.} mathematical thinking will be necessary to progress.
\end{itemize}
%
These make timely contributions to knowledge through the application of \glsf{mathematical-thinking} difficult. It's worth moderating your expectations of what can be achieved in your project – discussion with your supervisor of what their expectations are would be very worthwhile\footnote{What is often missing from the mathematical literature – or what isn't always visible to the new entrant – is the often vast timescales over which mathematical progress is made. Bertrand Russell and Alfred Whitehead spent over two decades of their professional lives in the creation of the three volume \emph{Principia Mathematica}. A fourth volume – on geometry – was begun but never completed. In another example, the proof of Fermat's Last Theorem took 358 years to complete.}.

%\begin{activity}[{Understanding your supervisor's \glsf{mathematical-thinking} expectations}] 
%	Schedule some time with your supervisor to discuss what they hope will be achieved through your research project.
%\end{activity}
%%Hack to correct tcbox behaviour
%\color{black}


%Mathematics is a very tight community with very high publication standards. The language of mathematics is dense\footnote{A mathematical statement indicative of this complexity is: \enquote{Let q be $x^{5}-x-1$. Let $G$ be its Galois group, which acts faithfully on the set of complex roots of $q$. Numbering the roots lets one identify $G$ with a subgroup of the symmetric group $\mathcal{S}_{5}$. Since $q{\bmod {2}}$ factors as $(x^{2}+x+1)(x^{3}+x^{2}+1)$ in $\mathbb{F}_{2}[x]$...}.}.
%
%Because of this, many mathematical research projects at master's level are designed to provide a way into the mathematical literature. A supervisor will set a mathematical task that may already have been solved. The contribution a research student might make, then, is not a contribution to knowledge in the formal sense of extending mathematics, but the widening of the mathematical community to include another researcher whose skills have expanded to be able to make a novel restatement of a problem, for instance, and research further. 
%
%The study of the development of \glsf{mathematical-thinking}, for instance, in schoolchildren is a very fruitful area with much still to be contributed.\todo{More here}
%

\paragraph{Limitations of mathematical \glsf{abstraction}} Mathematics abstracts from real-world complexity: in modelling traffic to improve flow through a complex junction, for instance, one would not necessarily consider the economy of individual cars, or the noise pollution created by a solution. This can reduce a real-world problem to a complexity that is approachable, but may also lead to non-solutions when applied back in the real world, for instance, leading to complaints from local home owners that noise pollution has risen through a solution. Therefore, you need to check your simplifying assumptions carefully against the real-world situation to avoid reaching invalid conclusions.

\subsection{Further reading}\label{ssect:MathematicalThinkingFurtherReading}

\ReadingList{mathematical-thinking}{stacey1982thinking}


\section{Statistical thinking}\label{sect:StatisticalThinking}

\Glsf{statistical-thinking} is problem solving with Statistics. You can use it to identify patterns, trends and relationships in \glsplf{datum} by using probabilistic reasoning, which acknowledges variability and uncertainty inherent in the \glsplf{datum}. 

%\ResModTechnique{\glsf{statistical-thinking}}

\Glsf{statistical-thinking} is particularly valuable for prediction and forecasting -- for instance, to predict the \glsf{spread} of a virus in a \glsf{population} or how the average house price may change in a geographical area and over a certain period of time, or to test \glsplf{hypothesis}\footnote{You will encounter statistical tests in~\Cref{sect:inferentialStatistics}} -- for instance whether a medical treatment is likely to be effective. \Glsf{statistical-thinking} is essential in quantitative analysis, which will be covered in~\Cref{sect:statisticalAnalysis}.

%
%\Glsf{statistical-thinking}\todo{Source: \textcite{chance2024statistical}.} involves designing a study to collect \glsplf{datum}, analyse patterns in the \glsplf{datum}, and draw conclusions that go beyond the observed \glsplf{datum}. \Glsf{sampling} is key to being able to generalise results, while random assignment is key for cause-and-effect conclusions. Probability models help assess random variation and estimate margin of error.



\subsection{Procedural considerations}\label{ssect:StatisticalThinkingProceduralConsiderations}

To build statistical models you need to consider:

\paragraph{\Glsf{sampling}} Statistical modelling requires relevant \glsplf{datum}, so that you have to consider how you will obtain such \glsplf{datum}. \Glsf{sampling}\footnote{See~\Cref{sect:sampling}.} is a standard way to do so, for which you need to worry about both \glsf{sample} size and the extent it is representative of the \glsf{population} of interest.

\paragraph{\Glsplf{datum} quality} You will need good quality \glsplf{datum} for your models, that is \glsplf{datum} with no missing, inconsistent or erroneous entries. Typically, you will need to pre-process your \glsplf{datum} to ensure this is the case before applying any statistical technique. 

\paragraph{Choice of techniques} You will need to decide which statistical techniques to apply in relation to the \glsf{research-problem} you are trying to address. Like Mathematics, Statistics too includes a vast repertoire of techniques applicable to different problems. If you do not possess sufficient expertise to be able to choose by yourself, then you will need to take expert advice, as applying inappropriate techniques will lead to invalid or misleading results.

\paragraph{Statistical software} The use of computational tools is the norm to support \glsf{statistical-thinking} and many bespoke statistical software applications are available. In addition, many current programming languages and environments\footnote{E.g., R or Python.} come equipped with full libraries of statistical functions ready for use. These allow you to perform most statistical modelling and testing alongside \glsplf{datum} manipulation and visualisation with graphs and charts.

\subsection{Other things to think about}\label{ssect:StatisticalThinkingOtherThingsTo}

To be successful with statistical modelling you also need to consider:

\paragraph{Foundational knowledge} Statistical techniques can be complex and require specialised knowledge and skills to apply them effectively, particularly for advanced modelling, but also to provide meaningful interpretation of outcomes. As for \glsf{mathematical-thinking}, unless you have sufficient foundational knowledge, you need to evaluate carefully whether you will be able to develop the required advanced knowledge and skills within the remit of your project.

\paragraph{\Glsplf{datum} \glsf{bias}} Even when your \glsplf{datum} are of good quality, they can still be biased in that they may over- or under-represent certain characteristics of the \glsf{population} of interest, leading to invalid or unreliable generalisations. For instance, if all \glsplf{datum} in a clinical trial for a new medicinal drug are from male participants, then the effectiveness or otherwise of the drug for female patients cannot be inferred from those \glsplf{datum}, so that any \glsf{generalisation} to the wider \glsf{population} may be unsound.

\paragraph{Confounding factors} Unfortunately, things can still go wrong even when you have good quality, unbiased \glsplf{datum} to start with. This may be due to {confounding factors} you may not have considered in your study. For instance, say you are interested in the possible relationship between physical activity and heart health. If you only focus on (some measures for) those two variables, you are likely to miss possible effects of, say, age and gender on heart health in addition to physical activity, and may, once again, infer the wrong conclusions. 

\paragraph{\Glsf{model} assumptions} Statistical models are usually based on specific assumptions made of \glsplf{datum} characteristics. For instance, many statistical tests assume that the \glsplf{datum} are normally distributed\footnote{You will study these topics in~\Cref{sect:inferentialStatistics}}. It is therefore essential for you to check that all required assumptions hold, otherwise your \glsf{model}, and any conclusions you derive from it, may be invalid.

\paragraph{Ethical considerations} As \glsf{statistical-thinking} relies on \glsplf{datum}, then ethical and legal issues arise in relation to how the \glsplf{datum} are obtained and used in your research, particularly around privacy and \glsplf{datum} protection. In addition, ethical issues arise in relation to  social implications of applying \glsf{statistical-thinking} in decision-making, particularly when decisions are increasingly taken by algorithms. This can lead to inequality and discrimination, as demonstrated by some shocking cases which have been reported widely -- such as the COMPAS \glsf{system}, discriminating against black offenders, or the Amazon's recruitment algorithm, discriminating against women. 

\subsection{Further reading}\label{ssect:StatisticalThinkingFurtherReading}

\ReadingList{statistical thinking}{chance2024statistical}

\section{Systemic thinking}\label{sect:SystemicThinking}

\Glsf{systemic-thinking}\footnote{The term \enquote{systems thinking} is also used in the literature.} is yet another problem solving approach which focuses on systems and their dynamics. A \glsf{system} is a set of elements coming together in a complex whole and whose behaviour stems from the interaction of those elements. Any kind of \glsf{system} is in scope, whether natural, social or artificial, with \glsf{systemic-thinking} aimed at understanding how the different elements influence each other and how the \glsf{system} behaviour emerges from their interaction. 

%\ResGenTechnique{\glsf{systemic-thinking}}

\Glsf{systemic-thinking} takes a holistic approach to understanding a \glsf{system}, encouraging different stakeholders' perspectives and a participative approach to develop a shared understanding. This also means that the enquiry process is iterative, with insights being revisited, reviewed and refined as more knowledge is acquired through ongoing analysis and interaction with stakeholders.

Like mathematical and \glsf{statistical-thinking}, \glsf{systemic-thinking} may be used in combination with \glsf{computational-thinking}, for instance, to develop a better understanding of a \glsf{system} of interest before creating a computational \glsf{simulation} of it, or as an aid to prototyping novel computational artefacts.

\Glsf{systemic-thinking} usually relies on \glsplf{system-model} based on diagrammatic notations, of which there are a great variety. Among the most common are\footnote{This is by no means a complete list!}:

\begin{itemize}
	\item \glsplf{systems-map}, which allow you to sketch the structure of a \glsf{system} by identifying key components and sub-systems
	\item \glsplf{influence-diagram}, which extend systems maps to show how those elements influence each other
	\item \glsplf{causal-loop-diagram}, which are used to capture cause-and-effect relations, and particularly feedback loops in the \glsf{system} dynamics which affect behaviour over time
	\item \glsplf{stock-and-flow-diagram}, which augment causal loop diagrams with quantitative information that can be exploited in computational simulations
	\item \Glsf{unified-modeling-language} (UML) diagrams, where UML is a standardised engineering modelling language specifically defined to capture and analyse various aspects of a software system\footnote{Although it is also used for other kind of design and engineering, beyond software.}, either in terms of its structure or behaviour.
\end{itemize}

\subsection{Procedural considerations}\label{ssect:SystemThinkingProceduralConsiderations}
The following considerations apply to \glsf{statistical-thinking}.

\paragraph{Scope and boundaries} Before you start your investigation you need to define clearly the scope of your \glsf{system} of interest, its boundaries and the purpose of your analysis, to inform your \glsplf{datum} generation and interaction with stakeholders. Your analysis should encompass different \glsf{system} dimensions, say social, cultural, economical or environmental. 

\paragraph{Stakeholders} You need to decide who you will involve in your research, focusing on stakeholders with an interest and understanding of the \glsf{system}, and ensuring that a plurality of views and perspectives are represented. If you aim for some form of intervention on an existing \glsf{system}, then you must also ensure you involve stakeholders who will champion or support it. In all cases, you should encourage collaboration and communication among stakeholders, including exchanging knowledge and ideas to foster a shared understanding of the \glsf{system}.

\paragraph{\Glsplf{datum} generation} To generate your \glsplf{datum} about the \glsf{system} you can apply any relevant method from~\Cref{ch:DataGenerationMethods}. Likely, you will need both qualitative and quantitative \glsplf{datum} as you are aiming for a comprehensive characterisation and analysis of your \glsf{system} from a plurality of perspectives.

\paragraph{Modelling} You need to decide which notations to use to \glsf{model} key aspects of the \glsf{system} relevant to your research. You should also consider whether a computational \glsf{simulation} would be appropriate, in which case, you should review the operational considerations related to \glsf{computational-thinking}\footnote{See~\Cref{ssect:ComputationalThinkingProceduralConsiderations}.}. For both, you will need to develop a certain level of expertise to be able to apply them effectively in your enquiry. This will require time, and ideally some expert guidance, which you must ensure are available within the constraints of your project.

\subsection{Other things to think about}\label{ssect:SystemThinkingOtherThingsTo}

For successful systemic modelling you should also consider the following issues.

\paragraph{Complexity} Analysing complex systems can be challenging, and taking a \glsf{systemic-thinking} approach is time consuming and resource intensive due to the need for large amount of \glsplf{datum} to gather and protracted interactions with many stakeholders. You need to ensure access to both \glsplf{datum} and people for your \glsf{systemic-thinking} enquiry to be meaningful.

\paragraph{Subjectivity and \glsf{bias}} As a systemic thinker, you establish the \glsf{system} boundaries, the perspectives to take, who to involve and how to gather and interpret your \glsplf{datum}. This leaves your research open to your own \glsf{bias}. Reflexive practices, alongside \glsf{triangulation}, say by cross-validating with independent \glsplf{datum}, are therefore necessary to support the \glsf{validity} of your findings.

\paragraph{\Glsf{model} \glsf{validity}} As for all modelling, your \glsf{system} characterisation will be based on simplifications and assumptions, which you will need to state explicitly and check carefully against the \glsf{system} being modelled. To validate both assumptions and resulting models you could apply \glsf{triangulation}, including asking independent experts to review them or compare \glsf{model} behaviour with real-world observations.  


\subsection{Further reading}\label{ssect:SystemThinkingFurtherReading}

\ReadingList{systemic-thinking}{simon2014systemic}


%\endinput
%\section{\Glsf{statistical-thinking}}\label{sect:StatisticalThinking}
%
%%\subsection{\Glsf{statistical-thinking} tools}\label{ssect:StatisticalThinkingTools}
%
%Like mathematics, \glsf{statistical-thinking} is a sophisticated discipline with much to offer the researcher. Statistics provides many tools of general applicability across the \glsf{research-process} and some form of \glsf{statistical-thinking} will apply to most quantitative research and in some qualitative research projects. To this end, we provide a separate section below on statistical modelling. 
%
%\ResGenTechnique{\glsf{statistical-thinking}}
%
%\subsection{Workflow}\label{ssect:Workflow}
%
%\subsection{Other things to think about}\label{ssect:OtherThingsTo}
%
%\ResGenExtras{\glsf{statistical-thinking}}{}

%\section{\Glsf{reflexivity}}\label{sect:reflexivity}
%
%\Glsf{reflexivity}\todo{Sources: \textcites{dodgson2019reflexivity, may2014reflexivity,patnaik2013reflexivity,palaganas2017reflexivity, darawsheh2014reflexivity,macbeth2001reflexivity}.} involves the researcher describing the relationships between themselves and research participants, and is crucial for increasing the credibility of findings and deepening understanding. Researchers should be explicit about their \glsplf{bias} and experiences as a means of demonstrating trustworthiness to readers. In addition, however, \glsf{reflexivity} also helps a researcher to recognise and take responsibility for their situatedness within the research, i.e., how they influence \glsplf{datum} collection and interpretation. The reflexive researcher should focus on self-knowledge, sensitivity, and understanding the role of the self for their knowledge contribution. 
%
%Introducing a **reflexive practice** into qualitative research allows researchers to examine the grounds of claims about the social world and explore the strengths and limitations of knowledge forms. This helps correct an instrumental approach to knowledge that seeks to control rather than understand the social world.
%
%\textcquote{dodgson2019reflexivity}{- **Qualitative research is contextual and occurs between two or more people in a specific time and place.**
%- **Researchers must describe the intersecting relationships between participants and themselves (\glsf{reflexivity}) to increase credibility and deepen understanding of the work.**
%- **\Glsf{reflexivity} is essential in qualitative research as the researcher's identity influences the findings.**
%- **Readers need to understand the researcher's positionality beyond their name and affiliations.**
%- **Researchers should address unconscious \glsplf{bias} and continually evaluate their positionality.**
%- **Participator process enhances non-exploitive processes and minimizes power differentials.**
%- **\Glsf{reflexivity} involves self-examination and understanding one's situatedness within the research.**
%- **Researchers must be aware of power differentials inherent in the researcher/participant relationship.**
%- **Describing contextual relationships between participants and researchers increases credibility and understanding of the work.**
%- **Researchers need to focus on self-knowledge, sensitivity, and understanding the role of the self in creating knowledge.**
%- **Researchers' position as an insider or outsider is crucial in considering similarities and differences with participants.**
%
%\Glsf{reflexivity} in qualitative research involves the researcher being conscious of their own \glsplf{bias}, positionality, and impact on the \glsf{research-process}. It requires continual self-reflection and transparency in addressing these factors throughout the research endeavor.
%
%- \Glsf{reflexivity} is a process that permeates the whole research endeavor.
%- Researchers need to address \glsf{reflexivity} in substantive ways to inform the reader about their processes.
%- The issues surrounding researchers' \glsf{reflexivity} are many and complex.
%- Power differentials between participants and researchers pose challenges related to \glsf{reflexivity}.
%- Researchers must be explicit about \glsf{reflexivity} and continually address trustworthiness criteria.
%- \Glsf{reflexivity} involves a continual internal dialogue and critical self-evaluation of the researcher's positionality.}
%
%\textcquote{may2014reflexivity}{
%- \Glsf{reflexivity} in qualitative research allows for examining the grounds of claims about the social world and exploring the strengths and limitations of knowledge.
%- **\Glsf{reflexivity} involves investigating the relationship between the knower and the known through inquiry itself.**
%- Calls for reflexive social inquiry challenge the separation between subject and object.
%- **\Glsf{reflexivity} enables researchers to correct an instrumental approach to knowledge that aims to control rather than understand the social world.**
%- The document discusses different social scientific approaches to \glsf{reflexivity} and their implications for research practices.
%- **Reflexive spaces are explored through various forms of qualitative work conducted over the years.**
%
%Workflow: - \Glsf{reflexivity} involves turning back on oneself in order that processes of knowledge production become the subject of investigation.
%- It is a way of thinking or critical ethos that aids interpretation, translation, and representation.
%- \Glsf{reflexivity} is not a method but a continuous characteristic of good research practice.
%- It includes endogenous and referential \glsf{reflexivity}, which focus on the actions and understandings of researchers and the meeting of points of view in the production and reception of accounts.
%
%Issues: - \Glsf{reflexivity} involves turning back on oneself in order that processes of knowledge production become the subject of investigation.
%- Endogenous \glsf{reflexivity} and referential \glsf{reflexivity} are two interrelated dimensions of reflexive practice.
%- A reflexive approach to analysis requires navigating between scientism and relativism, and deconstruction and reconstruction.
%- \Glsf{reflexivity} is not a method, but a critical ethos to aid interpretation, translation, and representation.}
%
%\textcquote{patnaik2013reflexivity}{
%
%- The paper explores **\glsf{reflexivity}** in social sciences for meaning making and knowledge claims.
%- **\Glsf{reflexivity}** is important for establishing credibility and richness in research.
%- **Introspective \glsf{reflexivity}** involves understanding how the researcher's experiences influence the research.
%- **\Glsf{reflexivity}** helps in monitoring assumptions, ethical considerations, and research rigour.
%- **Maintaining a reflective journal** helps in capturing the researcher's attitudes and \glsplf{bias}.
%- **\Glsf{bracketing}** is used to prevent the researcher's \glsplf{bias} from influencing the \glsf{research-process}.
%
%Workflow: - \Glsf{reflexivity} is essential in qualitative research to address the subjective nature of \glsplf{datum} interpretation and researcher \glsf{bias}.
%- It involves introspective \glsf{reflexivity} to understand how the researcher's experiences influence the \glsf{research-process}.
%- Epistemological \glsf{reflexivity} examines the knowledge claims being made and the researcher's role in shaping them.
%- Achieving \glsf{reflexivity} requires addressing personal history, values, and \glsplf{bias} that may impact the research.
%- Operationalizing \glsf{reflexivity} involves asking questions about the researcher's influence on the topic, \glsf{research-process}, and participant interactions.
%
%Issues: - The influence of the researcher's values and attitudes on the choice of topic
%- The exploration of epistemological foundations of knowledge claims
%- The role of the researcher in the process of knowledge construction
%- The presentation of \glsf{reflexivity} in research writing}
%
%\textcquote{palaganas2017reflexivity}{
%The document discusses \glsf{reflexivity} in qualitative research, highlighting the personal changes and influences experienced by researchers during \glsf{fieldwork}. It emphasizes how researchers are shaped by the \glsf{research-process} and how their involvement impacts the final study. The authors reflect on their positionality and personal backgrounds, acknowledging the importance of self-awareness in research.
%
%- The document discusses **\glsf{reflexivity} in qualitative research** and how it impacts researchers.
%- The researchers share their **journeys of learning** and how the \glsf{research-process} shaped them.
%- The paper focuses on understanding social \glsplf{phenomenon} like **poverty, development, gender, migration, and ill health** in the Philippines.
%- \Glsf{reflexivity} helps researchers become aware of their contribution to the construction of meanings and lived experiences.
%- **\Glsf{fieldwork}** is described as intensely personal, influenced by researchers' positionality and personal backgrounds.
%
%Workflow: - \Glsf{reflexivity} in research involves reflecting on how the researcher's values, beliefs, and perceptions shape the \glsf{research-process}.
%- It is important in feminist research to identify with the research participants and be aware of the power differentials that exist.
%- \Glsf{reflexivity} can take on various forms such as epistemological \glsf{reflexivity}, personal \glsf{reflexivity}, critical \glsf{reflexivity}, and feminist \glsf{reflexivity}.
%- Researchers need to consider how their own background and assumptions affect their research practice.
%- Practicing \glsf{reflexivity} is a significant component of qualitative research.
%
%Issues: - \Glsf{reflexivity} can disrupt the process of discovery
%- Researchers need to suspend the belief that a more personal story reveals a more authentic story
%- Sharing of ideas and experiences can become emotionally driven, leading to ethical issues
%- Researchers need to be aware of personal \glsplf{bias} and how they shape the \glsf{research-process}
%- Objectivity in research focuses on method rather than content
%- Researchers must maintain academic distance to avoid being influenced by emotions
%- \Glsf{reflexivity} challenges the assumption of studying social reality objectively
%- Researchers need to be critical of power differentials in research
%- Feminist \glsf{reflexivity} emphasizes reciprocity in research}
%
%\textcquote{darawsheh2014reflexivity}{AI: Qualitative research is contextual and occurs between two or more people in a specific time and place. **\Glsf{reflexivity}** in research, which involves the researcher describing the relationships between themselves and the participants, is crucial for increasing the credibility of findings and deepening understanding. Researchers need to be explicit about their positionality, \glsplf{bias}, and experiences to ensure trustworthiness. **\Glsf{reflexivity}** helps researchers recognize and take responsibility for their situatedness within the research, influencing \glsplf{datum} collection and interpretation. It is important for researchers to focus on self-knowledge, sensitivity, and understanding the role of the self in creating knowledge. The balance between personal experiences and universal knowledge must be maintained.
%
%- **Qualitative research is contextual and occurs between two or more people in a specific time and place.**
%- **Researchers must describe the intersecting relationships between participants and themselves (\glsf{reflexivity}) to increase credibility and deepen understanding of the work.**
%- **\Glsf{reflexivity} is essential in qualitative research as the researcher's identity influences the findings.**
%- **Readers need to understand the researcher's positionality beyond their name and affiliations.**
%- **Researchers should address unconscious \glsplf{bias} and continually evaluate their positionality.**
%- **Participator process enhances non-exploitive processes and minimizes power differentials.**
%- **\Glsf{reflexivity} involves self-examination and understanding one's situatedness within the research.**
%- **Researchers must be aware of power differentials inherent in the researcher/participant relationship.**
%- **Describing contextual relationships between participants and researchers increases credibility and understanding of the work.**
%- **Researchers need to focus on self-knowledge, sensitivity, and understanding the role of the self in creating knowledge.**
%- **Researchers' position as an insider or outsider is crucial in considering similarities and differences with participants.**
%
%\Glsf{reflexivity} in qualitative research involves the researcher being conscious of their own \glsplf{bias}, positionality, and impact on the \glsf{research-process}. It requires continual self-reflection and transparency in addressing these factors throughout the research endeavor.
%
%- \Glsf{reflexivity} is a process that permeates the whole research endeavor.
%- Researchers need to address \glsf{reflexivity} in substantive ways to inform the reader about their processes.
%- The issues surrounding researchers' \glsf{reflexivity} are many and complex.
%- Power differentials between participants and researchers pose challenges related to \glsf{reflexivity}.
%- Researchers must be explicit about \glsf{reflexivity} and continually address trustworthiness criteria.
%- \Glsf{reflexivity} involves a continual internal dialogue and critical self-evaluation of the researcher's positionality.
%}
%
%\textcquote{macbeth2001reflexivity}{**Summary:**
%The document discusses the concept of \glsf{reflexivity} in qualitative research, focusing on two main inflections: positional \glsf{reflexivity} and textual \glsf{reflexivity}. It also reintroduces Garfinkel's ethnomethodological \enquote{constitutive \glsf{reflexivity}} as an alternative perspective. The text explores the \glsf{diversity} of \glsf{reflexivity} in the literature and its role in deconstructing the intersections of author, other, text, and world. The document includes references to various authors and their works on \glsf{reflexivity} and qualitative research.
%
%- \Glsf{reflexivity} in qualitative research is a significant topic, with two main inflections: positional \glsf{reflexivity} and textual \glsf{reflexivity}.
%- **Positional \glsf{reflexivity}** involves examining the influence of place, biography, self, and others on the analysis.
%- **Textual \glsf{reflexivity}** focuses on disrupting the exercise of textual representation.
%- The article discusses the concept of **constitutive \glsf{reflexivity}** in social science, particularly Garfinkel's ethnomethodological approach.
%- **\Glsf{reflexivity}** is seen as a deconstructive exercise to understand the connections between author, text, and world.
%- The literature on \glsf{reflexivity} is diverse, with various perspectives and interpretations.
%- An example of constitutive \glsf{reflexivity} is analyzed through a videotaped sequence from a fifth-grade classroom.
%- The rush of interest in qualitative research has led to a broad consensus on the importance of \glsf{reflexivity}.
%- **Postmodern attachments** may influence the understanding of \glsf{reflexivity}, but it is suggested that there are commonalities with Enlightenment certainties.
%- **Garfinkel's work** in ethnomethodology is referenced as an alternative perspective on \glsf{reflexivity}.
%
%workflow: \Glsf{reflexivity} in qualitative research involves two main programs: positional \glsf{reflexivity} and textual \glsf{reflexivity}. Positional \glsf{reflexivity} focuses on examining how place, biography, self, and others shape the analytic exercise. Textual \glsf{reflexivity}, on the other hand, involves examining and disrupting the exercise of textual representation. Both programs aim to deconstruct and understand the intersections of author, other, text, and world in the \glsf{research-process}.
%
%issues: \Glsf{reflexivity} in qualitative research raises issues related to positional \glsf{reflexivity} and textual \glsf{reflexivity}, which focus on the impact of place, biography, self, and other on the analytic exercise, as well as the disruption of the exercise of textual representation. The discussion also introduces the concept of constitutive \glsf{reflexivity}, which dissolves binaries and representational language games into practical achievements of diverse settings and practices.}
%
%\ResGenTechnique{\glsf{reflexivity}}
%
%\subsection{\Glsf{reflexivity} tools}\label{ssect:ReflexivityTools}
%
%\subsection{Workflow}\label{ssect:Workflow}
%
%\subsection{Other things to think about}\label{ssect:OtherThingsTo}
%
%\ResGenExtras{\glsf{reflexivity}}{}


%\subsection{\Glsf{sampling}: what, who (and how) to choose}\label{sect:DG:sampling}\label{ssect:sampling}
%\todo{Section referenced, but commented out}
%
%A core prelude to many of the \glsplf{datum} generating techniques introduced above is to choose the \glsf{data-source}. You've already had experience of doing this when you conducted your literature search as part of Stage Cref{stage:?}\footnote{You might remember the relatively complex procedures for recording search terms, discovered papers, their relationships, and your growing collection of notes on them.}. \todo{More here} 
%
%Unless you had infinite amounts of time – which you don't – and infinite patience – which you might have – you could never be 100\% certain that your literature search collected \emph{all} relevant papers  the search space is infeasibly large (and not indexed particularly well). But you were systematic and achieved a practically good\footnote{By \emph{practically good}, we mean you found the most of the most important papers, some other papers, and didn't have to read \emph{every single paper}. I.e., you found a \emph{representative \glsf{sample}}.} coverage because of that.\todo{More here?}
%
%\Glsf{sampling} is the process of selecting a subset\footnote{We could have said \enquote{\glsf{sample}} but that would have been circular.} of an infeasibly large \glsf{population} of interest, and is used in the research strategies\todo{Update given~\Cref{stage3}.} that work by estimating or predicting the properties of that \glsf{population}. %, for instance, survey (who will fill in your questionnaire) or experimental (who will take part in your experiment) research. 
%
%\Glsf{sampling} can either be random or non-random. 
%
%In \emph{\glsf{random-sampling}}\footnote{\Glsf{random-sampling} is also called \emph{\glsf{probability-sampling}}.} some unbiased way of choosing, before the fact, subset members from the \glsf{population}, while in \emph{\glsf{non-random-sampling}}\footnote{\Glsf{non-random-sampling} is also called \emph{non-probability \glsf{sampling}}.}, the choice is based on a researcher's judgement and discretion and can be added to as the research progresses. The lack of \glsf{bias} in the former means that results tend to generalise from the \glsf{sample} to the \glsf{population}. The potential for \glsf{bias} in the latter means that results may not generalise, but things of interest, of depth, and of richness can be followed as they are discovered. As a result, the former is used more in quantitative research, and the latter in qualitative research\todo{Reword to separate the two?}. %Also, in quantitative research the tendency is to choose the \glsf{sample} upfront, before any analysis commences, while in qualitative research, the process is iterative, with more and more \glsf{sample} \glsplf{datum} collected and analysed until no more collection is possible or \textit{\glsf{saturation}} is reached, that is collecting more \glsplf{datum} would not bring more relevant information.
%
%\Glsf{random-sampling} techniques include:
%\begin{description}
%
%	\item [\glsf{simple-random-sampling}] where each member of the \glsf{population} has exactly the same chance to be selected. It is easy and efficient to implement, and given the complete randomness of the \glsf{sample}, \glsf{generalisation} is fairly reliable. However, if the \glsf{population} has large sub-groups, these may be over-represented in the \glsf{sample}, with minority groups being under-represented.\todo{Add strengths and weaknesses.}
%	
%	\item [\glsf{stratified-random-sampling}] where sub-groups of the \glsf{population} are identified based on common characteristics, the \textit{strata}, and \glsf{sampling} is random across those strata. The strata are not mutually exclusive: for instance, the \glsf{population} may have sub-groups defined by gender, ethnicity and level of education, which may overlap. This approach overcomes the over/under representation problem of \glsf{simple-random-sampling}.\todo{Add strengths and weaknesses.}
%	
%	\item [\glsf{cluster-sampling}] where the \glsf{population} is divided up in naturally occurring separate clusters, and the \glsf{sample} is obtained by randomly selecting some clusters and then randomly selecting members of those clusters. It is more cost-efficient than the other two approaches, but can introduce \glsf{bias} if the selected clusters are not representative of the whole \glsf{population}, so that the over/under representation problem remains.\todo{Add strengths and weaknesses.}
%\end{description}
%
%%%Candidate for moving to chapter preload
%
%\Glsf{non-random-sampling} techniques include:
%\begin{description}
%	\item [\glsf{purposive-sampling}] in which participants are selected by the researcher based on particular characteristics, knowledge, or expertise they have. It is often used for small populations, especially rare populations which may otherwise be difficult to access\todo{Clarify}. \Glsf{purposive-sampling} is particularly suited to studies which intend to be deep and narrow, and for which subsequent \glsf{generalisation} back to the parent \glsf{population} is not a concern. As the \glsf{sample} choice is made by the researcher, it is prone to \glsf{bias}.\todo{Add other strengths and weaknesses, or perhaps the weakness is with the class rathe than the instance?}
%
%	\item [\glsf{convenience-sampling}] where participants are selected based on their availability or accessibility. This is quick and easy, but unlikely to produce a representative \glsf{sample}, so, once again, \glsf{bias} is an issue.\todo{Add other strengths and weaknesses.}
%	
%	\item [\glsf{snowball-sampling}] which relies on referral from previous participants to recruit new ones. This is an effective approach when a \glsf{population} is difficult to access or when the topic is sensitive or tabu. This too is unlikely to generate a representative \glsf{sample}, and is prone to \glsf{bias}.\todo{Add other strengths and weaknesses.}
%\end{description}
%
%\begin{activity}[{Deep reading on \Glsf{non-random-sampling}}]
%Check back to your choice of \glsf{research-strategy}. If you've chosen one that uses \glsf{non-random-sampling}, then you should read the following sources for more details \cites{}	
%\end{activity}
%
%In summary, when choosing a \glsf{sample}, you need to consider various factors, including the aim of your study, the kind of methods you are applying, and the level of access you may have. Trade-offs are likely involved and you may not be able to obtain an ideal \glsf{sample}. Nevertheless, your \glsf{sample} will still be useful to your research, as long as you clearly explain and justify how it was obtained and what its limitations are.
%
%\begin{activity}[{Choosing your \glsf{sampling} approach}]
%Assuming your study requires you to perform some \glsf{sampling}, write down the approach you are going to take, with its justification in terms of what is needed to address your aim and objectives, and any trade-offs due to the practicality of accessing the \glsf{sample}. Record any possible weakness or limitation of your chosen approach.
%\begin{guidance}
%You can skip this activity if \glsf{sampling} is not indicated by your choice of \glsf{research-strategy}.
%\end{guidance}
%\end{activity}
%
%\todo{Add other core techniques here; which are there?}
%

\chapter{Analysis methods}\label{ch:AnalysisMethods}

%%%
\newcommand{\ResAnTechnique}[1]{\begin{activity}[{Do I need to know about #1?}]
Check back to your draft methodology from~\Cref{stage3}. Does it involve \glsplf{datum} analysis using #1? If so, read through the remainder of this section and complete the activities.
\end{activity}}
%%%

All research requires you to analyse \glsplf{datum}, so that you'll apply one or more \glsplf{datum} \glsplf{analysis-method} in your project. While there is a vast choice of methods for generating \glsplf{datum}, your choices for analysing them are more limited. Broadly speaking, \glsplf{analysis-method} fall into two main categories, quantitative and qualitative, depending on the nature of the \glsplf{datum} to analyse. This chapter introduces some of the most common \glsplf{analysis-method} in research: this introduction is neither complete nor very deep\footnote{Entire books have been written on any of them!}, but should give you a solid starting point for your project and references to sources you can access to learn more.

Before delving into common \glsplf{analysis-method}, you should consider techniques to summarise and present \glsplf{datum}, often a preparatory step for \glsplf{datum} analysis. In all cases, you will need some of these techniques to summarise and present \glsplf{datum} in your dissertation, so this will be time well spent.

\section{Summarising and presenting data}\label{sect:pesentingData}


\subsection{Using tables}\label{ssect:UsingTables}

Tables can be used to summarise and present both quantitative and \glsplf{quantitative-datum}, as a starting point for their analysis. The following kinds of tables are used extensively in research and often found in dissertations, hence they provide a good starting point.

\subsubsection{Pivot tables}\label{sssect:pivotTables}

\Glsplf{pivot-table} can be used to summarise, sort, filter, re-organise or group \glsplf{datum}. In a \glsf{pivot-table}, \glsplf{datum} are organised in rows and columns on which you can perform calculations, such as counting, generating totals or averages, and much more. Pivot tables are both powerful and versatile\footnote{In fact, they are so versatile that we'll only be able to provide few illustrative examples. Much, much more can be found online!}, and one of the most widespread tools to facilitate \glsplf{datum} analysis.

You can generate a \glsf{pivot-table} from any \glsplf{datum} set organised in rows and columns, regardless of whether the values are quantitative or qualitative: all common spreadsheet applications\footnote{From MS Excel to Apple Numbers to Google Sheets.} include this function.

%\todo{Jon to change style of pivot tables in example as there should be more than one header row and column}

\begin{example}{US housing market \glsplf{datum}}
\Cref{tab:exampledataset} gives the first few rows of Kaggle's housing price \glsplf{datum} set, a \glsplf{datum} set  related to the US housing market. Kaggle is possibly the largest and best known online community for \glsplf{datum} science and \glsf{machine-learning}, so that its \glsplf{datum} sets are often used in research and teaching.  

The housing price \glsplf{datum} set contains over 9,316 entries, each corresponding to a distinct US property, characterised by the following attributes: size in square feet, number of bedrooms and bathrooms, type of neighbourhood, the year it was built and its market price in US dollars. As you can see, the table includes both \glsdisp{numerical-variable}{numeric} and \glsdisp{non-numerical-variable}{non-numeric} variables.

\begin{SimpleNColTable}{tab:exampledataset}{7}{Example: the US housing market}[rccccrr]
	ID & Feet$^2$ & Beds & Baths & N.hood & Year & Price (\$)\\
	1 & 2126 & 4 & 1 & Rural & 1969 & 215,355\\
	2 & 2459 & 3 & 2 & Rural & 1980 & 195,014\\
	3 & 1860 & 2 & 1 & Suburb & 1970 & 306,891\\
	4 & 2294 & 2 & 1 & Urban & 1996 & 206,786\\
	5 & 2130 & 5 & 2 & Suburb & 2001 & 272,436\\
\end{SimpleNColTable}


Pivot tables can be used to summarise \glsplf{datum} to answer specific questions. For instance, you may ask what is the average house price by neighbourhood and number of bedrooms, which would result in the \glsf{pivot-table} in~\Cref{tab:pivot1}, which gives the \glsf{mean} price of each combination. From this you can analyse how the \glsf{mean} price is affected by each of the factor, for instance you can see how 5-bedroom houses in the suburbs tend to have the highest average price, while 2-bedroom houses in the same neighbourhood are the cheapest.

\begin{SimpleNColTable}{tab:pivot1}
	{5}
	[\narrowtablewidth]
	{\Glsf{pivot-table} of average property prices by neighbourhood and number of bedrooms}
	[Q[r]Q[r]Q[r]Q[r]Q[r]]
	[row{2,Z} = {bg=\frameColor,fg=white,font=\bfseries}]
	Neighbourhood & Rural & Suburb & Urban & \Glsf{mean} \\
	Beds & \SetCell[c=4]{c}{Price (\$)}\\
	2 & 218,323 & 216,300 & 220,050 & 215,355\\
	3 & 219,053 & 220,397 & 223,737 &  218,230\\
	4 & 227,774 & 224,609 & 230,086 & 221,057\\
	5 &  231,112 & 231,776 & 234,894 & 227,473\\
	\Glsf{mean} & 224,096 & 223,234 & 227,166 & 224,827\\
\end{SimpleNColTable}

Alternatively, you may be interested in how many properties of each kind have been built in each neighbourhood. In this case the resulting \glsf{pivot-table} would look like that in~\Cref{tab:pivot2}, which counts the number of house per combination of bedrooms/bathrooms and kind of neighbourhood in the \glsplf{datum} set.

%\begin{figure}[htbp]
%\centering
%\includegraphics[width=\textwidth]{figures/pivot2.pdf}
%\caption{\Glsf{pivot-table} of property counts by neighbourhood and number of bedrooms/bathrooms}
%\label{fig:pivot2}
%\end{figure}

\begin{SimpleNColTable}{tab:pivot2}{6}[\narrowtablewidth]{\Glsf{pivot-table} of property counts by neighbourhood and number of bedrooms/bathrooms}[rccccc][
	row{2,6,10,Z} = {bg=\frameColor,fg=white,font=\bfseries},
	column{2,Z} = {bg=\frameColor,fg=white,font=\bfseries},
	]
	&\SetCell[c=1]{r}Neighbourhood:  & Rural & Suburb & Urban & \\
	Beds & Baths &&&&Total \\
	2 & 1 & 180 & 220 & 261 & 661\\
	  & 2 & 129 & 214 & 283 & 626\\
	  & 3 & 155 & 45 & 75 & 257\\
	\SetCell[c=2]{r}2 bed totals: & 464 &479 & 619 & 1562 \\
%	\hline	
	3 & 1 & 96 & 130 & 129 & 355\\
	  & 2 & 397 & 149 & 245 & 791\\
	  & 3 & 127 & 139 & 889 & 1155\\
	\SetCell[c=2]{r}3 bed totals: & & 620 & 418 & 1263 & 2301 \\
%	\hline	
	4 & 1 & 142 & 306 & 228 & 676\\
	  & 2 & 453 & 379 & 410 & 1242\\
	  & 3 & 315 & 175 & 194 & 684\\
	\SetCell[c=2]{r}4 bed totals: & & 910 & 860 & 832 & 2602 \\
\end{SimpleNColTable}
\end{example}

%\end{document}

The example illustrates a couple of ways you can use pivot tables to address specific questions of your \glsplf{datum}, out of a vast range of the possibilities. If your \glsplf{datum} are organised in tables, then it is well worth spending some time becoming familiar with pivot tables.

\begin{activity}[{Pivot tables in Excel}] Download the housing price \glsplf{datum} set from Kaggle and re-create the pivot tables in our example. Come up with other questions you could ask of the \glsplf{datum} and generate related pivot tables.	
\begin{guidance}
%Hack to correct tcbox behaviour
\color{black}


%Hack to correct tcbox behaviour
\color{black}
Feel free to use your preferred spreadsheet application for this activity, as long as it supports pivot tables -- most do.

You may have to register with Kaggle to gain access to the \glsplf{datum} set. 

If you use Excel, its help facility and documentation provide all the info you need to create a \glsf{pivot-table}. In addition, you could browse some of the very many freely available online resources and tutorials on this topic.
\end{guidance}
\end{activity}	


\subsubsection{Frequency and contingency tables}\label{sssect:freqAndContTables}

\Glsplf{frequency-table} are used to summarise the frequency (or count) of values taken by a categorical variables in a \glsplf{datum} set, while \glsplf{contingency-table}\footnote{Also known as \glsf{cross-tabulation} tables.} extend frequency tables in order to tabulate the value frequencies of two categorical variables. 

\begin{example}{Frequency and contingency tables for degree classifications}
At the Open University, after studying a degree, a student's outcome is classed as one of distinction, merit, pass or fail. You could use a \glsf{frequency-table} to summarise the frequency of each class of outcome for a particular students' cohort, as shown in~\Cref{tab:frequencyTableExample}.
 
\begin{SimpleNColTable}{tab:frequencyTableExample}{5}[\narrowtablewidth]{Classification frequencies}[rcccc|]
	& Distinction & Merit & Pass & Fail \\
	Outcome & 12 & 26 & 42 & 5\\
\end{SimpleNColTable}

You could also use a \glsf{contingency-table} to tabulate the outcome value frequencies in the cohort against gender, as shown in~\Cref{tab:contingencyTableExample}. 


\begin{SimpleNColTable}{tab:contingencyTableExample}{4}[\narrowtablewidth]{Classification frequencies by gender} [rccc|]
	 & Female & Male & Other\\
	Distinction & 7 & 5 & 0\\
	Merit & 12 & 13 & 1\\
	Pass & 21 & 19 & 2\\
	Fail & 2 & 3 & 0\\
\end{SimpleNColTable}

where \enquote{Other} is used for students who don't associate with a binary gender or have declined to declare their gender. You can then use the table to investigate if there are significant differences in outcomes by gender.
\end{example}

Contingency tables are often used to summarise and analyse \glsplf{datum} collected in \glsf{survey-research}, and are a key tool in \glsf{statistical-analysis}.

Both frequency and contingency tables can be generated as pivot tables in a spreadsheet. In fact, the \glsf{pivot-table} in the example of \Cref{sssect:pivotTables} is a \glsf{contingency-table}. 

\subsubsection{Tabulating your data}\label{sssect:tabulatingData}

This is a good point to consider how you could tabulate your \glsplf{datum} for follow up analysis and presentation. 

\begin{activity}[{Creating your own tables}]
Consider how you could use different kinds of table to summarise your \glsplf{datum} both to help you in your follow-up \glsplf{datum} analysis and present your \glsplf{datum} in the body of your dissertation.
Apply your choices to your \glsplf{datum}. 
\begin{guidance}
%Hack to correct tcbox behaviour
\color{black}


%Hack to correct tcbox behaviour
\color{black}
You can start from the kind of tables included in this chapter. Should they not be suitable, you should conduct a web search to find examples of other kinds of table you may use.

In your table choices, you should consider:
\begin{itemize}
\item purpose of the table: you should decide what you intend to achieve by tabulating the \glsplf{datum}. For instance you may want to provide a concise summary or make it easier to analyse trends over time or relationships between \glsplf{datum}
\item  type of \glsplf{datum}: as you have seen in this section, some forms of tabulation may be more appropriate for quantitative rather than \glsplf{quantitative-datum}, so you need to think of the kind of \glsplf{datum} you want to tabulate
\item table design: you should choose rows and columns carefully to meet the purpose of the table, including providing clear and concise headings which appropriately describe the content of each cell
\item  consistency: for numerical values, ensure you use consistent units of measurement and level of precisions. If the same \glsplf{datum} are included in several tables ensure you use the same headings
\item highlight significant entries: you can use colours or other visual cues to draw attention to significant entries.
\end{itemize}
Your table won't be perfect at this point, but should provide a good starting point to analyse your \glsplf{datum}.
\end{guidance}
\end{activity}
%%Hack to correct tcbox behaviour
\color{black}

\subsection{Summarising qualitative data}\label{ssect:summarisingQuantitativeData}

\Glsplf{qualitative-datum} are heterogeneous in nature and cannot be easily set out in a standard manner.

Conveying the depth and richness of \glsplf{qualitative-datum} in a succinct way is challenging, so that you will need both selectivity and creativity in presenting your \glsplf{datum}. 

For textual \glsplf{datum}, like \glsf{interview} transcripts, \emph{verbatim} quotations are often used to illustrate specific themes or points, or support certain conclusions. However, an excessive use of quotations will result in overlong accounts, which may be difficult to follow or may even obscure the main findings. Therefore it is important that you select quotations which are particularly representative or poignant, avoiding verbose details that you can present more succinctly in the narrative surrounding those quotations.

For all kinds of \glsplf{qualitative-datum}, diagrams, schematics and drawings can also be used effectively and imaginatively for presenting them and to facilitate their analysis. \Glsplf{datum} visualisation is, in fact, a discipline in its own right\footnote{Edward Tufte is one of the most influential figures in this field. His books provide compelling examples on how to use visualisation to present and analyse highly complex \glsplf{datum}.}, with some visualisation techniques also applicable to \glsplf{qualitative-datum}.

\begin{activity}[{Visualisation techniques for \glsplf{qualitative-datum}}] 
Conduct a web search on techniques for visualising \glsplf{qualitative-datum}. List the techniques you have found and what they are used for. Consider whether they may be applicable in your project.
\begin{solution}
You may have encountered some or all of the following:
\begin{itemize}
	\item diagrams and schematics, to convey complex processes or structures
	\item graphic timelines, to summarise key events and their order
	\item word clouds, to summarise emerging themes or concepts from text, and their relative frequencies
	\item mind maps, to visualise how different ideas relate or contribute to a central concept or topic
	\item heat maps, to highlight trends or differences in tabulated \glsplf{datum}
	\item icons, alongside brief descriptions, to represent and quickly identify specific concepts 
	\item storyboards, to visualise narratives or sequencing of actions or events
	\item bespoke drawings, for \glsplf{datum} which cannot be easily visualised using other standard techniques
	\item pie charts and bar charts, to summarise proportions and counts – which are actually quantitative, but may be the result of \glsplf{qualitative-datum} analysis – of \glsplf{categorical-datum}.
\end{itemize}
\end{solution}
\end{activity}
%%Hack to correct tcbox behaviour
\color{black}

If you have concluded that visualisation may be a good way to present your \glsplf{qualitative-datum}, then this is a good time to have a go.

\begin{activity}[{Creating your own visualistations}]
Apply your chosen \glsplf{visualisation-technique} to your \glsplf{quantitative-datum}.
\begin{guidance}
%Hack to correct tcbox behaviour
\color{black}


%Hack to correct tcbox behaviour
\color{black}
Your visualisations should be:
\begin{itemize}
\item fit-for-purpose in relation to what you intend to convey
\item readable and accessible to a variety of readers with diverse abilities
\item accurate and honest in representing your \glsplf{datum}, so that your readers are not misled or likely to misinterpret your \glsplf{datum}.
\end{itemize}
\end{guidance}
\end{activity}
%%Hack to correct tcbox behaviour
\color{black}



\section{Statistical analysis for quantitative data}\label{sect:statisticalAnalysis}

\Glsf{statistical-analysis} is an umbrella term for a set of methods which you can apply to numerical and \glsplf{categorical-datum}. More precisely, in statistics, \glsplf{datum} are classified as:

\begin{itemize}
	\item \glsforce{scalar-datum}{scalar}, which includes all measurements and counts; with reference to the types in~\Cref{sect:evidenceAndData}, these are all \glsplf{numerical-datum}, continuous, discrete, interval and \glsplf{ratio-datum}; and
	\item \glsforce{categorical-datum}{categorical}, both ordinal and nominal.
	\end{itemize}

There are two broad categories of \glsplf{statistical-method}:
\begin{itemize}
	\item \glsf{descriptive-statistics}, whose aim is to describe \glsplf{datum}; and
	\item \glsf{inferential-statistics}, whose aim is to make predictions from \glsplf{datum}.
\end{itemize}


\subsection{Descriptive statistics} %\label{sect:descriptive} -- old
\label{ssect:DescriptiveStatistics}

\Glsf{descriptive-statistics} are used to describe various attributes of a \glsplf{datum} set. 

\ResAnTechnique{\glsf{descriptive-statistics}}

They include:
\begin{itemize}
	\item count, to establish how many entries there are in the \glsplf{datum} set
	\item \glsf{centrality}, to establish the \enquote{centre} of the \glsplf{datum} set. Three measures are commonly used: the \glsf{mean}, which provides the average value of the \glsplf{datum} set; the \glsf{median}, which provides its mid point\footnote{Remember that \glsplf{quantitative-datum} can be ordered.}; and the \glsf{mode}, which indicates the value that occurs most frequently, if any\footnote{There is no \glsf{mode} if no value is repeated in the \glsplf{datum} set.} 
	\item \glsf{dispersion}, to establish the \glsf{spread} of the \glsplf{datum} in the \glsplf{datum} set. There are two common measures: the \glsf{range}, which is the difference between smallest (minimum) and largest (maximum) values; and the \glsf{standard-deviation}, which is derived from the distance of each value in the \glsplf{datum} set from the \glsf{mean} through a mathematical formula\footnote{It is not essential for you to know such formula, which is automatically computed by spreadsheets and statistical software. Of course, you can always look it up in the literature...}. The larger the \glsf{standard-deviation}, the greater the dispersion
	\item \glsf{skewness}, to establish how symmetrically distributed the values in the data set are in relation to the centre, or, in other words, whether the distribution is somehow lopsided. In the case of perfect symmetry, \glsf{skewness} is equal to zero, and \glsf{mean} and \glsf{median} are equal. A perfectly symmetric distribution is usually referred to as a \glsf{normal-distribution} or \glsf{bell-curve}, from the shape of the line that you can obtain by plotting the \glsplf{datum} on a chart\footnote{This oversimplifies the topic in order to give you some intuition in case you have not come across these concepts before. A lot more could be said about the \glsf{normal-distribution} and its pivotal role in Statistics!}. When he distribution is asymmetric, \glsf{mean} and \glsf{median} are different. In particular, the distribution is right-skewed (positive \glsf{skewness}, with the `bell' having a longer tail towards large values), when the \glsf{mean} is greater than the \glsf{median}. Vice-versa, the distribution is left-skewed (negative \glsf{skewness}, with the `bell' having a longer tail towards small values), when the \glsf{mean} is smaller than the \glsf{median}).  
\end{itemize}

Not all \glsf{descriptive-statistics} apply to \glsplf{categorical-datum}. In particular, the \glsf{mode} is used as the main measure of centrality for \glsplf{nominal-datum}, while the \glsf{median} is used for \glsplf{ordinal-datum} which are not numeric.

These are lots of definitions to digest, particularly if you haven't encountered these terms before! The following activity should help.

\begin{activity}[{\Glsf{descriptive-statistics} in Excel}] Assume you have measured the weight in grams of each apple in a basket, obtaining the following numbers: 
%
\begin{displayquote}105, 120, 122, 125, 127, 128, 129, 130, 132, 133, 135, 135, 138, 140, 128\end{displayquote} 
%
Enter these values in an Excel sheet and use its built-in \glsplf{datum} analysis function to generate the related \glsf{descriptive-statistics}.
\begin{guidance}
%Hack to correct tcbox behaviour
\color{black}


%Hack to correct tcbox behaviour
\color{black}
In the version of Excel used for this activity, you can access this function from the \Glsplf{datum} tab, by pressing the \Glsplf{datum} Analysis button. If you find it difficult to locate this function, you should refer to the documentation or to some of the many tutorials on this topic which are freely available online.
\end{guidance}
\begin{solution}
You should have obtained the following values:
%
\begin{SimpleNColTable}*{tab:values}{2}{Values}[{rS[table-format=3.2]}]
{{{Attribute}}} & {{{Value}}} \\%%Triple {{{}}} on page 51 of tabularray manual
\glsf{mean} & 128.47 \\
\glsf{median} & 129 \\
\glsf{mode} & 128 \\
\glsf{standard-deviation} & 8.55 \\
\glsf{skewness} & -1.4 \\
\glsf{range} & 35 \\
minimum & 105 \\
maximum & 140 \\
count & 15 \\
\end{SimpleNColTable}

There are 15 entries in this \glsplf{datum} set (count), with \glsf{range} 35 -- the difference between maximum (140) and minimum (105) values.

In terms of centrality, the \glsf{mean} (128.47) is slightly smaller than the \glsf{median} (129). This is consistent with a negative \glsf{skewness}: the \glsf{mean} is smaller than \glsf{median}, so the \glsplf{datum} distribution is (slightly) left skewed.

Excel reports a \glsf{mode} at 128. In reality, if you look at the \glsplf{datum} you will see that there are two modes in this \glsplf{datum} set\footnote{Statisticians call this \textit{bi-modal}.}, with values 128 and 135, but Excel only returns the first encountered!

In terms of dispersion, the \glsf{standard-deviation} is telling us that most apple weights are within 8.55 grams of the \glsf{mean} (below or above), so the apple weights are similar in the apple baskets.
\end{solution}\end{activity}
%%Hack to correct tcbox behaviour
\color{black}

In your dissertation, you can easily present the \glsf{descriptive-statistics} of your \glsplf{datum} sets as a table, possibly adapting that automatically generated by your spreadsheet.

In addition, you can use the following standard charts to visualise the \glsplf{datum} and examine their \glsf{descriptive-statistics}. 

\paragraph{Histograms}
With \glsplf{scalar-datum}, like in the previous activity, you can use a \glsf{histogram}. The one in~\Cref{fig:histogramBin1} uses the apple weights from the activity: on the horizontal axis, you will find the distinct weights, and on the vertical axis, their frequencies, that is how many times each weight appears in the \glsplf{datum} set. Given the values you have obtained for the \glsf{descriptive-statistics} of the \glsplf{datum} set, you can easily locate on the chart min and max values, and \glsf{mean}, \glsf{median} and \glsf{mode}. In this case, the two \enquote{peaks} correspond to the two modes mentioned in a previous activity. You can also check that most of the values are within 8.55 grams from the \glsf{mean}, either way: the only values left out are 105 (to the left) and 138 and 140 (to the right). \Glsf{skewness} is not obviously notable on this chart: below you will use a different chart for \glsf{skewness}.

\begin{figure}[htbp]
\centering
\includegraphics[width=\textwidth]{Figures/Apple_Weights_BarChart}
\caption{\Glsf{histogram} for the apple weights (\glsf{bin-size} = 1)}
\label{fig:histogramBin1}
\end{figure}

It is worth noticing that given this is a small \glsplf{datum} set of discrete values, the \glsf{histogram} plots each individual apple weight. To do so it uses bins of size equal to one, where a \glsf{bin} is a way to group a number of values, with its \glsf{bin-size} establishing the \glsf{spread} of the \glsf{bin}. Frequencies are then calculated by \glsf{bin}. Grouping values in bins is necessary with large \glsplf{datum} sets and/or with continuous \glsplf{datum}. ~\Cref{fig:histogramBin5} illustrates a \glsf{histogram} for the apple weights, in which the \glsf{bin-size} is five: that is, each \glsf{bin} spans a set of five possible values. 

\begin{figure}[htbp]
\centering
\includegraphics[width=12cm]{Figures/Apple_Weights_histogram_BinSize5}
\caption{\Gls{histogram} for the apple weights (\gls{bin-size} = 5)}
\label{fig:histogramBin5}
\end{figure}

\paragraph{Boxplots}
In order to visualise both \glsf{spread} and \glsf{skewness}, you can use a \glsf{boxplot}, illustrated in~\Cref{fig:boxplot} for the apple weights. This is made of a \enquote{box} around the \glsf{median} of the \glsplf{datum}, and some \enquote{whiskers} on each side of the box\footnote{Which is why this chart is also called a \glsf{box-and-whiskers-plot}.}. It is obtained by dividing up the \glsplf{datum} into quartiles, each containing a quarter (or 25\%) of the \glsplf{datum}, with the \glsf{median} in the centre. The box includes the two quartiles on each side of the \glsf{median}, which, together, account for half of the values in the \glsplf{datum} set. The whiskers account for the other two quartiles, with a caveat: if there are very extreme values, these are treated as possible outliers and left out of the whiskers. This is, in fact, the case in our example where value 105 is treated as an outlier in the chart: it is a dot on its own, not included in the left whisker. The whisker length,  provides an indication of \glsf{spread}: the longer the whiskers, the more \glsf{spread} out the \glsplf{datum}. Instead, the position of the \glsf{median} in relation to the extreme of the box provides an indication of \glsf{skewness}, although possible outliers must also be considered. In the example the \glsf{median} is not in the centre of the box, indicating that there is some skewness; by taking the outlier into consideration, there is slight left-skewness (consistent with the negative \glsf{skewness} value in the \glsf{descriptive-statistics}). Note that without the outlier, the distribution would be right-skewed! You can check this by removing the outlier for the data set and re-calculating the descriptive statistics.

\begin{figure}[htbp]
\centering
\includegraphics[width=0.5\textheight]{Figures/Apple_Weights_Boxplot}
\caption{\Glsf{boxplot} for the apple weights}
\label{fig:boxplot}
\end{figure}

To be more precise, the relation between a \glsf{boxplot} and its underlying statistical features is illustrated in~\Cref{fig:boxplotFeatures}. The two quartiles around the \glsf{median} represent the interquartile \glsf{range} (IQR) of the \glsplf{datum} set. The whisker lengths, calculated based on the formulae in the figure,  allows the identification of lower and upper bounds beyond which values are seen as extreme and represented separately as outliers. An outlier, therefore, is just a value which is distant from most of the other values in the \glsplf{datum} set: it may point to an error, which should be corrected, or an anomaly, which may require further investigation, but that's not necessarily the case. However, it's good practice to investigate all outliers to understand why they have occurred. 

\begin{figure}[htbp]
\centering
\includegraphics[width=0.65\textwidth]{Figures/boxplotFeatures}
\caption{The elements of a \glsf{boxplot}}
\label{fig:boxplotFeatures}
\end{figure}


\begin{activity}[{Charts in Excel}] Go back to your Excel sheet from the previous activities and generate charts similar to those in the figures above.
\begin{guidance}
%Hack to correct tcbox behaviour
\color{black}

In the version of Excel used in this activity\footnote{Version 16.71.}, you can generate these charts from the Insert tab, by choosing from the Statistical charts menu. If you find it difficult to locate this function, you should refer to the documentation or check some of the many tutorials on this topic which are freely available online.
\end{guidance}
\end{activity}
%%Hack to correct tcbox behaviour
\color{black}

\paragraph{Commonly used charts}
~\Cref{tab:charts} summarises some of the most common charts used to visualise \glsplf{datum} and their \glsf{descriptive-statistics}. While they are not all illustrated in this chapter, they are so common that you can find plenty of study materials online should you wish to look them up.

\begin{SimpleNColTable}{tab:charts}{3}{Common charts to visualise \glsplf{datum} sets and their \glsf{descriptive-statistics}}[rcX[1]]
Chart & \Glsf{variable}(s) & Purpose \\
bar chart & one categorical & to visualise counts/frequencies/proportions/percentages \\
stacked bar chart & two categorical & to compare   counts/frequencies/proportions/percentages between two groups\\
\glsf{histogram} & one scalar & to visualise distribution, including centrality, dispersion and \glsf{skewness} \\
scatter diagram & two scalar & to visualise relationships and possible outliers \\
\glsf{boxplot} & one scalar or one categorical & to visualise \glsf{spread}, \glsf{skewness}, \glsf{median}, IQR and possible outliers \\
line chart & one scalar by time & to visualise change over time \\
\end{SimpleNColTable}


Calculating \glsf{descriptive-statistics} and visualising \glsplf{datum} in appropriate charts, should be the first step in your \glsplf{statistical-datum} analysis, as these provide useful summaries and visualisations of key properties of your \glsplf{datum} set. And, as you have found out in the activities, you do not need to be a statistician to be able to generate them!

\Glsf{descriptive-statistics} may also help you identify errors or anomalies in your \glsplf{datum}, and can inform possible follow-up analysis, including inferential \glsf{statistical-analysis}. Depending on your \glsf{research-aim} and objectives, they could also be all you need in your project. 

\subsection{Descriptive statistics applied to your data}\label{ssect:yourDescriptiveStatistics}

If you have collected scalar or \glsplf{categorical-datum}, it is time for you to have a go at analysing them using \glsf{descriptive-statistics} and charts.

\begin{activity}[{Applying \glsf{descriptive-statistics} to your \glsplf{datum}}] 
If your data set includes scalar or \glsplf{categorical-datum}, calculate their \glsf{descriptive-statistics} and generate appropriate charts.
\begin{guidance}
%Hack to correct tcbox behaviour
\color{black}


%Hack to correct tcbox behaviour
\color{black}
MS Excel is relatively straightforward to use for this purpose, but feel free to use other tools you may be already familiar with, including statistical or \glsplf{datum} analytics packages. Whichever tool you use, you should ensure it supports the functionalities discussed in this section.
\end{guidance}
\end{activity}
%%Hack to correct tcbox behaviour
\color{black}

\subsection{Further reading}\label{ssect:descriptiveStatistics}
%
\ReadingList{descriptive statistics}{cooksey2020descriptive}

\subsection{Inferential statistics}\label{sect:inferentialStatistics}\label{ssect:InferentialStatistics}

\Glsf{inferential-statistics} relies on the concepts of \glsf{population} and \glsf{sample}: the \glsf{population} is the entire group you are interested in studying -- say, all UK voters in a general election -- while the \glsf{sample} is the portion or subset of that group you have access to in your research. Then the aim of \glsf{inferential-statistics} is to establish whether patterns or effects you have observed in the \glsf{sample} can be generalised to, i.e., inferred for, the whole \glsf{population}, or whether they are the result of chance. In \glsf{inferential-statistics} this is achieved through statistical tests.

\ResAnTechnique{Inferential statistics}

You use a statistical test to find out whether a \glsf{proposition} is likely to be true in the \glsf{population} you are studying. You can think of a \glsf{proposition} as an educated guess you have made based on some observations, but that has yet to be supported by \glsf{evidence}. To test a \glsf{proposition}, you need a representative \glsf{sample} of the \glsf{population}\footnote{\Glsf{sampling} was discussed in~\Cref{sect:sampling}}.

You apply a statistical test to the \glsf{sample} and to provide \glsf{evidence} (or otherwise) that any pattern or effect you have observed in your \glsf{sample} is also likely to exist in the \glsf{population}, and is not just the effect of chance. Such evidence takes the form of a measure of \glsf{statistical-significance}, which indicates whether the effect is likely to exist in the \glsf{population}. Note that his is different from the common language meaning of significance as big or important: the effect in the population may well be small or unimportant, even if statistically significant.

As a corollary, if your \glsf{sample} is very large, almost all effects observed in the \glsf{sample} will be likely present in the \glsf{population}; vice-versa, if your \glsf{sample} is very small, most effects observed in the \glsf{sample} are unlikely to be present in the \glsf{population}, unless they are really very large effects. 

As a rule of thumb, most tests require a \glsf{sample} size of at least 30 observations, but more precise \glsf{sample} size estimates can be made based on \glsf{population} size and expected \glsf{significance} level\footnote{In the literature, you can easily find formulae for the ideal \glsf{sample} size.}.

You can apply statistical tests to both scalar and \glsplf{categorical-datum} and use them to compare values of specific statistics or to establish statistical relationships between variables, specifically:
\begin{itemize}
	\item an \glsf{association} between variables means that one \glsf{variable} can be used to provide some information about the other
	\item a \glsf{correlation} is a particular type of \glsf{association} in which the two associated variables always change together, for instance they both increase or decrease at the same time, or when one increases the other always decreases (or \emph{vice versa}). 
\end{itemize}

Statistical tests can be used to estimate the strength of an \glsf{association} (i.e., the extent changes in one \glsf{variable} correspond to changes in the other) and its direction (whether the variables change in the same or opposite ways).

Each statistical test comprises the following elements:

\begin{itemize}
	\item \glsplf{hypothesis}: there are two, the \glsforce{null-hypothesis}{null} and \glsforce{alternative-hypothesis}{alternative} \glsplf{hypothesis}. \Glsf{inferential-statistics} assumes you can't prove something to be true, but you can disprove something by finding an exception. Here is a classic example: you can't prove that all swans are white, but you can disprove they are by finding a black swan! So, you must  set the \glsf{null-hypothesis} to what you want to disprove about the \glsf{population}, with the \glsf{alternative-hypothesis} being what you are really interested in finding out. So, the \glsf{null-hypothesis} is usually a statement of no pattern/effect in the \glsf{population}
	\item \glsf{significance}: this is the level of \glsf{statistical-significance} for the test. It's known as the \textit{alpha} ($\alpha$) value from the Greek name of the mathematical \glsf{variable} used to express it. Most tests are run with $\alpha=0.05$, which gives a 5\% probability that you may infer that the \glsf{null-hypothesis} is disproved while in actuality it is correct\footnote{This is called a Type I error in Statistics.}
	\item \glsf{sample}(s): you need to have one or more (representative) samples of the \glsf{population} of interest  on which to perform the test. Multiple samples are used in some tests, typically to compare specific statistics in different groups within the \glsf{population} or changes within a group over time or after an intervention of interest, say treating patients with a new pharmacological drug
	\item \glsf{p-value}: this is the probability calculated for your test by your statistical package, and which is used to decide the outcome of the test
	\item \glsf{decision}: this is based on the \glsf{p-value} in relation to the $\alpha$ value. If the \glsf{p-value} is less than the $\alpha$ value, then the \glsf{null-hypothesis} is {rejected}, i.e. disproved, which means your \glsf{alternative-hypothesis} that there is an effect in the \glsf{population} is supported by statistical \glsf{evidence}.
\end{itemize}

There are very many statistical tests to choose from, depending on the kind of \glsplf{datum} you have and their distribution, the purpose of your analysis and the number of samples involved.  This introductory section does not detail all possible statistical tests --- once again, entire books have been written about them! Instead, it provides~\Cref{tab:testsForComparison} and~\Cref{tab:testsForAssociation} as summaries of the most common tests that you can then follow up in the literature, should you wish to apply any in your research.

%{\color{red} Bring the examples our of the table???}

\begin{SimpleNColTable}{tab:testsForComparison}{6}[\widetablewidth]{Common statistical tests for comparison. \emph{Parametric} tests apply to normally distributed \glsplf{datum} (see~\Cref{ssect:DescriptiveStatistics}); \emph{non parametric} tests to skewed distributions}[X[4,l]X[2,c]X[4,l]X[2,l]X[2,l]X[4,l]]
	Purpose & Variables & Example & Parametric & Non-parametric & Notes  \\
to compare the \glsf{sample} \glsf{mean} against a specific value & one scalar & to investigate whether AA batteries of a particular brand have the claimed lifespan  & one \glsf{sample} t-test & n/a  & \\
to compare the \glsf{sample} proportion against a specific value & one categorical & to investigate the proportion of people who voted for a particular party in a city against that for the whole country & one \glsf{sample} z-test & n/a  & \\
to compare the means of two independent samples & scalar & to compare the \glsf{mean} scores (dependent) of students studying the same subject with two different teaching approaches (explanatory) & independent t-test & Mann-Whitney test/Wil\-coxon rank sum  & two samples are \textit{independent} when there is no reason to believe that observations in one \glsf{sample} are influenced or determined by those in the other\\
to compare the means of three or more independent samples & scalar dependent; nominal explanatory & to compare the \glsf{mean} scores (dependent \glsf{variable}) of students studying the same subject with three or more different teaching approaches (explanatory \glsf{variable}) & one-way ANOVA & Kruskal-Wallis test \\

to compare the average difference between paired samples against a particular value & scalar dependent; time or condition as explanatory & to compare the blood pressure readings (dependent \glsf{variable}) of a group of people before and after exercising (explanatory \glsf{variable}) & paired t-test & Wilcoxon signed
rank test &  in paired samples each \glsplf{datum} point in one \glsf{sample} is uniquely matched to a \glsplf{datum} point in the other \glsf{sample}; this happens, for instance, when you measure a factor before and after an intervention, or take different readings for the same group of individuals. Because of this, paired samples are not independent\\
\end{SimpleNColTable}

\begin{SimpleNColTable}{tab:testsForAssociation}{6}[\widetablewidth]{Common statistical tests for \glsf{association}. \textit{Parametric} tests apply to normally distributed \glsplf{datum}; \textit{non parametric} tests to skewed distributions}[X[4,l]X[2,c]X[4,l]X[2,l]X[2,l]X[4,l]]
Purpose & Variables & Example & Parametric & Non parametric & Notes  \\
to investigate \glsf{correlation} between two continuous variables & scalar dependent and explanatory & to investigate the relation between blood pressure (dependent) and age (explanatory)  & Pearson’s \Glsf{correlation} Coefficient & Spearman’s \Glsf{correlation} Coefficient & \\

to investigate \glsf{association} between two categorical variables & categorical dependent and explanatory & to find out if there are gender (categorical) differences in the choice of modes of transport (categorical) in a city & chi-squared & n/a  &  \\

to investigate \glsf{association} between two categorical variables when the \glsf{sample} is small & categorical dependent and explanatory & to find out if there are gender (categorical) differences in the choice of modes of transport (categorical) in a city  & Fisher's Exact Test & n/a & the \glsf{sample} size $n$ should be less than 20 \\

to predict the value of one \glsf{variable} from that of one or more other variables & scalar dependent and any kind of explanatory & to predict house prices (dependent) based on location (explanatory, categorical) and number of bedrooms (explanatory, scalar) & linear regression & n/a & linear regression relies on \glsplf{association} between dependent and explanatory variables\\

to predict the value of a binary \glsf{variable} from that of two or more other variables & binary categorical dependent and any kind of explanatory & to predict whether a customer is likely or not to purchase a certain product (dependent) based on previous purchased products (explanatory, categorical) and average annual spent (explanatory, scalar) & logistic regression & n/a & a binary \glsf{variable} has only two possible values, so that logistic regression calculates the probability of each value based on the values of the explanatory variables. Because of this logistic regression can be used as a classification method \\
\end{SimpleNColTable}

Even if these tests are only a sub-set of all statistical tests available, there is a lot to digest. The next activity should help you use these tables to choose an appropriate test. 

\begin{activity}[{Choosing an appropriate test}] 
Consider each of the following scenarios and use the information in the tables to decide which test to apply and what the \glsf{null-hypothesis} should be. For each, write down your reasoning, choice and \glsf{null-hypothesis}.

\begin{description}
	\item [Scenario 1:] to investigate the amount of sugar contained in baby food of a particular brand against a recommended threshold, from a \glsf{sample} of 30 products of that brand.
	\item [Scenario 2:] to investigate the number of products per hour of two manufacturing machines in the same plant, by observing the two machines' output over 24 hours. 
	\item [Scenario 3:] to investigate the effect of temperature on the consumption of ice cream in a particular city over 12 months.
	\item [Scenario 4:] to investigate whether preference in chocolate types, say white vs milk vs dark, is related to gender in a particular country.
	\end{description}
\begin{guidance}
%Hack to correct tcbox behaviour
\color{black}


%Hack to correct tcbox behaviour
\color{black}
To simplify things, always assume normal distributions.
\end{guidance}
\begin{solution}
Assuming \glsplf{normal-distribution}, for each scenario, you should consider:
\begin{itemize}
	\item the kind of \glsplf{datum}
	\item number of samples and their size
	\item purpose of the investigation
\end{itemize}
From this you could conclude:
\begin{description}
	\item [Scenario 1:] scalar \glsf{variable} (amount of sugar); one \glsf{sample} of 30 products; to compare the \glsf{sample} \glsf{mean} against the recommended threshold. The test to use is a one \glsf{sample} t-test with \glsf{null-hypothesis} that the \glsf{sample} \glsf{mean} is above the threshold.
	\item [Scenario 2:]  scalar \glsf{variable} (number of products per hour); two samples (one per machine over the time span); to compare the means of products per hours for the two machines; there is no reason to think that the working of one machine may influence that of the other, so the two samples are independent. The test to use is an independent t-test with \glsf{null-hypothesis} that the two \glsf{sample} means are different.
	\item [Scenario 3:] scalar dependent (level of ice cream consumption) and scalar explanatory (temperature); to investigate any relationship between the two variables from one \glsf{sample} over the period. The test to apply is Pearson's \glsf{correlation} with \glsf{null-hypothesis} that there is no \glsf{association} between the two variables.  If, in addition, you wanted to make predictions on ice cream consumption based on temperature, then you could also apply linear regression.
	\item [Scenario 4:] both dependent (chocolate type preference) and explanatory (gender) variables are categorical; to investigate \glsf{association} based on one \glsf{sample} from a particular country. The test to use is a Chi-squared with \glsf{null-hypothesis} that gender has no \glsf{association} with chocolate taste.
\end{description}
\end{solution}
\end{activity}
%%Hack to correct tcbox behaviour
\color{black}

\subsection{Inferential statistics applied to your data}\label{ssect:yourDescriptiveStatistics}

It's time to apply \glsf{inferential-statistics} in your project.

\begin{activity}[{Applying \glsf{inferential-statistics} to your \glsplf{datum}}] 
Apply to your \glsplf{datum} any statistical test required by your methodology.
\begin{guidance}
%Hack to correct tcbox behaviour
\color{black}


%Hack to correct tcbox behaviour
\color{black}
You should make use of a statistical package to perform your test(s). In documenting each test, you should include:
\begin{itemize}
	\item the kind of test performed
	\item how your \glsplf{datum} met its conditions
	\item the \glsplf{hypothesis}, \glsf{sample}(s) and level of \glsf{significance}
	\item the outcome, in terms of \glsf{p-value} and \glsf{decision}
	\item how the outcome relates to your aim and objectives
\end{itemize} 
\end{guidance}
\end{activity}
%%Hack to correct tcbox behaviour
\color{black}

\subsection{Further reading}\label{ssect:inferentialStatistics}
%
\ReadingList{Inferential statistics}{casella2024statistical}


\section{Qualitative analysis}\label{sect:QualitativeAnalysis}

Qualitative analysis is used to extract meaning and insights from non \glsplf{numerical-datum}, be that text, images, audio or other. The most common types of qualitative analysis are:

\begin{itemize}
	\item \glsf{thematic-analysis}, which aims to identify recurring themes, their definition and relationships. It is applied particularly to text, e.g., transcriptions of interviews or answers to \glsplf{questionnaire} or existing text \glsplf{document}, for instance to find out something about people's views, opinions, knowledge,~\etc{}.
	\item \glsf{content-analysis}, which aims to identify patterns used for communication, whether in text, speech, images, videos, or other, for instance, focusing on the use of certain words, themes, or concepts within that content. It has many uses, from discovering and understanding patterns, to looking at intentions behind what is expressed, or to highlighting differences of use in different contexts
	\item \glsf{discourse-analysis}, which focuses on the use of language in conversations in a real-world context, including how this is influenced by historic or cultural factors, or power dynamics
	\item \glsf{narrative-analysis}, which focuses on stories made and told by people, to investigate their meaning and how people make sense of reality.
\end{itemize}

\ResAnTechnique{any of the above types of qualitative analysis}

While their goal may be different, all these types of analysis apply {\glsf{coding}} as a core method, discussed next.


\subsection{Coding qualitative data}\label{ssect:CodingQualitativeData}

A \glsf{code} is a label which describes an extract from a \glsplf{quantitative-datum} set, with \glsf{coding} the process of creating and assigning codes to categorise those extracts. 

\Glsf{coding} is important and it helps you ensure that your analysis is systematic, and the codes will help you explore themes and patterns in the \glsplf{datum}. However, codes are not themes: they are just labels used to group similar types of \glsplf{datum}, developed to support your follow-up analysis. 

There are two main approaches to \glsf{coding}. In \glsf{deductive-coding}, the codes are decided upfront, before looking at the \glsplf{datum}, and may be based on the \glsplf{phenomenon} in your \glsf{research-problem}, or may have emerged from your literature review, including codes possibly used in previous studies. In \glsf{inductive-coding}, the codes emerge from the \glsplf{datum} and are not pre-defined. Deducting and \glsf{inductive-coding} can also be combined: you could start with a set of pre-defined codes then add new codes as you review your \glsplf{datum}.

Whichever your approach, you should follow a multi-pass \glsf{coding} process. The first pass should consist of going through the whole \glsplf{datum} set in order to establish which codes to use. In the second pass, and any subsequent ones, you should apply the codes to the \glsplf{datum} bit by bit, say line by line in a text, or frame by frame in a video,~\etc{}. In the second pass and subsequent passes, you should also review your initial codes, which may become more or less detailed as a result.

You can choose your codes in various ways. For instance, \glsf{in-vivo} \glsf{coding} uses the exact language which occurs in the \glsplf{datum}: this is used, in particular, for participants' speech, especially when different languages are used. Instead, \glsf{descriptive-coding}\footnote{This is a very common approach, although there are others which you can research in the literature.} uses codes that encapsulate a general idea, such as \enquote{sport} or \enquote{running}: this is particularly useful for non-textual \glsplf{datum}, like images or videos. 

Whichever codes you end up with, you should ensure they are properly defined, so that their are unambiguous and can be applied consistently. You should use a \glsf{codebook} for this purpose, to list all your codes and their intended meaning, so that you can revisit and refine them throughout the \glsf{coding} process.

The last step before detailed analysis is \glsf{code-categorisation}, which is the process of reviewing what you have coded and organise it into categories. For instance, from codes such as \enquote{football}, \enquote{tennis} and \enquote{rugby} you may define a category \enquote{sports}. In this way, you both organise your \glsplf{datum} and establish connections between codes and coded information. 

Both \glsf{coding} and categorisation are iterative processes which carry on until you reach \glsf{saturation}, that is no more is gained from further \glsf{coding} or categorisation. At this point, you can proceed with your chosen \glsf{analysis-method}, whether content, thematic, narrative, \glsf{discourse-analysis} or other, in order to identify patterns and themes, and provide your own interpretation of the \glsplf{datum}.

\Glsf{coding} and categorising are time consuming tasks, particularly if you have a large amount of text to code. In most research, \glsf{coding} \glsplf{datum} by hand is impractical and you should at least make use of a word processor, perhaps using colours and comments to code fragments of your text. Better still, you could  make use of a bespoke \glsf{coding} tool for \glsplf{qualitative-datum}: many such tools are now available, some of which can also automate \glsf{coding} and categorisation to some extent.

\begin{activity}[{Investigating tools for \glsf{coding} \glsplf{qualitative-datum}}] 
Conduct a web search on tools which support \glsplf{qualitative-datum} \glsf{coding}. List up to four which appear most commonly used. For each, indicate which \glsf{coding} features it offers and the extent it is freely available for students' research projects.
\begin{solution}
Qualitative analysis tools are growing and changing rapidly, particularly due to the integration and exploitation of AI capabilities. 

At the time of writing this book, the most used commercial products include NVivo, ATLAS.it and MAXQDE\footnote{You can easily find their home page through a web search.}.
 
They all provide support for \glsf{coding}, with more or less extensive automation, alongside various other features such as \glsplf{datum} visualisation, \glsf{statistical-analysis}, automatic transcripts generation from audio and video files, to name just a few. These commercial products are quite sophisticated with a steep learning curve and are usually quite expensive. They are also geared towards large research efforts, possibly by teams of researchers.

An increasing number of lighter, free products are also available. These include, for instance, Taguette, which supports manual \glsf{coding} and is both open source and free to use, or QDE Miner Lite, which is a free limited version of its full commercial release, and also supports manual \glsf{coding}. Such free products may be sufficient for students' research projects, particularly at master's level.

You may have found other similar tools in your search.
\end{solution}
\end{activity}
%%Hack to correct tcbox behaviour
\color{black}

\subsection{Further reading}\label{ssect:QualitativeAnalysisFurtherReading}

There is a lot more for you to know on the different kinds of qualitative analysis, so that you should read more widely before applying your chosen techniques.
 
\ReadingList{qualitative analysis}{kielmann2012introduction,kiger2020thematic,drisko2016content,gee2014introduction,bamberg2020narrative}


\subsection{Analysing your quantitative data}\label{ssect:yourQualitativeAnalysis}

You should now be in a position to carry out your qualitative analysis.

\begin{activity}[{Your qualitative analysis}]
Apply your chosen qualitative analysis technique(s) to your \glsplf{datum} and write up your \glsplf{datum} analysis. When appropriate, apply visualisation techniques to summarise your \glsplf{datum}.
\begin{guidance}
%Hack to correct tcbox behaviour
\color{black}


%Hack to correct tcbox behaviour
\color{black}
In writing up your analysis you should describe:
\begin{itemize}
	\item the techniques and procedures your have used to analyse the \glsplf{datum}
	\item how you coded the \glsplf{datum} and the results of that process
	\item which major themes or patterns have emerged from your analysis
	\item how they relate to your aim and objectives
\end{itemize}
\end{guidance}
\end{activity}
%%Hack to correct tcbox behaviour
\color{black}



%%%LR -- I don't think we need this, which is a T802 thing
%\subsection{The extended abstract}\label{ssect:ExtendedAbstract} 
%
%An {extended abstract} is a summary of \glsf{academic-research} intended for a more general audience, so that it should be easily read and understood by someone with only a superficial knowledge of the topic. As with the abstract, it should be a stand-alone item without any reference to your full dissertation. However, it is a lengthier piece of academic writing, structured with headings and sub-headings, including citations and references, and possibly tables, figures and diagrams to help you present and summarise your work.
%
%\begin{activity}[{Drafting your extended abstract}] Write a draft extended abstract for your project, which should reflect your research progress to date.
%
%\begin{guidance}Your extended abstract should be 4 to 6 pages in length (once complete) and a common structure is as follows:
%
%\begin{itemize}
%\item title --- the same as your dissertation
%
%\item introduction and background --- an outline of your \glsf{research-problem} in its context, its \glsf{significance}, and the knowledge gap addressed by your research
%
%\item aim and objectives --- from your dissertation
%
%\item \glsf{research-design} --- an outline of your \glsf{research-design}
%
%\item results --- a summary of the \glsf{evidence} collected and analysed, and your key findings
%
%\item discussion --- how significant your findings are in relation to \glsf{research-problem} and knowledge gap
%
%\item conclusion and future work --- your overall conclusions and possible follow-up research
%
%\item references --- selected references cited in the body of your extended abstract
%
%\end{itemize}
%
%Your course may have different guidelines which you should check and follow to produce your extended abstract.
%
%\end{guidance}\end{activity}
%%%Hack to correct tcbox behaviour
%\color{black}

\chapter{Writing up}\label{ch:writingUpInStage4}

It's time for your end-of-stage report to consolidate your work so far and to provide some more content for your dissertation. Its recommended structure and content are indicated in~\Cref{stage4WritingOutcomes}. 

\begin{activity}[{Writing and assessing your report for~\Cref{stage4}}] Using your word processor of choice, revise and expand your~\Cref{stage3} report by applying the structure and guidance in~\Cref{stage4WritingOutcomes}, making good use of your notes and summaries from all related activities you have carried out.

Assess your report by applying the criteria in~\Cref{tab:criteriaForReport4}. Revise and iterate until you are ready to move on. 
\begin{guidance}
%Hack to correct tcbox behaviour
\color{black}


%Hack to correct tcbox behaviour
\color{black}
There are substantial additions to your previous report as a result of your work in this stage. In particular:
\begin{itemize}
	\item you should revise and expand your \glsf{research-design} chapter substantially, adding details of the procedures you have followed in your \glsplf{datum} generation and analysis. You should make all necessary adjustments to your methodology which may have resulted from carrying out those activities
	\item in your (new) analysis and interpretation chapter, you should provide appropriate summaries of the \glsplf{datum} you have generated, using suitable techniques to do so, such as tables, charts, visualisations,~\etc{}. You should also provide a succinct account of your \glsplf{datum} analysis and record its main findings. You should structure your narrative in this chapter to best convey what you have done, striving for clarity and rigour 
	\item if appropriate you may use appendices to include samples of your \glsplf{raw-datum}, or other information in relation to the methods you have applied, such as \glsplf{questionnaire} you have used, or code you have written.
\end{itemize} 
In evaluating your report, for each criteria in the table, you should consider the related prompts, write down any further work needed for your next stage, and update your  and risk assessment accordingly.
\end{guidance}\end{activity}
%%Hack to correct tcbox behaviour
\color{black}

\begin{SimpleNColTable}{tab:criteriaForReport4}{2}[\narrowtablewidth]{Criteria for reviewing your report}[R[2]R[8]]
Criteria & Prompts \\
Completeness & Are all sections included and their content complete? What is missing?\\
Academic writing & Is your writing clear, concise and precise? Should you improve it further? \\
Logical structure and flow & Have you structured your writing so that your narrative follows a logical flow? Which restructuring may be needed?\\
Supporting \glsf{evidence} & Are all your claims supported by appropriate references or other \glsf{evidence}? Which further \glsf{evidence} do you still need?\\
Citation and reference style & Do all your citations and references comply with the bibliographical style required by your course? \\
Avoiding \glsf{plagiarism} & Have you acknowledged the work of others? Is it clearly distinguished from your own? \\
Grammar and spelling & Have you proof-read your report carefully to remove typos and grammatical errors? \\
\end{SimpleNColTable}
 
 
\begin{takeaways}{Stage 4}\label{ch:Stage4Takeaways}
\Cref{stage4} has focussed on research methods for generating and analyse data. Here are takeaways of \Cref{stage4}:
\begin{itemize}
\item \glsplf{raw-datum} represent any \glsplf{datum} you generate and analyse as part of your research, and upon which your \glsf{evidence} and contribution to knowledge are based
\item you need to manage your \glsplf{datum} carefully, ensuring they are properly stored and organised
\item you must consider whether you will need to share your research \glsplf{datum}, and how to deal with confidential \glsplf{datum}, particularly \glsplf{personal-datum}
\item \glsf{sampling} is the process of selecting a \glsf{sample} from the \glsf{population} of interest, and is something used in many research strategies
\item all research needs methods to generate and analyse \glsplf{datum} and there is a vast choice of methods for you to choose from
\item you can use modelling methods if your research requires you to build models of natural, social or artificial \glsplf{phenomenon}  
\item each method requires you to consider procedural and feasibility issues, alongside potential research weaknesses and how to deal with them
\item tables are common ways to organise and present many kinds of \glsplf{datum}, and a good starting point for \glsplf{datum} analysis 
\item \glsplf{qualitative-datum} are heterogeneous in nature, and you many need bespoke approaches to summarise and present them
\item when your \glsplf{datum} are scalar or categorical you can use \glsf{descriptive-statistics} to capture key features of your \glsplf{datum} set, and use standard charts to visualise such features
\item you can use \glsf{inferential-statistics} to make predictions, specifically to establish whether patterns or effects observed in a \glsf{sample} can be inferred for the \glsf{population} from which the \glsf{sample} was taken
\item qualitative analysis comes in many flavour, depending on your goal in extracting meaning from \glsplf{qualitative-datum}
\item \glsf{coding} is the first step in all kinds of qualitative analysis. It is the process of assigning labels to extracts from \glsplf{qualitative-datum} to allow a systematic follow-up analysis
\item in writing up your data analysis you must decide how to summarise and present your data, how to report your analysis and  your findings and how to structure your arguments and narrative. 
\end{itemize}
\end{takeaways}

%%Sectional bibliography
\printbibliography[segment=\therefsegment,title=Stage IV \bibname]
