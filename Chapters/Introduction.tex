%\part{Introduction}
\chapter{So, you want to do a research project}\label{ch:SoYouWant}

Well, you've come to the right place{\ldots}

\section{What is academic research?}\label{sect:WhatIsAcademicResearch}
In its most general sense, research is investigation, fact-finding, exploration and analysis. However, academic research is first and foremost about the \emph{generation of new knowledge} with respect to what is already known in a field of study. New knowledge is what academic research must deliver.

Academic research is therefore about \emph{the process of knowledge generation} and how it is executed. The expectation is that of a \emph{systematic and rigorous process} of generating, analysing and interpreting data, of drawing well-founded conclusions from evidence and explicitly stated assumptions, and of presenting findings in a clear and logical manner. There is also an expectation that your work is legal and ethical in all respects, both in the treatment of other participants and of any sensitive information, and in the way you report your work. It is this rigorous process of academic research that you will learn and practise by following the guidance in this book.

Academic research is also a \emph{collaborative endeavour} within a field of study: even if you are conducting your project on your own, you will build on what researchers have done before --  standing on the shoulders of giants as one great scientist once said, adding one more piece to the jigsaw puzzle so that future researchers can build on what you have done -- by standing on yours!

With the completion of your research and by sharing your results, as a new researcher you will join a community of peers.

\begin{question}[subtitle={Activity: What is academic research?}] You will have been involved in a process of personal research already in your life, probably daily, discovering knowledge that is \emph{new to you}. Every time you search the web and find information about something you didn't know, for instance, your own personal knowledge grows.

Take five minutes to think about how this personal research process differs from academic research. Write down your answer\footnote{This book is full of activities like this. Each one is designed to help your personal research process about your project. Each asks you to think about something and then write down your answer, which you should always do. Writing down your answer means you can revisit it later.}.

\begin{solution}If the knowledge you seek is something you can easily find in existing sources, e.g., books, articles, the web,%\footnote{Of course, we're talking of reputable sources. All that is on the web isn't knowledge! Your studies will lead to new \emph{reputable} knowledge.},
~\etc{}, then it is already part of the existing body of knowledge, so there is no new knowledge generated in a general sense. It is also unlikely you will have followed a rigorous and systematic process in generating and analysing data to draw your conclusions, possibly because data are incomplete or of poor quality, or your analysis is not very rigorous. The issue you are investigating may also be of no relevance to a wider community. Rather than academic research, this is a process of personal learning.
\end{solution}\end{question}
%%Hack to correct tcbox behaviour
\color{black}

\subsection{Master's level research}\label{ssect:master'sLevelResearch}
Your first research project is likely to be part of a programme of master's study. You may therefore wonder what is different or special about master's level research compared to all other academic research. Its distinguishing feature is that this is research in the context of a master's programme, typically as a \gls{capstone-project}\footnote{A capstone is a large flat stone that completes a structure, typically a wall. In archeology, it is used to refer to a flat stone forming the roof of a tomb. A capstone project completes a degree or other programme of study. The two are not (intentionally) related.} project as its final component. 

In essence, master's research is no different from any other kind of academic research in that it you will be expected to generate some original knowledge within your field of study, perhaps by addressing a relevant, non-trivial issue or problem within that field, and in a rigorous, systematic and ethical manner, by applying accepted research methods and techniques.

However what makes your project master's level compared to, say, Doctorate or post-doctoral research is primarily its scope: you must complete your master's project within the time constraint of your master's course, which will be a small fraction of the time you would have to spend on a Doctorate or post-doctoral research\footnote{A master's capstone module is usually 600 hours of study. A Doctorate could be as much as 10 times that. Post-doctoral research may carry on for several years. Professional academic researchers may dedicate the whole of their career to research a particular topic.}. Therefore, at master's level you must trade off your ambition as a researcher with what you can realistically achieve within the available time. Often master's projects take an existing idea or approach and apply it in a novel way, while Doctorate and post-doctoral projects are likely to come up with a completely new, ground-breaking approach.

The constraints of master's research leave very little room for trial and error, hence it is critical that you identify an appropriate scope for your intended research very early in your studies. This might well involve choosing a topic proposed by an academic who will become your supervisor. In addition, your choice of topic will be limited by the degree for which you are studying. If you're doing a master's in Geography, for instance, a capstone project in the area of Information Security is not likely to be relevant\footnote{Unless you establish a clear link between Geography and Information Security, which is of interest to geographers!}.

Lastly, the range of methods and techniques\footnote{You'll hear a lot about methods and techniques that apply in research in the following chapters. The skills you will build on top of them -- critical thinking, problem solving, reflection, for instance -- are generally useful, particularly in your professional career. So, there are very good reasons to study them well, even if you might get a bit sick of them during your project.} you can apply at master's level is likely to be narrower than those used by more advanced researchers, as the latter may be too time consuming to be feasible. This, too, limits the type of the research problems you will be able to address in your project.

\begin{example}{Topic example}The recent explosion of AI solutions has been driven by the creation of technology which is able to do human-like things, for instance the creation of images and text that is at times indistinguishable from what you or I might write. Some might go as far as to say that AI is creating new knowledge!

A good topic for a master's project might be to investigate the extent to which AI technology can be said to be \enquote{doing research}.

The originality of such project may come from the novel combination and application of existing techniques, or might build a framework for distinguishing new knowledge from the simpler re-hash of old knowledge.

This project might come up with new Intellectual Property\footnote{Intellectual Property (IP) is the practical part of new knowledge that you can exploit as its creator, like an invention or a novel design. Many research projects have led to IP that has been licensed or sold, delivering economic benefits to its creators and others, such as businesses, customers or public bodies. Your university will have an IP policy that you should know about before signing up if there's any chance your IP will be valuable. We've written more about IP in~\Cref{sect:IntellectualProperty}.}.
\end{example}
\subsection{What you will have achieved as a master's graduate}\label{ssect:WhatYouWill}

Now that you know what is specific to master's research, it is worth looking at what you will have achieved by completing a master's research project. This is because master's research is in some sense the baseline of academic research skills and competencies, upon which you will build if you continue down the path of an academic researcher.

While there may be differences in the specifics of each master's course, there are benchmarks which define the knowledge and skills all master's graduates must attain. Many countries use standard frameworks to define their academic qualifications. In the UK, for instance, the UK Frameworks for Higher Education Qualifications align to the Framework for Qualifications of the European Higher Education Area, bringing together standards across the European Union to ensure comparability between European degrees\footnote{The ENIC--NARIC Networks are notable in ensuring the academic recognition of qualifications across 55 countries. See \href{https://www.enic-naric.net/page-framework-qualifications-europe-and-north-america-region}{https://www.enic-naric.net/page-framework-qualifications-europe-and-north-america-region}}.

The UK has adopted the following definitions:

\begin{displayquote}
Master's degrees are awarded to students who have demonstrated:

\begin{enumerate}
\item a systematic understanding of knowledge, and a critical awareness of current problems and/or new insights, much of which is at, or informed by, the forefront of their academic discipline, field of study or area of professional practice

\item a comprehensive understanding of techniques applicable to their own research or advanced scholarship

\item originality in the application of knowledge, together with a practical understanding of how established techniques of research and enquiry are used to create and interpret knowledge in the discipline

\item conceptual understanding that enables the student to evaluate:

\begin{itemize}
\item critically current research and advanced scholarship in the discipline

\item methodologies and develop critiques of them and, where appropriate, to propose new hypotheses.
\end{itemize}

\end{enumerate}

Typically, holders of the qualification will be able to:

\begin{enumerate}[resume]
\item deal with complex issues both systematically and creatively, make sound judgements in the absence of complete data, and communicate their conclusions clearly to specialist and non-specialist audiences

\item demonstrate self-direction and originality in tackling and solving problems, and act autonomously in planning and implementing tasks at a professional or equivalent level

\item continue to advance their knowledge and understanding, and to develop new skills to a high level.

\end{enumerate}

And holders will have:

\begin{enumerate}[resume]
\item the qualities and transferable skills necessary for employment requiring:

\begin{itemize}
\item the exercise of initiative and personal responsibility

\item decision-making in complex and unpredictable situations

\item the independent learning ability required for continuing professional development.

\end{itemize}

\end{enumerate}

\end{displayquote}

This definition establishes key \glspl{learning-outcome}\footnote{A \gls{learning-outcome} defines what a student should be able to do after their study. Learning outcomes are tested in assessment, exams and, in the case of research, the dissertation or thesis. Although, in this case, they are based on a standard, they may be described differently for your master's -- often useful detail is added. Knowing the learning outcomes for the Master's degree you are studying can give you a real advantage when it comes to writing up!} for a Master's graduate, principally:

\begin{itemize}
\item advanced knowledge at the leading edge of an academic discipline

\item critical thinking in evaluating existing knowledge, in choosing and applying research methods to generate new knowledge, and in making judgements and deriving conclusions

\item originality in applying knowledge and techniques to address complex problems and generate new knowledge

\item effective communication to diverse audiences

\item self direction, autonomy and independent learning.

\end{itemize}

\section{The role of a research supervisor}\label{sect:RoleOfASupervisor}
An academic supervisor will support you throughout your project -- a research expert highly knowledgeable in the subject area of your research. It may even be the case that your project topic comes from suggestions from, or previous research conducted by, your supervisor.

There's a hint in the role title -- your supervisor is there to supervise your studies: to mentor, guide, make suggestions, and engage in discussion on all aspects of your research. Your supervisor is unlikely to want to micro-manage your project (but it does sometimes happen!). Whatever, ensuring that you make the required progress and meet the required standards remains your responsibility.

It's sometimes the case that, as you make progress in your research your knowledge and expertise will grow to rival that of your supervisor. However, your supervisor, as an expert in the \gls{research-process}, will continue to provide invaluable guidance and support throughout your project.

Depending on your course regulations, your supervisor may have additional roles. For instance, your supervisor may be required to assess formally whether you are making the necessary progress as you reach each \gls{milestone} in your project, or may be required to validate that the research carried out is indeed your own work. Missing their assessment and validation is very likely to slow progress, and in the worst case, stop you completing your project.

So, to discharge these various roles, your supervisor must have sight of your work and how it develops throughout your project: engaging with your supervisor regularly, discussing your ideas and their development in detail will be a very important part of conducting your project.

It is therefore essential to your success that you develop a close working relationship with your supervisor as early as is possible and engage in an active dialogue on all aspects of your research on a continuing basis.

You supervisor may have more than one student conducting research with them. This means that there will also be an expectation that \emph{you} drive and manage the interaction with your supervisor.

\begin{question}[subtitle={Activity: Managing the interaction with your supervisor }] If you haven't already done so, get in touch with your supervisor to introduce yourself and discuss how to work together.

After the meeting, write down what you have agreed, and negotiate and create diary appointments in your favourite calendar app accordingly.

Make sure your supervisor formally agrees -- although, these days, this is usually just a matter of RSVPing to a calendar invitation.

\begin{guidance}Before contacting your supervisor, you should think about \emph{your} availability for a 30 minute meeting every week, for instance, or 60 minutes every fortnight. This will make it easier for you to identify and agree with your supervisor a regular slot to discuss your ideas and work progress. You should also account for periods when you or your supervisor may not be available.

It is important to book regular diary slots for the duration of the project. It may not be that you need every meeting you arrange, especially in the later stages of your research, but it will be good to know that the time is there if you do need it.

This activity is particularly important if you are a part-time student, possibly juggling study, work, and family commitments.
\end{guidance}\end{question}
%%Hack to correct tcbox behaviour
\color{black}

\section{What is expected of you as an academic researcher}\label{sect:WhatIsExpected}
There are key differences between doing a research project and studying a taught course. Awareness of those differences will help you prepare better for your research.

These differences arise because of the higher demands of an academic research project to demonstrate i) self-direction, ii) critical thinking, iii) time and task management, and iv) information management.

\subsection{Self-direction}\label{ssect:Self-direction}
A research project requires you to be in the driving seat\footnote{This includes \enquote{managing} the interaction with your supervisor, as mentioned earlier.}. While in your previous studies, you may have been given all necessary materials and directed in your studies, on a research project deciding what to study and when are left up to you to a large extent. In particular, you will need to identify relevant material in relation to your chosen topic, determining its relevance and how to use it in your research.

You will also be responsible for understanding the academic standards required, and for organising your research work. This is part of demonstrating that you can exercise self-direction in planning, executing, and critically reviewing your work, and that you are truly an independent learner. This book will provide plenty of advice to help you take control of your own research.

\begin{question}[subtitle={Activity -- Self-direction}] Write down three areas of your life where you need to be self-directed. For each, write down how you go about being self-directed, from how you start to how you finish. What does it feel like to be self-directed? Is there some emerging pattern?

\begin{guidance}Those areas might be from your work, your previous studies, your personal interests, hobbies, or some other aspects of your life.
\end{guidance}\end{question}
%%Hack to correct tcbox behaviour
\color{black}

\subsection{Critical thinking}\label{ssect:CriticalThinking}
Your research will require you to think critically about all aspects of your work. In essence, critical thinking is about asking questions systematically to look for evidence and good reasons in order to form your own judgements, instead of accepting what you read or hear at face value.

At the heart of critical thinking is an ability to maintain an objective position by weighing up all sides of an argument, evaluating its strengths and weaknesses, and testing how sound the claims made, and their supporting evidence, are. To think critically you will need to scrutinise arguments in great detail and with some degree of skepticism. By taking such an objective stance you will be able to judge good and bad arguments, regardless of whether you agree with them.

In your project, you will be expected to think critically about all aspects of your research and to capture such thinking in your writing. As the author of a dissertation, critical thinking will benefit your ability to build stronger arguments, avoid bias and link your claims to appropriate supporting evidence. As a reader of the \gls{academic-literature}, critical thinking will help you assess the strength of other researchers' arguments and identify unsupported claims and illogical reasoning.

Critical thinking is both an attitude and a skill essential in academic research, which this book will help you develop. It is also beneficial to your professional and private life. In fact, it is likely that you already think critically in many aspects of your life.

\begin{question}[subtitle={Activity -- Critical thinking}] Write down three areas of your life where you need to think critically. For each, write down how you go about thinking critically, from how you start to how you finish. What does it feel like to think critically? Is there some emerging pattern?

\begin{guidance}These areas might be the same areas as for the previous activity, or might be different.
\end{guidance}\end{question}
%%Hack to correct tcbox behaviour
\color{black}

\subsection{Time and task management}\label{ssect:TimeAndTask}
For your research project, you may be provided with some broad guidelines and deadlines\footnote{Deadlines on master's programmes tend to be absolute as they affect the cycle of master's assessment and validation. A little more flexibility is available at Doctoral level, where assessment and validation often occurs only at the very end.} around which you should conduct your project in general. However, you will be expected to organise your time, and plan and execute your work in detail to meet those deadlines.

Within the set duration of your project, you will need to plan, organise and execute your research and submit your final dissertation. It will be up to you to develop and keep under review your personal research plan, allocate time to tasks, and meet deadlines and \glspl{milestone}. You will find that one of the challenges of conducting academic research is that it is open ended. Therefore you will need to learn how to set boundaries and use your limited research time effectively.

This book will help you plan your project work based on a 5-stage framework developed for this purpose.

\begin{question}[subtitle={Activity -- Time and task management}] Write down three areas of your life where you need to manage your time and tasks. For each, write down how you go about managing them, from how you start to how you finish. What does it feel like to manage tasks and time? Is there some emerging pattern?

\begin{guidance}These areas might be the same areas as for the previous activity, or might be different.
\end{guidance}\end{question}
%%Hack to correct tcbox behaviour
\color{black}

\subsection{Information management }\label{ssect:InformationManagement}
The volume of materials you will need to read and study during your project is considerable: you will need to read extensively around the topic of your choice, as well as research methods to apply in your research. This will result in a large amount of information that you will need to gather, organise, store and make sense of, including making copious notes as you go along.

You will also need to submit a substantial final dissertation. In the UK, for instance, a master's research dissertation is usually in the region of 15,000 words. This is likely to be much longer than any other written work required by your previous studies, and you will need to structure your narrative appropriately and present your work in compliance with standards and conventions for academic writing.

All this will require you to develop and apply a range of new skills, which this book will help you develop.

\begin{question}[subtitle={Activity -- Information management}] Write down three areas of your life where you need to manage information. For each, write down how you go about managing information, from how you start to how you finish. What does it feel like to manage information? Is there some emerging pattern?

\begin{guidance}Again, these might be the same areas as for the previous activities, or might be different. You might even find that it is part of previous activities.
\end{guidance}\end{question}
%%Hack to correct tcbox behaviour
\color{black}

\section{Key skills}\label{sect:KeySkills}


\subsection{Writing skills}\label{ssect:WritingSkills}

%%JGH - changed solely to mainly
In all likelihood, your research will be assessed mainly on the basis of your dissertation, so that the quality of your writing is paramount\footnote{This is true of both master's and Doctoral research. In fact, I could say that this applies to research in general: the expectation is that all research is written up, usually in the form of an academic report or article that is then open to the scrutiny of other researchers in the field of study.}. 

Not only should your dissertation be well presented and free of grammatical errors and typos, but you will be expected to be clear, concise and precise throughout, and avoid conversational language\footnote{Except, perhaps, in the acknowledgements, where you thanks your family, friends, the authors of this book, \emph{etc}.}. If this is your first experience of academic research, your dissertation is going to be the longest and most demanding piece of writing to date. Therefore, you will be expected to structure your dissertation appropriately and organise your narrative in a logical and coherent manner. Last but not least, you will be required to demonstrate your critical thinking throughout and adopt a range of academic conventions, which I will cover in~\Cref{stage2} of the book.


\begin{question}[subtitle={Activity: Your writing practice}] Consider a piece of academic writing you have produced, possibly as part of your previous studies. Assess its quality in terms of its clarity, conciseness, precision, structure and logical narrative flow. Does the piece demonstrate your critical thinking? How?
\begin{guidance}
For each of the above qualities, find extracts from the piece which demonstrate your good practice, and extracts where things could be improved. Rewrite the latter accordingly. Reflect on your strengths and weaknesses, and writing skills you may need to develop further.
\end{guidance}\end{question}
%%Hack to correct tcbox behaviour
\color{black}




\subsection{Active reading and note taking}\label{ssect:ActiveReadingAnd}
Throughout your project, and particularly when reviewing the literature, you will read a great deal of material, and you are likely to spend more time reading and assimilating new content than you may have experienced in your previous studies.

Therefore it is important that you become both effective and efficient at reading and note taking. The key points you need to keep in mind are:

\begin{itemize}
\item you take notes as you read
\item you are disciplined and systematic in your note taking, so that you can easily locate and review your notes when you need to during your project
\item you won't need to study in depth all that you read, so you should develop reading techniques both to grasp the essence of an article very quickly and to dig deeper into content and meaning.
\end{itemize}


These practices will help you become an \gls{active-reader}, that is one who engages with the materials and is able to assimilate the important points in an effective manner.

\begin{question}[subtitle={Activity: Your note taking practice}] Think of how you take notes when reading new materials. Write down key practices you apply and how effective they are in helping you assimilate new materials.

\begin{solution}These are things I usually do. While reading, I highlight definitions and words or phrases of interest, and make annotations in the margin next to paragraphs which make significant points, or provide useful summaries or raise interesting questions. If reading a physical book, I attach post-it notes to pages which are of particular interest, so that I can return to them easily. On digital copies, I use digital bookmarks.

After reading a whole article or a chapter in a book, I jot down some bullet points summarising key insights from my reading, my overall judgement of the whole reading, and how it may be useful, or otherwise, for my work.

I usually keep my notes and summaries with the physical or digital copy of what I have read, and I also keep together related articles within my file system --- in the past, I used to have a physical filing cabinet for this, but these days I work almost exclusively with digital content appropriately organised in folders and sub-folders.

You may have come up with a different set of practices. What matters is that you reflect on how effective they are or where improvements may be needed.

\end{solution}\end{question}
%%Hack to correct tcbox behaviour
\color{black}

\subsection{Digital literacy and tools}\label{ssect:DigitalLiteracyAnd}
Unless you're thinking of handwriting your dissertation\footnote{And it has been done, although perhaps not that recently, check out \url{https://www.reddit.com/r/Handwriting/comments/b9vmss/my_mums_handwritten_thesis_from_1982/}, for instance.}, you're going to make use of digital tools for text preparation, bibliographic management, note taking,~\etc{}.

There is a vast range of tools, some of which you will be already familiar with (but might not be suitable for the very long document you're going to have to write) and some that you are going to have to learn how to use -- and find time for that learning. Among the latter, the most important are probably those for bibliography management.

So, why don't you start with these?

\subsection{Bibliographic Management Tools}\label{ssect:BibliographicManagementTools}
During your project, it is essential that you keep track of the articles and papers that you read, alongside your notes on their content and relevance, or otherwise, to your project. Some you will access repeatedly during your research; others you will only cite as part of your literature review; others still, you may simply discard as not relevant. Whatever their final use in your project might be, it is important that you keep track of what you have read and the use you can make of it as you go along.

%\todo{Check repeats}
%As the amount of reading grows, keeping track can quickly become overwhelming. This is where a \textbf{Bibliographic Management Tool}\footnote{These are also known as \enquote{Reference Management Tools},\enquote{Reference management software} or \enquote{Bibliographic management software}. Wikipedia has a comparison of bibliographic management tools here: url\{\href{https://en.wikipedia.org/wiki/Comparison_of_reference_management_software}{https://en.wikipedia.org/wiki/Comparison\_of\_reference\_management\_software}\}} \textbf{(BMT)} can help you. If you are not already using a BMT, this is the time for you to pick one and master its basic functionalities.

As the amount of reading grows, keeping track can quickly become overwhelming. This is where a \gls{bibliographic-management-tool}\footnote{These are also known as \enquote{Reference management tools}, \enquote{Reference management software} or \enquote{Bibliographic management software}. Wikipedia has a comparison of bibliographic management tools here: \url{https://en.wikipedia.org/wiki/Comparison of_reference_management_software}.} can help you. If you are not already using a BMT, this is the time for you to pick one and master its basic workings.

Briefly, a BMT is a software tool that can help you collect and save details of documents you have used in your research, and from which you can generate references easily, reference lists and bibliographies in a variety of bibliographical styles, to include in your dissertation and other reports you may have to produce during your studies. Its basic functionalities should include:

\begin{itemize}
\item a repository into which you can store the details of each document that you have read (authors, title and so on), your notes on the document, and even a copy of the document itself. Where the repository is online, the tool may allow you to share your collections with other users. In addition, in many cases, the reference sharing service will enable you to download many of the necessary citations from the database

\item an export mechanism to desktop word processors, such as MS Word, Apple Pages or Apache OpenOffice, so that you can include your citations in your text and generate your reference section automatically. The exact mechanism and its accuracy vary between tools, but it is likely to generate a list of references with most (if not all) the required elements of your chosen bibliographical style --- some manual checking and editing may still be needed to ensure that the output conforms to the required style.

\end{itemize}

\begin{question}[subtitle={Activity: Investigating BMTs}] Visit the Wikipedia article that compares notable reference management software (the link is in the related note above). Count how many are included, which ones are free or licensed, and which operating systems, devices and word processing software they are compatible with.

Choose a couple which work with your computing devices and software, and may be suitable for your project. Review their key features, and choose one that meets your needs and preferences. 

\begin{guidance} If you already know or use a BMT, you may wish to compare its features with those of others in the Wikipedia article. 

If you don't know where to start, you could consider Zotero first, as it is free and open-source.

It is also possible your university has a licensed BMT that you can use for your project, so you should investigate whether that is the case and what its key features are.
\end{guidance}\end{question}
%%Hack to correct tcbox behaviour
\color{black}

\subsection{Keeping track of your digital assets}\label{ssect:KeepingTrackOf}
Beside your growing repository of articles from the literature, which you should manage through your chosen BMT, during your project you will be collecting and producing lots of other digital \gls{research-assets} which are needed for your research and will contribute to your dissertation. These may include data, images, tables, your own notes, comments from your supervisor,~\etc{}. You will need to organise and manage all these assets in a disciplined and systematic manner.

The simplest and cheapest way to manage digital assets is to organise your file system appropriately. Here are two basic practices you could apply:

\begin{itemize}
\item establish a naming convention, so that the content or function of each file is clear from its name
\item establish a coherent folder structure, so that the relation between files within each folder is clear, as is the relation between folders.
\end{itemize}

\begin{question}[subtitle={Activity: Organising your digital assets}] Think of how you currently organise and manage your file system. Write down any rule or convention you usually apply and reflect on the extent it helps you to locate and keep track of the information you need. Are there aspects you could improve?

\begin{solution} There is no right or wrong: different people use different approaches which fit their own preferences and needs. For instance:

\begin{itemize}
\item keep your file system well organised and ensure you review and clean it up on a regular basis
\item partition your file system to separate work-related and personal assets
\item organise assets thematically, using folders and sub-folder structures
\item when folders become too crowded, refactor them by creating more sub-folders to group related assets
\item separate assets based on current, completed or terminated tasks, the latter for things you started but for one reason or other couldn't complete.
\end{itemize}

These are just few examples. You may have come up with your own set of practices. What matters is that you reflect on how effective they are or where improvements may be needed.
\end{solution}\end{question}
%%Hack to correct tcbox behaviour
\color{black}

\subsection{Managing document versions}\label{ssect:ManagingDocumentVersions}
As you progress in your project, you will generate several drafts of parts of your dissertation and other intermediate reports. This will happen as you make progress with your literature review or chose, refine and apply your research methods. 

It's important that you keep and manage different versions, including backing them up regularly, to ensure you can always access the latest one, but also refer back easily to previous work should you need to. 

The simplest option is for your to use your file system wisely by applying an appropriate versioning convention to keep successive versions of your files in good order: this will require you to be disciplined and do much of the work manually. In addition, you could make use of a word processor which already includes some functionalities to manage document versions, so that some of the manual work is handled by that tool. The most sophisticated approach it to use a bespoke \gls{version-control-system}, like Git\footnote{\href{https://git-scm.com}{https://git-scm.com}}, which is free and open-source. However, if you are not already using one, there may be a steep learning curve to start with, and you will need extra time to learn how to use and set up such a tool. 

\begin{question}[subtitle={Activity: Thinking about how you manage document versions}] Think of how you currently manage different versions of your documents or other digital assets. Summarise your practices and reflect on the extent they help you keep track and locate easily different digital versions. Are there aspects you could improve?

\begin{solution} 
If you are not going to use a bespoke version control system, then one of the reasons for keeping your file system well organised is to ensure you can keep track of the most recent version of any document, but also locate previous ones easily. 

In such case, you will need to come up with your own system to identify versions. This might be a very simple numbering system, say, using 1.0, 2.0, 3.0,~\etc{}.  at the end of a file name to identify subsequent versions, and only generate a new version when there is a significant departure from the current document, say you restructure it significantly, or remove a substantial amount of content, or want to freeze a version to send to your supervisor for comments while still carrying on working. You may also need to archive previous version so not to clutter your working folders.

Manual approaches are not perfect, but can be simple enough to work for you. Discipline and organisation are however necessary to make them work effectively.

You may have come up with a different set of practices. What matters is that you reflect on how effective they are or where improvements may be needed.
\end{solution}\end{question}
%%Hack to correct tcbox behaviour
\color{black}

\subsection{Choosing the right word processor}\label{ssect:ChoosingTheRight}
If you have not written a substantial document, like a dissertation, before you should put some thought into how you are going to write up your project work. Your final dissertation is likely to be in the region of 15,000 words, plus appendices, which makes it a substantial document. By following the advice in this book, you will accumulate materials as you progress, so that you will be editing, reshaping, extending and re-versioning content you have written throughout your project. Therefore, you will need to choose the right tool for the job.

You could start by considering whether any word processors you may be familiar with and possibly used in your previous studies, is appropriate for these requirements. It is possible that you will need to develop more sophisticated word processing skills or use more advanced functionalities of your word processor of choice. You should also take into consideration the format your course requires you to submit your work in, and ensure your chosen software can generate documents in that format.

Whichever word processor you choose, you should ensure it allows you to:
\begin{itemize}
\item manage multiple documents
\item structure and process large documents
\item back up your work regularly, possibly automatically, and restore documents which may have been corrupted by system errors or crashes
\item connect to your chosen BMT, so that citations and references can be managed easily and shared automatically between the two
\item generate documents in the formats required for submitting your dissertation and any other required assignment.
\end{itemize}


\begin{question}[subtitle={Activity: Investigating your word processor's advanced functionality }] Check the functionality of your current word processor against the list above. Is it powerful enough to meet the above requirements? Write down the key functionalities you will need to use and make a note of any new skills you will need to develop, then set some time aside to practice them.

\begin{solution} \LaTeX{}\footnote{The \LaTeX{} system is available from \url{https://tug.org}, the \TeX{} User Group} is my favourite document preparation and typesetting system. It is a very powerful tool, particularly suited to scientific and technical content, but also works well for other report types. It is also free and well supported by a large community of users.

\LaTeX{} meets all the requirements above. You can use it to structure and process large documents, which you can split up and organise in separate sub-documents: for instance you can have different files for different chapters or different versions of a chapter, which you can then include flexibly in your final document.

The \LaTeX{} system I use backs up my documents automatically and includes a recovery feature in case of system crash. It also connects with my BMT (I use BibDesk\footnote{BibDesk is free software, available from \url{https://bibdesk.sourceforge.io}). BibDesk is only available for the Mac Operating System}, but could use Zotero\footnote{Zotero is available from \url{https://www.zotero.org}}), so that all my citations and references are easy to manage and shared automatically between the two systems.

One of \LaTeX{}'s most powerful features is that it separates output format from source text, so that we can change the style of citations and references, and in fact of the whole document, very easily and generate outputs in different styles from the same source text.

\LaTeX{} also produces pdf, which is a widely accepted document format.
\end{solution}\end{question}
%%Hack to correct tcbox behaviour
\color{black}

\LaTeX{} is widely used across the scientific community and recommended especially if you're writing:

\begin{itemize}
\item a mathematical or scientific dissertation, as \LaTeX{} comes with fonts for these by default

\item a dissertation in which you will have lots of figures or tables, as these are managed automatically by the system

\item a dissertation in which you will be cross-referencing between sections frequently, as \LaTeX{} provides mechanisms for ease of cross-referencing

\item a dissertation that has a complex structure, as \LaTeX{} supports all sorts of document structures by default

\item a dissertation that has many bullet points or enumerated lists, as \LaTeX{} has powerful formatting options to manage them

\item a dissertation with many references and citations, as \LaTeX{} handles them automatically and allows you to style and re-style them very easily.
\end{itemize}

You can find out why \LaTeX{} is the document preparation system of choice at many blogs\footnote{Including this one \url{https://blog.orvium.io/LaTex-over-word/}}.

However, \LaTeX{} has a steep learning curve -- the investment of time that you will use for learning \LaTeX{} will be considerable but it will, if your dissertation has any of the above characteristics, save you time and tears in the end, especially when you're making the final touches.

%\begin{figure}[htbp]
%\centering
%\includegraphics[keepaspectratio,width=\textwidth,height=0.75\textheight]{what does word do?.jpg}
%\caption{You've got five minutes to submit your dissertation in Word. You want to move that figure a little. What happens to the rest of your dissertation? Source: \url{https://www.reddit.com/r/funny/comments/2glhbp/moving_a_picture_in_microsoft_word/}}
%\label{screenshot2023-05-10at113717}
%\end{figure}


\begin{question}[subtitle={Activity: Investigating \LaTeX{} editors }] \LaTeX{} is a mature system and numerous editors are now available, from open source to proprietary, from desktop to web-based. 

Conduct a web search to investigate its key features, and write down those which make it particularly suited to writing a research dissertation.
\begin{solution}
At the time of writing, \href{www.overleaf.com}{Overleaf}\footnote{Overleaf has a \enquote{freemium} business model, offering both free and premium subscription options. The free tier provides essential features suitable for basic use. Many universities have a full Overleaf licence; your university library should know if this is the case.} is one of the most popular \LaTeX{} editors for writing research documents of all sizes because it:
\begin{itemize}
\item has all the power of \LaTeX{}
\item is set up for research work
\item has a fantastic help system
\item has a range of templates -- and may even have one customised for the style your university expects
\item can be used to track changes between versions
\item backs up your work
\item works through a browser, and so is available on all your smart devices.
\end{itemize}
\end{solution}\end{question}
%%Hack to correct tcbox behaviour
\color{black}

\LaTeX{} also has a Word-like interface, through the LyX tool, which can reduce the learning curve. If you find it easier to use a Word-like interface, but want the benefits of \LaTeX{}, you might like to try LyX.%\todo{Check url}
\footnote{LyX is available at \url{lyx.org}.}

\begin{takeaways}{Introduction}\label{sect:IntroductionTakeaways}

\begin{itemize}
\item academic research is about the process of knowledge generation within a field of study
\item at master's level it is critical you establish the right scope for your research as early as possible, as you have very limited time for remedial actions later on 
\item master's research gives you a baseline upon which your can build further research skills and competences if you continue down the path of an academic researcher
\item your supervisor is your best research ally and you should establish an effective working relationship from the start
\item as an academic researcher you are in the driving seat and expected to demonstrate critical thinking, self-direction, and competent time, task and information management throughout 
\item there are key skills (see~\Cref{sect:KeySkills}) you must develop and practise from the onset 
\item choosing the right digital tools for your project will make your research life a lot easier in the long term, even if they require some investment upfront to learn how to use them effectively.
\end{itemize}
\end{takeaways}


%It is time to reflect on what you have learnt so far. Reflection is what makes your learning more effective, and relevant and useful to your own practice, so you shouldn't be surprised that reflection is a common theme in this book. You have already been asked to reflect in the practical activities you have carried out so far. In this closing section, you are asked to reflect of what you have learnt in this chapter as a whole.
%
%\begin{question}[subtitle={Activity -- Reflection on learning}] Consider the content of this chapter and write down key things you have learnt or that have surprised you. For each indicate why they are notable or relevant to you, and how you may apply them in new situations or to inform your future learning.

\endinput