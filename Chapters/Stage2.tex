%%%%2025-09-26: solutions removed as may be source of \abx@aux@cite in .aux file
\begin{PartTitlePage}{The Literature}{stage2}
The focus of~\Cref{stage2} is on completing your literature review and developing your understanding of research-design.	
\end{PartTitlePage}

You've now reached~\Cref{stage2} of the 5-stage framework.~\Cref{stage2} assumes you have completed your work at~\Cref{stage1} and have discussed it with your supervisor, who will have helped you assess whether it is appropriate for your course of study.\footnote{If the scope of your project is still unclear at this point, then you should spend more time in defining it before moving on.}~\Cref{stage2} also assumes you have now a fair idea of how the \glsf{research-process} works, having been through one iteration in~\Cref{stage1}, and that you have gained sufficient basic research skills to be able to carry on your research somewhat more independently, albeit with your supervisor's support.

The good news\footnote{The bad news is that if you haven't picked up everything, you may need to refer back to~\Cref{stage1}. You will find breadcrumbs throughout this stage to help you do this if you need to.} is that this assumption means that there is less to read in~\Cref{stage2}, which will not take you through the \glsf{research-process} step by detailed step, but focus on \textit{additional} knowledge and skills you'll need to progress.

\chapter[Stage II Activities and Outcomes]{\Cref{stage2} Activities and Outcomes}\label{c:Stage2outcomes}

As per~\Cref{stage1}, its Research Activity table and Writing Outcomes table will help you structure and guide your work in this stage.

\section{Your Research Activities for this stage}

The research activities which are in focus in~\Cref{stage2} are shown in~\Cref{stage2ResearchActivities}, which also provides some prompts for your interaction with your supervisor during this stage. 

Looking at the table, reviewing the literature is the most demanding activity, at 30\% of the overall stage effort: in this stage you will complete your first full literature review draft, building upon your groundwork from~\Cref{stage1}. As a result, you may have to adjust or refine your research scope, including \glsf{research-problem}, aim and objectives.

  The next major activity is considering your \glsf{research-design}, at 15\% of the total stage effort: this will allow you to reach an initial understanding of \glsf{research-design}, its building blocks, and how they may apply in your project. At this point, your understanding will be sketchy, but should be enough for you to make sense of what is reported in the \glsf{academic-literature}.

All other research activities are similar in effort to~\Cref{stage1}. You should ensure you complete each of them, even if they are not covered explicitly in this stage. In particular, you should engage in \glsf{reflection} and \glsf{reflexivity} as you go along, and revisit your risk assessment to ensure all major risk is accounted for.

\begin{SimpleNColTable}{stage2ResearchActivities}{4}{\RActivitiesTableCaption{2}}[R[4]cR[8]R[8]]
Research activity & Effort & Description&  Supervisor Interaction Focus\\
Identifying the \glsf{research-problem} &10\% &Adjust, if needed &\\
Reviewing the literature &30\% &{Complete full draft of literature review\\
Draft critical summary of key insights from literature review} &{Suitability of literature review structure and narrative flow}\\
Setting your aim and objectives &5\% &Adjust, if needed &\\
Developing the \glsf{research-design} &15\% &{Consider data and \glsf{evidence} you will need for your research, and possible research strategies and research methods you may apply\\}& Suitability of research strategies and methods considered\\
Generating and analysing \glsf{evidence} &0\% &n/a &\\
Interpreting and evaluating findings &0\% &n/a &\\
Writing up &20\% & Achieve the writing outcomes of~\Cref{stage2WritingOutcomes} & Demonstration of good academic writing and any improvements required\\
\Glsf{reflection} and \glsf{reflexivity} &10\% & Apply to~\Cref{stage2} work and experience as you go along &\\
Planning work &5\% &{Refine your project plan by detailing tasks, \glsplf{milestone} and \glsplf{deliverable} for this stage\\ Review progress} & Appropriateness of \glsf{work-plan} and progress\\
Managing risk &5\% &Review and adjust project risk &Any major adjustment required\\	
\end{SimpleNColTable}


\section{Your Writing Outcomes for this stage} 

\Cref{stage2WritingOutcomes} gives you the writing outcomes for this stage: the activities in this part of the book are designed to help you reach them.

Remember that the first column of the table gives you the expected full structure of the dissertation\footnote{If your course of study assumes a different structure, then use that instead, mapping the writing outcomes accordingly.}. Within that column, the greyed out parts are yet to be written and will be the focus of later stages, while those highlighted in red are to be written during this stage. The remaining parts are those you wrote in~\Cref{stage1}: depending on your work in this stage, you may need to revise or adjust them.

%%Edit stage 2
\part{Stage 2: Scope}

We now move onto Stage 2 of our 5-stage framework. Stage 2 assumes you have completed your work at Stage 1 and have discussed it with your supervisor, who will have helped you assess whether it is appropriate for your course.\footnote{If the scope of your project is still unclear at this point, then you should spend more time in defining it before moving on.} Stage 2 also assumes you have now a fair idea of how the research process works, having been through one iteration in Stage 1, and that you have gained sufficient basic research skills to be able to carry on your research somewhat more independently, albeit with your supervisor's support.

The good news\footnote{The bad news is that if you haven't picked up everything, you may need to refer back to Stage 1. We will leave breadcrumbs throughout the chapter to help you do this if you need to.} is that this assumption means that there is less to read in Stage 2 and we will not take you through the research process step by detailed step, but focus on \textit{additional} knowledge and skills you'll need to progress.

The research activities which are in focus in Stage 2 are shown in Table~\ref{tab:stage2}, which also provides some guidelines for your interaction with your supervisor during this stage. As you can see, the focus of this stage is on completing your literature review and developing your understanding of the basic blocks of research design.


\begin{table}[htbp]
\caption{Research activities addressed in Stage 2 (15\% of project length)\label{tab:stage2}}
\small
\begin{tabulary}{\tablewidth}{@{}LLLLL@{}} 
\toprule
 \textbf{Research process activities} & \textbf{Deliverables} & \textbf{Learning Outcomes: by the end of this stage you will:} & \textbf{Effort} & \textbf{Suggested focus of your interaction with your supervisor} \\
\midrule

 \textbf{Identifying the research problem} & Research problem statement, refined as needed & be able to express a research problem & 3\% & \\
 \textbf{Reviewing the literature} & Substantial draft of your literature review & know key skills and practices in academic writing, and the elements of an academic argument; be able to write academic arguments, to structure the narrative of your literature review and to assess and improve your current draft & 25\% & Suitability of literature review scope, structure and narrative, including appropriate logical argumentation, and demonstration of critical thinking \\
 \textbf{Setting your aim and objectives} & Aim and objectives, refined as needed & be able to express aim and objectives which relate to your research problem  & 2\% & \\
 \textbf{Developing the research design} & Candidate research strategies and methods for your project & know the building blocks of research design, and common research strategies and methods; be able to understand research strategies and methods applied in articles you have reviewed & 40\% & Suitability of candidate research strategies and methods \\
%\textbf{Gathering and analysing evidence} & n\slash a & & 0\% & \\
% \textbf{Interpreting and evaluating findings} & n\slash a & & 0\% & \\
 \textbf{Reflecting and reporting} & Stage 2 report & be able to assess your research progress and to write up a substantial report & 25\% & Any further improvements required, particularly in relation to critical thinking and academic writing \\
 \textbf{Planning work and managing risk} & Updated risk and work plan & be able to assess risk and draw a work plan & 5\% & Any major adjustment required to address deficiencies or manage risk \\
\bottomrule
\end{tabulary}
\end{table}

\begin{question}[subtitle={Activity: Understanding the effort needed in this stage}] Consider Table~\ref{tab:stage2} carefully, taking notice of the entries in the `Effort' column. Make a note of the activities which will be the most time consuming in this stage and what is expected – the suggested focus – for each. Update your work plan accordingly.

\begin{solution}As $40\%$ of your study time, developing the research design will constitute your major effort in this stage. There are two other research activities at $25\%$
%
\begin{itemize}
\item Reviewing the literature, which will get to an almost complete form; and
\item Reporting, critical reflection and conclusions, which will complete Stage 2.
\end{itemize}
%
To update your work plan, you should use the effort percentages to estimate the actual study time you will need to dedicate to each. 
\end{solution}\end{question}
%%Hack to correct tcbox behaviour
\color{black}

\chapter{Writing a full draft of your literature review}\label{sect:stage2literatureReview}
In Stage 2, you will build on your Stage 1 work to write a substantial draft of your literature review. Although it won't be in its final form, this draft will be close to complete – as you work through later stages, you might find a few more papers\footnote{Depending on how thorough you've been, of course.} to consider, but you'll probably be able to count them on one hand.

While in Stage 1 your main focus was on gathering and assimilating relevant articles, understanding their relationships and emerging themes, in Stage 2 your focus will be on consolidating all that you have learnt into a coherent, well-structured and critical review.  

This is the essence of \textit{synthesis} – the act of bringing ideas together into a cogent whole. That cogent whole is the spell-binding story you create to convince your readers\footnote{Eventually, your examiners, first your supervisor, family, friends, anyone else that will read it{\ldots}} that \textit{you have a contribution to knowledge to make}, based on the literature you have read and the knowledge gaps that you have found there. 

Given the number of articles you'll have to read, and the complexity of the relationships that exist between them, there's a long (and sometimes winding) road from the literature to your dissertation.  The knowledge, process and techniques you acquired and applied in Stage 1 will still be relevant\footnote{If needed, refresh your understanding by rereading Section~\ref{sect:litrev} in the previous chapter.}: you can make best use of the time you spend reading the literature by using the Keshav workflow\footnote{Introduced in Section~\ref{sect:Keshav}.}, iterating quickly between papers, and keeping notes that refine what you have as your understanding develops, and will eventually turn into text you can use for your dissertation, too. However, once you have selected, read and understood\footnote{At this point, you may have only a rudimentary understanding, but that will develop as you bring your synthesis together.} a good number of relevant articles, and identified and summarised relevant emerging themes, you can also start develop the narrative to include in your literature review. 

For this you will need further synthesis skills and techniques, which we consider in this section. 

While we present them in the context of writing your literature review, these synthesis skills and techniques apply more widely to any form of academic writing, so they will be essential when you write your narrative across your whole dissertation -- creating the spell-binding story to convince your readers that \textit{you have made a contribution} to knowledge. 

\section{Key skills for synthesising}

There are key skills you will be able to demonstrate through your synthesis of the literature. They are:

\paragraph{Critical thinking}
The first important skill is critical thinking, that is, that you have maintained an objective position by weighing up all sides of an argument, evaluating its strengths and weaknesses, and testing how sound the claims made and their supporting evidence are.

At its essence, critical thinking is the skill of systematically asking questions that get you ``under the hood'' of the research -- perhaps into the hidden crevices into which no one has looked before. Part of this is searching for a lack of evidence or poor reasoning behind an argument, evidence or the application of other research tools, instead of accepting what you read `at face value.'

Critical thinking develops with time and practice. The trick is a balance of scepticism: choosing the arguments or evidence to take issue with -- those that are less solid -- and leaving the rest.

%\begin{tip}You may not believe this, but there's a notion in software engineering of ``coding bad smells'' that was introduced around the turn of the century. Equally unbelievable is that coding bad smells arose from the parenthood experiences of one famous software engineer\footnote{Which you can read about here: \href{http://www-public.tem-tsp.eu/~gibson/Teaching/Teaching-ReadingMaterial/BeckFowler99.pdf}{http:\slash \slash www--public.tem--tsp.eu\slash \ensuremath{\sim}gibson\slash Teaching\slash Teaching--ReadingMaterial\slash BeckFowler99.pdf}\footnote{\href{http://www-public.tem-tsp.eu/~gibson/Teaching/Teaching-ReadingMaterial/BeckFowler99.pdf}{http:\slash \slash www-public.tem-tsp.eu\slash \ensuremath{\sim}gibson\slash Teaching\slash Teaching-ReadingMaterial\slash BeckFowler99.pdf}}}, he ``was under the influence of the odors\emph{[sic]} of his newborn daughter at the time{\ldots}''. The point was that an experienced coding specialists just know when something is wrong with code and needs to be looked at. At the beginning of your research career, you're yet to have that experience. But read, anyway, and see if anything `smells' to you. $<$This really isn't very good$>$\end{tip}
%
As an academic author, critical thinking will benefit your ability to build stronger arguments, avoid bias and link your claims to appropriate supporting evidence, so it is well worth picking it up as a skill.

\paragraph{Establishing connections}
The second skill you must demonstrate is that of being able to draw links between the articles that you have found, a process you started in Stage 1 by constructing a summary-comparison matrix\footnote{Refer back to Section~\ref{sect:Stage1AssimilatingAndAnalysing} if necessary.} and identifying emerging themes.

As you have experienced in Stage 1, this is more than summarising each article in turn: it is about identifying the things that two articles agree on and disagree on, then do the same for three, then four, etc., and begin grouping articles together under the various different themes that you have found, those themes being relevant to your research problem.

At some point, you'll arrive at a collection of more or less definitive unifying themes to which you can assign the papers you are reading. At this point, you'll have moved on from an article-based process to a theme-based process and your synthesis will be really coming together. So much so that, perhaps\footnote{We`re getting a little ahead of ourselves here -- you won't have a complete literature review at this point, but you may start to feel that you can see how, eventually, it could come together{\ldots}}, the themes that you have found could  be used as the titles of sections of your literature review. Your thematic critical reflection on the literature will then be the content of those sections, which will continue to grow and grow into the completed literature review.

\paragraph{Critical writing}

Many authors of fiction use descriptive writing to give vivid, detailed descriptions of their characters in the hope you will feel some empathy for them. Being able to write descriptively is a great skill for a fiction author to have.

Descriptive writing is also an essential part of academic writing, but its use is very different: in academic writing descriptive writing is used to set the context and to provide any existing evidence behind an argument you are developing. The key word here%\footnote{You might use different words to those used in the original, for instance, to shorten the original, but your intention will be to retain the original meaning.} 
is \emph{existing}: your descriptive writing should not change the sense of what someone else has written. 

This does not mean that you shouldn't interpret or reinterpret what someone else has written if your additions add value to what someone else has written. That's the key to analysis, synthesis, and evaluation, skills in which are good things to demonstrate!

A certain amount of description is therefore necessary in any academic writing. But to present new knowledge, your examiner will be looking for something more than description -- they will be looking for a critical approach, by which you will build\footnote{You will be able to add in your generated evidence too, of course.} new arguments using what has gone before by, for instance, analysing, synthesising, and evaluating.

A perfect critique of an article will have the following components:

\begin{itemize}
\item your introduction to the paper, by saying what is involved, where it takes place, or under which circumstances. %\textbf{Description answers the questions: what? who? where? when?}

\begin{example}{The perfect critique -- introduction} Kirlappos and Sasse (2012) discuss the implications of users visiting fake websites, concluding that trying to get them to stop doing so could prove difficult without appropriate user awareness training. [\ldots]
\end{example}


\item your analysis, which gives your perspective on the paper, perhaps highlighting how the paper comments on your focus in reading, including its strengths and weaknesses from the perspective of the topic. %\textbf{Analysis answers the questions: how? why? what if?}

\begin{example}{[\ldots] analysis} This training should be automated, involving ‘fake’ phishing e-mails being sent to users in order for them to compare the fake e-mail to a legitimate e-mail and understand the differences. [\ldots]
\end{example}


\item your synthesis, which explains how the parts fit into your research context, giving reasons, making comparisons, and highlighting relationships with other papers. %\textbf{Synthesis answers the questions: where else? which relationships?}

\begin{example}{[\ldots] synthesis} This can be contrasted with San Nicolas-Rocca\& Olfman (2013, p.84) who state that users should be trained on understanding laws and regulations as well as organisational policies and guidelines which define their specific responsibilities. [\ldots]
\end{example}


\item your evaluation of the strengths and weaknesses of the paper from your perspective, the implications that can be made for your purposes, and the impact and value to your research. %\textbf{Evaluation answers the questions: so what? what next?}

\begin{example}{[\ldots] evaluation} This false dichotomy is in danger of losing something, however, in that training can be both instructive on the nature of `fake' emails and on the formal understanding of users' responsibilities in regards to cyber breaches.	
\end{example}
\end{itemize}

The boundary between these critique elements can sometimes be fuzzy, of course, in the sense that in-depth descriptions may well start to be analytical and synthetic, and some analysis and synthesis may include a level of evaluation. Table~\ref{tab:kindOfAcademicWriting} provides some practical examples which may help you distinguish between description, analysis, synthesis and evaluation in academic writing.

\begin{table}[htbp]
\setlength{\tymax}{0.5\linewidth}
\centering
\small
\caption{Types of academic writing (adapted from \textcite{Cottrell2005}.)\label{tab:kindOfAcademicWriting}}
\begin{tabulary}{\tablewidth}{@{}LLLL@{}} \toprule
 \textbf{Descriptive} & \textbf{Analytic}&\textbf{Synthetic} & \textbf{Evaluative} \\
\midrule

 What happens or what something is like & & Makes comparisons & Identifies significance \\
 Tells a story or the order in which things occur & Explain why things work the way they do & & Demonstrate relevance \\
 Explains how to do something & & Gives reasons for choices & Draws conclusions \\
 What a theory says or how something works & & Structures information based on established criteria & Weighs pieces of information against one another \\
 Lists things, alternatives and options, etc. & Shows relations between pieces of information, parts of a system & & Highlights strengths and weaknesses \\
 Describes a system or its components & Demonstrates how a theory works & & Considers wider implications \\
\bottomrule

\end{tabulary}
\end{table}

\begin{example}{Revisiting the academic writing in our theme summary}
In Stage 1, we produced the following summary around the theme of modelling learning trajectories with curriculum analytics. This summary uses a mix of all the kinds of academic writing indicated in Table~\ref{tab:kindOfAcademicWriting}. You will consider the details in the next activity.
\begin{quotation}
	Several authors have considered modelling and analysing students' learning trajectory through a programme of study, in order to understand how students progress or otherwise through their study and the learning outcomes they achieve in doing so. Such an understanding can then be used to inform scholarship and reflection around curriculum and its design, and inform possible changes. 
	
For instance, Dawson and Hubball (2014) deploy social network analysis techniques to identify and visualise the most common learning pathways followed by students within complex curriculum structures, in which many links may exist between the various curriculum components. They suggest that their proposed tool could  be used by curriculum practitioners to study student progression and completion across different pathways, and the extent students acquired the expected learning outcomes, although their study does not evaluate the extent that might be the case. 

Similarly, Salazar-Fernandez et al. (2021) use process mining to extract students' educational trajectories from historic data: in this case, their aim is to understand which trajectories are more likely to result in late dropouts. In their proof-of-concept tool evaluation over a specific data set, they achieve some positive results with the tool providing a strong indication that students taking a study break before resitting a failed module are most likely to drop out. Further research is needed to apply and evaluate their tool in other settings. 

Somewhat distinct from these studies is the work of McEneaney and Morsink (2022), who propose a simulation tool, based on Coloured Petri Nets, to be used as a design tool to help curriculum practitioners to test the possible effect on learning of envisaged curriculum changes, such as including or removing modules or study pathways in an existing programme. As another proof-of-concept tool, the work still requires wider application and evaluation. 

Finally, both Greer et al (2016) and Molinaro et al. (2016) focus on potentially useful visualisations for curriculum practitioners. In particular, Greer et al (2016) introduces the ``Ribbon tool'', based on Sankey diagrams, for visualising student flows through academic programmes, with interactive capabilities which allow practitioners to study and compare specific student demographics.  The same tool is recommended by  Molinaro et al. (2016), alongside other visualisation tools, based on both students and course data, to allow practitioners explore both curriculum features and students' attainment. Both articles are part of the proceedings of the very first Curriculum Analytics workshop, in 2016, which also explain why they contain primarily poof-of-concept work and suggestions for future research.

Overall, this collection of articles contains some interesting ideas as to how curriculum and student data could be combined and analysed through the application and development of bespoke Curriculum Analytics tools. They all appear, however, quite preliminary studies, which is also an indication that this remains a young field of study where much research is still needed.
\end{quotation}
\end{example}

\begin{question}[subtitle={Activity: Distinguishing between different types of academic writing}] Consider the summary above. Write down examples of each kind of academic writing you can find with reference to Table~\ref{tab:kindOfAcademicWriting}.
\begin{solution}
We've picked the following examples:
\begin{itemize}
	\item descriptive: ``Dawson and Hubball (2014) deploy social network analysis techniques to identify and visualise the most common learning pathways followed by students within complex curriculum structures, in which many links may exist between the various curriculum components''. This simply summarises what those authors do.
	\item analytic: ``Several authors have considered modelling and analysing students' learning trajectory through a programme of study, in order to understand how students progress or otherwise through their study and the learning outcomes they achieve in doing so. Such an understanding can then be used to inform scholarship and reflection around curriculum and its design, and inform possible changes.'' This highlights the connections between modelling, understanding and scholarship across the different articles. 
	\item synthetic: ``both Greer et al (2016) and Molinaro et al. (2016) focus on potentially useful visualisations for curriculum practitioners. [...] Both articles are part of the proceedings of the very first Curriculum Analytics workshop, in 2016, which also explain why they contain primarily poof-of-concept work and suggestions for future research.'' This brings together work which shows strong similarities.
	\item evaluative: `` Overall, this collection of articles contains some interesting ideas as to how curriculum and student data could be combined and analysed through the application and development of bespoke Curriculum Analytics tools. They all appear, however, quite preliminary studies, which is also an indication that this remains a young field of study where much research is still needed.''. This expresses a judgement on the maturity of the work summarised and more widely its field of study.
\end{itemize}
\end{solution}\end{question}
%%Hack to correct tcbox behaviour
\color{black}

You should do a similar analysis of something you have written.

\begin{question}[subtitle={Activity: Types of academic writing in your summaries}] Consider a couple of theme summaries you have written. Which kinds of academic writing have you used?
\begin{guidance} 
If you find primarily descriptive writing, you should consider having another go at your summaries to improve the balance between description, analysis, synthesis and evaluation. 
\end{guidance}\end{question}
%%Hack to correct tcbox behaviour
\color{black}

Having mastered these key skills, your completed literature review will be a self-contained piece of academic writing, which shows your critical thinking and mastery of academic writing skills, and through which you:

\begin{itemize}
\item demonstrate your understanding of key ideas and their significance to your research, particularly to framing and justifying your research problem

\item relate different ideas to each other, including arguments and counter-arguments,

\item reason through the evidence to argue the possible contribution to knowledge your research can make.
\end{itemize}

In this way, a completed literature review demonstrates your ability to synthesise from the academic literature.


\section{Core practice for academic writing}

Academic writing has certain core practices that you should be aware of and apply consistently in your project work. They are:

\paragraph{Use critical rather than descriptive language}

Your writing should be based on well-formed arguments, i.e., on claims that are supported by evidence\footnote{Either that of others or that you have discovered yourself. We look at academic arguments in detail in Section~\ref{sect:arguments}.}, on comparing and evaluating alternative arguments, and on forming judgements on the basis of that evidence. Your writing will therefore favour analysis, synthesis and evaluation over descriptions.

In your descriptions you should also focus on essential details and keep general background information to a minimum.

\paragraph{Be clear and precise}

Your writing must make it easy for the reader to follow your arguments and grasp the points you are trying to make. You should:

\begin{itemize}
\item avoid long, over-complicated, poorly punctuated sentences;

\item be accurate in what you report;

\item be precise; avoid ambiguity and vagueness;

\item clearly define terms and concepts that may be open to more than one interpretation to avoid misunderstanding;

\item keep your audience in mind: avoid jargon and, when using technical terms\footnote{If there are a lot of technical terms, you might consider keeping a glossary.}, explain any that your readers are unlikely to be familiar with.
\end{itemize}

\paragraph{Order your topics appropriately}

You need to consider carefully which information your readers need to read first to make best sense of the topics you will introduce\footnote{We get to how to structure an argument later in this stage}. Start with less complex\footnote{You will have experienced which are the simpler and which are the more complex topics in your reading so you will already know in which order to place them.} topics and build on them towards more complex ones. Later topics may depend on earlier ones, and this can give a hint to their presentational order, too.

%%2023-09-28: jgh This will come later
%Often, it's good to present the points that support your own argument first so that you establish your case early in the mind of the reader.
%
%Getting the order right will take time and perhaps many iterations. As you write, you should take time to step back and think about whether your arguments follow logically from each other, or whether moving them around may make your line of reasoning clearer to the reader.

You can make your reader's life easier if similar topics are grouped together in your writing – otherwise, they may get the impression that you are repeating yourself, or they may miss important connections between them. So, group similar topics together.

%, including any analysis and evaluation that relates to them, as well as comparisons and counter arguments.

\paragraph{Signpost your narrative}

Another way of making your reader's life easier is to \textit{signpost} your writing, i.e., provide explicit clues to avoids the reader getting lost while reading your work. It applies both to the overall structure of your document and to individual arguments within each of its sections.

Common signposting practices at chapter, section or document level are:

\begin{itemize}
\item careful choice of headings and sub-headings

\item `setting the scene'\footnote{Journalists will often structure their narrative in the following way: they will
\begin{itemize}
\item tell you what they're going to tell you
\item tell you
\item tell you what they've told you.	
\end{itemize}
%
This is signposting at is best.} at the start of a chapter or major section to provide a roadmap of what comes next

\item summarising key points at the end of a section/chapter.

\end{itemize}

%todos ok to here

Another form of signposting is at the level of an individual argument, when appropriate words or phrases (see Table~\ref{tab:signalwords}) are used to help the reader understand where they are in an argument.

\begin{table}[htbp]
\caption{`Signal words' used in academic writing, inspired by \textcite{Cottrell2017}\label{tab:signalwords}}	
\begin{tabulary}{\textwidth}{@{}LL@{}}\toprule
\textbf{Function} & \textbf{Possible word or phrase to use} \\\midrule
\textbf{introducing an argument, a description, a section or a chapter} & first, firstly, first of all, to begin with, initially \\
\textbf{reinforcing similarities/arguments} & similarly, equally, in the same way, also, for example \\
\textbf{adding further evidence/arguments} & furthermore, moreover, in addition \\
\textbf{introducing alternative evidence/arguments} & alternatively, however, on the other hand, differently \\
\textbf{highlighting choices} & either/or, neither/nor \\
\textbf{contrasting ideas/arguments} & instead, by contrast, conversely, on the one hand [...] on the other \\
\textbf{drawing conclusions} & therefore, as a result, as a consequence, in conclusion, consequently, because of this \\
\bottomrule
\end{tabulary}
\end{table}


\begin{question}[subtitle={Activity: Signposting in our theme summary}] Consider, once again, our summary on modelling students' learning trajectories. Write down any form of signposting you can identify.
\begin{solution}
This is what we have noted:
\begin{itemize}
	\item there is an opening statement to set the scene 
	\item `For instance' is used to introduce a specific example which illustrates a point previously made
	\item `Similarly' is used to introduce work with a similar aim as the previous work mentioned
	\item `Finally' introduces the remaining work reviewed
	\item `Overall' opens the concluding statement, which offers some summary observations
\end{itemize}
\end{solution}\end{question}
%%Hack to correct tcbox behaviour
\color{black}

You can do the same for your own summaries.

\begin{question}[subtitle={Activity: Signposting in your summaries}] Consider a couple of theme summaries you have written. Which forms of signposting have you used?
\begin{guidance} 
Consider whether your signposting could be improved by applying the guidance in this section.  
\end{guidance}\end{question}
%%Hack to correct tcbox behaviour
\color{black}

\paragraph{Write good, grammatical text}

This is an essential characteristic of all written work you are expected to produce. You should therefore proofread carefully all your writing before submission to remove as many grammatical errors and typos as possible.

With modern tools, producing good grammatical text isn't difficult and your examiner will appreciate good grammatical writing. In fact, given the availability of good tools\footnote{Misquoting Star Wars: \enquote{Use the tools, Luke, use the tools.}}, they will, most likely, expect you to write perfect natural language prose. 

%todos ok to here



\paragraph{Use an appropriate format}

You should be sure to understand the required final format for your dissertation. For instance, all your citations and references should comply with the bibliographical style required. Your dissertation pages should be numbered, and all your figures and tables should also be numbered and accompanied by appropriate captions. 

\begin{question}[subtitle={Dissertation format requirements}]
	Check the requirements your course places on the final form of your dissertation then answer these questions\footnote{It may be that you can't answer these questions because they are not part of the requirements. In this case, we've suggested an answer for you.}:
	%
\begin{enumerate}%[start=0,label={(\bfseries R\arabic*):}]
\item How many words should your dissertation be?
\item What font size should you use?
\item What should the line spacing and margins be?
\item How should you number your dissertation pages?
\item How should you number figures and tables?
\item Are there any stylistic requirements on headings and sub-headings?
\item Are you allowed to include additional material in appendices?
\item How should references and citations be formatted?
\item Is there specific information that you should include in your dissertation title page?
\end{enumerate}

\begin{solution}
Our university has no single format for all Masters courses. However, looking at one specific capstone project guidelines, we have found the following:
%
\begin{enumerate}%[start=0,label={(\bfseries R\arabic*):}]
\item The dissertation should be between 10,000 and 15,000 words, excluding references and appendices
\item The font used should be 11 or 12-point Times New Roman, with 11 points recommended
\item 1.5 line spacing should be used and margins should be approximately 2cm 
\item Pages should be numbered, including references and appendices. Lower-case Roman numerals – iii, iv, v, etc., should be used on the preliminary pages, and Arabic numerals starting from page 1 should be used from the beginning of Chapter 1 
\item All figures and tables should be numbered, with sequential numbering  within each chapter, for instance, the figures in Chapter 2 would be Figures 2.1, 2.2, 2.3, etc. Also, they should have descriptive captions: below them for figures and above them for tables
\item Headings and sub-headings should be numbered with sequential numbering within each chapter, for instance, in Chapter 3, you could have sections 3.1, 3.2, etc. each with sub-sections, say, 3.1.1, 3.1.2, etc. You should avoid sub-subsections
\item Appendices are allowed and should be used to provide extra materials in support of the dissertation body. Appendices are not assessed and should be used sparingly
\item References and citations should use the Harvard bibliographical style
\item The title page should include:
%
\begin{itemize}
\item the full title of the dissertation
\item your full name
\item your university identifier
\item the degree for which it is submitted
\item the date (consisting of month and year) of your submission
\item the total word count.
\end{itemize}
In addition, there should be a short statement declaring that no part of the dissertation has been submitted for a degree or other qualification.
\end{enumerate}
\end{solution}
\end{question}
%%Hack to correct tcbox behaviour
\color{black}

\paragraph{Avoid plagiarism like the plague!}

In describing your research, you will need to distinguish it clearly from the work of others.

Part of this is to identify clearly the new contribution to knowledge your work makes: you can't make a new contribution to knowledge that someone else has already made\footnote{Duh!}.

Another part of this relates to behaving ethically in acknowledging all sources you have used. This is known as attribution and its role is to give credit where credit is due, avoiding any possible accusation of plagiarism, that is passing off the work of others as if it were your own. In a dissertation, the correct form of attribution is through referencing, and there are two common ways to report cited work:

\begin{itemize}
	\item Paraphrasing. This refers to rephrasing in your own words what an article says, still ensuring that the article is appropriately cited. 
\item Using quotation marks. You can use quotation marks to identify text reproduced without change from a cited article. In this case you can use the words of the article \textit{verbatim}, as long as they are enclosed within quotation marks.
\end{itemize}

Both can be used, but an excessive use of quotations may interrupt the flow of the narrative making it harder for the reader. As a rule of thumb, you should use quotations in your literature review if the wording used by the authors of the article really matters, perhaps because they introduce a significant new term or concept or use particularly suggestive language. Otherwise, we recommend paraphrasing, which may help you convey the essence of what you have read more concisely or even more clearly. 

If you follow these simple practices consistently, you will avoid most unintentional plagiarism and your work is more likely to comply with your course requirements and expectations. 

\begin{question}[subtitle={Ways to report cited content}]
Consider the following statements:

\blockquote{A summary-comparison matrix, introduced by \textcite{Sastry2013}, can be used as an efficient tool to keep track and compare the content of articles reviewed. Similar information is recorded for each paper, such as the research aim, research methods or key findings,  organised in a matrix form which facilitates their comparison.}

and 

\blockquote{\textcite{Sastry2013} define a summary-comparison matrix as `a data organizer that helps students to extract relevant information from research papers, categorize those extractions and visualize how these disparate extractions are related to each other' }

Write down how they differ in the way they report the cited work.
\begin{solution}
The former uses paraphrasing; the latter includes a verbatim quotation from the article.	
\end{solution}
\end{question}
%%Hack to correct tcbox behaviour
\color{black}

A third part is to avoid \textit{accidental} plagiarism, which may result from not keeping tabs on where ideas/arguments/logic/processes/etc. came from, and what your knowledge contribution to those is. This is to show that while intentional plagiarism is a deliberate attempt to deceive, something that both universities and publishers take very seriously and may have severe consequences, plagiarism doesn't need to be intentional to be classed as plagiarism. Accidental plagiarism is usually the result of poor organisation or sloppy review practices, and can be avoided by being careful and systematic when reading articles, and writing and organising your notes and summaries.

The final part is to avoid self-plagiarism, which is where you re-use already existing material that you yourself have published, without clear attribution\footnote{Clearly, this is not relevant if you haven't published before.}.


\begin{tip}
The guidance we give above is much easier to implement at the time of creation, not at the end of the writing up process. Imagine not knowing where a quote on page 35 came from when you're writing up and the difficulty you could have in trying to find it retrospectively.

Given the importance academics place on plagiarism, the risks of not being proactive in avoiding it are very high. Universities take plagiarism very seriously. You should too.
\end{tip}

%todos ok to here
%We distinguish between:
%%
%\begin{itemize}
%\item a citation, which is a short-cut that appears in the main body of the text to refer to a specific source
%
%\item a reference, which is the full bibliographic information of a source you cite in your text. References are usually collected in a section at the end of a dissertation, article, report, etc.,
%
%\end{itemize}
%

\begin{question}[subtitle={ACTIVITY: Looking up your University's plagiarism policy}] Look up your university policy on plagiarism and any disciplinary processes related to it.

\begin{solution}Our university has strict policy on plagiarism. Intentional plagiarism can lead to severe disciplinary actions, from failing study modules to be expelled from a course. Early on in study, poor academic practices can be addressed by providing extra study support. Repeated offences will incur in disciplinary action, decided on by an Academic Conduct Officer, including failure.
\end{solution}\end{question}
%%Hack to correct tcbox behaviour
\color{black}

\section{Develop your arguments!}\label{sect:arguments}
The development and presentation of academic arguments forms the core of academic writing. But what exactly is an academic argument? To answer this question, we look at the model proposed by Booth, Colomb and Williams in \textcite{Booth1995}: we will refer to this as the BCW model. 

\subsection{The BCW model}

In the BCW model, an academic argument has at least 2 parts – the \textit{claim} and the \textit{evidence} – and might, in more complex arguments, have up to 3 other parts – called the \textit{reason}, the \textit{warrant} and any \textit{qualifications}. 

Their model is shown in Figure~\ref{fig:argument}, and the parts are:
%
%
\begin{enumerate}%[label={(\bfseries \arabic*):}]
\item The \textit{claim}, which is a point of view and needs support with... 

\item ... \textit{evidence}, which provides the grounds on which the claim is made, and ...

\item ...the \textit{reason}, which is why we believe the claim to be true.

\item The \textit{warrant} explains how the reason is relevant to the claim.

\item The \textit{qualifications} are concessions which may limit what is being claimed, for instance by acknowledging objections, alternatives, etc.
\end{enumerate}
%

In dealing with qualifications, you might need to make further arguments, in which case the process of building an argument may become recursive and so much more complex to develop and present.

\begin{figure}[htbp]
\centering
\includegraphics[width=0.8\textwidth]{BCW argument form.pdf}
\caption{The five elements of an academic argument, adapted from Booth et al., 1995. \todo[inline]{LR: Needs redrawing as it is currently a merger from a couple of illustrations in that book - also needs to bring out which parts are mandatory.}\label{fig:arguments}
}
%Description
%
%The figure summarises the five elements of an academic argument and their relationship.
%
%End of description
\label{fig:argument}
\end{figure}

%todos ok to here

%In simple arguments, the warrant – that the reason is relevant to the claim – does not stating if it is obvious. Moreover, for such arguments, the qualification might also be missing.

A fully developed BCW argument is a beautiful thing. Here's a short example\footnote{Adapted from \textcite{BCW1995}.}


\begin{example}{The need for a TV `watershed'}%\todo[inline]{Needs revisiting}}
%
\begin{description}[style=nextline]
%\DrawEnumitemLabel
\item [Claim] showing violence on TV should  be allowed only after the 9pm 

\item [Reason] TV violence can have harmful psychological effects on children 

\item [Evidence] Smith (1997)\textcite{Smith1997} found that children ages 5–7 who watched more than three hours of violent television each day were 25 percent points more likely to say that what they saw on television was ``really happening''

\item [Warrant] If children are protected from watching violence on TV they will be less likely to see violence as a normal part of day-to-day life

\item [Qualifications] The following are qualifications that I have considered:%
\begin{itemize}
\item that a child interprets something on TV as ``really happening'' does not necessarily mean that they will try to emulate it. 
\item Not all children are impressionable. 
\item Violence \textit{is} a normal part of day to day life.
\end{itemize}
\end{description}
\end{example}

%todos ok to here

Qualifications can improve a claim – reducing its scope, for instance, making it more acceptable – but may also require further arguments to be made. For instance, a rebuttal to \enquote{Violence is a normal part of day to day life} might be the counterclaim that \enquote{To reduce violence in day to day life, we should insulate children from it so that the next generation isn't so likely to see violence as a justified response to problems.} and so to other arguments. These might be structured the same way until all qualifications are discharged to your satisfaction.

\begin{question}[subtitle={Activity: Differentiating between reason and evidence}] Think about the example above, write down what you think the difference between reason and evidence is.

\begin{solution}According to Booth, Colomb and Williams (1995), reasons are things we think up in our mind, while evidence is somewhat ``out there'' for everybody to see and examine. While in everyday casual conversation, we can often support a claim with just a reason, that should not be the case in academic research where reasons should be backed up by evidence, as your research audience is unlikely to accept your reasons at face value.
\end{solution}\end{question}
%%Hack to correct tcbox behaviour
\color{black}

\subsection{Arguments and narrative}

In your literature review, and your dissertation as a whole, your arguments won't be presented in the BCW form above, of course: the model is only a useful device to help you distinguish between the essential elements of an academic argument. Your dissertation presentation will be in the form of a continuous narrative instead. However, it is essential that there is a strong connection between your narrative and your academic arguments. In this section, we consider ways the BCW model can help you write narrative which allows your arguments to shine.

We can start for the theme summaries from Stage 1. Let's look at an example.    

\begin{example}{Finding the academic argument in our theme summary}
Let's consider once again our summary around the theme of modelling learning trajectories with curriculum analytics. Let's `extract' from the narrative the argument made using the BCW model. It goes like this:

\begin{description}[style=nextline]
%\DrawEnumitemLabel
\item [Claim] modelling and analysing students' learning trajectories can inform scholarship and reflection around curriculum, its design, and possible improvements.  

\item [Reason] such modelling and analysis helps us understand how students progress or otherwise through their study and the learning outcomes they achieve in doing so 

\item [Evidence] the works cited demonstrates how this can be done

\item [Warrant] if we understand which curriculum characteristics prevent students from succeeding on programme of study, then we could re-design the curriculum to address the problem 

\item [Qualifications] All cited work is based on limited case studies, so that we have no evidence of wider applicability or generality; also some of that work is little more than a proof-of-concept aimed at suggesting new avenues for research rather than proposing a mature approach.
\end{description}
\end{example}

In this example, by matching our summary narrative to the BCW structure, we can check that all necessary elements of the argument we were trying to make, particularly its claim and evidence, are in place. In the example, our argument was complete with respect to the BCW model, in that all elements were present. Should that not be the case, however, the BCW model would have helped us identify how the argument could be developed further. 

It's now time for you to have a go. To do so, you can start from the theme summaries you have already written: 

\begin{question}[subtitle={Activity: Constructing your academic arguments}] Consider your current theme summaries. Extract and write down their academic arguments according the BCW model. Identify those which are already well developed and those which require further work, indicating what is still needed.
\begin{guidance} Don't worry if you can't include all the five elements for each argument, but try to capture both your claim and any reasons and evidence in support in all cases.
It is possible that your summaries contain more than one academic argument each: in such case, ensure you apply the BCW model to each of them.
Constructing academic arguments is a fundamental skill which becomes easier with practice. Therefore we suggest you spend up to 2 hours on this activity.
\end{guidance}\end{question}
%%Hack to correct tcbox behaviour
\color{black}

 You have seen how, given some narrative you have written, it should be possible to clearly identify all the elements of the argument you were trying to convey. Vice versa, given an argument based on the BCW model, the real skill of writing it down is to find a form of narrative that combines all components together in prose that is easily digestible\footnote{Meaning that your reader – cough, examiner, cough – can easily engage with it.}. If you find this conversion difficult, you could start with the following template\footnote{Adapted from Figure~\ref{fig:argument}} for creating a narrative around a BCW model argument:

\newcommand{\BCW}[5]{I claim that #1, because #2, based on #3. The warrant that allows me to connect my reason and claim is #4. I acknowledge the following qualifications to my claim (which I deal with later): #5}

\BCW{\textit{my claim}}{\textit{my reasons}}{\textit{this evidence}}{\textit{my warrant}}{\begin{itemize}\item\textit{qualification 1}\item \textit{qualification 2} \item ...\end{itemize}}

\begin{question}[subtitle={Activity: Beginning an evidence narrative}]
Put the example TV violence argument into the BCW narrative form based on the template.
\begin{solution}
You should have something like:

\blockquote{
	\BCW{showing violence on TV should  be allowed only after the 9pm}{TV violence can have harmful psychological effects on children}{Smith (1997)\textcite{Smith1997} found that children ages 5–7 who watched more than three hours of violent television each day were 25 percent points more likely to say that what they saw on television was ``really happening''}{If children are protected from watching violence on TV they will be less likely to see violence as a normal part of day-to-day life}{\begin{itemize}\item that a child interprets something on TV as ``really happening'' does not necessarily mean that they will try to emulate it. 
\item Not all children are impressionable. 
\item Violence \textit{is} a normal part of day to day life.
\end{itemize}}}
\end{solution}
\end{question}

As you can see, the model gets close to a narrative, but there are some grammatical errors – \enquote{... based on \textcite{Smith1997} found that ...} – which need smoothing out, and it could get very repetitive if each and every argument you present has this form. 

There are many techniques for improving your narrative. One of the most powerful is to read what you have out loud to a live listener and have them try to follow your argument. 

\begin{question}[subtitle={Activity: A willing audience}]
Read the result in its raw form – \textbf{as written} – to a willing family member/friend/colleague and ask for their opinions on it. Reflect on both their reactions to your reading and the feedback they provide.
\begin{guidance}
	In your reflection you could consider the following questions. Was your audience engaged or did they fall asleep? Did their feedback help you understand how the narrative should improve? Will they come back and help with future versions? Did the template work for you? 
\end{guidance}
\end{question}

It's unlikely that your first cut at creating a narrative using the template will result in something you can use directly in your dissertation. Instead, you should iterate to make it more compelling, using the feedback from your 'critical friends' to help you to do so.

\subsection{Logical fallacies and cognitive bias}

The BCW model is a useful device to help you structure your academic arguments. In making you think deeply and critically about what you are trying to claim and why, you have a better chance to write good argument. There are, however, things you still need to guard against. 

Firstly, there may be logical fallacies in your arguments: these are errors of reasoning that can undermine your claim. The most common ones are\footnote{Many more exist -- you may like to do a web search to find out more.}:

\begin{itemize}
	\item circular reasoning\footnote{A.k.a. begging the question or begging the claim.}, in which the supporting evidence is just a restating of the claim.
	\item hasty generalisation, in which a claim is made based on insufficient evidence.
	\item sweeping generalisation, in which a claim obtained from evidence within a specific situation or contest is assumed to be true in other situations/contexts.
	\item \textit{post hoc ergo propter hoc}\footnote{This is a Latin phrase meaning ``after this, therefore because of this''. [More explanation here of why this is bad...]}, which claims a causal relation between two phenomena because one happens after the other.
	\item false dichotomy\footnote{AKA black and white fallacy or false dilemma.}, in which an either--or claim is constructed which assumes two phenomena are mutually exclusive where that's not the case.
\end{itemize}

\begin{question}[subtitle={Activity: Logical fallacies}]
Think of each definition above and write down a possible example.
\begin{guidance}
If you struggle to think of appropriate examples, do a web search to get some inspiration. However, you should still try and come up with your own examples too.
\end{guidance}
\begin{solution}
This is what we have come up with:
\begin{itemize}
	\item circular reasoning: collecting quality data is difficult (claim) because quality data are difficult to collect (evidence, restating the claim)
	\item hasty generalisation: camomile tea cures insomnia (claim) because when I drink camomile tea in the evening I sleep well (insufficient evidence).
	\item sweeping generalisation:  universities don't generate quality student data (claim) because I have observed this is the case in my university (evidence from a specific context)
	\item \textit{post hoc ergo propter hoc}: I should not watch Italy playing rugby (claim), because every time I watch them they loose (assumed causal relationship).
	\item false dichotomy: during a pandemic we either save lives or preserve the economy (claimed false dichotomy), because to save lives we must shut all economic activities down.
\end{itemize}
\end{solution}
\end{question}

Secondly, everybody is susceptible to cognitive bias, which prevents people from processing information objectively due to limited capabilities of our mind, or due to emotional responses or social norms and conditioning. You have already encountered cognitive bias in Section~\ref{sect:biasInResearch}, alongside examples of various forms of bias. Among them, confirmation bias is particularly relevant to the construction of academic arguments.

\begin{question}[subtitle={Activity: Revisiting confirmation bias}]
Go back to Section~\ref{sect:biasInResearch} and revisit the definition of confirmation bias. Write down ways in which it may affect you when constructing your academic arguments.
\begin{solution}
Confirmation bias may result in cherry-picking information which agrees most with our beliefs, opinions or preconceptions, and ignoring that which may support opposite views. This can lead us, as researchers, to conduct selective searches, actively looking for articles whose positions is in alignment with ours, or recall and interpret evidence in a way which reinforces our position, rather than maintaining an objective stance by weighing and contrasting available evidence.
\end{solution}
\end{question}


%\subsection{Organising your arguments}
%
%\textbf{LR-Note: I've reused some content above}
%
%Now that you know how a single argument can be structured, and what the main arguments from your summaries are, you should ensure they are presented in a logical manner.
%
%Here are few simple rules you could apply:
%\begin{itemize}
%	\item no earlier argument will depend on a later argument -- this avoid c
%	\item no circular argument
%	\item 
%\end{itemize}
%
%
%
%At a minumb, you should ensure that 
%
%In doing so, you should consider the following rules, 
%
%
%
%There are sequences, there are good sequences; and there are perfect sequences. Finding a perfect sequence is harder than finding a good sequence which is harder than finding a sequence. 
%
%The perfect sequence will meet the following rule:
%%todos ok to here
%
%%
%\paragraph{Rule 1 of argument construction:} \textit{no earlier argument will depend on a later argument}.
%
%That's it. With this rule, you're building an argument tree\footnote{Technically, a directed acyclic graph, but the difference shouldn't worry you.} making your reader's job easier: when they come to try to understand a later argument they will have everything they need to have to do so.
%
%Anything argument structure that breaks this rule – a tree with cycles, for instance, in which an earlier argument depends on a later one – makes your readers' job harder.\footnote{Not impossible, though.}. 
%
%\begin{example}{A circular argument}
%Let's look at a simple circular argument:
%
%%\paragraph{Argument:} 
%%\blockquote{Small businesses are the only way the economy can truly work well. The stimulation and employment they provide is invaluable. And, it is this stimulation and employment which allows small businesses to truly thrive.}
%%%
%%\begin{description}%[label={Premise \arabic*:}]
%%\item [Claim 1] Small businesses are required for an economy to function.
%%%\item [Reason] Small businesses employ people and strengthen the economy.
%%\item [Claim 2] A working economy is required for a small business to function.
%%\end{description}
%%
%%The above argues the importance of small businesses to an economy whilst claiming that the economy requires small business and entrepreneurs; and also that small business and entrepreneurs require a working economy. In the BCW model, we could restate this as:
%%
%
%\paragraph{Argument 1}
%\begin{description}%[label={Premise \arabic*:}]
%\item [Claim 1] Small businesses\footnote{Adapted from \url{https://helpfulprofessor.com/circular-reasoning-fallacy-examples/}} are required for an economy to function.
%\item [Reason 1]Small businesses employ people who pay taxes which strengthens the economy.
%\end{description}
%
%\paragraph{Argument 2}
%\begin{description}
%\item [Claim 2] A strong economy is required for a small business to function.
%\item [Reason 2]A strong economy is more able to provide the investment that small business need to grow.
%\end{description}
%\end{example}
%
%These arguments together give that small businesses are both the cause of a working economy and the consequence of one; they depend on each other. Inverting their order does change this, and so it looks like Rule 1 is broken no matter what we do.
%
%Thing is, there is an actual circular relationship between small business and economies – that they support each other is often the reason why an economy grows. The problem with this circular argument is that it tries to separate a single argument into two. Unresolvable circular arguments are often the result of trying to do this and so point to some additional thinking that is needed.
%
%\paragraph{Logical Fallacies} The structure of your argument is our focus at this point, but the construction of individual arguments is more complex, due to the existence of \textit{fallacies}.
%
%According to Wikipedia:
%\begin{quote}
%A fallacy is reasoning that is logically invalid, or that undermines the logical validity of an argument. All forms of human communication can contain fallacies. 
%\end{quote}
%
%%todos ok to here
%
%\begin{stolen}{\url{https://en.wikipedia.org/wiki/List_of_fallacies}}
%
%Fallacies are challenging to classify. They can be classified by their structure (formal fallacies) or content (informal fallacies). Informal fallacies, the larger group, may then be subdivided into categories such as improper presumption, faulty generalisation, and error in assigning causation and relevance, among others.
%
%The use of fallacies is common when the speaker's goal of achieving common agreement is more important to them than utilising sound reasoning. When fallacies are used, the premise should be recognised as not well-grounded, the conclusion as unproven (but not necessarily false), and the argument as unsound.
%\end{stolen}
%
%Unless you're trying to intentionally mislead your reader through the use of a fallacy\footnote{We recommend against this; your supervisor and examiner will be highly tuned to pick up fallacious arguments; in our experience, the finding and examination of a fallacious written argument gives academics great joy. They're easy to spot and point to poor academic research.}, it's going to be difficult for you to identify any that do creep into your writing – knowing they're there means you wouldn't use them.
%
%Look out for comments from comments from other readers – your supervisor, for instance – who might point them out. Wikipedia's page on fallacies\footnote{\url{https://en.wikipedia.org/wiki/List_of_fallacies}} is a great, regularly updated, resource that will help you understand any indicated fallacy, and the individual entries that that page links to give more information that will help you debug your writing.
%
%In the worst case, you could always ask the person that pointed a fallacy out whether the argument can be restructured to make sense, how you should do that if it is possible.
%%
%%
%%\paragraph{Spiral Arguments}
%%Another type of argument that looks circular but isn't is a spiral argument. A spiral argument again has (at least) two parts, the claim of which appears to depend on another claim which depends on the first claim. Add example here – there's one at \url{https://www.catholic.com/tract/proving-inspiration}. While there can be a get out clause of a spiral argument – a reason it doesn't break Rule 1 – a spiral argument can be less than convincing, even wrong, if the circularity in it isn't absolutely removed.
%%
%
%\paragraph{Cognitive biases}
%\begin{stolen}{\url{https://www.logicallyfallacious.com/\#null}}
%In the early 1970s, two behavioral researchers, Daniel Kahneman and Amos Tversky pioneered the field of behavioural economics through their work with cognitive biases and heuristics, which like logical fallacies, deal with errors in reasoning. The main difference, however, is that logical fallacies require an argument whereas cognitive biases and heuristics (mental shortcuts) refer to our default pattern of thinking. Sometimes there is crossover. Logical fallacies can be the result of a cognitive bias, but having biases (which we all do) does not mean that we have to commit logical fallacies. Consider the bandwagon effect, a cognitive bias that demonstrates the tendency to believe things because many other people believe them. This cognitive bias can be found in the logical fallacy, appeal to popularity.
%
%Everybody is doing X.
%
%Therefore, X must be the right thing to do.
%
%The cognitive bias is the main reason we commit this fallacy. However, if we just started working at a soup kitchen because all of our friends were working there, this wouldn't be a logical fallacy, although the bandwagon effect would be behind our behavior. The appeal to popularity is a fallacy because it applies to an argument.
%
%I would say that more often than not, cognitive biases do not lead to logical fallacies. This is because cognitive biases are largely unconscious processes that bypass reason, and the mere exercise of consciously evaluating an argument often causes us to counteract the bias.	
%\end{stolen}
%
%There is more about cognitive biases at \url{https://www.logicallyfallacious.com/} 
%
%\paragraph{Argument types}
%
%\todo{There's more stuff at \url{https://www.psychologytoday.com/us/blog/hide-and-seek/201906/arguments-and-how-they-fail} if we want to go that route.}
%
%%todos ok to here
%

\section{Developing your literature review from your theme summaries}

You should now have a collection of summaries based on the themes you have identified from the literature. You will have also revised those summaries to ensure that the academic arguments within are well-formed in relation to the BCW model. Congratulations! You now have some basic building blocks to put together your very first literature review draft.


\subsection{Developing the main body of your literature review}\label{sect:bodyOfLiteratureReview}

In piecing together your literature review it is important you consider the order in which different topics are introduced. You main goal in doing so is to ensure that the reader of your dissertation will find it easy to follow the `story' you are trying to tell --- one which provides sufficient context and justification for your research.

In doing so, you should think of your literature review as made of three distinct parts:
\begin{itemize}
	\item an introduction, which gives the reader, in outline, a sense of the detailed review you have conducted 
	\item a main body, which provides a detailed account of such review, and
	\item a summary conclusion, which provides the academic argument in support of your research, based on the evidence you have presented in the main body.
\end{itemize}

Your theme summaries are your starting point to populate the main body of your literature review. In doing so, we suggest you take the following steps\footnote{You should have both your theme summaries and concept map (see Section~\ref{sect:stage1Synthesising}) at hand}:

\begin{itemize}
\item list all the themes you have identified in no particular order

\begin{example}{Listing the themes} 
In our Stage 1 review of the Curriculum Analytics (CA) literature we identified the following themes:

\begin{enumerate}
	\item stakeholders and CA tool development
	\item modelling study trajectories and progression
	\item benefits of CA tools
	\item deriving insights from data
	\item curriculum metrics and quantitative assessment
	\item success factors for CA adoption
	\item capturing student voice
	\item CA definitions
\end{enumerate}
\end{example}

\item consider whether there are themes whose concepts overlap, and could be merged. If so, merge their themes summaries into a wider theme

\begin{example}{Merging themes} 
We found that stakeholders were prominent both in the development and adoption of CA, so we merged those two themes; we also found that the `capturing student voice' could be seen as a particular instance of the `benefits of CA tools', so we merged those two themes as well. As a result, we ended up with the following (changes in bold, also illustrated in Figure~\ref{fig:themeStructuring1}) :
\begin{enumerate}
	\item \textbf{importance of stakeholders in CA development and adoption}
	\item modelling study trajectories and progression
	\item \textbf{benefits of CA tools, including capturing student voice}
	\item deriving insights from data
	\item curriculum metrics and quantitative assessment
	\item CA definitions
\end{enumerate}
\end{example}

\begin{figure}[htbp]
\centering
\includegraphics[width=0.7\textwidth]{Figures/themeStructuring1.pdf}
\caption{Merging themes}
\label{fig:themeStructuring1}
\end{figure}

\item consider whether there are themes which rely on concepts introduced by other themes. If so, the latter should come first, and the former could follow closely. You should re-order your themes accordingly  

\begin{example}{Identifying dependencies between themes} 
All themes assumes an understanding of what CA are, so that CA definitions should come first. Also, `deriving insights from data' expands on some issues around stakeholders and CA adoption, so it should follow soon after that theme. This led to the following (changes in bold, also illustrated in Figure~\ref{fig:themeStructuring2}):
\begin{enumerate}
	\item \textbf{CA definitions}
	\item {importance of stakeholders in CA development and adoption}
	\item \textbf{deriving insights from data}
	\item modelling study trajectories and progression
	\item {benefits of CA tools, including capturing student voice}
	\item curriculum metrics and quantitative assessment
\end{enumerate}
\end{example}

\begin{figure}[htbp]
\centering
\includegraphics[width=0.7\textwidth]{Figures/themeStructuring2.pdf}
\caption{Theme dependencies}
\label{fig:themeStructuring2}
\end{figure}

\item consider which themes may be broad and generic vs those which may be narrow and  specific. Reorder the themes so that your narrative goes from the generic to the specific, while maintaining any dependencies you have already identified. 

\begin{example}{From generic to specific} 
Both CA benefits and metrics are quite generic theme, so that they could come early: this is illustrated in Figure~\ref{fig:themeStructuring3}, from more generic in top left to more specific in bottom right. Different re-ordering are possible. For instance:
\begin{enumerate}
	\item {CA definitions}
	\item \textbf{benefits of CA tools, including capturing student voice}
	\item \textbf{curriculum metrics and quantitative assessment}
	\item {importance of stakeholders in CA development and adoption}
	\item {deriving insights from data}
	\item modelling study trajectories and progression	
\end{enumerate}
\end{example}

\begin{figure}[htbp]
\centering
\includegraphics[width=0.7\textwidth]{Figures/themeStructuring3.pdf}
\caption{Theme generality}
\label{fig:themeStructuring3}
\end{figure}

\end{itemize}

By the end of this process, you should have a structure for your literature review body: each of the themes in your list can be a sub-section which you can populate with the narrative from your summaries. 

\begin{question}[subtitle={Activity: Structuring the body of your literature review}] Consider your current theme summaries. Apply the above steps to arrive at a possible structure for the main body of your literature review. Once you are satisfied with the structure, populate it with your theme summaries and read it through. Make sure there is a good flow from one section to the next. 
\begin{guidance} 
You may have to re-apply the steps to improve both structure and narrative flow between sections. Being able to structure a narrative is yet another fundamental skill which becomes easier with practice and that will serve you well when writing up your whole dissertation. Therefore, you should tread this as a substantial activity which is likely to require you several hours if not days.
\end{guidance}\end{question}
%%Hack to correct tcbox behaviour
\color{black}

\subsection{Choosing headings and sub-headings}

You should now consider the headings of your review sections and sub-sections, to ensure they are appropriate. Your headings should: 

\begin{itemize}
\item indicate, in outline, to the reader how the narrative develops from one section/sub-section to the next
\item express clearly the purpose of each
\item accurately reflect the content of each
\item be concise
\end{itemize}


\begin{question}[subtitle={Activity: Improving your current headings}] Consider your current draft of the main body of your literature review. Apply the guidelines above to improve its headings and sub-headings.
\begin{guidance}
You choice of headings and sub-headings should signpost to the reader where they are in your `story', help them follow the thread of your arguments, and allow them to locate efficiently specific content they wish to return to.
\end{guidance}\end{question}
%%Hack to correct tcbox behaviour
\color{black}

\subsection{Writing your review introduction and critical summary}
Now that the main body of your literature review is in place, at least in draft form, you can top-and-tail it with a brief introduction at the beginning and a critical summary at the end.

\begin{question}[subtitle={Activity: Writing the introduction}] Write a brief introduction to your literature review to introduce the topics you are going to cover in its main body, and their relevance in relation to your research problem.
\begin{guidance} 
Your introduction should be brief and indicate in outline the main topics you are going to cover and why. This section should be straightforward to write now that the literature review main body is in place.
\end{guidance}\end{question}
%%Hack to correct tcbox behaviour
\color{black}

\begin{question}[subtitle={Activity: Writing the critical summary}] Write the concluding section of your literature review which summarises the main findings from your reading in support of your proposed research.
\begin{guidance} 
Your critical summary should take the form of an academic argument which supports your claim that there is a knowledge gap your research is going to address. You should skim through Section~\ref{fig:arguments} to remind you of the key elements of an academic argument. Writing this section may take some time and require you to iterate until your argument is well-formed.
\end{guidance}\end{question}
%%Hack to correct tcbox behaviour
\color{black}



%%%%%%%
%LR -- used the structuring by theme above instead, so this content is not needed any longer. Instead we need something about assessing the extent the review covers the required scope
%%%%%%%
%Look back at the pacman diagram in Stage 1\todo{Introduce this in Stage 1 and explain.}. 
%
%You should now have most elements that you need to be able to describe the boundary of knowledge in the area of your research problem, including being able to focus more closely on the knowledge gap you hope to fill. 
%
%What remains is to be able to turn those elements into an engaging narrative that leads the reader to the conclusion that your contribution can – will – be a contribution to knowledge. Practising the techniques you have just learnt will be a big step towards being able to do this in a way that satisfies your reader.
%
%\begin{figure}[htbp]
%\centering
%\includegraphics[width=0.8\textwidth]{pacman}
%\caption{Pacman: the relationship between the literature and the research gap.\todo[inline]{Move to stage 1, and reference here?}}	
%\end{figure}
%
%To do this, your literature review should tell a story to the reader, supported by evidence from previous research and the work of others.
%
%
%Your argument structure in the literature review is the core of your pacman diagram, which you should now be able to describe. The fun trick to do this, is to print out\footnote{Remember printing?} – yes, literally print out, using a printer – the text of each argument and to lay them in a pacman diagram on your floor. Arguments that are far from your research focus – those that set the general context of your research – should go far from the \enquote{mouth}, those that are key to your research should give the shape of the \enquote{mouth}. 
%
%This will give you a structure for your literature review, starting from far away from the \enquote{mount} and working towards it.
%
%Unfortunately, not everyone has a printer, or a floor big enough, and so another\footnote{But much less fun!} way to achieve this is to use a mind map type structure, labelling – or otherwise distinguishing them – and put them into mind mapping software, or even something like presentation software\footnote{Such as PowerPoint for PCs, Keynote for Macs, or the free LibreOffice for all OSes.}, again building the shape of the pacman as suggested above.
%
%\begin{question}[subtitle={Activity: Developing your literature review outline}] Based on the arguments you developed in the previous activity, draw the pacman diagram for your current literature review and, from it, produce a linear outline of your narrative  which indicates which arguments to include and in which order. You should use sections and subsections to structure it.
%
%Review and revise your outline, until you are satisfied that all the arguments you wish to make are included and they follow logically from each other.
%
%\begin{guidance}
%There is no single way to go about doing this. Here are some techniques you may like to try if you find it difficult to get started:
%
%\todo[inline]{More guidance needed for this one?}
%%
%\begin{itemize}
%\item [\color{red}Added] Draw a mind map of the arguments you have developed and how they relate to each other. Use the map to group and sequence the arguments in your outline. Try overlaying your arguments onto a pacman diagram – place the more detailed ones closer to the `mouth'. Once you're happy with their positioning, you can use that to order them.
%
%\item Write a sentence/bullet point for each argument, then re-order them to ensure they follow from one another logically
%
%\item Try different ways of grouping your arguments, for instance by theme or chronologically
%
%\item Try some free writing, then read, review and re-organise to ensure there is a logical flow
%
%\item use sections (and subsections, if you need to) to group together related arguments.
%\end{itemize}
%
%You should spend up to 2 hours on this activity.
%\end{guidance}\end{question}
%%%Hack to correct tcbox behaviour
%\color{black}

%\paragraph{What's next?}
%
%Once you are happy with your outline, you can start filling in the details. Depending on how much you have read and the arguments you have been able to make, you should end up with a substantial initial draft of your literature review by the end of Stage 1\todo{or 2?}.


\section{Assessing your literature review}

It is time to assess whether you have done enough work on your literature review and can move on. If not, you will need to widen it to include what is currently missing -- there may be topics which you haven't explored sufficiently, or even some you haven't address yet. This means iterating the literature review process by searching, gathering, assimilating and synthesising more published work.

 The best person to assess your literature review is YOU\footnote{Of course, others can help too, as we shall see.}!  The following activities will help you assess your progress and guide any further work required.

\subsection{Your own assessment}

At this point, your literature review should be sufficient to describe the boundary of knowledge in the area of your research problem, including the knowledge gap you hope to fill. 

%\begin{figure}[htbp]
%\centering
%\includegraphics[width=0.7\textwidth]{pacman}
%\caption{Pacman: the relationship between the literature and the research gap.\todo[inline]{Move to stage 1, and reference here?}}	
%\end{figure}

Does your current review draft achieve this? To be able to answer this question, we recommend you apply the criteria in Table~\ref{tab:litrevcrit}.

\begin{table}[htbp]
\caption{Criteria for assessing your literature review\label{tab:litrevcrit}}
\begin{minipage}{\linewidth}
\setlength{\tymax}{0.5\linewidth}
\centering
\small
\begin{tabulary}{\tablewidth}{@{}ll@{}} \toprule
 \textbf{Criteria} & \textbf{Prompts} \\
\midrule

 \textbf{Research problem underpinning} & To which extent does it demonstrate your understanding of different facets of your research problem? \\
 \textbf{Research problem justification} & To which extent does it argue that the research problem is worth investigating? \\
 \textbf{Potential contribution to knowledge} & How clearly does it expose what the knowledge gap? How clearly does it articulate why it is significant to address it? \\
 \textbf{Logical progression} & Do sections and sub-sections structure the narrative appropriately? To which extent does it include a logical progression of arguments?  \\
 \textbf{Critical writing} & Are connections between ideas appropriately explored? Is there a good balance between description, analysis and evaluation? \\
 \textbf{Supporting references} & To which extent are all key arguments supported by appropriate references? \\
 \textbf{Format and proof reading} & Have you reviewed your writing carefully to remove typos and grammatical errors? Are all citations and references in correct bibliographical style? \\
\bottomrule

\end{tabulary}
\end{minipage}
\end{table}

\begin{question}[subtitle={Activity: Assessing your literature review}] Assess your current draft of the literature review by applying the criteria in Table~\ref{tab:litrevcrit}.

\begin{guidance}
For each criterium, use the prompts to write down your own assessment and to record what is still missing: the latter will help you identify further work you will need to carry out.
\end{guidance}\end{question}
%%Hack to correct tcbox behaviour
\color{black}


\subsection{Getting others to help you}

There are people around you that will be able to help. Gathering others' feedback will not only help you understand what still needs to be done, but it can also help you to find value in your work so far. Even if you feel the draft is scrappy\footnote{It certainly won't be in its final form as yet.}, for instance, others might be able to bring out things they like about it. 

Among them, the most important person to help you is your supervisor, as an experienced academic writer and topic expert.

\begin{question}[subtitle={ACTIVITY: Sharing your literature review with your supervisor}] Before going further with your literature review, it would be wise to ask your supervisor for comments on what you have already achieved. 

Share your current draft with them and ask them to find holes in your coverage and your arguments. Carefully gather any and all feedback you receive.
\begin{guidance}
As this point, it's unlikely that your supervisor will say \enquote{It's perfect!}\footnote{Although they may say \enquote{Well done!}, but don't worry if they don't:)} so be ready to hear one or more of the following comments\footnote{Or something close.}:
%
\begin{itemize}
\item \enquote{you should read/add the following papers to your review}
\item \enquote{this section seems out of order, perhaps it would be better elsewhere}
\item \enquote{you've missed this topic}
\item \enquote{this argument would be better expressed this way}
\item \enquote{Author X also made this argument}
\item \enquote{other comments}
\item \enquote{your conclusions are wrong because ...}
\item \enquote{you might like to speak to this colleague about this issue}
\end{itemize}
Make sure you consider their feedback carefully and take notes of all remaining work needed to develop your literature review further.
\end{guidance}\end{question}
%%Hack to correct tcbox behaviour
\color{black}


Other help might come from family members/friends/colleague that you can talk to. Like you did in Activity~\ref{act:???}, read what you have out loud to a live listener and have them try to follow your arguments. 

\begin{question}[subtitle={Activity: A willing audience}]
Read your current draft to a willing family member/friend/colleague and ask for their feedback on it. Take notes of improvements that may be needed.
\begin{guidance}
Write down their comments as they make them (but try not to interrupt the flow too much). What was their reactions? Were they engaged or did they fall asleep (again)? Will they come back and help for future versions? Make sure identify ways you could improve your draft for your readers. 
\end{guidance}
\end{question}


\section{Widening your literature review}

Our 5-stage framework assumes you develop a comprehensive initial draft of your literature survey by the end of Stage 2, with the bulk of gathering and assimilating articles happening in Stage 1. However, even if you hit gold first time with your literature search, you are likely to require more than one iteration of the literature review process in order to develop a full draft to your own satisfaction and that of your supervisor. Therefore, expect some iterations also in Stage 2. The focus of such iterations is to help you widen or deepen some aspects of your review --- something you and your supervisor will have identified while assessing your current draft.

\begin{question}[subtitle={Activity: Widening your literature review}] Select a number of topics or sub-topics for further exploration and iterate through the review process in order to gather further articles, assimilate and synthesise their content, and integrate it within your current literature review draft. Reassess the content at the end and iterate if necessary.
\begin{guidance}
The topics or sub-topics to investigate may include:
\begin{itemize}
\item those you identified while assessing your current draft
\item those from suggestions and feedback from your supervisor
\item themes or concepts currently under-explored in your concept map\footnote{The concept map was introduced in Section~\ref{sect:stage1Synthesising}.}
\item interesting ideas or sub-topics you came across in previous reading, but yet to explore
\item articles cited in work you have already reviewed, or that cite that work
\end{itemize}
At each iteration, make sure you apply the wide range of review techniques you have learnt in Stage 1 and 2. In particular, record all new entries in your BMT, alongside your notes and summaries; take good notes as you assimilate new content; update your summary-comparison and concept matrices, and your concept map; and produce appropriate theme summaries. 

As you integrate new content within your current draft, you may have to amend your arguments, or even rethink the structure of your draft. 

Alongside widening the content you should also improve your draft based on other feedback you have received, for instance improving the narrative flow or clarifying points your readers may have found obscure or poorly expressed.

Importantly, make sure you take your supervisor's comments seriously. We don't mean just taking note of what they say and implementing it, but – and this is a good habit to form – writing a response: thank them for each of their comments,  and tell them how the literature review has changed because of their input. Your response doesn't have to be long – the shortest response to a comment could be \enquote{Thank you, done!}. Doing this will make them – or anyone that comments – feel really valued. 

This is a substantial activity which, depending on how much material you still have to review, may take you several days, if not weeks.
\end{guidance}
\end{question}
%%Hack to correct tcbox behaviour
\color{black}

*******\chapter{Research design foundation}

To make a contribution to knowledge we do research. Practically, to do research, we combine a number of research tasks into a framework. Designing such framework is what we term research design. The framework will depend on the research area, the type of knowledge contribution you wish to make, your mindset as a researcher, and the opportunities and difficulties you may face along the way.

A research framework has many levels. At its foundations are its \enquote{ontology}, \enquote{epistemology}, and \enquote{methodology}: 

\begin{description}
\item [Ontology] is the philosophical study of the nature of existence and addresses the question: \enquote{What is the reality that I will research?}. Practically, ontology translates to determining what \emph{phenomena} exist in the context of  your research, the \emph{relations} that exist between them and how they group together into \textit{categories}.

\item [Epistemology] is the philosophical study of knowledge and addresses the question: \enquote{How is knowledge generated and from what sources?}. Practically, epistemology is about finding out \enquote{What people know?}, \enquote{What does it mean to say that people know something?}, and \enquote{How do people know that they know?}.

\item [Methodology] is the system of principles and methods by which you conduct research, that is, investigate, measure, and analyse your research’s aim and objectives. Methodology operationalises the \enquote{how} question of knowledge generation, so it is about devising concrete strategies to answer \enquote{How will I make my contribution to knowledge?}. 
\end{description}

As you might have guessed, given that the goal of research is to make a contribution to knowledge, epistemology and ontology are incredibly important in defining what knowledge is in any particular research context and what, in that context, can be known about. Once this choice is made, an appropriate methodology can be devised: hence, methodology depends on choices made in relation to ontology and epistemology. 

Fortunately, many others have thought very deeply about ontology and epistemology\footnote{For instance, if you're interested, you can find a fuller discussion of {Ontology} and Epistemology in the Stanford Encyclopedia of Philosophy.} and, in most areas and for the vast majority of masters-level research, their thinking will suffice. If not, we'd be left in a situation in which even an ostensibly simply statement like \enquote{That hat is blue} becomes in need of complex debate \parencite[Section 4.1]{steup2020epistemology}.

Methodology, on the other hand, is something we will spend some time on, particularly how individual research methods combine to produce knowledge contributions through research strategies. 

You should be aware that `methodology' has many meanings in the literature, including the study of research methods, which questions the assumptions that underpin their creation and application. Wikipedia says\footnote{It could almost be seen as a warning!}:
%
\blockquote{%Many discussions in methodology concern the question of whether the quantitative approach is superior, especially whether it is adequate when applied to the social domain. 
[...] A few theorists reject methodology as a discipline in general. For example, some argue that it is useless since methods should be used rather than studied. Others hold that it is harmful because it restricts the freedom and creativity of researchers. Methodologists often respond to these objections by claiming that a good methodology helps researchers arrive at reliable theories in an efficient way. The choice of method often matters since the same factual material can lead to different conclusions depending on one's method. Interest in methodology has risen in the 20th century due to the increased importance of interdisciplinary work and the obstacles hindering efficient cooperation.}
%
These are not unimportant issues to consider. However, and as for ontology and epistemology, we will leave their discussion to others, content to stand on those giants' shoulders – we take an unapologetically practical approach to research methods, limiting our discussions to what, we feel, are their important characteristics for practice. This doesn't ignore philosophical issues, however: where there are important philosophical considerations to be considered, we address them. This includes questions as to how to choose a particular research method, and what an experienced reader will expect to be answered by it. You can then craft your dissertation to meet those expectations.

%\bigskip
%
%Before we meet research methods, however, it's worth thinking more about what is the foundations of ontology, epistemology and methodology is the research \textit{paradigm}, the system of beliefs, ideas, values, and habits that guide (and constrain) a researcher's way of thinking about the world.


\section{Researcher mindsets}

Depending on your background, you may have begun your research studies with a particular mindset – that of a scientist, for instance, or as someone embedded within an organisation. This mindset will flavour your approach to research, but it shouldn't constrain it – there are many options for research and the right one for you might be outside of your current understanding.

%That's not a bad thing – many have been there before and constructed \emph{research paradigms} that characterise useful mindsets for researchers to have. 

Over time, researchers in different communities and disciplines have developed differing mindsets, which are known in the literature as research paradigms\todo{[ADAPTED from https://proofed.com/writing-tips/the-four-types-of-research-paradigms-a-comprehensive-guide/]}\footnote{A.k.a. philosophical traditions.} You can think of a research paradigm as a philosophical way of thinking, a set of shared beliefs which shape a worldview. 

We briefly outline the prevalent ones in this section --- there is a lot, lot more to be known around this topic, and this introduction only scratches the surface! We provide some references for you to start your own investigation into this fascinating and complex topic, should you wish to.

Each paradigm comes with its own ontological, epistemological and methodological choices. It is important for you to be aware of their existence as this may help you guide your research design choices, even if in practice you will mainly focus on methodological considerations. 

\subsection{Positivist and post-positivism}
The Positivist research paradigm assumes that there is a single, objective reality that can be accurately known, described and explained. 

%They also depends on \textit{deductive reasoning} to generate that knowledge: starting from general observations as assumptions they deduce specific conclusions through logical steps, and as long as the assumptions are true, then so are the conclusions reached. 

Positivism contributes knowledge as explanations of this reality, constructed from hypotheses which are confirmed through observations and measurements, hence becoming universal laws or facts. As an example, think of Newton's explanation of the action of forces on matter that is encoded as his Three Laws of Motion: these are meant as universal objective truths which apply to the natural world forever.

%Under this assumption, positivist researchers make claims of new knowledge that they compare against reality to determine the \enquote{truth}. Hence positivists seek to confirm their theories through their observations and measurements of an objective reality. 

In assuming a single, objectively knowable reality, positivism  removes the researcher as a variable in the research equation: research is necessarily limited to data generation, analysis and interpretation from an objective viewpoint as the basis of knowledge. As such, it befits research where a single objective reality can be assumed, such as the natural sciences, the physical sciences, or whenever very large sample sizes can be used to infer characteristics of a population. It leads the researcher towards quantitative methods.

%, including experiments, tests, surveys and simulations involving formal modelling based on mathematics, statistics or computational thinking. 

Positivism emerged in the late eighteenth and early nineteenth centuries in Western societies, fuelled by a growing optimism on the role and power of the natural sciences -- as witnessed, for instance, by the universal acceptance of Newton's Three Laws, and their explanatory and predictive capabilities backed up by empirical observations. So much so, that it was the predominant paradigm for almost a century and a bulkhead against a growing number of worrying observations, including the movements of the planet Mercury\footnote{See \textcite{enwiki:1193607156}, for instance.}, which didn't reinforce – indeed appear to contradict – Newton's Laws. How could an established truth lead that way? Indeed, Einstein's insight into the intimate connection between space and time inspired a substantial move away from the established Newtonian \enquote{laws} and \enquote{facts}, which were neither any longer \textcite{lakatos2014falsification}\textcite{laka}.

The need to rethink positivist objective truths was something of a crisis in the positivist movement~(see, for instance, \textcite{kuhn2012structure}), leading to post-positivism\footnote{Not the most creative name, you must admit.} which introduced the idea of falsification: any posited theory must make predictions which are testable, the currency of a theory being determined by whether or not it had yet been proven false\footnote{Note that falsifiable theories that have been tested and failed can still be useful, perhaps within a restricted context. For instance, Newton's Laws of motion provide a very good approximation at low energies.}. 

\bigskip

So, both positivism and post-positivism accrete knowledge by formulating generalisations and cause-effect linkages, based on objective, verifiable observations and measurements, and expressed as theories and laws. However, post-positivism acknowledges some of the limitations in such observations and measurement, so that a theory or law will only remain true for as long as it is not falsified by new observations or measurements. There is therefore a shift from certainty (positivism) to probability (post-positivism), with post-positivist researchers encouraged to take multiple measurements and observations, including triangulating their data, to arrive at an objective truth. Thus you might take a post-positivist approach to establishing the linkage between a drug and the alleviation of symptoms: once a generalisation or cause-effect linkage is established, it applies for as long as it remains un-falsified.

Both positivism and post-positivism assume an objective reality and do not admit that the researcher's own mindset and values may influence true knowledge: in being objective and verifiable, different researchers must necessarily arrive at the same truth, as long as the research process is reliable\footnote{We will discuss reliability in the next section}, that is different researchers can follow the same process to arrive at the same conclusions. 

This denial of the researcher's influence on the research is often levelled as a criticism of these paradigms, particularly by social scientists, and has led to new paradigms.

%\todo{See \cite{wikipedia-contributors2023positivism} for a more detail description.}

%\begin{question}[subtitle={Activity: Am I a post-positivist thinker?}]
%***to rethink****On a scale of 0 to 10, with 0 wholly in disagreement and 10 wholly in agreement, to which extent do you agree with the following statement?
%
%\textit{Even though the cat is blue, I can imagine situations in which it could be observed as some other colour.}
%\begin{guidance}
%	If you thought this was easy, you might be a post-positivist thinker. Post-positivist thinkers may tend towards quantitative research.
%\end{guidance}
%\end{question}
%\todo{Clarification of methods should go before activity, to make clear the distinction between posit and post-posit}

\subsection{Anti-positivist (interpretivism)}

The shift from positivism to post-positivism still preserves the absolute objectivity of reality. In contrast, anti-positivism asserts that different people experience and understand reality in different ways: while there may be only \enquote{one} reality, everyone interprets it according to their own views. Simply put, this might mean that generalisations and even cause-effect relationships are subject to individual experience. Think of the way that people interpret the (single) power structure within your organisation: typically, different people will describe it in different ways, as it applies to them.

Explaining the name, anti-positivists believe that all research is influenced and shaped by researchers’ worldviews, leading to differing interpretations of the same reality. Again, think of the questions you might ask of people within an organisation that leads them to describe the power structure. Different questions can lead to different descriptions.

As a result, anti-positivists gravitate towards qualitative research methods and techniques to understand the different perspectives, placed in an explicative context of their own perspective. 

%These may include interviews and focus groups, participant observations, and review of documentation on a phenomenon of interest (e.g., newspaper articles, reports, or information from websites).

In moving away from objective knowledge, however, anti-positivism raises questions of research validity\footnote{We will discuss validity in the next section.}, that is of how trustworthy and generalisable knowledge generated as subjective interpretation might be. 

%See \cite{wikipedia-contributors2023antipositivism} for a more detailed description.

\subsection{Constructivism}

The Constructivist research paradigm goes further and asserts that reality is a construct of our minds and so is absolutely subjective. Constructivists believe that all knowledge comes from our experiences and reflections on those experiences as formed in our mind. A distinction is also made between reality which is individually vs socially constructed, the latter being the result of social interaction within a specific cultural or historical context. 

Due to its focus on experiences and subjectivity, this paradigm is also mostly associated with qualitative research approaches. The researcher focuses on participants’ experiences, including their own, constructing knowledge through understanding, sense making and reconstruction.

%Knowledge accumulates through later research adding informed and sophisticated reconstructions, and vicarious or lived experience\todo{\url{https://image.slidesharecdn.com/lecture21-111207045819-phpapp02/95/research-paradigms-20-728.jpg?cb=1415227903}}.

Establishing research validity is an even more prominent issue with this paradigm.

\subsection{Critical Theory}
The Critical Theory research paradigm originated in the fields of sociology, philosophy and political theory, and asserts that social science can never be 100\% objective or value-free. Therefore, like interpretivism, it assumes multiple interpretations of reality in social contexts. However, it goes a step further by asserting that reality is shaped by those who are powerful, who legitimate particular ways of perceiving the world: `truth' is inherently political, defined by those in charge to the disadvantage of many, and challenged by those who wish to promote equality. As a result, critical researchers seek to challenge the status quo and perceive research as transformative at a social level\footnote{As a result, this paradigm is also called `transformative' in the literature.}, confronting ideology and trying to discover and challenge the mechanisms through which exploitation and disadvantage are perpetuated in society.
 
This paradigm focuses on enacting social change through scientific investigation. Critical theorists question knowledge and procedures, and acknowledge how power is used (or abused) in the phenomena or systems they’re investigating. Researchers using this paradigm offer historically situated insights into society and its power structures as the basis of knowledge, approaching knowledge contribution through inquiries which are both critical and transformative, aimed at emancipation and restitution to address historical injustices. The researcher values are acknowledged and welcomed as a formative influence on the research.

Rather than reliability and validity, the quality of critical theory research is judged in terms of how well it is situated in its historical context, and the extent it acts as a stimulus for transformation, and the diminuition of ignorance and misconceptions

\subsection{Indigenous}
The paradigms just described have attracted criticisms in that they are seen as Western-European centric, imposed on  indigenous cultures as a result of colonialism, hence marginalising indigenous traditions.

 In counterposition, an Indigenous paradigm is emerging with the aim of decolonising research. This paradigm  emphasises the connection between people, their culture, and the spiritual and natural worlds, valuing knowledge which is local to communities, and holistic in connecting all beings with nature and spirituality. 
 
 As a result, indigenous cultural practices and forms of expressions should be reflected in the way the research is conducted, including language, metaphors, oral traditions and indigenous knowledge systems. 

From an ontological perspective, therefore, both physical and spiritual realities and their connection matter, alongside reciprocal relations among all living beings. 

From an epistemological perspective, knowledge is relational, based on the connection between natural and spiritual worlds, and its generation is a fluid process based on oral traditions, such as storytelling, and inward exploration of personal experience in context. The codification of such knowledge is through community praxis, in which the `Elders' are often seen as key actors in the epistemological process.

Finally, indigenous methodology is one that favours the collective involvement of indigenous people in developing, approving and implementing the research, leading to knowledge of practical use.   

It is important to note that although we have tried to characterise this paradigm in relation to ontology, epistemology and methodology, some scholars reject any such classification, regarding this too as a form of colonialism imposed by a Western view of research paradigms. If you are interested in going more deeply into this debate, you could start from \todo{add references: Hart, M. A. (2010). Indigenous Worldviews, Knowledge, and Research: The Development of an Indigenous Research Paradigm. Journal of Indigenous Voices in Social Work, 1(1).}

\section{What's your mindset?}

Table~\ref{tab:researchParadigms} summarises the main paradigms we have discussed based on their ontological, epistemological and methodological standpoints. From a methodological perspective, we have indicated the main tendency of the paradigm, although the quantitative vs qualitative distinction is not as stark in practice, and a mix of methods often applies.

\begin{table}[htbp]
\small
\caption{Summarising research paradigms\label{tab:researchParadigms}}
\begin{tabulary}{\tablewidth}{@{}LLLLLLLL@{}} 
\toprule
&  \textbf{Positivism} & \textbf{Post-positivism} & \textbf{Anti-positivism (Interpretivism)} & \textbf{Constructivism} & \textbf{Critical theory} & \textbf{Indigenous} \\
\midrule
%\textbf{Aim} & to discover general laws and principles and predict behaviour through neutral and objective enquiry & to discover general laws and principles and predict behaviour through neutral and objective enquiry & to understand and predict social systems and their behaviour, acknowledging that the researcher's values and experience affect the enquiry & to understand and predict social systems and their behaviour, acknowledging that the researcher's values and experience affect the enquiry & to change society, challenge norms and emancipate and empower people & ***marginalised and post-colonial communities research ???\\
%\midrule
\textbf{Ontology} & one discoverable external reality & one discoverable external reality that can only be known imperfectly  &  one external reality which is interpreted subjectively & reality as the construct of one's mind & one external reality determined by socio, political and economic power factors &   physical and spiritual realities and their connection; reciprocal relations between all living beings \\
\midrule
\textbf{Epistemology} & objective laws and theories that can be confirmed empirically & objective laws and theories that can be falsified empirically & subjective interpretations & subjective constructions & social and historical constructions, acknowledging issues of power and social injustice & relational knowledge, indigenous knowledge systems based on oral traditions and inward exploration of experience\\ 
\midrule
\textbf{Researcher's role} & objective, neutral & objective, neutral, aware of cognitive limitations & subjective, bringing own values, experience and bias & subjective, bringing own values, experience and bias & subjective, aware of own social position & researcher as indigenous participant in collective research \\
\midrule
%\textbf{Main strategies} & experimental, survey, simulation, mathematical and logical proof & experimental, survey, simulation, mathematical and logical proof & case study, action research, ethnography, phenomenology, grounded theory & case study, action research, ethnography, phenomenology, grounded theory, systematic review & \\
%\midrule
\textbf{Main methods} & quantitative & quantitative, with triangulation & qualitative & qualitative & qualitative & qualitative\\
\bottomrule
\end{tabulary}
\end{table} 

Your own mindset may lead you to gravitate towards one or more of these paradigms, or even somewhere in between. The next activity should help you reflect on this point.   

\begin{question}[subtitle={Activity: What kind of thinker am I?}]
	Consider the following question and describe how you would go about answering it:
	
	``What is the colour of swans?''
	
	Then compare your approach to each of the paradigm. Which one is it closer to and why?
	\begin{guidance}
		If you can think of more than one way to approach the question, then describe and reflect on each of them in relation to the paradigms.
	\end{guidance}
	\begin{solution}
		I can think of a couple of ways I could tackle this question. 
		
		The first would be to start by observing the swans that live on the lake near my home, and record my observations. From that I would put forward an initial hypothesis, say that all `swans are white,' as those are the only ones I can observe locally. I would then look online for images of swans from around the world to see if they match my observations. Having found images of black swans alongside white ones, I would then revise my hypothesis to ``All swans are either white or black.'' This process would continue until I'm satisfied there is no further contradictory evidence I can find, hence conclude that in all probability swans are either white or black. I would have to admit that there may be swans of other colours I've yet to come across, so the statement is open to future challenges. I would also need to be convinced that I'm a neutral observer, able to determine the colour of a swan correctly and reliably. This approach closely aligns with the post-positivist paradigm, specifically: I've made observations, triangulated my direct swan observations with the review of online swan images, and formulated, rejected and then reformulated hypotheses as part of my enquiry process.
		
		My second approach would be to ask other people. For instance, I could set up a crowd-sourcing survey inviting participants to answer the question. By analysing their answers I could then decide if there is enough consensus on the colour of swans: for instance, most participants may have identified swans to be either white or black, although some may have provided more nuanced answers, like yellowish or other. From my analysis I would draw my conclusions which may or may not be the same as in my previous approach. In this case, I would have to worry about who participated in my research. Were there enough participants from around the world to provide sufficient and diverse evidence? To which extent may their colour perception differ? What else could I do to check the validity of the outcome? This approach aligns with the interpretivist paradigm: I have to accept that, like me, each observer in my study will make their own interpretation of what the colour of a swan is, so that I would have to account for this in my conclusions. 
\end{solution}
\end{question}


\section{Research strategies}

Each research discipline and area has its more-or-less well-worn paths to a successful knowledge contribution. In Stage 3, you're now at the point where you'll join researchers in your chosen area on one of those paths: as you get deeper and deeper into your research, the steps you'll take will become more and more specialised. 

To identify and take such steps, you will need to devise a research \textit{strategy}, by which we man a collection of recipes for doing research that will, if followed accurately, lead to a contribution to knowledge \emph{even in the presence of uncertainty}. When devised, a research strategy consists of research tasks that interact in more or less complex ways, but which are sufficiently detailed that the researcher knows what to do next, even if that means making a choice between two or more next steps. 

%With uncertainty a product of many contextual characteristics, including the researcher's view-point, the simpler research strategies stem from contexts with lesser uncertainty. Thus, highly constrained contexts, such as the natural sciences, tend to have simpler strategies as there is less uncertainty to contend with. \todo{to rephrase/expand, but not sure how}


There's good news and bad news in choosing a research strategy:
%
\begin{itemize}
\item The bad news is that there are many possible choices you could make at any point.
\item The good news is that, for your particular area in Masters research, there will likely be only a small subset that you need to know about.
\end{itemize}

To help you in your choice, our approach in this chapter is unapologetically practical. In Section~\ref{sect:standardResearchStrategies}, we will layout the options that you have together with reasons for choosing them and reasons for not choosing them. Each comes with a list of key evaluation questions the answers to which you will be expected to present as part of your dissertation. Amongst other things, the answers you give will justify how and why your work makes a contribution to knowledge. These evaluative questions in turn give you targets to aim for throughout your research, you will need to answer each of them – they will be the driver for your research and your writing up. 

Before we look at research strategy in detail, we are going to consider the importance of such choice in terms your ability to defend your claim that your research has contributed new knowledge.  
%
%For instance, social science research strategy could be as simple as a two group experiment while the strategy\todo{Andrews calls it a research model.} described in \textcite[p.~407]{andrews2005place} for education research has 12 components, arranged in complex interacting feedback loops, with the of including experiments.
%%
%\begin{figure}[h!]
%\centering{
%\begin{tabular}{ccc}
%  \includegraphics[height=8cm]{SimpleExperiment.jpg}&\qquad\qquad\qquad\qquad&
%  	\includegraphics[height=8cm]{andrews2005placefig1.jpg}
%\end{tabular}
%  \caption[Simple and complex research designs]{(left) A simple two-group experimental design~\parencite[, adapted p.~128]{marczyk2005essentials}; (right) A complex research strategy for education research~\parencite[, figure 1]{andrews2005place}
%  \label{fig:simpleandcomplexres}
%  }}
%\end{figure} 
%
%Although each research paradigm is sufficiently distinct as to indicate different strategies, strategies do overlap in their application. Every strategy will, for instance, generate data of some form, whether this is readings in some experimental setting or documentation of the lived experience of a community under focus. Generated research data provide a focus for qualitative and quantitative analysis and, thus, to the synthesis of new knowledge.****


%Recall from Stage 1 that:
%%
%\begin{itemize}
%\item your research aim tells the reader how your research will address the knowledge gap that you have found through your literature review, 
%\item your objectives break it down into 3 to 4 high-level goals you must reach to achieve the aim.
%\end{itemize}
%%
%You're now at the point where you need to think about how you will meet each of your research objectives. Research strategies are the recipes for good research, combining research tasks into meaningful ways of doing research to achieve those objectives.

\section{Developing the research design}\label{sect:stage1ResearchDesign}

Developing a design for your research will help you summarise, explain, and justify how your research is conducted to your examiners and other readers of your dissertation. In addition, it will be a touchstone for you to refer to at times of difficulty and allow you to plot your progress against your objectives. 

The research design will, like most other aspects of your project, evolve: at the start, it will be a collection of your initial ideas and intentions; by the end, it will be a detailed account of what you have actually done. 

Your research design will depend on many factors, including the type of research problem you are trying to address, the intended outcome of your research, the sort of evidence you will need, the resources and expertise you have, accepted research methods applied by other researchers in your field\footnote{And the philosophical beliefs which motivate them}. As you are a key participant in your own research, your personal views and values will also affect the choices you make while developing your research design.

Research design is also a field of study in its own right, one which has grown out of many diverse academic traditions and ways of thinking across academic disciplines and subject areas, and which is still evolving\footnote{In this young research area, there is still a lot of post--rationalisation of a particular course of research as authors looks for generalisable themes.}. As such, it is not an easy topic to digest and is one of the most challenging aspects of doing academic research. It can be puzzling for students embarking on academic research for the first time.

For this reason, in Stage 1, we will not consider research design in detail --- that will happen from Stage 2 instead. However, so that you can start to think about your research design, in this section, we introduce a range of topics which concern research design and should influence your follow-up work.

\subsection{Types of evidence and data}\label{sect:evidenceAndData}
The phenomena upon which your research will be based must be observed and this gives rise to data. Data can be interpreted to give information and evidence.

Thus, most academic research will be based on data and evidence. Data is the raw observations with no interpretation attached --- anything you may collect, observe or gather in your research. Evidence is information interpreted to support (or otherwise!) your academic arguments. Indeed, data forms the basis of evidence, so the two concepts are closely linked and often used interchangeably. This section recalls briefly the main types of data and evidence used in academic research.

\textbf{Quantitative data} are data that can be quantified or measured, and be given numerical values. They include the following types:

\begin{itemize}
\item \textbf{Numerical} data are numbers\footnote{Yes, they are!}, such as the number of students registered on a module or the temperature in the UK in July. Simplifying a little, when numerical data has a whole-number value it is called discrete, otherwise it is continuous\footnote{Given the fundamental nature of energy, and the vagaries of quantum physics, it may be that we're incorrect in stating that real--world temperature is actually a continuous variable. However, even if it isn't, its values lie on a continuous scale.}. In either case, appropriate mathematical and statistical operations can be applied to them, values can be ordered, and the interval between two numbers can be calculated exactly.

\item \textbf{Ordinal} data are non numerical data that can be arranged in an order. An example is the very widely used Likert scale\footnote{Almost certainly, the most recent survey you completed would have used the 5--point Likert scale mentioned here.} often used in questionnaires to elicit opinions. An example of a 5-point Likert scale is that ranging from `Strongly disagree' to `Strongly agree' with `Disagree', `Neither Disagree nor Agree', and `Agree' in the middle. While these values can be arranged in order\footnote{This might be done by giving `Strongly disagree' the numerical value 1, `Disagree' the numerical value 2, and so on.}, mathematical and statistical operations can only be applied in a limited way, for instance, taking the mean (or average) score of responses on a Likert scale.
\footnote{Read more about Likert scales use (and misuse) in \textcite{carifio2007ten}.}

\item \textbf{Interval} data are data which can be arranged on a scale, so that we can calculate the distance between any two data points. All numerical data are also interval data, but interval data may not be numerical. For instance, calendar dates are interval data as we can calculate the time interval between two given dates, e.g., the number of days in between.

\item \textbf{Ratio} data are numerical data with an absolute zero considered as a point of origin, that is no negative values are possible. Examples include a person's high, weight or their wage from employment: none of these can take a meaningful value less than zero.
\end{itemize}

\textbf{Qualitative data, on the other hand,} are descriptive in nature and defy ordering. Sentences, images, sounds, etc., are all examples of qualitative data. An important subclass of qualitative data is \textbf{nominal} data commonly used to denote categories, for instance, Dog, Cat, Alligator, etc. These data cannot be ordered and mathematical operations and functions don't apply to them.

Note that in Statistics ordinal data and nominal data are called \textbf{categorical} data, exactly because they are used to denote categories, which may or may not be ordered. This also means that categorical data span the quantitative/qualitative divide.

Data and evidence are also classed as:

- \textbf{primary,} when newly generated or collected during research; or

- \textbf{secondary}, when already available from previous research, and re-used during new research.

The academic literature that will be at the core of your literature review\footnote{A bit or a hint for the next Activity :)} is secondary evidence, as are all other published academic and non academic documents, e.g., laws, policies and procedures, official reports, etc.

\begin{example}{Curriculum Analytics example, cont'd} 
In our example, to conduct our research we would need to make use of a range of curriculum data which are likely to include both quantitative (e.g., number of students, pass rates, etc.) and qualitative data (e.g., components of a programme of study, learning outcomes, etc.). These will be all secondary data, in the sense that they will be provided by the target university for us to apply a range of modelling techniques.
In terms of outcomes of our research, we are envisaging some form of stakeholders' evaluation of the effectiveness of the techniques, so we are likely to generate, as primary evidence, mainly  qualitative data. 
\end{example}

\begin{question}[subtitle={Activity: Considering types of data and evidence in your project}] List and justify the kinds of data and evidence your research will make use of or generate. 
\begin{guidance}
Consider both existing data and evidence you will need and any new data/evidence your research may produce. Justify your answer in terms of the specific aim and objectives of your work.
\end{guidance}
\end{question}
%%Hack to correct tcbox behaviour
\color{black}

\subsection{Classes of research methods}
\textbf{Research methods} are the means used in research to collect, analyse, synthesise or present data and evidence, and to derive findings from them. Their purpose is to help you conduct your research in a systematic, rigorous, repeatable and reliable fashion.

Research methods can be classes based on their purpose into:

\begin{itemize}
\item \textbf{data} \textbf{collection methods}, used to gather data and evidence

\item \textbf{data analysis methods}, used to analyse data and evidence

\item \textbf{modelling methods}, used to build models of complex real-world situations, where many interrelated phenomena are at play and a holistic understanding is needed.

\end{itemize}

Methods are also classed based on the type of data and evidence they handle into:

\begin{itemize}
\item \textbf{quantitative methods}, which --  unsurprisingly -- are used when dealing with qualitative data;

\item \textbf{qualitative methods}, which -- again unsurprisingly --  are used for qualitative data.

\end{itemize}

Broadly speaking, quantitative methods are widely applied in the natural sciences, with their focus on measurement, natural phenomena and their simpler cause-and-effect relations, while qualitative methods are widely applied in the social sciences, with their focus on understanding human behaviour. In practice, however, this distinction is not as stark and often quantitative and qualitative methods are mixed in research, particularly when research spans several academic disciplines. Instead, modelling methods are often associated with design, computing, engineering and more generally the so-called `sciences of the artificial' (Simon, 1969), which consider technology and its development in its wider social context, focusing on addressing complex, messy socio-technical problems.

Within these broad classes, you will encounter several methods from Stage 2 onwards.

\begin{example}{Curriculum Analytics example, cont'd} 
In our example, we may need both quantitative and quantitative collection methods to extract relevant data sets from the data provided by the target university. In addition, modelling methods may be required in the application of the selected techniques to those data set. Qualitative data analysis methods should apply to stakeholders' evaluation data.
\end{example}

\begin{question}[subtitle={Activity: Considering research methods in your project}] List and justify the kinds of research methods you may apply in your research. 
\begin{guidance}
You should relate your choices to the kind of data and evidence you have identified in the previous activity.\end{guidance}
\end{question}
%%Hack to correct tcbox behaviour
\color{black}




\section{Developing your understanding of research design}\label{sect:stage2researchDesign}

Your literature review has set you on the right course to begin your research – you now know that there's a hole in knowledge into which a contribution can be made. 

While this tells you \textit{what} you are going to research, your research design should tell  \textit{how} you are going to do it. As you learnt in Stage 1, your research design will summarise, explain and justify how your research is conducted, developing into a detailed account of what you have done by the end of your project. An appropriate research design is one which allows you to make your contribution to knowledge in a way which meets the expectations of researchers in your field of study.

%Of course, the notion of correctness here is a loose one: innovation in research design in a particular domains happens more or less often. Indeed, it is often an innovation in the design of research by which new knowledge is contributed. And it's not unknown for a researcher new to the domain to bring fresh thinking – it may even be you. There are risks of innovating, however, including finding it difficult to publish – not a concern for you – having difficulty convincing a reader that's not ready to accept innovation – essentially, being accepted. Risk management is often a Good Thing.

In this section, we consider fundamental concepts in research design to help you develop your understanding and inform your project choices, and to prepare you for the detailed work in Stage 3 where you will apply your learning to your own project.

The basic building blocks of research design are research methods and research strategies. \textit{Research methods} are standard ways to collect, analyse, synthesise and present data and evidence, and to derive findings. You can think of them as basic techniques and procedures that researchers have come up with over time to deal with different kinds of data and evidence. Instead, \textit{research strategies} are ways to systematise how research methods can be used together. You can use them to guide you in selecting and combining research methods to address specific kinds of research problem. Both methods and strategies are therefore practical tools to do research. Those you will encounter in this section are summarised in Figure~\ref{fig:strategiesAndMethods}.

\begin{figure}[htbp]
\centering
\includegraphics[width=0.9\textwidth]{strategiesAndMethods.pdf}
\caption{Research strategies and methods}
\label{fig:strategiesAndMethods}
\end{figure}

At a theoretical level, strategies and methods are motivated and justified by different sets of beliefs on the nature of what we can study, how knowledge can be generated, and what is of value in research, called \textit{philosophical traditions}: although practically you could apply strategies and methods without referencing them, it is important for you as a researcher to be aware of the beliefs and values they embody, and the extent they align to your own -- something which may influence your own research design choices. In this section you will also read about the more prominent philosophical traditions.

\subsection{Research methods}
{Research methods} are the means used in research to collect, analyse, synthesise or present data and evidence, and to derive findings from them. %It is worth noting that there is a subtle distinction between data and evidence. Data is raw information with no interpretation attached --- anything you may collect, observe or gather in your research. Evidence is information interpreted to support your academic arguments. Indeed, data form the basis of evidence, so the two concepts are closely linked and often used interchangeably.
They help you conduct your research in a systematic, rigorous, repeatable and reliable fashion. So, research methods are important because they underpin how sound\footnote{This is the topic of research \textit{validity}, which will be covered in Stage 3.} your research is.

Research methods are many and vary greatly, which can be confusing to the novice researcher --- and even the seasoned one at times! One source of confusion is that the same term is often used to indicate both specific techniques and procedures, and broad {research strategies} combining many.

In this handbook we will use the term research method as a synonym for research technique\slash procedure, but be aware that you may encounter other meanings while reading the academic literature.

In this section, we recall a wide range of research methods commonly applied in research, particularly at Masters level. They are categorised as collection, analysis and modelling methods.

%%%JGH 2023-10-13: Commands to keep Activity text consistent across all research methods.
%\newcommand{\ActivityRMUse}[2][]{
%%%optional argument is addition information, should include initial ", " so fit in the sentence 
%%%mandatory argument is research method
%\begin{question}[subtitle={Activity: Considering #2\todo[inline]{templated: check for sense}}]%
%Consider the extent to which #2{} could be used in your research project, and in which form#1. Write down your answer.
%%
%\begin{guidance}
%If you think #2 will be helpful, you might like to look at the resources at the end of this chapter. If you don't think it will be useful for your project, write down the reason – it might come in handy later.
%\end{guidance}\end{question}
%%%Hack to correct tcbox behaviour
%\color{black}
%}

%%LR - Nov 2023, replaced individual questions in method section and replaced with activity in summary section
%\newcommand{\ActivityRMUse}[2][]{}
%
%\subsubsection{Data collection methods}
%These are methods that can be used to collect data and evidence, which might be qualitative, quantitative or both.
%
%\paragraph{Questionnaires}
%A {questionnaire} is a fixed set of questions organised in a particular order used to gather answers. It can be delivered face-to-face or distributed to respondents to gather their answers. The respondents' answers constitute the generated data that is subsequently analysed by the researcher.
%
%Questionnaires are a data collection technique applicable when:
%
%- you wish to obtain standardised data from many people
%
%- you seek relatively brief information from your respondents
%
%- you expect your respondents to be able to understand and interpret the questions in a straightforward manner.
%
%Questionnaires can help collect both quantitative and qualitative data, depending on their questions.
%They can be \textit{self-administered}, in the sense that the respondents complete the questionnaire without the researcher being present, or \textit{researcher-administered}, in which case the researcher asks the questions and writes down the responses.
%
%\ActivityRMUse{questionnaires}
%
%\paragraph{Interviews and focus groups}
%An {interview} is a form of conversation between the researcher and one or more interviewees, designed by the researcher to gain insights and opinions on a specific topic. The researcher guides and controls the conversation and asks the questions. The interviewees' answers constitute the generated data that is subsequently analysed by the researcher.
%
%An interview is a technique for data generation applicable when you wish to:
%
%- obtain detailed information on a specific issue or topic
%
%- ask open-ended, complex questions, which may be tackled or interpreted differently by different interviewees
%
%- investigate sensitive issues or privileged information that interviewees may not be willing to commit to writing.
%
%Interviews are primarily used to collect qualitative data. They can be one-on-one, between the researcher and one interviewee at a time, or can happen in a group, with several interviewees being interviewed together by the researcher. The latter is referred to as a {focus group}.
%
%Interviews can be fully planned or quite open-ended. The former are termed \textit{structured} and use pre-determined, identical questions with all the interviewees, while the latter are termed \textit{unstructured} and typically start by introducing a topic but then let the interviewee talk freely around their ideas, experience and beliefs. Somewhere in between are \textit{semi-structured} interviews, where the researcher selects some themes and related questions upfront, but then adapt them depending on how the conversation with the interviewee develops.
%
%\ActivityRMUse{interviews}
%
%\paragraph{Delphi technique}
%With the {Delphi} technique, a group of experts are consulted with a view of obtaining a consensus on a particular issue or topic. It involves an iterative process of collecting, synthesising and circulating anonymous judgements from those experts to eventually arrive at a consensual view. More precisely, each subject expert is initially consulted separately by the researcher, who then anonymises and collates the group responses and circulate them to the same group of experts. The process is repeated until a consensus is reached.
%
%This technique is based on the idea that a group of people are more likely to arrive at an informed and valid position than an individual, with anonymity preventing interpersonal relationships from influencing the outcome. The judgements and consensus gathered constitute the generated data that is subsequently analysed by the researcher.
%
%The Delphi technique is particularly suited to situations in which the researcher wishes to improve their understanding of an under-explored problem or issue in order to inform decision-making.
%
%\ActivityRMUse{the Delphi technique}
%
%\paragraph{Observations and measurements}
%{Observations} are used in research to find out what people actually do or what actually happens in a particular context, rather than what has been reported about it. Observations can be of people's behaviour or interactions, e.g., observing a formal meeting in an organisation, or of events and processes, for instance observing a queue at the post-office or a computer-controlled production plant. As such, observations can generate all kinds of data, qualitative and quantitative. Quantitative observations are often referred to as {measurements}, e.g., the length of time a particular customer has spent waiting in the queue at the post office.
%
%There are two main types of research observation, \textit{systematic} vs. \textit{participant}. The former is when the researcher decides in advance what to observe, the schedule of observations and what to record. For example, the observation of a queue at the post office could be planned to take place over a certain week or month, and recordings may include time of arrival and departure of each customer, average and maximum length of the queue, average service time, etc.
%
%In participant observations, the researcher participates directly in the situation under study and produces a rich description of what happens based on what they experience. For instance, in relation to the previous example, the researcher might join the queue at the post office and record their experience in great detail, or even join the staff in the post office to understand why queues are longer or shorter for certain tasks.
%
%\ActivityRMUse{observations and measurements}
%
%\paragraph{Use of secondary evidence}
%The previous techniques can be used to generate primary evidence\footnote{As defined in Section~\ref{sect:evidenceAndData}, primary evidences is newly generated during research, while secondary evidence is already available from previous research.}.
%
%Academic research, however, can also use secondary evidence as its starting point. This can be represented by existing documents of many forms, from academic articles to documents found in organisations, e.g. laws, policies and procedures, reports, formal minutes of meetings, informal communications, etc. Similarly, there are plenty of publicly available data sets that can be used, created by academic communities or private and public organisations, such as business financial data or statistical data from the UK Office for National Statistics or data from social network platforms. As already noted, the academic literature at the core of your literature review is a form of secondary evidence.
%
%\ActivityRMUse{reusing existing evidence}
%
%\subsubsection{Data analysis methods}\label{sect:dataAnalysisMethods}
%These are methods that can be used to analyse data and evidence after collection.
%
%\paragraph{Spreadsheets, tables, charts and graphs}
%They are the bread and butter of data analysis, are applicable to all kinds of data, and can be used to summarise and visualise data, and identify interesting patterns. You should be already familiar with this topic from your previous studies, so that this is only a brief overview to refresh your knowledge.
%
%A {spreadsheet} is a digital tool you can use to capture, display and manipulate data arranged in tables, that is arranged in rows and columns. Common spreadsheets include Microsoft Excel, Apple Numbers and Google Sheets. Spreadsheets are among the most used digital tools, so it is likely you are already familiar with at least their basic functionalities. Spreadsheets have become quite sophisticated tools, including all sort of charts and graphs, as well as programmatic capabilities which allow you to code quite complex data manipulation functions. Some of those advanced functionalities could be advantageous to your research, so it is worth spending some time considering what they can offer to your project. There are plenty of tutorials and other documentation online you can use to learn more.
%
%\ActivityRMUse{spreadsheets}
%
%
%Alongside spreadsheets a growing number of {data analytics} tools are also available: these are sophisticated digital tools which extend spreadsheet capabilities for collating and visualising data to include some degree of automated analysis, both statistical and based on Machine Learning algorithms. Tools like Tableau and Power BI are notable examples: both are available in free versions for community use and for study.
%
%\ActivityRMUse{data analytics tools}
%
%\paragraph{Statistical analysis} 
%This refers to a broad collection of techniques used to investigate trends, patterns, and relationships in quantitative data. It is a well established field of study with wide application across all kinds of research, and well developed tool support. In particular, both spreadsheets and data analytics tools include functionalities which allow you to calculate statistical measures on data, check statistical relationships between variables and data sets, and generate basic statistical models. Bespoke statistical tools also exist for more advanced statistical analysis and modelling, like IBM SSPS or Minitab, which are also available in free versions for students.
%
%Should your project require advanced statistical analysis, then you will need to become proficient in good time to carry out your analysis and interpretation of findings in Stages 4 and 5 or your project.
%
%\ActivityRMUse{statistical analysis}
%
%\paragraph{Thematic analysis}
%This is a way of analysing qualitative data, particularly texts, e.g., transcriptions of interviews or answers to questionnaires or existing text documents, in order to find out something about people's views, opinions, knowledge, etc.
%
%At its core is the identification by the researcher of recurring themes, their definition and relationships: this relies on the researcher's judgement and it is quite subjective.
%
%\ActivityRMUse{thematic analysis}
%
%\paragraph{Content analysis}
%This is used to identify patterns in non-numerical content used for communication, whether text, speech, images, videos, or other. For instance, in can be used to investigate certain words, themes, or concepts within that content.
%
%It can be either quantitative, where the focus is, for instance, on counting occurrences, or qualitative, where the focus is on interpreting and understanding meaning and relationships. As such it can be used for many purposes, from discovering and understanding patterns, to looking at intentions behind what is expressed, or to highlighting differences of use in different contexts. 
%
%\ActivityRMUse{content analysis}
%
%\paragraph{Discourse analysis}
%It focuses on the use of language in conversations within their real-world context. In the analysis, it is essential to consider the influence of history, culture and power dynamics within that context.
%
%\ActivityRMUse{discourse analysis}
%
%\paragraph{Narrative analysis}
%It focuses on stories which are told by people. The focus is on listening to such stories and how they are told to investigate their meaning, particularly how people make sense of reality.
%
%\ActivityRMUse{narrative analysis}
%
%\subsubsection{Modelling methods}
%At its essence, a \textit{model} is an abstraction or representation of something, be that a system, a structure or a behaviour. Modelling is used across many disciplines, so a vast repertoire of modelling techniques exist.
%
%Possibly the most important thing you must remember about modelling is expressed by the following oft-cited aphorism\footnote{Box, George E. P. (1976), ``Science and statistics'' (PDF), Journal of the American Statistical Association, 71 (356): 791--799.}:
%
%\begin{quotation}
%	\textit{All models are wrong, some are useful} (Box, 1976)
%\end{quotation}
%
%which makes clear that a model should not be regarded as a faithful replication of some reality, but as a tool to investigate some aspects of that reality.
%
%In this section, you will read about a small set of modelling techniques, which are particularly relevant in Master projects. A lot more can be found in the academic literature and beyond.
%
%
%\paragraph{Systems diagrams}
%You can use {systems diagrams} to help you understand the structure of a situation of interest that can be rendered as a system. The term `system' is meant in its widest possible meaning of a set of components interconnected for a purpose. This is a very general and versatile technique that you can apply to all sorts of real-world situations. If you have studied a systems thinking and practice module for your qualification, you will be already familiar with this technique.
%
%There are many different kinds of systems diagrams\footnote{You can find links to useful tutorials in Section~\ref{sect:readingList}.}. For examples, \textit{systems maps} allow you to sketch the structure of a system by identifying key components and sub-systems. They can be extended to show how those elements influence each other, in which case they are called \textit{influence diagrams}. On the other hand, \textit{causal loop diagrams} are used to capture cause-and-effect relations in a system, hence model certain dynamics of that system, particularly underlying feedback structures. They can be turned into \textit{stock and flow diagrams} by adding quantitative information, so that this type of diagram is useful both for analysis and simulation of systems behaviour.
%
%
%System diagrams have an accepted structure, format and notation but what you choose to describe and include within a system and its components will depend on your own viewpoint. Systems diagrams can be shared with others as learning devices to promote more understanding of a situation.
%
%\ActivityRMUse{systems diagrams}
%
%\paragraph{UML modelling}
%UML ({Unified Modeling Language}) is a graphical language for visualising, specifying or documenting various artefacts in the process of developing software systems. If you have studied a software engineering module for your qualification, you may be already familiar with UML.
%
%UML can be used as a `sketching' language, to capture elements of systems informally, like you can do with systems diagrams, or as a `blueprinting' language to specify precisely how elements of a software system will be developed. Different kinds of UML diagrams exist, so that you can model elements of software systems both in terms of their structures and behaviours, and their interactions with end-users.
%
%In a Masters project, UML can be used to help you understand existing systems in context, or plan the development of new innovative artefacts.
%
%\ActivityRMUse{UML modelling}
%
%\paragraph{Problem diagrams}
%These have their roots in software requirements engineering as a diagrammatic technique to capture requirements in a real-world context to inform the specification of a new software system to satisfy them. They have been subsequently generalised for application to general engineering problems for which some novel solution artefact is to be developed in a real-world context, to guide design and ensure fitness-for-purpose. Problem diagrams span the problem-solution space divide by focusing on those phenomena that characterise a problem and constrain its solution.
%
%In the context of Masters projects, problem diagrams can help you develop a good understanding of real-world requirements in context and explore both constraints and effects of designing new systems for that context.
%
%\ActivityRMUse{problem diagrams}
%
%\paragraph{Statistical modelling}
%Many {statistical modelling techniques} exist. The most commonly applied include those used to model relations between variables, e.g., how crop yields relate to environmental factors, such as soil quality or meteorological conditions, or to model real-world processes, e.g., the spreading of a disease in a population. Once a statistical model is defined, it can then be used to make predictions of what might happen in the real world.
%
%As per advanced statistical analysis, statistical modelling requires well developed statistical knowledge and skills, which you should already possess, or have the time to develop, if you are considering their use in your Masters project.
%
%\ActivityRMUse{statistical modelling}
%
%\paragraph{ML modelling}
%Machine Learning (ML) models are computational programs which can identify patterns in large data sets and use them to perform a great variety of tasks usually associated with human cognition, like recognising images or language, classifying objects, or generating speech or texts, to name just a few current applications. 
%
%They are increasingly applied to all sort of data-rich problems, so that several ML models are readily available for use, embedded in computational library\footnote{For instance, the \textit{Scikit-Learn} library for the Python programming language.}. 
%
%ML modelling tools require highly developed technical knowledge and skills, which you should already possess, or have the time to develop, if you are considering their use in your Masters project.
%
%\ActivityRMUse{ML modelling}
%
%\paragraph{Prototyping}
%
%A prototype is an early version of an artefact, which can be used to test early design ideas or properties before implementation. As such they can present various degrees of fidelity in relation to the end artefact.
%
%Prototypes are used in many contexts, particularly architecture, design, engineering or computing.  In academic research, they can be useful to explore both problems and their intended solutions. 
%
%Data generated from prototyping can be either qualitative or quantitative, depending on the nature of the prototype and its intended use. For instance, the prototype of a digital app for a smart phone may be used to evaluate some usability properties with end users, which may generate qualitative data, or to measure computational speed and efficiency, which would generate quantitative data. 
%
%\ActivityRMUse{prototyping}
%
%\paragraph{Simulations}
%
%A simulation is something that mimics a situation, a process or the behaviour of system. They may be used to make predictions, for education and learning as well as for exploration and discovery. For instance, weather simulations are used to inform forecasts, while simulations of crisis scenarios, say a fire or an earthquake, can be used to train people. 
%
%Computer simulations are widely used in research, and many types exist. Among the most common are: 
%\begin{itemize}
%	\item \textit{system dynamics}, which use continuous mathematical models to capture the dynamic behaviour of complex systems. Financial market simulations are often based on system dynamics.
%	\item \textit{agent-based}, which use collectives of interactive agents, whose behaviour can be programmed, to explore emergent properties of a system. Simulations of natural ecosystems are often based on multi agents, e.g., to study the balance between preys and predators over time, in which agents represent each the behaviour of ether a prey or a predator. 
%	\item \textit{statistical simulations}, which apply statistical models to mimic a process or a system. The spread of diseases in a population, for instance, is often simulated statistically.
%\end{itemize}
%
%\ActivityRMUse{simulations}
%
%
%\subsubsection{Summary of methods}
%
%We have covered quite a few methods in this section! Table~\ref{tab:methodsSummary} gives you a concise summary, including a brief description of each. 
%
%Before moving on, you should about which of these methods may find some application in your project.
%
%\begin{question}[subtitle={ACTIVITY: Identify candidate research methods}] Consider the methods in the table and which you may be able to use in your project. For each, write down what you would use them for and why.
%
%\begin{guidance}
%In considering candidate methods you should think of the sort of data and evidence your project may need, as well as the demands of each method in terms of access to people and other resources, and the level of expertise required to apply them. 
%Their descriptions in the previous sections, which you may like to revisit, should be sufficient for you to carry on this activity. In Stage 3, you will return to some of those methods, to investigate how to apply them in detail within your project.
%
%\end{guidance}\end{question}
%%%Hack to correct tcbox behaviour
%\color{black}
%
%
%\begin{table}[htbp]
%\small
%\caption{Research methods introduced in this section\label{tab:methodsSummary}}
%\begin{tabulary}{\textwidth}{@{}LLL@{}} 
%\toprule
% \textbf{Type} & \textbf{Name} & \textbf{Description}\\
%\midrule
%
% \textbf{Data collection} & \textbf{Questionnaires} & pre-defined set of questions used to gather answers from respondents \\
% & \textbf{Interviews and focus groups} & form of conversation between researcher interviewee(s) to gain insights and opinions around a specific topic \\
% & \textbf{Delphi} & iterative process of collecting and synthesising anonymous judgements from experts to arrive at a consensual view \\
% & \textbf{Observations\slash measurements} & direct observation\slash measurement of phenomena of interest \\
% & \textbf{Use of secondary evidence} & reusing data\slash evidence from previous research\\
%\\
% \textbf{Data analysis} & \textbf{Spreadsheets, tables, charts and graphs} & tools to summarise and visualise data in order to identify interesting patterns\\
% & \textbf{Statistical analysis} & set of techniques to investigate trends, patterns, and relationships in quantitative data \\
% & \textbf{Thematic analysis} & technique to identify recurring themes, their definition and relationships in qualitative data\\
% & \textbf{Content analysis} & technique to investigate certain words, themes, or concepts in qualitative data\\
% & \textbf{Discourse analysis} & technique to investigate how language is used in conversations\\
%& \textbf{Narrative analysis} & technique to investigate meaning behind \color{red}{people's storytelling?}\\
%\\
% \textbf{Modelling} & \textbf{Systems diagrams} & representation of the structure of a situation of interest, rendered as a system \\
% & \textbf{UML modelling} & graphical notation to visualise, specify or document software system-related artefacts \\
% & \textbf{Problem diagrams} & graphical notation to capture requirements for to-be-designed artefacts, including software \\
% & \textbf{Statistical modelling} & set of techniques to establish statistical relations between variables of interest \\
% & \textbf{ML modelling} & computational programs able to identify patterns in large data sets and use them to perform cognitive tasks \\
% & \textbf{Prototyping} & early version of an artefact used to test or investigate properties or early design ideas \\
% & \textbf{Simulations} & something that mimics a situation, a process or the behaviour of system \\
%\bottomrule
%\end{tabulary}
%\end{table} 

\section{Research strategies}
A {research strategy}\footnote{Our favourite take on \textit{strategy} is: \enquote{The essence of strategy is choosing what not to do} from the economist \textit{Michael Porter}. Your research strategy should do this – by giving you a focus – so choose carefully, and use its guidance wisely.} is a systematisation of a set of research methods, which can be applied together in order to address research problems of a particular kind. As such it can help you select and apply an appropriate mix of research methods. The term \emph{methodology} is also sometimes used in the literature with a similar meaning\footnote{And, confusingly, also as a synonym of 'method'...}, although methodology also means the study of research methods, so it is an overloaded term this handbook will avoid.

This section provides an overview of some of the best known and most commonly applied research strategies.

%\paragraph{Survey research}
%This aims to gain insights which are valid across a target population, by collecting data from a predefined sample in a standardised and systematic way.
%
%A typical application of survey research is to predict the outcome of an upcoming general election by polling data from a representative sample of voters.
%
%For your data collection, you need to identify upfront which data you will collect in a standardised matter, your target population and sample. So questionnaires or structured interviews are usually used for data collection.
%
%In your data analysis, you seek patterns in the sample data collected to arrive at generalisations to the wider population. Statistical analysis is usually applied, possibly complemented by some thematic analysis, if open-ended questions are also included.
%
%For survey research to be successful, you must be able to access an appropriate sample and generate a sufficient volume of data.
%
%The advantages of this strategy are that it can produce a lot of data in a relatively short time, and you can replicate your data collection process on different samples or on the same sample at a later time. However, among its disadvantages are the depth in the data that can sometimes be lacking, its focus on what can be measured, the fact that it cannot reveal cause-and-effect relationships, and can only provide a snapshot at a particular time.
%
%\ActivityRMUse{survey research}
%
%\paragraph{Experimental research}
%This is used to investigate cause and effect relationships between factors by testing hypothesis or proving\slash disproving causal links.
%
%For instance, you may run an experiment to test ways in which the use of mobile phones just before going to sleep affect people's sleeping patterns.
%
%There are two main kinds of experiments: \textit{laboratory experiments}, which are carried out in closed environments, such as a laboratory; and \textit{field experiments}, which are conducted in the `real world'. Laboratory experiments are often applied in engineering and computer science research, while {field experiments} are usually applied when people are involved.
%
%Possibly the best known kind of field experiment are clinical trials, widely applied in medicine. However, field experiments are also very popular in research which investigates technology in its social context or application of use.
%
%In experiments, first you would need to state the \textit{hypothesis} to be tested: this is a tentative statement about the relationship between phenomena to be tested in the experiment. In the example above, a hypothesis to be tested might be that ``the blue light emitted by a mobile phone reduces the production of melatonin.'' As melatonin is the hormone which controls a person's sleep-wake cycle, its reduction is likely to disrupt a person's sleeping pattern. After formulating the hypothesis, you would then make detailed observations and measurements of outcomes, e.g., the amount of melatonin released by the body, and any changes that take place when particular factors are introduced or removed, e.g., the length of exposure to the blue light.
%
%In analysing your experimental data you seek to explain causal links between factors under study, looking at your observations and measurements under different experimental conditions. Statistical analysis is widely used for data analysis.
%
%For experiments to be successful you must be able to control factors which can affect the outcome. This is possible in laboratory experiments, while the level of control in field experiments is diminished.
%
%Experimental research has well established processes and protocols and is particularly well suited to the consideration of cause-and-effect relations. However, it has its pros and cons. Laboratory experiments are very reliable due to the high level of control, but can be very artificial, with little or no relation to a real-world context. The opposite is true for field experiments.
%
%\ActivityRMUse{experimental research}
%
%\paragraph{Design science research} This seeks to generate new knowledge about a significant problem or its solution via the design of an artefact. It simultaneously generates knowledge about the problem, the artefact and the method used to design it. Artefact indicates anything made by humans, so this is a very broad definition, encompassing all that does not exist in nature.
%
%Lots of research in Computing is an expression of design science, for instance designing new algorithms able to emulate human cognition.
%
%More than data collection and analysis, in design science you need to follow a process of articulating the problem, and designing, constructing and evaluating a solution artefact. In doing so, you shed new insights on the problem, and argue how the solution and solution process contribute new knowledge. As a result, modelling techniques are widely applied, possibly informed by data collection techniques, like reviewing existing documents or interviews with stakeholders and experts. Prototyping is often used to produce proof-of-concept artefacts to test or demonstrate the design.
%
%For design science research to be successful you must be able to argue that it is not `normal' design, that is you are not simply re-implementing a solution to a well-known problem through a well-known development process.
%
%An advantage of design science research is that it leads to tangible artefacts which fit real-world contexts, and it is particularly suited to emerging and rapidly changing technology-related fields of study, where new problems emerge all the time and known solutions are sparse or become rapidly obsolete. The latter is also a disadvantage, of course, as new solutions may be short-lived. Also, it may be difficult to generalise outcomes to different real-world settings. Depending on the nature of the artefact being designed, advanced technical skills may be required.
%
%\ActivityRMUse{design science research}
%
%\paragraph{Case study research}
%A {case study} can be used to investigate in great depth a notable instance of what is under study, in its real-world context. Case studies focus on the `how?' and `why?', and what you seek can span from exploring possible questions or hypotheses for follow-up research, to providing a detailed account of a phenomenon in its natural context, to explaining why certain outcomes or phenomena have occurred.
%
%For instance, an example of case study could be a detailed investigation of the US Equifax social security breach of 2017, in which 143 million of their consumer records were stolen by hackers. This may be descriptive of the chain of events that took place or explicative of why things happened the way they did.
%
%Case studies require you to collect data from a great variety of sources, and to focus on depth rather than breadth. Therefore, all data collection techniques which allow you to do so may be used, from interviews to observations to studying existing documents forensically. This will lead to much qualitative data, so that qualitative methods are often needed for the analysis of the evidence.
%
%For a case study to be conducted successfully you must be able to analyse the chosen instance holistically and in its real-world context.
%
%Case studies allow you to study a complex situation where several factors are at play, and to explore alternative meanings and explanations. However, case studies are time-consuming, difficult to perform rigorously and with limited generalisation beyond the particular instance under study.
%
%\ActivityRMUse{case study research}
%
%\paragraph{Systemic inquiry}
%This is used to explore complex, messy problematic situations involving multiple and often contrasting perspectives, with the aim of transforming the situation for social improvement. Systemic inquiry is based on concepts and principles of systems thinking and systems practice.
%
%Situations for systemic inquiry can range from local to global. So, it may equally apply to exploring changes in practice within a local organisation, and to international responses to disruptive events such as climate change. Of course, it is highly unlikely that your Masters project will tackle a situation at a global scale!
%
%In systemic inquiry, you must be able to articulate your personal stake in the situation, for example, a deeply felt interest or active involvement, rather than assuming and claiming unbiased passive `neutral' observation. You must also keep your own journal during the course of your research inquiry, tracking changes in your own viewpoint and how you adapted your research as a result. In some sense, a systemic inquiry is a conceptualisation of your own learning system and how it adapts to change during the research. Therefore, a successful systemic inquiry should demonstrate \textbf{reflexivity} -- reflecting on your own changing viewpoint and impact on the wider research situation. In fact, systemic inquiry emphasises reflexivity and building trust relationships with stakeholders, in order to make sense of complex situations of change and uncertainty.
%
%To conduct your systemic inquiry you must articulate your research problem with reference to one or more systems which concern the problematic situation under study, and frame your research in terms of possible systems change. You must also have access to sources of different perspectives on the situation under study in order to generate your evidence. This may include both primary evidence from people involved or affected by the situation, and secondary evidence from official and grey literature\footnote{As explained in from Section~\ref{sect:howToAccessTheLiterature}, grey literature refers to information produced by organisations other than commercial publishers, such as academia, government bodies, or non-publishing businesses and industries, and can include pre-publication and non-peer-reviewed articles, theses and dissertations, research and committee reports, government reports, conference papers, accounts of ongoing research, etc.} associated with the situation; your own research journal will also be a source of evidence. In terms of methods, a systemic inquiry is primarily a qualitative endeavour, so you can apply any methods that deal with qualitative data. Distinctively, you can complement them with other tools and techniques which you may have developed through your own experience and professional practice: this is known as \textit{bricolage} research\footnote{Kincheloe, J. L. (2011). Describing the bricolage: Conceptualizing a new rigor in qualitative research. In Key works in critical pedagogy (pp. 177--189). Brill.}.
%
%\ActivityRMUse{systemic inquiry}
%
%\paragraph{Systematic review research}
%This is used to generate new insights from published work. A {systematic review} is a literature review linked to a clearly defined research problem or question. It uses a rigorous set of criteria to identify, select, and critically appraise relevant research from previously published studies in order to generate a scholarly synthesis of the evidence in relation to that problem or question. Such a synthesis is meant to advance a field of study.
%
%For example, a systematic review of randomised controlled trials on the effectiveness of a specific medical treatment could be used to advance evidence-based medicine.
%
%In a systematic review you only use secondary evidence from published studies. You must decide upfront your research problem\slash question and the set of criteria you will use to select, summarise and evaluate those studies. The type of analysis you will conduct will depend on the nature of the evidence you are considering and combining. In \textit{narrative reviews}, a narrative synthesis is produced, while in \textit{meta-analysis}, statistical techniques are used to analyse and combine results.
%
%To be successful, a systematic review has to be both systematic and extensive, which requires the researcher to have a very good grasp of the subject area in order to establish appropriate criteria and make a novel contribution to knowledge.
%
%Because of their explicit set of criteria, systematic reviews are considered transparent, reliable, and easy to replicate. However, they can be very time-consuming due to the large body of work to review. Also, in striving to piece together evidence from potentially very different studies, they may obscure important differences. Narrative reviews may also be subject to bias.
%
%\ActivityRMUse{systematic review research}
%
%
%\paragraph{Ethnography}
%This is used to study the culture of a group of people in their natural setting, and was originally developed within the discipline of anthropology. 
%
%It requires the researcher to characterise the culture being study by making detailed observations, gathering and recording detailed data, reflecting on what they have learnt, linking it to the existing literature. 
%
%The researcher is required to join the group in order to gain an insider's perspective by sharing what the group members' experience: the resulting cultural characterisation should therefore be one that the group members recognise and find familiar. This characterisation should be inclusive of various cultural facets, social and economical, rather than focusing on one specific aspect.
%
%Ethnography can lead to rich descriptions of complex social settings. However, it is very time consuming and demanding in terms of quantity of evidence to produce. The dual observer-participant role of the researcher can also make it hard to maintain an unbiased stance. Also, while the characterisation produced may be very deep in representing a particular group culture, it may be difficult to generalise to other social groups or settings. Because of these characteristics, ethnography is seldom used at Masters research.
%
%\ActivityRMUse{ethnography}
%
%\paragraph{Action research}
%This is used primarily to improve the researcher's own professional practice, focussing on practice change, and continuous learning and improvement via iterative `plan-act-reflect' cycle. As such, it is often applied with the education discipline. 
%
%The researcher is an active participant in the research, rather than solely an observer, alongside other collaborating practitioners: in fact, reflection and collaboration are two key elements of this strategy.  
%
%The research outcomes should make both a contribution to knowledge \textit{and} to practice. For instance, it is possible for new theories or methods to be outcomes of action research, which could be more generally applicable, alongside their direct implementation to improve practice within a specific professional setting. 
%
%Action research can bring immediate professional benefits, hence may have direct impact on practice. However, its application is constrained by the need to involve practitioners as collaborators in the research. Also, the researcher's own professional stake and involvement may increase the risk of personal bias distorting the research and its outcomes.
%
%
%\ActivityRMUse{action research}
%
%
%\paragraph{Mixed-methods research}
%This combines quantitative and qualitative research to gain different perspectives on phenomena of interest, by exploring connections and contradictions between quantitative and qualitative data\footnote{Mixed-method research should not be confused with \textit{multi-method research}, which simply indicates the use of many methods, possibly all qualitative or quantitative.}.
%
%For instance, in looking at acceptance of a new technology, mixed-methods research could consider both levels of adoption and demographics, and the reasons behind adoption or otherwise, possibly to inform further development of the technology.
%
%Data collection and analysis will depend on the particular combination of methods selected. An important aspect is the consideration of how connections between findings are established, through comparing and contrasting data from the different methods applied. This is also referred to as \textit{triangulation}.
%
%The main advantage of mixed-methods research is that it can provide a more holistic understanding of the phenomena under study, and facilitate different avenues for exploration. It is particularly suited to situations in which neither quantitative nor qualitative methods alone can provide sufficient insights. However, mixed-methods make research design more complex and demanding in terms of execution time, skills required and data variety to handle and analyse.
%
%\ActivityRMUse[they should be combined]{mixed methods research}
%
\subsection{Summary of research strategies}
Table~\ref{tab:resstrat} provides a summary of research strategies introduced in this section. As you did for methods, in the next activity you should reflect on those which are most relevant to your project.
 
\begin{question}[subtitle={ACTIVITY: Identify candidate research strategies }] Consider the strategies in the table and which you may be able to use in your project. For each, write down why you think they are applicable.
\begin{guidance}
In considering candidate strategies you should think about your research problem, aim and objectives. Their descriptions in the previous sections, which you may like to revisit, should be sufficient for you to carry out this activity. In Stage 3, you will return to some of those strategies, investigating them in greater depth and for their application to your project.
\end{guidance}\end{question}
%%Hack to correct tcbox behaviour
\color{black}


\begin{table}[htbp]
\caption{Research strategies introduced in this section\label{tab:resstrat}}
\centering
\small
\begin{ltabulary}{\textwidth}{@{}LL@{}}
\toprule
 \textbf{Name} & \textbf{Description} \\
\midrule
 \textbf{Survey research} & to gain insights which are valid across a target population, by collecting data from a predefined sample in a standardised and systematic way\\
 \textbf{Experimental research} & to investigate cause and effect relationships between factors by testing hypothesis or proving\slash disproving causal links\\
 \textbf{Design science research} & to generate new knowledge about a significant problem or its solution via the design of an artefact\\
 \textbf{Case study research} & to investigate in great depth a notable instance of what is under study, in its real-world context\\
 \textbf{Systemic inquiry} & to explore complex, messy problematic situations involving multiple and often contrasting perspectives, with the aim of transforming the situation for social improvement\\
 \textbf{Systematic review research} & to generate new insights from published academic work\\
 \textbf{Ethnography} & to study the culture of a group of people in their natural setting\\
 \textbf{Action research} & to improve the researcher's own professional practice\\
 \textbf{Mixed-methods research} & to combine quantitative and qualitative research to gain different perspectives on phenomena of interest\\
\end{ltabulary}
\end{table}

%\begin{table}[htbp]
%\caption{Research strategies introduced in this section\label{tab:resstrat}}
%\centering
%\small
%\begin{ltabulary}{\textwidth}{@{}p{2cm}LLLLLL@{}}
%\toprule
%
% \textbf{name} & \textbf{aim} & \textbf{data collection} & \textbf{data analysis} & \textbf{success factors} & \textbf{advantages} & \textbf{disadvantages} \\
%\midrule
%
% \textbf{survey research} & to gain insights which are valid across a target population, by collecting data from a predefined sample in a standardised and systematic way & you need to identify upfront which data you will collect in a standardise matter, your target population and sample & you seek patterns in the sample data collected to devise generalisations to the wider population & you must be able to access an appropriate sample and generate a sufficient volume of data & - can produce a lot of data in a short time - data collection can be replicated on different samples, or the same sample on a later time & * lack of depth * focus on what can be measured * provides a snapshot at a particular time, rather than a longitudinal view * can't reveal cause-and-effect relationships \\
% \textbf{experiment\-al research} & to investigate cause and effect relationships between factors by testing hypothesis or proving\slash disproving causal links & you need first to state the hypothesis to be tested, then make detailed observations and measurements of outcomes and any changes that take place when particular factors are introduced or removed & you seek to explain causal links between factors under study, looking at your observations and measurements under the different experimental conditions & you must be able to control factors which may affect the outcome. This is possible in laboratory experiments, while the level of control in field experiments is diminished. & * there are well established processes and protocols * tailored to the study of causal relations & * Laboratory experiments are very reliable due to the high level of control, but can be very artificial, with little or no relation to a real-world context * The opposite is true for field experiments \\
% \textbf{design science research} & to generate new knowledge about a significant problem or its solution via the design of an artefact. It simultaneously generates knowledge about the problem, the artefact and the method used to design it. By artefact is meant anything made by humans, so this is a very broad definition encompassing all that does not exist in nature & you need to both articulate the problem, and design, construct and evaluate the solution artefact & you need shed new insights on the problem, and argue how solution and solution process contribute new knowledge & you must be able to argue that it is not `normal' design & * leads to tangible artefacts which fit a real-world context * is particularly suited to emerging and rapidly changing technology-related fields of study & * might be difficult to generalise to other real-world settings * may require advanced technical skills * may lead to short shelf-life of the research, particularly in technological volatile fields of study where technology becomes quickly obsolete \\
% \textbf{case study research} & to investigate in great depth a notable instance of what is under study, in its real-world context & you must be able to articulate your personal stake in the situation, for example, a deeply felt interest or active involvement , rather than assuming and claiming unbiased passive `neutral' observation. You must also keep your own journal during the course of your research inquiry, & you may seek to explore questions or hypotheses for follow-up research, or provide a detailed account of a phenomenon in its natural context, or explain why certain outcomes or phenomena have occurred & you must be able to analyse the significant instance holistically and in context & * allows the study of a complex situation where several factors are at play * allows the researchers to explore alternative meanings and explanations & * can be time-consuming and access may be difficult to obtain * may be perceived as lacking rigour * insights may be difficult to generalise \\
% \textbf{systemic inquiry} & to explore complex, messy situations involving multiple and often contrasting perspectives, with the aim of transforming the situation for social betterment. & - bricolage research applies, in which you can complement traditional research methods with other tools and techniques from your existing repertoire of expertise and professional tradition - your research journal contributes evidence & * mainly a qualitative endeavour * bricolage research applies, in which you can complement traditional research methods with other tools and techniques from your existing repertoire of expertise and professional tradition & * you must articulate your research problem with reference to one of more systems of interest, and frame the research in terms of possible systems change. * you must articulate your personal stake in the situation * You must also have access to sources of different perspectives on the situation under study in order to generate your evidence & * helps you make sense of complex situations of change and uncertainty * support development of trust amongst participants, including trust with the researcher in co-exploring the situation & - acknowledges that the research will not deliver `certainty' in terms of `problem-solving' associated with complex situations \\
% \textbf{mixed methods research} & to gain different perspectives on phenomena of interest, by exploring connections and contradictions between quantitative and qualitative data & will depend on the particular combination of methods selected & a key aspect is consideration of how connections between findings are established, through comparing and contrasting data from the different methods applied & you must be able to apply competently different kinds of methods & * can provide a more holistic understanding of the phenomena under study, and facilitate different avenues for exploration * particularly suited to situations in which neither quantitative nor qualitative methods alone can provide sufficient insights & * add complexity to the research design * can be demanding and time consuming \\
% \textbf{systematic review research} & to generate a scholarly synthesis of evidence in relation to a specific research problem or question & you need to establish and apply a rigorous set of criteria to identify, select, and critically appraise relevant research from previously published studies & you need to generate a critical synthesis of evidence based on the selected set of criteria & - your review must be both systematic and extensive - you need to be a skilled critical thinker and academic writer & - is considered transparent, reliable, and easy to replicate & - can be very time-consuming * is only as reliable as the studies reviewed * can be difficult to synthesise findings from potentially very different studies \\
%\bottomrule
%\end{ltabulary}
%\end{table}


%\subsection{Philosophical traditions}
%Research methods and research strategies are strongly related to {philosophical traditions}\footnote{The term \emph{research paradigm} is also used in the literature with a similar meaning.}, which are world-views that inform how one should conduct research. Philosophical traditions may sound a bit esoteric, but they matter in that they make explicit assumptions behind research design choices, influencing what a researcher chooses to research and the way they may go about collecting evidence or interpreting findings.
%
%Each philosophical tradition embodies a set of beliefs around three fundamental philosophical issues:
%
%- The nature of our world, which relates to questions such as: What is there? What kind of categories do things belong to? How are those categories related? The part of philosophy dealing with these questions is called \textit{ontology}. In research design, ontology determines which phenomena are there to be studied as part of the research, and underlies our experience of the world. Hence, ontology is closely connected with the kind of observations we make or evidence we gather.
%
%- How knowledge is acquired, which relates to questions such as: What does it mean to know something? How can one claim to know something? What makes a belief justified? The part of philosophy dealing with these questions is called \textit{epistemology}. In research design, epistemology is closely related to research methods for knowledge creation and validation.
%
%- What are the values, especially in relation to ethics, which relates to questions such as: What is good or bad? What is right or wrong? Where do values come from? How do we justify our values? The part of philosophy dealing with these questions is called \textit{axiology}. In research design, axiology is closely related to ethical considerations when planning or executing research.
%
%In what follows, you will find a brief introduction to some of the better known and most often cited traditions. However, you should be aware that their definitions are not universal, their boundaries not clear-cut, and it is very rarely the case that a research design will fit a specific tradition neatly. You should, instead, consider each of these traditions as a `wrapper' of convenience for a set of beliefs on research practice which have emerged from different disciplines and cultures, and also be aware that such beliefs have changed over time, and continue to do so.
%
%\paragraph{Positivism}
%This is perhaps the oldest tradition, with roots in the natural sciences. It sees the world as ordered and regular, with universal laws governing its functioning, and assumes it can be investigated objectively.
%
%Specifically, positivism encompasses the following set of beliefs:
%
%\begin{itemize}
%\item There is a physical world which exists `out there' and can be observed and measured. This also implies that all researchers will observe and measure the same phenomena in exactly the same way.
%
%\item Through observations and measurements, the researcher can produce models of how the world functions, which are `true' explanations of the aspects of the world under study. This also implies that only one true explanation exists.
%
%\item Truths about the world are perfectively objective and independent of the researcher's values or beliefs. This means that all researchers will arrive at the same truth.
%
%\item Research is based on the empirical testing of theories or hypothesis, leading to either confirmation or rejection (a.k.a. `refutation'). As there can only be one truth, either the theory or hypothesis tested is that truth, in which case all subsequent tests will confirm it, or it is not that truth, in which case at some point a test will reject it. The term refutation is used to indicate that a truth, albeit universal, is always tentative: it will be valid until somebody comes up with a test to reject it.
%
%\item Research seeks universal laws and irrefutable facts. This means that re-testing such laws or facts should always confirm them, if they are indeed truths.
%
%\end{itemize}
%
%For instance, starting with the hypothesis that `all swans are white', a positivist researcher would set as a test to look for swans and observe their colour. If all swans are seen white, then the hypothesis would be confirmed, if not, then it would be rejected. If the hypothesis is confirmed, then the truth that `all swans are white' is added to the body of knowledge and will remain so until another test will lead to a rejection --- indeed that's what English people believed until they first spotted a black swan in Australia!
%
%\begin{question}[subtitle={Activity: Summarising positivism}] Given these beliefs, what does positivism assume of the nature of the world (ontology), how knowledge is acquired (epistemology), and what is of value in research (axiology)?
%
%\begin{solution}Ontology: the world exists independently of the researcher, and can be observed and measured objectively.
%
%Epistemology: there are universal truths, which can be acquired by empirical testing of theories and hypothesis. Tests can lead to either confirmation or rejection. Confirmed theories and hypothesis are added to the body of knowledge.
%
%Axiology: positivism values objectivity above all, and dismisses individual's subjective views or experience.
%\end{solution}\end{question}
%%%Hack to correct tcbox behaviour
%\color{black}
%
%Positivism has attracted criticism particularly from the social sciences, which consider some of its beliefs untenable, primarily that researchers are totally objective and not influenced by their own values and beliefs, or that knowledge is made of perfectly generalisable truths. This has led to other traditions, which we consider next.
%
%\paragraph{Interpretivism/Constructivism}
%With its roots in the social sciences, {interpretivism} seeks to identify, explore and explain phenomena in social settings, acknowledging that people perceive the world in different ways, mediated by their beliefs, attitudes and values.
%
%Specifically, interpretivism encompasses the following set of beliefs:
%
%\begin{itemize}
%\item Different individuals, groups or cultures perceive the world differently and what people consider real is a construction of their mind --- leading to the term \textit{constructivism} also being used.
%
%\item The researcher is not neutral, and their perceptions of the world are influenced by their values or beliefs. This implies that different researchers can perceive the same phenomena in different ways, and there is no single truth or single explanation of the world.
%
%\item As there are different perceptions of reality, communication among groups of individuals is the only way of constructing some shared meaning or understanding, and this will change over time.
%
%\item As researchers are influenced by their own values and beliefs, they will arrive at different interpretations as a result of their observations. The strengths of their interpretations will depend on the strengths of the evidence and arguments their interpretations are based upon.
%
%\item Research is based on studying people and other phenomena in their `natural' context. Such a context can be unique, so that interpretations based on observations may not be generalisable to other contexts.
%
%\end{itemize}
%
%\begin{question}[subtitle={Activity: Summarising interpretivism/constructivism}] Given these beliefs, what does interpretivism assume of the nature of the world (ontology), how knowledge is acquired (epistemology), and what is of value in research (axiology)?
%
%\begin{solution}Ontology: the researcher acknowledges that they perceive the world based on their belief, values and culture.
%
%Epistemology: the researcher will offer interpretations based on observations in a social context. Different researchers may offer different interpretations. All knowledge is constructed and shared understanding is reached through communication. Interpretations in one context may not be generalisable to other social contexts.
%
%Axiology: The researcher's values and beliefs matter. The strength of their interpretations will depend on the strengths of the evidence and arguments in support.
%
%\end{solution}\end{question}
%%%Hack to correct tcbox behaviour
%\color{black}
%
%\paragraph{Critical theory}
%Perhaps not as well established as the previous traditions, {critical theory} originated in the fields of sociology, philosophy and political theory.
%
%Like interpretivism, it assumes multiple interpretations of reality in social contexts. However, it goes a step further by asserting that reality is shaped by those who are powerful, who legitimate particular ways of perceiving the world: `truth' is inherently political, defined by those in charge to the disadvantage of many, and challenged by those who wish to promote equality.
%
%As a result, critical researchers seek to challenge the status quo and perceive research as transformative at a social level, confronting ideology and trying to discover and challenge the mechanisms through which exploitation and disadvantage are perpetuated in society.
%
%\begin{question}[subtitle={Activity: Summarising critical theory}] Given these characteristics, what does critical theory assume of the nature of the world (ontology) , how knowledge is acquired (epistemology), and what is of value in research (axiology)?
%
%\begin{solution}Ontology: reality is the product of power relations, shaped by those who are powerful and there are disadvantages for many.
%
%Epistemology: the researcher confronts ideology and tries to discover the truth of exploitation and the mechanisms by which disadvantage is perpetuated to challenge the status quo and promote social justice and equality.
%
%Axiology: The researcher has the moral responsibility to make things better in society.
%
%\end{solution}\end{question}
%%%Hack to correct tcbox behaviour
%\color{black}
%
%\paragraph{Indigenous}
%The traditions described so far are attracting increasing criticisms in that they are seen as Western-European centric and often imposed on other indigenous cultures as a result of colonialism.
%
%In counterposition, an indigenous research tradition has emerged with a social and political agenda of decolonising indigenous societies. It emphasises the connection between the researcher and their own culture, in the sense that cultural practices and forms of expressions should be reflected in the way the research is conducted, including language, metaphors, oral traditions and knowledge systems. It also advocates an holistic approach which strives to reach a balance between different areas of life, integrating intellectual, social, political, economic, psychological and spiritual dimensions.
%
%\begin{question}[subtitle={Activity: Summarising indigenous traditions}] Given these characteristics, what does the indigenous tradition assume of the nature of the world (ontology), how knowledge is acquired (epistemology), and what is of value in research (axiology)?
%
%\begin{solution}Ontology: reality is determined by the indigenous culture, to which the researcher is strongly connected.
%
%Epistemology: this is determined by indigenous knowledge systems, cultural practices and forms of expressions.
%
%Axiology: The researcher has a social and political agenda of decolonisation of indigenous societies.
%
%\end{solution}\end{question}
%%%Hack to correct tcbox behaviour
%\color{black}

\section{Understanding research methods and strategies in articles you have reviewed}
To ground what you have learnt so far on research methods and strategies, you should look back at some at the articles you have reviewed to see how they describe and use them. This may also give you some ideas on how to apply them within your own project. You should be aware, however, that terminology used in the literature may different from that of this handbook\footnote{We have already noted how some terms are used differently by different authors.}.

\begin{example}{Considering research methods and strategies in published work} 
Looking back to our curriculum analytics example, one of the papers we reviewed was:
\begin{quotation}
	\textit{Gray, G., Schalk, A. E., Cooke, G., Murnion, P., Rooney, P., \& O'Rourke, K. C. (2022). Stakeholders’ insights on learning analytics: Perspectives of students and staff. Computers \& Education, 187, 104550.}
\end{quotation}

In this paper there is a `Methodology' section which describes the study research design in detail. Specifically, it states that a mixed-methods strategy was used to collect and analyse qualitative and quantitative data. 
For data collection both questionnaires and a focus group were. For qualitative data analysis, thematic analysis was applied; for quantitative data analysis, tables and charts were used, including some calculation of differences in scores obtained from questionnaire answers.

The paper usefully includes some descriptions of the specific steps the researchers took to recruit participants and analyse the data, something that could be replicated in new studies.

We found differences in terminology:
\begin{itemize}
	\item mixed-method `approach' is used for what we call mixed-methods strategy, and 
	\item `survey' is used to mean a questionnaire: this is actually rather common in the literature, where the term survey is found to mean both.
\end{itemize}
\end{example}


\begin{question}[subtitle={Activity: Considering research methods and strategies in articles you have reviewed}] Go back to two or three articles you have reviewed, perhaps those you have found most interesting or closest to the research you intend to do. 

Look for research methods and strategies they use, and consider how these are presented and applied. Try to establish links to what you have learnt so far, including noting any differences in the terminology used. If appropriate, write down specific points which may help you apply them in your own project.  
\begin{guidance}
Often articles include a 'Methodology' or 'Methods' section where research methods and strategies are discussed: that was the case in our example. That's the section you should start from. It may be, however, that more relevant content is described elsewhere, so also look for sections that summarise data or evidence collected and analysed. 

You may want to skim through few articles before deciding which ones to consider in detail.
\end{guidance}
\end{question}
%%Hack to correct tcbox behaviour
\color{black}



%to re-write as understanding of research design found in reviewed articles. Plus considering candidate options for research design 

%%LR -- stopped here 9/11


%Table~\ref{tab:dataevidence} provides a summary of the type of data and evidence that were introduced in Stage 1. In the next activity you will reflect on those which are most relevant to your project.
%
%\begin{table}[htbp]
%\begin{minipage}{\linewidth}
%\setlength{\tymax}{0.5\linewidth}
%\centering
%\small
%\caption{Common types of data/evidence used in a research project}\label{tab:dataevidence}
%\begin{tabulary}{\textwidth}{@{}LL@{}} \toprule
% \textbf{Types of data\slash evidence} & \\
%\midrule
%
% \textbf{Quantitative} data that can be quantified or measured, and be given numerical values & \textbf{Numerical} numbers, either discrete or continuous \\
% & \textbf{Ordinal} can be arranged in an order, but are not necessarily numerical \\
% & \textbf{Interval} ordinal data for which we can calculate precisely the interval between any two data points \\
% \textbf{Qualitative} all other data which is descriptive in nature & \textbf{Categorical (or nominal)} correspond to categories which cannot be ordered and on which mathematical operations and function don't apply \\
% & \textbf{Other} e.g., texts, words, images, sounds, etc. \\
%\bottomrule
%
%\end{tabulary}
%\end{minipage}
%\end{table}
%
%\begin{question}[subtitle={ACTIVITY: Considering data and evidence}] Consider whether you will need to use qualitative or quantitative data in your project. Write down 
%
%\begin{guidance}You should use the second column to help you decide, i.e., \enquote{I'll be using a survey combining a Likert scale with free text answers, so that means I'll be using Ordinals, and so quantitative, and Other, and so qualitative.}
%
%For each needed in your project, you should also give specific examples and indicate the source.
%\end{guidance}\end{question}
%%%Hack to correct tcbox behaviour
%\color{black}


%\subsection{Putting it all together}
%You should now have enough material to sketch your overall research design.
%
%\begin{question}[subtitle={Activity: Sketching your overall research design}] Based on your judgements, expressed in the previous activities, summarise your research design by addressing each of the following questions:
%
%\begin{itemize}
%\item Which evidence and data will you need and why?
%
%\item Where will you source such data\slash evidence from?
%
%\item Which research strategy are you thinking of adopting and why?
%
%\item Which research methods are you thinking of applying for your data collection\slash analysis or modelling within that strategy?
%
%\end{itemize}
%
%Ensure that your answers are justified in term of your research problem, and intended aim and objectives, by indicating explicitly the rationale behind your choices.
%
%\begin{guidance}At this point, your choices will be tentative, but should provide a good starting point for further investigation and a meaningful conversation with your supervisor who will be able to advise you further.
%
%As well as your intended research, you should also keep in mind that you will have limited time and resources to complete your project, so you should limit your choices to be:
%
%\begin{itemize}
%\item manageable in terms of their application in the context in which you are going to conduct your research and the time you have available
%
%\item efficient in terms of the data\slash evidence they produce and your ability to process them with the resources and time you have available
%
%\item effective at producing data\slash evidence that you have the skills and expertise to analyse in the time you have available.
%\end{itemize}
%
%
%\end{guidance}\end{question}
%%%Hack to correct tcbox behaviour
%\color{black}


%% LR -- I think all these moves to Stage 3!!!!!!!!!

%%%%LR -- this to move to Stage 3, if kept at all
%\subsection{Investigating research strategies and methods further}
%The overview provided in this section was designed to help you develop a broad understanding of possible choices you can make to help you sketch your initial research design. It is now time to start looking a little deeper into your most likely research strategies and methods.
%
%\begin{question}[subtitle={Activity: Reviewing your chosen research strategies and methods}] Consider the research strategies and methods you have included in your sketched research design. Conduct a small literature review on each of them to help you confirm that they are indeed suitable for your research, and to help you articulate how and why they are suitable for your project.
%
%\begin{guidance}As this is a literature review, your should follow the process and practices you have already learnt and applied, including recording entries and notes in your BMT.
%
%You should focus on materials which will help you understand how they work, and their strength and weaknesses in relation to your project, something you will return to in the next stage of your project, where you will consider specific procedures for applying them.
%
%Your review does not need to be extensive: a couple of references for each strategy\slash method should suffice, as long as they provide the information required. You can use the annotated reading list in the next section as your starting point, but you should also explore the wider literature. Your supervisor should also be able to suggest literature you could start from.
%\end{guidance}\end{question}
%%%Hack to correct tcbox behaviour
%\color{black}
%
%\section{Annotated reading list}\label{sect:readingList}
%
%\todo{Move these into body, and make standard bibliographic references(!)}
%
%There are many resources which cover a variety of research strategies and methods in more detail that we can here. You could start from the following:
%
%\begin{itemize}
%\item GO-GN (2020), Research Methods handbook, \href{https://go-gn.net/wp-content/uploads/2020/07/GO-GN-Research-Methods.pdf}{https:\slash \slash go-gn.net\slash wp-content\slash uploads\slash 2020\slash 07\slash GO-GN-Research-Methods.pdf} This is a practical introduction to research methods for phd research, written with contributions from doctoral research students.
%
%\item Oates, B.J. (2006) Researching information systems and computing, SAGE. This is very good read for novice research students, particularly in information systems and computing disciplines. It provides a clear, practical and comprehensive introduction to academic research, including key definitions, methods and techniques.
%
%\item The SAGE research portal at \href{https://methods-sagepub-com.libezproxy.open.ac.uk}{https:\slash \slash methods-sagepub-com.libezproxy.open.ac.uk} contains a great variety of resources on both strategies and methods, from articles to video tutorials
%\end{itemize}
%
%\paragraph{For Design Science Research} start from:
%
%\begin{itemize}
%\item vom Brocke J., Hevner A., Maedche A. (2020) Introduction to Design Science Research. In: vom Brocke J., Hevner A., Maedche A. (eds) Design Science Research. Cases, Cham. \href{https://doi.org/10.1007/978-3-030-46781-4_1}{https:\slash \slash doi.org\slash 10.1007\slash 978-3-030-46781-4\_1}, which is possibly the most up-to-date introduction to the topic.
%
%\item The International Conference on Design Science Research in Information Systems and Technology (DESRIST) (since 2005) have tracked the development of this research strategy, with many seminal papers published its proceedings. These can be accessed via the OU Library.
%
%\end{itemize}
%
%\paragraph{For case study research} start from:
%
%\begin{itemize}
%\item Yin, R.K., 2009. Case study research: Design and methods. Applied social research methods series, Vol. 5. Fourth Edition, Sage.
%
%\end{itemize}
%
%\paragraph{For systemic inquiry} start from:
%
%\begin{itemize}
%\item Ison, R. (2017). Systemic inquiry. Ch. 10 in Part 3 Systems Practice: How to Act (pp. 251--274). Springer, London, which is also available as eBook Reading
%
%\item Simon, G and Chard, A. eds. (2014) Systemic Inquiry: Innovations in Reflexive Practice Research. Farnhill: Everything is Connected Press.
%
%\item Ison, R.L., Collins, K.B. and Iaquinto, B.L., 2021. Designing an inquiry-based learning system: Innovating in research praxis to transform science--policy--practice relations for sustainable development. Systems Research and Behavioral Science, 38(5), pp.610--624.
%
%\item McClintock, D., Ison, R. and Armson, R., 2003. Metaphors for reflecting on research practice: researching with people. Journal of Environmental Planning and Management, 46(5), pp.715--731.
%
%\item Kincheloe, J. L. (2011). Describing the bricolage: Conceptualizing a new rigor in qualitative research. In Key works in critical pedagogy (pp. 177--189). Brill.
%
%\end{itemize}
%
%
%\paragraph{Problem diagrams} are part of a wider approach called problem oriented engineering, with roots in software development, but more widely applicable to most forms of design and engineering. The following references are a good starting point:
%
%\begin{itemize}
%\item Jackson, M., 2005. Problem frames and software engineering. Information and Software Technology, 47(14), pp.903--912.
%
%\item Hall, J., Rapanotti, L. and Jackson, M., 2008. Problem oriented software engineering: Solving the package router control problem. IEEE Transactions on Software Engineering, 34(2), pp.226--241.
%
%\end{itemize}
%
%\paragraph{UML} is a modelling language with roots in software engineering. It is now an international standard that can be found at:
%
%\begin{itemize}
%\item \href{https://www.iso.org/standard/52854.html}{https:\slash \slash www.iso.org\slash standard\slash 52854.html}
%
%\item However, many tutorials are available online, so it should be relatively easy for you to find an introductory one. There are also many UML digital modelling tools, some of which are open source and free to use
%
%\end{itemize}
%
%\paragraph{Systems diagrams}
%
%The system thinking diagramming tutorials on the Open University's Open Learn site are a good starting point to look up these techniques. Link: \href{https://www.open.edu/openlearn/science-maths-technology/across-the-sciences/systems-thinking-diagramming-tutorials}{https:\slash \slash www.open.edu\slash openlearn\slash science-maths-technology\slash across-the-sciences\slash systems-thinking-diagramming-tutorials}. For an application of these techniques on a particular case study you can also consider \href{https://www.open.edu/openlearn/science-maths-technology/computing-ict/diagramming-development-1-bounding-realities/content-section-0?active-tab=description-tab}{https:\slash \slash www.open.edu\slash openlearn\slash science-maths-technology\slash computing-ict\slash diagramming-development-1-bounding-realities\slash content-section-0?active-tab=description-tab}
%
%
%\paragraph{For tables, graphs and charts} start with the following free resources: \href{https://www.open.edu/openlearn/science-maths-technology/mathematics-statistics/working-charts-graphs-and-tables/content-section-0?active-tab=description-tab}{`Working with charts, graphs and tables'} %
%%
%%\item  `More working with charts, graphs and tables' at:
%%
%%\href{https://www.open.edu/openlearn/science-maths-technology/mathematics-statistics/more-working-charts-graphs-and-tables/content-section-0?active-tab=content-tab}{https:\slash \slash www.open.edu\slash openlearn\slash science-maths-technology\slash mathematics-statistics\slash more-working-charts-graphs-and-tables\slash content-section-0?active-tab=content-tab}
%%
%or the UK BBC Skillswise site at: \href{https://www.bbc.co.uk/teach/skillswise/graphs/zmkpqp3}{https:\slash \slash www.bbc.co.uk\slash teach\slash skillswise\slash graphs\slash zmkpqp3}.\todo{Annotated bibliography needs completing for all research methods.}

\chapter{Managing risk in Stage 2}

More here

\chapter{Reflecting and reporting in Stage 2}

Well, you're reached the end of stage 2. You're really speeding along now. 

Before carrying on, it's time to reflect and write up your Stage 2 report.

\begin{question}[subtitle={Activity: Reflecting on your learning and practice}]
As you did at the end of Stage 1, in this activity you are asked to stand back and reflect deeply on what you have leant and done, the wider context of your work and your own attitude to it. Specifically, you are asked to think deeply about each of the following:

\begin{itemize}
	\item your study this far
	\item the way you work. Are you tidy and systematic, or let things happen organically? For instance, how does
	\item the context of your research
	\item your feelings about your project
\end{itemize}

You should also think of any significant changes with respect to your reflection in Stage 1
\begin{guidance}
You should refer back to the guidance to this activity in Stage 1, Section~\ref{sect:stage1Reflection}.
\end{guidance}\end{question}
%%Hack to correct tcbox behaviour
\color{black}

Your end-of-Stage 2 report will help you consolidate your work so far, and develop your dissertation incrementally. Its recommended structure and content is indicated in Table~\ref{tab:S2report}, which builds one that of your Stage 1 report.

\begin{table}[htbp]
\caption{Report structure and guidance guidance\label{tab:S2report}}
\centering
\begin{tabulary}{\tablewidth}{@{}LL@{}} \toprule
 \textbf{Report template} & \textbf{Guidance} \\
\midrule

 Proposed title & Your title should continue to capture succinctly your research problem and aim. \textit{It is likely this is the same as, or very similar to, that in Stage 1}\\
 Sect 1 - Introduction 1.1 Background to the research 1.2 Justification for the research 1.3 Fitness of the research & This section should continue to provide an introduction to your research topic in its wider context (as background) and your justification of why the research is worth pursuing. Its purpose is to introduce and justify your intended research in overview, before entering the detailed work of the subsequent sections. It should be well argued and supported by appropriate citations. In this section, you should also argue how the research fits within the scope of your qualification, and meets any other personal, professional or organisational criteria. \textit{You may review this section from Stage 1 to reflect your growing understanding of the topic in context derived from your literature review.} \\
 Sect 2 - Literature review 2.1 Introduction 2.2 Main body 2.3 Critical summary & \textit{This section should consist of your current literature review, developed in this stage by following the advice in Section~\ref{sect:stage2literatureReview}}. At this point, it should be a substantial, almost complete, draft, well structured and articulated through solid academic arguments. It should demonstrate your understanding of the main literature which relates to your research problem, clearly identifying the knowledge gap your project will address.\\
 Sect 3 - Research definition 3.1 Problem statement 3.2 Aim and objectives 3.3 Knowledge contribution & You should continue to ensure that your research problem is well articulated and appropriate for your course and your personal and professional circumstances, that your aim and objectives are consistent with research problem, and that the intended knowledge contribution of your research is clearly articulated. \textit{You may revise this section from Stage 1 in view of your increasing understanding from your literature review.} \\
 Sect 4 - Research design 4.1 Evidence and data 4.2 Research strategies and methods 4.3 Ethical, legal and EDI considerations & \textit{This section should extend your Stage 1 work with your considerations of candidate research strategies and methods for your project, based on the guidance in Section~\ref{sect:stage2researchDesign}}.  \\
 Sect 5 - Work planning and risk assessment 5.1 Statement of progress 5.2 Key priorities in follow-up stage 5.3 Risk assessment & In this section you should reflect on the progress you have made in Stage 2 and establish your priorities for the next stage. You should also review your risk assessment as appropriate. \\
 References & You should keep your growing references in good order and ensure you apply the required bibliographical style consistently. Ideally, you should use a BMT to generate and integrate your references within your report \\
 Appendix - Work schedule & You should include your revised work plan as an appendix \\
 Appendix - Risk assessment table & You should include your updated risk table as an appendix \\
\bottomrule
\end{tabulary}
\end{table}

\begin{question}[subtitle={Activity: Writing and assessing your report for Stage 2}] Using your word processor of choice, revise and expand your Stage 1 report by applying the structure and guidance in Table~\ref{tab:S2report}. 

Assess your report by applying the criteria in Table~\ref{tab:criteriaForStage2report}. Revise and iterate until you are ready to move on. 
\begin{guidance}
In completing your report, you should make good use of notes and summaries your wrote as part of the activities in this chapter.
In evaluating your report, for each criteria, you should consider the related prompts, write down any further work needed for your next stage, and update your work plan and risk assessment table accordingly.
\end{guidance}\end{question}
%%Hack to correct tcbox behaviour
\color{black}

\begin{table}
	\caption{Criteria for reviewing your research proposal \label{tab:criteriaForStage2report}}
\begin{tabulary}{\tablewidth}{@{}LL@{}} \toprule
 \textbf{Criteria} & \textbf{Prompts} \\
\midrule
 \textbf{Completeness} & Are all sections included and their content complete? What is missing?\\
 \textbf{Academic writing} & Have you applied good academic writing practices throughout? Which main issues do you still have to address? \\
 \textbf{Logical structure and flow} & Have you structured your writing appropriately to ensure a logical flow of arguments? Which restructuring may be needed?\\
 \textbf{Supporting evidence} & Are your key arguments supported by appropriate references or other evidence? Which further evidence is needed?\\
 \textbf{Citation and reference style} & Do all your citations and references comply with the required bibliographical style? \\
 \textbf{Avoiding plagiarism} & Have you acknowledged the work of others and distinguished it from your own appropriately? \\
 \textbf{Grammar and spelling} & Have you proof-read your report carefully to remove all typos and grammatical errors? \\
\bottomrule
\end{tabulary}
\end{table}


\chapter{Stage 2 Takeaways}
\begin{itemize}
\item the fundamental skills for synthesising the literature are critical thinking and writing, and the ability to establish connections between ideas and arguments
\item academic writing requires you to observed a number of practices, which ensure your writing is clear, precise, logical and well-structured
\item an academic argument is a structured argument whose key elements help you ensure your claims are carefully reasoned and supported
\item your theme summarised from Stage 1 are your starting point when writing your literature review
\item your literature review will be adequate if it contextualise and justify your research in the context of related academic literature, and is well structured and logically argued
\item the building blocks of research design are research methods and strategies, together with their underlying philosophical traditions
\item several research methods exist to help you collect and analyse data and evidence, or model real-world scenarios or systems and artefacts
\item several research strategies exist to help you meet your research objectives
\item philosophical traditions capture different ways of thinking about knowledge generation and values in research
\item knowing the building blocks of research design helps you understand how research reported in academic articles was conducted
\item the template provided can help you structure your Stage 2 report
\end{itemize}

\begin{ReportTable}{stage2WritingOutcomes}[\WOCaption{2}{20}]
%%\ReportTitle1
\ReportTitle*
	&\begin{itemize}\item Update your provisional title if necessary \end{itemize}\&\\

%%\ReportAbstract12345
\ReportAbstract-----
	&\begin{itemize}\item\tablenocontent\end{itemize}&
\\

%%1-\ReportIntroduction12345
\ReportIntroduction***!
& \begin{itemize}\item Update background and justification based on your increasing understanding from reviewing the literature 
	\item Add a definitions section, which explains your use of acronyms and other technical terms to assist the reader\end{itemize}
	&%
\\

%%2-\ReportLitRev---
\ReportLitRev**!
	&\begin{itemize}\item Complete a full draft of your literature review. There may be small additions and adjustments required later on in your project, but the main body of your literature review should now be completed. Your review should be the outcome of your deep engagement with the academic (and other) literature around your chosen topic, building on the work you started in Stage 1
	\item Write a critical summary of the key insights, including articulating and justifying the knowledge gap your project will address. 
	\end{itemize}&\Cref{sect:DevelopingYourLiterature}
\\

%%3-\ReportResearchDef----
\ReportResearchDef****
	&\begin{itemize}\item Revise your problem statement, aim and objectives, and knowledge contribution based on your increased understanding from your literature review \end{itemize}&\\

%%4-\ReportResearchDes-----
\ReportResearchDes*!!-*
	&\begin{itemize}\item Include your initial consideration of the data and evidence you will need in your project and where they may come from.
	
	\item Outline possible research methods and strategy you could apply, justifying why you think they may be appropriate for your project. At this point this is only a tentative account that you will review and develop in the next stage 
	\end{itemize}&\Cref{sect:evidenceAndData,sect:stage2methodology}
\\

%%5-\ReportAnalysisInterp----
\ReportAnalysisInterp----
	&\begin{itemize}\item\tablenocontent\end{itemize}
\\

%%6-\ReportEvalConc-------
\ReportEvalConc-------
& \begin{itemize}\item\tablenocontent\end{itemize}\\

%%7-\ReportRefs-
\ReportRefs*
	&\begin{itemize}\item \tablecites{}\end{itemize}\\

%%%A-\ReportDissertationAppendices
\ReportDissertationAppendices--
	& \begin{itemize}\item\tablenocontent\end{itemize}\\
	
%%\ReportProgressTracking123456789
<<<<<<< HEAD
\ReportProgressTracking***?*??
&\begin{itemize}\item Revise the content of this appendix in view of your increased understanding and the progress you have made in this stage, paying particular attention to feasibility, work plan and risk assessment
\end{itemize}&\\
=======
\ReportProgressTracking**???*
& Revise the content of this appendix in view of your increased understanding and the progress you have made in this stage, paying particular attention to feasibility, work plan and risk assessment
&\\
>>>>>>> 7ceb27cef66abd429ac2080cd0974e740e9793ce
%
%%\ReportReflection123
\ReportReflection*?
 &\begin{itemize}\item Update your personal statement based on actions and outcomes in this stage 
 \end{itemize}&\\
\end{ReportTable}
\endinput



\section{Planning your work for this stage}\label{sect:stage2WorkPlan}

This is a good point to refine your project \glsf{work-plan} to include more detail on your work for~\Cref{stage2}. 


%For this, you will need to:
%\begin{itemize}
%	\item make sure you have completed all the work for~\Cref{stage1} or make the necessary adjustments to your plan
%	\item identify the main tasks under each activity for this stage, allocate them time and include them in your plan. For complex tasks, you may also include some sub-tasks,~\etc{}, but you should avoid making your plan too complicated
%	\item establish key \glsplf{milestone} and \glsplf{deliverable} and include them in your plan
%	\item optimise your plan by considering dependencies and tasks which may overlap. 
%\end{itemize}


\begin{question}[subtitle={Activity: Revising your work plan}] 
Consider the activities in~\Cref{stage2ResearchActivities} and the writing outcomes in~\Cref{stage2WritingOutcomes}: 

\begin{itemize}
	\item for each activity, identify a number of tasks which capture the work needed, decide how much time to spend on each, and include them in your \glsf{work-plan}, also taking into consideration their possible dependencies
	\item for each \glsf{writing-outcome}, identify corresponding \glsplf{deliverable} and set related \glsplf{milestone} in your \glsf{work-plan}.
\end{itemize}

At the end, review your overall plan, also considering the progress you made in the previous stage, and make all necessary adjustments. 
\begin{guidance}
%Hack to correct tcbox behaviour
\color{black}


%Hack to correct tcbox behaviour
\color{black}
Make sure you:
\begin{itemize}
	\item focus on a small number of key tasks for each activity, so as to keep your plan light
	\item when allocating time to tasks, ensure that tasks fit within the overall time for their corresponding activity 	
	\item consider task dependencies and things you can progress in parallel, so to optimise your project time 
	\item break down new content you will need to write into \glsplf{deliverable}, setting appropriate \glsplf{milestone} in your plan.
\end{itemize} 
\end{guidance}
\end{question}
%%Hack to correct tcbox behaviour
\color{black}

If after reviewing your progress you find that you are well behind, then talk to your supervisor who will be able to advise you on how to bring your project back on track and improve your planning.

\chapter{Writing a full draft of your literature review}\label{sect:stage2literatureReview}
In~\Cref{stage2}, you will build on your~\Cref{stage1} work to write a substantial draft of your literature review. Although it won't be in its final form, this draft will be close to complete – as you work through later stages, you might find a few more papers\footnote{Depending on how thorough you've been, of course.} to consider, but you'll probably be able to count them on one hand.

While in~\Cref{stage1} your main focus was on gathering and assimilating relevant articles, understanding their relationships and emerging themes, in~\Cref{stage2} your focus will be on consolidating all that you have learnt into a coherent, well-structured and critical narrative.  

This is the essence of \glsf{synthesis} – the act of bringing ideas together into a cogent whole. That cogent whole is the spell-binding story you create to convince your readers\footnote{Eventually, your examiners, first your supervisor, family, friends, and anyone else that will read it{\ldots}} that \textit{you have a contribution to knowledge to make}, based on the literature you have read and the knowledge gaps that you have found there. 

Given the number of articles you'll have to read, and the complexity of the relationships that exist between them, there's a long (and sometimes winding) road from the literature to your dissertation.  The knowledge, process and techniques you acquired and applied in~\Cref{stage1} will still be relevant\footnote{If needed, refresh your understanding by rereading~\Cref{sect:stage1LiteratureReview} in the previous chapter.}: you can make best use of the time you spend reading the literature by using \citeauthor{keshav2007read}'s workflow\footnote{Introduced in~\Cref{sect:Keshav}.}, iterating quickly between papers, and keeping notes that refine what you have as your understanding develops, and will eventually turn into text you can use for your dissertation, too. However, once you have selected, read and understood\footnote{At this point, you may have only a rudimentary understanding, but that will develop as you bring your \glsf{synthesis} together.} a good number of relevant articles, and identified and summarised relevant emerging themes, you can also start to develop the narrative to include in your literature review. 

For this you will need further \glsf{synthesis} skills and techniques, discussed in this chapter. 

While you will encounter them in the context of writing your literature review, these \glsf{synthesis} skills and techniques apply more widely to any form of academic writing, so they will be essential when you write your narrative across your whole dissertation. 

\section{Key skills for synthesising}\label{sect:KeySkillsFor}

There are key skills you will be able to demonstrate through your \glsf{synthesis} of the literature. They are:

\paragraph{\Glsf{critical-thinking}}
You must demonstrate that you have maintained an objective position by weighing up all sides of an argument, evaluating its strengths and weaknesses, and testing how sound the claims made and their supporting \glsf{evidence} are.

At its essence, \glsf{critical-thinking} is the skill of systematically asking questions that get you \enquote{under the hood} of the research -- perhaps into the hidden crevices into which no one has looked before. Part of this is searching for a lack of \glsf{evidence} or poor reasoning behind an argument, instead of accepting what you read \enquote{at face value.}

\Glsf{critical-thinking} develops with time and practice. The trick is a balance of scepticism: choosing the arguments or \glsf{evidence} to take issue with -- those that are less solid -- and leaving the rest.

%\begin{tip}You may not believe this, but there's a notion in software engineering of \enquote{\glsf{coding} bad smells} that was introduced around the turn of the century. Equally unbelievable is that \glsf{coding} bad smells arose from the parenthood experiences of one famous software engineer\footnote{Which you can read about here: \href{http://www-public.tem-tsp.eu/~gibson/Teaching/Teaching-ReadingMaterial/BeckFowler99.pdf}{http://www--public.tem--tsp.eu/\ensuremath{\sim}gibson/Teaching/Teaching--ReadingMaterial/BeckFowler99.pdf}\footnote{\href{http://www-public.tem-tsp.eu/~gibson/Teaching/Teaching-ReadingMaterial/BeckFowler99.pdf}{http://www-public.tem-tsp.eu/\ensuremath{\sim}gibson/Teaching/Teaching-ReadingMaterial/BeckFowler99.pdf}}}, he \enquote{was under the influence of the odors\emph{[sic]} of his newborn daughter at the time{\ldots}}. The point was that an experienced \glsf{coding} specialists just know when something is wrong with code and needs to be looked at. At the beginning of your research career, you're yet to have that experience. But read, anyway, and see if anything \enquote{smells} to you. $<$This really isn't very good$>$\end{tip}
%
As an academic author, \glsf{critical-thinking} will benefit your ability to build stronger arguments, avoid \glsf{bias} and link your claims to appropriate supporting \glsf{evidence}, so it is well worth picking it up as a skill.

\paragraph{Establishing connections}
You must be able to draw links between the articles that you have found, a process you started in~\Cref{stage1} by constructing a \glsf{summary-comparison-matrix}\footnote{Refer back to~\Cref{sect:stage1LiteratureReview} if necessary.} and identifying emerging themes.

As you have experienced in~\Cref{stage1}, this is more than summarising each article in turn: it is about identifying the things that two articles agree and disagree on, then do the same for three, then four,~\etc{}, and begin grouping articles together under the various different themes that you have found, those themes being relevant to your \glsf{research-problem}.

At some point, you'll arrive at a collection of more or less definitive unifying themes to which you can assign the papers you are reading. At this point, you'll have moved on from an article-based process to a theme-based process and your \glsf{synthesis} will be really coming together. So much so that, perhaps\footnote{We're getting a little ahead of ourselves here -- you won't have a complete literature review at this point, but you may start to feel that you can see how, eventually, it could come together{\ldots}}, the themes that you have found could  be used as the titles of sections of your literature review. Your thematic critical \glsf{reflection} on the literature will then be the content of those sections, which will continue to grow and grow into the completed literature review.

\paragraph{\Glsf{critical-writing}}

Many authors of fiction use descriptive writing to give vivid, detailed descriptions of their characters in the hope you will feel some empathy for them. Being able to write descriptively is a great skill for a fiction author to have.

Descriptive writing is also an essential part of academic writing, but its use is very different: in academic writing descriptive writing is used to set the context and to provide any existing \glsf{evidence} behind an argument you are developing. The keyword here is \emph{existing}: your descriptive writing should not change the sense of what someone else has written. 
This does not mean that you shouldn't interpret or reinterpret what someone else has written if your additions add value to what someone else has written.

A certain amount of description is therefore necessary in any academic writing. But to present new knowledge, your examiner will be looking for something more than description -- they will be looking for a critical approach, by which you will build\footnote{You will be able to add in your generated \glsf{evidence} too, of course.} new arguments using what has gone before by, for instance, analysing, synthesising, and evaluating.

A perfect critique of an article will have the following components:

\begin{itemize}
\item your introduction to the paper, which discusses what is involved, where it takes place, or under which circumstances. %\textbf{Description answers the questions: what? who? where? when?}

%TODO: change spacing for this and subsequent paragraphs
\begin{example}{The perfect critique -- introduction} Kirlappos and Sasse (2012) discuss the implications of users visiting fake websites, concluding that trying to get them to stop doing so could prove difficult without appropriate user awareness training. [\ldots]
\end{example}


\item your analysis, which gives your perspective on the paper, perhaps highlighting how the paper comments on your focus in reading, including its strengths and weaknesses from the perspective of the topic. %\textbf{Analysis answers the questions: how? why? what if?}

\begin{example}{[\ldots] analysis} This training should be automated, involving ‘fake’ phishing e-mails being sent to users in order for them to compare the fake e-mail to a legitimate e-mail and understand the differences. [\ldots]
\end{example}


\item your \glsf{synthesis}, which explains how the parts fit into your research context, giving reasons, making comparisons, and highlighting relationships with other papers. %\textbf{\Glsf{synthesis} answers the questions: where else? which relationships?}

\begin{example}{[\ldots] \glsf{synthesis}} This can be contrasted with San Nicolas-Rocca\& Olfman (2013, p.84) who state that users should be trained on understanding laws and regulations as well as organisational policies and guidelines which define their specific responsibilities. [\ldots]
\end{example}


\item your evaluation of the strengths and weaknesses of the paper from your perspective, the implications that can be made for your purposes, and the impact and value to your research. %\textbf{Evaluation answers the questions: so what? what next?}

\begin{example}{[\ldots] evaluation} This \glsf{false-dichotomy} is in danger of losing something, however, in that training can be both instructive on the nature of \enquote{fake} emails and on the formal understanding of users' responsibilities in regards to cyber breaches.	
\end{example}
\end{itemize}

The boundary between these critique elements can sometimes be fuzzy, of course, in the sense that in-depth descriptions may well start to be analytical and synthetic, and some analysis and \glsf{synthesis} may include a level of evaluation.~\Cref{tab:kindOfAcademicWriting} provides some practical examples which may help you distinguish between description, analysis, \glsf{synthesis} and evaluation in academic writing.

\begin{SimpleNColTable}{tab:kindOfAcademicWriting}{2}[\narrowtablewidth]{Types of academic writing (adapted from \textcite{cottrell2023critical})}[rR[4]]
\\Descriptive & {\begin{itemize}
\item describes what happens or what something is like or how something works
\item tells a story or the order in which things occur
\item indicates how to do something
\item states what a \glsf{theory} says 
\item lists things, alternatives and options,~\etc{}.
\item describes a \glsf{system} or its components
\end{itemize}
} \\
Analytic & {\begin{itemize}
\item explains why things happen
\item explain why things work the way they do 
\item shows relations between pieces of information, parts of a \glsf{system}
\item demonstrates how a \glsf{theory} works
\end{itemize}
}\\
Synthetic & {\begin{itemize}
\item makes comparisons
\item weighs pieces of information against one another 
\item highlights strengths and weaknesses 
\item gives reasons for choices 
\item structures information based on established criteria 
\end{itemize}
}\\
Evaluative & {
\begin{itemize}
\item identifies \glsf{significance}
\item demonstrates relevance 
\item draws conclusions
\item considers wider implications	
\end{itemize}
}\\
\end{SimpleNColTable}

Let's look at how these different types are used in an example of academic writing.

\begin{example}{Revisiting the academic writing in our thematic summary}
In~\Cref{stage1}, you encountered following summary around the theme of modelling learning trajectories with Curriculum Analytics. 
\begin{quotation}
	Several authors have considered modelling and analysing students' learning trajectory through a programme of study, in order to understand how students progress or otherwise through their study and the learning outcomes they achieve in doing so. Such an understanding can then be used to inform scholarship and \glsf{reflection} around curriculum and its design, and inform possible changes. 
	
For instance, Dawson and Hubball (2014) apply social network analysis techniques to identify and visualise the most common learning pathways followed by students within complex curriculum structures, in which many links may exist between the various curriculum components. They suggest that their proposed tool could  be used by curriculum practitioners to study student progression and completion across different pathways, and the extent students acquired the expected learning outcomes, although their study does not evaluate the extent that might be the case. 

Similarly, Salazar-Fernandez et al.~(2021) use process mining to extract students' educational trajectories from historic data: in this case, their aim is to understand which trajectories are more likely to result in late dropouts. In their proof-of-concept tool evaluation over a specific data set, they achieve some positive results with the tool providing a strong indication that students taking a study break before resitting a failed module are most likely to drop out. Further research is needed to apply and evaluate their tool in other settings. 

Somewhat distinct from these studies is the work of McEneaney and Morsink (2022), who propose a \glsf{simulation} tool, based on Coloured Petri Nets, to be used as a design tool to help curriculum practitioners to test the possible effect on learning of envisaged curriculum changes, such as including or removing modules or study pathways in an existing programme. As another proof-of-concept tool, the work still requires wider application and evaluation. 

Finally, both Greer et al (2016) and Molinaro et al.~(2016) focus on potentially useful visualisations for curriculum practitioners. In particular, Greer et al (2016) introduces the \enquote{Ribbon tool}, based on Sankey diagrams, for visualising student flows through academic programmes, with interactive capabilities which allow practitioners to study and compare specific student demographics.  The same tool is recommended by  Molinaro et al.~(2016), alongside other visualisation tools, based on both students and course data, to allow practitioners explore both curriculum features and students' attainment. Both articles are part of the proceedings of the very first Curriculum Analytics workshop, in 2016, which also explains why they contain primarily poof-of-concept work and suggestions for future research.

Overall, this collection of articles contains some interesting ideas as to how curriculum and student data could be combined and analysed through the application and development of bespoke Curriculum Analytics tools. They all appear, however, to be quite preliminary studies, which is also an indication that this remains a young field of study where much research is still needed.
\end{quotation}
\end{example}

\begin{question}[subtitle={Activity: Distinguishing between different types of academic writing}] Consider the example above. Write down instances of each kind of academic writing you can find with reference to~\Cref{tab:kindOfAcademicWriting}.
\begin{solution}
You may have found some of the following:
%%Changes for JW352
\begin{itemize}
	\item descriptive: 
 	%
	\begin{displayquote}Dawson and Hubball (2014) apply social network analysis techniques to identify and visualise the most common learning pathways followed by students within complex curriculum structures, in which many links may exist between the various curriculum components. 	
	 \end{displayquote}
	%
	i.e., a summary of what those authors did.
	\item analytic:  \begin{displayquote}
  Several authors have considered modelling and analysing students' learning trajectory through a programme of study, in order to understand how students progress or otherwise through their study and the learning outcomes they achieve in doing so. Such an understanding can then be used to inform scholarship and \glsf{reflection} around curriculum and its design, and inform possible changes.	
  \end{displayquote}
  which highlights the connections between modelling, understanding and scholarship across the different articles. 
	\item synthetic: 	
  \begin{displayquote}
  	both Greer et al (2016) and Molinaro et al.~(2016) focus on potentially useful visualisations for curriculum practitioners. [...] Both articles are part of the proceedings of the very first Curriculum Analytics workshop, in 2016, which also explains why they contain primarily poof-of-concept work and suggestions for future research.		
  \end{displayquote}
  %
  which brings together work which shows strong similarities. Finally,
	\item evaluative: 
	\begin{displayquote}
 	Overall, this collection of articles contains some interesting ideas as to how curriculum and student data could be combined and analysed through the application and development of bespoke Curriculum Analytics tools. They all appear, however, to be quite preliminary studies, which is also an indication that this remains a young field of study where much research is still needed. 			
	 \end{displayquote}
 	%
	which expresses a judgement on the maturity of the work summarised and, more widely, the field of study.
\end{itemize}
\end{solution}
\end{question}
%%Hack to correct tcbox behaviour
\color{black}

You should do a similar analysis of something you have written.

\begin{question}[subtitle={Activity: Types of academic writing in your thematic summaries}] Consider a couple of thematic summaries you have written. Which kinds of academic writing have you used?
\begin{guidance}
%Hack to correct tcbox behaviour
\color{black}
 
If you find primarily descriptive writing, you should consider having another go at your summaries to improve the balance between description, analysis, \glsf{synthesis} and evaluation. 
\end{guidance}\end{question}
%%Hack to correct tcbox behaviour
\color{black}

By developing and applying these key skills, your completed literature review will be a self-contained piece of academic writing, which shows your \glsf{critical-thinking} and mastery of academic writing skills, and through which you:

\begin{itemize}
\item demonstrate your understanding of key ideas and their \glsf{significance} to your research, particularly to framing and justifying your \glsf{research-problem}

\item relate different ideas to each other, including arguments and counter-arguments

\item \glsf{reason} through the \glsf{evidence} to argue the possible contribution to knowledge your research can make.
\end{itemize}

In this way, your completed literature review demonstrates your ability to synthesise from the \glsf{academic-literature}.


\section{Core practices for academic writing}\label{sect:CorePracticeFor}

In your academic writing you should apply a number of well-established core practices, discussed in this section.

\paragraph{Use critical more often than descriptive language}

Your writing should be based on well-formed arguments, i.e., on claims that are supported by \glsf{evidence}\footnote{Either that of others or that you have discovered yourself. You will take a closer look at academic arguments in~\Cref{sect:DevelopYourArguments}.}, on comparing and evaluating alternative arguments, and on forming judgements on the basis of that \glsf{evidence}. Your writing should therefore favour analysis, \glsf{synthesis} and evaluation over descriptions.

In your descriptions you should also focus on essential details and keep general background information to a minimum.

\paragraph{Be clear and precise}

Your writing must make it easy for the reader to follow your arguments and grasp the points you are trying to make. You should:

\begin{itemize}
\item avoid long, over-complicated, poorly punctuated sentences

\item be accurate in what you report

\item be precise, and avoid ambiguity and vagueness

\item clearly define terms and concepts that may be open to more than one interpretation to avoid misunderstanding

\item keep your audience in mind, by avoiding jargon and, when using technical terms\footnote{If there are a lot of technical terms, you might consider including a glossary.}, explaining any that your readers are unlikely to be familiar with.
\end{itemize}

\paragraph{Order your topics appropriately}

You need to consider carefully which information your readers need to read first to make best sense of the topics you will introduce\footnote{You will learn ways to structure your narrative in~\Cref{sect:DevelopingYourLiterature}.}. Start with less complex\footnote{You will have experienced which are the simpler and which are the more complex topics in your reading so you will already know in which order to place them.} topics and build on them towards more complex ones. Later topics may depend on earlier ones, and this can give a hint to their presentational order.

%%2023-09-28: jgh This will come later
%Often, it's good to present the points that support your own argument first so that you establish your case early in the mind of the reader.
%
%Getting the order right will take time and perhaps many iterations. As you write, you should take time to step back and think about whether your arguments follow logically from each other, or whether moving them around may make your line of reasoning clearer to the reader.

You can make your reader's life easier if similar topics are grouped together in your writing – otherwise, they may get the impression that you are repeating yourself, or they may miss important connections between them.

%, including any analysis and evaluation that relates to them, as well as comparisons and counter arguments.

\paragraph{Signpost your narrative}

Another way of making your reader's life easier is to \textit{signpost} your writing, i.e., provide explicit clues to avoid the reader getting lost while reading your work. This applies both to the overall structure of your document and to individual arguments within each of its sections.

Common signposting practices at chapter, section or document level are:

\begin{itemize}
\item careful choice of headings and sub-headings

\item \enquote{setting the scene}\footnote{Journalists will often structure their narrative in the following way. They will:
\begin{itemize}
\item tell you what they're going to tell you
\item tell you
\item tell you what they've told you.	
\end{itemize}
%
This is signposting at is best.} at the start of a chapter or major section to provide a roadmap of what comes next

\item summarising key points at the end of a section/chapter.

\end{itemize}

%todos ok to here

Another form of signposting is at the level of an individual argument, where appropriate words or phrases (see~\Cref{tab:signalWords}) are used to help the reader understand where they are in the argument.

\begin{SimpleNColTable}{tab:signalWords}{2}[\narrowtablewidth]{\enquote{Signal words} used in academic writing, inspired by \textcite{cottrell2023critical}}[R[2]R[4]]
Function & Signal words \\
Introducing an argument, a description, a section or a chapter & first, firstly, first of all, to begin with, initially \\
Reinforcing similarities/arguments & similarly, equally, in the same way, also, for example \\
Adding further \glsf{evidence}/arguments & furthermore, moreover, in addition \\
Introducing alternative \glsf{evidence}/arguments & alternatively, however, on the other hand, differently \\
Highlighting choices & either/or, neither/nor \\
Contrasting ideas/arguments & instead, by contrast, conversely, on the one hand [...] on the other \\
Drawing conclusions & therefore, as a result, as a consequence, in conclusion, consequently, because of this \\
\end{SimpleNColTable}


\begin{question}[subtitle={Activity: Signposting in our thematic summary}] Consider, once again, the example summary on modelling students' learning trajectories. Write down any form of signposting you can identify.
\begin{solution}
You may have noted:
\begin{itemize}
	\item there is an opening statement to set the scene 
	\item \enquote{For instance} is used to introduce a specific example which illustrates a point previously made
	\item \enquote{Similarly} is used to introduce work with a similar aim as the previous work mentioned
	\item \enquote{Finally} introduces the remaining work reviewed
	\item \enquote{Overall} opens the concluding statement, which offers some summary observations
\end{itemize}
\end{solution}
\end{question}
%%Hack to correct tcbox behaviour
\color{black}

You can do the same for your own summaries.

\begin{question}[subtitle={Activity: Signposting in your summaries}] Consider a couple of thematic summaries you have written. Which forms of signposting have you used?
\begin{guidance}
%Hack to correct tcbox behaviour
\color{black}
 
Consider whether your signposting could be improved by applying the guidance in this section.  
\end{guidance}\end{question}
%%Hack to correct tcbox behaviour
\color{black}

\paragraph{Write good, grammatical text}

This is an essential characteristic of all written work you are expected to produce. You should therefore proofread carefully all your writing before submission to remove as many grammatical errors and typos as possible.

With modern digital tools, producing good grammatical text isn't difficult and your examiner will appreciate good grammatical writing. In fact, given the availability of good tools\footnote{Misquoting Star Wars: \enquote{Use the tools, Luke, use the tools.}}, they will, most likely, expect you to write perfect natural language prose. 

%todos ok to here



\paragraph{Use an appropriate format}

You should be sure to understand the required final format for your dissertation. For instance, all your citations and references should comply with the bibliographical style required. Your dissertation pages should be numbered, and all your figures and tables should also be numbered and accompanied by appropriate captions. 

\begin{question}[subtitle={Dissertation format requirements}]
	Check the requirements your course of study places on the final form of your dissertation then answer these questions\footnote{It may be that you can't answer these questions because they are not part of the requirements. In this case, we've suggested an answer for you.}:
	%
\begin{enumerate}%[start=0,label={(\bfseries R\arabic*):}]
\item how many words should your dissertation be?
\item what font size should you use?
\item what should the line spacing and margins be?
\item how should you number your dissertation pages?
\item how should you number figures and tables?
\item are there any stylistic requirements on headings and sub-headings?
\item are you allowed to include additional material in appendices?
\item how should references and citations be formatted?
\item is there specific information that you should include in your dissertation title page?
\end{enumerate}

\begin{solution}
The Open University has no single format for research dissertation. However, one specific \glsf{capstone-project} dissertation guidelines include:
%
\begin{enumerate}%[start=0,label={(\bfseries R\arabic*):}]
\item the dissertation should be between 10,000 and 15,000 words, excluding references and appendices
\item the font used should be 11 or 12-point Times New Roman, with 11 points recommended
\item 1.5 line spacing should be used and margins should be approximately 2cm 
\item pages should be numbered, including references and appendices. Lower-case Roman numerals – iii, iv, v,~\etc{}, should be used on the preliminary pages, and Arabic numerals starting from page 1 should be used from the beginning of Chapter 1 
\item all figures and tables should be numbered, with sequential numbering  within each chapter, for instance, the figures in Chapter 2 would be Figures 2.1, 2.2, 2.3,~\etc{}. Also, they should have descriptive captions: below them for figures and above them for tables
\item headings and sub-headings should be numbered with sequential numbering within each chapter, for instance, in Chapter 3, you could have sections 3.1, 3.2,~\etc{}. each with sub-sections, say, 3.1.1, 3.1.2,~\etc{}. You should avoid sub-subsections
\item appendices are allowed and should be used to provide extra materials in support of the dissertation body. Appendices are not assessed and should be used sparingly
\item references and citations should use the Harvard bibliographical style
\item the title page should include:
%
\begin{itemize}
\item the full title of the dissertation
\item your full name
\item your university identifier
\item the \glsf{qualification} for which it is submitted
\item the date (consisting of month and year) of your submission
\item the total word count.
\end{itemize}
In addition, there should be a short statement declaring that no part of the dissertation has been submitted for any other degree.
\end{enumerate}
\end{solution}
\end{question}
%%Hack to correct tcbox behaviour
\color{black}

\paragraph{Avoid \glsf{plagiarism}!}

In describing your research, you will need to distinguish it from the work of others.

Part of this is to identify clearly the new contribution to knowledge your work makes: you can't make a new contribution to knowledge in a context in which that knowledge already exists.

Another part of this relates to behaving ethically in acknowledging all sources you have used. This is known as attribution and its role is to give credit where credit is due, avoiding any possible accusation of \glsf{plagiarism}, that is passing off the work of others as if it were your own. 

While intentional \glsf{plagiarism} is a deliberate attempt to deceive, something that both universities and publishers take very seriously and may have severe consequences\footnote{Your university may prevent you from graduating. A publisher may refuse to publish your work or withdraw previously published articles.}, \glsf{plagiarism} doesn't need to be intentional to be classed as such. \Glsf{accidental-plagiarism} is usually the result of poor organisation or sloppy review practices, and can be avoided by being careful and systematic when reading articles, and writing and organising your notes and summaries, making sure you keep tabs on where ideas/arguments/logic/processes/\etc{}. come from, and what your knowledge contribution to those is. 

\begin{tip}
Such practices are much easier to implement at the time of creation than at the end of the writing up process. Imagine not knowing where a quote on page 35 came from when you're writing up and the difficulty you could have in trying to find it retrospectively.

Given the importance academics place on \glsf{plagiarism}, the risk of not being proactive in avoiding it is very high. Universities take \glsf{plagiarism} very seriously. You should too.
\end{tip}

You should also avoid \glsf{self-plagiarism}, which is where you re-use existing material that you yourself have published, without clear attribution\footnote{Clearly, this is not relevant if you haven't published before.}.

In a dissertation, the correct form of attribution is through referencing, and there are two common ways to report cited work:

\begin{itemize}
	\item paraphrasing, that is rephrasing in your own words, what an article says, still ensuring that the article is appropriately cited
	\item using quotation marks to identify text reproduced without change from a cited article. In this case you can use the words of the article \textit{verbatim}, as long as they are enclosed within those quotation marks.
\end{itemize}

Both can be used, but an excessive use of quotations may interrupt the flow of your narrative making it harder for the reader. As a rule of thumb, you should use quotations if the wording used by the authors of the article really matters, perhaps because they introduce a significant new term or concept or use particularly suggestive language. Otherwise, yoiu should paraphrase, which may help you convey the essence of what you have read more concisely or even more clearly. 

If you follow these simple practices consistently, you will avoid most \glsf{accidental-plagiarism} and your work will more likely comply with your course requirements and expectations. 

\begin{question}[subtitle={Ways to report cited content}]
Consider the following statements:

\blockquote{A \glsf{summary-comparison-matrix}, introduced by \textcite{sastry2013summary}, can be used as an efficient tool to keep track and compare the content of articles reviewed. Similar information is recorded for each paper, such as the \glsf{research-aim}, research methods or key findings, organised in a matrix form which facilitates their comparison.}

and 

\blockquote{\textcite{sastry2013summary} define a \glsf{summary-comparison-matrix} as \enquote{a data organizer that helps students to extract relevant information from research papers, categorize those extractions and visualize how these disparate extractions are related to each other.}}

Write down how they differ in the way they report the cited work.
\begin{solution}
The former uses paraphrasing; the latter includes a verbatim quotation from the article.	
\end{solution}
\end{question}
%%Hack to correct tcbox behaviour
\color{black}


%todos ok to here
%We distinguish between:
%%
%\begin{itemize}
%\item a citation, which is a short-cut that appears in the main body of the text to refer to a specific source
%
%\item a reference, which is the full bibliographic information of a source you cite in your text. References are usually collected in a section at the end of a dissertation, article, report,~\etc{},
%
%\end{itemize}
%

\begin{question}[subtitle={Activity: Looking up your University's \glsf{plagiarism} policy}] Look up your university policy on \glsf{plagiarism} and any disciplinary processes related to it.

\begin{solution}
The Open University has a strict policy on \glsf{plagiarism}. 

\begin{itemize}
	\item intentional \glsf{plagiarism} can lead to severe disciplinary actions, from failing study modules to being expelled from a course
	\item \glsf{accidental-plagiarism} resulting from poor academic practices is addressed by providing extra study support, particularly early on in a programme of studies. However, repeated offences will incur in disciplinary action, as per intentional \glsf{plagiarism}.
\end{itemize}
\end{solution}
\end{question}
%%Hack to correct tcbox behaviour
\color{black}

\section{Develop your arguments!}\label{sect:DevelopYourArguments}
The development and presentation of academic arguments forms the core of academic writing. But what exactly is an academic argument? To answer this question, this book used the \glsf{model} proposed by Booth, Colomb and Williams in \textcite{booth1995making} -- referred to as the \glsf{bcw-model} (BCW for short). 

\subsection{The BCW model}\label{ssect:BCWModel}

In the BCW model, an academic argument has at least two parts – the \textit{\glsf{claim}} and the \textit{\glsf{evidence}} – and might, in more complex arguments, have up to three other parts – called the \textit{\glsf{reason}}, the \textit{\glsf{warrant}} and any \textit{qualifications}. 

The \glsf{model} is shown in~\Cref{fig:arguments}, with its parts:
%
%
\begin{enumerate}%[label={(\bfseries \arabic*):}]
\item the \textit{\glsf{claim}}, which is a point of view and needs support with... 

\item ... \textit{\glsf{evidence}}, which provides the grounds on which the \glsf{claim} is made, and ...

\item ...the \textit{\glsf{reason}}, which is why you believe the \glsf{claim} to be true. In addition ...

\item the \textit{\glsf{warrant}} explains how the \textit{\glsf{reason}} is relevant to the \textit{\glsf{claim}}, while ...

\item the \textit{qualifications} are concessions which may limit what is being claimed, for instance by acknowledging objections, alternatives,~\etc{}.
\end{enumerate}
%

In dealing with qualifications, you might need to make further arguments, in which case the process of building an argument may become recursive and so much more complex to develop and present.

\begin{figure}[htbp]
\centering
\includegraphics[width=0.8\textwidth]{Figures/argumentParts.jpeg}
\caption{The five elements of an academic argument, adapted from \textcite{booth1995making}.\label{fig:arguments}
}
%Description
%
%The figure summarises the five elements of an academic argument and their relationship.
%
%End of description
\end{figure}

%todos ok to here

%In simple arguments, the \glsf{warrant} – that the \glsf{reason} is relevant to the \glsf{claim} – does not stating if it is obvious. Moreover, for such arguments, the \glsf{qualification} might also be missing.

A fully developed BCW argument is a beautiful thing. Here's an example\footnote{Adapted from \textcite{booth1995making}.}.


\begin{example}{The need for a TV \enquote{watershed}}%\todo[inline]{Needs revisiting}}
%
\begin{description}[style=nextline]
%\DrawEnumitemLabel
\item [\Glsf{claim}] Showing violence on TV should  be allowed only after the 9pm watershed

\item [\Glsf{reason}] TV violence can have harmful psychological effects on children 

\item [\Glsf{evidence}] Research by \textcite{smith1997understanding} found that children ages 5–7 who watched more than three hours of violent television each day were 25 percent points more likely to say that what they saw on television was \enquote{really happening}

\item [\Glsf{warrant}] If children are protected from watching violence on TV they will be less likely to see violence as a normal part of day-to-day life

\item [Qualifications] The following qualifications could be considered:%
\begin{itemize}
\item that a child interprets something on TV as \enquote{really happening} does not necessarily mean that they will try to emulate it
\item not all children are impressionable. 
\item violence \textit{is} a normal part of day to day life.
\end{itemize}
\end{description}
\end{example}

%todos ok to here

Qualifications can improve a \glsf{claim} – reducing its scope, for instance, making it more acceptable – but may also require further arguments to be made. For instance, a rebuttal to \enquote{violence is a normal part of day to day life} might be the counterclaim that \enquote{to reduce violence in day to day life, society should insulate children from it so that the next generation isn't so likely to see violence as a justified response to problems}, and so on to other arguments. These might be structured in the same way until all qualifications are discharged to your satisfaction.

\begin{question}[subtitle={Activity: Differentiating between \glsf{reason} and \glsf{evidence}}] Consider the example above and write down what you think the difference between \glsf{reason} and \glsf{evidence} is.

\begin{solution}According to \textcite{booth1995making}, reasons are things you think up in your mind, while \glsf{evidence} is somewhat \enquote{out there} for everybody to see and examine. While in everyday casual conversation, a \glsf{claim} can often be supported with just a \glsf{reason}, that is not the case in \glsf{academic-research} where reasons must be backed up by \glsf{evidence} -- 
your research audience will not accept your reasons at face value.
\end{solution}
\end{question}
%%Hack to correct tcbox behaviour
\color{black}

\subsection{Arguments and narrative}\label{ssect:ArgumentsAndNarrative}

In your literature review, and your dissertation as a whole, your arguments won't be presented in the BCW form above, of course: the \glsf{model} is only a useful device to help you distinguish between the essential elements of an academic argument. Your dissertation presentation will be in the form of a continuous narrative instead. However, it is essential that there is a strong connection between your narrative and your academic arguments. In this section, you will consider ways the BCW \glsf{model} can help you write narrative which allows your arguments to shine.

Let's go back to our running example.    

\begin{example}{Finding the academic argument in our thematic summary}
Consider once again the example summary around the theme of modelling learning trajectories with Curriculum Analytics. Let's use the BCW \glsf{model} to highlight the argument within the narrative. It goes like this:

\begin{description}[style=nextline]
%\DrawEnumitemLabel
\item [\Glsf{claim}] Modelling and analysing students' learning trajectories can inform scholarship and \glsf{reflection} around curriculum, its design, and possible improvements  

\item [\Glsf{reason}] Such modelling and analysis helps us understand how students progress or otherwise through their studies and the learning outcomes they achieve in doing so 

\item [\Glsf{evidence}] The works cited demonstrate how this can be done

\item [\Glsf{warrant}] By understanding which curriculum characteristics prevent students from succeeding on a programme of studies, then the curriculum could be re-designed to address the problem. 

\item [Qualifications]
The following should be considered:
\begin{itemize}
	\item all cited work is based on limited case studies, so that there is no \glsf{evidence} of wider applicability or generality
	\item some cited work is little more than a proof-of-concept aimed at suggesting new avenues for research rather than proposing a mature approach. 
\end{itemize}
\end{description}
\end{example}

In this example, the BCW \glsf{model} was used to confirm that all necessary elements of the argument were in place. If not, the BCW \glsf{model} would have still been useful to identify how to develop the argument further. 

It's now time for you to have a go. 

\begin{question}[subtitle={Activity: Constructing your academic arguments}] Consider your current thematic summaries. Extract and write down their academic arguments according the BCW \glsf{model}. Identify those which are already well developed and those which require further work, indicating what is still needed.
\begin{guidance}
%Hack to correct tcbox behaviour
\color{black}
 Don't worry if you can't include all five elements for each argument. Instead, try to capture both your \glsf{claim} and any reasons and \glsf{evidence} in support for each case. If the latter are lacking, then think of ways you can support your claims better.
It is possible that your summaries contain more than one academic argument: if so, ensure you apply the BCW \glsf{model} to each of them.
Constructing academic arguments is a fundamental skill which becomes easier with practice. Therefore you should spend up to two hours on this activity.
\end{guidance}\end{question}
%%Hack to correct tcbox behaviour
\color{black}

 You have now seen how, given something you have written, you can use the BCW \glsf{model} to identify clearly the arguments within and their elements, and any gaps you may still need to address.
 
 Equally, given an argument based on the BCW \glsf{model}, you can develop a narrative that combines all components together in prose. Of course, it is a real skill to be able to come up with a narrative that is easily digestible\footnote{Meaning that your reader – cough, examiner, cough – can easily engage with it.}. If you find this conversion difficult, you could start with the following template\footnote{Adapted from~\Cref{fig:arguments}} for creating a narrative around a BCW \glsf{model} argument:

\newcommand{\BCW}[5]{I \glsf{claim} #1, because #2, based on #3. The \glsf{warrant} that allows me to connect my \glsf{reason} and \glsf{claim} is #4. I acknowledge the following qualifications to my \glsf{claim} (which I deal with later): #5}

\BCW{\textit{my \glsf{claim}}}{\textit{my reasons}}{\textit{this \glsf{evidence}}}{\textit{my \glsf{warrant}}}{\begin{itemize}\item\textit{\glsf{qualification} 1}\item \textit{\glsf{qualification} 2} \item ...\end{itemize}}

\begin{question}[subtitle={Activity: Beginning an evidence narrative}]
Put the example TV violence argument into the BCW narrative form based on the template.
\begin{solution}
You should have something like:

\blockquote{
	\BCW{showing violence on TV should  be allowed only after the 9pm}{TV violence can have harmful psychological effects on children}{\textcite{smith1997understanding} found that children ages 5–7 who watched more than three hours of violent television each day were 25 percent points more likely to say that what they saw on television was \enquote{really happening}}{that if children are protected from watching violence on TV they will be less likely to see violence as a normal part of day-to-day life}{\begin{itemize}\item  a child interprets something on TV as \enquote{really happening} does not necessarily mean that they will try to emulate it. 
\item not all children are impressionable 
\item violence \textit{is} a normal part of day to day life.
\end{itemize}}}
\end{solution}
\end{question}
%%Hack to correct tcbox behaviour
\color{black}

As you can see, the \glsf{model} gets close to a narrative, but there are some grammatical errors – \enquote{... based on \textcite{smith1997understanding} found that ...} – which need smoothing out, and it could get very repetitive if each and every argument you present has this form. 

There are many techniques for improving your narrative. One of the most powerful is to read what you have out loud to a live listener and have them try to follow your argument. 

\begin{question}[subtitle={Activity: A willing audience}]
Read your argument in its raw form – \emph{as written} – to a willing family member/friend/colleague and ask for their opinions on it. Reflect on both their reactions to your reading and the feedback they provide.
\begin{guidance}
%Hack to correct tcbox behaviour
\color{black}


%Hack to correct tcbox behaviour
\color{black}
	In your \glsf{reflection} you could consider the following questions. Was your audience engaged or did they fall asleep? Did their feedback help you understand how the narrative should improve? Will they come back and help with future versions? Did the template work for you? 
\end{guidance}
\end{question}
%%Hack to correct tcbox behaviour
\color{black}

It's unlikely that your first cut at creating a narrative using the template will result in something you can use directly in your dissertation. Instead, you should iterate to make it more compelling, using the feedback from your \enquote{critical friends} to help you to do so.

\subsection{Logical fallacy and cognitive-bias}{\Glsplf{logical-fallacy} and \glsf{cognitive-bias}}\label{ssect:LogicalFallaciesCognitive}

The BCW \glsf{model} is a useful device to help you structure your academic arguments. In making you think deeply and critically about what you are trying to \glsf{claim} and why, you have a better chance to write good arguments. There are, however, things you still need to guard against. 

Firstly, there may be \glsplf{logical-fallacy} in your arguments: these are errors of reasoning that can undermine your \glsf{claim}. The most common ones are\footnote{Many more exist -- you may like to do a web search to find out more.}:

\begin{itemize}
	\item \Glsf{circular-reasoning}\footnote{AKA begging the question or begging the \glsf{claim}.}, in which the supporting \glsf{evidence} is just a restating of the \glsf{claim}. As a result the \glsf{claim} is not supported by \glsf{evidence} at all
	\item \Glsf{hasty-generalisation}, in which a \glsf{claim} is made based on insufficient \glsf{evidence}. As a result, new \glsf{evidence} may be easily found which contradicts it 
	\item \Glsf{sweeping-generalisation}, in which a \glsf{claim} obtained from \glsf{evidence} within a specific situation or context is assumed to be true in other situations/contexts. Again, the \glsf{claim} could be easily dismissed by considering counter-arguments from other contexts
	\item \Glsf{post-hoc-ergo-propter-hoc} ––– \enquote{after, therefore because of}%%JW375
	 ––– which claims a causal relation between two \glsplf{phenomenon} because one happens after the other. Only because two things happen one after the other does not necessarily mean that the first causes the second: it may be random chance, or there may be other factors which cause or influence both.
	\item \Glsf{false-dichotomy}, in which an \emph{either--or} \glsf{claim} is made which assumes that there are two mutually exclusive \glsplf{phenomenon} where that's not the case --- the \glsplf{phenomenon} may not be mutually exclusive at all. 
\end{itemize}

\begin{question}[subtitle={Activity: \Glsplf{logical-fallacy}}]
Think of each definition above and write down a possible example.
\begin{guidance}
%Hack to correct tcbox behaviour
\color{black}


%Hack to correct tcbox behaviour
\color{black}
If you struggle to think of appropriate examples, do a web search to get some inspiration. However, you should still try and come up with your own examples too.
\end{guidance}
\begin{solution}
This is what I have come up with:
\begin{itemize}
	\item \Glsf{circular-reasoning}: collecting quality data is difficult (\glsf{claim}) because quality data are difficult to collect (\glsf{evidence}, restating the \glsf{claim})
	\item \Glsf{hasty-generalisation}: camomile tea cures insomnia (\glsf{claim}) because when I drink camomile tea in the evening I sleep well (insufficient \glsf{evidence})
	\item \Glsf{sweeping-generalisation}:  universities don't generate quality student data (\glsf{claim}) because I have observed this is the case in my university (\glsf{evidence} from a specific context)
	\item \Glsf{post-hoc-ergo-propter-hoc}: Italy used to loose in the rugby Six Nations tournament till I started to watch them play. Therefore, my watching is causing them to win (assumed false causal relation)
	\item \Glsf{false-dichotomy}: during a pandemic the government can either save lives or preserve the economy (claimed \glsf{false-dichotomy}), because to save lives the government must shut all economic activities down: this argument excludes the possibility that those actions could be combined.
\end{itemize}
\end{solution}
\end{question}
%%Hack to correct tcbox behaviour
\color{black}

Secondly, everybody is susceptible to \glsf{cognitive-bias}, which prevents people from processing information objectively due to limited capabilities of our mind, or due to emotional responses or social norms and conditioning. You have already encountered \glsf{cognitive-bias} in~\Cref{sect:biasInResearch}, alongside examples of various forms of \glsf{bias}. Among them, \glsf{confirmation-bias} is particularly relevant to the construction of academic arguments.

\begin{question}[subtitle={Activity: Revisiting \glsf{confirmation-bias}}]
Go back to~\Cref{sect:biasInResearch} and revisit the definition of \glsf{confirmation-bias}. Write down ways in which it may affect you when constructing your academic arguments.
\begin{solution}
\Glsf{confirmation-bias} may result in cherry-picking information which agrees most with your beliefs, opinions or preconceptions, and ignoring that which may support opposite views. This can lead you, as a researcher, to conduct selective searches, actively looking for articles whose position is in alignment with yours, or recall and interpret \glsf{evidence} in a way that reinforces your position, rather than weighing and contrasting all available \glsf{evidence}.
\end{solution}
\end{question}
%%Hack to correct tcbox behaviour
\color{black}


%\subsection{Organising your arguments}\label{ssect:OrganisingYourArguments}
%
%\textbf{LR-Note: I've reused some content above}
%
%Now that you know how a single argument can be structured, and what the main arguments from your summaries are, you should ensure they are presented in a logical manner.
%
%Here are few simple rules you could apply:
%\begin{itemize}
%	\item no earlier argument will depend on a later argument -- this avoid c
%	\item no circular argument
%	\item 
%\end{itemize}
%
%
%
%At a minumb, you should ensure that 
%
%In doing so, you should consider the following rules, 
%
%
%
%There are sequences, there are good sequences; and there are perfect sequences. Finding a perfect sequence is harder than finding a good sequence which is harder than finding a sequence. 
%
%The perfect sequence will meet the following rule:
%%todos ok to here
%
%%
%\paragraph{Rule 1 of argument construction:} \textit{no earlier argument will depend on a later argument}.
%
%That's it. With this rule, you're building an argument tree\footnote{Technically, a directed acyclic graph, but the difference shouldn't worry you.} making your reader's job easier: when they come to try to understand a later argument they will have everything they need to have to do so.
%
%Anything argument structure that breaks this rule – a tree with cycles, for instance, in which an earlier argument depends on a later one – makes your readers' job harder.\footnote{Not impossible, though.}. 
%
%\begin{example}{A circular argument}
%Let's look at a simple circular argument:
%
%%\paragraph{Argument:} 
%%\blockquote{Small businesses are the only way the economy can truly work well. The stimulation and employment they provide is invaluable. And, it is this stimulation and employment which allows small businesses to truly thrive.}
%%%
%%\begin{description}%[label={Premise \arabic*:}]
%%\item [\Glsf{claim} 1] Small businesses are required for an economy to function.
%%%\item [\Glsf{reason}] Small businesses employ people and strengthen the economy.
%%\item [\Glsf{claim} 2] A working economy is required for a small business to function.
%%\end{description}
%%
%%The above argues the importance of small businesses to an economy whilst claiming that the economy requires small business and entrepreneurs; and also that small business and entrepreneurs require a working economy. In the BCW \glsf{model}, we could restate this as:
%%
%
%\paragraph{Argument 1}
%\begin{description}%[label={Premise \arabic*:}]
%\item [\Glsf{claim} 1] Small businesses\footnote{Adapted from \url{https://helpfulprofessor.com/circular-reasoning-fallacy-examples/}} are required for an economy to function.
%\item [\Glsf{reason} 1]Small businesses employ people who pay taxes which strengthens the economy.
%\end{description}
%
%\paragraph{Argument 2}
%\begin{description}
%\item [\Glsf{claim} 2] A strong economy is required for a small business to function.
%\item [\Glsf{reason} 2]A strong economy is more able to provide the investment that small business need to grow.
%\end{description}
%\end{example}
%
%These arguments together give that small businesses are both the cause of a working economy and the consequence of one; they depend on each other. Inverting their order does change this, and so it looks like Rule 1 is broken no matter what we do.
%
%Thing is, there is an actual circular relationship between small business and economies – that they support each other is often the \glsf{reason} why an economy grows. The problem with this circular argument is that it tries to separate a single argument into two. Unresolvable circular arguments are often the result of trying to do this and so point to some additional thinking that is needed.
%
%\paragraph{\Glsplf{logical-fallacy}} The structure of your argument is our focus at this point, but the construction of individual arguments is more complex, due to the existence of \textit{\glsplf{fallacy}}.
%
%According to Wikipedia:
%\begin{displayquote}
%A \glsf{fallacy} is reasoning that is logically invalid, or that undermines the logical \glsf{validity} of an argument. All forms of human communication can contain \glsplf{fallacy}. 
%\end{displayquote}
%
%%todos ok to here
%
%\begin{stolen}{\url{https://en.wikipedia.org/wiki/List_of_\glsplf{fallacy}}}
%
%\Glsplf{fallacy} are challenging to classify. They can be classified by their structure (formal \glsplf{fallacy}) or content (informal \glsplf{fallacy}). Informal \glsplf{fallacy}, the larger group, may then be subdivided into categories such as improper presumption, faulty \glsf{generalisation}, and error in assigning causation and relevance, among others.
%
%The use of \glsplf{fallacy} is common when the speaker's goal of achieving common agreement is more important to them than utilising sound reasoning. When \glsplf{fallacy} are used, the premise should be recognised as not well-grounded, the conclusion as unproven (but not necessarily false), and the argument as unsound.
%\end{stolen}
%
%Unless you're trying to intentionally mislead your reader through the use of a \glsf{fallacy}\footnote{We recommend against this; your supervisor and examiner will be highly tuned to pick up fallacious arguments; in our experience, the finding and examination of a fallacious written argument gives academics great joy. They're easy to spot and point to poor \glsf{academic-research}.}, it's going to be difficult for you to identify any that do creep into your writing – knowing they're there means you wouldn't use them.
%
%Look out for comments from comments from other readers – your supervisor, for instance – who might point them out. Wikipedia's page on \glsplf{fallacy}\footnote{\url{https://en.wikipedia.org/wiki/List_of_\glsplf{fallacy}}} is a great, regularly updated, resource that will help you understand any indicated \glsf{fallacy}, and the individual entries that that page links to give more information that will help you debug your writing.
%
%In the worst case, you could always ask the person that pointed a \glsf{fallacy} out whether the argument can be restructured to make sense, how you should do that if it is possible.
%%
%%
%%\paragraph{Spiral Arguments}
%%Another type of argument that looks circular but isn't is a spiral argument. A spiral argument again has (at least) two parts, the \glsf{claim} of which appears to depend on another \glsf{claim} which depends on the first \glsf{claim}. Add example here – there's one at \url{https://www.catholic.com/tract/proving-inspiration}. While there can be a get out clause of a spiral argument – a \glsf{reason} it doesn't break Rule 1 – a spiral argument can be less than convincing, even wrong, if the circularity in it isn't absolutely removed.
%%
%
%\paragraph{\Glsplf{cognitive-bias}}
%\begin{stolen}{\url{https://www.logicallyfallacious.com/\#null}}
%In the early 1970s, two behavioral researchers, Daniel Kahneman and Amos Tversky pioneered the field of behavioural economics through their work with \glsplf{cognitive-bias} and heuristics, which like \glsplf{logical-fallacy}, deal with errors in reasoning. The main difference, however, is that \glsplf{logical-fallacy} require an argument whereas \glsplf{cognitive-bias} and heuristics (mental shortcuts) refer to our default pattern of thinking. Sometimes there is crossover. \Glsplf{logical-fallacy} can be the result of a \glsf{cognitive-bias}, but having \glsplf{bias} (which we all do) does not mean that we have to commit \glsplf{logical-fallacy}. Consider the bandwagon effect, a \glsf{cognitive-bias} that demonstrates the tendency to believe things because many other people believe them. This \glsf{cognitive-bias} can be found in the \glsf{logical-fallacy}, appeal to popularity.
%
%Everybody is doing X.
%
%Therefore, X must be the right thing to do.
%
%The \glsf{cognitive-bias} is the main \glsf{reason} we commit this \glsf{fallacy}. However, if we just started working at a soup kitchen because all of our friends were working there, this wouldn't be a \glsf{logical-fallacy}, although the bandwagon effect would be behind our behaviour. The appeal to popularity is a \glsf{fallacy} because it applies to an argument.
%
%I would say that more often than not, \glsplf{cognitive-bias} do not lead to \glsplf{logical-fallacy}. This is because \glsplf{cognitive-bias} are largely unconscious processes that bypass \glsf{reason}, and the mere exercise of consciously evaluating an argument often causes us to counteract the \glsf{bias}.	
%\end{stolen}
%
%There is more about \glsplf{cognitive-bias} at \url{https://www.logicallyfallacious.com/} 
%
%\paragraph{Argument types}
%
%\todo{There's more stuff at \url{https://www.psychologytoday.com/us/blog/hide-and-seek/201906/arguments-and-how-they-fail} if we want to go that route.}
%
%%todos ok to here
%

\section{Developing your literature review from your thematic summaries}\label{sect:DevelopingYourLiterature}

You should now have a collection of summaries based on the themes you have identified from the literature. You will have also revised those summaries to ensure that the academic arguments within are well-formed in relation to the BCW \glsf{model}. Congratulations! You now have some basic building blocks to put together your very first literature review draft.


\subsection{Developing the main body of your literature review}\label{sect:bodyOfLiteratureReview}

In piecing together your literature review it is important you consider the order in which different topics are introduced. You main goal in doing so is to ensure that the reader of your dissertation will find it easy to follow the \enquote{story} you are trying to tell --- one which provides sufficient context and justification for your research.

In doing so, you should think of your literature review as made of three distinct parts:
\begin{itemize}
	\item an introduction, which gives the reader, in outline, a sense of the detailed review you have conducted 
	\item a main body, which provides a detailed account of such review, and
	\item a summary conclusion, which provides the academic argument in support of your research, based on the \glsf{evidence} you have presented in the main body.
\end{itemize}

Your thematic summaries are your starting point to populate the main body of your literature review. In doing so, you should take the following steps\footnote{You should have both your thematic summaries and \glsf{concept-map} (see~\Cref{ssect:Synthesising}) at hand.}:

\begin{description}
\item[Step 1:] list all the themes you have identified in no particular order

\begin{example}{Listing the themes...} 
In the example review of the Curriculum Analytics (CA) literature the following themes emerged:

\begin{enumerate}
	\item stakeholders and CA tool development
	\item modelling study trajectories and progression
	\item benefits of CA tools
	\item deriving insights from data
	\item curriculum metrics and quantitative assessment
	\item success factors for CA adoption
	\item capturing the student voice
	\item cA definitions
\end{enumerate}
\end{example}

\item[Step 2:] consider whether there are themes whose concepts overlap, and could be merged. If so, merge their themes into a wider theme

\begin{example}{...Merging themes} 
As stakeholders were prominent both in the development and adoption of CA, those two themes could me merged; also  \enquote{capturing the student voice} could be seen as a particular instance of the \enquote{benefits of CA tools}, so those two themes could be merged as well. The resulting themes are (new combined themes in bold):
\begin{enumerate}
	\item \textbf{importance of stakeholders in CA development and adoption} (by merging \enquote{stakeholders and CA tool development} and \enquote{success factors for CA adoption})
	\item modelling study trajectories and progression
	\item \textbf{benefits of CA tools} (which now includes \enquote{capturing the student voice})
	\item deriving insights from data
	\item curriculum metrics and quantitative assessment
	\item cA definitions
\end{enumerate}
\end{example}
%%Hack to correct tcbox behaviour
\color{black}

\item[Step 3:] consider whether there are themes which rely on concepts introduced by other themes. If so, the latter should come first, and the former could follow closely. You should re-order your themes accordingly  

\begin{example}{...Identifying dependencies between themes} 
All themes assume an understanding of what CA are, so that CA definitions should come first. Also, \enquote{deriving insights from data} expands on some issues around stakeholders and CA adoption, so it should follow soon after that theme. This led to the following order:
\begin{enumerate}
	\item CA definitions
	\item importance of stakeholders in CA development and adoption
	\item deriving insights from data
	\item modelling study trajectories and progression
	\item benefits of CA tools
	\item curriculum metrics and quantitative assessment
\end{enumerate}
\end{example}
%%Hack to correct tcbox behaviour
\color{black}


\item[Step 4:]consider which themes may be broad and generic vs those which may be narrow and  specific. Reorder the themes so that your narrative goes from the generic to the specific, while maintaining any dependencies you have already identified. 

\begin{example}{...From generic to specific} 
Both CA benefits and metrics are quite generic themes, so that they could come early, leading to the new order:
\begin{enumerate}
	\item CA definitions
	\item benefits of CA tools
	\item curriculum metrics and quantitative assessment
	\item importance of stakeholders in CA development and adoption
	\item deriving insights from data
	\item modelling study trajectories and progression	
\end{enumerate}
Different orders are also possible, of course.
\end{example}
%%Hack to correct tcbox behaviour
\color{black}

\end{description}

By the end of this process, you should have a structure for your literature review body: each of the themes in your list can be a sub-section which you can populate with the narrative from your summaries. 

\begin{question}[subtitle={Activity: Structuring the body of your literature review}] Consider your current thematic summaries. Apply the above steps to arrive at a possible structure for the main body of your literature review. Once you are satisfied with the structure, populate it with your thematic summaries and read it through. Make sure there is a good flow from one section to the next. 
\begin{guidance}
%Hack to correct tcbox behaviour
\color{black}
 
You may have to re-apply the steps to improve both structure and narrative flow between sections. Being able to structure a narrative is yet another fundamental skill which becomes easier with practice and that will serve you well when writing up your whole dissertation. Therefore, you should treat this as a substantial activity which may take you several hours if not days.
\end{guidance}\end{question}
%%Hack to correct tcbox behaviour
\color{black}

\subsection{Choosing headings and sub-headings}\label{ssect:ChoosingHeadings}

You should now consider the headings of your review sections and sub-sections, to ensure they are appropriate. Your headings should: 

\begin{itemize}
\item indicate, in outline, to the reader how the narrative develops from one section/sub-section to the next
\item express clearly the purpose of each
\item accurately reflect the content of each
\item be concise.
\end{itemize}


\begin{question}[subtitle={Activity: Improving your current headings}] Consider your current draft of the main body of your literature review. Apply the guidelines above to improve its headings and sub-headings.
\begin{guidance}
%Hack to correct tcbox behaviour
\color{black}


%Hack to correct tcbox behaviour
\color{black}
You choice of headings and sub-headings should signpost to the reader where they are in your \enquote{story}, help them follow the thread of your arguments, and allow them to locate efficiently specific content they wish to return to.
\end{guidance}\end{question}
%%Hack to correct tcbox behaviour
\color{black}

\subsection{Writing your review introduction and critical summary}\label{ssect:WritingYourReview}
Now that the main body of your literature review is in place, at least in draft form, you can top-and-tail it with a brief introduction at the beginning and a critical summary at the end.

\begin{question}[subtitle={Activity: Writing the introduction}] Write a brief introduction to your literature review to introduce the topics you are going to cover in its main body, and their relevance to your \glsf{research-problem}.
\begin{guidance}
%Hack to correct tcbox behaviour
\color{black}
 
Your introduction should be brief and indicate in outline the main topics you are going to cover and why. This section should be straightforward to write now that the literature review main body is in place.
\end{guidance}\end{question}
%%Hack to correct tcbox behaviour
\color{black}

\begin{question}[subtitle={Activity: Writing the critical summary}] Write the concluding section of your literature review which summarises the main findings from your reading in support of your proposed research.
\begin{guidance}
%Hack to correct tcbox behaviour
\color{black}
 
Your critical summary should take the form of an academic argument which supports your \glsf{claim} that there is a knowledge gap your research is going to address. You should skim through~\Cref{fig:arguments} to remind you of the key elements of an academic argument. Writing this section may take some time and require you to iterate until your argument is well-formed.
\end{guidance}\end{question}
%%Hack to correct tcbox behaviour
\color{black}



%%%%%%%
%LR -- used the structuring by theme above instead, so this content is not needed any longer. Instead we need something about assessing the extent the review covers the required scope
%%%%%%%
%Look back at the pacman diagram in~\Cref{stage1}\todo{Introduce this in~\Cref{stage1} and explain.}. 
%
%You should now have most elements that you need to be able to describe the boundary of knowledge in the area of your \glsf{research-problem}, including being able to focus more closely on the knowledge gap you hope to fill. 
%
%What remains is to be able to turn those elements into an engaging narrative that leads the reader to the conclusion that your contribution can – will – be a contribution to knowledge. Practising the techniques you have just learnt will be a big step towards being able to do this in a way that satisfies your reader.
%
%\begin{figure}[htbp]
%\centering
%\includegraphics[width=0.8\textwidth]{pacman}
%\caption{Pacman: the relationship between the literature and the research gap.\todo[inline]{Move to~\Cref{stage1}, and reference here?}}	
%\end{figure}
%
%To do this, your literature review should tell a story to the reader, supported by \glsf{evidence} from previous research and the work of others.
%
%
%Your argument structure in the literature review is the core of your pacman diagram, which you should now be able to describe. The fun trick to do this, is to print out\footnote{Remember printing?} – yes, literally print out, using a printer – the text of each argument and to lay them in a pacman diagram on your floor. Arguments that are far from your research focus – those that set the general context of your research – should go far from the \enquote{mouth}, those that are key to your research should give the shape of the \enquote{mouth}. 
%
%This will give you a structure for your literature review, starting from far away from the \enquote{mount} and working towards it.
%
%Unfortunately, not everyone has a printer, or a floor big enough, and so another\footnote{But much less fun!} way to achieve this is to use a mind map type structure, labelling – or otherwise distinguishing them – and put them into mind mapping software, or even something like presentation software\footnote{Such as PowerPoint for PCs, Keynote for Macs, or the free LibreOffice for all OSes.}, again building the shape of the pacman as suggested above.
%
%\begin{question}[subtitle={Activity: Developing your literature review outline}] Based on the arguments you developed in the previous activity, draw the pacman diagram for your current literature review and, from it, produce a linear outline of your narrative  which indicates which arguments to include and in which order. You should use sections and subsections to structure it.
%
%Review and revise your outline, until you are satisfied that all the arguments you wish to make are included and they follow logically from each other.
%
%\begin{guidance}
%There is no single way to go about doing this. Here are some techniques you may like to try if you find it difficult to get started:
%
%\todo[inline]{More guidance needed for this one?}
%%
%\begin{itemize}
%\item [\color{red}Added] Draw a mind map of the arguments you have developed and how they relate to each other. Use the map to group and sequence the arguments in your outline. Try overlaying your arguments onto a pacman diagram – place the more detailed ones closer to the \enquote{mouth}. Once you're happy with their positioning, you can use that to order them.
%
%\item write a sentence/bullet point for each argument, then re-order them to ensure they follow from one another logically
%
%\item try different ways of grouping your arguments, for instance by theme or chronologically
%
%\item try some free writing, then read, review and re-organise to ensure there is a logical flow
%
%\item use sections (and subsections, if you need to) to group together related arguments.
%\end{itemize}
%
%You should spend up to 2 hours on this activity.
%\end{guidance}\end{question}
%%%Hack to correct tcbox behaviour
%\color{black}

%\paragraph{What's next?}
%
%Once you are happy with your outline, you can start filling in the details. Depending on how much you have read and the arguments you have been able to make, you should end up with a substantial initial draft of your literature review by the end of~\Cref{stage1}\todo{or 2?}.


\section{Assessing your literature review}\label{sect:AssessingYourLiterature}

It is time to assess whether you have done enough work on your literature review and can move on. If not, you will need to widen it to include what is currently missing -- there may be topics which you haven't explored sufficiently, or even some you haven't addressed yet. This means iterating the literature review process by searching, gathering, assimilating and synthesising more published work.

 The best person to assess your literature review is YOU!\footnote{Of course, others can help too, as you will see.} The following activities will help you assess your progress and guide any further work required.

\subsection{Your own assessment}\label{ssect:YourOwnAssessment}

At this point, your literature review should be sufficient to describe the boundary of knowledge in the area of your \glsf{research-problem}, including the knowledge gap you hope to fill. 

%\begin{figure}[htbp]
%\centering
%\includegraphics[width=0.7\textwidth]{pacman}
%\caption{Pacman: the relationship between the literature and the research gap.\todo[inline]{Move to~\Cref{stage1}, and reference here?}}	
%\end{figure}

Does your current review draft achieve this? To be able to answer this question, you should apply the criteria in~\Cref{tab:litrevcrit}, which also provides some prompts to help you apply each criterion.

\begin{SimpleNColTable}{tab:litrevcrit}{2}{Criteria for assessing your literature review}[lR[4]]
Criteria & Prompts \\
\Glsf{research-problem} underpinning & To which extent does it demonstrate your understanding of different facets of your \glsf{research-problem}? \\
\Glsf{research-problem} justification & To which extent does it argue that the \glsf{research-problem} is worth investigating? \\
Potential contribution to knowledge & How clearly does it expose  the knowledge gap? How clearly does it articulate why it is significant to address it? \\
Logical progression & To which extent does it include a logical progression of arguments? Do sections and sub-sections structure the narrative appropriately?  \\
\Glsf{critical-writing} & Are connections between ideas appropriately explored? Is there a good balance between description, analysis, \glsf{synthesis} and evaluation? Is your writing concise and precise?\\
Supporting references & To which extent are all key arguments supported by appropriate references? \\
Format and proofreading & Have you reviewed your writing carefully to remove typos and grammatical errors? Are all citations and references in the correct bibliographical style? \\

\end{SimpleNColTable}

\begin{question}[subtitle={Activity: Assessing your literature review}] Assess your current draft of the literature review by applying the criteria in~\Cref{tab:litrevcrit}.

\begin{guidance}
%Hack to correct tcbox behaviour
\color{black}


%Hack to correct tcbox behaviour
\color{black}
For each criterion, use the prompts to write down your own assessment and to record what is still missing: the latter will help you identify further work you will need to carry out.
\end{guidance}\end{question}
%%Hack to correct tcbox behaviour
\color{black}

\subsection{Getting others to help you}\label{ssect:GettingOthersHelp}

There are people around you who will be able to help. Gathering others' feedback will not only help you understand what still needs to be done, but it can also help you to find value in your work so far. Even if you feel the draft is scrappy\footnote{It certainly won't be in its final form as yet.}, for instance, others might be able to bring out things they like about it. 

Among them, the most important person to help you is your supervisor, as an experienced academic writer and topic expert.

\begin{question}[subtitle={ACTIVITY: Sharing your literature review with your supervisor}] Before going further with your literature review, it would be wise to ask your supervisor for comments on what you have already achieved. 

Share your current draft with them and ask them to find holes in your coverage and your arguments. Carefully gather any and all feedback you receive.
\begin{guidance}
%Hack to correct tcbox behaviour
\color{black}


%Hack to correct tcbox behaviour
\color{black}
As this point, it's unlikely that your supervisor will say \enquote{It's perfect!}\footnote{Although they may say \enquote{Well done!}, but don't worry if they don't:)} so be ready to hear one or more of the following comments:
%
\begin{itemize}
\item \enquote{you should read/add the following references to your review}
\item \enquote{this section seems out of order, perhaps it would be better elsewhere}
\item \enquote{you've missed this topic}
\item \enquote{this argument would be better expressed this way}
\item \enquote{Author X also made this argument}
\item \enquote{your conclusions are wrong because ...}
\item \enquote{you might like to speak to this colleague about this issue}
\end{itemize}
Likely, there are other comments not included in this list. Make sure you consider all feedback carefully and take notes of further work needed to develop your literature review further.
\end{guidance}\end{question}
%%Hack to correct tcbox behaviour
\color{black}


Other help might come from family members/friends/colleague that you can talk to. Read what you have out loud to a live listener and have them try to follow your arguments. 

\begin{question}[subtitle={Activity: A willing audience}]
Read your current draft to a willing family member/friend/colleague and ask for their feedback on it. Take notes of improvements that may be needed.
\begin{guidance}
%Hack to correct tcbox behaviour
\color{black}


%Hack to correct tcbox behaviour
\color{black}
Write down their comments as they make them (but try not to interrupt the flow too much). What was their reaction? Were they engaged or did they fall asleep (again)? Will they come back and help for future versions? Make sure you identify ways to improve your draft for your readers. 
\end{guidance}
\end{question}
%%Hack to correct tcbox behaviour
\color{black}


\section{Widening your literature review}\label{sect:WideningYourLiterature}

The 5-stage framework in this book assumes you develop a comprehensive initial draft of your literature review by the end of~\Cref{stage2}, with the bulk of gathering and assimilating articles happening in~\Cref{stage1}. However, even if you hit gold first time with your literature search, you are likely to require more than one iteration of the literature review process in order to develop a full draft to your own satisfaction and that of your supervisor. Therefore, expect some iterations also in~\Cref{stage2}. The focus of such iterations is to help you widen or deepen some aspects of your review --- something you and your supervisor will have identified while assessing your current draft.

\begin{question}[subtitle={Activity: Widening your literature review}] Select a number of topics or sub-topics for further exploration and iterate through the review process in order to gather further articles, assimilate and synthesise their content, and integrate it within your current literature review draft. Re-assess the content at the end and iterate if necessary.
\begin{guidance}
%Hack to correct tcbox behaviour
\color{black}


%Hack to correct tcbox behaviour
\color{black}
The topics or sub-topics to investigate may include:
\begin{itemize}
\item those you identified while assessing your current draft
\item those from suggestions and feedback from your supervisor
\item themes or concepts currently under-explored in your \glsf{concept-map}
\item interesting ideas or sub-topics you came across in previous reading, but yet to explore
\item articles cited in work you have already reviewed, or that cite that work.
\end{itemize}
At each iteration, make sure you apply the wide range of review techniques you have learnt in Stages 1 and 2. In particular, record all new entries in your BMT, alongside your notes and summaries; take good notes as you assimilate new content; update your summary-comparison and \glsplf{concept-matrix}, and your \glsf{concept-map}; and produce appropriate thematic summaries. 

As you integrate new content within your current draft, you may have to amend your arguments, or even rethink the structure of your draft. 

Alongside widening the content you should also improve your draft based on other feedback you have received, for instance improving the narrative flow or clarifying points your readers may have found obscure or poorly expressed.

Importantly, make sure you take your supervisor's comments seriously, not just by taking note of what they say and implementing it, but – and this is a good habit to form – writing a response: thank them for each of their comments,  and tell them how the literature review has changed because of their input. Your response doesn't have to be long – the shortest response to a comment could be \enquote{Thank you, done!}. Doing this will make them – or anyone that comments – feel really valued. 

This is a substantial activity which, depending on how much material you still have to review, may take you several days, if not weeks.
\end{guidance}
\end{question}
%%Hack to correct tcbox behaviour
\color{black}

\chapter{Research design foundation}\label{ch:ResearchDesignFoundation}

You do research to make a contribution to knowledge. Practically, to do research, you combine a number of research tasks into a framework. Designing such a framework is what is known as \glsf{research-design}. The framework will depend on your research area, the type of knowledge contribution you wish to make, your mindset as a researcher, and the opportunities and difficulties you may face along the way.

\Glsf{research-design} has many elements, discussed in this chapter.

\section{Philosophical foundation}\label{sect:philosophicalFoundation}

The philosophical pillars of \glsf{research-design} are \glsf{ontology}, \glsf{epistemology}, and \glsf{research-methodology}: 

\begin{description}
\item [\Glsf{ontology}] is the philosophical study of the nature of existence and addresses the question: \enquote{What is the reality that I will research?}. Practically, \glsf{ontology} translates to determining which \emph{\glsplf{phenomenon}} exist in the context of  your research, the \emph{relations} that exist between them and how they group together into \textit{categories}.

\item [\Glsf{epistemology}] is the philosophical study of knowledge and addresses the question: \enquote{How is knowledge generated and from what sources?}. Practically, \glsf{epistemology} is about addressing \enquote{What do people know?}, \enquote{What does it mean to say that people know something?}, and \enquote{How do people know that they know?}.

\item [\Glsf{research-methodology}] is the \glsf{system} of principles and methods by which you conduct research, that is, investigate \glsplf{phenomenon}, and generate and analyse data and \glsf{evidence}. Methodology operationalises the \enquote{how} question of knowledge generation, so it is about devising concrete ways to answer \enquote{How will I make my contribution to knowledge?}. 
\end{description}

As you might have guessed, given that the goal of research is to make a contribution to knowledge, \glsf{epistemology} and \glsf{ontology} are incredibly important in defining what knowledge is in any particular research context and what, in that context, can be known. Once this choice is made, an appropriate methodology can be devised: hence, methodology depends on choices made in relation to \glsf{ontology} and \glsf{epistemology}. 

Fortunately, many scholars have thought very deeply about \glsf{ontology} and \glsf{epistemology}\footnote{For instance, if you're interested, you can find a fuller discussion of {\glsf{ontology}} and \glsf{epistemology} in the Stanford Encyclopaedia of Philosophy.} and, in most areas and for the vast majority of master's or doctoral level research, their thinking will suffice. If not, you'd be left in a situation in which even an ostensibly simply statement like \enquote{That hat is blue} becomes in need of complex debate \parencite[Section 4.1]{steup2020epistemology}.

Methodology, on the other hand, is something you will spend some time on, particularly how individual research methods combine to produce knowledge contributions through research strategies. 

You should be aware that \glsf{research-methodology} has many meanings in the literature, including the study of research methods, which questions the assumptions that underpin their creation and application. Wikipedia says\footnote{It could almost be seen as a warning!}:
%
\blockquote{%Many discussions in methodology concern the question of whether the quantitative approach is superior, especially whether it is adequate when applied to the social domain. 
[...] A few theorists reject methodology as a discipline in general. For example, some argue that it is useless since methods should be used rather than studied. Others hold that it is harmful because it restricts the freedom and creativity of researchers. Methodologists often respond to these objections by claiming that a good methodology helps researchers arrive at reliable theories in an efficient way. The choice of method often matters since the same factual material can lead to different conclusions depending on one's method. Interest in methodology has risen in the 20th century due to the increased importance of interdisciplinary work and the obstacles hindering efficient cooperation.}
%
These are not unimportant issues to consider. However, and as for \glsf{ontology} and \glsf{epistemology}, this book will leave their discussion to others, content to stand on those giants' shoulders – it takes an unapologetically practical approach, limiting all discussions to what are their important considerations in practice, particularly in relation to how to choose an appropriate \glsf{research-method}, and what an experienced reader will expect to be answered by it. You can then craft your dissertation to meet those expectations.

%\bigskip
%
%Before we meet research methods, however, it's worth thinking more about what is the foundations of \glsf{ontology}, \glsf{epistemology} and methodology is the research \textit{paradigm}, the \glsf{system} of beliefs, ideas, values, and habits that guide (and constrain) a researcher's way of thinking about the world.

\section{Data and evidence}\label{sect:evidenceAndData}

The \glsplf{phenomenon} upon which your research will be based must be observable and this gives rise to data. \Glsplf{datum} are the raw observations with no interpretation attached --- anything you may collect, capture or gather in your research. \Glsf{evidence} is information interpreted to support (or otherwise!) your academic arguments. Indeed, data forms the basis of \glsf{evidence}, so the two concepts are closely linked and often used interchangeably. This section recalls briefly the main types of data and \glsf{evidence} used in \glsf{academic-research}.

\Glsplf{quantitative-datum} are data that can be quantified or measured, and be given numerical values. They include the following types:

\begin{itemize}
\item \glsplf{numerical-datum} are numbers\footnote{Yes, they are!}, such as the number of students registered on a module or the temperature in the UK in July. Simplifying a little, when \glsplf{numerical-datum} has a whole-number value it is called discrete, otherwise it is continuous\footnote{Given the fundamental nature of energy, and the vagaries of quantum physics, it may be incorrect to state that real--world temperature is actually a continuous \glsf{variable}. However, even if it isn't, its values lie on a continuous scale.}. In either case, you can apply appropriate mathematical and statistical operations, we can order values and calculate the interval between two numbers.

\item \glsplf{ordinal-datum} are non \glsplf{numerical-datum} that can be arranged in an order. An example is the very widely used Likert scale you often encounter in \glsplf{questionnaire} to elicit opinions, like a 5-point Likert scale ranging from \enquote{Strongly disagree} to \enquote{Strongly agree} with \enquote{Disagree}, \enquote{Neither Disagree nor Agree}, and \enquote{Agree} in the middle\footnote{Almost certainly, the most recent survey you completed would have used a 5--point Likert scale similar to this.}. While these values can be arranged in order\footnote{This might be done by giving \enquote{Strongly disagree} the numerical value 1, \enquote{Disagree} the numerical value 2, and so on.}, you can only apply mathematical and statistical operations in a limited way, for instance, taking the mean (or average) score of responses on the Likert scale.\footnote{You can read more about Likert scales use (and misuse) in \textcite{carifio2007ten}.}

\item \glsplf{interval-datum} are data which can be arranged on a scale, so that we can calculate the distance between any two data points. All \glsplf{numerical-datum} are also \glsplf{interval-datum}, but \glsplf{interval-datum} may not be numerical. For instance, calendar dates are \glsplf{interval-datum} as you can calculate the time interval between two given dates, e.g., the number of days in between.

\item \glsplf{ratio-datum} are \glsplf{numerical-datum} with an absolute zero considered as a point of origin, that is no negative values are possible. Examples include a person's high, weight or their wage from employment: none of these can take a meaningful value less than zero.
\end{itemize}

\Glsplf{qualitative-datum}, on the other hand, are descriptive in nature and defy ordering. Sentences, images, sounds,~\etc{}, are all examples of \glsplf{quantitative-datum}. An important subclass of \glsplf{qualitative-datum} is \glsplf{nominal-datum} commonly used to denote categories, for instance, Dog, Cat, Alligator,~\etc{}. You cannot order these data or apply mathematical operations and functions.

Note that in Statistics \glsplf{ordinal-datum} and \glsplf{nominal-datum} are called \glsplf{categorical-datum}, exactly because they are used to denote categories, which may or may not be ordered. This also means that \glsplf{categorical-datum} span the quantitative/qualitative divide.

Data and \glsf{evidence} are also classed as:
%%JW 409
\begin{description}
	\item \glsplf{primary-datum} and \glsf{primary-evidence}, when newly generated or collected during research; or
	\item \glsplf{secondary-datum} and \glsf{secondary-evidence}, when already available from previous research, and re-used during new research.
\end{description}

The \glsf{academic-literature} at the core of your literature review is \glsf{secondary-evidence}, as are all other published academic and non academic documents, e.g., laws, policies and procedures, official reports,~\etc{}.

\begin{example}{Thinking about types of data and \glsf{evidence}} 
In the Curriculum Analytics example, the research aims to develop novel and effective techniques to apply to curriculum data to support stakeholders in their day-to-day curricular decisions.

Such research will need to make use of a range of curriculum data which are likely to be both quantitative (e.g., number of students, pass rates,~\etc{}) and qualitative (e.g., components of a programme of study, learning outcomes,~\etc{}). These will be all \glsplf{secondary-datum}, in the sense that they will be provided by the target university so that new techniques can developed, applied and evaluated.

Stakeholders should be involved in the evaluation, to assess the effectiveness of those techniques, and this may lead to \glsplf{quantitative-datum} as \glsf{primary-evidence}. The evaluation may also compare predictions made by the techniques to outcomes captured by historic data, so that some \glsf{primary-evidence} may also be quantitative in nature.
\end{example}

\begin{question}[subtitle={Activity: Types of data and \glsf{evidence} in your project}] List and justify the kinds of data and \glsf{evidence} your research will make use of or generate. 
\begin{guidance}
%Hack to correct tcbox behaviour
\color{black}


%Hack to correct tcbox behaviour
\color{black}
Consider both existing data and \glsf{evidence} you will need and any new data/\glsf{evidence} your research may produce. Justify your answer in terms of the specific aim and objectives of your work.
\end{guidance}
\end{question}
%%Hack to correct tcbox behaviour
\color{black}


\section{Researcher mindsets}\label{sect:ResearcherMindsets}

Depending on your background, you may have begun your research studies with a particular mindset – that of a scientist, for instance, or as someone embedded within an organisation. This mindset will flavour your approach to research, but it shouldn't constrain it – there are many options for research and the right one for you might be outside of your current understanding.

%That's not a bad thing – many have been there before and constructed \emph{\glsf{research-paradigms}} that characterise useful mindsets for researchers to have. 

Over time, researchers in different communities and disciplines have developed differing mindsets, which are known in the literature as \glsf{research-paradigms}\footnote{AKA philosophical traditions.}: you can think of a research paradigm as a philosophical way of thinking, a set of shared beliefs which shape a worldview. 

This books briefly outlines the prevalent ones --- there is a lot, lot more to be known around this topic, and this introduction only scratches the surface\footnote{The book includes some references for you to start your own investigation into this fascinating and complex topic, should you wish to.}!

Each paradigm comes with its own ontological, epistemological and methodological choices. It is important for you to be aware of their existence as this may help guide your \glsf{research-design} choices, even if in practice you will mainly focus on methodological considerations. 

\subsection{Positivism and post-positivism}\label{ssect:PositivistAndPost-positivism}
\Glsf{positivism} assumes that there is a single, objective reality that can be accurately known, described and explained. It contributes knowledge as explanations of this reality, constructed from \glsplf{hypothesis} which are confirmed through observations and measurements, hence becoming universal laws or facts.

%They also depends on \textit{\glsf{deductive-reasoning}} to generate that knowledge: starting from general observations as assumptions they deduce specific conclusions through logical steps, and as long as the assumptions are true, then so are the conclusions reached. 

As an example, think of Newton's explanation of the action of forces on matter that is captured in his Three Laws of Motion: these are meant as universal objective truths which apply to the natural world forever.

%Under this assumption, positivist researchers make claims of new knowledge that they compare against reality to determine the \enquote{truth}. Hence positivists seek to confirm their theories through their observations and measurements of an objective reality. 

In assuming a single, objectively knowable reality, \glsf{positivism}  removes the researcher as a \glsf{variable} in the research equation: research is necessarily limited to data generation, analysis and interpretation from an objective viewpoint as the basis of knowledge. As such, it befits research where a single objective reality can be assumed, such as the natural sciences, the physical sciences, or whenever very large \glsf{sample} sizes can be used to infer characteristics of a \glsf{population}\footnote{You will return to samples and populations in~\Cref{stage3} and \Cref{stage4}.}. It leads the researcher towards quantitative data and \glsf{evidence}.

%, including experiments, tests, surveys and simulations involving formal modelling based on mathematics, statistics or \glsf{computational-thinking}. 

\Glsf{positivism} emerged in the late eighteenth and early nineteenth centuries in Western societies, fuelled by a growing optimism on the role and power of the natural sciences -- as witnessed, for instance, by the universal acceptance of Newton's Three Laws, and their explanatory and predictive capabilities backed up by empirical observations. So much so, that it was the predominant paradigm for almost a century and a bulkhead against a growing number of worrying observations, including the movements of the planet Mercury\footnote{See \textcite{wikipedia-contributors2025mercury}, for instance.}, which didn't reinforce – indeed appear to contradict – Newton's Laws. How could an established truth lead that way? Indeed, Einstein's insight into the intimate connection between space and time inspired a substantial move away from the established Newtonian \enquote{laws} and \enquote{facts}, which were neither any longer~\parencite{lakatos2014falsification}.

The need to rethink positivist objective truths was something of a crisis in the positivist movement~(see, for instance, \textcite{kuhn2012structure}), leading to \glsf{post-positivism}\footnote{Not the most creative of names.} which introduced the idea of \glsf{falsification}: any posited \glsf{theory} must make predictions which are testable, the currency of a \glsf{theory} being determined by whether or not it had yet been proven false\footnote{Note that falsifiable theories that have been tested and failed can still be useful, perhaps within a restricted context. For instance, Newton's Laws of Motion provide a very good approximation at low energies.}. 

\bigskip

So, both \glsf{positivism} and \glsf{post-positivism} accrete knowledge by formulating generalisations and cause-effect linkages, based on objective, verifiable observations and measurements, and expressed as theories and laws. However, \glsf{post-positivism} acknowledges some of the limitations in such observations and measurements, so that a \glsf{theory} or law will only remain true for as long as it is not falsified by new observations or measurements. There is therefore a shift from certainty (\glsf{positivism}) to probability (\glsf{post-positivism}), with post-positivist researchers encouraged to take multiple measurements and observations, including triangulating\footnote{\Glsf{triangulation} is covered in detail in~\Cref{stage3}.} their data, to arrive at an objective truth. Thus you might take a post-positivist approach to establishing the linkage between a drug and the alleviation of symptoms: once a \glsf{generalisation} or cause-effect linkage is established, it applies for as long as it remains un-falsified.

Both \glsf{positivism} and \glsf{post-positivism} assume an objective reality and do not admit that the researcher's own mindset and values may influence true knowledge: in being objective and verifiable, different researchers must necessarily arrive at the same truth, as long as the \glsf{research-process} is reliable\footnote{You will encounter reliability in the next section.}: different researchers should be able to follow the same process to arrive at the same conclusions. 

This denial of the researcher's influence on the research is often levelled as a criticism of these paradigms, particularly by social scientists, and has led to new paradigms.

%\todo{See \cite{wikipedia-contributors2023positivism} for a more detail description.}

%\begin{question}[subtitle={Activity: Am I a post-positivist thinker?}]
%***to rethink****On a scale of 0 to 10, with 0 wholly in disagreement and 10 wholly in agreement, to which extent do you agree with the following statement?
%
%\textit{Even though the cat is blue, I can imagine situations in which it could be observed as some other colour.}
%\begin{guidance}
%	If you thought this was easy, you might be a post-positivist thinker. Post-positivist thinkers may tend towards quantitative research.
%\end{guidance}
%\end{question}
%\todo{Clarification of methods should go before activity, to make clear the distinction between posit and post-posit}

\subsection{Anti-positivist}\label{ssect:Anti-positivist}

The shift from \glsf{positivism} to \glsf{post-positivism} still preserves the absolute objectivity of reality. In contrast, \glsf{anti-positivism}\footnote{AKA interpretivism.} asserts that different people experience and understand reality in different ways: while there may be only \enquote{one} reality, everyone interprets it according to their own views. Simply put, this might mean that generalisations and even cause-effect relationships are subject to individual experience. Think of the way that people interpret the (single) power structure within your organisation: typically, different people will describe it in different ways, as it applies to them.

Explaining the name, anti-positivists believe that all research is influenced and shaped by researchers’ worldviews, leading to differing interpretations of the same reality. Again, think of the questions you might ask of people within an organisation that leads them to describe the power structure. Different people will give different descriptions.

As a result, anti-positivists gravitate towards \glsplf{quantitative-datum} and \glsf{evidence} to characterise the
 different perspectives, each placed in its own explicative context. 

%These may include interviews and focus groups, participant observations, and review of documentation on a \glsf{phenomenon} of interest (e.g., newspaper articles, reports, or information from websites).

In moving away from objective knowledge, however, \glsf{anti-positivism} raises questions of research \glsf{validity}\footnote{You will encounter \glsf{validity} in~\Cref{ch:DefendingYourClaim}.}, that is of how trustworthy and generalisable knowledge generated as subjective interpretation might be. 

%See \cite{wikipedia-contributors2023antipositivism} for a more detailed description.

\subsection{Constructivism}\label{ssect:constructivism}

\Glsf{constructivism} goes further than \glsf{anti-positivism} and asserts that reality is a construct of our minds and so is absolutely subjective. Constructivists believe that all knowledge comes from our experiences and reflections on those experiences as formed in our mind. A distinction is also made between reality which is individually vs socially constructed, the latter being the result of social interaction within a specific cultural or historical context. 

Due to its focus on experiences and subjectivity, this paradigm is also mostly associated with \glsplf{quantitative-datum} and \glsf{evidence}. The researcher focuses on participants’ experiences, including their own, constructing knowledge through understanding, sense making and reconstruction.

%Knowledge accumulates through later research adding informed and sophisticated reconstructions, and vicarious or lived experience\todo{\url{https://image.slidesharecdn.com/lecture21-111207045819-phpapp02/95/research-paradigms-20-728.jpg?cb=1415227903}}.

Establishing research \glsf{validity} is an even more contentious issue with this paradigm.

\subsection{Critical theory}\label{ssect:CriticalTheory}
The \glsf{critical-theory} research paradigm originated in the fields of sociology, philosophy and political theory, and asserts that social science can never be 100\% objective or value-free. Like \glsf{anti-positivism}, it assumes multiple interpretations of reality in social contexts. In fact, the researcher's values are acknowledged and welcomed as a formative influence on the research.

Critical theory asserts that reality is shaped by those who are powerful, who legitimate particular ways of perceiving the world: \enquote{truth} is inherently political, defined by those in charge to the disadvantage of many, and challenged by those who wish to promote \glsf{equality}. Therefore, critical researchers seek to challenge the status quo and perceive research as transformative at a social level\footnote{Which is why this paradigm is also called \enquote{transformative} in the literature.}, confronting ideology and trying to discover and challenge the mechanisms through which exploitation and disadvantage are perpetuated in society. Hence, critical theory focuses on enacting social change through scientific investigation. 
 
In particular, critical theorists question knowledge and procedures, and acknowledge how power is used (or abused) in the context they’re investigating. As the basis of knowledge, they offer insights into society which are rooted in history and power structures, and that are both critical and transformative, aimed at emancipation and restitution to address historical injustices. 

The quality of \glsf{critical-theory} research is judged in terms of how well it is situated in its historical context, and the extent it acts as a stimulus for transformation, and for reducing ignorance and misconceptions.

\subsection{The indigenous paradigm}\label{ssect:indigenous}
The paradigms just described have attracted criticisms in that they are seen as Western-European centric, imposed on  \glsf{indigenous} cultures as a result of colonialism, hence marginalising \glsf{indigenous} traditions.

 In counterposition, an \glsf{indigenous} paradigm is emerging with the aim of decolonising research. This paradigm  emphasises the connection between people, their culture, and the spiritual and natural worlds, valuing knowledge which is local to communities, and holistic in connecting all beings with nature and spirituality. 
 
 As a result, the way the research is conducted should reflect \glsf{indigenous} cultural practices and forms of expressions, including language, metaphors, oral traditions and \glsf{indigenous} knowledge systems. 

It follows that, from an ontological perspective, both physical and spiritual realities and their connection matter, alongside reciprocal relations among all living beings. 

From an epistemological perspective, knowledge is relational, based on the connection between natural and spiritual worlds, and its generation is a fluid process based on oral traditions, such as storytelling, and inward exploration of personal experience in context. The codification of such knowledge is through community praxis, in which the \enquote{Elders} are often seen as key actors in the epistemological process.

Methodologically, the indigenous paradigm favours the collective involvement of indigenous people in developing, approving and implementing the research, leading to knowledge of practical use.   

It is important to note that characterising this paradigm in relation to \glsf{ontology}, \glsf{epistemology} and methodology is rejected by some scholars, seen as a form of colonialism imposed by a Western view of \glsf{research-paradigms}\footnote{If you are interested in this debate, you could start from \cite{hart2010indigenous}.}.

\subsection{What's your mindset?}\label{ssect:WhatsYourMindset}

\Cref{tab:researchParadigms} summarises the main paradigms discussed in this section based on their ontological, epistemological and methodological standpoints. For the latter, their quantitative vs. qualitative tendency is indicated, although such a distinction is not as stark in practice, and a mix of the two often applies.

\begin{SimpleNColTable}{tab:researchParadigms}{7}{Summarising \glsf{research-paradigms}} [lR[4]R[4]R[4]R[4]R[4]R[4]]
&  \Glsf{positivism} & \Glsf{post-positivism} & \Glsf{anti-positivism} & \Glsf{constructivism} & \Glsf{critical-theory} & \Glsf{indigenous} \\

%\textbf{Aim} & to discover general laws and principles and predict behaviour through neutral and objective enquiry & to discover general laws and principles and predict behaviour through neutral and objective enquiry & to understand and predict social systems and their behaviour, acknowledging that the researcher's values and experience affect the enquiry & to understand and predict social systems and their behaviour, acknowledging that the researcher's values and experience affect the enquiry & to change society, challenge norms and emancipate and empower people & ***marginalised and post-colonial communities research ???\\
%\midrule
\Glsf{ontology} & one discoverable external reality & one discoverable external reality that can only be known imperfectly  &  one external reality which is interpreted subjectively & reality as the construct of one's mind & one external reality determined by historic power factors &   physical and spiritual realities and their connection; reciprocal relations between all living beings \\

\Glsf{epistemology} & objective laws and theories that can be confirmed empirically & objective laws and theories that can be falsified empirically & subjective interpretations & subjective constructions & social and historical constructions, acknowledging issues of power and social injustice & relational knowledge, \glsf{indigenous} knowledge systems based on oral traditions and inward exploration of experience\\ 

Researcher's role & objective, neutral & objective, neutral, aware of cognitive limitations & subjective, bringing own values, experience and \glsf{bias} & subjective, bringing own values, experience and \glsf{bias} & subjective, aware of own social position & researcher as \glsf{indigenous} participant in collective research \\

%\textbf{Main strategies} & experimental, survey, \glsf{simulation}, \glsf{mathematical-logical-proof} & experimental, survey, \glsf{simulation}, \glsf{mathematical-logical-proof} & case study, \glsf{action-research}, \glsf{ethnography}, \glsf{phenomenology}, \glsf{grounded-theory} & case study, \glsf{action-research}, \glsf{ethnography}, \glsf{phenomenology}, \glsf{grounded-theory}, \glsf{systematic-review} & \\
%\midrule
Methodology & quantitative & quantitative, with \glsf{triangulation} & qualitative & qualitative & qualitative & qualitative\\

\end{SimpleNColTable} 

Your own mindset may lead you to gravitate towards one or more of these paradigms, or even somewhere in between. The next activity should help you reflect on this point.   

\begin{question}[subtitle={Activity: What kind of thinker am I?}]
	Consider the following question and describe how you would go about answering it:
%	
	\begin{displayquote}
	What colour are swans?
	\end{displayquote}

	
	Then compare your approach to each of the paradigm. Which one is it closer to and why?
	\begin{guidance}
		If you can think of more than one way to approach the question, then describe and reflect on each of them in relation to the paradigms.
	\end{guidance}
	\begin{solution}
		Below are are a couple of ways you could tackle this question. 
		
		The first is to start by observing the swans that live on a lake near your home, and record your observations. Assuming they are all white, you could put forward an initial \glsf{hypothesis}, say that all \enquote{swans are white}. You could then look online for images of swans from around the world to see if they match your observations. Having found images of black swans alongside white ones, you would then revise your \glsf{hypothesis} to \enquote{All swans are either white or black.} This process could continue until you're satisfied there is no further contradictory \glsf{evidence}, hence conclude that in all probability swans are either white or black. You would have to admit that there may be swans of other colours you've yet to come across, so the statement is open to future challenges. You would also need to be convinced that you are a neutral observer, able to determine the colour of a swan correctly and reliably. This approach closely aligns with the \glsf{post-positivism} paradigm, specifically: you make observations, triangulate them by reviewing online swan images, and formulate, rejected and then reformulate \glsplf{hypothesis} as part of your enquiry process.
		
		A second approach would be to ask other people. For instance, you could set up a crowd-sourcing survey inviting participants to answer the question. By analysing their answers you could then decide if there is enough consensus on the colour of swans: for instance, most participants may have identified swans to be either white or black, although some may have provided more nuanced answers, like yellowish or other. From your analysis of the answers, you would draw your conclusions which may or may not be the same as in the previous approach. In this case, you would have to worry about who participated in your research. Were there enough participants from around the world to provide sufficient and diverse evidence? To which extent may their colour perception differ? Were their trustworthy in their answers? What else could you do to check the \glsf{validity} of the outcome? This approach aligns with the anti-positivist paradigm: you have to accept that, each observer in your study will make their own interpretation of what the colour of a swan is, so that you would have to account for possible differences in your conclusions. 
\end{solution}
\end{question}
%%Hack to correct tcbox behaviour
\color{black}


\section{Methodology}\label{sect:stage2methodology}

The most practical aspect of your \glsf{research-design} is your methodology. While your literature review helps you establish \textit{what} you are going to research, your methodology tells \textit{how} you are going to do it, that is how to go about generating, analysing and interpreting data and \glsf{evidence} to meet your \glsf{research-aim} and objectives. 

An appropriate methodology is one which will allow you to make your contribution to knowledge in a way which meets the expectations of researchers in your field of study, these in turn being motivated by the mindsets you encountered in~\Cref{sect:ResearcherMindsets}: although practically you could apply them without referencing such mindsets, it is important for you as a researcher to be aware of the beliefs and values they embody, and the extent they align to your own -- something which will influence your own \glsf{research-design} choices. 


%Of course, the notion of correctness here is a loose one: innovation in \glsf{research-design} in a particular domains happens more or less often. Indeed, it is often an innovation in the design of research by which new knowledge is contributed. And it's not unknown for a researcher new to the domain to bring fresh thinking – it may even be you. There are risks of innovating, however, including finding it difficult to publish – not a concern for you – having difficulty convincing a reader that's not ready to accept innovation – essentially, being accepted. Risk management is often a Good Thing.

The building blocks of your methodology are research methods and research strategies, briefly introduced here: you will return to them in much more detail in~\Cref{stage3} and \Cref{stage4}.


%Those you will encounter in this book are summarised in Cref{fig:strategiesAndMethods}: we will look at them in detail in Stages 3 and 4.
%
%\begin{figure}[htbp]
%\centering
%\includegraphics[width=0.9\textwidth]{strategiesAndMethods.pdf}
%\caption{Research strategies and methods}
%\label{fig:strategiesAndMethods}
%\end{figure}


\subsection{Research methods}\label{sect:resMethIntro}

\Glsplf{research-method} are standard ways to collect, analyse, synthesise, present and interpret data to generate \glsf{evidence} and derive findings\footnote{This section only provides an overview. You will study research methods in detail in~\Cref{stage4}.} You can think of them as basic techniques and procedures that researchers have developed over time to deal with different kinds of data and \glsf{evidence}. They help you conduct your research in a systematic, rigorous and reliable fashion, ensuring you can defend your claimed contribution to knowledge at the end of your project.

Research methods are many and vary greatly, which can be confusing to the novice researcher --- and even the seasoned one at times! One source of confusion is that the same term is often used by different academic authors to indicate both specific techniques and procedures, and broad {research strategies} combining many. So, be aware that the classification in this book may be different from others you may find in the literature.

This book classifies methods into the following categories (see~\Cref{tab:generationMethods,tab:modellingMethods,tab:analysisMethods}):
%
\begin{itemize}
	\item \glsplf{generation-method} are used to generate data. You can apply them to generate \glsplf{primary-datum} from \glsplf{phenomenon} you observe in the real-world or from the people participating in your research
	\item \glsplf{modelling-method} are used to build models of complex real-world situations, which can then be used to generate further data. For instance, you may want to predict how a new aircraft you are designing is likely to respond to severe wether conditions: you can do this safely by building a \glsf{model} of the aircraft and testing it under various meteorological conditions
	\item \glsplf{analysis-method} are used to analyse data. Once you have your data, you can apply these methods to ensure your analysis is systematic and rigorous.
\end{itemize}


\begin{SimpleNColTable}{tab:generationMethods}{2}{Generation methods discussed in this book}[X[-1,r]X[l]]
Name & Description\\
\Glsf{observations-measurements} & direct \glsf{observation}/measurement of \glsplf{phenomenon} of interest \\ 
\Glsplf{questionnaire} & pre-defined sets of questions used to gather answers from respondents \\
Interviews & form of conversation between researcher and interviewee(s) to gain insights and opinions around a specific topic \\
Focus groups & interactive group discussion to develop an understanding of complex \glsplf{phenomenon} informed by multiple perspectives \\
\Glsf{delphi} & iterative process of collecting and synthesising anonymous judgements from experts to arrive at a consensual view \\
\Glsf{journaling} & written personal records of participants' experiences and observations during a study, using \glsf{reflection} and \glsf{reflexivity} to surface inner thoughts, feelings, motivations and perceptions\\
\Glsf{fieldwork} & direct collection of data in a natural setting, including observations, measurements, samples, specimens, or other \\
\Glsplf{document} & use of existing documents to develop new insights or answer research questions\\
\end{SimpleNColTable} 


\begin{SimpleNColTable}{tab:modellingMethods}{2}{Modelling methods discussed in this book}[X[-1,r]X[l]]
Name & Description\\
\Glsf{computational-thinking} & problem solving approach aimed at computational solutions in the form of computer programmes and systems\\
\Glsf{mathematical-thinking} &  problem solving approach aimed at mathematical solutions\\
\Glsf{statistical-thinking} & problem solving approach aimed at statistical solutions, particularly for prediction and forecasting\\
\Glsf{systemic-thinking} & problem solving approach focused on understanding and improving complex systems and their dynamics \\
\end{SimpleNColTable} 


\begin{SimpleNColTable}{tab:analysisMethods}{2}{Analysis methods discussed in this book}[X[-1,r]X[l]]
Name & Description\\ 
\Glsf{tables-and-charts} & tools to summarise and visualise data in order to identify interesting patterns\\
\Glsf{statistical-analysis} & set of techniques to investigate trends, patterns, and relationships in quantitative data \\
\Glsf{thematic-analysis} & technique to identify recurring themes, their definition and relationships in \glsplf{quantitative-datum}\\
\Glsf{content-analysis} & technique to investigate certain words, themes, or concepts in \glsplf{quantitative-datum}\\
\Glsf{discourse-analysis} & technique to investigate how language is used in conversations\\
\Glsf{narrative-analysis} & technique to investigate meaning behind people's narrative accounts\\
\Glsf{coding} & the process of creating and assigning codes to categorise  extracts from \glsplf{quantitative-datum}\\
\end{SimpleNColTable} 


\begin{question}[subtitle={Activity: Considering possible methods for your project}] Consider the methods in~\Cref{tab:generationMethods,tab:modellingMethods,tab:analysisMethods}. Read their description and make a note of those you find potentially suitable for your project.  
\begin{guidance}
%Hack to correct tcbox behaviour
\color{black}


%Hack to correct tcbox behaviour
\color{black}
This activity is designed to help you become familiar with the methods you will encounter later in the book. Don't worry if you do not understand the details or you can't decide yet whether they may be applicable to your project. You will return to them in more detail in~\Cref{stage3}, which will also provide criteria to help you choose. 
\end{guidance}
\end{question}
%%Hack to correct tcbox behaviour
\color{black}


\subsection{Research strategies}


%, by which we mean a collection of research tasks that interact in more or less complex ways, but which are sufficiently detailed that the researcher knows what to do next, even if that means making a choice between two or more next steps. You will need to choose your own \glsf{research-strategy} for your project.

%With uncertainty a product of many contextual characteristics, including the researcher's view-point, the simpler research strategies stem from contexts with lesser uncertainty. Thus, highly constrained contexts, such as the natural sciences, tend to have simpler strategies as there is less uncertainty to contend with. \todo{to rephrase/expand, but not sure how}

%There's good news and bad news in choosing a \glsf{research-strategy}:
%%
%\begin{itemize}
%\item the bad news is that there are many possible choices you could make at any point.
%\item the good news is that, for your particular area in master's research, there will likely be only a small subset that you need to know about.
%\end{itemize}
%
%To help you in your choice, our approach in~\Cref{stage3} of this book is unapologetically practical. It will layout the options that you have together with reasons for choosing them and reasons for not choosing them. Each comes with a list of key evaluation questions the answers to which you will be expected to present as part of your dissertation. Amongst other things, the answers you give will justify how and why your work makes a contribution to knowledge. These evaluative questions in turn give you targets to aim for throughout your research, you will need to answer each of them – they will be the driver for your research and your writing up. 
%
%Before we look at \glsf{research-strategy} in detail, we are going to consider the importance of such choice in terms your ability to defend your \glsf{claim} that your research has contributed new knowledge.  

A \glsf{research-strategy}\footnote{A well-known take on {strategy} is that of the economist \textit{Michael Porter}: \enquote{The essence of strategy is choosing what not to do}. Your \glsf{research-strategy} should do this – by giving you a focus – so choose carefully, and use its guidance wisely.} is a systematisation of a set of research methods, which can be applied together in order to address research problems of a particular kind. As such it can help you select and apply an appropriate mix of research methods to generate and analyse data, and derive insights. 

Each research discipline and area has its more-or-less well-worn paths to a successful knowledge contribution and has devised strategies to do so. Those discussed in this book are summarised in~\Cref{tab:strategiesSummary}. 

\begin{question}[subtitle={Activity: Considering possible research strategies for your project}] Consider the strategies in~\Cref{tab:strategiesSummary}. Read their brief description and make a note of those you find potentially suitable for your project.  
\begin{guidance}
%Hack to correct tcbox behaviour
\color{black}


%Hack to correct tcbox behaviour
\color{black}
Like that on research methods, this activity is designed to help you become familiar with the strategies you will encounter later in the book. As per research methods, don't worry if you do not understand the details or you can't decide whether they may be applicable to your project. We will return to each of them in detail in~\Cref{stage3}, also providing criteria to help you decide. 
\end{guidance}
\end{question}
%%Hack to correct tcbox behaviour
\color{black}

\begin{SimpleNColTable}{tab:strategiesSummary}{2}{Research strategies discussed in this book} [X[-1,r]X[l]]
Name & Used to \\
 \Glsf{survey-research} & uncover patterns that can be generalised from a \glsf{sample} to the target \glsf{population}\\
\Glsf{design-science-research} & generate insights from the design and creation of an artefact as the novel solution to a practical problem\\
\Glsf{experimental-research} & establish cause and effect relationships between real-world \glsplf{phenomenon}\\
\Glsf{case-study-research} & provide a detailed insightful account of a \glsf{phenomenon} in its natural context\\
\Glsf{action-research} & trigger action to address important problems that people experience in their professional practice\\
%%JW421
%\Glsf{ethnography} & provide a cultural characterisation of a group of people in their natural setting\\
\Glsf{ethnography} & in-situ, participatory study of cultures, including practices, meanings and interactions, reported as rich (or \enquote{thick}) description\\
\Glsf{systematic-review} & generate new insights from accumulated published academic work, not contained in individual articles\\
\Glsf{grounded-theory} & formulate theories concerning complex social \glsplf{phenomenon} grounded in people's own accounts and interpretations\\
\Glsf{phenomenology} & provide insights into people's lived experience\\
\Glsf{simulation} & study a simulated artefact or \glsf{system} under different conditions, to answer \enquote{What if?} questions, make predictions or gain insights on its behaviour or properties\\
\Glsf{mathematical-logical-proof} & generate true propositions which grow the scope and applicability of Mathematics\\
\Glsf{mixed-methods-research} & combine quantitative and qualitative methods to increase both breadth and depth of understanding of a \glsf{phenomenon} under study\\
\end{SimpleNColTable}

%%Old version
%\section{Research strategies}\label{sect:ResearchStrategiesA}
%To help you in your choice, our approach in this chapter is unapologetically practical. 
%
%In~\Cref{sect:standardResearchStrategies}\todo{This section? Check!}, we will layout the options that you have together with reasons for choosing them and reasons for not choosing them. Each comes with a list of key evaluation questions the answers to which you will be expected to present as part of your dissertation. Amongst other things, the answers you give will justify how and why your work makes a contribution to knowledge. These evaluative questions in turn give you targets to aim for throughout your research, you will need to answer each of them – they will be the driver for your research and your writing up. 
%%
%For instance, social science \glsf{research-strategy} could be as simple as a two group experiment while the strategy\todo{Andrews calls it a research \glsf{model}.} described in \textcite[p.~407]{andrews2005place} for education research has 12 components, arranged in complex interacting feedback loops, with the of including experiments.
%%
%\begin{figure}[h!]
%\centering{
%\begin{tabular}{ccc}
%%%2025-09-26: Missing figures???
%%  \includegraphics[height=8cm]{Figures/SimpleExperiment.jpg}&\qquad\qquad\qquad\qquad&
%%  	\includegraphics[height=8cm]{Figures/andrews2005placefig1.jpg}
%\end{tabular}
%  \caption[Simple and complex research designs]{(left) A simple two-group experimental design~\parencite[, adapted p.~128]{marczyk2005essentials}; (right) A complex \glsf{research-strategy} for education research~\parencite[, figure 1]{andrews2005place}
%  \label{fig:simpleandcomplexres}
%  }}
%\end{figure} 
%
%Although each research paradigm is sufficiently distinct as to indicate different strategies, strategies do overlap in their application. Every strategy will, for instance, generate data of some form, whether this is readings in some experimental setting or documentation of the lived experience of a community under focus. Generated research data provide a focus for qualitative and quantitative analysis and, thus, to the \glsf{synthesis} of new knowledge.****
%
%
%Recall from~\Cref{stage1} that:
%%
%\begin{itemize}
%\item your \glsf{research-aim} tells the reader how your research will address the knowledge gap that you have found through your literature review, 
%\item your objectives break it down into 3 to 4 high-level goals you must reach to achieve the aim.
%\end{itemize}
%%
%You're now at the point where you need to think about how you will meet each of your research objectives. Research strategies are the recipes for good research, combining research tasks into meaningful ways of doing research to achieve those objectives.
%
%
%
%Developing a design for your research will help you summarise, explain, and justify how your research is conducted to your examiners and other readers of your dissertation. In addition, it will be a touchstone for you to refer to at times of difficulty and allow you to plot your progress against your objectives. 
%
%The \glsf{research-design} will, like most other aspects of your project, evolve: at the start, it will be a collection of your initial ideas and intentions; by the end, it will be a detailed account of what you have actually done. 
%
%Your \glsf{research-design} will depend on many factors, including the type of \glsf{research-problem} you are trying to address, the intended outcome of your research, the sort of \glsf{evidence} you will need, the resources and expertise you have, accepted research methods applied by other researchers in your field\footnote{And the philosophical beliefs which motivate them}. As you are a key participant in your own research, your personal views and values will also affect the choices you make while developing your \glsf{research-design}.
%
%\Glsf{research-design} is also a field of study in its own right, one which has grown out of many diverse academic traditions and ways of thinking across academic disciplines and subject areas, and which is still evolving\footnote{In this young research area, there is still a lot of post--rationalisation of a particular course of research as authors looks for generalisable themes.}. As such, it is not an easy topic to digest and is one of the most challenging aspects of doing \glsf{academic-research}. It can be puzzling for students embarking on \glsf{academic-research} for the first time.
%
%For this \glsf{reason}, in~\Cref{stage1}, we will not consider \glsf{research-design} in detail --- that will happen from~\Cref{stage2} instead. However, so that you can start to think about your \glsf{research-design}, in this section, we introduce a range of topics which concern \glsf{research-design} and should influence your follow-up work.
%

\subsection{Understanding methodology in articles you have reviewed}\label{ssect:understandingMethodology}

To ground what you have learnt so far on methodology, and research methods and strategies, you should look back at some at the articles you have reviewed to see how they describe and use them. This may also give you some ideas on how to apply them within your own project. You should be aware, however, that terminology used in the literature may different from that of this book\footnote{We have already noted how some terms are used differently by different authors.}.

\begin{example}{Considering methodology in published work} 
Looking back to the Curriculum Analytics example, one of the papers reviewed was:
\begin{quotation}
	\textit{Gray, G., Schalk, A. E., Cooke, G., Murnion, P., Rooney, P., \& O'Rourke, K. C. (2022). Stakeholders’ insights on learning analytics: Perspectives of students and staff. Computers \& Education, 187, p.104550.}
\end{quotation}

In this paper there is a \enquote{Methodology} section which states that a mixed-methods strategy was used to collect and analyse qualitative and quantitative data. 
For data collection both \glsplf{questionnaire} and a \glsf{focus-group} were used. For \glsplf{quantitative-datum} analysis, \glsf{thematic-analysis} was applied; for quantitative data analysis, \glsf{tables-and-charts} were used, including some calculation of differences in scores obtained from questionnaire answers.

The paper usefully includes some descriptions of the specific steps the researchers took to recruit participants and analyse the data, something that could be replicated in new studies.

There are differences in terminology:
\begin{itemize}
	\item mixed-method \enquote{approach} is used instead of mixed methods research strategy, and 
	\item \enquote{Survey} is used to mean a questionnaire: this is rather common in the literature, where the term survey is often found to mean both.
\end{itemize}
\end{example}


\begin{question}[subtitle={Activity: Considering methodology in articles you have reviewed}] Go back to two or three articles you have reviewed, perhaps those you have found most interesting or closest to the research you intend to do. 

Look for research methods and strategies they use, and consider how these are presented and applied. Try to establish links to what you have learnt so far, including noting any differences in the terminology used. If appropriate, write down specific points which may help you apply them in your own project.  
\begin{guidance}
%Hack to correct tcbox behaviour
\color{black}


%Hack to correct tcbox behaviour
\color{black}
As in the example, often articles include a \enquote{Methodology} or \enquote{Methods} section where research methods and strategies are discussed. That's the section you should start from. It may be, however, that more relevant content is described elsewhere, so also look for sections that summarise data or \glsf{evidence} collected and analysed. 

You may like to skim through few articles before deciding which ones to consider in detail.
\end{guidance}
\end{question}
%%Hack to correct tcbox behaviour
\color{black}



%%%%%%
%%%LR: not used here; content in~\Cref{stage3} instead
%%%%%%%%%
%We have covered quite a few methods in this section!~\Cref{tab:methodsSummary} gives you a concise summary, including a brief description of each. 
%
%Before moving on, you should about which of these methods may find some application in your project.
%
%\begin{question}[subtitle={ACTIVITY: Identify candidate research methods}] Consider the methods in the table and which you may be able to use in your project. For each, write down what you would use them for and why.
%
%\begin{guidance}
%In considering candidate methods you should think of the sort of data and \glsf{evidence} your project may need, as well as the demands of each method in terms of access to people and other resources, and the level of expertise required to apply them. 
%Their descriptions in the previous sections, which you may like to revisit, should be sufficient for you to carry on this activity. In~\Cref{stage3}, you will return to some of those methods, to investigate how to apply them in detail within your project.
%
%\end{guidance}\end{question}
%%%Hack to correct tcbox behaviour
%\color{black}
%
%


%%%JGH 2023-10-13: Commands to keep Activity text consistent across all research methods.
%\newcommand{\ActivityRMUse}[2][]{
%%%optional argument is addition information, should include initial \enquote{, } so fit in the sentence 
%%%mandatory argument is \glsf{research-method}
%\begin{question}[subtitle={Activity: Considering #2\todo[inline]{templated: check for sense}}]%
%Consider the extent to which #2{} could be used in your research project, and in which form#1. Write down your answer.
%%
%\begin{guidance}
%If you think #2 will be helpful, you might like to look at the resources at the end of this chapter. If you don't think it will be useful for your project, write down the \glsf{reason} – it might come in handy later.
%\end{guidance}\end{question}
%%%Hack to correct tcbox behaviour
%\color{black}
%}

%%LR - Nov 2023, replaced individual questions in method section and replaced with activity in summary section
%\newcommand{\ActivityRMUse}[2][]{}
%
%\subsubsection{Data collection methods}\label{sssect:DataCollectionMethods}
%These are methods that can be used to collect data and \glsf{evidence}, which might be qualitative, quantitative or both.
%
%\paragraph{\Glsplf{questionnaire}}
%A {questionnaire} is a fixed set of questions organised in a particular order used to gather answers. It can be delivered face-to-face or distributed to respondents to gather their answers. The respondents' answers constitute the generated data that is subsequently analysed by the researcher.
%
%\Glsplf{questionnaire} are a data collection technique applicable when:
%
%- you wish to obtain standardised data from many people
%
%- you seek relatively brief information from your respondents
%
%- you expect your respondents to be able to understand and interpret the questions in a straightforward manner.
%
%\Glsplf{questionnaire} can help collect both quantitative and \glsplf{quantitative-datum}, depending on their questions.
%They can be \textit{self-administered}, in the sense that the respondents complete the questionnaire without the researcher being present, or \textit{researcher-administered}, in which case the researcher asks the questions and writes down the responses.
%
%\ActivityRMUse{\glsplf{questionnaire}}
%
%\paragraph{Interviews and focus groups}
%An {\glsf{interview}} is a form of conversation between the researcher and one or more interviewees, designed by the researcher to gain insights and opinions on a specific topic. The researcher guides and controls the conversation and asks the questions. The interviewees' answers constitute the generated data that is subsequently analysed by the researcher.
%
%An \glsf{interview} is a technique for data generation applicable when you wish to:
%
%- obtain detailed information on a specific issue or topic
%
%- ask open-ended, complex questions, which may be tackled or interpreted differently by different interviewees
%
%- investigate sensitive issues or privileged information that interviewees may not be willing to commit to writing.
%
%Interviews are primarily used to collect \glsplf{quantitative-datum}. They can be one-on-one, between the researcher and one interviewee at a time, or can happen in a group, with several interviewees being interviewed together by the researcher. The latter is referred to as a {\glsf{focus-group}}.
%
%Interviews can be fully planned or quite open-ended. The former are termed \textit{structured} and use pre-determined, identical questions with all the interviewees, while the latter are termed \textit{unstructured} and typically start by introducing a topic but then let the interviewee talk freely around their ideas, experience and beliefs. Somewhere in between are \textit{semi-structured} interviews, where the researcher selects some themes and related questions upfront, but then adapt them depending on how the conversation with the interviewee develops.
%
%\ActivityRMUse{interviews}
%
%\paragraph{\Glsf{delphi} technique}
%With the {\Glsf{delphi}} technique, a group of experts are consulted with a view of obtaining a consensus on a particular issue or topic. It involves an iterative process of collecting, synthesising and circulating anonymous judgements from those experts to eventually arrive at a consensual view. More precisely, each subject expert is initially consulted separately by the researcher, who then anonymises and collates the group responses and circulate them to the same group of experts. The process is repeated until a consensus is reached.
%
%This technique is based on the idea that a group of people are more likely to arrive at an informed and valid position than an individual, with anonymity preventing interpersonal relationships from influencing the outcome. The judgements and consensus gathered constitute the generated data that is subsequently analysed by the researcher.
%
%The \Glsf{delphi} technique is particularly suited to situations in which the researcher wishes to improve their understanding of an under-explored problem or issue in order to inform decision-making.
%
%\ActivityRMUse{the \Glsf{delphi} technique}
%
%\paragraph{Observations and measurements}
%{Observations} are used in research to find out what people actually do or what actually happens in a particular context, rather than what has been reported about it. Observations can be of people's behaviour or interactions, e.g., observing a formal meeting in an organisation, or of events and processes, for instance observing a queue at the post-office or a computer-controlled production plant. As such, observations can generate all kinds of data, qualitative and quantitative. Quantitative observations are often referred to as {measurements}, e.g., the length of time a particular customer has spent waiting in the queue at the post office.
%
%There are two main types of research \glsf{observation}, \textit{systematic} vs. \textit{participant}. The former is when the researcher decides in advance what to observe, the schedule of observations and what to record. For example, the \glsf{observation} of a queue at the post office could be planned to take place over a certain week or month, and recordings may include time of arrival and departure of each customer, average and maximum length of the queue, average service time,~\etc{}.
%
%In participant observations, the researcher participates directly in the situation under study and produces a rich description of what happens based on what they experience. For instance, in relation to the previous example, the researcher might join the queue at the post office and record their experience in great detail, or even join the staff in the post office to understand why queues are longer or shorter for certain tasks.
%
%\ActivityRMUse{observations and measurements}
%
%\paragraph{Use of \glsf{secondary-evidence}}
%The previous techniques can be used to generate \glsf{primary-evidence}\footnote{As defined in Cref{sect:evidenceAndData}, primary evidences is newly generated during research, while \glsf{secondary-evidence} is already available from previous research.}.
%
%\Glsf{academic-research}, however, can also use \glsf{secondary-evidence} as its starting point. This can be represented by existing documents of many forms, from academic articles to documents found in organisations, e.g. laws, policies and procedures, reports, formal minutes of meetings, informal communications,~\etc{}. Similarly, there are plenty of publicly available data sets that can be used, created by academic communities or private and public organisations, such as business financial data or \glsplf{statistical-datum} from the UK Office for National Statistics or data from social network platforms. As already noted, the \glsf{academic-literature} at the core of your literature review is a form of \glsf{secondary-evidence}.
%
%\ActivityRMUse{reusing existing \glsf{evidence}}
%
%\subsubsection{Data analysis methods}\label{sect:dataAnalysisMethods}\label{sssect:DataAnalysisMethods}
%These are methods that can be used to analyse data and \glsf{evidence} after collection.
%
%\paragraph{Spreadsheets, tables, charts and graphs}
%They are the bread and butter of data analysis, are applicable to all kinds of data, and can be used to summarise and visualise data, and identify interesting patterns. You should be already familiar with this topic from your previous studies, so that this is only a brief overview to refresh your knowledge.
%
%A {spreadsheet} is a digital tool you can use to capture, display and manipulate data arranged in tables, that is arranged in rows and columns. Common spreadsheets include Microsoft Excel, Apple Numbers and Google Sheets. Spreadsheets are among the most used digital tools, so it is likely you are already familiar with at least their basic functionalities. Spreadsheets have become quite sophisticated tools, including all sort of charts and graphs, as well as programmatic capabilities which allow you to code quite complex data manipulation functions. Some of those advanced functionalities could be advantageous to your research, so it is worth spending some time considering what they can offer to your project. There are plenty of tutorials and other documentation online you can use to learn more.
%
%\ActivityRMUse{spreadsheets}
%
%
%Alongside spreadsheets a growing number of {data analytics} tools are also available: these are sophisticated digital tools which extend spreadsheet capabilities for collating and visualising data to include some degree of automated analysis, both statistical and based on \Glsf{machine-learning} algorithms. Tools like Tableau and Power BI are notable examples: both are available in free versions for community use and for study.
%
%\ActivityRMUse{data analytics tools}
%
%\paragraph{\Glsf{statistical-analysis}} 
%This refers to a broad collection of techniques used to investigate trends, patterns, and relationships in quantitative data. It is a well established field of study with wide application across all kinds of research, and well developed tool support. In particular, both spreadsheets and data analytics tools include functionalities which allow you to calculate statistical measures on data, check statistical relationships between variables and data sets, and generate basic statistical models. Bespoke statistical tools also exist for more advanced \glsf{statistical-analysis} and modelling, like IBM SSPS or Minitab, which are also available in free versions for students.
%
%Should your project require advanced \glsf{statistical-analysis}, then you will need to become proficient in good time to carry out your analysis and interpretation of findings in Stages 4 and 5 or your project.
%
%\ActivityRMUse{\glsf{statistical-analysis}}
%
%\paragraph{\Glsf{thematic-analysis}}
%This is a way of analysing \glsplf{quantitative-datum}, particularly texts, e.g., transcriptions of interviews or answers to \glsplf{questionnaire} or existing text documents, in order to find out something about people's views, opinions, knowledge,~\etc{}.
%
%At its core is the identification by the researcher of recurring themes, their definition and relationships: this relies on the researcher's judgement and it is quite subjective.
%
%\ActivityRMUse{\glsf{thematic-analysis}}
%
%\paragraph{\Glsf{content-analysis}}
%This is used to identify patterns in non-numerical content used for communication, whether text, speech, images, videos, or other. For instance, in can be used to investigate certain words, themes, or concepts within that content.
%
%It can be either quantitative, where the focus is, for instance, on counting occurrences, or qualitative, where the focus is on interpreting and understanding meaning and relationships. As such it can be used for many purposes, from discovering and understanding patterns, to looking at intentions behind what is expressed, or to highlighting differences of use in different contexts. 
%
%\ActivityRMUse{\glsf{content-analysis}}
%
%\paragraph{\Glsf{discourse-analysis}}
%It focuses on the use of language in conversations within their real-world context. In the analysis, it is essential to consider the influence of history, culture and power dynamics within that context.
%
%\ActivityRMUse{\glsf{discourse-analysis}}
%
%\paragraph{\Glsf{narrative-analysis}}
%It focuses on stories which are told by people. The focus is on listening to such stories and how they are told to investigate their meaning, particularly how people make sense of reality.
%
%\ActivityRMUse{\glsf{narrative-analysis}}
%
%\subsubsection{Modelling methods}\label{sssect:ModellingMethods}
%At its essence, a \textit{\glsf{model}} is an \glsf{abstraction} or representation of something, be that a \glsf{system}, a structure or a behaviour. Modelling is used across many disciplines, so a vast repertoire of modelling techniques exist.
%
%Possibly the most important thing you must remember about modelling is expressed by the following oft-cited aphorism\footnote{Box, George E. P. (1976), \enquote{Science and statistics} (PDF), Journal of the American Statistical Association, 71 (356): 791--799.}:
%
%\begin{quotation}
%	\textit{All models are wrong, some are useful} (Box, 1976)
%\end{quotation}
%
%which makes clear that a \glsf{model} should not be regarded as a faithful replication of some reality, but as a tool to investigate some aspects of that reality.
%
%In this section, you will read about a small set of modelling techniques, which are particularly relevant in Master projects. A lot more can be found in the \glsf{academic-literature} and beyond.
%
%
%\paragraph{Systems diagrams}
%You can use {systems diagrams} to help you understand the structure of a situation of interest that can be rendered as a \glsf{system}. The term \enquote{\glsf{system}} is meant in its widest possible meaning of a set of components interconnected for a purpose. This is a very general and versatile technique that you can apply to all sorts of real-world situations. If you have studied a systems thinking and practice module for your \glsf{qualification}, you will be already familiar with this technique.
%
%There are many different kinds of systems diagrams\footnote{You can find links to useful tutorials in Cref{sect:readingList}.}. For examples, \textit{systems maps} allow you to sketch the structure of a \glsf{system} by identifying key components and sub-systems. They can be extended to show how those elements influence each other, in which case they are called \textit{influence diagrams}. On the other hand, \textit{causal loop diagrams} are used to capture cause-and-effect relations in a \glsf{system}, hence \glsf{model} certain dynamics of that \glsf{system}, particularly underlying feedback structures. They can be turned into \textit{stock and flow diagrams} by adding quantitative information, so that this type of diagram is useful both for analysis and \glsf{simulation} of systems behaviour.
%
%
%\Glsf{system} diagrams have an accepted structure, format and notation but what you choose to describe and include within a \glsf{system} and its components will depend on your own viewpoint. Systems diagrams can be shared with others as learning devices to promote more understanding of a situation.
%
%\ActivityRMUse{systems diagrams}
%
%\paragraph{UML modelling}
%UML ({Unified Modeling Language}) is a graphical language for visualising, specifying or documenting various artefacts in the process of developing software systems. If you have studied a software engineering module for your \glsf{qualification}, you may be already familiar with UML.
%
%UML can be used as a \enquote{sketching} language, to capture elements of systems informally, like you can do with systems diagrams, or as a \enquote{blueprinting} language to specify precisely how elements of a software \glsf{system} will be developed. Different kinds of UML diagrams exist, so that you can \glsf{model} elements of software systems both in terms of their structures and behaviours, and their interactions with end-users.
%
%In a master's project, UML can be used to help you understand existing systems in context, or plan the development of new innovative artefacts.
%
%\ActivityRMUse{UML modelling}
%
%\paragraph{Problem diagrams}
%These have their roots in software requirements engineering as a diagrammatic technique to capture requirements in a real-world context to inform the specification of a new software \glsf{system} to satisfy them. They have been subsequently generalised for application to general engineering problems for which some novel solution artefact is to be developed in a real-world context, to guide design and ensure fitness-for-purpose. Problem diagrams span the problem-solution space divide by focusing on those \glsplf{phenomenon} that characterise a problem and constrain its solution.
%
%In the context of master's projects, problem diagrams can help you develop a good understanding of real-world requirements in context and explore both constraints and effects of designing new systems for that context.
%
%\ActivityRMUse{problem diagrams}
%
%\paragraph{Statistical modelling}
%Many {statistical modelling techniques} exist. The most commonly applied include those used to \glsf{model} relations between variables, e.g., how crop yields relate to environmental factors, such as soil quality or meteorological conditions, or to \glsf{model} real-world processes, e.g., the spreading of a disease in a \glsf{population}. Once a statistical \glsf{model} is defined, it can then be used to make predictions of what might happen in the real world.
%
%As per advanced \glsf{statistical-analysis}, statistical modelling requires well developed statistical knowledge and skills, which you should already possess, or have the time to develop, if you are considering their use in your master's project.
%
%\ActivityRMUse{statistical modelling}
%
%\paragraph{ML modelling}
%\Glsf{machine-learning} (ML) models are computational programs which can identify patterns in large data sets and use them to perform a great variety of tasks usually associated with human cognition, like recognising images or language, classifying objects, or generating speech or texts, to name just a few current applications. 
%
%They are increasingly applied to all sort of data-rich problems, so that several ML models are readily available for use, embedded in computational library\footnote{For instance, the \textit{Scikit-Learn} library for the Python programming language.}. 
%
%ML modelling tools require highly developed technical knowledge and skills, which you should already possess, or have the time to develop, if you are considering their use in your master's project.
%
%\ActivityRMUse{ML modelling}
%
%\paragraph{Prototyping}
%
%A prototype is an early version of an artefact, which can be used to test early design ideas or properties before implementation. As such they can present various degrees of fidelity in relation to the end artefact.
%
%Prototypes are used in many contexts, particularly architecture, design, engineering or computing.  In \glsf{academic-research}, they can be useful to explore both problems and their intended solutions. 
%
%Data generated from prototyping can be either qualitative or quantitative, depending on the nature of the prototype and its intended use. For instance, the prototype of a digital app for a smart phone may be used to evaluate some usability properties with end users, which may generate \glsplf{quantitative-datum}, or to measure computational speed and efficiency, which would generate quantitative data. 
%
%\ActivityRMUse{prototyping}
%
%\paragraph{Simulations}
%
%A \glsf{simulation} is something that mimics a situation, a process or the behaviour of \glsf{system}. They may be used to make predictions, for education and learning as well as for exploration and discovery. For instance, weather simulations are used to inform forecasts, while simulations of crisis scenarios, say a fire or an earthquake, can be used to train people. 
%
%Computer simulations are widely used in research, and many types exist. Among the most common are: 
%\begin{description}
%	\item [\glsf{system} dynamics:] which usinge continuous mathematical models to capture the dynamic behaviour of complex systems. Financial market simulations are often based on \glsf{system} dynamics.
%	\item [agent-based:] which use collectives of interactive agents, whose behaviour can be programmed, to explore emergent properties of a \glsf{system}. Simulations of natural ecosystems are often based on multi agents, e.g., to study the balance between preys and predators over time, in which agents represent each the behaviour of ether a prey or a predator. 
%	\item [statistical simulations:] which  apply statistical models to mimic a process or a \glsf{system}. The \glsf{spread} of diseases in a \glsf{population}, for instance, is often simulated statistically.
%\end{description}
%
%\ActivityRMUse{simulations}
%
%\subsubsection{Summary of methods}\label{sssect:SummaryOfMethods}
%We have covered quite a few methods in this section!~\Cref{tab:methodsSummary} gives you a concise summary, including a brief description of each. 



%\paragraph{\Glsf{survey-research}}
%This aims to gain insights which are valid across a target \glsf{population}, by collecting data from a predefined \glsf{sample} in a standardised and systematic way.
%
%A typical application of \glsf{survey-research} is to predict the outcome of an upcoming general election by polling data from a representative \glsf{sample} of voters.
%
%For your data collection, you need to identify upfront which data you will collect in a standardised matter, your target \glsf{population} and \glsf{sample}. So \glsplf{questionnaire} or structured interviews are usually used for data collection.
%
%In your data analysis, you seek patterns in the \glsf{sample} data collected to arrive at generalisations to the wider \glsf{population}. \Glsf{statistical-analysis} is usually applied, possibly complemented by some \glsf{thematic-analysis}, if open-ended questions are also included.
%
%For \glsf{survey-research} to be successful, you must be able to access an appropriate \glsf{sample} and generate a sufficient volume of data.
%
%The advantages of this strategy are that it can produce a lot of data in a relatively short time, and you can replicate your data collection process on different samples or on the same \glsf{sample} at a later time. However, among its disadvantages are the depth in the data that can sometimes be lacking, its focus on what can be measured, the fact that it cannot reveal cause-and-effect relationships, and can only provide a snapshot at a particular time.
%
%\ActivityRMUse{\glsf{survey-research}}
%
%\paragraph{\Glsf{experimental-research}}
%This is used to investigate cause and effect relationships between factors by testing \glsf{hypothesis} or proving/disproving causal links.
%
%For instance, you may run an experiment to test ways in which the use of mobile phones just before going to sleep affect people's sleeping patterns.
%
%There are two main kinds of experiments: \textit{laboratory experiments}, which are carried out in closed environments, such as a laboratory; and \textit{field experiments}, which are conducted in the \enquote{real world}. Laboratory experiments are often applied in engineering and computer science research, while {field experiments} are usually applied when people are involved.
%
%Possibly the best known kind of field experiment are clinical trials, widely applied in medicine. However, field experiments are also very popular in research which investigates technology in its social context or application of use.
%
%In experiments, first you would need to state the \textit{\glsf{hypothesis}} to be tested: this is a tentative statement about the relationship between \glsplf{phenomenon} to be tested in the experiment. In the example above, a \glsf{hypothesis} to be tested might be that \enquote{the blue light emitted by a mobile phone reduces the production of melatonin.} As melatonin is the hormone which controls a person's sleep-wake cycle, its reduction is likely to disrupt a person's sleeping pattern. After formulating the \glsf{hypothesis}, you would then make detailed observations and measurements of outcomes, e.g., the amount of melatonin released by the body, and any changes that take place when particular factors are introduced or removed, e.g., the length of exposure to the blue light.
%
%In analysing your experimental data you seek to explain causal links between factors under study, looking at your observations and measurements under different experimental conditions. \Glsf{statistical-analysis} is widely used for data analysis.
%
%For experiments to be successful you must be able to control factors which can affect the outcome. This is possible in laboratory experiments, while the level of control in field experiments is diminished.
%
%\Glsf{experimental-research} has well established processes and protocols and is particularly well suited to the consideration of cause-and-effect relations. However, it has its pros and cons. Laboratory experiments are very reliable due to the high level of control, but can be very artificial, with little or no relation to a real-world context. The opposite is true for field experiments.
%
%\ActivityRMUse{\glsf{experimental-research}}
%
%\paragraph{\Glsf{design-science-research}} This seeks to generate new knowledge about a significant problem or its solution via the design of an artefact. It simultaneously generates knowledge about the problem, the artefact and the method used to design it. Artefact indicates anything made by humans, so this is a very broad definition, encompassing all that does not exist in nature.
%
%Lots of research in Computing is an expression of design science, for instance designing new algorithms able to emulate human cognition.
%
%More than data collection and analysis, in design science you need to follow a process of articulating the problem, and designing, constructing and evaluating a solution artefact. In doing so, you shed new insights on the problem, and argue how the solution and solution process contribute new knowledge. As a result, modelling techniques are widely applied, possibly informed by data collection techniques, like reviewing existing documents or interviews with stakeholders and experts. Prototyping is often used to produce proof-of-concept artefacts to test or demonstrate the design.
%
%For \glsf{design-science-research} to be successful you must be able to argue that it is not \enquote{normal} design, that is you are not simply re-implementing a solution to a well-known problem through a well-known development process.
%
%An advantage of \glsf{design-science-research} is that it leads to tangible artefacts which fit real-world contexts, and it is particularly suited to emerging and rapidly changing technology-related fields of study, where new problems emerge all the time and known solutions are sparse or become rapidly obsolete. The latter is also a disadvantage, of course, as new solutions may be short-lived. Also, it may be difficult to generalise outcomes to different real-world settings. Depending on the nature of the artefact being designed, advanced technical skills may be required.
%
%\ActivityRMUse{\glsf{design-science-research}}
%
%\paragraph{\Glsf{case-study-research}}
%A {case study} can be used to investigate in great depth a notable instance of what is under study, in its real-world context. Case studies focus on the \enquote{how?} and \enquote{why?}, and what you seek can span from exploring possible questions or \glsplf{hypothesis} for follow-up research, to providing a detailed account of a \glsf{phenomenon} in its natural context, to explaining why certain outcomes or \glsplf{phenomenon} have occurred.
%
%For instance, an example of case study could be a detailed investigation of the US Equifax social security breach of 2017, in which 143 million of their consumer records were stolen by hackers. This may be descriptive of the chain of events that took place or explicative of why things happened the way they did.
%
%Case studies require you to collect data from a great variety of sources, and to focus on depth rather than breadth. Therefore, all data collection techniques which allow you to do so may be used, from interviews to observations to studying existing documents forensically. This will lead to much \glsplf{quantitative-datum}, so that qualitative methods are often needed for the analysis of the \glsf{evidence}.
%
%For a case study to be conducted successfully you must be able to analyse the chosen instance holistically and in its real-world context.
%
%Case studies allow you to study a complex situation where several factors are at play, and to explore alternative meanings and explanations. However, case studies are time-consuming, difficult to perform rigorously and with limited \glsf{generalisation} beyond the particular instance under study.
%
%\ActivityRMUse{\glsf{case-study-research}}
%
%\paragraph{Systemic inquiry}
%This is used to explore complex, messy problematic situations involving multiple and often contrasting perspectives, with the aim of transforming the situation for social improvement. Systemic inquiry is based on concepts and principles of systems thinking and systems practice.
%
%Situations for systemic inquiry can range from local to global. So, it may equally apply to exploring changes in practice within a local organisation, and to international responses to disruptive events such as climate change. Of course, it is highly unlikely that your master's project will tackle a situation at a global scale!
%
%In systemic inquiry, you must be able to articulate your personal stake in the situation, for example, a deeply felt interest or active involvement, rather than assuming and claiming unbiased passive \enquote{neutral} \glsf{observation}. You must also keep your own journal during the course of your research inquiry, tracking changes in your own viewpoint and how you adapted your research as a result. In some sense, a systemic inquiry is a conceptualisation of your own learning \glsf{system} and how it adapts to change during the research. Therefore, a successful systemic inquiry should demonstrate \textbf{\glsf{reflexivity}} -- reflecting on your own changing viewpoint and impact on the wider research situation. In fact, systemic inquiry emphasises \glsf{reflexivity} and building trust relationships with stakeholders, in order to make sense of complex situations of change and uncertainty.
%
%To conduct your systemic inquiry you must articulate your \glsf{research-problem} with reference to one or more systems which concern the problematic situation under study, and frame your research in terms of possible systems change. You must also have access to sources of different perspectives on the situation under study in order to generate your \glsf{evidence}. This may include both \glsf{primary-evidence} from people involved or affected by the situation, and \glsf{secondary-evidence} from official and \glsf{grey-literature}\footnote{As explained in from Cref{sect:howToAccessTheLiterature}, \glsf{grey-literature} refers to information produced by organisations other than commercial publishers, such as academia, government bodies, or non-publishing businesses and industries, and can include pre-publication and non-peer-reviewed articles, theses and dissertations, research and committee reports, government reports, conference papers, accounts of ongoing research,~\etc{}} associated with the situation; your own research journal will also be a source of \glsf{evidence}. In terms of methods, a systemic inquiry is primarily a qualitative endeavour, so you can apply any methods that deal with \glsplf{quantitative-datum}. Distinctively, you can complement them with other tools and techniques which you may have developed through your own experience and professional practice: this is known as \textit{bricolage} research\footnote{Kincheloe, J. L. (2011). Describing the bricolage: Conceptualizing a new rigor in qualitative research. In Key works in critical pedagogy (pp. 177--189). Brill.}.
%
%\ActivityRMUse{systemic inquiry}
%
%\paragraph{\Glsf{systematic-review} research}
%This is used to generate new insights from published work. A {\glsf{systematic-review}} is a literature review linked to a clearly defined \glsf{research-problem} or question. It uses a rigorous set of criteria to identify, select, and critically appraise relevant research from previously published studies in order to generate a scholarly \glsf{synthesis} of the \glsf{evidence} in relation to that problem or question. Such a \glsf{synthesis} is meant to advance a field of study.
%
%For example, a \glsf{systematic-review} of randomised controlled trials on the effectiveness of a specific medical treatment could be used to advance evidence-based medicine.
%
%In a \glsf{systematic-review} you only use \glsf{secondary-evidence} from published studies. You must decide upfront your \glsf{research-problem}/question and the set of criteria you will use to select, summarise and evaluate those studies. The type of analysis you will conduct will depend on the nature of the \glsf{evidence} you are considering and combining. In \textit{narrative reviews}, a narrative \glsf{synthesis} is produced, while in \textit{\glsf{meta-analysis}}, statistical techniques are used to analyse and combine results.
%
%To be successful, a \glsf{systematic-review} has to be both systematic and extensive, which requires the researcher to have a very good grasp of the subject area in order to establish appropriate criteria and make a novel contribution to knowledge.
%
%Because of their explicit set of criteria, systematic reviews are considered transparent, reliable, and easy to replicate. However, they can be very time-consuming due to the large body of work to review. Also, in striving to piece together \glsf{evidence} from potentially very different studies, they may obscure important differences. Narrative reviews may also be subject to \glsf{bias}.
%
%\ActivityRMUse{\glsf{systematic-review} research}
%
%
%\paragraph{\Glsf{ethnography}}
%This is used to study the culture of a group of people in their natural setting, and was originally developed within the discipline of anthropology. 
%
%It requires the researcher to characterise the culture being study by making detailed observations, gathering and recording detailed data, reflecting on what they have learnt, linking it to the existing literature. 
%
%The researcher is required to join the group in order to gain an insider's perspective by sharing what the group members' experience: the resulting cultural characterisation should therefore be one that the group members recognise and find familiar. This characterisation should be inclusive of various cultural facets, social and economical, rather than focusing on one specific aspect.
%
%\Glsf{ethnography} can lead to rich descriptions of complex social settings. However, it is very time consuming and demanding in terms of quantity of \glsf{evidence} to produce. The dual observer-participant role of the researcher can also make it hard to maintain an unbiased stance. Also, while the characterisation produced may be very deep in representing a particular group culture, it may be difficult to generalise to other social groups or settings. Because of these characteristics, \glsf{ethnography} is seldom used at master's research.
%
%\ActivityRMUse{\glsf{ethnography}}
%
%\paragraph{\Glsf{action-research}}
%This is used primarily to improve the researcher's own professional practice, focussing on practice change, and continuous learning and improvement via iterative \enquote{\glsf{plan-act-reflect-cycle}} cycle. As such, it is often applied with the education discipline. 
%
%The researcher is an active participant in the research, rather than solely an observer, alongside other collaborating practitioners: in fact, \glsf{reflection} and collaboration are two key elements of this strategy.  
%
%The research outcomes should make both a contribution to knowledge \textit{and} to practice. For instance, it is possible for new theories or methods to be outcomes of \glsf{action-research}, which could be more generally applicable, alongside their direct implementation to improve practice within a specific professional setting. 
%
%\Glsf{action-research} can bring immediate professional benefits, hence may have direct impact on practice. However, its application is constrained by the need to involve practitioners as collaborators in the research. Also, the researcher's own professional stake and involvement may increase the risk of personal \glsf{bias} distorting the research and its outcomes.
%
%
%\ActivityRMUse{\glsf{action-research}}
%
%
%\paragraph{Mixed-methods research}
%This combines quantitative and qualitative research to gain different perspectives on \glsplf{phenomenon} of interest, by exploring connections and contradictions between quantitative and \glsplf{quantitative-datum}\footnote{Mixed-method research should not be confused with \textit{\glsf{multi-method-research}}, which simply indicates the use of many methods, possibly all qualitative or quantitative.}.
%
%For instance, in looking at acceptance of a new technology, mixed-methods research could consider both levels of adoption and demographics, and the reasons behind adoption or otherwise, possibly to inform further development of the technology.
%
%Data collection and analysis will depend on the particular combination of methods selected. An important aspect is the consideration of how connections between findings are established, through comparing and contrasting data from the different methods applied. This is also referred to as \textit{\glsf{triangulation}}.
%
%The main advantage of mixed-methods research is that it can provide a more holistic understanding of the \glsplf{phenomenon} under study, and facilitate different avenues for exploration. It is particularly suited to situations in which neither quantitative nor qualitative methods alone can provide sufficient insights. However, mixed-methods make \glsf{research-design} more complex and demanding in terms of execution time, skills required and data variety to handle and analyse.
%
%\ActivityRMUse[they should be combined]{\glsf{mixed-methods-research}}
%

%\begin{table}[htbp]
%\caption{Research strategies introduced in this section\label{tab:resstrat}}
%\centering
%\small
%\begin{ltabulary}{\textwidth}{@{}p{2cm}LLLLLL@{}}
%\toprule
%
% \textbf{name} & \textbf{aim} & \textbf{data collection} & \textbf{data analysis} & \textbf{success factors} & \textbf{advantages} & \textbf{disadvantages} \\
%\midrule
%
% \textbf{\glsf{survey-research}} & to gain insights which are valid across a target \glsf{population}, by collecting data from a predefined \glsf{sample} in a standardised and systematic way & you need to identify upfront which data you will collect in a standardise matter, your target \glsf{population} and \glsf{sample} & you seek patterns in the \glsf{sample} data collected to devise generalisations to the wider \glsf{population} & you must be able to access an appropriate \glsf{sample} and generate a sufficient volume of data & - can produce a lot of data in a short time - data collection can be replicated on different samples, or the same \glsf{sample} on a later time & * lack of depth * focus on what can be measured * provides a snapshot at a particular time, rather than a longitudinal view * can't reveal cause-and-effect relationships \\
% \textbf{experiment\-al research} & to investigate cause and effect relationships between factors by testing \glsf{hypothesis} or proving/disproving causal links & you need first to state the \glsf{hypothesis} to be tested, then make detailed observations and measurements of outcomes and any changes that take place when particular factors are introduced or removed & you seek to explain causal links between factors under study, looking at your observations and measurements under the different experimental conditions & you must be able to control factors which may affect the outcome. This is possible in laboratory experiments, while the level of control in field experiments is diminished. & * there are well established processes and protocols * tailored to the study of causal relations & * Laboratory experiments are very reliable due to the high level of control, but can be very artificial, with little or no relation to a real-world context * The opposite is true for field experiments \\
% \textbf{\glsf{design-science-research}} & to generate new knowledge about a significant problem or its solution via the design of an artefact. It simultaneously generates knowledge about the problem, the artefact and the method used to design it. By artefact is meant anything made by humans, so this is a very broad definition encompassing all that does not exist in nature & you need to both articulate the problem, and design, construct and evaluate the solution artefact & you need shed new insights on the problem, and argue how solution and solution process contribute new knowledge & you must be able to argue that it is not \enquote{normal} design & * leads to tangible artefacts which fit a real-world context * is particularly suited to emerging and rapidly changing technology-related fields of study & * might be difficult to generalise to other real-world settings * may require advanced technical skills * may lead to short shelf-life of the research, particularly in technological volatile fields of study where technology becomes quickly obsolete \\
% \textbf{\glsf{case-study-research}} & to investigate in great depth a notable instance of what is under study, in its real-world context & you must be able to articulate your personal stake in the situation, for example, a deeply felt interest or active involvement , rather than assuming and claiming unbiased passive \enquote{neutral} \glsf{observation}. You must also keep your own journal during the course of your research inquiry, & you may seek to explore questions or \glsplf{hypothesis} for follow-up research, or provide a detailed account of a \glsf{phenomenon} in its natural context, or explain why certain outcomes or \glsplf{phenomenon} have occurred & you must be able to analyse the significant instance holistically and in context & * allows the study of a complex situation where several factors are at play * allows the researchers to explore alternative meanings and explanations & * can be time-consuming and access may be difficult to obtain * may be perceived as lacking rigour * insights may be difficult to generalise \\
% \textbf{systemic inquiry} & to explore complex, messy situations involving multiple and often contrasting perspectives, with the aim of transforming the situation for social betterment. & - bricolage research applies, in which you can complement traditional research methods with other tools and techniques from your existing repertoire of expertise and professional tradition - your research journal contributes \glsf{evidence} & * mainly a qualitative endeavour * bricolage research applies, in which you can complement traditional research methods with other tools and techniques from your existing repertoire of expertise and professional tradition & * you must articulate your \glsf{research-problem} with reference to one of more systems of interest, and frame the research in terms of possible systems change. * you must articulate your personal stake in the situation * You must also have access to sources of different perspectives on the situation under study in order to generate your \glsf{evidence} & * helps you make sense of complex situations of change and uncertainty * support development of trust amongst participants, including trust with the researcher in co-exploring the situation & - acknowledges that the research will not deliver \enquote{certainty} in terms of \enquote{problem-solving} associated with complex situations \\
% \textbf{\glsf{mixed-methods-research}} & to gain different perspectives on \glsplf{phenomenon} of interest, by exploring connections and contradictions between quantitative and \glsplf{quantitative-datum} & will depend on the particular combination of methods selected & a key aspect is consideration of how connections between findings are established, through comparing and contrasting data from the different methods applied & you must be able to apply competently different kinds of methods & * can provide a more holistic understanding of the \glsplf{phenomenon} under study, and facilitate different avenues for exploration * particularly suited to situations in which neither quantitative nor qualitative methods alone can provide sufficient insights & * add complexity to the \glsf{research-design} * can be demanding and time consuming \\
% \textbf{\glsf{systematic-review} research} & to generate a scholarly \glsf{synthesis} of \glsf{evidence} in relation to a specific \glsf{research-problem} or question & you need to establish and apply a rigorous set of criteria to identify, select, and critically appraise relevant research from previously published studies & you need to generate a critical \glsf{synthesis} of \glsf{evidence} based on the selected set of criteria & - your review must be both systematic and extensive - you need to be a skilled critical thinker and academic writer & - is considered transparent, reliable, and easy to replicate & - can be very time-consuming * is only as reliable as the studies reviewed * can be difficult to synthesise findings from potentially very different studies \\
%\bottomrule
%\end{ltabulary}
%\end{table}


%\subsection{Philosophical traditions}\label{ssect:PhilosophicalTraditions}
%Research methods and research strategies are strongly related to {philosophical traditions}\footnote{The term \emph{research paradigm} is also used in the literature with a similar meaning.}, which are world-views that inform how one should conduct research. Philosophical traditions may sound a bit esoteric, but they matter in that they make explicit assumptions behind \glsf{research-design} choices, influencing what a researcher chooses to research and the way they may go about collecting \glsf{evidence} or interpreting findings.
%
%Each philosophical tradition embodies a set of beliefs around three fundamental philosophical issues:
%
%- The nature of our world, which relates to questions such as: What is there? What kind of categories do things belong to? How are those categories related? The part of philosophy dealing with these questions is called \textit{\glsf{ontology}}. In \glsf{research-design}, \glsf{ontology} determines which \glsplf{phenomenon} are there to be studied as part of the research, and underlies our experience of the world. Hence, \glsf{ontology} is closely connected with the kind of observations we make or \glsf{evidence} we gather.
%
%- How knowledge is acquired, which relates to questions such as: What does it mean to know something? How can one \glsf{claim} to know something? What makes a belief justified? The part of philosophy dealing with these questions is called \textit{\glsf{epistemology}}. In \glsf{research-design}, \glsf{epistemology} is closely related to research methods for knowledge creation and validation.
%
%- What are the values, especially in relation to \glsf{ethics}, which relates to questions such as: What is good or bad? What is right or wrong? Where do values come from? How do we justify our values? The part of philosophy dealing with these questions is called \textit{axiology}. In \glsf{research-design}, axiology is closely related to ethical considerations when planning or executing research.
%
%In what follows, you will find a brief introduction to some of the better known and most often cited traditions. However, you should be aware that their definitions are not universal, their boundaries not clear-cut, and it is very rarely the case that a \glsf{research-design} will fit a specific tradition neatly. You should, instead, consider each of these traditions as a \enquote{wrapper} of convenience for a set of beliefs on research practice which have emerged from different disciplines and cultures, and also be aware that such beliefs have changed over time, and continue to do so.
%
%\paragraph{\Glsf{positivism}}
%This is perhaps the oldest tradition, with roots in the natural sciences. It sees the world as ordered and regular, with universal laws governing its functioning, and assumes it can be investigated objectively.
%
%Specifically, \glsf{positivism} encompasses the following set of beliefs:
%
%\begin{itemize}
%\item there is a physical world which exists \enquote{out there} and can be observed and measured. This also implies that all researchers will observe and measure the same \glsplf{phenomenon} in exactly the same way.
%
%\item through observations and measurements, the researcher can produce models of how the world functions, which are \enquote{true} explanations of the aspects of the world under study. This also implies that only one true explanation exists.
%
%\item truths about the world are perfectively objective and independent of the researcher's values or beliefs. This means that all researchers will arrive at the same truth.
%
%\item research is based on the empirical testing of theories or \glsf{hypothesis}, leading to either confirmation or rejection (AKA \enquote{refutation}). As there can only be one truth, either the \glsf{theory} or \glsf{hypothesis} tested is that truth, in which case all subsequent tests will confirm it, or it is not that truth, in which case at some point a test will reject it. The term refutation is used to indicate that a truth, albeit universal, is always tentative: it will be valid until somebody comes up with a test to reject it.
%
%\item research seeks universal laws and irrefutable facts. This means that re-testing such laws or facts should always confirm them, if they are indeed truths.
%
%\end{itemize}
%
%For instance, starting with the \glsf{hypothesis} that \enquote{all swans are white}, a positivist researcher would set as a test to look for swans and observe their colour. If all swans are seen white, then the \glsf{hypothesis} would be confirmed, if not, then it would be rejected. If the \glsf{hypothesis} is confirmed, then the truth that \enquote{all swans are white} is added to the body of knowledge and will remain so until another test will lead to a rejection --- indeed that's what English people believed until they first spotted a black swan in Australia!
%
%\begin{question}[subtitle={Activity: Summarising \glsf{positivism}}] Given these beliefs, what does \glsf{positivism} assume of the nature of the world (\glsf{ontology}), how knowledge is acquired (\glsf{epistemology}), and what is of value in research (axiology)?
%
%\begin{solution}\Glsf{ontology}: the world exists independently of the researcher, and can be observed and measured objectively.
%
%\Glsf{epistemology}: there are universal truths, which can be acquired by empirical testing of theories and \glsf{hypothesis}. Tests can lead to either confirmation or rejection. Confirmed theories and \glsf{hypothesis} are added to the body of knowledge.
%
%Axiology: \glsf{positivism} values objectivity above all, and dismisses individual's subjective views or experience.
%\end{solution}\end{question}
%%%Hack to correct tcbox behaviour
%\color{black}
%
%\Glsf{positivism} has attracted criticism particularly from the social sciences, which consider some of its beliefs untenable, primarily that researchers are totally objective and not influenced by their own values and beliefs, or that knowledge is made of perfectly generalisable truths. This has led to other traditions, which we consider next.
%
%\paragraph{Interpretivism/\Glsf{constructivism}}
%With its roots in the social sciences, {interpretivism} seeks to identify, explore and explain \glsplf{phenomenon} in social settings, acknowledging that people perceive the world in different ways, mediated by their beliefs, attitudes and values.
%
%Specifically, interpretivism encompasses the following set of beliefs:
%
%\begin{itemize}
%\item different individuals, groups or cultures perceive the world differently and what people consider real is a construction of their mind --- leading to the term \textit{\glsf{constructivism}} also being used.
%
%\item the researcher is not neutral, and their perceptions of the world are influenced by their values or beliefs. This implies that different researchers can perceive the same \glsplf{phenomenon} in different ways, and there is no single truth or single explanation of the world.
%
%\item as there are different perceptions of reality, communication among groups of individuals is the only way of constructing some shared meaning or understanding, and this will change over time.
%
%\item as researchers are influenced by their own values and beliefs, they will arrive at different interpretations as a result of their observations. The strengths of their interpretations will depend on the strengths of the \glsf{evidence} and arguments their interpretations are based upon.
%
%\item research is based on studying people and other \glsplf{phenomenon} in their \enquote{natural} context. Such a context can be unique, so that interpretations based on observations may not be generalisable to other contexts.
%
%\end{itemize}
%
%\begin{question}[subtitle={Activity: Summarising interpretivism/\glsf{constructivism}}] Given these beliefs, what does interpretivism assume of the nature of the world (\glsf{ontology}), how knowledge is acquired (\glsf{epistemology}), and what is of value in research (axiology)?
%
%\begin{solution}\Glsf{ontology}: the researcher acknowledges that they perceive the world based on their belief, values and culture.
%
%\Glsf{epistemology}: the researcher will offer interpretations based on observations in a social context. Different researchers may offer different interpretations. All knowledge is constructed and shared understanding is reached through communication. Interpretations in one context may not be generalisable to other social contexts.
%
%Axiology: The researcher's values and beliefs matter. The strength of their interpretations will depend on the strengths of the \glsf{evidence} and arguments in support.
%
%\end{solution}\end{question}
%%%Hack to correct tcbox behaviour
%\color{black}
%
%\paragraph{\Glsf{critical-theory}}
%Perhaps not as well established as the previous traditions, {\glsf{critical-theory}} originated in the fields of sociology, philosophy and political \glsf{theory}.
%
%Like interpretivism, it assumes multiple interpretations of reality in social contexts. However, it goes a step further by asserting that reality is shaped by those who are powerful, who legitimate particular ways of perceiving the world: \enquote{truth} is inherently political, defined by those in charge to the disadvantage of many, and challenged by those who wish to promote \glsf{equality}.
%
%As a result, critical researchers seek to challenge the status quo and perceive research as transformative at a social level, confronting ideology and trying to discover and challenge the mechanisms through which exploitation and disadvantage are perpetuated in society.
%
%\begin{question}[subtitle={Activity: Summarising \glsf{critical-theory}}] Given these characteristics, what does \glsf{critical-theory} assume of the nature of the world (\glsf{ontology}) , how knowledge is acquired (\glsf{epistemology}), and what is of value in research (axiology)?
%
%\begin{solution}\Glsf{ontology}: reality is the product of power relations, shaped by those who are powerful and there are disadvantages for many.
%
%\Glsf{epistemology}: the researcher confronts ideology and tries to discover the truth of exploitation and the mechanisms by which disadvantage is perpetuated to challenge the status quo and promote social justice and \glsf{equality}.
%
%Axiology: The researcher has the moral responsibility to make things better in society.
%
%\end{solution}\end{question}
%%%Hack to correct tcbox behaviour
%\color{black}
%
%\paragraph{\Glsf{indigenous}}
%The traditions described so far are attracting increasing criticisms in that they are seen as Western-European centric and often imposed on other \glsf{indigenous} cultures as a result of colonialism.
%
%In counterposition, an \glsf{indigenous} research tradition has emerged with a social and political agenda of decolonising \glsf{indigenous} societies. It emphasises the connection between the researcher and their own culture, in the sense that cultural practices and forms of expressions should be reflected in the way the research is conducted, including language, metaphors, oral traditions and knowledge systems. It also advocates an holistic approach which strives to reach a balance between different areas of life, integrating intellectual, social, political, economic, psychological and spiritual dimensions.
%
%\begin{question}[subtitle={Activity: Summarising \glsf{indigenous} traditions}] Given these characteristics, what does the \glsf{indigenous} tradition assume of the nature of the world (\glsf{ontology}), how knowledge is acquired (\glsf{epistemology}), and what is of value in research (axiology)?
%
%\begin{solution}\Glsf{ontology}: reality is determined by the \glsf{indigenous} culture, to which the researcher is strongly connected.
%
%\Glsf{epistemology}: this is determined by \glsf{indigenous} knowledge systems, cultural practices and forms of expressions.
%
%Axiology: The researcher has a social and political agenda of decolonisation of \glsf{indigenous} societies.
%
%\end{solution}\end{question}
%%%Hack to correct tcbox behaviour
%\color{black}
%to re-write as understanding of \glsf{research-design} found in reviewed articles. Plus considering candidate options for \glsf{research-design} 

%%LR -- stopped here 9/11


% Cref{tab:dataevidence} provides a summary of the type of data and \glsf{evidence} that were introduced in~\Cref{stage1}. In the next activity you will reflect on those which are most relevant to your project.
%
%\begin{table}[htbp]
%\begin{minipage}{\linewidth}
%\setlength{\tymax}{0.5\linewidth}
%\centering
%\small
%\caption{Common types of data/\glsf{evidence} used in a research project}\label{tab:dataevidence}
%\begin{tabulary}{\textwidth}{@{}LL@{}} \toprule
% \textbf{Types of data/\glsf{evidence}} & \\
%\midrule
%
% \textbf{Quantitative} data that can be quantified or measured, and be given numerical values & \textbf{Numerical} numbers, either discrete or continuous \\
% & \textbf{Ordinal} can be arranged in an order, but are not necessarily numerical \\
% & \textbf{Interval} \glsplf{ordinal-datum} for which we can calculate precisely the interval between any two data points \\
% \textbf{Qualitative} all other data which is descriptive in nature & \textbf{Categorical (or nominal)} correspond to categories which cannot be ordered and on which mathematical operations and function don't apply \\
% & \textbf{Other} e.g., texts, words, images, sounds,~\etc{}. \\
%\bottomrule
%
%\end{tabulary}
%\end{minipage}
%\end{table}
%
%\begin{question}[subtitle={ACTIVITY: Considering data and \glsf{evidence}}] Consider whether you will need to use qualitative or quantitative data in your project. Write down 
%
%\begin{guidance}You should use the second column to help you decide, i.e., \enquote{I'll be using a survey combining a Likert scale with free text answers, so that means I'll be using Ordinals, and so quantitative, and Other, and so qualitative.}
%
%For each needed in your project, you should also give specific examples and indicate the source.
%\end{guidance}\end{question}
%%%Hack to correct tcbox behaviour
%\color{black}


%\subsection{Putting it all together}\label{ssect:PuttingItAll}
%You should now have enough material to sketch your overall \glsf{research-design}.
%
%\begin{question}[subtitle={Activity: Sketching your overall \glsf{research-design}}] Based on your judgements, expressed in the previous activities, summarise your \glsf{research-design} by addressing each of the following questions:
%
%\begin{itemize}
%\item which \glsf{evidence} and data will you need and why?
%
%\item where will you source such data/\glsf{evidence} from?
%
%\item which \glsf{research-strategy} are you thinking of adopting and why?
%
%\item which research methods are you thinking of applying for your data collection/analysis or modelling within that strategy?
%
%\end{itemize}
%
%Ensure that your answers are justified in term of your \glsf{research-problem}, and intended aim and objectives, by indicating explicitly the rationale behind your choices.
%
%\begin{guidance}At this point, your choices will be tentative, but should provide a good starting point for further investigation and a meaningful conversation with your supervisor who will be able to advise you further.
%
%As well as your intended research, you should also keep in mind that you will have limited time and resources to complete your project, so you should limit your choices to be:
%
%\begin{itemize}
%\item manageable in terms of their application in the context in which you are going to conduct your research and the time you have available
%
%\item efficient in terms of the data/\glsf{evidence} they produce and your ability to process them with the resources and time you have available
%
%\item effective at producing data/\glsf{evidence} that you have the skills and expertise to analyse in the time you have available.
%\end{itemize}
%
%
%\end{guidance}\end{question}
%%%Hack to correct tcbox behaviour
%\color{black}


%% LR -- I think all these moves to~\Cref{stage3}!!!!!!!!!

%%%%LR -- this to move to~\Cref{stage3}, if kept at all
%\subsection{Investigating research strategies and methods further}\label{ssect:InvestigatingResearchStrategies}
%The overview provided in this section was designed to help you develop a broad understanding of possible choices you can make to help you sketch your initial \glsf{research-design}. It is now time to start looking a little deeper into your most likely research strategies and methods.
%
%\begin{question}[subtitle={Activity: Reviewing your chosen research strategies and methods}] Consider the research strategies and methods you have included in your sketched \glsf{research-design}. Conduct a small literature review on each of them to help you confirm that they are indeed suitable for your research, and to help you articulate how and why they are suitable for your project.
%
%\begin{guidance}As this is a literature review, your should follow the process and practices you have already learnt and applied, including recording entries and notes in your BMT.
%
%You should focus on materials which will help you understand how they work, and their strength and weaknesses in relation to your project, something you will return to in the next stage of your project, where you will consider specific procedures for applying them.
%
%Your review does not need to be extensive: a couple of references for each strategy/method should suffice, as long as they provide the information required. You can use the annotated reading list in the next section as your starting point, but you should also explore the wider literature. Your supervisor should also be able to suggest literature you could start from.
%\end{guidance}\end{question}
%%%Hack to correct tcbox behaviour
%\color{black}
%
%\section{Annotated reading list}\label{sect:readingList}\label{sect:AnnotatedReadingList}
%
%\todo{Move these into body, and make standard bibliographic references(!)}
%
%There are many resources which cover a variety of research strategies and methods in more detail that we can here. You could start from the following:
%
%\begin{itemize}
%\item gO-GN (2020), Research Methods handbook, \href{https://go-gn.net/wp-content/uploads/2020/07/GO-GN-Research-Methods.pdf}{https://go-gn.net/wp-content/uploads/2020/07/GO-GN-Research-Methods.pdf} This is a practical introduction to research methods for phd research, written with contributions from doctoral research students.
%
%\item oates, B.J. (2006) Researching information systems and computing, SAGE. This is very good read for novice research students, particularly in information systems and computing disciplines. It provides a clear, practical and comprehensive introduction to \glsf{academic-research}, including key definitions, methods and techniques.
%
%\item the SAGE research portal at \href{https://methods-sagepub-com.libezproxy.open.ac.uk}{https://methods-sagepub-com.libezproxy.open.ac.uk} contains a great variety of resources on both strategies and methods, from articles to video tutorials
%\end{itemize}
%
%\paragraph{For \Glsf{design-science-research}} start from:
%
%\begin{itemize}
%\item vom Brocke J., Hevner A., Maedche A. (2020) Introduction to \Glsf{design-science-research}. In: vom Brocke J., Hevner A., Maedche A. (eds) \Glsf{design-science-research}. Cases, Cham. \href{https://doi.org/10.1007/978-3-030-46781-4_1}{https://doi.org/10.1007/978-3-030-46781-4\_1}, which is possibly the most up-to-date introduction to the topic.
%
%\item the International Conference on \Glsf{design-science-research} in Information Systems and Technology (DESRIST) (since 2005) have tracked the development of this \glsf{research-strategy}, with many seminal papers published its proceedings. These can be accessed via the OU Library.
%
%\end{itemize}
%
%\paragraph{For \glsf{case-study-research}} start from:
%
%\begin{itemize}
%\item yin, R.K., 2009. \Glsf{case-study-research}: Design and methods. Applied social research methods series, Vol. 5. Fourth Edition, Sage.
%
%\end{itemize}
%
%\paragraph{For systemic inquiry} start from:
%
%\begin{itemize}
%\item ison, R. (2017). Systemic inquiry. Ch. 10 in Part 3 Systems Practice: How to Act (pp. 251--274). Springer, London, which is also available as eBook Reading
%
%\item simon, G and Chard, A. eds. (2014) Systemic Inquiry: Innovations in Reflexive Practice Research. Farnhill: Everything is Connected Press.
%
%\item ison, R.L., Collins, K.B. and Iaquinto, B.L., 2021. Designing an inquiry-based learning \glsf{system}: Innovating in research praxis to transform science--policy--practice relations for sustainable development. Systems Research and Behavioral Science, 38(5), pp.610--624.
%
%\item mcClintock, D., Ison, R. and Armson, R., 2003. Metaphors for reflecting on research practice: researching with people. Journal of Environmental Planning and Management, 46(5), pp.715--731.
%
%\item kincheloe, J. L. (2011). Describing the bricolage: Conceptualizing a new rigor in qualitative research. In Key works in critical pedagogy (pp. 177--189). Brill.
%
%\end{itemize}
%
%
%\paragraph{Problem diagrams} are part of a wider approach called problem oriented engineering, with roots in software development, but more widely applicable to most forms of design and engineering. The following references are a good starting point:
%
%\begin{itemize}
%\item jackson, M., 2005. Problem frames and software engineering. Information and Software Technology, 47(14), pp.903--912.
%
%\item hall, J., Rapanotti, L. and Jackson, M., 2008. Problem oriented software engineering: Solving the package router control problem. IEEE Transactions on Software Engineering, 34(2), pp.226--241.
%
%\end{itemize}
%
%\paragraph{UML} is a modelling language with roots in software engineering. It is now an international standard that can be found at:
%
%\begin{itemize}
%\item \href{https://www.iso.org/standard/52854.html}{https://www.iso.org/standard/52854.html}
%
%\item however, many tutorials are available online, so it should be relatively easy for you to find an introductory one. There are also many UML digital modelling tools, some of which are open source and free to use
%
%\end{itemize}
%
%\paragraph{Systems diagrams}
%
%The \glsf{system} thinking diagramming tutorials on the Open University's Open Learn site are a good starting point to look up these techniques. Link: \href{https://www.open.edu/openlearn/science-maths-technology/across-the-sciences/systems-thinking-diagramming-tutorials}{https://www.open.edu/openlearn/science-maths-technology/across-the-sciences/systems-thinking-diagramming-tutorials}. For an application of these techniques on a particular case study you can also consider \href{https://www.open.edu/openlearn/science-maths-technology/computing-ict/diagramming-development-1-bounding-realities/content-section-0?active-tab=description-tab}{https://www.open.edu/openlearn/science-maths-technology/computing-ict/diagramming-development-1-bounding-realities/content-section-0?active-tab=description-tab}
%
%
%\paragraph{For tables, graphs and charts} start with the following free resources: \href{https://www.open.edu/openlearn/science-maths-technology/mathematics-statistics/working-charts-graphs-and-tables/content-section-0?active-tab=description-tab}{`Working with charts, graphs and tables'} %
%%
%%\item  \enquote{More working with charts, graphs and tables} at:
%%
%%\href{https://www.open.edu/openlearn/science-maths-technology/mathematics-statistics/more-working-charts-graphs-and-tables/content-section-0?active-tab=content-tab}{https://www.open.edu/openlearn/science-maths-technology/mathematics-statistics/more-working-charts-graphs-and-tables/content-section-0?active-tab=content-tab}
%%
%or the UK BBC Skillswise site at: \href{https://www.bbc.co.uk/teach/skillswise/graphs/zmkpqp3}{https://www.bbc.co.uk/teach/skillswise/graphs/zmkpqp3}.\todo{Annotated bibliography needs completing for all research methods.}


\chapter{Writing up}\label{ch:writingUpInStage2}

Well, you're reaching the end of~\Cref{stage2}. You're really speeding along now. 

Your end of \Cref{stage2} report will help you consolidate your work so far and provide some more content for your dissertation. Its recommended structure and content are indicated in~\Cref{stage2WritingOutcomes}, which builds on those of your~\Cref{stage1} report. 

\begin{question}[subtitle={Activity: Writing and assessing your report for~\Cref{stage2}}] Using your word processor of choice, revise and expand your~\Cref{stage1} report by applying the structure and guidance in~\Cref{stage2WritingOutcomes}, making good use of your notes and summaries from all~\Cref{stage2} activities you have carried out.

Assess your report by applying the criteria in~\Cref{tab:criteriaForReport2}. Revise and iterate until you are ready to move on. 
\begin{guidance}
%Hack to correct tcbox behaviour
\color{black}


%Hack to correct tcbox behaviour
\color{black}
You should ensure that each required section of your report includes at least a draft of the main points you wish to make. In writing your~\Cref{stage2} report:
\begin{itemize}
	\item you should review and revise all content from~\Cref{stage1} in light of your increasing understanding from engaging with the \glsf{academic-literature}, ensuring all related research elements remain coherent
	\item you may like to add a glossary section to your introductory chapter to collect key technical or other definitions which may recur in your work. You can then expand this section as you  progress through the follow-up stages 
	\item by the end of~\Cref{stage2} your literature review should be a substantial, almost complete, draft, well structured and articulated through solid academic arguments, appropriately supported by citations. It should demonstrate your understanding of the main literature which relates to your \glsf{research-problem}, showing your awareness of what is known in the field and which knowledge gaps still remain. Your critical summary should synthesise the key insights from your review, and highlight and justify the expected knowledge contribution of your research in relation to them
	\item in this first engagement with \glsf{research-design}, you should reflect on the kind of data and \glsf{evidence} you will need for your research and show awareness of which methodology to apply, including potential research strategies and methods. Although you will develop your methodology in detail in later stages, you should be able to understand methodological choices in the literature you have reviewed and how they may be relevant to your project  
	\item you should continue to monitor your progress and ensure your project remains viable by revisiting your project risk and \glsf{work-plan}
	\item equally you should continue to practise \glsf{reflection} and \glsf{reflexivity} to further develop your skills and attitude to research
\end{itemize}
In evaluating your report, for each criteria in the table, you should consider the related prompts, write down any further work needed for your next stage, and update your \glsf{work-plan} and risk assessment accordingly.
\end{guidance}\end{question}
%%Hack to correct tcbox behaviour
\color{black}

\begin{SimpleNColTable}{tab:criteriaForReport2}{2}{Criteria for reviewing your report}[X[-1,r]X[l]]
Criteria & Prompts \\
Completeness & Are all sections included and their content complete? What is missing?\\
Academic writing & Is your writing clear, concise and precise? Should you improve it further? \\
Logical structure and flow & Have you structured your writing so that your narrative follows a logical flow? Which restructuring may be needed?\\
Supporting \glsf{evidence} & Are all your claims supported by appropriate citations or other \glsf{evidence}? Which further \glsf{evidence} do you still need?\\
Citation and reference style & Do all your citations and references comply with the bibliographical style required by your course of study? \\
Avoiding \glsf{plagiarism} & Have you acknowledged the work of others? Is it clearly distinguished from your own? \\
Grammar and spelling & Have you proof-read your report carefully to remove all typos and grammatical errors? \\
\end{SimpleNColTable}


\begin{takeaways}{\Cref{stage2}}\label{ch:Stage2Takeaways}
\Cref{stage2} focuses on completing your literature review through synthesis and well structured arguments, and on developing your understanding of research design. Here are the takeaways from~\Cref{stage2}:
\begin{itemize}
\item fundamental skills for synthesising the literature include \glsf{critical-thinking} and writing, and the ability to establish logical connections between ideas and arguments
\item academic writing requires you to observe a number of practices, which ensure your writing is clear, precise, logical, concise and well-structured
\item all forms of \glsf{plagiarism} are unacceptable in \glsf{academic-research}
\item an academic argument is a structured argument whose key elements help you ensure your claims are well reasoned and supported
\item beware of \glsplf{logical-fallacy} in your academic arguments
\item in developing your literature review you must consider how best organise your thematic summaries to take the reader from your introduction to your conclusions, including appropriate structuring, narrative flow and signposting within and between sub-sections
\item your literature review will be adequate if it contextualises and justifies your research in relation to the \glsf{academic-literature}, and is well structured and logically argued
\item the philosophical pillars of \glsf{research-design} are \glsf{ontology}, \glsf{epistemology} and methodology. Differing views on them has led to different research mindsets
\item research mindsets capture ways of thinking about the nature of knowledge and its generation
\item your methodology is the \glsf{system} of research strategies and methods which defines how you conduct your research
\item you can choose among several research methods to collect and analyse data, \glsf{model} real-world scenarios or systems and artefacts, and derive insights from your findings
\item you can choose among several research strategies to systematise your use of research methods in order to reach your \glsf{research-aim} 
\item knowing the building blocks of \glsf{research-design} helps you understand how research reported in the \glsf{academic-literature} was conducted.
\end{itemize}
\end{takeaways}

%%Sectional bibliography, hardcoded part title
\printbibliography[segment=\therefsegment,title=Stage II \bibname]