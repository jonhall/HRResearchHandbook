\begin{Dedication}
To Felicity and Harriet.

May they grow to explore.
\end{Dedication}

\chapter{Foreword}
You have just about embarked in your very first research project, possibly as the last component of your master's studies. That's both exciting and daunting.

Exciting because you will be able to focus on a topic of your own choice and to investigate in depth an issue or problem which is of particular interest to you, either personally or professionally. In doing so, you will acquire a deep knowledge of that topic, conduct a unique and novel research study, and develop and apply a wide range of research skills.

Daunting because your success will depend on you demonstrating your mastering of the topic and of the research process, that you can exercise competently a wide range of research skills, and can communicate your work effectively through an academic dissertation, possibly the largest and most demanding piece of writing you will ever undertake.

The aim of this book is to support you in taking your first steps into academic research. It will provide you with a solid scaffolding for you to become a competent and confident researcher. It will demystify the language around academic research and, through practical advice and activities, will help you plan, manage and execute your project work successfully from start to finish.

But the benefits don't stop with the completion of your project. In succeeding in your research, you will have also gained and demonstrated a wide range of skills which are professionally relevant and valued by employers, from problem solving to effective communication, digital and information literacy, self-management and resilience. These transferable skills will serve you well in your profession and your life, regardless of the path you will take after you've completed your studies.

\paragraph{Who should read this book}
This book is written for those who are taking their first steps into academic research, typically master's students -- which is why master's level research is particularly prominent. However, its content is equally valuable to first-year Doctoral students, particularly those who are required to complete an initial research feasibility study as part of their probationary requirements. 

We would also argue that the book would be valuable to anyone, in academia and industry, who is interested in research in the more general sense. This is because most of the advice in the book is applicable beyond academic research.

\paragraph{Who can read this book}
We have tried to be as inclusive as possible, including using gender-neutral language where possible, using the \LaTeX{} \verb|cblind| package to change colours. We have plans to create an audio version, when funds allow – we do not charge for the book, but any donation you could make would be allow us to be even more inclusive.

\chapter{Acknowledgements}

The writing of this book has been enriched by the support, encouragement, and insights of many, and it whose contributions it is our pleasure to acknowledge.

First\footnote{In alphabetical order.}, our research students Assia Alexandrova, John Briers, Michael Clark, Silvana Costantini, Nigel Eve, Anne Nkwocha, Georgi Markov, Mercy Williams, and many others, provided invaluable early perspectives on the teaching of research. If this book resonates with its intended audience, it's because of them.

Second, to our colleagues in the taught postgraduate programme at the Open University, all passionate educators, thank you for sharing teaching strategies that informed our chapters on mentoring research students. Thanks also to the students who studied with us from that programme who were instrumental in making those strategies practical.

Third, thanks go to John Watkins, who provided many hours' worth of feedback, giving more clarity and precision in our writing. 

Fourth, our thanks also go to the Open University which offered its extensive library and helped us source many critical resources.

Finally, we thank Michael Jackson, whose mentorship and guidance over the years has deeply influenced our approach to research.

This book stands as a testament to the collaboration that defines academic research, and we are grateful to all those who played a role in its creation. While we have done our utmost to recognise every contribution, any omissions are entirely unintentional.

\bigskip

To all remaining inFelicitys, including that one, we 'fess up.
