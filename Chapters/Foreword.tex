\chapter{Foreword}
You have just about embarked in your very first research project, possibly as the last component of your master's studies. That's both exciting and daunting.

Exciting because you will be able to focus on a topic of your own choice and to investigate in depth an issue or problem which is of particular interest to you, either personally or professionally. In doing so, you will acquire a deep knowledge of that topic, conduct a unique and novel research study, and develop and apply a wide range of research skills.

Daunting because your success will depend on you demonstrating your mastering of the topic and of the research process, that you can exercise competently a wide range of research skills, and can communicate your work effectively through an academic dissertation, possibly the largest and most demanding piece of writing you will ever undertake.

The aim of this book is to support you in taking your first steps into academic research. It will provide you with a solid scaffolding for you to become a competent and confident researcher. It will demystify the language around academic research and, through practical advice and activities, will help you plan, manage and execute your project work successfully from start to finish.

But the benefits don't stop with the completion of your project. In succeeding in your research, you will have also gained and demonstrated a wide range of skills which are professionally relevant and valued by employers, from problem solving to effective communication, digital and information literacy, self-management and resilience. These transferable skills will serve you well in your profession and your life, regardless of the path you will take after you've completed your studies.

\paragraph{Who should read this book}
This book is written for those who are taking their first steps into academic research, typically master's students -- which is why master's level research is particularly prominent. However, its content is equally valuable to first-year Doctoral students, particularly those who are required to complete an initial research feasibility study as part of their probationary requirements. 

We would also argue that the book would be valuable to anyone, in academia and industry, who is interesting in research in the more general sense. This is because most of the advice in the book is applicable beyond academic research.

\paragraph{Who can read this book}
We have tried to be as inclusive as possible, including using gender-neutral language where possible, using the \LaTeX{} \verb|cblind| package to change colours. We have plans to create an audio version, when funds allow – we do not charge for the book, but any donation you could make would be allow us to be even more inclusive.

\chapter{Acknowledgements}

The journey of writing this book has been enriched by the support, encouragement, and insights of many individuals, and it is our pleasure to acknowledge their contributions here.

First and foremost, we express our deepest gratitude to John Watkins, who provided many hours worth of invaluable feedback on early drafts and inspired clarity in the writing. 

Assia Alexandrova, Michael Clark, Nigel Eve, Anne Nkwocha, John Briers, Silvana Costantini, Georgi Markov, and Mercy Williams and so many others, provided invaluable early perspectives that ensured the book resonated with its intended audience. Their reflections on their own research journeys were woven into the text as relatable anecdotes.

To our colleagues in the taught postgraduate programme at the Open University, all passionate educators, thank you for sharing teaching strategies that informed our chapters on mentoring research students. 

%To John Watkins, whose detailed feedback from the perspective of industry practitioners added a practical lens to our discussions, we extend our heartfelt gratitude.

%We are particularly grateful to ???, a dear friend and fellow researcher, whose rigorous questioning and constructive critique pushed us to sharpen our arguments. ???, a colleague with unmatched expertise, generously shared their knowledge, helping us refine key chapters on research methodologies.

Our thanks also go to the Open University, who offered their extensive library and helped us source many critical resources.

%We acknowledge Michael Jackson, whose mentorship and guidance over the years has deeply influenced our approach to academic research.

This book stands as a testament to the collaborative spirit of academic research, and we are profoundly grateful to all those who played a role in its creation. While we have done our utmost to recognise every contribution, any omissions are entirely unintentional. To each of you, thank you.

To all remaining errors as, we 'fess up.
