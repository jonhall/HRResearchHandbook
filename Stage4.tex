\chapter*{Stage 4: Analysing data and evidence}

You've now reached Stage 4, which means your project end is rapidly approaching. In this stage you are in the midst of your data gathering and analysis, which is possibly the most exciting, yet demanding, part of your research: this is where you get the chance to make your original contribution to knowledge.

With reference to our 5-stage framework, the activities which are in focus in Stage 4 are summarised in Table~\ref{tab:stage4}, which also provides some guidelines for your interaction with your supervisor during this stage.


\begin{table}[htbp]
\caption{Research activities addressed in Stage 4 (20\% of project length)\label{tab:stage4}}
\small
\begin{tabulary}{\tablewidth}{@{}LLLLL@{}} 
\toprule
 \textbf{Research process activities} & \textbf{Deliverables} & \textbf{Learning Outcomes: by the end of this stage you will:} & \textbf{Effort} & \textbf{Suggested focus of your interaction with your supervisor} \\
\midrule

 \textbf{Identifying the research problem} & Research problem statement, refined as needed & be able to assess and improve your research problem statement & 1\% & \\
 \textbf{Reviewing the literature} & Substantial draft of your literature review, refined as needed & be able to assess and improve your current draft & 1\% &  \\
 \textbf{Setting your aim and objectives} & Aim and objectives, refined as needed & be able to assess and improve your aim, objectives and related tasks & 2\% & \\
 \textbf{Developing the research design} & Research design description, refined as needed &  select and justify analysis methods based on the data you have gathered; describe data analysis procedures  & 2\% & Suitability of methods and procedures \\
\textbf{Gathering and analysing evidence} & summary of the data/evidence gathered and their analysis & apply appropriate data analysis methods; present your data and evidence in a concise and effective way & 50\% & Appropriateness of data analysis and presentation\\
\textbf{Interpreting and evaluating findings} & draft summary of findings from data/evidenced gathered & derive findings from data analysis and critically assess them in relation to research aim and objectives & 15\% & Critical and logical thinking\\
 \textbf{Reflecting and reporting} & Stage 4 report; draft abstract and extended abstract for your project & know the difference between an abstract and an extended abstract, be able to assess your research progress and write up a substantial report, an abstract and an extended abstract & 25\% & Any further improvements required, particularly in relation to critical thinking and academic writing \\
 \textbf{Planning work and managing risk} & Updated risk and work plan & be able to assess risk and draw a work plan & 5\% & Any major adjustment required to address deficiencies or manage risk \\
\bottomrule
\end{tabulary}
\end{table}

\begin{question}[subtitle={Activity: Understanding the effort needed in this stage}] Consider Table~\ref{tab:stage4} carefully, paying particular attention to the entries in the `Effort' column. Make a note of the activities which are most prominent in this stage and what their deliverables and learning outcomes are.

\begin{solution}
Gathering and analysing evidence will constitute by far your major effort in this stage (50\% of study time): in particular, the framework assumes that you will have already gathered some data and evidence in Stage 3, so that you can start selecting and applying appropriate data analysis methods. The amount of evidence will depend on how much work you were able to complete in the previous stage, and it is very likely that you are still gathering more data and evidence. Nevertheless, you should have enough to start analysing, and thinking of possible findings in relation to your aim and objectives. 
\end{solution}\end{question}
%%Hack to correct tcbox behaviour
\color{black}

Note that your data analysis and interpretation may also prompt you to look for more evidence, including, perhaps, reviewing more academic literature or even re-think or adjust your current aim an objectives. Therefore, so you should expect some level of iteration back to activities you have carried out in previous stages, and revision of things you have written.

By the end of Stage 4 your data collection, analysis and interpretation should be on a solid ground, and consistent with your aim and objectives. Your research design description should also be close to its final form. If that's not the case, then you will need to re-consider carefully both your project risk and work plan. Given the criticality of this stage, it is essential that you work very closely with your supervisor throughout.


\section{Choosing and applying data analysis methods}
This section assumes that you have gathered a certain amount of data and evidence, so that you can proceed with some analysis. \footnote{If that's not the case, then, you should go back to Stage 3, Section~\ref{sect:???} which focuses on data collection methods. You should also discuss your progress with your supervisor, revisiting your project risk around access to the required data and evidence.}

Your choice of data analysis methods will depend on two key factors: the kind of data and evidence you have, and what you are trying to achieve, that is your aim and objectives. 

This section provides an introduction to the most commonly applied data analysis methods: it is far from complete and does not go very deeply into the details of each method. Instead, its purpose is to help you make your initial selection, after which you should review the related specialised literature to help you apply your chosen methods correctly. Your tutor's guidance will also be essential to this.

\subsection{Quantitative data analysis}
Methods for qualitative data analysis are commonly used to describe characteristics of quantitative data\footnote{Refer back to Section~\ref{sect:??} for a definition.}, to measure differences and assess relationships between variables, and to test hypothesis.

Among them, statistical analysis methods are the most common, usually accompanied by tables, charts and graphs to summarise data or visualise patterns. More recently, they have been complemented by Machine Learning (ML) methods, emerged from the rapid expansion of AI approaches in the past decade.

We look briefly at both, starting from statistical methods. 

\subsubsection{Statistical analysis}
There are two broad categories:
\begin{itemize}
	\item descriptive statistics, whose aim is to describe data; and
	\item inferential statistics, whose aim is to make predictions from the data
\end{itemize}

\paragraph{Descriptive statistics} 
This is used to describe various attributes of a data set. Primarily:
\begin{itemize}
	\item centrality, to establish the `centre' of the data set. Three measures are commonly used: the \textit{mean}, which provides the average value of the data set; the \textit{median}, which provides its mid point\footnote{Remember that quantitative data can be ordered.}; and the \textit{mode}, which indicates the value that occurs the most, if any\footnote{There is no mode if no value is repeated in the data set.}
	\item dispersion, to establish how close to the mean or otherwise the values in a the data set are. The higher the standard deviation, the greater the dispersion.  The \textit{standard deviation} is the measure used for this purpose\footnote{It is based on mathematical formulae which considers the distance of each value in the data from the mean. It is not essential for you to know such formula, which is automatically computed by spreadsheets or statistical software. Of course, you can always look it up the literature...}
	\item skewness, to establish how symmetrical the values in the data set are in reference to a bell curve (see Figure~\ref{fig:skewness}). A value of 0 indicates perfect symmetry (centre of the figure); negative values indicate left skewness, that is the tail is longer to the left and the mean is smaller than the median; vice versa, positive values indicate right skewness, that is the tail is longer to the right and the mean is greater than the median
	\item range, which is the different from the smallest (minimum) and largest (maximum) values in the data set
	\item count, which is how many individual values there are in the data set
\end{itemize}


***to do: insert skewness figure

These are lots of definitions to digest, particularly if you haven't' encountered many of these terms before! The following activity should help.

\begin{question}[subtitle={Activity: Descriptive statistics in Excel}] Assume you have have measured the weight in grams of each apple in a basket, obtaining the following numbers: 105, 120, 122, 125, 127, 128, 129, 130, 132, 133, 135, 135, 138, 140, 128. Enter these data in an Excel sheet and use the in-built data analysis function to generate the descriptive statistics.
\begin{guidance}
In the current version of Excel, you can access this function from the Data tab, by pressing the Data Analysis button. If you find it difficult to locate this function, you should check some of the many tutorials on this topic which are freely available online.
\end{guidance}
\begin{solution}
You should have obtained the following values:

\begin{tabulary}{\tablewidth}{@{}LL@{}} 
\textbf{Measure} & \textbf{Value} \\
mean & 128.47 \\
median & 129 \\
mode & 128 \\
standard deviation & 8.55 \\
skewness & -1.4 \\
range & 35 \\
minimum & 105 \\
maximum & 140 \\
count & 15 \\
\end{tabulary}

Note that there are actually two modes in this data set\footnote{Statisticians call this bi-modal.}, 128 and 135, but Excel only returns the first encountered!
Note that the skewness is negative (hence the mean is smaller than the median), so that the data are not perfectly symmetric. 
\end{solution}\end{question}
%%Hack to correct tcbox behaviour
\color{black}

Calculating descriptive statistics is often the first step in quantitative data analysis, as these measures provide useful summaries of key properties of a data set. They can also be useful to identify errors or anomalies in the data, and can inform possible follow-up inferential statistical analysis.

Depending on your research aim and objectives, they could also be all you need in your project, and as you have found out in the activity, you do not need to be a statistician to be able to calculate them.


\paragraph{Inferential statistics}
Inferential statistics relies on the concepts of population and sample: the \textit{population} is the entire group you are interested in studying -- say, all voters in a general election; while the \textit{sample} is the portion or subset of that group you have access to in your research.
Then the aim of inferential statistics is to make predictions about the population based on the sample.




********* cut and paste from somewhere, just to to miss the ideaas

In inferential statistics, data are analyzed from a sample to make inferences (deductions) and generalize the results to the population. Many of the tests were created in political and social sciences, using polling data to understand and predict behaviors (hypothesis) within a population, for example, the results of an election. The purpose of inferential statistics is to answer or test the hypotheses. The null hypothesis (H0 “H-naught,” “H-null”) states that there is no difference between the control and the study group and the alternative hypothesis (H1 and Ha) states that there is a difference among the groups, or there is some association between the predictor and the outcome.2 During a statistical test, a P-value is interpreted as the probability of the event (for example difference between means) occurring by chance if the null hypothesis is true (Tables 32.1 and 32.2), the latter is defined by parameters set by the investigator during power calculations and selection of sample size, which have been described elsewhere in this book (Chapter 26).
************
The kind of predictions 


common types of predictions:
- about differences between groups
- relationships between variables

it helps to connect the dots and make predictions about what you expect to see in the population based on your sample data.  

they can be used to test hypothesis.

The most common inferential statistical methods:
- t-test: to assess statistically significant differences between groups in terms of their means, standard deviations and skewness
- ANOVA: same as t-test but for multiple groups
- correlation: to assess the relationship between two variables (correlation is not causation)
- regression: to understand cause and effect between variables

Example and activity to follow; including how to present the analysis

\subsubsection{Sampling}
Clearly if your sample is not representative, then you can't make prediction about the population.

\subsection{ML methods???}

Complementary to statistical methods, an increasing variety of Machine Learning algorithms can also be used for data analysis. Two broad application concern the recognition of patterns in data and making predictions based on historical data.

**** more about this????


issue -- black box nature




\subsection{Choosing the right method for your research}
Type of data and your research aim and objective


type of data table, and which method apply!!!!!

https://stats.oarc.ucla.edu/other/mult-pkg/whatstat/


Skewness -- and methods which apply


Nature of research questions, and related hypothesis

1) assess characteristics of your sample --> descriptive
2) predict outcomes --> both descriptive and inferential




\subsection{Qualitative data analysis -- to do}



\section{Writing up your analysis --- to do}

Building on your pilot work, in this stage you will apply your chosen methods and procedures to gather and analyse your data and evidence. Remember that this may well include modelling, design or prototyping work, depending on your chosen research strategy.

You are expected to make significant progress on this aspect of your research during Stage 4, therefore, it is important that you plan your work carefully in consultation with your supervisor.

Depending on your chosen research strategy and methods, you may end up gathering a large quantity of data or evidence, so you will need to think carefully at how you will handle it. In this respect, you will need to consider:

- How to organise and store your `raw' data and evidence. These represent anything you gather directly, say, full transcripts of participants' interviews, survey responses, logs of experimental data, copies of documents, etc. It is essential that you manage them carefully to ensure you don't loose track of important information, and that you can always refer back to them in your subsequent analysis. You may need to include such evidence or a sample of it in an appendix of your dissertation as demonstration of the work you have done.

- How to summarise data and evidence in the body of your stage report, and later in your final dissertation. This will depend on the nature of your data and evidence, and you will need to ensure that your summaries are appropriate to convey the essence of the evidence you have generated and to support your analysis, so that you can build academic arguments which relate evidence to findings, and findings to aim and objectives.

- How to structure your report. Depending on your chosen research strategies and methods, different structures are possible. For instance, you may choose to start with a section which summarises all your evidence followed by one in which you analyse it, which may work well, for instance, for Survey Research. Alternatively, you could have separate sections each including a summary and analysis of a sub-set of your evidence: this may be appropriate for mixed methods research, with each section dealing with a different kind of data, or for Design Science Research, with each section addressing a different design cycle. Whatever you choose, it is important that your report is effective in presenting your evidence and findings in a clear and rigorous manner.

\begin{question}[subtitle={Activity: Gathering and analysing your data and evidence}] Plan and execute your research on gathering and analysing data and evidence, and write up your summary and analysis. Ensure that you plan carefully how to manage your raw data\slash evidence and how to structure your report.

You should aim to complete the bulk of this work by the end of Stage 4.

\begin{guidance}To complete this work successfully, it is essential that you discuss and agree your detailed work plan with your supervisor from the start, and monitor your progress on a regular basis.

This is a substantial activity, which will take up around 50\% of your study time in this stage.
\end{guidance}\end{question}
%%Hack to correct tcbox behaviour
\color{black}

\section{Interpreting and evaluating findings}
Having gathered and analysed your data and evidence, you must now interpret your findings in relation to your aim and objectives, and generally evaluate them in terms of their contribution to knowledge and possible limitations.

By interpretation I mean being able to answer the following questions:

\begin{itemize}
\item What does the analysis of your data\slash evidence you have conducted show? What are your findings?

\item How do your findings relate to your aim and objectives?

\item How do your findings relate to what you know from the literature or from professional practice?

\item Which new knowledge do your findings contribute?

\item What do your findings fail to achieve?

\end{itemize}

\subsubsection{Activity: Interpreting and evaluating your initial findings}
Consider your analysis of data and evidence, and based on it, address each of the above questions. In doing so, you should express well-formed academic arguments, with explicit reference to the data and evidence you have gathered and analysed.

\begin{guidance}You can refresh your understanding of academic arguments by looking back at the related materials for Stage 1.

Your interpretation and evaluation of findings will be, of course, limited by the data\slash evidence you have gathered and analysed up to this point. You will revisit and expand this work in Stage 5 in order to complete your project.

Depending on the extend of your current data\slash evidence gathering and analysis, this activity could take you up to 15\% of your study time.

\end{guidance}

\section{Drafting an abstract for your project}

An abstract is a common way to summarise academic research. Abstracts are an integral parts of all published academic articles -- you will have encountered many abstracts while reviewing the literature. They are also very common in academic dissertations, therefore it is highly likely you will be required to include one at the beginning of yours.

An {abstract} provides a short summary of the whole research written for a specialist audience, that is you can assume the reader has good knowledge of the topic and field of study. It should be a stand-alone item, so that it can be understood without reference to any other part of the article or the dissertation.

Its content should convey succinctly the research problem, how and where it arises and its significance, the research aim and research design, key results obtained by the research, their evaluation and their implications for further research or professional practice. 

Writing an abstract for your research is a good exercise, even if one is not needed for your dissertation, as it gives you an opportunity to write a logical argument that connects all key elements of your research. This can help you check that all the pieces fit together in a coherent manner. It is also something you can share with your supervisor and critical friends to communicate succinctly the essence of what you have done and achieved.  

\begin{question}[subtitle={Activity: Drafting your abstract}]
Write a draft abstract for your project, which should reflect your research progress to date. 
\begin{guidance}
You should go back to some of the articles you have reviewed to consider the content and structure of their abstract. Choose a structure which may fit your project and write up your draft abstract accordingly.

As your research is yet to be completed, you will not be able to write up the full abstract, but you should end up with a draft that you can easily complete by the end of your project.
\end{guidance}
\end{question}
%%Hack to correct tcbox behaviour
\color{black}


%%%LR -- I don't think we need this, which is a T802 thing
%\subsection{The extended abstract} 
%
%An {extended abstract} is a summary of academic research intended for a more general audience, so that it should be easily read and understood by someone with only a superficial knowledge of the topic. As with the abstract, it should be a stand-alone item without any reference to your full dissertation. However, it is a lengthier piece of academic writing, structured with headings and sub-headings, including citations and references, and possibly tables, figures and diagrams to help you present and summarise your work.
%
%\begin{question}[subtitle={Activity: Drafting your extended abstract}] Write a draft extended abstract for your project, which should reflect your research progress to date.
%
%\begin{guidance}Your extended abstract should be 4 to 6 pages in length (once complete) and a common structure is as follows:
%
%\begin{itemize}
%\item Title --- the same as your dissertation
%
%\item Introduction and background --- an outline of your research problem in its context, its significance, and the knowledge gap addressed by your research
%
%\item Aim and objectives --- from your dissertation
%
%\item Research design --- an outline of your research design
%
%\item Results --- a summary of the evidence collected and analysed, and your key findings
%
%\item Discussion --- how significant your findings are in relation to research problem and knowledge gap
%
%\item Conclusion and future work --- your overall conclusions and possible follow-up research
%
%\item References --- selected references cited in the body of your extended abstract
%
%\end{itemize}
%
%Your course may have different guidelines which you should check and follow to produce your extended abstract.
%
%\end{guidance}\end{question}
%%%Hack to correct tcbox behaviour
%\color{black}

\section{Reflecting and reporting in Stage 4 -- to complete}

It's time to write your Stage 4 report. As in the previous stages, before you do, it is time for more reflection. 

\begin{question}[subtitle={Activity: Reflecting on your learning and practice}]
As you did at the end of the previous stages, in this activity you are asked to stand back and reflect deeply on what you have leant and done, the wider context of your work and your own attitude to it. Specifically, you are asked to think deeply about each of the following:

\begin{itemize}
	\item your study this far
	\item the way you work. Are you tidy and systematic, or let things happen organically? For instance, how does
	\item the context of your research
	\item your feelings about your project
\end{itemize}

You should also think of any significant changes with respect to your reflection in the previous stages
\begin{guidance}
You should be accomplished at reflection by now. However, should you need to, you can refer back to the guidance to this activity in Stage 1, Section~\ref{sect:stage1Reflection}.
\end{guidance}\end{question}
%%Hack to correct tcbox behaviour
\color{black}

Your end-of-Stage 4 report will help you consolidate your work so far, adding yet another increment toward you full dissertation. We recommend you follow the guidance in Table~\ref{tab:S4report} to write your report.

\textbf{LR -- table and follow up activity still to do -- reuse stage 3 materials}

At the end of Stage 4, you should complete a report, extending that of Stage 3 and covering the work you have carried on in this stage. Its recommended structure and content are indicated in Table 1.

Table 1 -- Report structure and content guidance

\begin{table}[htbp]
\begin{minipage}{\linewidth}
\setlength{\tymax}{0.5\linewidth}
\centering
\small
\begin{tabulary}{\textwidth}{@{}ll@{}} \toprule
 \textbf{Structure} & \textbf{Content guidance} \\
\midrule

 Proposed title & Your title should continue to capture succinctly research problem and aim \\
 Abstract & This should provide a succinct summary of your research aimed at a specialised audience \\
 Sect 1 - Introduction 1.1 Background to the research 1.2 Justification for the research & This section should provide an introduction to your research topic in its wider context (as background) and your justification of why the research is worth pursuing. It should be well articulated and supported by evidence \\
 Sect 2 - Literature review 2.1 Review of existing relevant knowledge 2.2 Critical summary, including knowledge gap to be addressed by the research & Your review should provide a critical account of your in-depth engagement with the academic (and other) relevant literature, including identifying key trends, ideas and possible knowledge gaps. Most of your citations should point to academic articles. Your critical summary should highlight key insights from your review and provide a strong justification for your proposed research. Both coverage and depth of your review matter. You should ensure that your review is well structured, with a logical narrative flow and your arguments are well supported by evidence \\
 Sect 3 - Research definition 3.1 Problem statement 3.2 Aim, objectives, tasks and deliverables 3.3 Knowledge contribution & You should ensure that your research problem is well articulated and appropriate for your course and your personal and professional circumstances, that your aim and objectives are consistent with research problem, that tasks and deliverables break down your objectives appropriately and are clearly related to your chosen research methods, and that the intended knowledge contribution of your research is clearly articulated \\
 Sect 4 - Research design 4.1 Evidence and data 4.2 Research strategy and methods 4.3 Research procedures 4.4 Ethical, legal and EDI considerations & This section should demonstrated your critical engagement with all elements of research design, including a detailed account of the data and evidence needed in your research, the research methods and research strategies you will to apply, and how you will apply them within your project. Your account should be supported by a clear rationale and insights from the related literature, and appropriately justified in relation to your research problem, aim and objectives. It should also demonstrate your careful consideration of ethical and legal matters, and that your research will comply with your course and university requirements \\
 Sect 5 - Analysis and interpretation 5.1 Summary and analysis of evidence 5.2 Summary of key findings 5.3 Interpretation in relation to aim and objectives & This section should demonstrate substantial progress towards gathering and analysing your data and evidence, and interpreting them in relation to aim and objectives. It should demonstrate a competent execution of your research design, present appropriate summaries of evidence and data, supported by raw data in an appendix if needed. Key findings should be clearly identified and logically connected to evidence, with good critical reflection on their implications for aim and objectives. \\
 Sect 6 - Assessment of your proposed research 6.1 Qualification fit 6.2 Personal and professional fit 6.3 Technical skills and resources needed 6.4 Statement of feasibility 6.5 Personal reflection on research process & In this section you should continue to argue how your research is a good fit across all criteria. You should provide a clear rationale as to why you think what you are proposing is feasible. You should also reflect on your growing understanding of the research process, including key learning and aspects you have found particularly challenging. \\
 Sect 7 - Planning, scheduling and risk assessment 7.1 Statement of progress 7.2 Key priorities in follow-up stage 7.3 Risk assessment & In this section you should reflect on the progress you have made in Stage 2 and establish your priorities for the next stage. You should also review your risk assessment as appropriate. \\
 References & You should keep your growing references in good order and ensure you apply the required bibliographical style consistently. Ideally, you should use a BMT to generate and integrate your references within your report \\
 Appendix - Extended abstract & If needed, you should include your draft extended abstract as an appendix. This should provide a structured summary of your research aimed at a generalist audience. \\
 Appendix - Raw evidence & If relevant, you should include a sample of your raw data as an appendix \\
 Appendix - Work schedule & You should include your revised work plan as an appendix \\
 Appendix - Risk assessment table & You should include your updated risk table as an appendix \\
\bottomrule

\end{tabulary}
\end{minipage}
\end{table}

\begin{question}[subtitle={Activity: Putting your report together}] Using your word processor of choice, and starting from your previous report, complete your Stage 4 report by applying the structure and guidance in Table 1, and making good use of your notes and summaries from all related activities you have carried out so far.

\begin{guidance}In this first pass at putting together your report, you should focus primarily on completeness, ensuring that each section includes at least draft content.
\end{guidance}\end{question}
%%Hack to correct tcbox behaviour
\color{black}

As in the previous stages, after you have filled in your report you should review and revise it iteratively until you are happy with your account, and are ready to move on.

Table 1 - Criteria to review your report

\begin{table}[htbp]
\begin{minipage}{\linewidth}
\setlength{\tymax}{0.5\linewidth}
\centering
\small
\begin{tabulary}{\textwidth}{@{}ll@{}} \toprule
 \textbf{Criteria} & \textbf{Prompts} \\
\midrule

 \textbf{Completeness} & Are all sections of the suggested structure completed in line with the guidance provided? \\
 \textbf{Good academic writing practices} & Have you applied good academic writing practices throughout? \\
 \textbf{Logical structure and flow} & Have you structured your narrative appropriately to ensure a logical flow of arguments? \\
 \textbf{Supporting references or evidence} & Are your key arguments supported by appropriate references or other evidence? \\
 \textbf{Citation and reference style} & Do all your citations and references comply with the required bibliographical style? \\
 \textbf{Avoiding plagiarism} & Have you acknowledged the work of others and distinguished it from your own appropriately? \\
 \textbf{Standard of English (or any modern language you use)} & Have you proof-read your report carefully to remove all typos and grammatical errors? \\
\bottomrule

\end{tabulary}
\end{minipage}
\end{table}

\begin{question}[subtitle={Activity: Reviewing your report}] Apply the criteria in Table 1 to review your current report and write up a summary of your assessment.

\begin{guidance}For each criteria, consider the related prompts to help you assess your report overall, and write down any further work needed for your next stage.
\end{guidance}\end{question}
%%Hack to correct tcbox behaviour
\color{black}

\section{Takeaways -- to do }


