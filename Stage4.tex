\chapter*{Stage 4: Third research increment}

\section{Introducing stage 4}
Your Stage 4 work will build on your pilot work in Stage 3 and any adjustments to your research design as a result. With reference to our 5-stage framework, the activities which are in focus in Stage 4 are summarised in Table 1, which also provides some guidelines for your interaction with your supervisor during this stage.

Table 1 - Stage 4 activities

\begin{table}[htbp]
\begin{minipage}{\linewidth}
\setlength{\tymax}{0.5\linewidth}
\centering
\small
\begin{tabulary}{\textwidth}{@{}llll@{}} \toprule
 \textbf{Research activities} & \textbf{Stage 4} \textbf{(20\% of project length)} & \textbf{Effort within stage} & \textbf{Suggested focus of your interaction with your supervisor} \\
\midrule

 \textbf{Identifying the research problem} & Adjust, if needed & 1\% & \\
 \textbf{Reviewing the literature} & Adjust, if needed & 1\% & \\
 \textbf{Setting research aim and objectives} & Adjust, if needed & 1\% & \\
 \textbf{Choosing the research design} & Adjust, if needed & 2\% & \\
 \textbf{Gathering and analysing evidence} & Conduct initial data\slash evidence generation and analysis & 50\% & Initial application of collection and analysis methods, and any improvements required \\
 \textbf{Interpreting and evaluating findings} & Critically assess findings up to this point & 15\% & Critical thinking in assessing findings, and any improvements required \\
 \textbf{Reporting, critical reflection and conclusions} & Assess research progress and write up Stage 4 report & 25\% & Any further improvements required \\
 \textbf{Work planning and risk management} & At stage start, review work from previous stage and project risk; adjust plan as needed If you have received feedback from supervisor on your previous stage work, adjust plan to include any revision recommended & 5\% & Any major adjustment required \\
\bottomrule

\end{tabulary}
\end{minipage}
\end{table}

\begin{question}[subtitle={Activity: Understanding the effort needed in this stage}] Consider Table 1 carefully, taking notice of the entries in the `Effort within stage' column. Make a note of the activities which are most prominent in this stage and what is expected under each.

\begin{solution}Gathering and analysing evidence will constitute by far your major effort in this stage (50\% of study time): the expectation is that the main part of your data and evidence collection and analysis should take place in this stage.
\end{solution}\end{question}
%%Hack to correct tcbox behaviour
\color{black}

\section{Gathering, analysing and presenting evidence}
Building on your pilot work, in this stage you will apply your chosen methods and procedures to gather and analyse your data and evidence. Remember that this may well include modelling, design or prototyping work, depending on your chosen research strategy.

You are expected to make significant progress on this aspect of your research during Stage 4, therefore, it is important that you plan your work carefully in consultation with your supervisor.

Depending on your chosen research strategy and methods, you may end up gathering a large quantity of data or evidence, so you will need to think carefully at how you will handle it. In this respect, you will need to consider:

- How to organise and store your `raw' data and evidence. These represent anything you gather directly, say, full transcripts of participants' interviews, survey responses, logs of experimental data, copies of documents, etc. It is essential that you manage them carefully to ensure you don't loose track of important information, and that you can always refer back to them in your subsequent analysis. You may need to include such evidence or a sample of it in an appendix of your dissertation as demonstration of the work you have done.

- How to summarise data and evidence in the body of your stage report, and later in your final dissertation. This will depend on the nature of your data and evidence, and you will need to ensure that your summaries are appropriate to convey the essence of the evidence you have generated and to support your analysis, so that you can build academic arguments which relate evidence to findings, and findings to aim and objectives.

- How to structure your report. Depending on your chosen research strategies and methods, different structures are possible. For instance, you may choose to start with a section which summarises all your evidence followed by one in which you analyse it, which may work well, for instance, for Survey Research. Alternatively, you could have separate sections each including a summary and analysis of a sub-set of your evidence: this may be appropriate for mixed methods research, with each section dealing with a different kind of data, or for Design Science Research, with each section addressing a different design cycle. Whatever you choose, it is important that your report is effective in presenting your evidence and findings in a clear and rigorous manner.

\begin{question}[subtitle={Activity: Gathering and analysing your data and evidence}] Plan and execute your research on gathering and analysing data and evidence, and write up your summary and analysis. Ensure that you plan carefully how to manage your raw data\slash evidence and how to structure your report.

You should aim to complete the bulk of this work by the end of Stage 4.

\begin{guidance}To complete this work successfully, it is essential that you discuss and agree your detailed work plan with your supervisor from the start, and monitor your progress on a regular basis.

This is a substantial activity, which will take up around 50\% of your study time in this stage.
\end{guidance}\end{question}
%%Hack to correct tcbox behaviour
\color{black}

\section{Interpreting and evaluating findings}
Having gathered and analysed your data and evidence, you must now interpret your findings in relation to your aim and objectives, and generally evaluate them in terms of their contribution to knowledge and possible limitations.

By interpretation I mean being able to answer the following questions:

\begin{itemize}
\item What does the analysis of your data\slash evidence you have conducted show? What are your findings?

\item How do your findings relate to your aim and objectives?

\item How do your findings relate to what you know from the literature or from professional practice?

\item Which new knowledge do your findings contribute?

\item What do your findings fail to achieve?

\end{itemize}

\subsubsection{Activity: Interpreting and evaluating your initial findings}
Consider your analysis of data and evidence, and based on it, address each of the above questions. In doing so, you should express well-formed academic arguments, with explicit reference to the data and evidence you have gathered and analysed.

\begin{guidance}You can refresh your understanding of academic arguments by looking back at the related materials for Stage 1.

Your interpretation and evaluation of findings will be, of course, limited by the data\slash evidence you have gathered and analysed up to this point. You will revisit and expand this work in Stage 5 in order to complete your project.

Depending on the extend of your current data\slash evidence gathering and analysis, this activity could take you up to 15\% of your study time.

\end{guidance}

\section{Drafting your abstract and extended abstract}
Within your dissertation you will need to include both an \textbf{abstract} at the beginning. You course may also asked you to produce an \textbf{extended abstract} as an appendix. In this section, we discuss the differences between the two and provide guidance to help your write them.

Both are summaries of your research, but they have a different structure and purpose.

The \textbf{abstract} provides a short summary of your whole research for a specialist audience, that is you can assume the reader has good knowledge of the topic and field of study. It should be a stand-alone item, so that it can be understood without reference to any other part of your dissertation.

Its content should convey succinctly your chosen research problem, how and where it arises and its significance, your research aim and research design, key results obtained by your research, their evaluation and their implications for further research or professional practice. You will have seen many examples of abstracts in reviewing the academic literature.

\subsubsection{Activity: Drafting your abstract}
Write a draft abstract for your project, which should reflect your research progress to date.

\begin{guidance}Typically an abstract should be a piece of text of approximately 200--300 words (once complete), without headings and sub-headings, so only using paragraph breaks, if needed. It should not include any citations or references, and abbreviation and acronyms should be kept to a minimum.

Your course may have different guidelines which you should check and follow to produce your abstract.
\end{guidance}

\paragraph{}
The \textbf{extended abstract} is a précis of your dissertation intended for a more general audience, so that it should be easily read and understood by someone with only a superficial knowledge of your topic. As with the abstract, it should be a stand-alone item without any reference to your full dissertation. However, it is a lengthier piece of academic writing, structured with headings and sub-headings, including citations and references, and possibly tables, figures and diagrams to help you present and summarise your work.

\begin{question}[subtitle={Activity: Drafting your extended abstract}] Write a draft extended abstract for your project, which should reflect your research progress to date.

\begin{guidance}Your extended abstract should be 4 to 6 pages in length (once complete) and a common structure is as follows:

\begin{itemize}
\item Title --- the same as your dissertation

\item Introduction and background --- an outline of your research problem in its context, its significance, and the knowledge gap addressed by your research

\item Aim and objectives --- from your dissertation

\item Research design --- an outline of your research design

\item Results --- a summary of the evidence collected and analysed, and your key findings

\item Discussion --- how significant your findings are in relation to research problem and knowledge gap

\item Conclusion and future work --- your overall conclusions and possible follow-up research

\item References --- selected references cited in the body of your extended abstract

\end{itemize}

Your course may have different guidelines which you should check and follow to produce your extended abstract.

\end{guidance}\end{question}
%%Hack to correct tcbox behaviour
\color{black}

\section{Reporting in Stage 4}
At the end of Stage 4, you should complete a report, extending that of Stage 3 and covering the work you have carried on in this stage. Its recommended structure and content are indicated in Table 1.

Table 1 -- Report structure and content guidance

\begin{table}[htbp]
\begin{minipage}{\linewidth}
\setlength{\tymax}{0.5\linewidth}
\centering
\small
\begin{tabulary}{\textwidth}{@{}ll@{}} \toprule
 \textbf{Structure} & \textbf{Content guidance} \\
\midrule

 Proposed title & Your title should continue to capture succinctly research problem and aim \\
 Abstract & This should provide a succinct summary of your research aimed at a specialised audience \\
 Sect 1 - Introduction 1.1 Background to the research 1.2 Justification for the research & This section should provide an introduction to your research topic in its wider context (as background) and your justification of why the research is worth pursuing. It should be well articulated and supported by evidence \\
 Sect 2 - Literature review 2.1 Review of existing relevant knowledge 2.2 Critical summary, including knowledge gap to be addressed by the research & Your review should provide a critical account of your in-depth engagement with the academic (and other) relevant literature, including identifying key trends, ideas and possible knowledge gaps. Most of your citations should point to academic articles. Your critical summary should highlight key insights from your review and provide a strong justification for your proposed research. Both coverage and depth of your review matter. You should ensure that your review is well structured, with a logical narrative flow and your arguments are well supported by evidence \\
 Sect 3 - Research definition 3.1 Problem statement 3.2 Aim, objectives, tasks and deliverables 3.3 Knowledge contribution & You should ensure that your research problem is well articulated and appropriate for your course and your personal and professional circumstances, that your aim and objectives are consistent with research problem, that tasks and deliverables break down your objectives appropriately and are clearly related to your chosen research methods, and that the intended knowledge contribution of your research is clearly articulated \\
 Sect 4 - Research design 4.1 Evidence and data 4.2 Research strategy and methods 4.3 Research procedures 4.4 Ethical, legal and EDI considerations & This section should demonstrated your critical engagement with all elements of research design, including a detailed account of the data and evidence needed in your research, the research methods and research strategies you will to apply, and how you will apply them within your project. Your account should be supported by a clear rationale and insights from the related literature, and appropriately justified in relation to your research problem, aim and objectives. It should also demonstrate your careful consideration of ethical and legal matters, and that your research will comply with your course and university requirements \\
 Sect 5 - Analysis and interpretation 5.1 Summary and analysis of evidence 5.2 Summary of key findings 5.3 Interpretation in relation to aim and objectives & This section should demonstrate substantial progress towards gathering and analysing your data and evidence, and interpreting them in relation to aim and objectives. It should demonstrate a competent execution of your research design, present appropriate summaries of evidence and data, supported by raw data in an appendix if needed. Key findings should be clearly identified and logically connected to evidence, with good critical reflection on their implications for aim and objectives. \\
 Sect 6 - Assessment of your proposed research 6.1 Qualification fit 6.2 Personal and professional fit 6.3 Technical skills and resources needed 6.4 Statement of feasibility 6.5 Personal reflection on research process & In this section you should continue to argue how your research is a good fit across all criteria. You should provide a clear rationale as to why you think what you are proposing is feasible. You should also reflect on your growing understanding of the research process, including key learning and aspects you have found particularly challenging. \\
 Sect 7 - Planning, scheduling and risk assessment 7.1 Statement of progress 7.2 Key priorities in follow-up stage 7.3 Risk assessment & In this section you should reflect on the progress you have made in Stage 2 and establish your priorities for the next stage. You should also review your risk assessment as appropriate. \\
 References & You should keep your growing references in good order and ensure you apply the required bibliographical style consistently. Ideally, you should use a BMT to generate and integrate your references within your report \\
 Appendix - Extended abstract & If needed, you should include your draft extended abstract as an appendix. This should provide a structured summary of your research aimed at a generalist audience. \\
 Appendix - Raw evidence & If relevant, you should include a sample of your raw data as an appendix \\
 Appendix - Work schedule & You should include your revised work plan as an appendix \\
 Appendix - Risk assessment table & You should include your updated risk table as an appendix \\
\bottomrule

\end{tabulary}
\end{minipage}
\end{table}

\begin{question}[subtitle={Activity: Putting your report together}] Using your word processor of choice, and starting from your previous report, complete your Stage 4 report by applying the structure and guidance in Table 1, and making good use of your notes and summaries from all related activities you have carried out so far.

\begin{guidance}In this first pass at putting together your report, you should focus primarily on completeness, ensuring that each section includes at least draft content.
\end{guidance}\end{question}
%%Hack to correct tcbox behaviour
\color{black}

\paragraph{}
As in the previous stages, after you have filled in your report you should review and revise it iteratively until you are happy with your account, and are ready to move on.

Table 1 - Criteria to review your report

\begin{table}[htbp]
\begin{minipage}{\linewidth}
\setlength{\tymax}{0.5\linewidth}
\centering
\small
\begin{tabulary}{\textwidth}{@{}ll@{}} \toprule
 \textbf{Criteria} & \textbf{Prompts} \\
\midrule

 \textbf{Completeness} & Are all sections of the suggested structure completed in line with the guidance provided? \\
 \textbf{Good academic writing practices} & Have you applied good academic writing practices throughout? \\
 \textbf{Logical structure and flow} & Have you structured your narrative appropriately to ensure a logical flow of arguments? \\
 \textbf{Supporting references or evidence} & Are your key arguments supported by appropriate references or other evidence? \\
 \textbf{Citation and reference style} & Do all your citations and references comply with the required bibliographical style? \\
 \textbf{Avoiding plagiarism} & Have you acknowledged the work of others and distinguished it from your own appropriately? \\
 \textbf{Standard of English (or any modern language you use)} & Have you proof-read your report carefully to remove all typos and grammatical errors? \\
\bottomrule

\end{tabulary}
\end{minipage}
\end{table}

\begin{question}[subtitle={Activity: Reviewing your report}] Apply the criteria in Table 1 to review your current report and write up a summary of your assessment.

\begin{guidance}For each criteria, consider the related prompts to help you assess your report overall, and write down any further work needed for your next stage.
\end{guidance}\end{question}
%%Hack to correct tcbox behaviour
\color{black}

