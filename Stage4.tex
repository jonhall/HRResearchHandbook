\chapter*{Stage 4: Analysing data and evidence}

You've now reached Stage 4, which means your project end is rapidly approaching. In this stage you are in the midst of your data gathering and analysis, which is possibly the most exciting, yet demanding, part of your research: this is where you get an opportunity to make your original contribution to knowledge.

With reference to our 5-stage framework, the activities which are in focus in Stage 4 are summarised in Table~\ref{tab:stage4}, which also provides some guidelines for your interaction with your supervisor during this stage.


\begin{table}[htbp]
\caption{Research activities addressed in Stage 4 (20\% of project length)\label{tab:stage4}}
\small
\begin{tabulary}{\tablewidth}{@{}LLLLL@{}} 
\toprule
 \textbf{Research process activities} & \textbf{Deliverables} & \textbf{Learning Outcomes: by the end of this stage you will:} & \textbf{Effort} & \textbf{Suggested focus of your interaction with your supervisor} \\
\midrule

 \textbf{Identifying the research problem} & Research problem statement, refined as needed & be able to assess and improve your research problem statement & 1\% & \\
 \textbf{Reviewing the literature} & Substantial draft of your literature review, refined as needed & be able to assess and improve your current draft & 1\% &  \\
 \textbf{Setting your aim and objectives} & Aim and objectives, refined as needed & be able to assess and improve your aim, objectives and related tasks & 2\% & \\
 \textbf{Developing the research design} & Research design description, refined as needed &  select and justify analysis methods based on the data you have gathered; describe data analysis procedures  & 2\% & Suitability of methods and procedures \\
\textbf{Gathering and analysing evidence} & summary of the data/evidence gathered and their analysis & apply appropriate data analysis methods; present your data and evidence in a concise and effective way & 50\% & Appropriateness of data analysis and presentation\\
\textbf{Interpreting and evaluating findings} & draft summary of findings from data/evidenced gathered & derive findings from data analysis and critically assess them in relation to research aim and objectives & 15\% & Critical and logical thinking\\
 \textbf{Reflecting and reporting} & Stage 4 report; draft abstract and extended abstract for your project & know the difference between an abstract and an extended abstract, be able to assess your research progress and write up a substantial report, an abstract and an extended abstract & 25\% & Any further improvements required, particularly in relation to critical thinking and academic writing \\
 \textbf{Planning work and managing risk} & Updated risk and work plan & be able to assess risk and draw a work plan & 5\% & Any major adjustment required to address deficiencies or manage risk \\
\bottomrule
\end{tabulary}
\end{table}

\begin{question}[subtitle={Activity: Understanding the effort needed in this stage}] Consider Table~\ref{tab:stage4} carefully, paying particular attention to the entries in the `Effort' column. Make a note of the activities which are most prominent in this stage and what their deliverables and learning outcomes are.

\begin{solution}
Gathering and analysing evidence will constitute by far your major effort in this stage (50\% of study time): in particular, the framework assumes that you will have already gathered some data and evidence in Stage 3, so that you can start selecting and applying appropriate data analysis methods. The amount of evidence will depend on how much work you were able to complete in the previous stage, and it is very likely that you are still gathering more data and evidence. Nevertheless, you should have enough to start analysing, and thinking of possible findings in relation to your aim and objectives. 
\end{solution}\end{question}
%%Hack to correct tcbox behaviour
\color{black}

Note that your data analysis and interpretation may also prompt you to look for more evidence, including, perhaps, reviewing more academic literature or even re-think or adjust your current aim an objectives. Therefore, so you should expect some iteration back to activities you have carried out in previous stages, and revision of things you have written.

By the end of Stage 4 your data collection, analysis and interpretation should be on a solid ground, and consistent with your aim and objectives. Your research design description should also be close to its final form. If that's not the case, then you will need to re-consider carefully both your project risk and work plan. Given the criticality of this stage, it is essential that you work very closely with your supervisor throughout.

\section{Choosing and applying data analysis methods}
This section assumes that you have gathered a certain amount of data and evidence, so that you can proceed with some analysis\footnote{If that's not the case, then, you should go back to Stage 3, Section~\ref{sect:???} which focuses on data collection methods. You should also discuss your progress with your supervisor, revisiting your project risk around access to the required data and evidence.}.

Your choice of data analysis methods will depend on the following factors: the kind of data and evidence you have, and what you are trying to achieve, that is your aim and objectives, in the context of the research strategy(ies) you have chosen to apply.

This section provides an introduction to some common analysis methods. It is far from complete and does not go very deeply into the details of each method: entire books have been written on any of them! By studying this section, you won't become an expert in any of these methods, but you will have gained enough understanding to be able to make a judicious selection for your project. After that, you should review the related specialised literature to help you apply your chosen methods appropriately. You should also talk regularly to your supervisor for further guidance.

\subsection{Using tables}
In your research you may collect or make use of large amounts of raw data, possibly stored in a spreadsheet or database, so that a first step is for you to decide how to process them in order to facilitate your analysis and to generate summaries for your dissertation.

To this end, tables are used extensively in research and often found in dissertations.

\paragraph{Pivot tables}

Pivot tables can be used to summarise, sort, filter, re-organise or group data organised in rows and columns, and perform calculations on them, such as counting, generating totals or averages, and much more. Pivot tables are both powerful and versatile\footnote{In fact, they are so versatile that we'll only be able to provide few illustrative examples. Much, much more can be found online!}, and one of the most widespread tools for data analysis.

You can generate a pivot table from any data set organised in rows and columns, regardless of whether the values are quantitative or qualitiative: all common spreadsheet applications\footnote{From MS Excel to Apple Numbers to Google Sheets.} include this function.

Figure~\ref{fig:exampleDataSet} gives an example: these are the first few raws of a data set related to the US housing market\footnote{It was taken from one of Kaggle's free datasets, the housing price dataset. Kaggle is possibly the largest and best known online community for data science and machine learning.}. The dataset contains over 9,316 entries, each corresponding to a distinct property. Each property is characterised by a number of attributes: size in square feet, number of bedrooms and bathrooms, type of neighbourhood, the year it was built and its market price in US dollars. As you can see, this table includes both numerical and categorical variables.

\begin{figure}[htbp]
\centering
\includegraphics[width=\textwidth]{Figures/exampleDataSet.pdf}
\caption{First few rows of the example dataset}
\label{fig:exampleDataSet}
\end{figure}

Pivot tables can be used to summarise such data to answer certain questions. For instance, we may be interested in the average house price by neighbourhood and number of bedrooms, which would result in the pivot table in Figure~\ref{fig:pivot1}, which gives the average price of each combination. The `grand totals' in the table are also averages, by row and by column.

\begin{figure}[htbp]
\centering
\includegraphics[width=\textwidth]{Figures/pivot1.pdf}
\caption{Pivot table of average property prices by neighbourhood and number of bedrooms}
\label{fig:pivot1}
\end{figure}

Alternatively, we may be interested in finding out how many properties of each kind have been built in each neighbourhood. In this case the pivot table would look like that in Figure~\ref{fig:pivot2}. The grand totals in this case are counts. Note how we have added combinations of bedroom and bathroom numbers to characterise each type of property.

\begin{figure}[htbp]
\centering
\includegraphics[width=\textwidth]{Figures/pivot2.pdf}
\caption{Pivot table of property counts by neighbourhood and number of bedrooms/bathrooms}
\label{fig:pivot2}
\end{figure}

These are just but two examples of questions about the data you can address by using pivot tables, out of a vast range of the possibilities. If your data are organised in tables, then it is well worth spending some time becoming familiar with pivot tables.

\begin{question}[subtitle={Activity: Pivot tables in Excel}] Download the housing price data set from Kaggle and re-create the pivot tables in our example. Come up with other questions you could ask of the data and generate related pivot tables.	
\begin{guidance}
Feel free to use your preferred spreadsheet application for this activity, as long as it supports pivot tables -- most do.

You may have to register with Kaggle to gain access to the data set. 

The Excel help facility and documentation provides all the info you need to create a pivot table. However, you could also browse some of the very many freely available online resources and tutorials on this topic.
\end{guidance}
\end{question}	


\paragraph{Frequency and contingency tables}
Frequency tables are used to summarise the frequency (or count) of values taken by a categorical variables in a data set. For instance, after studying a degree, a student's outcome may be classed as distinction, merit, pass or fail. A frequency table can then be used to summarise the frequency of each class of outcome for a particular students' cohort, as shown in Table~\ref{tab:frequencyTableExample}.
 
\begin{table}[htbp]
\caption{Example of frequency table\label{tab:frequencyTableExample}}
\small
\begin{tabulary}{\tablewidth}{@{}LLLLL@{}} 
\toprule
  & \textbf{Distinction} & \textbf{Merit} & \textbf{Pass} & \textbf{Fail} \\
\midrule
 \textbf{Outcome} & 12 & 26 & 42 & 5\\
\bottomrule
\end{tabulary}
\end{table}

Contingency tables\footnote{Also known as \textit{cross-tabulation} tables.} are a form of frequency tables used to tabulate the value frequencies of two categorical variables. For instance, following from our previous example, we may like to tabulate the outcome value frequencies in the cohort against gender, as shown in Table~\ref{tab:contingencyTableExample}. 

\begin{table}[htbp]
\caption{Example of contingency table\label{tab:contingencyTableExample}}
\small
\begin{tabulary}{\tablewidth}{@{}LLLLL@{}} 
\toprule
 \textbf{Outcome by Gender} & \textbf{Distinction} & \textbf{Merit} & \textbf{Pass} & \textbf{Fail} \\
\midrule
\textbf{Female} & 7 & 12 & 21 & 2\\
\textbf{Male} & 5 & 13 & 19 & 3\\
\textbf{Other} & 0 & 1 & 2 & 0\\
\midrule
\textbf{Totals} & 12 & 26 & 42 & 5\\
\bottomrule
\end{tabulary}
\end{table}

Contingency tables are frequently used to summarise and analyse data collected in survey research, and are a key tool in statistical analysis.

Both frequency and contingency tables can be generated as pivot tables in a spreadsheet. In fact, the table in Figure~\ref{fig:pivot2} is a contingency table. 

\subsection{Statistical analysis}

Statistical analysis in an umbrella terms for a set of methods which can be applied to quantitative and categorical data\footnote{Refer back to Section~\ref{sect:??} for a definition.}. There are two broad categories:
\begin{itemize}
	\item descriptive statistics, whose aim is to describe data; and
	\item inferential statistics, whose aim is to make predictions from data.
\end{itemize}

\subsubsection{Descriptive statistics} 
This is used to describe various attributes of a data set. The basics are:
\begin{itemize}
	\item count, to establish how many entries there are in the data set
	\item range, to establish the span of the data set as the difference between its smallest (minimum) and largest (maximum) values 
	\item centrality, to establish the `centre' of the data set. Three measures are commonly used: the \textit{mean}, which provides the average value of the data set; the \textit{median}, which provides its mid point\footnote{Remember that quantitative data can be ordered.}; and the \textit{mode}, which indicates the value that occurs most frequently, if any\footnote{There is no mode if no value is repeated in the data set.}
	\item dispersion, to establish how close to the mean or otherwise the values in the data set are.   The \textit{standard deviation} is a common measure used for this purpose\footnote{It is based on a mathematical formula which considers the distance of each value in the data from the mean. It is not essential for you to know such formula, which is automatically computed by spreadsheets and statistical software. Of course, you can always look it up the literature...}. The larger the standard deviation, the greater the dispersion.
	\item skewness, to establish how symmetrically distributed the values in the data set are in relation to the centre. In the case of perfect symmetry, skewness is equal to zero, and mean and median are equal. When asymmetric, mean and median are different and the distribution may be either right (mean smaller than median, and negative skewness) or left (mean greater than median, and positive skewness) skewed.   
\end{itemize}

%In addition, two other attributes are defined relative to a \textit{normal distribution}\footnote{Or \textit{Gaussian distribution}, from one of the mathematicians, Carl Friedrich Gauss, who helped develop this concept.}: you will hear a lot about the normal distribution if you study or apply any statistics, so it is well worth becoming familiar with this concept.
%
%A normal distribution represents data that when plotted on a graph results in a \textit{bell curve}, exemplified in Figure~\ref{fig:bellcurve}. 

These are lots of definitions to digest, particularly if you haven't encountered these terms before! The following activity should help.

\begin{question}[subtitle={Activity: Descriptive statistics in Excel}] Assume you have measured the weight in grams of each apple in a basket, obtaining the following numbers: 105, 120, 122, 125, 127, 128, 129, 130, 132, 133, 135, 135, 138, 140, 128. Enter these data in an Excel sheet and use its built-in data analysis function to generate the related descriptive statistics.
\begin{guidance}
In the current version of Excel, you can access this function from the Data tab, by pressing the Data Analysis button. If you find it difficult to locate this function, you should check some of the many tutorials on this topic which are freely available online.
\end{guidance}
\begin{solution}

You should have obtained the following values:

\begin{tabulary}{\tablewidth}{@{}LL@{}} 
\textbf{Attribute} & \textbf{Value} \\
mean & 128.47 \\
median & 129 \\
mode & 128 \\
standard deviation & 8.55 \\
skewness & -1.4 \\
range & 35 \\
minimum & 105 \\
maximum & 140 \\
count & 15 \\
\end{tabulary}

There are 15 values in this data set, with range 35 (the difference between maximum and minimum values).
In terms of centrality, the mean (128.47) is slightly smaller than the median (129), and Excel reports a mode at 128. In reality, if you look at the data you will see that there are two modes in this data set\footnote{Statisticians call this bi-modal.}, 128 and 135, but Excel only returns the first encountered!
In terms of dispersion, the standard deviation is telling us that most apple weights are within 8.55 grams below or above the mean, so the apple weights are similar in the apple baskets.
Note that the skewness is negative, which is consistent with the mean being smaller than median, so the data distribution is right skewed. 
\end{solution}\end{question}
%%Hack to correct tcbox behaviour
\color{black}

In your dissertation, you can easily present such descriptive statistics as a table, possibly adapting that automatically generated by your spreadsheet.

In addition, charts can be used to visualise the data and examine their descriptive statistics. 

With discrete data, like in our example, you can use a \textit{bar chart}. The one in Figure~\ref{fig:barchart} uses the apple weights from the previous activity: on the horizontal, we have the distinct weights, and on the vertical, their frequencies, that is how many times each weight appears in the data set. Given the values you have obtained for the data set descriptive statistics, you can easily locate on the chart min and max values, and mean, median and mode. In this case, the two `peaks' correspond to the two modes we mentioned in the activity. You can also check that most of the values are within 8.55 from the mean, either way: the only values left out are 105 (to the left) and 138 and 140 (to the right). Skewness is not obviously visualised on this bar chart, so that we will use a different chart for that.

\begin{figure}[htbp]
\centering
\includegraphics[width=\textwidth]{Figures/barchart.pdf}
\caption{Bar chart for the apple weights}
\label{fig:barchart}
\end{figure}

Before we do that, however, it is worth noticing that while a bar chart works for this example, more generally you should use a \textit{histogram} for  quantitative data, particularly for continuous numbers.  Figure~\ref{fig:histogram} illustrates a possible histogram for our example. In this case, the apple weights are aggregated in `bins', each spanning a set of values, which depends on the size of the bin: we have used a bin size of 5 in the figure. The frequencies are then by bin, rather than individual weights. In our example, however, it is easier to visualise and investigate the descriptive statistics with a bar chart.

\begin{figure}[htbp]
\centering
\includegraphics[width=12cm]{Figures/histogram.pdf}
\caption{Histogram for the apple weights, with bin size equal to 5}
\label{fig:histogram}
\end{figure}

In order to visualise skewness, a useful chart is the boxplot, illustrated in Figure~\ref{fig:boxplot} for our example. This is made of a `box' around the median of the data, and some `whiskers' on each side of the box\footnote{Which is why this chart is also called a \textit{box and whiskers} plot.}. It is obtained by dividing up the data into quartiles, each containing a quarter (or 25\%) of the data, with the median in the centre. The box includes the two quartiles on each side of the median, which, together, account for half of the values in the data set. The whiskers account for the two other quartiles, with a caveat: if there are very extreme values, these are treated as possible outliers and left out of the whiskers. That is, in fact, the case in our example where value 105 is treated as an outlier in the chart: it is a dot on its own, not included in the left whisker. The whisker length provides an indication of skewness: in our example the right whisker is longer, so the data distribution is right skewed (consistent with the negative skewness value in the descriptive statistics).

\begin{figure}[htbp]
\centering
\includegraphics[width=\textwidth]{Figures/boxplot.pdf}
\caption{Boxplot for the apple weights}
\label{fig:boxplot}
\end{figure}

To be more precise, the relation between a boxplot and its underlying statistical features is illustrated in Figure~\ref{fig:boxplotFeatures}. The two quartiles around the median represent the interquartile range (IQR) of the data set, with the whisker lengths calculated based on the formulae in the figure, which allow the identification of extreme values, represented separately as outliers. An outlier, therefore, is just a value which is distant from most of the other values in the data set: it may point to an error, which should be corrected, or an anomaly, which may require further investigation, but that's not necessarily the case. In all cases, it's good practice to investigate outliers to understand why they have occurred. 

\begin{figure}[htbp]
\centering
\includegraphics[width=\textwidth]{Figures/boxplotFeatures.pdf}
\caption{The features of a boxplot --- LR to redraw as taken from the web}
\label{fig:boxplotFeatures}
\end{figure}


\begin{question}[subtitle={Activity: Charts in Excel}] Go back to your Excel sheet from the previous activities and generate charts similar to those in the figures above.
\begin{guidance}
In the current version of Excel, you can generate these charts from the Insert tab, by choosing from the Statistical charts menu. If you find it difficult to locate this function, you should check some of the many tutorials on this topic which are freely available online.
\end{guidance}
\end{question}

Calculating descriptive statistics and visualising the data in bar charts, histograms and boxplots, is often the first step in quantitative data analysis, as these provide useful summaries and visualisations of key properties of a data set. And as you have found out in the activities, you do not need to be a statistician to be able to generate them!

Descriptive statistics may also help you identify errors or anomalies in the data, and can inform possible follow-up analysis, including inferential statistical analysis. Depending on your research aim and objectives, they could also be all you need in your project.

If you have collected qualitative data, it is time for you to have a go at analysing them using descriptive statistics.

\begin{question}[subtitle={Activity: Applying descriptive statistics to your data}] 
Calculate descriptive statistics for your data set, and generate appropriate charts.
\begin{guidance}
MS Excel is relatively straightforward to use for this purpose, but feel free to use other tools  you may be familiar with, including statistical or data analytics packages. If you do, you should ensure they support the functions we have discussed in this section.
\end{guidance}
\end{question}


***to do -- summarise which descriptive statistics apply to different kind of data: nominal, interval/ratio and ordinal

\subsubsection{Inferential statistics}
Inferential statistics relies on the concepts of population and sample: the \textit{population} is the entire group you are interested in studying -- say, all UK voters in a general election; while the \textit{sample} is the portion or subset of that group you have access to in your research. Then the aim of inferential statistics is to establish whether patterns or effects you have observed in the sample can be generalised to, i.e., inferred for, the whole population, or whether they are the result of chance. For this, it is essential that the sample is representative of the population, something we will consider in some detail in Section~\ref{sect:sampling}.

In inferential statistics this is achieved through statistical tests. A statistical test tells you whether the proposition\footnote{You can think of a proposition as an educated guess you have made based on some observations, but that has yet to be supported by evidence.} you wish to test on your sample is likely to be true in the population under study. To do so, a statistical test returns a measure of \textit{statistical significance}, which is used to provide evidence (or otherwise) that the pattern or effect you see in your sample is also likely to exist in the population, and is not just the effect of chance\footnote{Contrary to the common language meaning of 'significance' as big or important. Statistical significance is only an indication that the effect is likely to exist in the population, where it may well be small or unimportant!}. As a corollary, if your sample is very large, almost all effects observed in the sample will be likely present in the population; vice-versa, if your sample is very small, most effects observed in the sample are unlikely to be present in the population, unless they are really very large.

Each statistical test comprises the following elements.

\paragraph{Hypotheses} There are two,  \textit{null and alternative} hypotheses. Inferential statistics assumes you can't prove something to be true, but you can disprove something by finding an exception. The classic example: you can't prove that all swans are white, but you can disprove they are by finding a black swan! So, you must  set the null hypothesis to what you want to disprove about the population, with the alternative hypothesis being what you are really interested to find out. So, the null hypothesis is a statement of no pattern/effect in the population.

\paragraph{Significance} This is level of statistical significance for the test. It's known as the alpha ($\alpha$) value from the Greek name of the mathematical variable used to express it. Most tests are run with $\alpha=0.05$, which gives a 5\% probability that we may infer that the null hypothesis is disproved while in actually it is correct\footnote{This is called a Type I error in Statistics.}.

\paragraph{Sample} You need to have a (representative) sample of the population on which to perform the test.

\paragraph{p-value} This is the probability calculated for you test by your statistical package, and which is used to decide the outcome of the test.

\paragraph{Decision} This is based on the p-value in relation to the $\alpha$ value: if the p-value is less than the $\alpha$ value, then the null hypothesis is \textit{rejected}, i.e. disproved, which means your alternative hypothesis that there is an effect in the population is supported by statistical evidence.








The measure of statistical significance returned by a statistical test is the \textit{p-value}, which is a probability.


In inferential statistics there are very many tests to choose from, which depend on the data you have, say nominal vs numerical, how many samples are involved in the analysis, and the purpose of your analysis, say, comparing samples or identifying relationships.


***go through the Dr Nic's tutorials and then summarise

Among the most common inferential statistical tests are:

tests for means, tests for proportions and testes for relationships

choice factors: data; sample; purpose



\begin{itemize}
\item 1 sample t-test for mean
\item 1 sample z-test for mean	
\end{itemize}


\begin{itemize}
	\item t-test: to compare the means of two groups to establish if they are genuinely different or if the difference is just the effect of chance. For instance,  a university may like to test whether there are significant differences in mean scores of students studying the same subject with different teachers. The underlying assumption is that the two student groups are \textit{independent}, that is there is no reason to believe that the performance of one group is influenced or determined by that of the other.
	\item paired t-test: the same as the t-test, but focusing on the performance of one particular group over time. For instance, to establish if the mean scores of a particular student group change from one study module to the next within a degree.
	\item one-way ANOVA: to compare the means of multiple groups. For instance, in a country with standardised school tests, an education authority may like to compare the mean scores of students from different schools taking the same test. There are several flavours of ANOVA tests.
	\item chi-square: to establish if there are differences in the proportion of different categories, so this test applies to categorical variables\footnote{Which can be either quantitative (ordinal) or qualitative (nominal), so that this test can also be used as a qualitative analysis method.}. In this way, this test allows you to establish whether there are relationships between categorical variables. For instance, a local authority may like to find out if there are gender differences in the choice of modes of transport.
	\item correlation: to assess the relationship between two numerical variables, to establish whether they change together in some way, specifically if an increase in one  is associated with an increase or decrease in the other. For instance, you could test if there is a correlation between the time a person spends watching television and their physical fitness. This test produces a coefficient between -1 and 1, which tells you both the strength of the correlation and its direction: a coefficient close to 1 means that the two variables always change together and in the same direction (either up or down); a coefficient close to -1 indicates that they change together, but in opposite directions (one up, one down); and a coefficient close to 0 means that they change independently. Importantly, correlation is not causation: only because two variables change together, does not mean that changes in one cause the other to change.
	\item regression: to make predictions on the value of one variable (known as the \textit{dependent} variable) based on the value of another variable or a set of other variables (known as the \textit{independent} variable(s)). For instance, you may use regression to estimate house prices (dependent variable) based on a number of variables such as location, type of house, number of bedrooms, etc. (independent variables). Regression too is based on correlation and does not imply causation between independent and dependent variables. There are many different types of regression.
\end{itemize}

\paragraph{Choosing an appropriate test}

** to do -- criteria and activities to test understanding of the different tests based on given scenarios

\subsubsection{Sampling}\label{sect:sampling}
Inferential statistics require the sample to be representative of the population of interest, with sampling the process of selecting a sample from the population. 

Broadly, sampling is classified into random\footnote{Also known as \textit{probability} sampling.} and non random. In random sampling an algorithm is established to extract the sample from the population, while in non-random sampling, the choice is based on the judgement and discretion of the researcher. The former is used when the ability to generalise to the population is paramount, while the latter when the depth and richness of the sample is more important. As a result, the former is used more in quantitative research, and the latter in qualitative research.

Random sampling include:
\begin{itemize}
	\item simple random sampling, where each member of the population has exactly the same chance to be selected. It is easy and efficient to implement, and given the complete randomness of the sample, generalisation is fairly reliable. However, if the population has large sub-groups, these may be over-represented in the sample, with minority groups being under-represented 
	\item stratified random sampling, where sub-groups of the population are defined based on common characteristics, the \textit{strata}, and sampling is random across those strata. The strata not mutually exclusive: for instance, the population may have sub-groups defined by gender, ethnicity and level of education, which may overlap. This gets round the over/under representation problem of simple random sampling.
	\item cluster sampling, where the population is divided up in naturally occurring separate clusters, and the sample is obtained by randomly selecting some clusters and then randomly selecting members of those clusters. 
\end{itemize}

Non random sampling include:
\begin{itemize}
	\item subjective\footnote{aka purpose -- check terminology in Rosen's boob!} sampling, the researcher chooses based on the researcher's own judgement and research aim. Used for small populations or especially rare population; target individuals with particular knowledge or expertise; prone to bias and unlikely to lead to generalisable results. Better suite for narrow and deep
	\item convenient sampling, participants selected based on their availability or accessibility. Quick and easy, but unlikely to reproduce a representative sample, and prone to bias
	\item snowball sampling, rely on referral from previous participants to recruit new ones. When difficult to identify or access a population or when the topic is sensitive or tabu. Highly prone to bias and unlikely to reproduce a representative sample
\end{itemize}


how to choose --- 
factors: aim and resources

aim: generalisable findings vs riche and deep insight
resources and practical constraints: multiple options 

Trade-offs are needed. As long as you explain and justify and acknowledge the limitations, it's ok.


\subsection{ML methods???}


%%%use of generative AI to add to ethics section
%constraints on the use of generative AI in research:
%- university policies: check with your university!!!! generative AI and IP issues
%- technology capabilities: still developing; as a human researcher you need to check and double check anything that AI does


Complementary to statistical methods, an increasing variety of Machine Learning algorithms can also be used for data analysis. Two broad application concern the recognition of patterns in data and making predictions based on historical data.


tasks that you can fast track with AI
- 
**** more about this????


issue -- black box nature




\subsection{Choosing the right method for your research}
Type of data and your research aim and objective


type of data table, and which method apply!!!!!

https://stats.oarc.ucla.edu/other/mult-pkg/whatstat/


Skewness -- and methods which apply


Nature of research questions, and related hypothesis

1) assess characteristics of your sample --> descriptive
2) predict outcomes --> both descriptive and inferential




\subsection{Qualitative data analysis -- to do}



\section{Writing up your analysis --- to do}

Building on your pilot work, in this stage you will apply your chosen methods and procedures to gather and analyse your data and evidence. Remember that this may well include modelling, design or prototyping work, depending on your chosen research strategy.

You are expected to make significant progress on this aspect of your research during Stage 4, therefore, it is important that you plan your work carefully in consultation with your supervisor.

Depending on your chosen research strategy and methods, you may end up gathering a large quantity of data or evidence, so you will need to think carefully at how you will handle it. In this respect, you will need to consider:

- How to organise and store your `raw' data and evidence. These represent anything you gather directly, say, full transcripts of participants' interviews, survey responses, logs of experimental data, copies of documents, etc. It is essential that you manage them carefully to ensure you don't loose track of important information, and that you can always refer back to them in your subsequent analysis. You may need to include such evidence or a sample of it in an appendix of your dissertation as demonstration of the work you have done.

- How to summarise data and evidence in the body of your stage report, and later in your final dissertation. This will depend on the nature of your data and evidence, and you will need to ensure that your summaries are appropriate to convey the essence of the evidence you have generated and to support your analysis, so that you can build academic arguments which relate evidence to findings, and findings to aim and objectives.

- How to structure your report. Depending on your chosen research strategies and methods, different structures are possible. For instance, you may choose to start with a section which summarises all your evidence followed by one in which you analyse it, which may work well, for instance, for Survey Research. Alternatively, you could have separate sections each including a summary and analysis of a sub-set of your evidence: this may be appropriate for mixed methods research, with each section dealing with a different kind of data, or for Design Science Research, with each section addressing a different design cycle. Whatever you choose, it is important that your report is effective in presenting your evidence and findings in a clear and rigorous manner.

\begin{question}[subtitle={Activity: Gathering and analysing your data and evidence}] Plan and execute your research on gathering and analysing data and evidence, and write up your summary and analysis. Ensure that you plan carefully how to manage your raw data\slash evidence and how to structure your report.

You should aim to complete the bulk of this work by the end of Stage 4.

\begin{guidance}To complete this work successfully, it is essential that you discuss and agree your detailed work plan with your supervisor from the start, and monitor your progress on a regular basis.

This is a substantial activity, which will take up around 50\% of your study time in this stage.
\end{guidance}\end{question}
%%Hack to correct tcbox behaviour
\color{black}

\section{Interpreting and evaluating findings}
Having gathered and analysed your data and evidence, you must now interpret your findings in relation to your aim and objectives, and generally evaluate them in terms of their contribution to knowledge and possible limitations.

By interpretation I mean being able to answer the following questions:

\begin{itemize}
\item What does the analysis of your data\slash evidence you have conducted show? What are your findings?

\item How do your findings relate to your aim and objectives?

\item How do your findings relate to what you know from the literature or from professional practice?

\item Which new knowledge do your findings contribute?

\item What do your findings fail to achieve?

\end{itemize}

\subsubsection{Activity: Interpreting and evaluating your initial findings}
Consider your analysis of data and evidence, and based on it, address each of the above questions. In doing so, you should express well-formed academic arguments, with explicit reference to the data and evidence you have gathered and analysed.

\begin{guidance}You can refresh your understanding of academic arguments by looking back at the related materials for Stage 1.

Your interpretation and evaluation of findings will be, of course, limited by the data\slash evidence you have gathered and analysed up to this point. You will revisit and expand this work in Stage 5 in order to complete your project.

Depending on the extend of your current data\slash evidence gathering and analysis, this activity could take you up to 15\% of your study time.

\end{guidance}

\section{Drafting an abstract for your project}

An abstract is a common way to summarise academic research. Abstracts are an integral parts of all published academic articles -- you will have encountered many abstracts while reviewing the literature. They are also very common in academic dissertations, therefore it is highly likely you will be required to include one at the beginning of yours.

An {abstract} provides a short summary of the whole research written for a specialist audience, that is you can assume the reader has good knowledge of the topic and field of study. It should be a stand-alone item, so that it can be understood without reference to any other part of the article or the dissertation.

Its content should convey succinctly the research problem, how and where it arises and its significance, the research aim and research design, key results obtained by the research, their evaluation and their implications for further research or professional practice. 

Writing an abstract for your research is a good exercise, even if one is not needed for your dissertation, as it gives you an opportunity to write a logical argument that connects all key elements of your research. This can help you check that all the pieces fit together in a coherent manner. It is also something you can share with your supervisor and critical friends to communicate succinctly the essence of what you have done and achieved.  

\begin{question}[subtitle={Activity: Drafting your abstract}]
Write a draft abstract for your project, which should reflect your research progress to date. 
\begin{guidance}
You should go back to some of the articles you have reviewed to consider the content and structure of their abstract. Choose a structure which may fit your project and write up your draft abstract accordingly.

As your research is yet to be completed, you will not be able to write up the full abstract, but you should end up with a draft that you can easily complete by the end of your project.
\end{guidance}
\end{question}
%%Hack to correct tcbox behaviour
\color{black}


%%%LR -- I don't think we need this, which is a T802 thing
%\subsection{The extended abstract} 
%
%An {extended abstract} is a summary of academic research intended for a more general audience, so that it should be easily read and understood by someone with only a superficial knowledge of the topic. As with the abstract, it should be a stand-alone item without any reference to your full dissertation. However, it is a lengthier piece of academic writing, structured with headings and sub-headings, including citations and references, and possibly tables, figures and diagrams to help you present and summarise your work.
%
%\begin{question}[subtitle={Activity: Drafting your extended abstract}] Write a draft extended abstract for your project, which should reflect your research progress to date.
%
%\begin{guidance}Your extended abstract should be 4 to 6 pages in length (once complete) and a common structure is as follows:
%
%\begin{itemize}
%\item Title --- the same as your dissertation
%
%\item Introduction and background --- an outline of your research problem in its context, its significance, and the knowledge gap addressed by your research
%
%\item Aim and objectives --- from your dissertation
%
%\item Research design --- an outline of your research design
%
%\item Results --- a summary of the evidence collected and analysed, and your key findings
%
%\item Discussion --- how significant your findings are in relation to research problem and knowledge gap
%
%\item Conclusion and future work --- your overall conclusions and possible follow-up research
%
%\item References --- selected references cited in the body of your extended abstract
%
%\end{itemize}
%
%Your course may have different guidelines which you should check and follow to produce your extended abstract.
%
%\end{guidance}\end{question}
%%%Hack to correct tcbox behaviour
%\color{black}

\section{Reflecting and reporting in Stage 4 -- to complete}

It's time to write your Stage 4 report. As in the previous stages, before you do, it is time for more reflection. 

\begin{question}[subtitle={Activity: Reflecting on your learning and practice}]
As you did at the end of the previous stages, in this activity you are asked to stand back and reflect deeply on what you have leant and done, the wider context of your work and your own attitude to it. Specifically, you are asked to think deeply about each of the following:

\begin{itemize}
	\item your study this far
	\item the way you work. Are you tidy and systematic, or let things happen organically? For instance, how does
	\item the context of your research
	\item your feelings about your project
\end{itemize}

You should also think of any significant changes with respect to your reflection in the previous stages
\begin{guidance}
You should be accomplished at reflection by now. However, should you need to, you can refer back to the guidance to this activity in Stage 1, Section~\ref{sect:stage1Reflection}.
\end{guidance}\end{question}
%%Hack to correct tcbox behaviour
\color{black}

Your end-of-Stage 4 report will help you consolidate your work so far, adding yet another increment toward you full dissertation. We recommend you follow the guidance in Table~\ref{tab:S4report} to write your report.

\textbf{LR -- table and follow up activity still to do -- reuse stage 3 materials}

At the end of Stage 4, you should complete a report, extending that of Stage 3 and covering the work you have carried on in this stage. Its recommended structure and content are indicated in Table 1.

Table 1 -- Report structure and content guidance

\begin{table}[htbp]
\begin{minipage}{\linewidth}
\setlength{\tymax}{0.5\linewidth}
\centering
\small
\begin{tabulary}{\textwidth}{@{}ll@{}} \toprule
 \textbf{Structure} & \textbf{Content guidance} \\
\midrule

 Proposed title & Your title should continue to capture succinctly research problem and aim \\
 Abstract & This should provide a succinct summary of your research aimed at a specialised audience \\
 Sect 1 - Introduction 1.1 Background to the research 1.2 Justification for the research & This section should provide an introduction to your research topic in its wider context (as background) and your justification of why the research is worth pursuing. It should be well articulated and supported by evidence \\
 Sect 2 - Literature review 2.1 Review of existing relevant knowledge 2.2 Critical summary, including knowledge gap to be addressed by the research & Your review should provide a critical account of your in-depth engagement with the academic (and other) relevant literature, including identifying key trends, ideas and possible knowledge gaps. Most of your citations should point to academic articles. Your critical summary should highlight key insights from your review and provide a strong justification for your proposed research. Both coverage and depth of your review matter. You should ensure that your review is well structured, with a logical narrative flow and your arguments are well supported by evidence \\
 Sect 3 - Research definition 3.1 Problem statement 3.2 Aim, objectives, tasks and deliverables 3.3 Knowledge contribution & You should ensure that your research problem is well articulated and appropriate for your course and your personal and professional circumstances, that your aim and objectives are consistent with research problem, that tasks and deliverables break down your objectives appropriately and are clearly related to your chosen research methods, and that the intended knowledge contribution of your research is clearly articulated \\
 Sect 4 - Research design 4.1 Evidence and data 4.2 Research strategy and methods 4.3 Research procedures 4.4 Ethical, legal and EDI considerations & This section should demonstrated your critical engagement with all elements of research design, including a detailed account of the data and evidence needed in your research, the research methods and research strategies you will to apply, and how you will apply them within your project. Your account should be supported by a clear rationale and insights from the related literature, and appropriately justified in relation to your research problem, aim and objectives. It should also demonstrate your careful consideration of ethical and legal matters, and that your research will comply with your course and university requirements \\
 Sect 5 - Analysis and interpretation 5.1 Summary and analysis of evidence 5.2 Summary of key findings 5.3 Interpretation in relation to aim and objectives & This section should demonstrate substantial progress towards gathering and analysing your data and evidence, and interpreting them in relation to aim and objectives. It should demonstrate a competent execution of your research design, present appropriate summaries of evidence and data, supported by raw data in an appendix if needed. Key findings should be clearly identified and logically connected to evidence, with good critical reflection on their implications for aim and objectives. \\
 Sect 6 - Assessment of your proposed research 6.1 Qualification fit 6.2 Personal and professional fit 6.3 Technical skills and resources needed 6.4 Statement of feasibility 6.5 Personal reflection on research process & In this section you should continue to argue how your research is a good fit across all criteria. You should provide a clear rationale as to why you think what you are proposing is feasible. You should also reflect on your growing understanding of the research process, including key learning and aspects you have found particularly challenging. \\
 Sect 7 - Planning, scheduling and risk assessment 7.1 Statement of progress 7.2 Key priorities in follow-up stage 7.3 Risk assessment & In this section you should reflect on the progress you have made in Stage 2 and establish your priorities for the next stage. You should also review your risk assessment as appropriate. \\
 References & You should keep your growing references in good order and ensure you apply the required bibliographical style consistently. Ideally, you should use a BMT to generate and integrate your references within your report \\
 Appendix - Extended abstract & If needed, you should include your draft extended abstract as an appendix. This should provide a structured summary of your research aimed at a generalist audience. \\
 Appendix - Raw evidence & If relevant, you should include a sample of your raw data as an appendix \\
 Appendix - Work schedule & You should include your revised work plan as an appendix \\
 Appendix - Risk assessment table & You should include your updated risk table as an appendix \\
\bottomrule

\end{tabulary}
\end{minipage}
\end{table}

\begin{question}[subtitle={Activity: Putting your report together}] Using your word processor of choice, and starting from your previous report, complete your Stage 4 report by applying the structure and guidance in Table 1, and making good use of your notes and summaries from all related activities you have carried out so far.

\begin{guidance}In this first pass at putting together your report, you should focus primarily on completeness, ensuring that each section includes at least draft content.
\end{guidance}\end{question}
%%Hack to correct tcbox behaviour
\color{black}

As in the previous stages, after you have filled in your report you should review and revise it iteratively until you are happy with your account, and are ready to move on.

Table 1 - Criteria to review your report

\begin{table}[htbp]
\begin{minipage}{\linewidth}
\setlength{\tymax}{0.5\linewidth}
\centering
\small
\begin{tabulary}{\textwidth}{@{}ll@{}} \toprule
 \textbf{Criteria} & \textbf{Prompts} \\
\midrule

 \textbf{Completeness} & Are all sections of the suggested structure completed in line with the guidance provided? \\
 \textbf{Good academic writing practices} & Have you applied good academic writing practices throughout? \\
 \textbf{Logical structure and flow} & Have you structured your narrative appropriately to ensure a logical flow of arguments? \\
 \textbf{Supporting references or evidence} & Are your key arguments supported by appropriate references or other evidence? \\
 \textbf{Citation and reference style} & Do all your citations and references comply with the required bibliographical style? \\
 \textbf{Avoiding plagiarism} & Have you acknowledged the work of others and distinguished it from your own appropriately? \\
 \textbf{Standard of English (or any modern language you use)} & Have you proof-read your report carefully to remove all typos and grammatical errors? \\
\bottomrule

\end{tabulary}
\end{minipage}
\end{table}

\begin{question}[subtitle={Activity: Reviewing your report}] Apply the criteria in Table 1 to review your current report and write up a summary of your assessment.

\begin{guidance}For each criteria, consider the related prompts to help you assess your report overall, and write down any further work needed for your next stage.
\end{guidance}\end{question}
%%Hack to correct tcbox behaviour
\color{black}

\section{Takeaways -- to do }


