\chapter{Stage 3: Second research increment}

By now you will have some mastery of the techniques and tools that you need to \enquote{do} research at masters level. You may have ideas about what you still need to do\footnote{If not, don't worry – we've got you covered with this chapter!} and so will have ideas about what come next.

With those skills, you're developing into an independent researcher\footnote{Being an independent researcher isn't one of the examined outcomes of masters research, but that you're feeling confident in your research is good.} and you may feel that this book holds nothing more for you. 

Stay with us a little longer though: the next sections aren't as long as those that you've studied already – you'll be doing more yourself, honing the skills you've picked up as you go along – but they might help to keep you systematic and on the path to submission. 

\newcommand{\percentthroughbook}{
%\thepage
%\getpagerefnumber{LastPage}
\intcalcDiv{\intcalcMul{\thepage}{100}}{\getpagerefnumber{LastPage}}\%
}

You won't be surprised to know that stage 3 comes next; there's another research increment coming.

You're now $\percentthroughbook{}$ through this book – not long to go before you're finished. Stick with us for a while longer.

\section{Introducing stage 3}

In Stage 3 you will focus on adding detail to both your aim and objectives and your research design. Stage 3 assumes that you have completed your Stage 2 work, and possibly discussed it with your supervisor\footnote{If your proposal still requires some `remedial' work to fully satisfy your course requirements then you should carry that out before moving on.}, particularly your research design choices.

With reference to our 5-stage framework, the activities which are in focus in Stage 3 are recalled in Table~\ref{tab:stage3acts}, which also provides some guidelines for your interaction with your supervisor during this stage.

\begin{table}[htbp]
\caption{Stage 3 activities\label{tab:stage3acts}}
\centering
\small
\begin{tabulary}{\tablewidth}{@{}LLLL@{}} \toprule
 \textbf{Research activities} & \textbf{Stage 3} \textbf{(20\% of project length)} & \textbf{Effort within stage} & \textbf{Suggested focus of your interaction with your supervisor} \\
\midrule

 \textbf{Identifying the research problem} & Adjust, if needed & 2\% & \\
 \textbf{Reviewing the literature} & Adjust, if needed & 3\% & \\
 \textbf{Setting research aim and objectives} & Finalise aim and objectives, and define tasks and deliverables & 10\% & Suitability of tasks and deliverables from objectives \\
 \textbf{Choosing the research design} & Complete research design, with detailed consideration of data and evidence, research strategy, research methods and procedures & 20\% & Suitability of research procedures \\
 \textbf{Gathering and analysing evidence} & Conduct pilot work to test aspects of your research design & 35\% & Scope of your pilot work \\
 \textbf{Interpreting and evaluating findings} & n/a & 0\% & \\
 \textbf{Reporting, critical reflection and conclusions} & Assess research progress and write up Stage 3 report & 25\% & Any further improvements required \\
 \textbf{Work planning and risk management} & At stage start, review work from previous stage and project risk; adjust plan as needed If you have received feedback from supervisor on your previous stage work, adjust plan to include any revision recommended & 5\% & Any major adjustment required \\
\bottomrule

\end{tabulary}
\end{table}

\begin{question}[subtitle={Activity: Understanding the effort needed in this stage}] Consider Table~\ref{tab:stage3acts} carefully, taking notice of the entries in the `Effort within stage' column. Write down the most time-consuming activities in this stage and what is expected under each.

\begin{solution}Developing your research design further and conducting your pilot work will constitute your major effort in this stage (55\% of the study time in total): your pilot work will be an initial test of some aspects of your research design, including a proof-of-concept application of some of your chosen methods.
\end{solution}\end{question}
%%Hack to correct tcbox behaviour
\color{black}

\section{Identifying tasks and deliverables}
Recall from Stage 1 that 
%
\begin{itemize}
\item your research aim tells the reader how your research will address the knowledge gap that you have found through your literature review, 
\item your objectives help you break it down into 3 to 4 high-level steps you must take to achieve the aim.
\end{itemize}
%

\paragraph{}
It is now time to break your research objectives down even further, into tasks to help you plan your detailed research work to meet each of them. While a research objective indicates what you need to achieve, \textbf{research tasks} specify the work that you need to complete to get there, so they address the question of what you need to do to meet the objective.

For the sake of argument, we'll be working with the following objectives, which you saw in Example~\ref{ex:machinelearning} in Stage 1\footnote{See page~\pageref{ex:machinelearning}.}\todo{What was the research problem?}:

\begin{example}{Recap: Applying Machine Learning}
Given the following research aim:

\emph{to apply Machine Learning (ML) to improve the accuracy of resources and time forecasting in the context of small engineering plants}

Some related objectives could be the following:
%
\begin{description}%[start=0,label={(\bfseries R\arabic*):}]
\item [Objective 1] To identify which ML techniques are applicable to resource and time forecasting in the context of small engineering plants

\item [Objective 2] To test the accuracy of forecasting of those techniques

\item [Objective 3] To provide recommendations as to how integrate those techniques effectively in engineering practice in order to improve forecasting accuracy
\end{description}
%
where
%
\begin{description}%[start=0,label={(\bfseries R\arabic*):}]
\item [Objective 1] allows us to identify specific ML techniques to be used in the project, to ensure the work is feasible within the time-frame of the project. 
\item [Objective 2] allows us to investigate and compare how accurate the chosen techniques are in their forecasting application. 
\item [Objective 3] allows us to draw some conclusions from the research conducted and make recommendations to improve professional practice.
\end{description}
%

Note how those objectives build on each other and, if successfully completed, they contribute to meet the overall aim.
\end{example}


Closely related to tasks are \textbf{deliverables}: these are the tangible things produced by your task work.

\begin{example}{Example -- cont'd}In our example, the first objective is met once we have identified the relevant ML techniques. There are two complementary ways to do this: to look at the literature and to ask practitioners. As a result, we could break this objective down into the tasks, and deliverables, indicated in the following table

\paragraph{Table}

\textbar{} \textbf{Objective} \textbar{} \textbf{Tasks} \textbar{} \textbf{Deliverables} \textbar{}
\textbar{} :----- \textbar{} :----- \textbar{} :----- \textbar{}
\textbar{} \textbf{To identify which ML techniques are applicable to resource and time forecasting in the context of small engineering plants} \textbar{} to investigate the academic literature in order to identify relevant ML techniques \textbar{} a collection of relevant ML techniques reported in the literature \textbar{}
\textbar{} \textbar{} to ask practitioners which techniques they employ \textbar{} a collection of relevant ML techniques used in professional practice \textbar{}\end{example}

\begin{question}[subtitle={ACTIVITY: Establishing tasks and deliverables}] Consider your research objectives. For each, identify related tasks and deliverables.

\begin{guidance}You should draw a table similar to that in our running example. You should ensure that the tasks provide a reasonable break down of your objectives into discrete pieces of work.
\end{guidance}\end{question}
%%Hack to correct tcbox behaviour
\color{black}

\paragraph{}
Your tasks and deliverables don't need to be perfect at this stage, and are likely to be revised as you progress through your project. However, it is important that you have thought about specific work you will need to carry out to meet your objectives.

\section{Relating tasks to research methods}
The way to carry out your tasks and meet your objectives is through the application of research methods.

\begin{example}{EXAMPLE - cont'd }Following on from our previous example, we have extended the table to include an indication and justification of candidate research methods for each task.

\paragraph{TABLE}
\textbar{} \textbf{Objective} \textbar{} \textbf{Tasks} \textbar{} \textbf{Deliverables} \textbar{} \textbf{Relevant research methods} \textbar{} \textbf{Justification and feasibility} \textbar{}
\textbar{} :----- \textbar{} :----- \textbar{} :----- \textbar{} :----- \textbar{} :----- \textbar{}
\textbar{} \textbf{To identify which ML techniques are applicable to resource and time forecasting in the context of small engineering plants} \textbar{} to investigate the academic literature in order to identify relevant ML techniques \textbar{} a collection of relevant ML techniques reported in the literature \textbar{} review of existing literature \textbar{} I can access relevant literature via my university library \textbar{}
\textbar{} \textbar{} to ask practitioners which techniques they employ \textbar{} a collection of relevant ML techniques used in professional practice \textbar{} questionnaire, possibly followed by interviews \textbar{} I have access to professional networks, which I could use to distribute the questionnaire, and possibly to recruit participants for follow-up interviews \textbar{}\end{example}

\paragraph{}
Note that the choice of research methods in relation to your research tasks is an essential part of your research design. In fact, the two influence each other: your objectives and related tasks direct you towards specific research methods, which in turn have to be part of your overall research design.

\begin{question}[subtitle={ACTIVITY: Associating methods to tasks and deliverables}] Extend your tasks and deliverables table with your candidate research methods, including stating why they apply and are feasible for your project. Revise your research design draft from Stage 2 so that is consistent with those choices.

\paragraph{GUIDANCE }
Refresh your understanding of chosen research methods from the study work you carried out in Stage 2. It is important you keep reviewing your choices with your supervisor.\end{question}
%%Hack to correct tcbox behaviour
\color{black}

\section{Detailing research procedures}
Once you have chosen the set of research methods you will apply, you must establish exactly how you will do that, something we refer to as \textbf{research procedures}.

Your research procedures will be method specific, in that each method you choose to apply will come with recommended practices, which you will need to contextualise to your own project needs, including your access to participants, data or other kind of evidence. For instance, there are plenty of guidelines in the literature on how to design questionnaires, including which type of questions to include and how to phrase them. There are also recommendations concerning testing the questionnaire design prior to its use, and of course, there are many ways a questionnaire can be administered. In writing your procedures for this research method, you would have to be specific on how each of the above applies in your project.

It is important, therefore, that you master the research methods of your choice, starting by reviewing once again the related academic literature.

\begin{question}[subtitle={Activity: Sketching research procedures}] Consider the research methods you intend to apply, and the related review you conducted in Stage 2. Reconsider those materials, possibly going back to the literature sources, to learn how to apply the methods effectively within your project.

For each method, sketch possible procedures of application, ensuring you make appropriate reference to the literature you have reviewed and best practice guidelines therein.

\begin{guidance}It is possible that the review you conducted in Stage 2 is not sufficient, in which case you will need to extend it to complete this activity.

You should focus on practical aspects of applying the methods, including specific processes and techniques to gather, summarise and present your evidence in your reports.

Depending on the extent you need to review further academic literature, this activity could be quite substantial, so you should set aside up to 20\% of your study time to complete it.

\end{guidance}\end{question}
%%Hack to correct tcbox behaviour
\color{black}

\section{Assessing validity}
As your intended research design becomes clearer, you will soon be testing some aspects of it in your pilot work. Before you do that, however, you need to consider if the choices you have made will allow you to gather evidence and derive findings in a systematic, rigorous, repeatable and reliable fashion so as to address your research problem. This is referred to as assessing the overall \textbf{validity} of your research design, which is broken down into the following considerations.

\textbf{Construct validity} asks whether you have put your design together logically by focusing on the relationship between evidence and research problem. Here you ask yourself whether the evidence you will generate through your chosen research design will be accurate and relevant to address your research problem. This tests the logical coherence of your aim, objectives, tasks, methods and deliverables in relation to the research problem and the knowledge gap you intend to address. With construct validity, you are asking: \emph{have I designed my research in the right way?}

\textbf{Internal validity} is concerned with the way you gather and analyse evidence. All research strategies and methods come with recommendations of good practice to ensure that your research is both systematic, repeatable and reliable. In your work, you need to ensure that you follow such practices and are aware of possible pitfalls. For instance, in experimental research you need to control all factors which may effect outcomes beyond those under study: failing to exercise such control will lead to observations and measurements which are unreliable. In assessing internal validity, you should also take into account limitations of human perception and cognition, and any potential personal bias. With internal validity, you are asking: \emph{have I executed my research in the right way?}

\textbf{External validity} relates to the extent you will be able to generalise your findings beyond the immediate context of your research. For instance, you may conduct a case study within a specific organisation, so here you are asking whether and how what you have discovered may apply to other organisations. With external validity, you are asking: \emph{will my research lead to findings that apply somewhere else?}

Anything that gets in the way of validity in research is termed a \textbf{threat to validity}. Different research strategies and methods are exposed to different threats, something you should have encountered in your review of the literature on your chosen methods.

\begin{question}[subtitle={Activity: Assessing validity of research design}] Conduct an initial assessment of your chosen research design in relation to the three kind of validity discussed above. Write down a short summary of your thinking in support of each, and of possible threats to validity you envisage.

\begin{guidance}You may need to refer back to the literature you have reviewed to identify specific threats which apply to your chosen research methods and strategies.

You won't be able to address this in full at this point in your project, particularly the internal validity, which refers to the execution of your research design. Nevertheless, it is important for you to consider validity and possible threats from the onset. You will return to this topic at the end of your project, as part of the overall assessment of your research, to reflect on the validity of your completed research.

\end{guidance}\end{question}
%%Hack to correct tcbox behaviour
\color{black}

\section{Conducting your pilot work}

Your \textbf{pilot work} will be a small scale test of some of the methods and procedures you will apply in the next stage of your project. Its main function is to help you assess the feasibility of your research design, or at least some aspects of it, and build your confidence in the approach you have chosen.

As such, your pilot work may not contribute directly to your aim and objectives, but it should help you decide whether you can actually do what you have planned to do, or inform how your research design and project plan should change instead.

There are no constraints on what you can do for your pilot work, other than you should exercise some aspect of your research design. It is therefore essential that you you agree what you are going to do with your supervisor first.

\begin{question}[subtitle={Activity: Planning and executing your pilot work}] Plan your pilot work and discuss your plan with your supervisor.

Once you have agreed the way forward, execute your plan and write a summary of both its execution and outcomes.

\begin{guidance}Your summary should include:

- an indication of which aspects of your research design your pilot work was concerned with

- any methods and procedures applied

- any data or evidence gathered, including possible modelling, artefact design or prototyping, appropriately presented and summarised

- lessons learnt and any resulting revision to your research design and project plan, particularly in relation to construct and internal validity of your research design.

To complete this activity successfully, it is essential that you agree your pilot work plan with your supervisor upfront, and discuss your progress on a regular basis.

This is a substantial activity, which will take you up to 35\% of your study time.

\end{guidance}\end{question}
%%Hack to correct tcbox behaviour
\color{black}

\section{Reporting in Stage 3}
At the end of Stage 3, you should complete a report, extending that of Stage 2 and covering the work you have carried on in this stage. Its recommended structure and content are indicated in Table 1.

Table 1 -- Report structure and content guidance

\begin{table}[htbp]
\begin{minipage}{\linewidth}
\setlength{\tymax}{0.5\linewidth}
\centering
\small
\begin{tabulary}{\textwidth}{@{}ll@{}} \toprule
 \textbf{Structure} & \textbf{Content guidance} \\
\midrule

 Proposed title & Your title should continue to capture succinctly research problem and aim \\
 Sect 1 - Introduction 1.1 Background to the research 1.2 Justification for the research & This section should provide an introduction to your research topic in its wider context (as background) and your justification of why the research is worth pursuing. It should be well articulated and supported by evidence \\
 Sect 2 - Literature review 2.1 Review of existing relevant knowledge 2.2 Critical summary, including knowledge gap to be addressed by the research & Your review should provide a critical account of your in-depth engagement with the academic (and other) relevant literature, including identifying key trends, ideas and possible knowledge gaps. Most of your citations should point to academic articles. Your critical summary should highlight key insights from your review and provide a strong justification for your proposed research. Both coverage and depth of your review matter. You should ensure that your review is well structured, with a logical narrative flow and your arguments are well supported by evidence \\
 Sect 3 - Research definition 3.1 Problem statement 3.2 Aim, objectives, tasks and deliverables 3.3 Knowledge contribution & You should ensure that your research problem is well articulated and appropriate for your course and your personal and professional circumstances, that your aim and objectives are consistent with research problem, that tasks and deliverables break down your objectives appropriately and are clearly related to your chosen research methods, and that the intended knowledge contribution of your research is clearly articulated \\
 Sect 4 - Research design 4.1 Evidence and data 4.2 Research strategy and methods 4.3 Research procedures 4.4 Ethical, legal and EDI considerations & This section should demonstrated your critical engagement with all elements of research design, including a detailed account of the data and evidence needed in your research, the research methods and research strategies you will to apply, and how you will apply them within your project. Your account should be supported by a clear rationale and insights from the related literature, and appropriately justified in relation to your research problem, aim and objectives. It should also demonstrate your careful consideration of ethical and legal matters, and that your research will comply with your course and university requirements \\
 Sect 5 - Analysis and interpretation 5.1 Pilot work & This section should report on a well thought-out pilot work which clearly and competently test some significant aspect of your research design. It should demonstrate good critical reflection on outcomes and highlight any adjustments needed as a result. \\
 Sect 6 - Assessment of your proposed research 6.1 Qualification fit 6.2 Personal and professional fit 6.3 Technical skills and resources needed 6.4 Statement of feasibility 6.5 Personal reflection on research process & In this section you should continue to argue how your research is a good fit across all criteria. You should provide a clear rationale as to why you think what you are proposing is feasible. You should also reflect on your growing understanding of the research process, including key learning and aspects you have found particularly challenging. \\
 Sect 7 - Planning, scheduling and risk assessment 7.1 Statement of progress 7.2 Key priorities in follow-up stage 7.3 Risk assessment & In this section you should reflect on the progress you have made in Stage 2 and establish your priorities for the next stage. You should also review your risk assessment as appropriate. \\
 References & You should keep your growing references in good order and ensure you apply the required bibliographical style consistently. Ideally, you should use a BMT to generate and integrate your references within your report \\
 Appendix - Work schedule & You should include your revised work plan as an appendix \\
 Appendix - Risk assessment table & You should include your updated risk table as an appendix \\
\bottomrule

\end{tabulary}
\end{minipage}
\end{table}

\begin{question}[subtitle={Activity: Putting your report together}] Using your word processor of choice, and starting from your previous report, complete your Stage 3 report by applying the structure and guidance in Table 1, and making good use of your notes and summaries from all related activities you have carried out so far.

\begin{guidance}In this first pass at putting together your report, you should focus primarily on completeness, ensuring that each section includes at least draft content.
\end{guidance}\end{question}
%%Hack to correct tcbox behaviour
\color{black}

\subsubsection{}
As in the previous stages, after you have filled in your report you should review and revise it iteratively until you are happy with your account, and are ready to move on.

Table 1 - Criteria to review your report

\begin{table}[htbp]
\begin{minipage}{\linewidth}
\setlength{\tymax}{0.5\linewidth}
\centering
\small
\begin{tabulary}{\textwidth}{@{}ll@{}} \toprule
 \textbf{Criteria} & \textbf{Prompts} \\
\midrule

 \textbf{Completeness} & Are all sections of the suggested structure completed in line with the guidance provided? \\
 \textbf{Good academic writing practices} & Have you applied good academic writing practices throughout? \\
 \textbf{Logical structure and flow} & Have you structured your narrative appropriately to ensure a logical flow of arguments? \\
 \textbf{Supporting references or evidence} & Are your key arguments supported by appropriate references or other evidence? \\
 \textbf{Citation and reference style} & Do all your citations and references comply with the required bibliographical style? \\
 \textbf{Avoiding plagiarism} & Have you acknowledged the work of others and distinguished it from your own appropriately? \\
 \textbf{Standard of English (or any modern language you use)} & Have you proof-read your report carefully to remove all typos and grammatical errors? \\
\bottomrule

\end{tabulary}
\end{minipage}
\end{table}

\begin{question}[subtitle={Activity: Reviewing your report}] Apply the criteria in Table 1 to review your current report and write up a summary of your assessment.

\begin{guidance}For each criteria, consider the related prompts to help you assess your report overall, and write down any further work needed for your next stage.
\end{guidance}\end{question}
%%Hack to correct tcbox behaviour
\color{black}

\section{Reflection: Stage 3}

%%More here

%%Repeated reflection activity
%%Repeated Activity for all reflections
\begin{question}[subtitle={Activity}]
$<$Needs assessing for content and structuring into activity + guidance$>$

This activity has four parts: the first is something you should be doing regularly, but won't make you into a disobedient or indocile thinker. The second, third and fourth may help you get started and keep going.

Part 1: Think about your study this far -- using this book and anything you've done for your dissertation in parallel -- as a journey. More from elsewhere, including   !!.

Part 2: think about yourself and the way you think. How does your desk look? Is it messy or tidy? Do the same for your computer desktop. Is it empty or are there hundreds of files strewn across it? Do you think your tidiness or untidiness will affect the way you do your research? How about how you keep your -- critically important -- bibliographic database which may contain up to a hundred academic\footnote{It's not unknown to have more than a hundred.} and other articles by the time you're finished?

Part 3: think about the context of your research. Which professional pressures are there on you to succeed in solving your research problem? Pressures could come in many forms: financial -- there's a promotion for you at the end of it; peer -- your colleagues know that you are studying will have good expectations of your result and you'll want to prove them right\footnote{Or wrong, depending on the colleague!}. Are you sponsored by your employer? Will you be able to report a negative outcomes to your research, for instance, that there is no solution to our problem using the current technology stack? A negative result is a very good research outcome, even if it tends to satisfy fewer non-academics than a positive result.

Which family pressures do you feel? It's' not unusual that you will have given up a paying role to study, moving the responsibility to provide onto another member of your family. What are their expectations?

Part 4: what's that thought nagging at the back of your mind? Is it ``How will I start?'' Or ``Will I be able to dedicate enough time to this?'' Or ``Can I really do this?''. Or ''Is ``shouldn't I be bringing in a wage rather than studying?''

You may be one of the lucky ones that doesn't have such negative thoughts, but negative thoughts are a very natural part of steps into the unknown. And research is precisely that, a step into the unknown. At least if you are aware of the doubts you naturally have, you can manage them. Think about making even the tiniest of steps forward in your research visible and celebrated! Work with Kansan boards where progress is encouragingly visible as you move a task from the inbox to the outbox. If you have concerns about managing your time, start using one of the many tools out there that break time up into manageable units and help manage it for you. If your concerns are about how to organise your thoughts, look into mind maps, lists, todo lists.

Thinking early and often through reflection is a powerful way of doing better. Do it well and your final report will be better than you will have expected.

It's worth saying that, at the end of what could be an exhausting journey, you will not fully appreciate your achievements. That realisation may have to wait until you are rested, graduated, or some distant time later.

But it will come.

\begin{guidance}
%Hack to correct tcbox behaviour
\color{black}
Something here
\end{guidance}\end{question}
%%Hack to correct tcbox behaviour
\color{black}



