%%Candidate to move to headers
%\DeclareExpandableDocumentCommand{\tabletitle}{O{2} m}{\\[1ex]\multicolumn{#1}{@{}p{\textwidth}@{}}{\textbf{#2}}\\ }%

%\newcommand{\hwidth}[1]{\makebox[0pt][l]{#1}}
%\DeclareExpandableDocumentCommand{\tabletitle}{m}{\SetCell[c=2]{\textbf{#1}}&\\ }

%%Structures the report table, headings to the left in a list, explanation to the right
%%#1 = caption label

%%Resource type = fulltext cite in itemized list
\DeclareCiteCommand{\resourcelistcite}
  {\usebibmacro{prenote}\begin{itemize}\item[] }
  {\usedriver
     {\DeclareNameAlias{sortname}{default}}
     {\thefield{entrytyper}}
  } 
%     {%
%		\footnote{\thefield{annotation}}
%	 }
  {\item[] \addspace}
  {\end{itemize}\usebibmacro{postnote}}

\newenvironment{ReportStructureTable}[1]{%
	\setlist{nosep} %compress any lists
	\begin{longtable}{@{}p{0.3\tablewidth}>{\vspace{-3pt}}p{0.7\tablewidth}@{}}%%JGH \vspace{-3pt} is a hack to align first lines
	\caption{Report structure and content guidance\label{#1}}\\
	\toprule
	\textbf{Suggested structure} & \textbf{Content guidance} \\
	\midrule%
}%
{\end{longtable}}

\newenvironment{MoreDetails}[0]{\subsubsection{More details} The following citations provide good starting points for further reading: }{}

\chapter{Stage 3: Developing your research design}

By now you will have some mastery of the techniques and tools that you need to \emph{do} research at masters level. You may also have ideas about what you still need to do in the next step\footnote{If not, don't worry – we've got you covered with this chapter!}.

With the skills you have so far gained, you're developing into an independent researcher\footnote{Being an independent researcher isn't one of the examined outcomes of masters research, but if you're feeling confident in your research that's a good thing.} and you may feel that this book holds nothing more for you. 

Stay with us a little longer though: the next sections aren't as long as those that you've studied already – you'll be doing more yourself, honing the skills you've picked up as you go along – but they might help to keep you systematic and on the path to submission. 

%\newcommand{\percentthroughbook}{
%%\thepage
%%\getpagerefnumber{LastPage}
%\intcalcDiv{\intcalcMul{\thepage}{100}}{\getpagerefnumber{LastPage}}\%
%}

You won't be surprised to know that stage 3 comes next; there's another research increment coming.

%%Removed for now
%You're now \(\percentthroughbook{}\) through this book – not long to go before you're finished. Stick with us for a while longer.

\section{Introducing stage 3}

\todo{LR: update at the end}
\todo{LR: to check all activity titles at the end}

\begin{table}[htbp]
\caption{Stage 3 activities\label{tab:stage3acts} {\color{red}Update as necessary)}}
\centering
\small
\begin{tblr}{colspec={X[10]X[1,r]X[4]},
row{even}={font=\bfseries,},
row{odd}={2-Z}{preto={\quad},},
row{1}={font=\bfseries},
}% \toprule
Research activity
 	& \SetCell[c=1]{c}{Effort within stage} & Suggested supervisor focus\\
%\midrule
%
\SetCell[c=2]{l}{Identifying the research problem}\\
	Adjust, if needed & 2\% & \\
\SetCell[c=2]{l}{Reviewing the literature}\\ 
	Adjust, if needed & 3\% & \\
\SetCell[c=2]{l}{Setting research aim and objectives}\\ 
	finalise aim and objectives, and define tasks and deliverables & 10\% & Suitability of tasks and deliverables from objectives \\
\SetCell[c=2]{l}{\textbf{Choosing the research design}}\\ Complete research design, with detailed consideration of data and evidence, research strategy, research methods and procedures & 20\% & Suitability of research procedures \\
\SetCell[c=2]{l}{\textbf{Gathering and analysing evidence}}\\ Conduct pilot work to test aspects of your research design & 35\% & Scope of your pilot work \\
\SetCell[c=2]{l}{\textbf{Interpreting and evaluating findings}}\\ n/a & 0\% & \\
\SetCell[c=2]{l}{\textbf{Reporting, critical reflection and conclusions}}\\ Assess research progress and write up Stage 3 report & 25\% & Any further improvements required \\
\SetCell[c=2]{l}{\textbf{Work planning and risk management}}\\ At stage start, review work from previous stage and project risk; adjust plan as needed If you have received feedback from supervisor on your previous stage work, adjust plan to include any revision recommended & 5\% & Any major adjustment required \\
%\bottomrule
\end{tblr}
\end{table}

%\todo{This is a repeated activity through all stages, and so should be templated.}
\begin{question}[subtitle={Activity: Understanding the effort needed in this stage}] Consider Table~\ref{tab:stage3acts} carefully, taking notice of the entries in the `Effort within stage' column. Write down the most time-consuming activities in this stage and what is expected under each.

\begin{solution}Developing your research design further and conducting your pilot work will constitute your major effort in this stage (55\% of the study time in total): your pilot work will be an initial test of some aspects of your research design, including a proof-of-concept application of some of your chosen methods.
\end{solution}\end{question}
%%Hack to correct tcbox behaviour
\color{black}

%\end{document}

\section{Research design foundation}

To make a contribution to knowledge we do research. Practically, to do research, we combine a number of research tasks into a framework. Designing such framework is what we term research design. The framework will depend on the research area, the type of knowledge contribution you wish to make, your mindset as a researcher, and the opportunities and difficulties you may face along the way.

A research framework has many levels. At its foundations are the research framework's \enquote{ontology}, \enquote{epistemology}, and \enquote{methodology}\todo{[ADAPTED from \url{https://proofed.com/writing-tips/the-four-types-of-research-paradigms-a-comprehensive-guide/}}: 

\begin{description}
\item [Ontology] is the philosophical study of the nature of existence\footnote{If you're interested, there's a fuller discussion of \emph{Ontology} in the Stanford Encyclopedia of Philosophy \parencite{hofweber2023logic}.} and addresses the question: \enquote{What is the reality that I will research?}. Practically, ontology translates to determining what \emph{phenomena} exist in the context of  your research, the \emph{relations} that exist between them and how they group together into \textit{categories}.

\item [Epistemology] is the philosophical study of knowledge\footnote{If you're interested, there's a fuller discussion of \emph{Epistemology} in the Stanford Encyclopedia of Philosophy \parencite{steup2020epistemology}.} and addresses the question: \enquote{How is knowledge generated and from what sources?}. Practically, epistemology is about finding out \enquote{What people know?}, \enquote{What does it mean to say that people know something?}, and \enquote{How do people know that they know?}.

\item [Methodology] is the system of principles and methods\footnote{Even if you are interested, there is (at the time of writing) no Stanford Encyclopedia entry for methodology, unfortunately. But, unlike ontology and epistemology, we go into more detail of methodology below.} by which you conduct research, that is, investigate, measure, and analyse your research’s aim and objectives. Methodology operationalises the \enquote{how} question of knowledge generation, so it is about devising concrete strategies to answer \enquote{How will I make my contribution to knowledge?}. 
\end{description}

As you might have guessed, given that the goal of research is to make a contribution to knowledge, epistemology and ontology are incredibly important in defining what knowledge is in any particular research context and what, in that context, can be known about. Once this choice is made, an appropriate methodology can be devised: hence, methodology depends on choices made in relation to ontology and epistemology. 

Fortunately, many others have thought very deeply about ontology and epistemology and, in most areas and for the vast majority of masters-level research, their answers will suffice. If not, we'd be left in a situation in which even an ostensibly simply statement like \enquote{That hat is blue} becomes in need of complex debate \parencite[Section 4.1]{steup2020epistemology}.

Methodology, on the other hand, is something we will spend some time on, particularly how individual research methods combine to produce knowledge contributions through research strategies. 

You should be aware that `methodology' has many meanings in the literature, including the study of research methods, which questions the assumptions that underpin their creation and application. Wikipedia says\footnote{It could almost be seen as a warning!}:
%
\blockquote{%Many discussions in methodology concern the question of whether the quantitative approach is superior, especially whether it is adequate when applied to the social domain. 
[...] A few theorists reject methodology as a discipline in general. For example, some argue that it is useless since methods should be used rather than studied. Others hold that it is harmful because it restricts the freedom and creativity of researchers. Methodologists often respond to these objections by claiming that a good methodology helps researchers arrive at reliable theories in an efficient way. The choice of method often matters since the same factual material can lead to different conclusions depending on one's method. Interest in methodology has risen in the 20th century due to the increased importance of interdisciplinary work and the obstacles hindering efficient cooperation.}
%
These are not unimportant issues to consider. However, and as for ontology and epistemology, we will leave their discussion to others, content to stand on those giants' shoulders – we take an unapologetically practical approach to research methods, limiting our discussions to what, we feel, are their important characteristics for practice. This doesn't ignore philosophical issues, however: where there are important philosophical considerations to be considered, we address them. This includes questions as to how to choose a particular research method, and what an experienced reader will expect to be answered by it. You can then craft your dissertation to meet those expectations.

%\bigskip
%
%Before we meet research methods, however, it's worth thinking more about what is the foundations of ontology, epistemology and methodology is the research \textit{paradigm}, the system of beliefs, ideas, values, and habits that guide (and constrain) a researcher's way of thinking about the world.

\section{Researcher mindsets}

Depending on your background, you may have begun your research studies with a particular mindset – that of a scientist, for instance, or as someone embedded within an organisation. This mindset will flavour your approach to research, but it shouldn't constrain it – there are many options for research and the right one for you might be outside of your current understanding.

%That's not a bad thing – many have been there before and constructed \emph{research paradigms} that characterise useful mindsets for researchers to have. 

Over time, researchers in different communities and disciplines have developed differing mindsets, which are known in the literature as research paradigms\todo{[ADAPTED from https://proofed.com/writing-tips/the-four-types-of-research-paradigms-a-comprehensive-guide/]}\footnote{A.k.a. philosophical traditions.} You can think of a research paradigm as a philosophical way of thinking, a set of shared beliefs which shape a worldview. 

We briefly outline the prevalent ones in this section --- there is a lot, lot more to be known around this topic, and this introduction only scratch the surface! We provide some references for you to start your own investigation into this fascinating and complex topic, should you wish to.

Each paradigm comes with its own ontological, epistemological and methodological choices. Being aware of them, may help you guide your research design choices, so it is important for you to be aware of their existence, even if in practice, you will mainly focus on methodological considerations. 

\subsection{Positivist and post-positivism}
The Positivist Research Paradigm  assumes that there is a single, objective reality that can be accurately known, described and explained. Positivist researchers depend on observations and measurements of this reality to gain knowledge of it. 
%They also depends on \textit{deductive reasoning} to generate that knowledge: starting from general observations as assumptions they deduce specific conclusions through logical steps, and as long as the assumptions are true, then so are the conclusions reached. 

Under the assumption of a single objective reality, positivist researchers make claims that they compare against reality to determine the \enquote{truth}. Hence positivists seek to confirm their theories through their observations of an objective reality. This removes the researcher as a variable in the research equation so that positivist research is necessarily limited to data generation, and their analysis, and interpretation from an objective viewpoint as the basis of knowledge.

The positivist research paradigm is mostly used in situations where a \enquote{single objective reality} is most expected, i.e., the natural sciences, the physical sciences, or whenever very large sample sizes can be used to infer characteristics of a population. It leads the researcher towards quantitative methods, including experiments, tests, surveys and simulations involving formal modelling based on mathematics, statistics or computational thinking. 

Positivism\todo{From \url{https://image.slidesharecdn.com/lecture21-111207045819-phpapp02/95/research-paradigms-20-728.jpg?cb=1415227903}} leads to knowledge as explanations constructed from hypotheses thus established as laws or facts. As an example, think of Newton's explanation of the action of forces on matter that is encoded as his Three Laws of Motion. These are meant as universal objective truths which apply to the natural world forever.

With this in mind, and the – almost – universal acceptance of Newton's Three Laws based on their predictive capability, positivism was, for a very long time, the dominant mode of Western thought. So much so, that it was a bulkhead against a growing number of worrying observations, including the movements of the planet Mercury\footnote{See \textcite{enwiki:1193607156}, for instance.}, which didn't reinforce – indeed appear to contradict – Newton's Laws. How could an established truth lead that way? Indeed, Einstein's insight into the intimate connection between space and time inspired a substantial move away from the established Newtonian \enquote{laws} and \enquote{facts}, which were neither any longer \textcite{lakatos2014falsification}\textcite{laka}.

The need to rethink positivist objective truths was something of a crisis in the positivist movement~(see, for instance, \textcite{kuhn2012structure}), leading to post-positivism\footnote{Not the most creative name, you must admit.} which introduced the idea of falsification: any posited theory must make predictions which are testable, the currency of a theory being determined by whether or not it had yet been proven false\footnote{Note that falsifiable theories that have been tested and failed can still be useful, perhaps within a restricted context. For instance, Newton's Laws of motion provide a very good approximation at low energies.}. 

\bigskip

So, both positivism and post-positivism accrete knowledge by formulating generalisations and cause-effect linkages, based on objective, verifiable observations and measurements, and expressed as theories and laws. However, post-positivism acknowledges some of the limitations in such observations and measurement, so that a theory or law will only remain true for as long as it is not falsified by new observations or measurements. There is therefore a shift from certainty (positivism) to probability (post-positivism), with post-positivist researchers encouraged to take multiple measurements and observations, including triangulating their data, to arrive at an objective truth. Thus you might take a post-positivist approach to establishing the linkage between a drug and the alleviation of symptoms: once a generalisation or cause-effect linkage is established, it applies for as long as it remains un-falsified.

Both positivism and post-positivism assumes an objective reality and do not admit that the researcher's own mindset and values may influence true knowledge: in being objective and verifiable, different researchers must necessarily arrive at the same truth. This denial is often levelled as a criticism of the paradigms, particularly by social scientists, and has led to new ones.

\todo{See \cite{wikipedia-contributors2023positivism} for a more detail description.}

%\begin{question}[subtitle={Activity: Am I a post-positivist thinker?}]
%***to rethink****On a scale of 0 to 10, with 0 wholly in disagreement and 10 wholly in agreement, to which extent do you agree with the following statement?
%
%\textit{Even though the cat is blue, I can imagine situations in which it could be observed as some other colour.}
%\begin{guidance}
%	If you thought this was easy, you might be a post-positivist thinker. Post-positivist thinkers may tend towards quantitative research.
%\end{guidance}
%\end{question}
%\todo{Clarification of methods should go before activity, to make clear the distinction between posit and post-posit}

\subsection{Anti-positivist (interpretivism)}

The shift from positivism to post-positivism still preserves the absolute objectivity of reality. In contrast, anti-positivism asserts that different people experience and understand reality in different ways: while there may be only \enquote{one} reality, everyone interprets it according to their own views. Simply put, this might mean that generalisations and even cause-effect relationships are subject to individual experience. Think of the way that people interpret the (single) power structure within your organisation: typically, different people will describe it in different ways, as it applies to them.

Explaining the name, anti-positivists believe that all research is influenced and shaped by researchers’ worldviews, leading to differing interpretations of the same reality. Again, think of the questions you might ask of people within an organisation that leads them to describe the power structure. Different questions can lead to different descriptions.

As a result, anti-positivists gravitate towards qualitative research methods and techniques to understand the different perspectives, placed in an explicative context of their own perspective. These may include interviews and focus groups, participant observations, and review of documentation on a phenomenon of interest (e.g., newspaper articles, reports, or information from websites).

In moving away from objective knowledge, however, anti-positivism raises questions of research validity, that is of how knowledge generated as subjective interpretation can be relied upon. We will discuss validity in the next section.

See \cite{wikipedia-contributors2023antipositivism} for a more detailed description.

\subsection{Constructivism}

The Constructivist Research Paradigm asserts that reality is a construct of our minds and so is absolutely subjective. Constructivists believe that all knowledge comes from our experiences and reflections on those experiences. A distinction is also made between reality which is individually vs socially constructed, the latter being the result of social interaction within a specific cultural or historical context. 

Due to its focus on experiences and subjectivity, this paradigm is mostly associated with qualitative research approaches. The researcher focuses on participants’ experiences, including their own, constructing knowledge as \enquote{individual reconstruction, coalescing around consensus}\todo{\url{https://image.slidesharecdn.com/lecture21-111207045819-phpapp02/95/research-paradigms-20-728.jpg?cb=1415227903}} through understanding, sense making and reconstruction.

Knowledge accumulates through later research adding informed and sophisticated reconstructions, and vicarious or lived experience\todo{\url{https://image.slidesharecdn.com/lecture21-111207045819-phpapp02/95/research-paradigms-20-728.jpg?cb=1415227903}}.

Research validity is also an issue with this paradigm.

See \cite{}.

\subsection{Critical theory}
The Critical Theory Research Paradigm originated in the fields of sociology, philosophy and political theory, and asserts that social science can never be 100\% objective or value-free. Therefore, like interpretivism, it assumes multiple interpretations of reality in social contexts. However, it goes a step further by asserting that reality is shaped by those who are powerful, who legitimate particular ways of perceiving the world: `truth' is inherently political, defined by those in charge to the disadvantage of many, and challenged by those who wish to promote equality. As a result, critical researchers seek to challenge the status quo and perceive research as transformative at a social level\footnote{As a result, this paradigm is also called `transformative' in the literature.}, confronting ideology and trying to discover and challenge the mechanisms through which exploitation and disadvantage are perpetuated in society.
 
This paradigm is focused on enacting social change through scientific investigation. Critical theorists question knowledge and procedures and acknowledge how power is used (or abused) in the phenomena or systems they’re investigating.

Researchers using this paradigm can offer structural and historical insights as the basis of knowledge, approaching knowledge contributions through \enquote{critique and transformation, restitution and emancipation}\todo{see \url{https://image.slidesharecdn.com/lecture21-111207045819-phpapp02/95/research-paradigms-20-728.jpg?cb=1415227903} for this, needs further investigation}. The recognition of researcher values is welcomed as a formative influence on the research.

Even though very different in approach, Critical Theory research can lead to generalisations via similarly, although these tend to be historically situated. This of the\todo{Add example here}.

Quality judgements in Critical Theory tend to be \enquote{Historical
situatenedness; erosion of ignorance and misapprehensions,
action stimulus}\todo{\url{https://image.slidesharecdn.com/lecture21-111207045819-phpapp02/95/research-paradigms-20-728.jpg?cb=1415227903}}

See \cite{wikipedia-contributors2023critical} for a more detailed description.

\subsection{Indigenous}
The traditions described so far are attracting increasing criticisms in that they are seen as Western-European centric and often imposed on other indigenous cultures as a result of colonialism.

In counterposition, an indigenous research tradition has emerged with a social and political agenda of decolonising indigenous societies. It emphasises the connection between the researcher and their own culture and the natural world, in the sense that cultural practices and forms of expressions should be reflected in the way the research is conducted, including language, metaphors, oral traditions and knowledge systems. 

It also advocates an holistic approach which strives to reach a balance between different areas of life, integrating intellectual, social, political, economic, psychological and spiritual dimensions.

In being a relatively recent paradigm, the ontological and epistemological stance of indigenous research is still the subject of debate, so that a consensus has yet to emerge. 

More can be found at... \todo{add references}

\subsection{What's your mindset?}

Table~\ref{tab:researchParadigms} summarises the main paradigms we have discussed based on their ontological, epistemological and methodological standpoints\footnote{Indigenous is omitted given the ongoing debate around its ontological and epistemological stance.}. From a methodological perspective, we have indicated the main tendency of the paradigm, although the quantitative vs qualitative distinction is not as stark in practice, and a mix of methods is often applied.

\begin{table}[htbp]
\small
\caption{Summarising research paradigms\label{tab:researchParadigms}}
\begin{tabulary}{\tablewidth}{@{}LLLLLLL@{}} 
\toprule
&  \textbf{Positivism} & \textbf{Post-positivism} & \textbf{Anti-positivism (Interpretivism)} & \textbf{Constructivism} & \textbf{Critical theory} \\
\midrule
%\textbf{Aim} & to discover general laws and principles and predict behaviour through neutral and objective enquiry & to discover general laws and principles and predict behaviour through neutral and objective enquiry & to understand and predict social systems and their behaviour, acknowledging that the researcher's values and experience affect the enquiry & to understand and predict social systems and their behaviour, acknowledging that the researcher's values and experience affect the enquiry & to change society, challenge norms and emancipate and empower people & ***marginalised and post-colonial communities research ???\\
%\midrule
\textbf{Ontology} & one discoverable external reality & one discoverable external reality that can only be known imperfectly  &  one external reality which is interpreted subjectively & reality as the construct of one's mind & one external reality determined by socio, political and economic power factors \\
\midrule
\textbf{Epistemology} & objective laws and theories that can be confirmed empirically & objective laws and theories that can be falsified empirically & subjective interpretations & subjective constructions & social and historical constructions, acknowledging issues of power and social injustice \\ 
\midrule
\textbf{Researcher's role} & objective, neutral & objective, neutral, aware of cognitive limitations & subjective, bringing own values, experience and bias & subjective, bringing own values, experience and bias & subjective, aware of own social position \\
\midrule
%\textbf{Main strategies} & experimental, survey, simulation, mathematical and logical proof & experimental, survey, simulation, mathematical and logical proof & case study, action research, ethnography, phenomenology, grounded theory & case study, action research, ethnography, phenomenology, grounded theory, systematic review & \\
%\midrule
\textbf{Main methods} & quantitative & quantitative, with triangulation & qualitative & qualitative & qualitative\\
\bottomrule
\end{tabulary}
\end{table} 

Your own mindset may lead you to gravitate towards one or more of these paradigm, or even somewhere in between. The next activity should help you reflect on this point.   

\begin{question}[subtitle={Activity: What kind of thinker am I?}]
	Consider the following question and describe how you would go about answering it:
	
	``What is the colour of swans?''
	
	Then compare your approach to each of the paradigm. Which one is it closer to and why?
	\begin{guidance}
		If you can think of more than one way to approach the question, then describe and reflect on each of them in relation to the paradigms.
	\end{guidance}
	\begin{solution}
		I can think of a couple of ways I could tackle this question. 
		
		The first would be to start by observing the swans that live on the lake near my home, and record my observations. From that I would put forward an initial hypothesis, say that all 'swans are white,' as those are the only ones I can observe locally. I would then look online for images of swans from around the world to see if they match my observations. Having found images of black swans alongside white ones, I would then revise my hypothesis to ``All swans are either white or black.'' This process would continue until I'm satisfied there is no further contradictory evidence I can find, hence conclude that in all probability swans are either white or black. I would have to admit that there may be swans of other colours I've yet to come across, so the statement is open to future challenges. I would also need to be convinced that I'm a neutral observer, able to determine the colour of a swan correctly and reliably. This approach closely aligns with the post-positivist paradigm, specifically: I've made observations, triangulated my direct swan observations with the review of online swan images, and formulated, rejected and then reformulated hypotheses as part of my enquiry process.
		
		My second approach would be to ask other people. For instance, I could set up a crowd-sourcing survey inviting participants to answer the question. By analysing their answers I could then decide if there is enough consensus on the colour of swans, like most participants may have identified swans to be either white or black, although some may have provided more nuanced answers, like yellowish or other. From my analysis I would draw my conclusions which may or may not be the same as in my previous approach. In this case, I would have to worry about who participated in my research. Were there enough participants from around the world to provide sufficient and diverse evidence? To which extent may their colour perception may differ? What else could I do to check the validity of this outcome? This approach aligns with the interpretivist paradigm: I have to accept that, like me, each observer in my study will make their own interpretation of what the colour of a swan is, so that I would have to account for this in my conclusions. 
\end{solution}
\end{question}


\section{Research strategies}

Each research area has its more-or-less well-worn paths to a successful knowledge contribution. In Stage 3, you're now at the point where you'll join researchers in your chosen area on one of those paths: as you get deeper and deeper into your research, the steps you'll take will become more and more specialised. 

To identify and take such steps, you will need to devise a research \textit{strategy}, by which we man a collection of recipes for doing research that will, if followed accurately, lead to a contribution to knowledge \emph{even in the presence of uncertainty}. When devised, a research strategy consists of research tasks that interact in more or less complex ways, but which are sufficiently detailed that the researcher knows what to do next, even if that means making a choice between two or more next steps. 

With uncertainty a product of many contextual characteristics, including the researcher's view-point, the simpler research strategies stem from contexts with lesser uncertainty. Thus, highly constrained contexts, such as the natural sciences, tend to have simpler strategies as there is less uncertainty to contend with. \todo{to rephrase/expand, but not sure how}

There's good news and bad news in choosing a research strategy:
%
\begin{itemize}
\item The bad news is that there are many possible choices you could make at any point.
\item The good news is that, for your particular area in masters research, there will likely be only a small subset that you need to know about.
\end{itemize}

To help you in your choice, our approach in this chapter is unapologetically practical. We will layout the options that you have together with reasons for choosing them and reasons for not choosing them. Each comes with a list of evaluation questions\todo{Comment on the completeness of the questions?} the answers to which you will be expected to present as part of your dissertation. Amongst other things, the answers you give will justify how and why your work makes a contribution to knowledge. These evaluative questions in turn give you targets to aim for throughout your research, you will need to answer each of them – they will be the driver for your research and your writing up. 

%
%For instance, social science research strategy could be as simple as a two group experiment while the strategy\todo{Andrews calls it a research model.} described in \textcite[p.~407]{andrews2005place} for education research has 12 components, arranged in complex interacting feedback loops, with the of including experiments.
%
%\todo{we need a glossary as I'm getting confused to the meaning of strategy. I have assumed a research strategy was something like 'survey research' or 'ethnography', while what's in the figure is more like the framework designed to conduct a specific study, possible by combining strategies. We need to agree on the terms and use them considently}
%
%\begin{figure}[h!]
%\centering{
%\begin{tabular}{ccc}
%  \includegraphics[height=8cm]{SimpleExperiment.jpg}&\qquad\qquad\qquad\qquad&
%  	\includegraphics[height=8cm]{andrews2005placefig1.jpg}
%\end{tabular}
%  \caption[Simple and complex research designs]{(left) A simple two-group experimental design~\parencite[, adapted p.~128]{marczyk2005essentials}; (right) A complex research strategy for education research~\parencite[, figure 1]{andrews2005place}
%  \label{fig:simpleandcomplexres}
%  }}
%\end{figure} 
%
%Although each research paradigm is sufficiently distinct as to indicate different strategies\todo{??? I don't understand this at all}, strategies do overlap in their application. Every strategy will, for instance, generate data of some form, whether this is readings in some experimental setting or documentation of the lived experience of a community under focus. Generated research data provide a focus for qualitative and quantitative analysis and, thus, to the synthesis of new knowledge.****


%Recall from Stage 1 that:
%%
%\begin{itemize}
%\item your research aim tells the reader how your research will address the knowledge gap that you have found through your literature review, 
%\item your objectives break it down into 3 to 4 high-level goals you must reach to achieve the aim.
%\end{itemize}
%%
%You're now at the point where you need to think about how you will meet each of your research objectives. Research strategies are the recipes for good research, combining research tasks into meaningful ways of doing research to achieve those objectives.

While your own research strategy will be specific and unique to your project in the way it informs the research you will conduct, standard research strategies have emerged over time, influenced both by research paradigms and research practice within specific disciplines. There are many of them, often with many variants: those we consider in this book are detailed in Section~\ref{sect:standardResearchStrategies}. For each, after a brief description of the strategy, we delve into the following details:
%
\begin{itemize}
\item describe what the kind of knowledge contribution it makes
%\item describes its focus\footnote{meaning?}
\item ask \enquote{Is this strategy right for me?}
\item describe any variants that exist and the choices that constitute them
\item describe the ways in which data is generated for the strategy
\item describe how a contribution to knowledge using the strategy will be evaluated
\item a number of references that give more detail if you are seriously considering the strategy.
\end{itemize}
%

The outcome of working through this chapter should be your choice of a research strategy that:
%
\begin{itemize}
\item is a good fit for your research problem, i.e., that will allow you to develop a contribution to knowledge arising from your research problem
\item makes the most of your current research skills and resources, i.e., the background knowledge and skills you bring to the research, the time that is available to you, and it fits with your research context.
\item a list of questions that could be asked of it by a knowledgeable evaluator, such as an examiner.
\end{itemize}
%
From the first two of these, you will gain an understanding of which steps you will be required to take to generate and analyse research data that, when complete, will make your contribution to knowledge. From the third of these, you'll be able to structure your research report – your dissertation – by describing your answers to the evaluative questions that will be asked of your research process.


\section{Defending your claim of new knowledge}

Being able to assert that you have made a contribution to knowledge is the point of structuring your research through a well thought-out research strategy. Even having made your assertion, however, you still need to defend it in your dissertation. That means considering, essentially, everything that could have gone wrong with the application of your research strategy – any weaknesses in the research strategy. This section shows you how to address such weaknesses.

\begin{figure}[hbtp]
\centering{
  \includegraphics[width=\textwidth]{Figures/ResearchVulnerabilities.pdf}
  \caption[]{Research vulnerabilities
  \label{fig:RVs}
  	}  
  }
\end{figure}

There are four main classes of weakness in all claimed knowledge contributions (see figure~\ref{fig:RVs}):
%
\begin{itemize}
\item novelty weaknesses, i.e., the hole in the literature that you claimed existed doesn't actually exist – perhaps you missed some key papers in your literature survey or, perhaps in the time that you've taken to complete your research, someone else has made a similar contribution to knowledge as that you claim. If there is no hole, then you cannot have made a contribution to knowledge.

\item validity weaknesses, i.e., the claim you have made to new knowledge isn't sufficiently credible, trustworthy, or accurate to be considered knowledge.

\item reliability weaknesses, i.e., the procedures that you have used to establish your claim of new knowledge are not sufficiently repeatable in different circumstances.

\item bias weaknesses, i.e., the claim you have made to new knowledge has been affected by your implicit or explicit biases, making the new knowledge unreliable.
\end{itemize}

Each potential weakness needs to be considered in turn – ignoring them leaves yourself open to a negative outcome of expert scrutiny – and, for each, arguments made as to why your claim doesn't suffer them, or if it does to some extent, that there is still a contribution to knowledge arising from your research.

You can deal with potential weaknesses in three ways\footnote{There is actually a fourth way, which is to be aware of the weakness but to ignore it. We do not recommend this as your examiner of your dissertation is likely to have detailed understanding of the research strategy you have chosen, including its potential weaknesses, and is likely to pick any methodological omissions up.}:
%
\begin{itemize}
\item avoid the weakness, i.e., choose a research strategy which is not troubled by the weakness. Part of the justification for the choice of research strategy can then be a discussion, if necessary, that the weakness doesn't arise.% This would be reported as part of the research method.

\item address the weakness, i.e., be aware of the weakness during the research and put in place further strengthening research. This might be, for instance, a second of further iteration of the research strategy which addresses discovered weaknesses in earlier research. This would be reported as part of the research design.

\item acknowledge and defer\footnote{Although it may seem to have similar outcomes, this is a much better strategy than simply ignoring the weakness as, although you don't address it, you make the examiner aware that you are aware of it. It can also give you a very neat way of filling out your future work.}, most usually at the end of the research period when the research is complete, i.e., write a reflection on the effect the weakness had on the outcomes and commit to addressing that weakness in future research. This would be reported as part of your \enquote{Discussion} and \enquote{Conclusion and future work chapter}.
\end{itemize}

If you can't avoid a weakness and you can't defer it, you have to address it. Addressing it means that your examiner will have their questions answered about the weaknesses they know occur in the type of research you're doing. Their evaluation will be through the questions they ask of your research and you must be prepared to answer them. There are two places at which they should be answered:
%
\begin{itemize}
\item in your dissertation, in all cases
\item in any \emph{viva voce} associated with your research course.
\end{itemize}
%

%As not all masters research have an associated \emph{viva voce}, we will assume that weaknesses should be addressed in the dissertation itself. Even if your course does have a \emph{viva voce}, it can be a nerve-racking experience to be confronted by an examiner asking questions to which you have no answer because you haven't thought about it!\footnote{And, although examiners are very good at asking surprising questions, you should have the ones you suspect they'll asked already answered.} 

In general, an examiner will explore weaknesses in your claimed contribution to knowledge through a number of questions they ask of your dissertation. For each research strategy, many of these questions\footnote{If not all; although examiners will have their own way of asking them!} can be predicted. Somewhere in your dissertation, then, you will need to expose your research strategy weakness and argue how your research addressed it.

Here is an example paragraph taken from an actual dissertation~\parencite{miles2019dispelling} with our commentary on specific points to the right, in the margin:

%\bigskip

\begin{adjustwidth}{4cm}{4cm}
%\subsection*{Limitations of Study}

My observational study focuses solely\footnote{Being specific on which phenomena are studied...} on the external elements of the embouchure and what can be seen in real time with the human eye\footnote{...and on the observations made of them...}, through the recording of video images. My analysis, and the conclusions that come from it, has been made from a purely visual perspective, captured by combinations of camera angles, without needing the use of any complex and expensive technologies\footnote{...thus correcting any expectations of what might have been achieved...}. In embarking on this research project, the initial intention was to measure facial muscle activity using Electromyography. This method proved to be too costly\footnote{...contextual factors prevented more sophisticated observations...} and the heavily mathematic and science based analysis process, out of the current skill set of this researcher\footnote{...and initial investigations reveals how difficult this would be}. Furthermore, due to the significant evidence found in the literature regarding the internal embouchure, the concept of the tongue being a pivotal element in facilitating pitch change has been accepted as fact and deemed unnecessary for further study in this project\footnote{There was no knowledge contribution to be made in this particular area...}. Therefore the ultimate goal of my research is to inform the teaching and learning of brass wind performance, with particular reference to the role of the embouchure\footnote{...and so the knowledge contribution was ...}. With this in mind, it is therefore important that the data obtained through this study be identifiable through the simplest means possible, so that it can be of the most benefit to the brass-playing community\footnote{...and our research goals were set accordingly.}.
\end{adjustwidth}


\todo{An activity?? I didn't find the above clear. Is its witnesses in the strategy chosen or has the strategy been chosen to address the witenesses? rather confusing}


%We go into more detail of the forms of weakness and how they can be treated %– through triangulation, reflexivity and critical literature – 
%below.
%
%
%There are three ways of dealing with the vulnerabilities in your research. You can:
%%
%\begin{itemize}
%\item patch: 
%\end{itemize}
%%
%

%\subsection{Introduction}\footnote{Probably not here.}
%
%There's still a great debate raging about what knowledge actually is. 
%
%The analysis of knowledge involves identifying the components that make up knowledge and understanding the structure of the concept of knowledge. The traditional definition of knowledge is as \emph{justified true belief}. This combines three conditions, usually presented out of the order they written in:
%%
%\begin{itemize}
%\item the \emph{belief} condition: we must \emph{believe} that something is true for it to be knowledge; we cannot know something if we don't believe it to be true, even if it is, in actuality, false;
%\item the \emph{justification} condition: a belief is justified if we have a reason to believing it, i.e., in the words of Wikipedia, \enquote{if another mental state supports it}~\parencite{enwiki:1204227228}, such as \enquote{a sensory experience, a memory, or a second belief}. This says that there is a reason or evidence for believing something;
%\item the \emph{truth} condition: that something \emph{believed} is actually true. Although this condition doesn't prevent from \emph{believing} something that is, in actuality, false, it prevents us \emph{knowing} something that is false. To know something means that that something is true;
%\end{itemize}
%
%\begin{question}[subtitle={Activity: Understanding the effort needed in this stage}]
%Do we want to explore this concept more deeply?
%\end{question}
%
%It's a pretty neat definition that stems from ancient Greece. Unfortunately, it has more recently been argued that it's incorrect. 
%
%The Gettier problem~\parencite{gettier1963is-justified}, for instance, challenges this traditional definition of knowledge. The Gettier problem shows that even if a belief is justified and true, it can still fail to qualify as knowledge. This problem has led to debates about whether the justification condition is sufficient for knowledge, and some have proposed adding additional conditions to the analysis. However, there is also skepticism about whether any analysis of knowledge can fully capture its complexity and nuances. 
%
%This raises real problems for the naive researcher and justifies the need for a lot of additional argument to establish that we have new knowledge.
%
%\begin{question}[subtitle={Optional activity: The Gettier Problem}]
%\textcite{gettier1963is-justified} is only 3 pages long, give it a read and try to understand the examples Gettier uses to question the traditional definition of knowledge.
%\end{question}
%
%The result is that we must be careful to ensure that what we claim to be new knowledge is actually both \emph{new} and \emph{knowledge}.
%
%Against this background of the meaning of knowledge, part of your mission\footnote{...should you choose to accept it.} is to justify your claim to have made a contribution to knowledge.  Given the traditional view of knowledge, for new knowledge you must convince your reader that:
%%
%\begin{enumerate}
%\item the thing you claim to be knowledge is actually true;
%\item that you believe it to be true,
%\item you have a solid justification for believing it,
%\item it hasn't been claimed to be true before.
%\end{enumerate}
%%
%
%Working backwards, the literature review
%
%Let's look at the belief condition first: do you believe your contribution to knowledge to be true. You've expended a great deal of effort in learning, understanding, and writing in convincing yourself that what you have created is believable, and your investment is testimony to the fact that you believe it to be true. Therefore your dissertation is evidence that your contribution to knowledge is something you believe in\todo{Needs to work in validity, reliability, bias and the other elements of this part}.
%
%The thing you ...
%
%You are justified in believing it ...
%
%\subsection{Literature review}
%
%As\todo{refer reader to previous section on this topic.} 
%
%\subsection{In detail: dealing with validity weaknesses}

%Reliability and validity\footnote{Adapted AI summary of \textcite{roberts2006reliability}.} are important – linked – concepts in research that serve to demonstrate the rigour and trustworthiness of both quantitative and qualitative research. Thus, researcher's planning of and reflection on reliability and validity and its reporting should be a part of all research strategies.
%
%Reliability and validity – and the effort needed by you to address them – varies greatly depending on the research strategy chosen. For logical proof, for instance, reliability and validity may be as simple as having a local community of logicians check your work, or even through the use of a computerised proof checker. At another extreme, in grounded theory, for instance, the treatment of reliability and validity will typically form large component of the reported research.
%
%\begin{tblr}{
%colspec={lXX},
%row{1}={font=\bfseries},
%}
%Threat&Meaning&Example\\
%History&An unrelated event influences the outcomes.&A week before the end of the study, all employees are told that there will be layoffs. The participants are stressed on the date of the post-test, and performance may suffer.\\
%Maturation&The outcomes of the study vary as a natural result of time.&Most participants are new to the job at the time of the pre-test. A month later, their productivity has improved as a result of time spent working in the position.\\
%Instrumentation&Different measures are used in pre-test and post-test phases.	&In the pre-test, productivity was measured for 15 minutes, while the post-test was over 30 minutes long.\\
%Testing&The pre-test influences the outcomes of the post-test.	&Participants showed higher productivity at the end of the study because the same test was administered. Due to familiarity, or awareness of the study’s purpose, many participants achieved high results.\\
%\end{tblr}


%\paragraph{Validity} refers to the extent to which a measure accurately represents the concept it claims to measure. It is concerned with the relevance and representativeness of the items or questions in a study. There are different levels of validity, including content validity, criterion-related validity, and construct validity. Content validity focuses on the relevance and representativeness of individual items, while criterion-related validity involves comparing a measure to other similar validated measures. Construct validity examines the underlying theoretical concepts and constructs being measured. Validity is important in ensuring that research findings are accurate and trustworthy.

%Validity\footnote{Adapted AI summary of \textcite{ihantola2011threats}.}, as already mentioned, refers to the extent to which a research study measures what it intends to measure and accurately reflects the concept or phenomenon under investigation. It is a crucial aspect of research quality and involves establishing the credibility, trustworthiness, and accuracy of the findings. 
%
%%Research is vulnerable to claims that it is not valid, and invalid research cannot be a contribution to knowledge. A good researcher will therefore consider each \textit{potential} vulnerability in turn and design their research so that it is robust should that potential vulnerability doesn't threaten their claimed knowledge contribution.
%
%The two main forms of validity are: \emph{internal} validity and \emph{external} validity, with each having a number of weaknesses.
%
%\subsubsection{Confusion reigns: validity} Although, or perhaps because, validity is an extremely important concept, its use as an analytic framework has spawned many different forms. \textcite{wortman1983evaluation}, for instance, say:
%%
%\blockquote{At the level of specific threats to validity, however, the sheer number is both overwhelming and somewhat confusing. Some threats seem a bit esoteric, especially for evaluators (e.g. \enquote{ambiguity about the direction of casual influence} and \enquote{hypothesis-guessing within experimental conditions,} for example); some seem to differ only in small degree (\enquote{compensatory rivalry by respondents receiving less desirable treatments} and \enquote{resentful demoral­ization of respondents receiving less desirable treatments,} for example); and still others (noted above) seem to be miscategorized, thereby blurring the differences among the major validity types. In addition, some of these threats are relevant during the design and planning for an evaluative study while others are more appropriate to the management and conduct of the study (e.g. multiple statistical testing, program implementation, diffusion, etc). Perhaps some consolidation is needed to make the whole structure less cumbersome and \enquote{threatening.}}
%
%Because of this, the treatment of validity below is partial, focussing on three main areas: experiment, observations, and instruments. It would be wise to discuss which expectations your supervisor has of validity\footnote{Schedule a meeting with your supervisor, with the subject \enquote{Validity discussion}.}.
%%
%
%%The validity of a piece of research can be evaluated in many different ways. Those we consider here cover the most often found, and include\footnote{Adapted from \textcite[and enclosing material]{bhandari2023construct}.} internal validity (including), external validity (including), construct validity, content validity, criterion validity, concurrent validity, discriminant validity, face validity, convergent validity, population validity, and predictive validity. 
%
%\paragraph{Internal validity} refers to the extent to which the research design and methodology accurately measure what they are intended to measure. For a conclusion to be internally valid, you need to be able to rule out other explanations (including control, extraneous, and confounding variables) for your results. 
%
%Internal validity applies across research strategies and research instruments, each involving the analysis of a phenomenon, having different interpretations according to each. The various areas of validity are illustrated in figure~\ref{fig:Validity}.
%
%\begin{figure}[hbtp]
%\centering{
%  \includegraphics[width=\textwidth]{figures/InternalValidityTypes}
%  \caption{Types of validity
%  \label{fig:Validity}
%  	}  
%  }
%\end{figure}
%
%\paragraph{Experiment} The forms of validity for experiments\footnote{The acronym for experimental internal validity is \emph{THIS MESS} \parencite{wortman1983evaluation}, hence the addition of \emph{statistical} to \emph{regression toward mean}. Subsequently, other authors have added further validity types; \textcite{cook1979quasi}, for instance, lists 33 potential weaknesses.}, i.e., research designed to establish a cause and effect relationhip, are:
%%
%\begin{itemize}
%\item testing: [if helpful, explanations needed throughout]
%\item history:
%\item instrument change:
%\item (statistical) regression towards mean:
%\item maturation:
%\item experimental mortality:
%\item selection:
%\item selection interaction:
%\end{itemize}
%%
%
%\paragraph{Observations} The forms of validity for observations are:
%%
%\begin{itemize}
%\item selective participation: [explanations needed throughout]
%\item selective recall: 
%\item accentuated perception:
%\end{itemize}
%%
%
%\paragraph{Instruments} The forms of validity for data generation instruments are \parencite[, adapted]{middleton2022the4types}:
%%
%\begin{itemize}
%\item Construct validity: Does the instrument measure the concept that it’s intended to measure?
%\item Content validity: Is the instrument fully representative of what it aims to measure?
%\item Face validity: Does the content of the instument appear to be suitable to its aims?
%\item Criterion validity: Do the results accurately measure the concrete outcome they are designed to measure?
%\end{itemize}
%%
%
%\paragraph{External validity} is the extent to which you can generalise the findings of a study to other situations, people, settings, and conditions \parencite{wortman1983evaluation}. In other words, external validity means the extent to which you can apply your knowledge contribution in a broader context\footnote{In qualitative studies, external validity is also referred to as \emph{transferability}, i.e., how transferable are results.}. 
%
%Although poor external validity may not disqualify your knowledge contribution as novel, it will require you to be extremely careful about its context of applications. Instead, for instance, of being able to say:
%%
%\blockquote{This result applies to all children of school age studying computing}
%%
%you may need to say
%%
%\blockquote{The result applies to all children between 12 and 13 years old, studying computing at the such-and-such a school with teacher A.}
%
%Clearly, the restricted nature of the new knowledge reduces its applicability and use as predictive, for instance.

%External validity breaks down further into population validity and ecological validity.

%%
%\begin{itemize}
%\item Population validity refers to whether you can reasonably generalise the findings from your sample to a larger group of people (the population). Population validity depends on the choice of population and on the extent to which the study sample mirrors that population. Population validity is established by showing that the sample and the population are similar.
%
%\item Ecological validity refers to the extent to which the measures measures how generalisable experimental findings are to the real world, such as situations or settings typical of everyday life. Ecological validity is demonstrated by showing that the restricted research context sufficiently \enquote{mimicked} the real world.
%\end{itemize}
%

%
%\begin{itemize}
%\item Construct validity refers to the extent to which the measures used in the study accurately capture the underlying constructs or concepts of interest.
%
%\item Content validity refers to the extent to which the measures used in the study accurately capture the underlying constructs or concepts of interest.
%
%\item Criterion validity refers to the extent to which the measures used in the study accurately capture the underlying constructs or concepts of interest.
%
%\item Concurrent validity refers to the extent to which the measures used in the study accurately capture the underlying constructs or concepts of interest.
%
%\item Discriminant validity refers to the extent to which the measures used in the study accurately capture the underlying constructs or concepts of interest.
%
%\item Face validity refers to the extent to which the measures used in the study accurately capture the underlying constructs or concepts of interest.
%
%\item Convergent validity refers to the extent to which the measures used in the study accurately capture the underlying constructs or concepts of interest.
%
%\item Predictive validity refers to the extent to which the measures used in the study accurately capture the underlying constructs or concepts of interest.
%\end{itemize}
%%
%



%Validity\footnote{Adapted AI summary of \textcite{wolming2010the-concept}.} is a concept that has evolved over time and has become more complex. Initially, validity was considered to be a fixed property of a test, with adequate correlations between test scores and an external criterion being seen as evidence of validity. However, the definition of validity has now expanded to include different types of validity related to the purpose of the test. These types include \emph{content validity}, \emph{criterion-related validity}, and \emph{construct validity}. 
%
%\paragraph{Content validity} is used for tests that describe an individual researcher's performance on a defined subject and is, for instance, concerned with the relevance and representativeness of items in a questionnaire.
%
%\paragraph{Criterion-related validity} is used for tests that predict future performance. It is established when a research tool can be compared to other similar validated measures for instance, when the results of two questionnaires deliver by two independent samples agree\footnote{If not exactly, then at least within statistical bounds.}.
% 
%\paragraph{Construct validity} is used to make inferences about psychological traits.

\subsection{Weakness types}
\todo{this section is either to develop or to remove; to rethink after the research strategy bit is done} In this section we recall many of the common weaknesses under each of the categories we have introduced. All research projects are subject to some or even most of them, so awareness of them will help you better inform your choice of research strategies and the step to take to address them. 

\subsubsection{Novelty weakness}

To identify your research problem, you will have found a hole in current knowledge through the literature you will have surveyed and reflected critically upon. During subsequent research you may find that you were:
%
\begin{itemize}
\item unable to contribute knowledge in that area, or not able to contribute as much as you had initially hoped;
\item found further sources that had already made a contribution.
\end{itemize}
%
If you do encounter this weakness, you will not be the first – virtually all researchers find that their initial aspirations for a knowledge contribution has to be reduced or altered as their research – and understanding – progresses.



\subsubsection{Validity weaknesses}

%\subsection{Dealing with validity weaknesses}
%%[Distinguish and add Validation of evaluation...]

There are a number of recognised validity weaknesses\footnote{These are often called \enquote{threats to validity} in the literature. We prefer \emph{weakness} as it suggests that there is an issue caused by the research strategy application rather than by the environment.} upon which an evaluation of research validity is made. \todo{why this list? where does it come from?}
%
\begin{itemize}
\item Mismatch between quantitative and qualitative samples
\item Imbalance between an insider's and outsider's views
\item Insufficient knowledge of research question, theory, hypotheses, statistical tests, and analysis
\item Occurrence of unrelated events or conditions during data collection
\item Insufficient or biased knowledge of earlier studies and theories
\item Lack of descriptive validity of settings and events during data analysis and interpretation
\item Population, time, and environmental validity in quantitative research
\item Lack of cognitive and empathy training of researchers
\item Value or ideologically based conflicts in collaboration between quantitative and qualitative researchers
\item Difficulty in persuading consumers to value the meta-inferences from both qualitative and quantitative findings.
\end{itemize}
%

\subsubsection{Reliability weaknesses} refers to the trustworthiness and consistency of the procedures and data generated in research. It is concerned with the extent to which the results of a study or measure are repeatable in different circumstances. Reliability can be demonstrated through methods such as inter-rater reliability, test-retest reliability, and internal consistency.

\subsubsection{Bias}

[Adapted from \cite{simundic2013bias}]

\textcquote{simundic2013bias}{%
Bias is any trend or deviation from the truth in data collection, data analysis, interpretation and publication which can cause false conclusions. Bias can occur either intentionally or unintentionally (1). Intention to introduce bias into someone’s research is immoral. Nevertheless, considering the possible consequences of a biased research, it is almost equally irresponsible to conduct and publish a biased research unintentionally.

It is worth pointing out that every study has its confounding variables and limitations. Confounding effect cannot be completely avoided. Every scientist should therefore be aware of all potential sources of bias and undertake all possible actions to reduce and minimise the deviation from the truth. If deviation is still present, authors should confess it in their articles by declaring the known limitations of their work.
%
%It is also the responsibility of editors and reviewers to detect any potential bias. If such bias exists, it is up to the editor to decide whether the bias has an important effect on the study conclusions. If that is the case, such articles need to be rejected for publication, because its conclusions are not valid.
}

\textcite{simundic2013bias} then goes onto detailed four forms of bias: 
%
\begin{itemize}
\item data collection bias: including selection bias, volunteer bias, admission bias, survivor bias, and misclassification bias

\item data analysis bias: including%
	\begin{itemize}
	\item data fabrication: reporting non-existing data from experiments that were never done;
	\item data elimination: eliminating data which do not support a hypothesis (outliers, or even whole subgroups);
	\item using inappropriate statistical tests to test your data;
	\item performing multiple testing (\enquote{fishing for P}) by pair-wise comparisons, testing multiple end-points and performing secondary or subgroup analyses, which were not part of the original plan in order \enquote{to find} statistically significant differences regardless of hypothesis.
	\end{itemize}
%

\item data interpretation bias: including
\begin{itemize}
\item discussing observed differences and associations even if they are not statistically significant (the often used expression is \enquote{borderline significance});
\item discussing differences which are statistically significant but are not otherwise meaningful;
\item drawing conclusions about causality, even if the study was not designed as an experiment;
\item drawing conclusions about values outside the range of observed data (extrapolation);
\item over-generalisation of the study conclusions to the entire general population, even if a study was confined to the population subset;
\item Type I (the expected effect is found significant, when actually there is none) and type II (the expected effect is not found significant, when it is actually present) errors.
\end{itemize}

\item publication bias: including
%
\begin{itemize}
\item funding bias: due to the prevailing number of studies funded by the same company, related to the same scientific question and supporting the interests of the sponsoring company
\item anti-negative bias: scientific journals are much more likely to accept for publication a study which reports some positive than a study with negative findings. Such behaviour creates false impression in the literature and may cause long-term consequences to the entire scientific community. Also, if negative results would not have so many difficulties to get published, other scientists would not unnecessarily waste their time and financial re- sources by re-running the same experiments.
\end{itemize}
\end{itemize}
%



\subsection{Addressing weaknesses}


\subsubsection{Addressing research weakness: critical literature review}

Alongside reflexivity, described below, addressing novelty weaknesses means returning to your literature survey as your research and understanding increases to cast an increasingly  critical eye over it. Each source should be reconsidered for what you thought it originally said and what you now think it says, using any difference\footnote{In the best case, there will, of course, be no difference!} to drive further reflection on your findings, methods, data generation, or even problem. 

As in the example above, your reader can be made aware of this process and how it has altered your research. Deepening the critical nature of your literature review allows your reader to understand that you are a reflective researcher and can turn any novelty weakness into a research strength!

\begin{question}[subtitle={Activity: more on critical review reflection}]
Periods of critical review reflection.
\end{question}


\subsubsection{Addressing research weakness: triangulation}

Triangulation\footnote{Adapted AI summary of \textcite{mathison1988triangulate}.} is a strategy used to increase the validity and reliability of research. It involves using multiple data sources and methods to arrive at a singular proposition about the phenomenon being studied. Triangulation is necessary to withstand critique by colleagues and enhance the credibility of research. It involves using multiple methods, data sources, and researchers. Triangulation in social science was first introduced in \textcite{campbell1959convergent}.

Triangulation is 
%
\textcquote[, adapted]{denzin1978research}{the combination of methodologies in the study of the same phenomenon}. Triangulation involves different types\footnote{These are explained below.} such as \emph{data triangulation}, \emph{investigator triangulation}, and \emph{methodological triangulation}. Triangulation:

\begin{itemize}
\item  involves using multiple data sources and methods to reach, ideally,  a singular conclusion about the phenomenon being studied
\item  helps to enhance the validity and reliability of research findings, and withstand critique by colleagues by showing that many different methods of reaching a conclusion agree
\item should use multiple methods and sources of data, regardless of the philosophical or methodological perspective
%\item  requires the researcher to make sense of any inconsistent and contradictory evidence.
\item can improve research practice by aiding in the elimination of bias and allowing the dismissal of rival explanations.
\end{itemize}
%
Because triangulation applies many techniques or derives from many sources it can, however, result in inconsistent or contradictory findings, which may themselves pose risks to the validity of research or evaluation findings. It is important to understand that triangulation does not guarantee convergence on a single proposition about a phenomenon. Instead, it provides a rich and complex picture that requires careful interpretation and explanation by the researcher. Triangulation should be used cautiously and researchers should be prepared to explain and make sense of the various outcomes it may produce.


\paragraph{Data triangulation} refers simply to using several data sources, the obvious example being the inclusion of more than one individual as a source of data. However, Denzin expands the notion of data triangulation to include time and space based on the assumption that understanding a social phenomenon requires its examination under a variety of conditions. So, for example, to study the effect of an inservice program on teachers, one should observe teachers at different times of the school day or year and in different settings such as the classroom and the teachers' lounge.

\paragraph{Investigator triangulation} involves more than one investigator in the research process\footnote{Because there is more than one researcher involved, it is unlikely that you will be required to perform this form of triangulation. You may, however, be a researcher in the triangulation of another's research – your supervisor, for instance – which means that you should be prepared to be involved. Be sure to schedule some time with your supervisor to discuss their needs, should this be the case.}, is also considered good practice. This perhaps more than other types of triangulation is usually built into the research process because most studies simply require more than one individual to accomplish the necessary data collection. However, the decision about who these multiple researchers should be and what their roles should be in the research process is problematic \parencite{denzin1978research}. How much hands-on data collection the principal investigator needs to do in order to analyse the data, and how much data analysis is relegated to field workers because much of the analysis occurs as data are collected, are both relevant and not easily answered questions.

\paragraph{Methodological triangulation} is the most discussed type of triangulation and refers to the use of multiple – or mixed – methods in the examination of a social phenomenon\footnote{We deal with mixed method research later in this Stage.}. Psychologists have long used Denzin's notion of within-method triangulation in assessing psychological traits. Multiple scales comprise a psychological assessment such as an intelligence test in an effort to assess the different aspects of intelligence. The lie detector scale in some psychological inventories is another example. Denzin suggests that the within-methods triangulation approach has limited value, because essentially only one method is being used, and finds the between-methods triangulation strategy more satisfying. Other researchers seem to follow this lead and focus primarily on between methods triangulation. \textcquote[p.~302]{denzin1978research}{The rationale for this strategy is that the flaws of one method are often the strengths of another: and by combining methods, observers can achieve the best of each while overcoming their unique deficiencies}. It is with this type of triangulation that Denzin relies most heavily on the work of Webb et al. (1966)\todo{find ref} to suggest that the use of appropriate multiple methods will result in more valid research findings.

%Researchers use various methods, such as pilot studies and comparison to other validated measures, to establish reliability and validity. Computerised data analysis packages can also enhance reliability. It is important to balance standardisation with maintaining the context and meaning of the data.

%\subsubsection{Evaluation of Reliability} ...
%
%% Messick's model introduced the idea of social consequences of measurement outcomes as an important aspect of validity. The introduction of social consequences raised debates about the feasibility and inclusion of consequences in the validity framework. Kane's argumentative approach to validity emphasises the practical feasibility and provides guidance on how to allocate research efforts and gauge the validation process. The integration of validity evidence and the need for guidance on prioritising validity questions are also discussed.
%%

\begin{question}[subtitle={Activity: More on triangulation}]
Read these sources on triangulation
\end{question}



\subsubsection{Addressing research weaknesses: reflexivity}

\textcquote[, adapted]{usher2023case}{Reflexivity\todo{Source material, needs adapting} is the process of a researcher acknowledging how they influence the study (Creswell and Miller 2000; May and Perry 2017; Berger 2015). Part of this process includes the researcher acknowledging their position of power and privilege from which they approach the research study (LeCompte and Schensul 2010). Reflexivity admits that the researcher isn't necessarily an unbiased observer of truth, as might be asserted in the more positivist approaches, but someone interacts with and influences their surroundings. Reflexivity has applications in all types of research, but its recording is most prevalent in the more qualitative forms.

As potentially biased instruments of data collection and analysis, researchers should critically reflect on the ways in which they shape their \enquote{study topics, sampling, interpretation of the findings, and conclusions through their writing} (Lincoln, Lynham, and Guba 2018; Creswell 2013; Johnson 2009).

There are a variety of ways to research reflexively (Genoe and Liechty 2016), for instance through the keeping of field notes, when \enquote{the researcher reflects on their actions, feelings, opinions, and assumptions throughout the process of observation} (May and Perry 2017) and even to questioning the theories, methods, and research strategy used. 

The goal of reflexive practice is to achieve a more honest and transparent research process and product (May and Perry 2017). In qualitative research, typically authors will include a section about positioning themselves in the study, where they reveal their various identities and the ways these may have influenced their interpretations of the data (Creswell 2013)\todo{Rewrite for our purposes}.}

\begin{question}[subtitle={Activity: The reflexive researcher}]
Read these sources on reflexivity.
\end{question}

It is the nature of research that new weaknesses are being discovered regularly. Some are specific and to the point, others of only marginal importance in specific research designs. The presentation below will not bring you up to the leading edge in \enquote{research weakness} research, but it will allow you to address the most important, research-strategy-specific ones.


\section{Your initial research strategy candidate list}\label{sect:standardResearchStrategies}

%\todo{This is a repeated activity through all stages, and so should be templated.}
\begin{question}[subtitle={Activity: Research strategy choice}] Take a look at Table~\ref{tab:ResearchStrategyChoice} and count the number of research strategies that you are going to read about.

Make a note of what you are being asked to complete.
\end{question}
%%Hack to correct tcbox behaviour
\color{black}

\newcommand{\tick}{$\square$}

\begin{question}[subtitle={Activity: Choosing a research strategy}] The next activity you should complete is lengthy and may take a long time to complete, with auxiliary reading, up to one week. 

The aim of this activity is to complete the following table, Table~\ref{tab:ResearchStrategyChoice}, which will track your progress in choosing a research strategy. Although it make not look much, completing each line is a substantial piece of work.
\end{question}
%%Hack to correct tcbox behaviour
\color{black}

\begin{table}
\caption{Research strategy choice\label{tab:ResearchStrategyChoice}}
\begin{tabulary}{\tablewidth}{rccl}
	\textbf{Research Strategy candidate}&\textbf{Considered}&\textbf{Excluded}&\textbf{Reason excluded}\\\toprule
	Survey&\tick&\tick\\
	Design and Creation&\tick&\tick\\
	Experiment&\tick&\tick\\
	Case study&\tick&\tick\\
	Action research&\tick&\tick\\
	Ethnography&\tick&\tick\\
	Systematic research&\tick&\tick\\
	Grounded theory&\tick&\tick\\
	Phenomenology&\tick&\tick\\
	Simulation&\tick&\tick\\
	Mathematical and logical proof&\tick&\tick\\
	Mixed methods&\tick&\tick
\end{tabulary}
%%Hack to correct tcbox behaviour
\end{table}

\subsection{Research strategy introduction}

Below we detail 12 candidate research strategies. 

For each, we give a brief introduction – a brief paragraph explaining the focus of the strategy – followed by the type of knowledge contribution that can be made through it. You will be able to compare these with your research problem to check whether that research strategy should be a candidate. When you have done this, you can fill in the first tickbox column in Table~\ref{tab:ResearchStrategyChoice} – I've considered the strategy. If there's a clear mismatch between the knowledge contribution and your research problem, you might even be able to fill in the second tickbox column – that that research strategy has been excluded – and give a reason why you have excluded it – the knowledge contribution it makes is not of the correct form – and you can move onto the next research strategy.

The \enquote{Reason excluded} column will be used in the your dissertation to justify your choice of research strategy so think deeply about what you write here – you can use the text of the knowledge contribution and subsequent subsections to frame your reason for excluding it. Whatever you do, don't leave it blank!

If you have not been able to exclude the research strategy, then you should read further – next come the techniques you would have to use for data generation. This gives you another reason to exclude a research strategy\footnote{Of course, you will need to choose \emph{one} research strategy, so be careful not to exclude something that wouldn't be too access or to gain skills for.} – that you do not have access to the data generation techniques or the skills to perform them. If this analysis leads you to exclude the research strategy, complete\footnote{This time, the reason will be something to do with data generation technique not accessible.} the second tickbox column and record the reason and you can move onto the next research strategy.

If you get to the \enquote{Evaluation} section, then you've got a candidate strategy. Read through the evaluation section\footnote{Perhaps taking notes on things you haven't immediately understood.}.

The \enquote{Is this strategy right for me} section is next and lists a number of other things you should consider that might allow you to exclude it. If this leads you to exclude it, then fill in tickbox column 2 and give a reason for the exclusion and you can move on to the next research strategy.

Otherwise, you'll have identified a(nother) candidate research strategy and will – through the final activity – have a number of academic resources that you should read to deepen your understanding of it. 

There are three possible outcomes:
%
\begin{itemize}
\item you find yourself with a single candidate research strategy: in which case you should go for it!

\item you find yourself with a number of candidate research strategies: in which case you can make a choice based on your skillset, how much fun you think you could have applying it, or any other criteria you wish. You may also like to think about mixing up bits of each to give a mixed methods research strategy

\item you find yourself without a choice: in which case you've probably been too picky... and you should try again, perhaps discussing your decision with your supervisor – you can't do research without a research strategy and you're unlikely to come up with one not on this list – a completely novel one.
\end{itemize}
%

\begin{question}[subtitle={Activity: discuss your choice with your supervisor}]
Title says it all, really.	
\end{question}


\subsection{Survey research}

Survey research provides – potentially large – amounts of up-to-date, focussed, real-world data, collected in a systematic way, that represent in summary form characteristics of research subjects that are statistically valid and accurate for some population forming the focus of the research~\parencite[p.196]{secor2010social}.

\subsubsection{Knowledge contribution}

The contribution to knowledge of survey research is to uncover patterns that can be  generalised to the population of focus.

%\subsubsection{Variants}

\subsubsection{Data Generation} 

Suggested by the name, a survey – a standardised set of questions administered to a number of respondents – allows the researcher to gather information about a population. Surveys can take many forms, from interviews to questionnaires to focus groups, but authors vary on what they consider appropriate.\footnote{Be sure to consider any supplied preparatory reading on the survey research strategy to ensure that you meet your supervisor's (or other's) expectations of what will be appropriate.}

Typical data generation methods for surveys are via the internet (more traditionally by mail), over the phone, and through face-to-face interviews, although surveys can also be conducted via pre-existing documents. Mixed-mode surveys combine these options into more complex instruments, perhaps using a broader but simpler questionnaire to identify potential participants for a deeper face-to-face interview.

\subsubsection{Evaluation} The following questions are typically asked of the knowledge contribution made through survey research
~\parencite[p.105 (adapted)]{oates2008researching}:
	\begin{enumerate}%[start=0,label={(\bfseries R\arabic*):}]
	\item Is coverage of the parent population appropriately wide and inclusive? Typically, this reduces to the question of whether the sample size is large enough, but may need more analysis in some cases (see stats section).
	
	\item Is there a reflection on the adequacy of the sample frame? Would additional data would have been useful?
	
	\item Are the data generation methods used appropriate and feasible? Has their design and construction been adequately described?
	
	\item Are the sampling frame – the source database from which a sample is drawn\footnote{Although it may not be an actual computer database} – and sampling techniques used adequately explained?
	
	\item Is the response rate adequate? If not, was there appropriate reflection on the effects on the current survey and on future surveys? How were non-respondents handled?
	
	\item Is there an adequate discussion on the differences between respondents and non-respondents? Is any significant differences discussed in the context of this and future work?
	
	\item If generalisations have been made about the parent population, are they appropriate? What reasoning chains have led to the generalisations made?
	
	\item What limitations, flaws, errors and/or omission are there in your use of surveys?
	
	\item Overall, has the survey research strategy been successful?
	\end{enumerate}

\subsubsection{Is the survey research strategy right for me?}

%\endinput

Evaluation sets certain requirements of the researcher for them to be successful. These include that:
	\begin{enumerate}
	\item you're in a position in which you have access to the population so that deep analysis can be performed. If this is not possible, for instance, because this is your first time accessing the population, you might like to consider use case study research instead;
	\item the phenomena and characteristics of the population are measurable through questions asked through a survey. If this is not the case that then you're not going to be able to make a contribution to knowledge about those phenomena or characteristics: if this is the case, you might like to consider phenomena that can be measured, or a different population for which those phenomena can be measured.
	\item conducting a survey means that you'll be analysing phenomena using point data, i.e., data that were collected at a point in time – that time at which the survey was answered. If your research requires longitudinal data, i.e., data that could change over time, then survey research becomes more difficult as you might need two or more surveys to collect the changing data. It's not impossible to do this, but it adds many complications: earlier participants might not be available for later surveys, their mindsets might have changed in the intervening period, etc. If this is the case, then consider whether the choice of phenomena is appropriate. Alternatively, you might like to consider one of the experimental research strategies described below.
	\item the difficulty of conducting two or more surveys also mitigates against trying to establish causes and effects. Again, reconsider phenomena or use an experimental research strategy instead;
	\item conclusions from survey research rely on the veracity of the responses received. If there's any reason to doubt your respondents' honesty, additional care should be taken. There are techniques to avoid this (see \textcite{}, for instance) but they add complexity to the strategy. If the development of a relationship of trust between research and population is critical to the research, then some form of ethnographic research might be more appropriate.
	\end{enumerate}
	
\newcommand{\RSActivity}[2]{\begin{question}[subtitle={Activity: Considering #1}]
	Having read the above subsection, do you consider #1 to be a serious candidate for your research strategy?
%	
\begin{guidance}
	If so, add the following references to your list of reading:
\resourcelistcite{#2}
	\end{guidance}
	\end{question}
%%Hack to correct tcbox behaviour
\color{black}}

\RSActivity{Survey Research}{dillman2014internet,
	%oates2008researching,
	%johannesson2014research,
	kalaian2008encyclopedia}

\subsection{Design and creation research}

The design and creation research strategy\footnote{AKA Design science research strategy.} focuses on developing new solutions to problem, a problem being a need in context. The new solution can be an artefact, process, policy, model, method, anything that satisfies the need in context.

\subsubsection{Knowledge contribution}

The contribution to knowledge is that which can be learned from the creation of an artefact as the solution of a problem. Knowledge contributions therefore come from an exploration of the artefact itself, or from its design, development, use, or other characteristics of the real-world problem solving process.

%\subsubsection{Focus}

%\subsubsection{Variants}

\subsubsection{Data Generation}

Typically the process of problem solving – the interactions between actors (customers, clients, designers, others), technologies and/or knowledge – is the source of data. This includes the nature of the problem solving process, whether it is linear or iterative, for instance, or the ways in which problem and solution understanding and validation are conducted.


\subsubsection{Evaluation}

Evaluation of the design and creation research strategy typically consists of the following questions~\parencite[p.122, adapted]{oates2008researching}:
					%
	\begin{enumerate}%[start=0,label={(\bfseries R\arabic*):}]
	\item Is an artifice designed and created to the satisfaction of the  stakeholders?

	\item Is there novelty in its design, development, and/or creation?

	\item Have you described the development methods used in adequate detail for someone else to follow?
%					\item Is this enough information? Is it an appropriate method?

	\item Do you discuss all stages of the problem solving process, including interactions with stakeholders? Is the discussion convincing?

	\item How is the artifice evaluated? Are the evaluation criteria documented? Are the criteria appropriate? How were they determined?

	\item Do you make generalisation from the design and creation of the artifice? Are the generalisations appropriate?

	\item Overall, how effectively do you think the design and creation strategy has been reported and used?
	\end{enumerate}

\subsubsection{Is the design and creation research strategy right for me?}

%\endinput

Although the design and creation research strategy may explore the development of new skills for a product, most often those skills will need to be learned by the developer (if you don't already possess them). This might be valuable from a personal perspective, but will not, by their nature, make a contribution to knowledge – learning them means that they exist already! Learning new skills is a time consuming task even for those with the right propensity for development. Ask yourself if the skills you will need can be fitted into your research, and assess carefully whether they are a realistic addition to the time that your research will take.
 
The design and creation research strategy will typically work with in a customer/developer relationship, with developer and creator not being the same person. The customer/client will set the requirements and context for the product, with the developer working on its development for that context to emmet those requirements. If you do not have access to a real-world problem owner then this strategy is not applicable. 

If there is a customer, but their need for a solution to their problem is urgent, it may be that its embedding within a research project will not lead to a timely solution. This might be the case if your employer, for instance, is looking to you to solve a pressing business problem, and looks to your research studies as the way to achieve this.

To make a contribution to knowledge there should be demonstrable novelty needed to produce the solution, or in the process, or som other aspect of the development. If you cannot clearly identify that novelty, then you will not be able to claim a contribution to knowledge.

\RSActivity{Design and creation research strategy}{oates2008researching,brocke2020introduction}

\subsection{Experimental research}

Experimental research provides a controlled environment in which cause and effect relationships can be investigated, expressed as a hypothesis. The strength of an experiment is that it can reduce the influence of confounding factors on a cause-effect relationship. 
%From wikipedia:

%\textcquote{enwiki:1195800578}{An experiment is a procedure carried out to support or refute a hypothesis, or determine the efficacy or likelihood of something previously untried. Experiments provide insight into cause-and-effect by demonstrating what outcome occurs when a particular factor is manipulated. Experiments vary greatly in goal and scale but always rely on repeatable procedure and logical analysis of the results.}

The potential scope of application of the experimental research strategy is wide, ranging from scientific experiments under laboratory conditions to natural experimental studies in which \textcquote{enwiki:1195800578}{in which individuals (or clusters of individuals) are exposed to the experimental and control conditions that are determined by nature or by other factors outside the control of the investigators.}

Typically, an experiment will be repeatable, so that the codification of the experimental and its procedures must be described as part of the research in a level of detail for an independent third party to repeat it. 

\subsubsection{Knowledge contribution}

The experiment strategy contributes to knowledge through allowing cause and effect relationships between real-world phenomena to be established. 

%\subsubsection{Focus}
%
%\subsubsection{Variants}
%
%There are many variants of the experimental research strategy, including\footnote{Add context of application for each}:
%%
%	\begin{description}
% 	\item [True] see~\textcite[p.126]{oates2008researching}
%	\item [Quasi] see~\textcite[p.133]{oates2008researching}
%	\item [Uncontrolled] see~\textcite[p.134]{oates2008researching}
%	\end{description}
%	%
%\noindent and, even more specialised, for social science applications
%	%
%	\begin{description}
%	\item [One group, pre-test and post-test] see~\textcite[p.135]{oates2008researching}
%	\item [Static group comparison] see~\textcite[p.135]{oates2008researching}
%	\item [Pre-test/post-test control group] see~\textcite[p.135]{oates2008researching}
%	\item [Solomon four-group design] see~\textcite[p.126]{oates2008researching}
%	\item see \textcite{field2002design}
%	\end{description}
%	
%%\endinput
		

\subsubsection{Data Generation} 

The experimental research strategy involves around making an intervention within tightly controlled parameters. Observations are made of before the intervention and after the intervention and a comparison is made. The difference between observations is assumed associated with the intervention made.

Depending on the complexity of the relationship between cause and effect, more or less complex experimental designs can be used. Those involving an inaccessibly large population of individuals, as might be the case for a medical drug trial, use sophisticated techniques to choose representative samples. Such techniques can use sophisticated statistics.

However, even simpler \enquote{local} cause-effect hypotheses may rely on the availability of a fully equipped scientific laboratory to work.

\subsubsection{Evaluation}

Evaluation of the experimental research strategy typically consists of the following questions \parencite[p.138, adapted]{oates2008researching}%
	\begin{enumerate}%[start=0,label={(\bfseries R\arabic*):}]
	\item Was a hypothesis or predicted outcome of the experiment clearly stated?% in the introduction to the research?
%	\item Is the precise form of the experiment adequately described: for instance, is it a true experiment, a quasi-experiment, or an uncontrolled trial? Was it conducted under laboratory conditions, or in a wild or social setting?
	\item Are the independent and dependent variables manipulated or measured in the study adequately described? What additional information is given?
	\item Has sufficient information been supplied so that the experiment can be repeated by an independent third party? What are the experimental protocols? 
	\item do I have access to all equipment I will need to successful test the hypothesis?
	\item do I have or can I gain the statistical skills I need to be able to complete data gathering and analysis?
	\item In a social setting, what information is given about any participants and how they were found? 
	\item In information about how representative the sample is of the wider population sufficient to draw conclusions on representativeness? Are you satisfied that the sample is representative?
	\item What information is given about the apparatus and the process the researchers used to make measurements? What additional information would you like?
	\item What limitations in their experimental research do the researchers recognise and document?
	\item Have flaws or omissions in the researchers' experimental protocols or reporting of their experiment been identified and acknowledged?
	\item Is the experimenter's statistical analysis adequate, have the statistical tools been used and their use justified? Are the statistical and other analyses convincing of the conclusions?
%	\item Overall, how effectively do you think the experiment strategy has been reported and%
	\end{enumerate}

\subsubsection{Is this strategy right for me?}

Although widely applicable, the experimental research strategy is not universally so. Counter-indications to its use include:
%
	\begin{itemize}
	\item can my research question be expressed as an hypothesis between a cause and an effect – for instance, as \textit{Does phenomenon X cause phenomenon Y?}. If not, cross experiments off you candidate list, and consider other methods.
	\item when the cause/effect relationship is complex, for instance, depending on many factors;
	\item when confounding factors and variable cannot be isolated;
	\item when no falsifiable hypothesis can be identified;
%	\item {\color{red}others?}
	\end{itemize}

In these cases, another research strategy should be chosen\footnote{Can we give guidance on which one?}

\RSActivity{experimental research strategy}{%oates2008researching,
	%johannesson2014research,
	field2002design}	

%\endinput

\subsection{Case study research}

Case study research proceeds through the study of a single instance of the phenomenon to be investigated. The study of a single phenomenon requires the researcher to delve deeply into the context of that phenomenon, whether that be a project, an organisation, an engineered system, a policy, or any other thing, relationship, or context.

\subsubsection{Knowledge contribution}

The knowledge contribution is a detailed insightful description of the phenomenon, including when appropriate its relationships with other phenomena and the processes in which it engages.

%\subsubsection{Focus}

\subsubsection{Variants}

Case studies come in many forms:%
	\begin{itemize}
	\item exploratory: in which the researcher explores a research problem sufficiently to be able to conduct a further study. If you're considering studying for a PhD after your masters research, then this might provide a head start for future research ~\parencite[p.143, adapted]{oates2008researching};

	\item  multiple: in which the researcher provides a \enquote{rich, detailed analysis of a phenomenon and its context}. This provides an opportunity, for instance, for multiple stakeholder views to be taken into account as they experience the phenomena in context in different ways, in some ways moving toward a phenomenological account or in which the relationships between phenomena can be aanlysed~\parencite[p.143, adapted]{oates2008researching}; 

	\item longitudinal: in which the researcher considers the state of a phenomenon over time. This offers a natural storytelling context in which change in the phenomenon and/or its context can be analysed~\parencite[p.143, adapted]{oates2008researching};

	\item in combination: in which combinations of the above are analysed, including how relationships between phenomena and/or stakeholders develop over time or in response to contextual factors.
	\end{itemize}
%and have many possible foci \parencite[p.44, adapted]{johannesson2014research}:
%	\begin{itemize} 
%	\item Focus on One Instance: in which a single  The idea is \enquote{To see a World in a Grain of Sand, And a Heaven in a Wild flower, Hold Infinity in the palm of your hand, And Eternity in an hour} as expressed by William Blake (2012).
%	
%	\item Focus on Depth. As much information as possible about the instance studied should be obtained, without shying away from any details.
%	
%	\item Natural Setting. The instance exists before and independently of the research project, and it should be studied in its ordinary context; it should not be moved to, or created in, a laboratory.
%	
%	\item Relationships and Processes. The instance should be studied in a holistic way, taking into account all the relationships and processes within the instance as well as in its environment.
%	
%	\item Multiple Sources and Methods. Multiple information sources should be consulted in order to obtain rich, many-faceted knowledge about the instance; when doing this, different data collection methods could be used, such as interviews and observation.
%	\end{itemize}
					
\subsubsection{Data Collection}

In case study research, data collection can happen through any appropriate data collection technique, whether through observation of the phenomena \textit{in situ} and the context and process in which it participates, surveys of those that deal with the phenomena (through interviews, questionnaires, \textit{etc}), documentation that directly or indirectly describe the phenomena. 

Case studies are particularly appropriate for those embedded alongside the phenomenon of study as might be the case, for instance, of an employee of an organisation takes to investigate a phenomena through a research study; in this case the techniques of ethnography and action research might also apply.

Depending on context, both qualitative and quantitative data will be collected and data analysis can be very rich and complex.

\subsubsection{Evaluation}

An experienced researcher evaluating case study research will ask the following questions:

\textcite[p.151, adapted]{oates2008researching}%
	\begin{enumerate}%[start=0,label={(\bfseries R\arabic*):}]
	\item Have the criteria for choosing the particular case study been described and justified? Is the choice appropriate for the phenomenon studied?
	
	\item Has the variant of case study research been clearly described?
	
	\item Which data generation methods were uses? Did they generate the right type of data about the phenomenon in sufficient quantities?
	
	\item Was the researcher able to work in the case study context? If so, how long was spent there? If not, how was a detailed investigation of the phenomena conducted? 
	
	\item Have you commented on the limitations experienced in the case study research due to any limitations on access to the phenomena? How was, for instance, commercial-in-confidence information handled?
	
	\item Does the research adequately describe any dynamic relationships between phenomena and the processes in which the phenomena participate?
	
	\item What generalisation were made from the case study research? Are the generalisations appropriate for the phenomena and its context?
	
	\item What use of theory of the phenomenon is made in the case study? IS the theory chosen appropriate? If no theory was used, how is the theoretical basis of the research covered?
	
	\item What limitations in the case study research have been recognised?
	
	\item Overall, how effectively was the case study research strategy applied and reported?
	\end{enumerate}
	%
	
%\endinput
					
\subsubsection{Is this strategy right for me?}

From the evaluation, you'll see that an experienced case study researcher will be looking for rich, detailed descriptions of the phenomena and its relationships. Case study research therefore requires you to have access the a phenomena at an intensity at which such richness and detail can be  perceived. 

As an example, if you're not a teacher, it might be difficult to gain access to a classroom to study student/teacher interactions. Should you not have appropriate access then another research strategy would be a better choice in which such access isn't detrimental. This might include systematic research reviews, for instance, which work from secondary sources.

As an other example of potential difficulties, as well as access to the engineering context understanding the processes by which an engine controller in an aircraft is designed may require detailed understanding of technical documentation, language and even mathematical or computational theories. Constructing this knowledge background from zero as part of your research studies may consume a lot of time; the success of your research will depend critically on climbing any learning curve quickly and successfully\footnote{Even if that learning curve looks like El Capitan!}.

Access to policy, processes, and procedures within an organisation will require interaction with others. Even if you already have a good relationship with them they might not have the time to assist you sufficiently for your data generation to be successful.

%%
%\begin{enumerate}%[start=0,label={(\bfseries R\arabic*):}]
%\item large amounts of data are needed: use ??? instead
%\item ???
%\end{enumerate}
%			

\RSActivity{case study research}{oates2008researching}

\subsection{Action research}

\textcquote[3.1.6, adapted]{johannesson2014research}{The Action Research strategy is used to address practical problems that appear in real-world settings. An action researcher does not only strive to generate new  knowledge but also to solve important problems that people experience in their practices.}

\subsubsection{Knowledge contribution}

The knowledge contributed through the action research strategy originates in real-world needs, which might a problem that a stakeholder or collection of stake holders experience. The action research strategy \textcquote[p.168, adapted]{oates2008researching}{immerses the researcher in real-world situations, rich contexts, and the actual problems experienced therein.}

%\subsubsection{Focus}

%\subsubsection{Variants}

\subsubsection{Data Generation}

???

\subsubsection{Evaluation}

An experienced researcher evaluating action research will ask the following questions:

\parencite[p.169, adapted]{oates2008researching}%
	\begin{itemize}
	\item Has the work used an iterative cycle of plan-act-reflect? How many cycles were conducted? 

	\item Did the action research reach its goal? Is any shortfall accounted for?
	
	\item Do you make explicit your \textit{framework of ideas}, \textit{methodology} and \textit{area of application}?
	
	\item What data generation methods do you used? Were they appropriate, and was enough data collected?
	
	\item Is the level of participation achieved discussed? Are any limitations in your outcomes due to a shortfall in participation? Which were they? Are they substantially accounted for?
	
	\item Is there a reflection on self-delusion and group-think of participants? How was this mitigated? Was the mitigation successful? If not, what was the outcome?
	
	\item Are any claims for practical and research outcomes appropriate for action research context?
	
	\item Which limitations, flaws, and omissions in the use of action research have been identified?
	
	\item Overall, has the effectiveness of the action research strategy been reported and used?
	\end{itemize}
					

\subsubsection{Is this strategy right for me?}

Adapted from~\textcite[p.168]{oates2008researching}%
					\begin{itemize}
					\item Will your supervisor/external/readership be receptive to the use of the action research strategy? If not, consider the case study research strategy or the design and creation research strategy as this is 
					
					\item Do you need high levels of rigour? High levels of rigour might not be possible using the action research strategy, or may only be possible if you already have them in your skill set and bring them to the research. 
					
					\item Do you need to establish a cause and effect between phenomena? Action research is ostensibly a real-world problem solving research strategy and may not be suitable complex causeal-effect relationships If so, consider experimental strategy
					
					\item Do you need to be able got generalise your research widely? If so, consider case study strategy.
					
					\item Will the organisation in which you are embedded require you to work for them, rather than to conduct research? If so, ensure they are clear that you are not a consultant.
					
					\item Do envisage issues working with others in complex, problematic and unpredictable real-world situations? If so, consider ???
					\end{itemize}

\RSActivity{action research}{oates2008researching,johannesson2014research}

%\endinput

\subsection{Ethnography}

The Ethnography research strategy attempts to describe people or cultures. It has its roots in anthropology.

\subsubsection{Knowledge contribution}

Ethnography contributes to knowledge through the study of phenomena in their natural epistemo-socio-technological setting, where their context influences how they are reacted to~\parencite[p.182, adapted]{oates2008researching}, providing a rich, detailed picture of a particular situation or work practices, placing them in their real-world context~\parencite[p.181]{oates2008researching}.

%\subsubsection{Focus}

%\subsubsection{Variants}

\subsubsection{Data Generation}

***

\subsubsection{Evaluation}

\parencite[p.180, adapted]{oates2008researching}%
	\begin{itemize}
	\item Have the lifestyles, meanings and beliefs of the epistemo-socio-technological setting been described adequately?
	
	\item Are the data generation methods that were used described? Did they lead to sufficient data having been collected?
	
	\item Was adequate time spent in the field? What reflection has been done on the time spent in the field?
	
	\item Is the approach holistic \parencite{}, semiotic \parencite{},  or critical \parencite{}?
	
	\item Is the ethnography a standalone description, or has it been linked to theory, other ethnographies or issues in other cultures?
	
	\item Does the research include a reflective account of the researcher?
	
	\item To which extent is the research presented as an ethnographic construction rather than as a literal description?
	
	\item What limitations in the ethnography have been recognised?
	
	\item Which other flaws and/or omissions in the reporting of the ethnography have been described?
	
	\item Overall, how effectively has the ethnography research strategy been?
	\end{itemize}					

\subsubsection{Is this strategy right for me?}

\parencite[p.182,adapted]{oates2008researching}
Ethnography requires you to be a researcher located within the context of your situated research. This can take extensive amounts of time, such as might be the case if the context of your research is the organisation for which you work. However, if you have yet to have identified the context, or have yet to reach out to it, then this requirement may mean that ethnographic research will not be successful. If you are not already close to your context of research, you may wish to consider case study research instead.

Even if you are already located within the context of your ethnographic research, the context must be accepting of an ethnographic approach for your research to be successful. An organisation, for instance, in which there is a culture of strict compartmentalisation may not provide sufficient opportunities for ethnographic research. 

In ethnographic research you allow the culture to determine the outcomes of the research. This precludes bias and prejudice. If there is any possibility that you could be biased to a particular outcome – as might happen if you feel you already know the outcome and are simply trying to confirm this – then ethnography is unlikely to lead to a successful outcome for your research. Any competent ethnographer will be particularly sensitive to expressions of bias, even if they aren't even intentional. Indeed, such bias may preclude any successful research strategy being applied.

Ethnography is analytical in the extreme. Should you not have an analytical mindset, then ethnography should be avoided.

%					\begin{itemize}
%%					\item you don't have extensive periods of time to be a located researcher: try case studies instead;
%%					\item are you especially biased, perhaps not intentionally, so as to be unable to give a account? If so, try ???
%%					\item are you sure you have an analytical mindset that will lead to useful research outputs? If not, try ???
%%					\item is the research context in which you are located accepting of the values of ethnography?
%					\end{itemize}
%					%
					

\RSActivity{ethnography}{oates2008researching,johannesson2014research}


\subsection{Systematic research reviews}

A systematic research review provides a definitive guide to the literature for a specified research topic. MAy allow further conclusions from across the literature to be drawn where, for instance, there is insight from across the literature not contained in individual research papers.

\subsubsection{Knowledge contribution}

Systematic research reviews relies on explicit, reproducible methods for identifying all relevant primary research the world over in all languages; similarly for the critical appraisal of identified research; the results of studies systematically brought together. Acts through literature searching, screening, data extraction, and analysis~\parencite[p.24]{wright2007write}.
				
\subsubsection{Focus}

See~\textcite{moher2009preferred} for PRISMA statement, a 27-item check-list. The aim of the PRISMA Statement is to help authors improve the reporting of systematic reviews and meta-analyses. PRISMA may also be useful for critical appraisal of published systematic reviews. However, the PRISMA checklist is not a quality assessment instrument to gauge the quality of a systematic review.

%\subsubsection{Variants}

\subsubsection{Data Collection}

Systematic review research involves the process of systematically bringing together the results of any research, including qualitative or mixed methods research studies; the identification of all primary research relevant to the defined review (research) question, a critical appraisal of this research, and a synthesis of the findings \parencite{pollock2018systematic}.
				
\subsubsection{Evaluation}

Evaluation of the systematic research review will involve answers to the following:

%
	\begin{itemize}
	\item has the researcher accessed all relevant primary research in the area? Are the criteria they have used to find the primary research explicit and reproducible? Were there any deviations from this protocol and, if so, is there clear reasons where and why documented?
	
	\item Have the relative strengths and weaknesses of the research reviewed been described? To which extent have any identified conflicts between sources been identified and resolved?
	
	\item To what extent has a definitive synthesis from the literature been achieved? 	\item To which extent have statistics been used across the survey to, for instance, produce overarching conclusions? To which extent has precision and/or generalisability been improved through the systematic research review?

	\item given the research question, to which extent is the review type the most appropriate ~\parencite[p.142]{pollock2018systematic}?
	
	\item have implications for future research and practice been discussed both for the research question and in other areas that were raised by the review?
	
	\item do you discuss the updating of the review~\parencite[p.142]{pollock2018systematic}?
	
	\item To which extent is the limits of current knowledge described?
	\end{itemize}

%{\parencite[adapted]{wright2007write}: 
%%
%	\begin{enumerate}%[start=0,label={(\bfseries R\arabic*):}]
%%	\item Clarifying the relative strengths and weaknesses of the literature on the question, 
%%	\item Summarising a large amount of literature, 
%%	\item Resolving literature conflicts, 
%%	\item Evaluating the need for large interventions, 
%%	\item Avoiding a redundant unnecessary interventions, 
%%	\item Increasing the statistical power of smaller studies, 
%	\item Improving the precision or identify of smaller interventions, and 
%	\item Improving the generalisability of intervention outcomes.
%	\end{enumerate}
%	%
					
\subsubsection{Is this strategy right for me?}
%
Are you able to consult all available literature, i.e., do you have access to a university library with a large research collection in your chosen discipline? If not, although there are ways of obtaining research that isn't in your library's collection, such as by contacting the author(s) directly. Although most authors will be happy to sent their published research to you, the round trip time can introduce lengthy delays in the research process, or make analysis of the literature in a particular area more difficult and systematic as you wait for the requested research to arrive. You may also need to be persistent to ensure that a busy author is aware of your research need.

The most successful systematic research surveys begin from a broad focus that is, perhaps, narrowed down as research progresses. The original question should not therefore be so narrow as to restrict the initial literature down too far – there must be some opportunity for conclusions from \emph{across} the research to be reached. Moreover, in an area in which there is little room for disagreement, it may be that study heterogeneity~\parencite[p.27]{wright2007write} precludes meta-analysis.

You will be expected to have stated the explicit inclusion and exclusion criteria for the survey research so that another research would be able to find the same collection. The particular choice of focus may make this a difficult task. For instance, survey research in the area of information technology (IT) may lead to a search term in which \enquote{IT} is part of your search terms, and this may have the unfortunate effect of not revealing anything useful – \enquote{it} is a very common word.

You should not allow any inherent bias that you have influence your documentation of the literature. This is often difficult to do.


%	\begin{itemize}
%%	\item unable to consult all relevant literature; if so perhaps you're only trying to complete a preliminary literature review? \parencite[p.414]{andrews2005place}
%%	\item identified focus is too narrow to provide generalisable results~\parencite{wright2007write}
%%	\item difficult to construct strict inclusion and exclusion criteria for the survey
%%	\item no meaningful conclusion reached due to paucity of literature
%%	\item study heterogeneity precludes meta-analysis, the authors of the systematic review need to summarise the findings based on the strength of the individual studies and reach conclusions if indicated~\parencite[p.27]{wright2007write}
%%	\item won't be more comprehensive than what exists~\parencite[p.405]{andrews2005place}
%%	\item there is some latent bias possible~\parencite[p.405]{andrews2005place}
%%	\item it will not transparent or nor replicable~\parencite[p.405]{andrews2005place}
%	\end{itemize}
%	%
					

\RSActivity{systematic research survey}{wright2007write,moher2009preferred,pollock2018systematic}


%\endinput

\subsection{Grounded theory}

\textcquote{drew202310-grounded}{Grounded theory aims to understand social phenomena by systematically collecting and analyzing data without preconceived notions or theoretical frameworks. The process involves iterative coding and constant comparison of data to generate concepts, categories, and relationships.}

\textcquote[3.1.5]{johannesson2014research}{Grounded theory is a research strategy that strives to develop theories through the analysis of empirical data. In contrast to experiments, grounded theory does not start with a hypothesis to be tested but instead with data from which a theory can be generated. Grounded theory also differs from research strategies, such as ethnography, which are content to provide rich descriptions of particular situations, but no theories. Grounded theory challenges a top-down theorising approach, in which the researcher first develops a theory and then checks whether it conforms to empirical data. Instead, grounded theory insists that empirical data is the starting point, upon which theories are to be built. Theory emerges through analysis and is grounded in the data.}

\textcquote{drew202310-grounded}{Grounded theory is an innovative way to gather qualitative data that can help introduce new thoughts, theories, and ideas into academic literature. While it has its strength in allowing the “data to do the talking”, it also has some key limitations – namely, often, it leads to results that have already been found in the academic literature. Studies that try to build upon current knowledge by testing new hypotheses are, in general, more laser-focused on ensuring we push current knowledge forward. Nevertheless, a grounded theory approach is very useful in many circumstances, revealing important new information that may not be generated through other approaches. So, overall, this methodology has great value for qualitative researchers, and can be extremely useful, especially when exploring specific case study projects. I also find it to synthesize well with action research projects.}

%\endinput

\subsubsection{Knowledge contribution}

A theory derived from empirical data\footnote{Examples here: \url{https://helpfulprofessor.com/grounded-theory-examples/}}
%\subsubsection{Focus}

%\subsubsection{Variants}

\subsubsection{Data Collection}

%\endinput

Empirical data is extant;~\textcquote[p.13]{strauss1998basics}{In fact, Patton (1990), a qualitative evaluation researcher, made the comment, \enquote{Qualitative evaluation inquiry draws on both critical and creative thinking – both the science and the art of analysis} (p. 434). He went on to provide a list of behaviors that he found useful for promoting creative thinking, something every analyst should keep in mind. These include (a) being open to multiple possibilities; (b) generating a list of options; (c) exploring various possibilities before choosing any one; (d) making use of multiple avenues of expression such as art, music, and metaphors to stimulate thinking; (e) using nonlinear forms of thinking such as going back and forth and circumventing around a subject to get a fresh perspective; (f) diverging from one's usual ways of thinking and working, again to get a fresh perspective; (g) trusting the process and not holding back; (h) not taking shortcuts but rather putting energy and effort into the work; and (i) having fun while doing it (pp. 434–435).~\parencite{patton1990qualitative}}

%\endinput

\subsubsection{Evaluation}

%\endinput

\textcquote{corbin1990grounded}{The success of a research project is judged by its products. Except in unusual instances when these are only orally presented, the study design and methods, findings, theoretical formulations, and conclusions are judged through publication. Yet, how are these to be evaluated and by what criteria? When judging qualitative research it is not appropriate, we have asserted, to use criteria ordinarily used to judge the procedures and canons of quantitative studies. It has been one of the aims of this paper to show how the grounded theory approach accepts the usual scientific canons but redefines them carefully to make them appropriate to its specific procedures. In the instance of any grounded theory study, the specific procedures and canons as described above should be part of its evaluation.}

%\endinput

%\textcite{smith1997understanding}
%					\begin{enumerate}%[start=0,label={(\bfseries R\arabic*):}]
%					\item are all processes~\parencite[remove when checked]{strauss1998basics} made explicit? 
%					\item have a range of data sources been used?
%					\item {}[add as general initial item]Was the research question which originated the study included in the final report? 
%					\item were the phenomena to be studied observable in social interaction?
%					\item {}[general]does the literature review allow the reader to identify the issues that the researcher found interesting initially? ~\parencite{smith1997understanding}
%					\item {}[general?]does the literature review provide a backdrop against which the new findings can be evaluated?~\parencite{smith1997understanding}
%					\item have the outcomes been triangulated - has more than one way of arriving at the theory been used?~\parencite{smith1997understanding}
%					\item has a wide range of participants been used? Did the importance of an issue come through repeatedly? Did the participants agree with the analysis process?~\parencite{smith1997understanding}
%					\item has the extent to which the theory is supported by the extant data been described?~\parencite{smith1997understanding} 
%					\item what new insight do the developed theories provide?~\parencite{smith1997understanding} 
%					\item {}[general and specific]have ideas for further study and implication for practice been discussed~\parencite{smith1997understanding}?
%					\end{enumerate} 
%					%
%					and from \textcite[p.425]{corbin1990grounded}
%					%
%					%
%					\begin{itemize}
%					\item [The Research Process:]
%					\begin{enumerate}[label={Criterion \arabic*:}]
%						\item How was the original sample selected? What grounds (selective sampling)?
%						\item What major categories emerged? 
%						\item What were some of the events, incidents, actions, and so on that (as indicators) pointed to some of these major categories?
%						\item On the basis of what categories did theoretical sampling proceed? That is, how did theoretical formulations guide some of the data collection? After the theoretical sampling was done, how representative did these categories prove to be?
%						\item What were some of the hypotheses pertaining to conceptual relations (that is, among categories), and on what grounds were they formulated and tested?
%						\item Were there instances when hypotheses did not hold up against what was actually seen? How were these discrepancies accounted for? How did they affect the hypotheses?
%						\item How and why was the core category selected? Was this selection sudden or gradual, difficult or easy? On what grounds were the final analytic decisions made?
%					\end{enumerate}
%
%					\item [Empirical Grounding of findings]
%\begin{enumerate}[label={Criterion \arabic*:}]
\paragraph{Are concepts generated?}

\textcquote{corbin1990grounded}{Since the basic building blocks of any grounded theory is a set of concepts grounded in the data, the first question to be asked of any publication is: Does it generate (via coding-categorizing activity) or at least use concepts, and what is or are their source or sources? If concepts are drawn from common usage (such as, “uncertainty”) but not put to technical use, then these are not concepts in the sense of being part of a grounded theory, for they are not actually grounded in the data themselves.}
					
\paragraph{Are the concepts systematically related?}
					
\textcquote{corbin1990grounded}{The name of the scientific game is systematic conceptualization through conceptual linkages. So, the questions to ask here of a grounded theory publication are whether such linkages have been made and do they seem to be grounded in the data? Furthermore, are the linkages systematically carried out? As in other qualitative writing, the linkages are unlikely to be presented as a listing of hypotheses or in propositional or other formal terms but will be woven throughout the text of the publication.}
					
					\paragraph{Are there many conceptual linkages and are the categories well developed? Do they have conceptual density?}
					
\textcquote{corbin1990grounded}{If there are only a few specified conceptual relationships, even if grounded and identified systematically, this leaves something to be desired in terms of the overall grounding of the theory. A grounded theory should be tightly linked, both in terms of categories to their subcategories and between categories in the final integration in terms of the paradigm features conditions, context, action/interaction (strategies) and consequences. Also categories, as mentioned in the body of the paper, should be theoretically dense (have many properties that are dimensionalized). It is the tight linkages, in terms of the paradigm features and density of the categories, that give a theory its explanatory power. Without these, the theory is less than satisfactory.}
					
					\paragraph{Is there much variation built into the theory?}
					
\textcquote{corbin1990grounded}{Some qualitative studies report only about a single phenomenon and establish only a very few conditions under which it appears, and specify only a few actions/interactions that characterize it, and a limited number or range of consequences. By contrast, a grounded theory monograph should be judged in terms of the range of its variations and the specificity with which these are spelled out in relation to the data that are their source. In a published paper, the range of variations touched upon may be more limited, but the author should at least suggest that the fully study included their specification.}
					
					\paragraph{Are the broader conditions that affect the phenomenon under study built into its explanation?}
					
\textcquote{corbin1990grounded}{The grounded theory mode of research requires that the explanatory conditions brought into analysis are not restricted to those that seem to have immediate bearing on the phenomenon under study. That is, the analysis should not be so “microscopic” as to disregard conditions that derive from more “macroscopic” sources: for instance, those such as economic conditions, social movements, trends, cultural values, and so forth.
					
These also must not simply be listed as background material but directly linked to phenomena through their effect on action/interaction, and through these latter to consequences. Therefore, any grounded theory publication that either omits these broader conditions or fails to explicate their specific connections to the phenomenon(a) under investigation, falls short in its empirical grounding.}


					
					\paragraph{Has “process” been taken into account?}
					
\textcquote{corbin1990grounded}{Identifying and specifying change or movement in the form of process is an important part of grounded theory research. Any change must be linked to the conditions that gave rise to it. Process may be described as stages or phases and also as fluidity or movement of action/interaction over the passage of time in response to prevailing conditions.}
					
					\paragraph{Do the theoretical findings seem significant and to what extent?}
					
\textcquote{corbin1990grounded}{The question of significance is generally thought of in terms of the relative importance of a theory for stimulating further studies and for giving useful explanations of a range of phenomena. We have in mind here, however, the adequacy of a study’s empirical grounding in relation to its actual analysis insofar as this combination of activities succeeds or fails, in some degree, at producing useful theoretical findings. If the researcher simply follows the grounded theory procedures/canons without any imagination or insight into what the data are reflecting - because he or she fails to see what they are really saying except in terms of trivial or well known phenomena - then the published findings can be judged as failing on this criterion. Recollect that there is an interplay between the researcher and the data, and no method, certainly not the grounded theory one, can insure that the interplay will be creative. This depends on three characteristics of the researcher: analytic ability, theoretical sensitivity, and sensitivity to the subtleties of the action/interaction (plus sufficient writing ability to convey the findings). Of course, a creative interplay also depends on the other pole of the re­searcher-data equation: the quality of the data collected or utilized. An unimaginative analysis may in a technical sense be adequately grounded in the data, but actually it is insufficiently grounded for the researcher’s theoretical purposes. This is because the researcher either does not draw on the fuller resources of data or fails to push data collection far enough.}

%\endinput

\textcquote{corbin1990grounded}{This double set of criteria, for the research process and for the empirical grounding of the theoretical findings, bear directly on the issues of how verified any given grounded theory study is and how this is to be ascertained. When the study is published,, if components of the research process are clearly laid out and if there are sufficient cues in the publication itself, then the presented theory or theoretical formulations can be assessed in terms of degrees of plausibility. We can judge under what conditions the theory might fit with “reality”, give understanding, and be useful (practically and in theoretical terms). Researchers themselves can be rendered more aware of precisely what their operations have been and the possible inadequacies of these operations. In other words, they would be able to identify and convey what were the limitations of their study.}

%\endinput
\subsubsection{Is this strategy right for me?}

%
\begin{itemize}
\item Without extant data, grounded theory cannot get started, use Experiment instead
\item A theory is not the goal state: if no theory is required, use ethnography instead
\item the research question is too focussed\footnote{Meaning?}: 
\end{itemize}
%
				
\RSActivity{case study research}{smith1997understanding,corbin1990grounded,strauss1998basics,gibson2013rediscovering,charmaz2014constructing}


\subsection{Phenomenology}

\textcquote[3.1.7]{johannesson2014research}{Phenomenology is a research strategy that focuses on the perceptions and experiences of people as well as their feelings and emotions.}

\subsubsection{Knowledge contribution}

\textcquote[3.1.7]{johannesson2014research}{The goal of a phenomenological study is not to establish cause and effect relationships or to describe a population through statistical means. Instead, it aims to describe and understand the lived experiences of people and thereby provide insight about the topic being studied.}

%\subsubsection{Focus}
%
%\subsubsection{Variants}

\subsubsection{Data Generation}

Data generation for the Phenomelogy research strategy is mainly through the long, unstructured interview, designed for the subject to: \textcquote[3.1.7]{johannesson2014research}{really tell their own story without being unduly influenced by the researchers.}
				
				\textcquote[, 26m29s]{office2020the-phenomenological}{interviews, follow-up interviews (to address any gaps in data like misunderstandings, missing information, unclear information, etc.), focus groups, field notes, journaling, audio recording, and video recording.
				A combination of instruments is ideal rather than one so that findings are rich, but dissertation students should also be realistic about choosing various instruments so that they do not overwhelm themselves with unrealistic expectations.}

\subsubsection{Evaluation}

\parencite[1h0m39s]{office2020the-phenomenological} 
				%
					\begin{itemize}
					\item Will the experience as described be understandable to any reader and can be identified by anyone who has had that particular experience?
					\item Is the description of the phenomenon clearly presented so that experience differs from other experiences that are similar? {\color{red}what does this mean?}
					\item Are quotations from the data used to demonstrate the emergence of themes?
					\item Is there a discussion of discrepancies among participants and how those discrepancies were factored in data analysis?
					\item Have meaning units, themes, and summaries been described?
					\item Are meaning units grouped together to form themes?
					\item Are themes combined to form a composite summary of the phenomenon?
					\item Are quotes used to support the findings?
					\item Research participants will have their individual ways of experiencing a certain phenomenon. Have you looked for these common to all or most of the participants and not clustered meaning units together where significant differences exist?
					\end{itemize}
					%
					and the research process
					%
					\begin{itemize}
					\item Bracketing/Epoché/Phenomenological reduction - have you discussed how judgments were suspended to focus on analysis of experience. How did you use suspend your judgments to focus on the analysis of participants' experiences?
					\item Horizon - during data analysis, what was your present experience, your horizon? The horizon cannot be bracketed so you will need to discuss that not everything could have been realized by you, the researcher. This discussion might also lead into a discussion about future research implications in Chapter 5.
					\item Intentionality - discuss your level of scrutiny of the data you analyzed. How did you keep your focus on the topic you were studying? Perhaps you slowed down and dwelled on each narrative and did not pass over the details of the account as if you understood it already.
					\item Dasein - How has your Dasein (being-there) affected the research? How did the research affect your Dasein?
					\item Fore-sight/Fore-conception - What was your preconceived knowledge about the phenomenon you were studying?
					\item Hermeneutic Circle - How did were your understandings revised as you analyzed the data?
					\end{itemize}
					%

					%

%				Phenomenology is sometimes presented as having similarities with ethnography \parencite[p.52]{johannesson2014research}, and this leads us to the following evaluation criteria{\color{red} sense?}:
%					\parencite[p.180, adapted for phenomology]{oates2008researching}%
%					\begin{itemize}
%					\item Does the research focus on lifestyle, meaning, and belief?
%					\item Are the data generation methods that were used described? Did they lead to sufficient data having been collected?
%					\item How long did you spend in the field? Do you think this was long enough?
%					\item Do you describe your approach holistic, semiotic or critical?
%					\item Is the phenomenology a standalone description, or is it linked to theory, other phenomenology or issues in your culture?
%					\item Have you included an account of you as researcher?
%					\item Have you presented an phenomenological construction rather than a literal description?
%					\item What limitations in the phenomenology have you recognised?
%					\item Which other flaws and/or omissions in your reporting of the phenomenology do you admit?
%					\item Overall, how effectively has your phenomenological strategy been?
%					\end{itemize}
					and \parencite[29m \textit{ff}]{office2020the-phenomenological}
					%
						\begin{itemize}
						\item have you described the \enquote{what, when, where, and how} of the study? What has be done? When the steps were sequenced? Where each step happened? How each step happened?
						\item have you described where will data be collected? Who collected the data? How often and how much data was collected? How long it took to collect the data? How the data was recorded? (ex: transcriptions, video recordings, audio recordings) Were there follow ups to interviews?
						\end{itemize}
						%
				

\subsubsection{Is this strategy right for me?}

%
\begin{itemize}
\item is your audience expecting scientific rigour? If so, choose ???
\item does your data (source) allow analysis above the data, or will the outputs be mostly descriptive?
\item are you short of time for data collection? If so, consider ???
\item do you appreciate the value of deep philosophical discourse?
\item ...
\end{itemize}

\RSActivity{phenomenology}{merleau1956phenomenology,anderson1991qualitative,smith2018phenomenology,shudak2018phenomenology,academic-educational-materials2019understanding,office2020the-phenomenological,groenewald2004a-phenomenological,hycner1985some}


\subsection{Simulation}

The simulation research strategy builds an explicative mechanism to imitate the behaviour of a real-world artefact.

\subsubsection{Knowledge contribution}

\textcquote{dooley2017simulation}{[S]imulation helps answer the question \enquote{What if?}} 
%\subsubsection{Focus}
%
\subsubsection{Variants}

The three main schools of simulation practice are \parencite{dooley2017simulation}:
		%
	\begin{itemize}
	\item Discrete event simulation, which involves modeling the organizational system as a set of entities evolving over time according to the availability of resources and the triggering of events.
	\item System dynamics, which involves identifying the key “state” variables that define the behavior of the system, and then relating those variables to one another through coupled, differential equations.
	\item Agent-based simulation, which involves agents that attempt to maximize their fitness (utility) functions by interacting with other agents and resources; agent behavior is determined by embedded schema which are both interpretive and action-oriented in nature.
	\end{itemize}
	%

\subsubsection{Data Generation}

\textcquote{dooley2017simulation}{Simulation enables studies of more complex systems because it creates observations by \enquote{moving forward} into the future, whereas other research methods attempt to look backwards across history to determine what happened, and how.}

Depending on the context in which the simulation is studied or used. May include:
	%
	\begin{itemize}
	\item observations, interviews, questionnaires, documents, in-depth description when supporting social objectives
	\item accuracy, predicative capability, when used to model real-world processes – think weather forecasting
	\item ...
	\end{itemize}
	%
				
\subsubsection{Evaluation}

For computer simulations:

\begin{itemize}
\item Good development: has a documented set of requirements been maintained? Has a change control process been implemented?Is there a corresponding document (or version) control process.

\item Has the architecture been documented? What is its relationship to the model?

\item Have a variety of testing methods, including code walk-throughs, scenario testing, and user testing been used to establish code quality?

\item Was a project plan for coding and testing developed?

\item how close is the simulation's behaviour to the \enquote{real} answer? Do the results make sense?

\item has the simulation been compared to any extant quantitative behaviour available? Does it match exactly, distributionally (a variable of interest has statistically similar characteristics), or pattern-wise (variables are generally related to one another in a valid manner, but perhaps differ from reality)?

\item Which experimental set-up was used? Was it appropriate?

\item Have observations from analysis been noted, and results discussed in order to sense-make? Has over-interpretation of the results been avoided so that retrofitting to theories is avoided?
\end{itemize}

For simulation:

%
\begin{itemize}
\item ...
\end{itemize}
%

\subsubsection{Is this strategy right for me?}

Do I already have the computing/mathematical/statistical skills that I need to underpin the research?

Do I have access to documentation sufficient to allow a simulation to be built?

\RSActivity{simulation}{dooley2017simulation}

\subsection{Mathematical and logical proof}

\textcquote[3.1.9]{johannesson2014research}{A proof is a rigorous deductive argument that demonstrates the truth of a certain proposition.}

\subsubsection{Knowledge contribution}

Establishes the absolute truth status of a proposition.

%\subsubsection{Focus}
%
%\subsubsection{Variants}

\subsubsection{Data Generation}

No data generation, although iteration and extensive exploration through examples may be needed to identify the extent to which proof is possible.

\subsubsection{Evaluation}

Formal evaluation by the community of mathematicians within some logical system. Potential for automated checking should computational tools exist.

\subsubsection{Is this strategy right for me?}

Do I have a formal background in logic? One way to determine whether my background is suitable would be to read the first few pages of \textcite{lakatos2015proofs}\footnote{Up to page 9  is available through google books} and consider my level of engagement with the process of proof.

Do I have a community of mathematicians that would be willing to check my proof as it develops?

Is the situation to which I want to apply formal proof amenable to formalisation? If not, then the notion of formal proof might not apply. Even if a formalisation is possible, does it already exist or would I also need to formalise the area first. If you have difficulty answering this question, it may be that this research strategy is not for you. 

Am I trying to predict future behaviours of a system? If so, mathematical, statistical or computational modelling might be a better option.

\RSActivity{mathematical and logical proof}{Kleene1964introduction,lakatos2015proofs,antonini2011examples}

\subsection{Mixed methods research}

\textcquote{johnson2007toward}{Mixed methods research is the type of research in which a researcher or team of researchers combines elements of qualitative and quantitative research approaches (e.g., use of qualitative and quantitative viewpoints, data collection, analysis, inference techniques) for the broad purposes of breadth and depth of understanding and corroboration.}

\subsubsection{Knowledge contribution}

The Mixed methods research strategy combined the knowledge contributions of each method used. In addition, \textcquote[p.119, quoting \emph{Huey Chen}]{johnson2007toward}{[the methods] can be adapted, altered, or synthesized to fit the research and cost situations of the study (modified form mixed methods).}

\textcquote[p.119, quoting \emph{Valerie Caracelli}]{johnson2007toward}{[P]lanfully [combining] methods of different types (qualitative and quantitative) to provide a more elaborated understanding of the phenomenon of interest (including its context) and, as well, to gain greater confidence in the conclusions generated by the evaluation study.}

%\subsubsection{Focus}
%
%\subsubsection{Variants}
%
\subsubsection{Data Generation}

Those of the individual methods, combined with those of the methods mixed. The latter form provides for triangulation: \textcquote[p.291]{denzin1978research}{the combination of methodologies in the study of the same phenomenon}: \textcquote[p.3]{webb2000unobtrusive}{If a proposition can survive the onslaught of a series of imperfect measures, with all their irrelevant error, confidence should be placed in it. Of course, this confidence is increased by minimizing error in each instrument and by a reasonable belief in the different and divergent effects of the sources of error.}

\subsubsection{Evaluation}

\subsubsection{Is this strategy right for me?}

Mixed methods research requires competence in more than a single research method, which takes time; it is unlikely that at this stage in your research career you will have a developed understanding sufficient to apply the mixed methods research strategy. However, it may be that your work is part of broader mixed methods research, perhaps led by your supervisor. If that is the case, then refer to the description of the particular method you are being asked to work with.

Even if you are contemplating mixed methods research, perhaps as an extended research agenda leading to a PhD, at this point it may be that a focus on a single method as part of the mixed methods research will achieve what is required. If this is that case, then you should discuss with your supervisor. 

%\subsubsection{More details}
%
%[Add summary of decision process as a diagram]
%
\section{What to do now}

\begin{question}[subtitle={Activity: Mixed methods follow up}]
	Schedule some time with you supervisor to discuss the thought processes and outcomes of choosing a research strategy. 
\end{question}

\begin{figure}[hbtp]
\centering{
  \includegraphics[width=0.7\textwidth]{Figures/researchStrategies}
  \caption{Research strategy choices
  \label{fig:researchStrategies}
  	}  
  }
\end{figure}

%\subsection{Research skills audit}
%
%\begin{question}[subtitle={Activity: Skills audit}]
%Something here
%\end{question}

\subsection{For your chosen research strategy}

\begin{question}[subtitle={Activity: Dissertation structure}]
Whichever tool you've chosen in which to write your dissertation, create chapters entitled \enquote{research strategy}, \enquote{method}, and \enquote{Evaluation}. 

\begin{guidance}
	For the research strategy chapter, make notes from the paper you've read on the general form of the research strategy. 

For the method chapter, add details on the methods that are used in the research strategy. For a complex strategy such as ethnography, you may not use all of them, but you will need to be explicit – when you come to complete it – as to which you have excluded and the reasons for their exclusion. 

For the evaluation chapter, create subsections for each of the questions of your chosen research strategy from the lists above.
\end{guidance}
	
\end{question}
\color{black}
\endinput


\subsection{Structuring research}

Like a recipe, research needs to be structured. 

To a large extent, the structure you use depends on your resources: massive research labs with many hundreds, even thousands, of researchers may need to run many different strands of research at the same time\footnote{According to \url{https://www.nature.com/articles/nature.2015.17567}, there are 5,154 authors on the paper \fullcite{aad2015combined}, in a collaboration between ATLAS and CMS, \blockquote{two massive detectors at the Large Hadron Collider (LHC) at CERN, Europe’s particle-physics lab near Geneva, Switzerland}. \textit{George Aad}, the first author, has the perfect surname for an academic.}, each contributing a small part towards an overall research goal. 

Imagine having to manage that collaboration: 5154 researchers working in parallel.

You aren't likely to have access to such resources. That's good, in a way, because you can keep the research simple and your research can be linear: one step after another. 

Given that you're resource limited means we can plot your research as a single line
%
\newcommand{\taskline}[0]{\ \rule[0.5ex]{0.8cm}{1pt}\ }
\[T\taskline T\taskline\cdots\taskline T\taskline T	
\]
%
each node $T$ being one task, taken from one of the entries in Table~\ref{tab:researchTasks}. The design is simple because it's linear, and so there's not much to think about:
%
\begin{itemize}
\item how long the line should be;
\item what each $T$ will be.
\end{itemize}
%
\todo{Need to talk about relationship between methods, design, tasks, etc}.
More complex research designs, those involving multiple researchers, for instance, will require some amount of sophisticated project management to ensure that the sequencing of parallel research tasks is done correctly.

The table below introduces more than 30 research tasks – the possible $T$s above – and for each gives a brief introduction and a key reference from which you can find out more\todo{Paraphrase descriptions in the table; more to come here.}.

Your literature review may have thrown up papers with an explicit \enquote{methods} section that describes the research design – how you performed the research\footnote{There are examples from the APA here: \url{https://www.scribbr.com/apa-style/methods-section/}.}\todo{Work this in more}. 

The point of a methods section is to report 
\begin{stolen}{https://www.scribbr.com/apa-style/methods-section/}
enough information to understand and replicate your study, including detailed information on the sample, measures, and procedures used.	
\end{stolen}

You may have come across\todo{complete.}.

\subsection{A simple research design: Experimental Research}

As an example of a simple research design, the shortest research design possible is
%
\[R\taskline O\taskline X \taskline O\]
%
which means (from the table\footnote{...where you can read more now, or wait until later in this chapter.}):
%
\begin{itemize}
\item [R] Randomise sample from population, then
\item [O] Observe, then
\item [X] Change experimental variable, then
\item [O] Observe, again.
\end{itemize}
%
This form of research design is called \enquote{Experimental research}\footnote{There are simpler research designs} and is what you might think of as the quintessential scientific research – doing an experiment ($X$) on a random sample ($R$) population, a drug trial, for instance and observing the consequences ($O$).

Even though we have called this a simple research design, it doesn't mean that the results that you can obtain by using it will be simple, it could be that the drug you're testing will make amazing strides in curing some illness, improving the lives of millions of people. What we mean by \emph{simple} is simply that there are few steps in the research. Simple doesn't mean, either, that the work that will need to go onto each step is simple, quick, or trivial. Observation, for instance, is an immensely difficult thing to do correctly; in the worst case, it may take many months of work to get to the point where you \enquote{change the experimental variable} – administer the drug, for instance – and so have something to observe.

The text we recommend for experimental design is \textcite{marczyk2005essentials}\footnote{Page~124 is a good place to start in that edition.}, who speak from a social science background. .\todo{Do we need to write more here about the source?} 

\todo{more here, using the above as a template, and the table below as source...}

\subsection{Another research design: Quasi-experimental Research}

More here from \textcite{marczyk2005essentials}.

\subsection{Another research design: Non-experimental Research}

More here from \textcite{marczyk2005essentials}.

\subsubsection{Single-study or Case-study research}

More here from \textcite{yin2009case}.

\subsection{Design Science Research}

More here from \textcite{oates2007researching}.

\subsubsection{Completing your research design}

Add Validity, BIAS, Reporting.

\newcommand{\midtitle}[2]{\SetCell[c=3]{l,clear=preto}{\textbf{Research Strategy: #1}~\parencite{#2}}\\*%
	Code&Description&Comments\\*%
	}
\SetCiteCommand{\parencite}
\begin{longtblr}[%
	expand=\midtitle,%\midtitle needs to be expanded so that pattern matching from LaTeX3 can work
	caption={Research: tasks, codes and descriptions},%caption is an outer key
	label={tab:researchTasks},%we can add a label
]{%
	width=\tablewidth,
	colspec = {>{\arabic{rownum}–}r|X[2,l]X[7]},%first column right aligned, then 2/7 of remaining width
%	column{1}={preto={\qquad}},%this doesn't seem to work
%	row{1} = {font=\bfseries},%first row is bold, but don't need it because of \midtitle
	measure=vbox,%needed to allow lists, \UseTblrLibrary{varwidth} added above
	}
%	Code&Description&Comments\\
\midtitle{Experimental}{marczyk2005essentials,oates2007researching}
	EXP&Experiment&\textcquote[p.35]{oates2007researching}{\textbf{Experiment}: focuses on investigating cause and effect relationships, testing hypotheses and seeking to prove or disprove a causal link between a factor and an observed outcome. There is 'before' and 'after' measurement, and all factors that might affect the results are carefully excluded from the study, other than the one factor that is thought will cause the 'after' result. (See Chapter 9.)}\\
	 R&	Randomise sample from population& \textcquote[p.124]{marczyk2005essentials}{A true experimental design is one in which study participants are randomly assigned to experimental and control groups. We have discussed randomization in previous chapters, so this chapter will simply highlight the importance of randomization in terms of the strength of a research design. Although randomization is typically described using examples such as rolling dice, flipping a coin, or picking a number out of a hat, most studies now rely on the use of random numbers tables to help them assign their research participants (as discussed in Chapters 2 and 3).}\\
	 O&	Observe phenomenon&\textcquote[p.119]{marczyk2005essentials}{Observation is another versatile approach to data collection. This approach relies on the direct observation of the construct of interest, which is often some type of behavior. In essence, if you can observe it, you can find some way of measuring it. The use of this approach is widespread in a variety of research, educational, and treatment settings.}\\
	 X&	Change experimental variable&\textcquote[p.127]{marczyk2005essentials}{experimental manipulation (independent variable)}\\
	 Y&	Change other variable&\textcquote[p.127]{marczyk2005essentials}{experimental manipulation (other variable)}\\
	\\
\midtitle{Quasi-experimental}{marczyk2005essentials}
	 NR&Non-random sampling&\textcquote[p.138]{marczyk2005essentials}{when randomized designs are not feasible, researchers must often make use of quasi-experimental designs. A good rule of thumb is that researchers should attempt to use the most rigorous research design possible, striving to use a randomized experimental design whenever possible (Campbell, 1969).
	 
	 Cook and Campbell (1979) present a variety of quasi-experimental designs, which can be divided into two main categories: nonequivalent comparison-group designs and interrupted time-series designs. In this section, we will discuss these two major groups of quasi-experimental designs, followed by a brief overview of single-subjects designs.}\\
	 REV&	Before the intervention, then after&\textcquote[p.142]{marczyk2005essentials}{\textbf{Reversal Time-Series Design} Also known as an ABA design (detailed on page 145), the reversal time-series design is basically a multi-subject variation of the single-subject reversal design, which will be discussed later in this chapter. The basic goal of this design is to establish causality by presenting and withdrawing an intervention, or independent variable, one to several times while concurrently measuring change in the dependent variable (as depicted in the following). As in the simple time-series design, this design begins with a series of pretests to observe normal fluctuations in baseline. The name “reversal” refers to the idea that causality can be inferred if changes that occur following the presentation of an intervention diminish or “reverse” when the independent variable is withdrawn.}\\
	 ABA&	Before, Intervention, After&See REV.\\
	 ABABA&	Iterated ABA&See REV.\\
	 ABABA...&Further Iterated ABA	&See REV.\\
	 EC&	Establish control&\textcquote[p.144]{marczyk2005essentials}{As with time-series designs, single-subject designs typically begin by establishing a stable baseline. Establishing a stable baseline involves taking repeated measures of a participant’s behavior (dependent variable) prior to the administration of any intervention to make certain that the participant’s behavior is occurring at a consistent rate. To obtain a stable baseline, the researcher must make special efforts to control all relevant environmental variables that otherwise might affect the participant’s responses. If the researcher does not know, or is uncertain, about which variables are relevant, the researcher must attempt to keep the participant’s environment as constant as possible by maintaining highly controlled conditions.}\\
	 1P&	Single participant&\textcquote[p.144]{marczyk2005essentials}{Not to be confused with non-experimental single-subject case studies, which are covered later in this chapter, the single-subject experimental design has a long and respected tradition in empirical research. According to Kazdin (2003c), single-subject experiments might be seen as true experiments because they “can demonstrate causal relationships and can rule out or make implausible threats to validity with the same elegance of group research” (p. 273). Similar to other experimental designs, the single subject design seeks to (1) establish that changes in the dependent variable occur following introduction of the independent variable (temporal precedence) and (2) identify differences between study conditions.
	 
	 The one way that single-subject designs differ from other experimental designs is in how they establish control, and thereby demonstrate that changes in a dependent variable are not due to extraneous variables. For example, experimental designs rely on randomization to equally distribute extraneous variables and on statistical techniques to control for such factors if they are found. Alternatively, single-subject designs eliminate between-subject variables by using only one participant, and they control for relevant environmental factors by establishing a stable baseline of the dependent variable. If change occurs following the introduction of the intervention, or independent variable, the researcher can reasonably assume that the change was due to the intervention and not to extraneous factors.}\\
	 SB&	Stable Baseline&See 1P\\
	 RC&	Retain control of Env&See 1P\\
	 \\
\midtitle{Non-experimental}{yin2009case,oates2007researching}
	CASE&Case Study&\textcquote[p.35]{oates2007researching}{\textbf{Case study}: focuses on one instance of the 'thing' that is to be investigated: an organization, a department, an information system, a discussion forum, a systems developer, a development project, a decision and so on. The aim is to obtain a rich, detailed insight into the 'life' of that case and its complex relationships and processes. (See Chapter 10.)}\\
	AR&Action research&\textcquote[p.35]{oates2007researching}{\textbf{Action research}: focuses on research into action. The researchers plan to do something in a real-world situation, do it, and then reflect on what happened or was learnt, and then begin another cycle of plan-act-reflect. (See Chapter 11.)}\\
	ETH&Ethanography&\textcquote[p.35]{oates2007researching}{Ethnography: focuses on understanding the culture and ways of seeing of a particular group of people. The researcher spends time in the field, taking part in the life of the people there, rather than being a detached observer. (See Chapter 12.)}\\
	CS& 	Choose subject&\textcquote[p.144]{marczyk2005essentials}{single-subject designs eliminate between-subject variables by using only one participant, and they control for relevant environmental factors by establishing a stable baseline of the dependent variable. If change occurs following the introduction of the intervention, or independent variable, the researcher can reasonably assume that the change was due to the intervention and not to extraneous factors.
	
	As with time-series designs, single-subject designs typically begin by establishing a stable baseline. Establishing a stable baseline involves taking repeated measures of a participant’s behavior (dependent variable) prior to the administration of any intervention to make certain that the participant’s behavior is occurring at a consistent rate. To obtain a stable baseline, the researcher must make special efforts to control all relevant environmental variables that otherwise might affect the participant’s responses. If the researcher does not know, or is uncertain, about which variables are relevant, the researcher must attempt to keep the participant’s environment as constant as possible by maintaining highly controlled conditions.}\\
	COMP&	Comprehensive description&\textcquote[p.148]{marczyk2005essentials}{the focus of the case-study approach is on individuality and describing the individual as comprehensively as possible. The case study requires a considerable amount of information, and therefore conclusions are based on a much more detailed and comprehensive set of information than is typically collected by experimental and quasi-experimental studies.}\\
	IDIP&	In-depth interviews with participants&\textcquote[p.148]{marczyk2005essentials}{Case studies of individual participants often include in-depth interviews with participants ...}\\
	IDIC&	In-depth interviews with collaterals&\textcquote[p.148]{marczyk2005essentials}{...and collaterals (e.g., friends, family members, colleagues), review of medical records, observation, and excerpts from participants’ personal writings and diaries}\\
	SUR& Surveys&\textcquote[p.33]{oates2007researching}{\textbf{Survey}: focuses on obtaining the same kinds of data from a large group of people (or events), in a standardized and systematic way. You then look for patterns in the data using statistics so that you can generalize to a larger population than the group you targeted. (See Chapter 7.)}\\
	RA&	Review of artefacts&\textcquote[p.148]{marczyk2005essentials}{According to Kazdin (1982), the major characteristics of case studies are the following:
	\begin{itemize}
		\item They involve the intensive study of an individual, family, group, institution, or other level that can be conceived of as a single unit.
		\item The information is highly detailed, comprehensive, and typically reported in narrative form as opposed to the quantified scores on a dependent measure.
		\item They attempt to convey the nuances of the case, including specific contexts, extraneous influences, and special idiosyncratic details.
		\item The information they examine may be retrospective or archival.
	\end{itemize}}\\
%	(RQ)&	Research question??\\
	PROPS&	Identify propositions&\textcquote[p.28]{yin2009case}{\textbf{Study propositions} [...] each proposition directs attention to something that should be examined within the scope of study.}\\
	UNITS&	Identify units&\textcquote[p.29]{yin2009case}{\textbf{Unit of analysis} [...] related to the fundamental problem of defining what the \enquote{case} is [... what the primary unit of analysis is].
	
Without such questions and propositions, you might be tempted to cover \enquote{everything} about the individual(s), which is impossible to do.}\\
	LINKS&	Identify how is data linked to propositions&\textcquote[p.34ff]{yin2009case}{be aware of the main choices and how they might suit your case study]}\\
	CRITS&Which are criteria to interpret findings&\textcquote[p.34]{yin2009case}{Criteria for interpreting a study's findings}\\
	THD&Theory Development&\textcquote[p.35]{yin2009case}{[including types on p.37]}\\
	GEN&Generalisation&\textcquote[p.38]{yin2009case}{[including fig 2.2]}\\
	NAR&Narrative&\textcquote[p.121]{yin2009case}{Certain types of narrative, produced by a case study investigator upon completion of all data collection, also may be considered a formal part of the database and not part of the final case study report. The narrative reflects a special practice that should be used more frequently: to have case study investigators compose open-ended answers to the questions in the case study protocol. This practice has been used on several occasions in multiple-case studies designed by the author (see BOX 24). 
	
	[Box 24]
	
		In such a situation, each answer represents your attempt to integrate the available evidence and to converge upon the facts of the matter or their tentative interpretation. The process is actually an analytic one and is the start of the case study analysis. }\\
	{NSC\\NEI\\NID}&Nuance from the specific context/extraneous influences/idiosyncratic details&\textcquote{kazdin1982single}{According to Kazdin(1982), the major characteristics of case studies are the following:
		\begin{itemize}
			\item  They involve the intensive study of an individual, family, group, institution, or other level that can be conceived of as a single unit.
			\item The information is highly detailed, comprehensive, and typically reported in narrative form as opposed to the quantified scores on a dependent measure.
			\item They attempt to convey the nuances of the case, including specific contexts, extraneous influences, and special idiosyncratic details.
			\item The information they examine may be retrospective or archival.
		\end{itemize}}\\ 
%	NEI&Nuance from extraneous influences\\ 
%	NID&Nuance from idiosyncratic details\\
	\\
\midtitle{Design Science Research}{oates2007researching}
	D\&C&Design and creation&\textcquote{oates2007researching}{\textbf{Design and creation}: focuses on developing new IT products, or artefacts. Often the new IT product is a computer-based system, but it can also be some element of the development process such as a new construct, model or method. (See Chapter 8.)}\\
%	PSA&Problem solving awareness&\textcquote[p.111]{oates2007researching}{Awareness is the recognition and articulation of a problem, which can come from studying the literature where authors identify areas for further research, or reading, about new findings in another discipline, or from practitioners or clients expressing the need for something, or from field research or from new developments in technology.}\\
%	PSS&Problem solving suggestion&\textcquote[p.112]{oates2007researching}{Suggestion involves a creative leap from curiosity about the problem to offering a very tentative idea of how the problem might be addressed}\\
%	PSD&Problem solving development&\textcquote[p.112]{oates2007researching}{Development is where the tentative design idea is implemented. How this is done depends on the kind of IT artefact being proposed. For example, an algorithm might need the construction of a formal proof. A new user interface embodying novel theories about human cognition will require software development. A systems development method will need to be captured in a manual that can then be followed in a systems development project. A new approach in digital art might require the development of an art portfolio tracing the development of the artist's creative ideas.}\\
%	PSE&Problem solving evaluation&\textcquote[p.112]{oates2007researching}{Evaluation examines the developed artefact and looks for an assessment of its worth and deviations from expectations.}\\
%	PSC&Problem solving conclusion&\textcquote[p.112]{oates2007researching}{Conclusion is where the results from the design process are consolidated and written up, and the knowledge gained is identified, together with any loose ends - unexpected or anomalous results that cannot yet be explained and could be the subject of further research.}\\
	PRU&Problem Understanding&\textcquote[p.49]{hall2017a-design}{gaining an understanding of the real-world environment in which the problem is located, and of the problem owner’s identified need}\\
	PRV&Problem Validation&\textcquote[p.49]{hall2017a-design}{agreeing with the problem owner that the problem is representative, a form of validation}\\
	SOU&Solution Understanding&\textcquote[p.49]{hall2017a-design}{producing the solution}\\
	SOV&Solution Validation&\textcquote[p.49]{hall2017a-design}{convincing the problem owner that the solution meets the agreed recognised need in the agreed real-world environment to their satisfaction, another form of validation}\\
\midtitle{General}{}
	LITREV& Literature Review&\textcquote[p.33]{oates2007researching}{literature review in figure 3.1}\\
	VALID&	Threats to validity&\textcquote[p.40]{yin2009case}{fours (general) tests for validity}\\
	BIAS&	Reflection on bias&\textcquote[p.72]{yin2009case}{[Avoiding bias for case studies]}\\ 
	REP&	Reporting&Something here\\
	TRI&	Triangulation&\textcquote[p.37]{oates2007researching}{The use of more than one data generation method to corroborate findings and enhance their validity is called method triangulation. Many types of triangulation are possible in a research project:
	%
\begin{itemize}
\item Method triangulation: the study uses two or more data generation methods.
\item Strategy triangulation: the study uses two or more research strategies.
\item Time triangulation: the study takes place at two or more different points in time.
\item Space triangulation: the study takes place in two or more different countries or cultures to overcome the parochialism of a study based in just one country or culture.
\item Investigator triangulation: the study is carried out by two or more researchers who then compare their accounts.
\item Theoretical triangulation: the study draws on two or more theories rather than one theoretical perspective only.
\end{itemize}
%


Researching Information Systems and Computing
Triangulation gives researchers multiple modes of \enquote{attack} on their research question.
However, researchers differ over whether they should expect triangulation of method or time or space, and so on, to lead to consistency of findings. It depends on their underlying research philosophy (see Chapters 19 and 20 for a detailed explanation). \enquote{Positivists} subscribe to the idea of a single \enquote{truth} or \enquote{reality} and would expect the multiple lines of attack to lead to a consistent set of findings. Interpretivists', on the other hand, do not subscribe to the idea of a single reality, believing any notion of 'reality' to be constructed by individuals and groups, so there are multiple realities for people in our world, and different research approaches are likely to lead to different findings. Interview data about recollections of a meeting and company minutes of the same meeting, for example, are two different \enquote{stories}, created by different people for different audiences. Interpretivists would not always expect to see convergence in the data they generate using triangulation.}\\
\midtitle{Data Generation}{oates2007researching}
	INT&Interview&\textcquote[p.36]{oates2007researching}{\textbf{Interview}: a particular kind of conversation between people where, at least at the beginning of the interview if not all the way through, the researcher controls both the agenda and the proceedings and will ask most of the questions. One-to-one and group interviews are possible. (See Chapter 13.)}\\
	OBS&Observation&\textcquote[p.36]{oates2007researching}{\textbf{Observations}: watching and paying attention to what people actually do, rather than what they report they do. Often involves looking, but it can involve the other senses too: hearing, smelling, touching and tasting. (See Chapter 14.)}\\
	QUES&Questionnaire&\textcquote[p.36]{oates2007researching}{\textbf{Questionnaire}: a pre-defined set of questions assembled in a pre-determined order. Respondents are asked to answer the questions, often via multiple choice options, thus providing the researcher with data that can be analysed and interpreted. (See Chapter 15.)}\\
	DOC&Documents&\textcquote[p.36]{oates2007researching}{\textbf{Documents}: documents that already exist prior to the research (for example, policy documents, minutes of meetings and job descriptions) and documents that are made solely for the purposes of the research task (for example, a researcher's logbook or design models). Also includes \enquote{multimedia documents}: visual data sources (for example, photographs, diagrams, videos and animations), aural sources (for example, sounds and music) and electronic sources (for example, websites, computer games and electronic bulletin boards). (See Taspic
	Chapter 16.)}\\
	EVAL&Evaluate&\textcquote[p.40]{oates2007researching}{\textbf{Evaluating the Research Process}
		Now that you know something of the research process, you can start to analyse and evaluate how well other researchers have described their process. Use the Evaluation Guide' below to help you.

EVALUATION GUIDE: RESEARCH PROCESS
%
\begin{enumerate}%[start=0,label={(\bfseries R\arabic*):}]
\item Do the researchers make clear their research question(s)?
\item Do the researchers explain the theory (ies) they use to conceptualize the research topic?
\item Do the researchers make clear both their strategy and the data generation method(s) within that strategy?
\item Do the researchers indicate their criteria for judging the success or usefulness of their work?
\item Is there a clear process summarized, from the original motivation and literature review through to final outcome(s)? If not, how does that affect your confidence in the research and its reporting?
\end{enumerate}}\\
\end{longtblr}

\subsection{Alternative models of the research process}

See \parencite[p.39]{oates2007researching} for two more models for the research process (in the context of Information Systems and Computing)
%
\begin{itemize}
\item Conceptualise, operationalise, generalise;
\item The SLDC (Software Development Life Cycle) Analogy.
\end{itemize}
%



%As a running example, we'll be working with the following research objectives, which you saw in Example~\ref{ex:machinelearning} in Stage 1\footnote{See page~\pageref{ex:machinelearning}.}\todo{What was the research problem?}:

\subsection{Decomposing objectives into tasks}

You've chosen your research design based on area, and you've got your research objectives from Section~\ref{sect:???}. How do you go about mapping one into the other?

Our suggested template for creating objectives had three components: identify, assess, and recommend.
%
%
\begin{description}
\item [identify:] literature review; questionnaire, interviews; problem solving awareness, ...
\item [assess:]	what goes here? problem solving suggestions; interviews; problem solving development; ...
\item [recommend:] what goes here? problem solving evaluation; problem solving conclusion; validity; bias; ...
\end{description}
%

\todo{Turn this example into identifying which research design.}

\begin{example}{Recap: Applying Machine Learning}
In Stage 2, we refined Clara's research aim, which was:
%
\blockquote{to apply Machine Learning (ML) to improve the accuracy of resources and time forecasting in the context of small engineering plants}
%
to three following three objectives:\todo[inline]{JGH: needs doing if not already done}

\begin{description}%[start=0,label={(\bfseries R\arabic*):}]
\item [Objective 1] to identify which ML techniques are applicable to resource and time forecasting in the context of small engineering plants, which will allow us to identify specific ML techniques to be used in the project, to ensure the work is feasible within the time-frame of the project. 

\item [Objective 2] to test the accuracy of forecasting of those techniques which will allow us to investigate and compare how accurate the chosen techniques are in their forecasting application. 

\item [Objective 3] to provide recommendations as to how integrate those techniques effectively in engineering practice in order to improve forecasting accuracy which will allows us to draw some conclusions from the research conducted and make recommendations to improve professional practice.
\end{description}

Note how those objectives were designed to build on each other and, when successfully completed, they'd contribute to meet the overall aim.
\end{example}

\begin{example}{Example – cont'd}In our example, the first objective is met once we have identified the relevant ML techniques. There are two complementary ways to do this: to look at the literature and to ask practitioners. As a result, we could break this objective down into the tasks, and deliverables, indicated in the following table

\begin{longtable}{@{}p{0.1\textwidth}@{}p{0.9\textwidth}@{}}
\caption{Objective 1: to identify which ML techniques...}\\
\toprule
\textbf{Task} & \textbf{Deliverable} \\\midrule
\tabletitle{to identify relevant ML techniques in the academic literature} & a collection of relevant ML techniques reported in the literature \\\\
\tabletitle{to ask practitioners which techniques they employ} & a collection of relevant ML techniques used in professional practice \\
\bottomrule
\end{longtable}
\end{example}

\begin{question}[subtitle={ACTIVITY: Establishing tasks and deliverables}] Consider your research objectives. For each, identify related tasks and deliverables.\todo[inline]{I don't think I could do this at this point.}

\begin{guidance}You should draw a table similar to that in our running example. You should ensure that the tasks provide a reasonable break down of your objectives into discrete pieces of work.
\end{guidance}\end{question}
%%Hack to correct tcbox behaviour
\color{black}

Your tasks and deliverables don't need to be perfect in stage 3 – there are two more stages to perfect them after all – and are likely to be revised as you progress through your project. However, it is important that you have thought about specific work you will need to carry out to meet your objectives.

\subsubsection{Relating tasks to research methods}
The way to carry out your tasks and meet your objectives is through the application of research methods.

\begin{example}{EXAMPLE - cont'd }Following on from our previous example, we have extended the table to include an indication and justification of candidate research methods for each task.

\begin{tblr}{colspec={XXXX},
row{1}={font=\bfseries},
}
%\caption{\textbf{Objective 1: to identify which ML techniques...}
%\hline[1pt]
Task&Deliverable&Relevant research methods&Justification and feasibility\\
%\hline[0.5pt]
\SetCell[c=3]{l}{to investigate the academic literature in order to identify relevant ML techniques}\\ 
&a collection of relevant ML techniques reported in the literature & review of existing literature & I can access relevant literature via my university library\\
\SetCell[c=3]{l}{to ask practitioners which techniques they employ}\\ 
& a collection of relevant ML techniques used in professional practice & questionnaire, possibly followed by interviews & I have access to professional networks, which I could use to distribute the questionnaire, and possibly to recruit participants for follow-up interviews \\
%\hline[1pt]
\end{tblr}
\end{example}

Note that the choice of research methods in relation to your research tasks is an essential part of your research design. In fact, the two influence each other: your objectives and related tasks direct you towards specific research methods, which in turn have to be part of your overall research design.

\begin{question}[subtitle={ACTIVITY: Associating methods to tasks and deliverables}] Extend your tasks and deliverables table with your candidate research methods, including stating why they apply and are feasible for your project. Revise your research design draft from Stage 2 so that is consistent with those choices.

\begin{guidance}
Refresh your understanding of chosen research methods from the study work you carried out in Stage 2. It is important you keep reviewing your choices with your supervisor.
\end{guidance}
\end{question}
%%Hack to correct tcbox behaviour
\color{black}

\subsubsection{Research task deliverables}

\todo[inline]{Add something here}

\section{Research procedures}
\todo{What's the relationship to objectives and tasks?}

Once you have chosen the set of research methods you will apply, you must establish exactly how you will do that, something we refer to as \textbf{research procedures}.

Your research procedures will be method specific, in that each method you choose to apply will come with recommended practices, which you will need to contextualise to your own project needs, including your access to participants, data or other kind of evidence. For instance, there are plenty of guidelines in the literature on how to design questionnaires, including which type of questions to include and how to phrase them. There are also recommendations concerning testing the questionnaire design prior to its use, and of course, there are many ways a questionnaire can be administered. In writing your procedures for this research method, you would have to be specific on how each of the above applies in your project.

It is important, therefore, that you master the research methods of your choice, starting by reviewing once again the related academic literature.

\begin{question}[subtitle={Activity: Sketching research procedures}] Consider the research methods you intend to apply, and the related review you conducted in Stage 2. Reconsider those materials, possibly going back to the literature sources, to learn how to apply the methods effectively within your project.

For each method, sketch possible procedures of application, ensuring you make appropriate reference to the literature you have reviewed and best practice guidelines therein.

\begin{guidance}It is possible that the review you conducted in Stage 2 is not sufficient, in which case you will need to extend it to complete this activity.

You should focus on practical aspects of applying the methods, including specific processes and techniques to gather, summarise and present your evidence in your reports.

Depending on the extent you need to review further academic literature, this activity could be quite substantial, so you should set aside up to 20\% of your study time to complete it.

\end{guidance}\end{question}
%%Hack to correct tcbox behaviour
\color{black}

\section{Assessing validity}
As your intended research design becomes clearer, you will soon be testing some aspects of it in your pilot work. Before you do that, however, you need to consider if the choices you have made will allow you to gather evidence and derive findings in a systematic, rigorous, repeatable and reliable fashion so as to address your research problem. This is referred to as assessing the overall \textbf{validity} of your research design, which is broken down into the following considerations.

\textbf{Construct validity} asks whether you have put your design together logically by focusing on the relationship between evidence and research problem. Here you ask yourself whether the evidence you will generate through your chosen research design will be accurate and relevant to address your research problem. This tests the logical coherence of your aim, objectives, tasks, methods and deliverables in relation to the research problem and the knowledge gap you intend to address. With construct validity, you are asking: \emph{have I designed my research in the right way?}

\textbf{Internal validity} is concerned with the way you gather and analyse evidence. All research strategies and methods come with recommendations of good practice to ensure that your research is both systematic, repeatable and reliable. In your work, you need to ensure that you follow such practices and are aware of possible pitfalls. For instance, in experimental research you need to control all factors which may effect outcomes beyond those under study: failing to exercise such control will lead to observations and measurements which are unreliable. In assessing internal validity, you should also take into account limitations of human perception and cognition, and any potential personal bias. With internal validity, you are asking: \emph{have I executed my research in the right way?}

\textbf{External validity} relates to the extent you will be able to generalise your findings beyond the immediate context of your research. For instance, you may conduct a case study within a specific organisation, so here you are asking whether and how what you have discovered may apply to other organisations. With external validity, you are asking: \emph{will my research lead to findings that apply somewhere else?}

Anything that gets in the way of validity in research is termed a \textbf{threat to validity}. Different research strategies and methods are exposed to different threats, something you should have encountered in your review of the literature on your chosen methods.

\begin{question}[subtitle={Activity: Assessing validity of research design}] Conduct an initial assessment of your chosen research design in relation to the three kind of validity discussed above. Write down a short summary of your thinking in support of each, and of possible threats to validity you envisage.

\begin{guidance}You may need to refer back to the literature you have reviewed to identify specific threats which apply to your chosen research methods and strategies.

You won't be able to address this in full at this point in your project, particularly the internal validity, which refers to the execution of your research design. Nevertheless, it is important for you to consider validity and possible threats from the onset. You will return to this topic at the end of your project, as part of the overall assessment of your research, to reflect on the validity of your completed research.

\end{guidance}\end{question}
%%Hack to correct tcbox behaviour
\color{black}

\section{Conducting your pilot work}

Your \textbf{pilot work} will be a small scale test of some of the methods and procedures you will apply in the next stage of your project. Its main function is to help you assess the feasibility of your research design, or at least some aspects of it, and build your confidence in the approach you have chosen.

As such, your pilot work may not contribute directly to your aim and objectives, but it should help you decide whether you can actually do what you have planned to do, or inform how your research design and project plan should change instead.

There are no constraints on what you can do for your pilot work, other than you should exercise some aspect of your research design. It is therefore essential that you agree what you are going to do with your supervisor first.

\begin{question}[subtitle={Activity: Planning and executing your pilot work}] Plan your pilot work and discuss your plan with your supervisor.

Once you have agreed the way forward, execute your plan and write a summary of both its execution and outcomes.

\begin{guidance}
This is a substantial activity, which will take you up to 35\% of your study time.

Your summary should include:
%
\begin{itemize}
\item an indication of which aspects of your research design your pilot work was concerned with
\item any methods and procedures applied
\item any data or evidence gathered, including possible modelling, artefact design or prototyping, appropriately presented and summarised
\item lessons learnt and any resulting revision to your research design and project plan, particularly in relation to construct and internal validity of your research design.
\end{itemize}
%
To complete this activity successfully, it is essential that you agree your pilot work plan with your supervisor upfront, and discuss your progress on a regular basis.
\end{guidance}\end{question}
%%Hack to correct tcbox behaviour
\color{black}

\section{Reporting in Stage 3}
At the end of Stage 3, you should complete a report, extending that of Stage 2 and covering the work you have carried on in this stage. The structure we suggest and an indication of the contents are shown in Table~\ref{tab:reportStructure}.

%%Report Structure Table is repeated throughout the thesis. This is the template
%%Format is:
%%\begin{ReportStructureTable}
%%	\tabletitle{Section 1: Introduction}\\
%%	\begin{enumerate}[label={1.\arabic*:}]
%%	\item Background to the research 
%%	\item Justification for the research 
%%	\end{enumerate}
%%	& This section should provide an introduction to your research topic in its wider context (as background) and your justification of why the research is worth pursuing. It should be well articulated and supported by evidence \\
%%\end{ReportStructureTable}
%%Still to do: remove space from above enumerate environment
%%Sets the chapter across two columns in bold
\begin{ReportStructureTable}{tab:reportStructure}
\tabletitle{Title} & Your title should succinctly capture your research problem and aim\\\\
\tabletitle{Section 1: Introduction}\\
\begin{enumerate}[label={1.\arabic*:}]
\item Background to the research 
\item Justification for the research 
\end{enumerate}
& This section should provide an introduction to your research topic in its wider context (as background) and your justification of why the research is worth pursuing. It should be well articulated and supported by evidence \\\\
\tabletitle{Section 2: Literature review}\\
\begin{enumerate}[label={2.\arabic*:}]
\item Review of existing relevant knowledge 
\item Critical summary, including knowledge gap to be addressed by the research 
\end{enumerate}
& Your review should provide a critical account of your in-depth engagement with the academic (and other) relevant literature, including identifying key trends, ideas and possible knowledge gaps. Most of your citations should point to academic articles. Your critical summary should highlight key insights from your review and provide a strong justification for your proposed research. Both coverage and depth of your review matter. You should ensure that your review is well structured, with a logical narrative flow and your arguments are well supported by evidence  \\\\
\tabletitle{Section 3: Research definition}\\
\begin{enumerate}[label={3.\arabic*:}]
\item Problem statement 
\item Aim, objectives, tasks and deliverables
\item Knowledge contribution
\end{enumerate}
& You should ensure that your research problem is well articulated and appropriate for your course and your personal and professional circumstances, that your aim and objectives are consistent with research problem, that tasks and deliverables break down your objectives appropriately and are clearly related to your chosen research methods, and that the intended knowledge contribution of your research is clearly articulated \\
\tabletitle{Section 4: Research design}\\\\
\begin{enumerate}[label={4.\arabic*:}]
\item Evidence and data 
\item Research strategy and methods
\item Research procedures
\item Ethical, legal and EDI considerations
\end{enumerate}
& This section should demonstrated your critical engagement with all elements of research design, including a detailed account of the data and evidence needed in your research, the research methods and research strategies you will to apply, and how you will apply them within your project. Your account should be supported by a clear rationale and insights from the related literature, and appropriately justified in relation to your research problem, aim and objectives. It should also demonstrate your careful consideration of ethical and legal matters, and that your research will comply with your course and university requirements\\\\
\tabletitle{Section 5: Analysis and interpretation}\\
\begin{enumerate}[label={5.\arabic*:}]
\item Pilot work
\end{enumerate}
& This section should report on a well thought-out pilot work which clearly and competently test some significant aspect of your research design. It should demonstrate good critical reflection on outcomes and highlight any adjustments needed as a result. \\\\
\tabletitle{Section 6: Assessment of your proposed research}\\
\begin{enumerate}[label={6.\arabic*:}]
\item Qualification fit
\item Personal and professional fit
\item Technical skills and resources needed
\item Statement of feasibility
\item Personal reflection on research process
\end{enumerate}
& In this section you should continue to argue how your research is a good fit across all criteria. You should provide a clear rationale as to why you think what you are proposing is feasible. You should also reflect on your growing understanding of the research process, including key learning and aspects you have found particularly challenging. \\\\
\tabletitle{Section 7: Planning, scheduling and risk assessment}\\
\begin{enumerate}[label={7.\arabic*:}]
\item Key priorities in follow-up stage
\item Personal and professional fit
\item Risk assessment
\end{enumerate}
& In this section you should reflect on the progress you have made in Stage 2 and establish your priorities for the next stage. You should also review your risk assessment as appropriate.\\\\
\tabletitle{Section 8: References}\\ & You should keep your growing references in good order and ensure you apply the required bibliographical style consistently. Ideally, you should use a BMT to generate and integrate your references within your report\\\\
\textbf{Appendix A: Work schedule}& Your revised work plan\\\\
\textbf{Appendix B: Risk assessment table}& Your revised risk table \\
\bottomrule
\end{ReportStructureTable}

\endinput

\begin{question}[subtitle={Activity: Putting your report together}] Using your word processor of choice, and starting from your previous report, complete your Stage 3 report by applying the structure and guidance in Table~\ref{tab:ReviewCrit}, and making good use of your notes and summaries from all related activities you have carried out so far.

\begin{guidance}In this first pass at putting together your report, you should focus primarily on completeness, ensuring that each section includes at least draft content.
\end{guidance}\end{question}
%%Hack to correct tcbox behaviour
\color{black}

As in the previous stages, after you have filled in your report you should review and revise it iteratively until you are happy with your account, and are ready to move on. 

\begin{table}[htbp]
\begin{minipage}{\linewidth}
\setlength{\tymax}{0.5\linewidth}
\centering
\caption{Criteria to review your report\label{tab:ReviewCrit}}
\small
\begin{tabulary}{\tablewidth}{@{}LL@{}} \toprule
 \textbf{Criteria} & \textbf{Prompts} \\
\midrule

 \tabletitle{Completeness} & Are all sections of the suggested structure completed in line with the guidance provided? \\
 \tabletitle{Good academic writing practices} & Have you applied good academic writing practices throughout? \\
 \tabletitle{Logical structure and flow} & Have you structured your narrative appropriately to ensure a logical flow of arguments? \\
 \tabletitle{Supporting references or evidence} & Are your key arguments supported by appropriate references or other evidence? \\
 \tabletitle{Citation and reference style} & Do all your citations and references comply with the required bibliographical style? \\
 \tabletitle{Avoiding plagiarism} & Have you acknowledged the work of others and distinguished it from your own appropriately? \\
 \tabletitle{Standard of English (or any modern language you use)} & Have you proof-read your report carefully to remove all typos and grammatical errors? \\
\bottomrule

\end{tabulary}
\end{minipage}
\end{table}

\begin{question}[subtitle={Activity: Reviewing your report}] Apply the criteria in Table 1 to review your current report and write up a summary of your assessment.

\begin{guidance}For each criteria, consider the related prompts to help you assess your report overall, and write down any further work needed for your next stage.
\end{guidance}\end{question}
%%Hack to correct tcbox behaviour
\color{black}

\section{Reflection: Stage 3}

%%More here

%%Repeated reflection activity
%%Repeated Activity for all reflections
\begin{question}[subtitle={Activity}]
$<$Needs assessing for content and structuring into activity + guidance$>$

This activity has four parts: the first is something you should be doing regularly, but won't make you into a disobedient or indocile thinker. The second, third and fourth may help you get started and keep going.

Part 1: Think about your study this far -- using this book and anything you've done for your dissertation in parallel -- as a journey. More from elsewhere, including   !!.

Part 2: think about yourself and the way you think. How does your desk look? Is it messy or tidy? Do the same for your computer desktop. Is it empty or are there hundreds of files strewn across it? Do you think your tidiness or untidiness will affect the way you do your research? How about how you keep your -- critically important -- bibliographic database which may contain up to a hundred academic\footnote{It's not unknown to have more than a hundred.} and other articles by the time you're finished?

Part 3: think about the context of your research. Which professional pressures are there on you to succeed in solving your research problem? Pressures could come in many forms: financial -- there's a promotion for you at the end of it; peer -- your colleagues know that you are studying will have good expectations of your result and you'll want to prove them right\footnote{Or wrong, depending on the colleague!}. Are you sponsored by your employer? Will you be able to report a negative outcomes to your research, for instance, that there is no solution to our problem using the current technology stack? A negative result is a very good research outcome, even if it tends to satisfy fewer non-academics than a positive result.

Which family pressures do you feel? It's' not unusual that you will have given up a paying role to study, moving the responsibility to provide onto another member of your family. What are their expectations?

Part 4: what's that thought nagging at the back of your mind? Is it ``How will I start?'' Or ``Will I be able to dedicate enough time to this?'' Or ``Can I really do this?''. Or ''Is ``shouldn't I be bringing in a wage rather than studying?''

You may be one of the lucky ones that doesn't have such negative thoughts, but negative thoughts are a very natural part of steps into the unknown. And research is precisely that, a step into the unknown. At least if you are aware of the doubts you naturally have, you can manage them. Think about making even the tiniest of steps forward in your research visible and celebrated! Work with Kansan boards where progress is encouragingly visible as you move a task from the inbox to the outbox. If you have concerns about managing your time, start using one of the many tools out there that break time up into manageable units and help manage it for you. If your concerns are about how to organise your thoughts, look into mind maps, lists, todo lists.

Thinking early and often through reflection is a powerful way of doing better. Do it well and your final report will be better than you will have expected.

It's worth saying that, at the end of what could be an exhausting journey, you will not fully appreciate your achievements. That realisation may have to wait until you are rested, graduated, or some distant time later.

But it will come.

\begin{guidance}
%Hack to correct tcbox behaviour
\color{black}
Something here
\end{guidance}\end{question}
%%Hack to correct tcbox behaviour
\color{black}

