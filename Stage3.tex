%%Candidate to move to headers
\DeclareExpandableDocumentCommand{\tabletitle}{O{2} m}{\\[1ex]\multicolumn{#1}{@{}p{\textwidth}@{}}{\textbf{#2}}\\}%

\newcommand{\hwidth}[1]{\makebox[0pt][l]{#1}}
\DeclareExpandableDocumentCommand{\tabletitle}{m}{\SetCell[c=2]{\textbf{#1}}&\\}

%%Structures the report table, headings to the left in a list, explanation to the right
%%#1 = caption label
\newenvironment{ReportStructureTable}[1]{%
	\setlist{nosep} %compress any lists
	\begin{longtable}{@{}p{0.3\tablewidth}>{\vspace{-3pt}}p{0.7\tablewidth}@{}}%%JGH \vspace{-3pt} is a hack to align first lines
	\caption{Report structure and content guidance\label{#1}}\\
	\toprule
	\textbf{Suggested structure} & \textbf{Content guidance} \\
	\midrule%
}%
{\end{longtable}}

\chapter{Stage 3: Second research increment}

By now you will have some mastery of the techniques and tools that you need to \emph{do} research at masters level. You may also have ideas about what you still need to do in the next step\footnote{If not, don't worry – we've got you covered with this chapter!}.

With the skills you have so far gained, you're developing into an independent researcher\footnote{Being an independent researcher isn't one of the examined outcomes of masters research, but that you're feeling confident in your research is good.} and you may feel that this book holds nothing more for you. 

Stay with us a little longer though: the next sections aren't as long as those that you've studied already – you'll be doing more yourself, honing the skills you've picked up as you go along – but they might help to keep you systematic and on the path to submission. 

\newcommand{\percentthroughbook}{
%\thepage
%\getpagerefnumber{LastPage}
\intcalcDiv{\intcalcMul{\thepage}{100}}{\getpagerefnumber{LastPage}}\%
}

You won't be surprised to know that stage 3 comes next; there's another research increment coming.

%%Removed for now
%You're now \(\percentthroughbook{}\) through this book – not long to go before you're finished. Stick with us for a while longer.

\section{Introducing stage 3}

\begin{table}[htbp]
\caption{Stage 3 activities\label{tab:stage3acts}}
\centering
\small
\begin{tblr}{colspec={X[10]X[1,r]X[4]},
row{even}={font=\bfseries,},
row{odd}={2-Z}{preto={\quad},},
row{1}={font=\bfseries},
}% \toprule
Research activity
 	& \SetCell[c=1]{c}{Effort within stage} & Suggested supervisor focus\\
%\midrule
%
\SetCell[c=2]{l}{Identifying the research problem}\\
	Adjust, if needed & 2\% & \\
\SetCell[c=2]{l}{Reviewing the literature}\\ 
	Adjust, if needed & 3\% & \\
\SetCell[c=2]{l}{Setting research aim and objectives}\\ 
	Finalise aim and objectives, and define tasks and deliverables & 10\% & Suitability of tasks and deliverables from objectives \\
\SetCell[c=2]{l}{\textbf{Choosing the research design}}\\ Complete research design, with detailed consideration of data and evidence, research strategy, research methods and procedures & 20\% & Suitability of research procedures \\
\SetCell[c=2]{l}{\textbf{Gathering and analysing evidence}}\\ Conduct pilot work to test aspects of your research design & 35\% & Scope of your pilot work \\
\SetCell[c=2]{l}{\textbf{Interpreting and evaluating findings}}\\ n/a & 0\% & \\
\SetCell[c=2]{l}{\textbf{Reporting, critical reflection and conclusions}}\\ Assess research progress and write up Stage 3 report & 25\% & Any further improvements required \\
\SetCell[c=2]{l}{\textbf{Work planning and risk management}}\\ At stage start, review work from previous stage and project risk; adjust plan as needed If you have received feedback from supervisor on your previous stage work, adjust plan to include any revision recommended & 5\% & Any major adjustment required \\
%\bottomrule
\end{tblr}
\end{table}

%\todo{This is a repeated activity through all stages, and so should be templated.}
\begin{question}[subtitle={Activity: Understanding the effort needed in this stage}] Consider Table~\ref{tab:stage3acts} carefully, taking notice of the entries in the `Effort within stage' column. Write down the most time-consuming activities in this stage and what is expected under each.

\begin{solution}Developing your research design further and conducting your pilot work will constitute your major effort in this stage (55\% of the study time in total): your pilot work will be an initial test of some aspects of your research design, including a proof-of-concept application of some of your chosen methods.
\end{solution}\end{question}
%%Hack to correct tcbox behaviour
\color{black}

\section{Your research design}

As you get deeper and deeper into the detail of your research, the steps you'll need to take will become more and more specialised, determined by the area that you are researching in. In all the possible research areas that there are, there could be hundreds of different types of research tasks. So we need to approach breaking down your research objectives into tasks in a systematic way.

First, we'll show you how objectives can be structured into tasks through the running example that you know so well by now: Clara's \textit{Applying Machine Learning} that you first saw in Example~\ref{ex:machinelearning}. 

Then, we'll examine some of the more general research tasks as building blocks, picking out where particular objectives will need more specialist ones.

Finally, we'll talk about the \textbf{deliverables}: these are the tangible things produced by your work. You'll spend a substantial part of your dissertation writing about your research task deliverables as they form the basis of your knowledge contribution and everything else – your evidence/arguments/rationale – will be derived from them.

\subsection{Research \enquote{recipes}}

For a couple of minutes, we want you to kick back, and think about your favourite dishes: perhaps a nice cheesecake, or a succulent stew, perfect pizza, or a bouncy bouillabaisse.

\bigskip

Now, think about the ingredients you'd need to make them: the cream cheese and biscuits for the base; some potatoes, carrots, and blue cheese\footnote{Yum!}; quality olive oil, mozzarella, and tomato; a leek, some fresh fish and garlic.

\bigskip

Your mouth is probably watering at the thought of good food. Good food is made from good ingredients and good ingredients have to be of good quality to make the best dishes. Having a good recipe really makes it all worthwhile.

With good ingredients and the right recipe, you can eat well.

It's the same with research – good research needs the best ingredients: and this section is a walk-through for finding the best ingredients to make your research tasty. With the right ingredients in the right recipe, you will write well and communicate with your readers well. 

There are nigh-on 40 research ingredients – research tasks, they're called – that you might come across in the literature\footnote{More, if you they use alternative names.}, and they need to be combined in the right recipes – research designs – for the tastiest research combinations. There are some – validity and bias, for instance – that – like olive oil and onion – will be a part of pretty much every (savoury) recipe you cook. Others, like \enquote{X} and \enquote{Y} add their own flavour, and will allow you to communicate with research experts in their own language.

\bigskip

\todo{Reused from below.}Recall from Stage 1 that 
%
\begin{itemize}
\item your research aim tells the reader how your research will address the knowledge gap that you have found through your literature review, 
\item your objectives break it down into 3 to 4 high-level steps you must take to achieve the aim.
\end{itemize}
%
You're now at the point where you need to think about how you will meet each of your research objectives. That means breaking them down even further into \textit{research tasks}. Research tasks may not be simple to achieve, but they're last step in your research structure.

\subsection{Research Design}

While a research objective indicates what you need to achieve, \textbf{research tasks} tell you the work that you need to complete to get there, so they address the question of what you need to do to meet the objective. Research tasks are the raw ingredients from which research is built. 

Research designs are the recipes for good research, combining research tasks into meaningful ways of doing research.

In this book, we cover 4 of the most important research designs in science: Experimental, Quasi-experimental, Non-experimental, and Design Science Research.

\subsection{Structuring research}

Like a recipe, research needs to be structured. 

To a large extent, the structure you use depends on your resources: massive research labs with many hundreds, even thousands, of researchers may need to run many different strands of research at the same time\footnote{According to \url{https://www.nature.com/articles/nature.2015.17567}, there are 5,154 authors on the paper \fullcite{aad2015combined}, in a collaboration between ATLAS and CMS, \blockquote{two massive detectors at the Large Hadron Collider (LHC) at CERN, Europe’s particle-physics lab near Geneva, Switzerland}. \textit{George Aad}, the first author, has the perfect surname for an academic.}, each contributing a small part towards an overall research goal. 

Imagine having to manage that collaboration: 5154 researchers working in parallel.

You aren't likely to have access to such resources. That's good, in a way, because you can keep the research simple and your research can be linear: one step after another. 

Given that you're resource limited means we can plot your research as a single line
%
\newcommand{\taskline}[0]{\ \rule[0.5ex]{0.8cm}{1pt}\ }
\[T\taskline T\taskline\cdots\taskline T\taskline T	
\]
%
each node $T$ being one task, taken from one of the entries in Table~\ref{tab:researchTasks}. The design is simple because it's linear, and so there's not much to think about:
%
\begin{itemize}
\item how long the line should be;
\item what each $T$ will be.
\end{itemize}
%
\todo{Need to talk about relationship between methods, design, tasks, etc}.
More complex research designs, those involving multiple researchers, for instance, will require some amount of sophisticated project management to ensure that the sequencing of parallel research tasks is done correctly.

The table below introduces more than 30 research tasks – the possible $T$s above – and for each gives a brief introduction and a key reference from which you can find out more\todo{Paraphrase descriptions in the table; more to come here.}.

Your literature review may have thrown up papers with an explicit \enquote{methods} section that describes the research design – how you performed the research\footnote{There are examples from the APA here: \url{https://www.scribbr.com/apa-style/methods-section/}.}\todo{Work this in more}. 

The point of a methods section is to report 
\begin{stolen}{https://www.scribbr.com/apa-style/methods-section/}
enough information to understand and replicate your study, including detailed information on the sample, measures, and procedures used.	
\end{stolen}

You may have come across\todo{complete.}.

\subsection{A simple research design: Experimental Research}

As an example of a simple research design, the shortest research design possible is
%
\[R\taskline O\taskline X \taskline O\]
%
which means (from the table\footnote{...where you can read more now, or wait until later in this chapter.}):
%
\begin{itemize}
\item [R] Randomise sample from population, then
\item [O] Observe, then
\item [X] Change experimental variable, then
\item [O] Observe, again.
\end{itemize}
%
This form of research design is called \enquote{Experimental research}\footnote{There are simpler research designs} and is what you might think of as the quintessential scientific research – doing an experiment ($X$) on a random sample ($R$) population, a drug trial, for instance and observing the consequences ($O$).

Even though we have called this a simple research design, it doesn't mean that the results that you can obtain by using it will be simple, it could be that the drug you're testing will make amazing strides in curing some illness, improving the lives of millions of people. What we mean by \emph{simple} is simply that there are few steps in the research. Simple doesn't mean, either, that the work that will need to go onto each step is simple, quick, or trivial. Observation, for instance, is an immensely difficult thing to do correctly; in the worst case, it may take many months of work to get to the point where you \enquote{change the experimental variable} – administer the drug, for instance – and so have something to observe.

The text we recommend for experimental design is \textcite{marczyk2005essentials}\footnote{Page~124 is a good place to start in that edition.}, who speak from a social science background. .\todo{Do we need to write more here about the source?} 

\todo{more here, using the above as a template, and the table below as source...}

\subsection{Another research design: Quasi-experimental Research}

More here from \textcite{marczyk2005essentials}.

\subsection{Another research design: Non-experimental Research}

More here from \textcite{marczyk2005essentials}.

\subsubsection{Single-study or Case-study research}

More here from \textcite{yin2009case}.

\subsection{Design Science Research}

More here from \textcite{oates2007researching}.

\subsubsection{Completing your research design}

Add Validity, BIAS, Reporting.

\newcommand{\midtitle}[2]{\SetCell[c=3]{l}{\textbf{Research Strategy: #1}~\parencite{#2}}\\*%
	Code&Description&Comments\\*%
	}
\SetCiteCommand{\parencite}
\begin{longtblr}[%
	expand=\midtitle,%\middditle needs to be expanded so that pattern matching from LaTeX3 can work
	caption={Research: tasks, codes and descriptions},%caption is an outer key
	label={tab:researchTasks},%we can add a label
]{%
	colspec = {r|X[2,l]X[7]},%first column right aligned, then 2/7 of remaining width
%	column{1}={preto={\qquad}},%this doesn't seem to work
%	row{1} = {font=\bfseries},%first row is bold, but don't need it because of \midtitle
	measure=vbox,%needed to allow lists, \UseTblrLibrary{varwidth} added above
	}
%	Code&Description&Comments\\
\midtitle{Experimental}{marczyk2005essentials,oates2007researching}
	EXP&Experiment&\textcquote[p.35]{oates2007researching}{\textbf{Experiment}: focuses on investigating cause and effect relationships, testing hypotheses and seeking to prove or disprove a causal link between a factor and an observed outcome. There is 'before' and 'after' measurement, and all factors that might affect the results are carefully excluded from the study, other than the one factor that is thought will cause the 'after' result. (See Chapter 9.)}\\
	 R&	Randomise sample from population& \textcquote[p.124]{marczyk2005essentials}{A true experimental design is one in which study participants are randomly assigned to experimental and control groups. We have discussed randomization in previous chapters, so this chapter will simply highlight the importance of randomization in terms of the strength of a research design. Although randomization is typically described using examples such as rolling dice, flipping a coin, or picking a number out of a hat, most studies now rely on the use of random numbers tables to help them assign their research participants (as discussed in Chapters 2 and 3).}\\
	 O&	Observe phenomenon&\textcquote[p.119]{marczyk2005essentials}{Observation is another versatile approach to data collection. This approach relies on the direct observation of the construct of interest, which is often some type of behavior. In essence, if you can observe it, you can find some way of measuring it. The use of this approach is widespread in a variety of research, educational, and treatment settings.}\\
	 X&	Change experimental variable&\textcquote[p.127]{marczyk2005essentials}{experimental manipulation (independent variable)}\\
	 Y&	Change other variable&\textcquote[p.127]{marczyk2005essentials}{experimental manipulation (other variable)}\\
	\\
\midtitle{Quasi-experimental}{marczyk2005essentials}
	 NR&Non-random sampling&\textcquote[p.138]{marczyk2005essentials}{when randomized designs are not feasible, researchers must often make use of quasi-experimental designs. A good rule of thumb is that researchers should attempt to use the most rigorous research design possible, striving to use a randomized experimental design whenever possible (Campbell, 1969).
	 
	 Cook and Campbell (1979) present a variety of quasi-experimental designs, which can be divided into two main categories: nonequivalent comparison-group designs and interrupted time-series designs. In this section, we will discuss these two major groups of quasi-experimental designs, followed by a brief overview of single-subjects designs.}\\
	 REV&	Before the intervention, then after&\textcquote[p.142]{marczyk2005essentials}{\textbf{Reversal Time-Series Design} Also known as an ABA design (detailed on page 145), the reversal time-series design is basically a multi-subject variation of the single-subject reversal design, which will be discussed later in this chapter. The basic goal of this design is to establish causality by presenting and withdrawing an intervention, or independent variable, one to several times while concurrently measuring change in the dependent variable (as depicted in the following). As in the simple time-series design, this design begins with a series of pretests to observe normal fluctuations in baseline. The name “reversal” refers to the idea that causality can be inferred if changes that occur following the presentation of an intervention diminish or “reverse” when the independent variable is withdrawn.}\\
	 ABA&	Before, Intervention, After&See REV.\\
	 ABABA&	Iterated ABA&See REV.\\
	 ABABA...&Further Iterated ABA	&See REV.\\
	 EC&	Establish control&\textcquote[p.144]{marczyk2005essentials}{As with time-series designs, single-subject designs typically begin by establishing a stable baseline. Establishing a stable baseline involves taking repeated measures of a participant’s behavior (dependent variable) prior to the administration of any intervention to make certain that the participant’s behavior is occurring at a consistent rate. To obtain a stable baseline, the researcher must make special efforts to control all relevant environmental variables that otherwise might affect the participant’s responses. If the researcher does not know, or is uncertain, about which variables are relevant, the researcher must attempt to keep the participant’s environment as constant as possible by maintaining highly controlled conditions.}\\
	 1P&	Single participant&\textcquote[p.144]{marczyk2005essentials}{Not to be confused with non-experimental single-subject case studies, which are covered later in this chapter, the single-subject experimental design has a long and respected tradition in empirical research. According to Kazdin (2003c), single-subject experiments might be seen as true experiments because they “can demonstrate causal relationships and can rule out or make implausible threats to validity with the same elegance of group research” (p. 273). Similar to other experimental designs, the single subject design seeks to (1) establish that changes in the dependent variable occur following introduction of the independent variable (temporal precedence) and (2) identify differences between study conditions.
	 
	 The one way that single-subject designs differ from other experimental designs is in how they establish control, and thereby demonstrate that changes in a dependent variable are not due to extraneous variables. For example, experimental designs rely on randomization to equally distribute extraneous variables and on statistical techniques to control for such factors if they are found. Alternatively, single-subject designs eliminate between-subject variables by using only one participant, and they control for relevant environmental factors by establishing a stable baseline of the dependent variable. If change occurs following the introduction of the intervention, or independent variable, the researcher can reasonably assume that the change was due to the intervention and not to extraneous factors.}\\
	 SB&	Stable Baseline&See 1P\\
	 RC&	Retain control of Env&See 1P\\
	 \\
\midtitle{Non-experimental}{yin2009case,oates2007researching}
	CASE&Case Study&\textcquote[p.35]{oates2007researching}{\textbf{Case study}: focuses on one instance of the 'thing' that is to be investigated: an organization, a department, an information system, a discussion forum, a systems developer, a development project, a decision and so on. The aim is to obtain a rich, detailed insight into the 'life' of that case and its complex relationships and processes. (See Chapter 10.)}\\
	AR&Action research&\textcquote[p.35]{oates2007researching}{\textbf{Action research}: focuses on research into action. The researchers plan to do something in a real-world situation, do it, and then reflect on what happened or was learnt, and then begin another cycle of plan-act-reflect. (See Chapter 11.)}\\
	ETH&Ethanography&\textcquote[p.35]{oates2007researching}{Ethnography: focuses on understanding the culture and ways of seeing of a particular group of people. The researcher spends time in the field, taking part in the life of the people there, rather than being a detached observer. (See Chapter 12.)}\\
	CS& 	Choose subject&\textcquote[p.144]{marczyk2005essentials}{single-subject designs eliminate between-subject variables by using only one participant, and they control for relevant environmental factors by establishing a stable baseline of the dependent variable. If change occurs following the introduction of the intervention, or independent variable, the researcher can reasonably assume that the change was due to the intervention and not to extraneous factors.
	
	As with time-series designs, single-subject designs typically begin by establishing a stable baseline. Establishing a stable baseline involves taking repeated measures of a participant’s behavior (dependent variable) prior to the administration of any intervention to make certain that the participant’s behavior is occurring at a consistent rate. To obtain a stable baseline, the researcher must make special efforts to control all relevant environmental variables that otherwise might affect the participant’s responses. If the researcher does not know, or is uncertain, about which variables are relevant, the researcher must attempt to keep the participant’s environment as constant as possible by maintaining highly controlled conditions.}\\
	COMP&	Comprehensive description&\textcquote[p.148]{marczyk2005essentials}{the focus of the case-study approach is on individuality and describing the individual as comprehensively as possible. The case study requires a considerable amount of information, and therefore conclusions are based on a much more detailed and comprehensive set of information than is typically collected by experimental and quasi-experimental studies.}\\
	IDIP&	In-depth interviews with participants&\textcquote[p.148]{marczyk2005essentials}{Case studies of individual participants often include in-depth interviews with participants ...}\\
	IDIC&	In-depth interviews with collaterals&\textcquote[p.148]{marczyk2005essentials}{...and collaterals (e.g., friends, family members, colleagues), review of medical records, observation, and excerpts from participants’ personal writings and diaries}\\
	SUR& Surveys&\textcquote[p.33]{oates2007researching}{\textbf{Survey}: focuses on obtaining the same kinds of data from a large group of people (or events), in a standardized and systematic way. You then look for patterns in the data using statistics so that you can generalize to a larger population than the group you targeted. (See Chapter 7.)}\\
	RA&	Review of artefacts&\textcquote[p.148]{marczyk2005essentials}{According to Kazdin (1982), the major characteristics of case studies are the following:
	\begin{itemize}
		\item They involve the intensive study of an individual, family, group, institution, or other level that can be conceived of as a single unit.
		\item The information is highly detailed, comprehensive, and typically reported in narrative form as opposed to the quantified scores on a dependent measure.
		\item They attempt to convey the nuances of the case, including specific contexts, extraneous influences, and special idiosyncratic details.
		\item The information they examine may be retrospective or archival.
	\end{itemize}}\\
%	(RQ)&	Research question??\\
	PROPS&	Identify propositions&\textcquote[p.28]{yin2009case}{\textbf{Study propositions} [...] each proposition directs attention to something that should be examined within the scope of study.}\\
	UNITS&	Identify units&\textcquote[p.29]{yin2009case}{\textbf{Unit of analysis} [...] related to the fundamental problem of defining what the \enquote{case} is [... what the primary unit of analysis is].
	
Without such questions and propositions, you might be tempted to cover \enquote{everything} about the individual(s), which is impossible to do.}\\
	LINKS&	Identify how is data linked to propositions&\textcquote[p.34ff]{yin2009case}{be aware of the main choices and how they might suit your case study]}\\
	CRITS&Which are criteria to interpret findings&\textcquote[p.34]{yin2009case}{Criteria for interpreting a study's findings}\\
	THD&Theory Development&\textcquote[p.35]{yin2009case}{[including types on p.37]}\\
	GEN&Generalisation&\textcquote[p.38]{yin2009case}{[including Fig 2.2]}\\
	NAR&Narrative&\textcquote[p.121]{yin2009case}{Certain types of narrative, produced by a case study investigator upon completion of all data collection, also may be considered a formal part of the database and not part of the final case study report. The narrative reflects a special practice that should be used more frequently: to have case study investigators compose open-ended answers to the questions in the case study protocol. This practice has been used on several occasions in multiple-case studies designed by the author (see BOX 24). 
	
	[Box 24]
	
		In such a situation, each answer represents your attempt to integrate the available evidence and to converge upon the facts of the matter or their tentative interpretation. The process is actually an analytic one and is the start of the case study analysis. }\\
	{NSC\\NEI\\NID}&Nuance from the specific context/extraneous influences/idiosyncratic details&\textcquote{kazdin1982single}{According to Kazdin(1982), the major characteristics of case studies are the following:
		\begin{itemize}
			\item  They involve the intensive study of an individual, family, group, institution, or other level that can be conceived of as a single unit.
			\item The information is highly detailed, comprehensive, and typically reported in narrative form as opposed to the quantified scores on a dependent measure.
			\item They attempt to convey the nuances of the case, including specific contexts, extraneous influences, and special idiosyncratic details.
			\item The information they examine may be retrospective or archival.
		\end{itemize}}\\ 
%	NEI&Nuance from extraneous influences\\ 
%	NID&Nuance from idiosyncratic details\\
	\\
\midtitle{Design Science Research}{oates2007researching}
	D\&C&Design and creation&\textcquote{oates2007researching}{\textbf{Design and creation}: focuses on developing new IT products, or artefacts. Often the new IT product is a computer-based system, but it can also be some element of the development process such as a new construct, model or method. (See Chapter 8.)}\\
	PSA&Problem solving awareness&\textcquote[p.111]{oates2007researching}{Awareness is the recognition and articulation of a problem, which can come from studying the literature where authors identify areas for further research, or reading, about new findings in another discipline, or from practitioners or clients expressing the need for something, or from field research or from new developments in technology.}\\
	PSS&Problem solving suggestion&\textcquote[p.112]{oates2007researching}{Suggestion involves a creative leap from curiosity about the problem to offering a very tentative idea of how the problem might be addressed}\\
	PSD&Problem solving development&\textcquote[p.112]{oates2007researching}{Development is where the tentative design idea is implemented. How this is done depends on the kind of IT artefact being proposed. For example, an algorithm might need the construction of a formal proof. A new user interface embodying novel theories about human cognition will require software development. A systems development method will need to be captured in a manual that can then be followed in a systems development project. A new approach in digital art might require the development of an art portfolio tracing the development of the artist's creative ideas.}\\
	PSE&Problem solving evaluation&\textcquote[p.112]{oates2007researching}{Evaluation examines the developed artefact and looks for an assessment of its worth and deviations from expectations.}\\
	PSC&Problem solving conclusion&\textcquote[p.112]{oates2007researching}{Conclusion is where the results from the design process are consolidated and written up, and the knowledge gained is identified, together with any loose ends - unexpected or anomalous results that cannot yet be explained and could be the subject of further research.}\\
\midtitle{General}{}
	LITREV& Literature Review&\textcquote[p.33]{oates2007researching}{literature review in Figure 3.1}\\
	VALID&	Threats to validity&\textcquote[p.40]{yin2009case}{fours (general) tests for validity}\\
	BIAS&	Reflection on bias&\textcquote[p.72]{yin2009case}{[Avoiding bias for case studies]}\\ 
	REP&	Reporting&Something here\\
\midtitle{Data Generation}{oates2007researching}
	INT&\textcquote[p.36]{oates2007researching}{\textbf{Interview}: a particular kind of conversation between people where, at least at the beginning of the interview if not all the way through, the researcher controls both the agenda and the proceedings and will ask most of the questions. One-to-one and group interviews are possible. (See Chapter 13.)}\\
	OBS&Observation&\textcquote[p.36]{oates2007researching}{\textbf{Observations}: watching and paying attention to what people actually do, rather than what they report they do. Often involves looking, but it can involve the other senses too: hearing, smelling, touching and tasting. (See Chapter 14.)}\\
	QUES&Questionnaire&\textcquote[p.36]{oates2007researching}{\textbf{Questionnaire}: a pre-defined set of questions assembled in a pre-determined order. Respondents are asked to answer the questions, often via multiple choice options, thus providing the researcher with data that can be analysed and interpreted. (See Chapter 15.)}\\
	DOC&Documents&\textcquote[p.36]{oates2007researching}{\textbf{Documents}: documents that already exist prior to the research (for example, policy documents, minutes of meetings and job descriptions) and documents that are made solely for the purposes of the research task (for example, a researcher's logbook or design models). Also includes 'multimedia documents': visual data sources (for example, photographs, diagrams, videos and animations), aural sources (for example, sounds and music) and electronic sources (for example, websites, computer games and electronic bulletin boards). (See Chapter 16.)}\\
\end{longtblr}
\end{document}


%As a running example, we'll be working with the following research objectives, which you saw in Example~\ref{ex:machinelearning} in Stage 1\footnote{See page~\pageref{ex:machinelearning}.}\todo{What was the research problem?}:

\subsection{Decomposing objectives into tasks}

You've chosen your research design based on area, and you've got your research objectives from Section~\ref{sect:???}. How do you go about mapping one into the other?

Our suggested template for creating objectives had three components: identify, assess, and recommend.
%
%
\begin{description}
\item [identify:] literature review; questionnaire, interviews; problem solving awareness, ...
\item [assess:]	what goes here? problem solving suggestions; interviews; problem solving development; ...
\item [recommend:] what goes here? problem solving evaluation; problem solving conclusion; validity; bias; ...
\end{description}
%

\todo{Turn this example into identifying which research design.}

\begin{example}{Recap: Applying Machine Learning}
In Stage 2, we refined Clara's research aim, which was:
%
\blockquote{to apply Machine Learning (ML) to improve the accuracy of resources and time forecasting in the context of small engineering plants}
%
to three following three objectives:\todo[inline]{JGH: needs doing if not already done}

\begin{description}%[start=0,label={(\bfseries R\arabic*):}]
\item [Objective 1] to identify which ML techniques are applicable to resource and time forecasting in the context of small engineering plants, which will allow us to identify specific ML techniques to be used in the project, to ensure the work is feasible within the time-frame of the project. 

\item [Objective 2] to test the accuracy of forecasting of those techniques which will allow us to investigate and compare how accurate the chosen techniques are in their forecasting application. 

\item [Objective 3] to provide recommendations as to how integrate those techniques effectively in engineering practice in order to improve forecasting accuracy which will allows us to draw some conclusions from the research conducted and make recommendations to improve professional practice.
\end{description}

Note how those objectives were designed to build on each other and, when successfully completed, they'd contribute to meet the overall aim.
\end{example}

\begin{example}{Example – cont'd}In our example, the first objective is met once we have identified the relevant ML techniques. There are two complementary ways to do this: to look at the literature and to ask practitioners. As a result, we could break this objective down into the tasks, and deliverables, indicated in the following table

\begin{longtable}{@{}p{0.1\textwidth}@{}p{0.9\textwidth}@{}}
\caption{Objective 1: to identify which ML techniques...}\\
\toprule
\textbf{Task} & \textbf{Deliverable} \\\midrule
\tabletitle{to identify relevant ML techniques in the academic literature} & a collection of relevant ML techniques reported in the literature \\\\
\tabletitle{to ask practitioners which techniques they employ} & a collection of relevant ML techniques used in professional practice \\
\bottomrule
\end{longtable}
\end{example}

\begin{question}[subtitle={ACTIVITY: Establishing tasks and deliverables}] Consider your research objectives. For each, identify related tasks and deliverables.\todo[inline]{I don't think I could do this at this point.}

\begin{guidance}You should draw a table similar to that in our running example. You should ensure that the tasks provide a reasonable break down of your objectives into discrete pieces of work.
\end{guidance}\end{question}
%%Hack to correct tcbox behaviour
\color{black}

Your tasks and deliverables don't need to be perfect in stage 3 – there are two more stages to perfect them after all – and are likely to be revised as you progress through your project. However, it is important that you have thought about specific work you will need to carry out to meet your objectives.

\subsubsection{Relating tasks to research methods}
The way to carry out your tasks and meet your objectives is through the application of research methods.

\begin{example}{EXAMPLE - cont'd }Following on from our previous example, we have extended the table to include an indication and justification of candidate research methods for each task.

\begin{tabulary}{\textwidth}{@{}LLLL@{}}
%\caption{\textbf{Objective 1: to identify which ML techniques...}
\toprule
\textbf{Task\hspace{1cm}} & \textbf{Deliverable} & \textbf{Relevant research methods} &\textbf{Justification and feasibility}\\\midrule
\tabletitle{to investigate the academic literature in order to identify relevant ML techniques} & a collection of relevant ML techniques reported in the literature & review of existing literature & I can access relevant literature via my university library \\\\
\tabletitle{to ask practitioners which techniques they employ} & a collection of relevant ML techniques used in professional practice & questionnaire, possibly followed by interviews & I have access to professional networks, which I could use to distribute the questionnaire, and possibly to recruit participants for follow-up interviews \\\bottomrule
\end{tabulary}
\end{example}

Note that the choice of research methods in relation to your research tasks is an essential part of your research design. In fact, the two influence each other: your objectives and related tasks direct you towards specific research methods, which in turn have to be part of your overall research design.

\begin{question}[subtitle={ACTIVITY: Associating methods to tasks and deliverables}] Extend your tasks and deliverables table with your candidate research methods, including stating why they apply and are feasible for your project. Revise your research design draft from Stage 2 so that is consistent with those choices.

\begin{guidance}
Refresh your understanding of chosen research methods from the study work you carried out in Stage 2. It is important you keep reviewing your choices with your supervisor.
\end{guidance}
\end{question}
%%Hack to correct tcbox behaviour
\color{black}

\subsubsection{Research task deliverables}

\todo[inline]{Add something here}

\section{Research procedures}
\todo{What's the relationship to objectives and tasks?}

Once you have chosen the set of research methods you will apply, you must establish exactly how you will do that, something we refer to as \textbf{research procedures}.

Your research procedures will be method specific, in that each method you choose to apply will come with recommended practices, which you will need to contextualise to your own project needs, including your access to participants, data or other kind of evidence. For instance, there are plenty of guidelines in the literature on how to design questionnaires, including which type of questions to include and how to phrase them. There are also recommendations concerning testing the questionnaire design prior to its use, and of course, there are many ways a questionnaire can be administered. In writing your procedures for this research method, you would have to be specific on how each of the above applies in your project.

It is important, therefore, that you master the research methods of your choice, starting by reviewing once again the related academic literature.

\begin{question}[subtitle={Activity: Sketching research procedures}] Consider the research methods you intend to apply, and the related review you conducted in Stage 2. Reconsider those materials, possibly going back to the literature sources, to learn how to apply the methods effectively within your project.

For each method, sketch possible procedures of application, ensuring you make appropriate reference to the literature you have reviewed and best practice guidelines therein.

\begin{guidance}It is possible that the review you conducted in Stage 2 is not sufficient, in which case you will need to extend it to complete this activity.

You should focus on practical aspects of applying the methods, including specific processes and techniques to gather, summarise and present your evidence in your reports.

Depending on the extent you need to review further academic literature, this activity could be quite substantial, so you should set aside up to 20\% of your study time to complete it.

\end{guidance}\end{question}
%%Hack to correct tcbox behaviour
\color{black}

\section{Assessing validity}
As your intended research design becomes clearer, you will soon be testing some aspects of it in your pilot work. Before you do that, however, you need to consider if the choices you have made will allow you to gather evidence and derive findings in a systematic, rigorous, repeatable and reliable fashion so as to address your research problem. This is referred to as assessing the overall \textbf{validity} of your research design, which is broken down into the following considerations.

\textbf{Construct validity} asks whether you have put your design together logically by focusing on the relationship between evidence and research problem. Here you ask yourself whether the evidence you will generate through your chosen research design will be accurate and relevant to address your research problem. This tests the logical coherence of your aim, objectives, tasks, methods and deliverables in relation to the research problem and the knowledge gap you intend to address. With construct validity, you are asking: \emph{have I designed my research in the right way?}

\textbf{Internal validity} is concerned with the way you gather and analyse evidence. All research strategies and methods come with recommendations of good practice to ensure that your research is both systematic, repeatable and reliable. In your work, you need to ensure that you follow such practices and are aware of possible pitfalls. For instance, in experimental research you need to control all factors which may effect outcomes beyond those under study: failing to exercise such control will lead to observations and measurements which are unreliable. In assessing internal validity, you should also take into account limitations of human perception and cognition, and any potential personal bias. With internal validity, you are asking: \emph{have I executed my research in the right way?}

\textbf{External validity} relates to the extent you will be able to generalise your findings beyond the immediate context of your research. For instance, you may conduct a case study within a specific organisation, so here you are asking whether and how what you have discovered may apply to other organisations. With external validity, you are asking: \emph{will my research lead to findings that apply somewhere else?}

Anything that gets in the way of validity in research is termed a \textbf{threat to validity}. Different research strategies and methods are exposed to different threats, something you should have encountered in your review of the literature on your chosen methods.

\begin{question}[subtitle={Activity: Assessing validity of research design}] Conduct an initial assessment of your chosen research design in relation to the three kind of validity discussed above. Write down a short summary of your thinking in support of each, and of possible threats to validity you envisage.

\begin{guidance}You may need to refer back to the literature you have reviewed to identify specific threats which apply to your chosen research methods and strategies.

You won't be able to address this in full at this point in your project, particularly the internal validity, which refers to the execution of your research design. Nevertheless, it is important for you to consider validity and possible threats from the onset. You will return to this topic at the end of your project, as part of the overall assessment of your research, to reflect on the validity of your completed research.

\end{guidance}\end{question}
%%Hack to correct tcbox behaviour
\color{black}

\section{Conducting your pilot work}

Your \textbf{pilot work} will be a small scale test of some of the methods and procedures you will apply in the next stage of your project. Its main function is to help you assess the feasibility of your research design, or at least some aspects of it, and build your confidence in the approach you have chosen.

As such, your pilot work may not contribute directly to your aim and objectives, but it should help you decide whether you can actually do what you have planned to do, or inform how your research design and project plan should change instead.

There are no constraints on what you can do for your pilot work, other than you should exercise some aspect of your research design. It is therefore essential that you agree what you are going to do with your supervisor first.

\begin{question}[subtitle={Activity: Planning and executing your pilot work}] Plan your pilot work and discuss your plan with your supervisor.

Once you have agreed the way forward, execute your plan and write a summary of both its execution and outcomes.

\begin{guidance}
This is a substantial activity, which will take you up to 35\% of your study time.

Your summary should include:
%
\begin{itemize}
\item an indication of which aspects of your research design your pilot work was concerned with
\item any methods and procedures applied
\item any data or evidence gathered, including possible modelling, artefact design or prototyping, appropriately presented and summarised
\item lessons learnt and any resulting revision to your research design and project plan, particularly in relation to construct and internal validity of your research design.
\end{itemize}
%
To complete this activity successfully, it is essential that you agree your pilot work plan with your supervisor upfront, and discuss your progress on a regular basis.
\end{guidance}\end{question}
%%Hack to correct tcbox behaviour
\color{black}

\section{Reporting in Stage 3}
At the end of Stage 3, you should complete a report, extending that of Stage 2 and covering the work you have carried on in this stage. The structure we suggest and an indication of the contents are shown in Table~\ref{tab:reportStructure}.

%%Report Structure Table is repeated throughout the thesis. This is the template
%%Format is:
%%\begin{ReportStructureTable}
%%	\tabletitle{Section 1: Introduction}\\
%%	\begin{enumerate}[label={1.\arabic*:}]
%%	\item Background to the research 
%%	\item Justification for the research 
%%	\end{enumerate}
%%	& This section should provide an introduction to your research topic in its wider context (as background) and your justification of why the research is worth pursuing. It should be well articulated and supported by evidence \\
%%\end{ReportStructureTable}
%%Still to do: remove space from above enumerate environment
%%Sets the chapter across two columns in bold
\begin{ReportStructureTable}{tab:reportStructure}
\tabletitle{Title} & Your title should succinctly capture your research problem and aim\\\\
\tabletitle{Section 1: Introduction}\\
\begin{enumerate}[label={1.\arabic*:}]
\item Background to the research 
\item Justification for the research 
\end{enumerate}
& This section should provide an introduction to your research topic in its wider context (as background) and your justification of why the research is worth pursuing. It should be well articulated and supported by evidence \\\\
\tabletitle{Section 2: Literature review}\\
\begin{enumerate}[label={2.\arabic*:}]
\item Review of existing relevant knowledge 
\item Critical summary, including knowledge gap to be addressed by the research 
\end{enumerate}
& Your review should provide a critical account of your in-depth engagement with the academic (and other) relevant literature, including identifying key trends, ideas and possible knowledge gaps. Most of your citations should point to academic articles. Your critical summary should highlight key insights from your review and provide a strong justification for your proposed research. Both coverage and depth of your review matter. You should ensure that your review is well structured, with a logical narrative flow and your arguments are well supported by evidence  \\\\
\tabletitle{Section 3: Research definition}\\
\begin{enumerate}[label={3.\arabic*:}]
\item Problem statement 
\item Aim, objectives, tasks and deliverables
\item Knowledge contribution
\end{enumerate}
& You should ensure that your research problem is well articulated and appropriate for your course and your personal and professional circumstances, that your aim and objectives are consistent with research problem, that tasks and deliverables break down your objectives appropriately and are clearly related to your chosen research methods, and that the intended knowledge contribution of your research is clearly articulated \\
\tabletitle{Section 4: Research design}\\\\
\begin{enumerate}[label={4.\arabic*:}]
\item Evidence and data 
\item Research strategy and methods
\item Research procedures
\item Ethical, legal and EDI considerations
\end{enumerate}
& This section should demonstrated your critical engagement with all elements of research design, including a detailed account of the data and evidence needed in your research, the research methods and research strategies you will to apply, and how you will apply them within your project. Your account should be supported by a clear rationale and insights from the related literature, and appropriately justified in relation to your research problem, aim and objectives. It should also demonstrate your careful consideration of ethical and legal matters, and that your research will comply with your course and university requirements\\\\
\tabletitle{Section 5: Analysis and interpretation}\\
\begin{enumerate}[label={5.\arabic*:}]
\item Pilot work
\end{enumerate}
& This section should report on a well thought-out pilot work which clearly and competently test some significant aspect of your research design. It should demonstrate good critical reflection on outcomes and highlight any adjustments needed as a result. \\\\
\tabletitle{Section 6: Assessment of your proposed research}\\
\begin{enumerate}[label={6.\arabic*:}]
\item Qualification fit
\item Personal and professional fit
\item Technical skills and resources needed
\item Statement of feasibility
\item Personal reflection on research process
\end{enumerate}
& In this section you should continue to argue how your research is a good fit across all criteria. You should provide a clear rationale as to why you think what you are proposing is feasible. You should also reflect on your growing understanding of the research process, including key learning and aspects you have found particularly challenging. \\\\
\tabletitle{Section 7: Planning, scheduling and risk assessment}\\
\begin{enumerate}[label={7.\arabic*:}]
\item Key priorities in follow-up stage
\item Personal and professional fit
\item Risk assessment
\end{enumerate}
& In this section you should reflect on the progress you have made in Stage 2 and establish your priorities for the next stage. You should also review your risk assessment as appropriate.\\\\
\tabletitle{Section 8: References}\\ & You should keep your growing references in good order and ensure you apply the required bibliographical style consistently. Ideally, you should use a BMT to generate and integrate your references within your report\\\\
\textbf{Appendix A: Work schedule}& Your revised work plan\\\\
\textbf{Appendix B: Risk assessment table}& Your revised risk table \\
\bottomrule
\end{ReportStructureTable}

\endinput

\begin{question}[subtitle={Activity: Putting your report together}] Using your word processor of choice, and starting from your previous report, complete your Stage 3 report by applying the structure and guidance in Table~\ref{tab:ReviewCrit}, and making good use of your notes and summaries from all related activities you have carried out so far.

\begin{guidance}In this first pass at putting together your report, you should focus primarily on completeness, ensuring that each section includes at least draft content.
\end{guidance}\end{question}
%%Hack to correct tcbox behaviour
\color{black}

As in the previous stages, after you have filled in your report you should review and revise it iteratively until you are happy with your account, and are ready to move on. 

\begin{table}[htbp]
\begin{minipage}{\linewidth}
\setlength{\tymax}{0.5\linewidth}
\centering
\caption{Criteria to review your report\label{tab:ReviewCrit}}
\small
\begin{tabulary}{\tablewidth}{@{}LL@{}} \toprule
 \textbf{Criteria} & \textbf{Prompts} \\
\midrule

 \tabletitle{Completeness} & Are all sections of the suggested structure completed in line with the guidance provided? \\
 \tabletitle{Good academic writing practices} & Have you applied good academic writing practices throughout? \\
 \tabletitle{Logical structure and flow} & Have you structured your narrative appropriately to ensure a logical flow of arguments? \\
 \tabletitle{Supporting references or evidence} & Are your key arguments supported by appropriate references or other evidence? \\
 \tabletitle{Citation and reference style} & Do all your citations and references comply with the required bibliographical style? \\
 \tabletitle{Avoiding plagiarism} & Have you acknowledged the work of others and distinguished it from your own appropriately? \\
 \tabletitle{Standard of English (or any modern language you use)} & Have you proof-read your report carefully to remove all typos and grammatical errors? \\
\bottomrule

\end{tabulary}
\end{minipage}
\end{table}

\begin{question}[subtitle={Activity: Reviewing your report}] Apply the criteria in Table 1 to review your current report and write up a summary of your assessment.

\begin{guidance}For each criteria, consider the related prompts to help you assess your report overall, and write down any further work needed for your next stage.
\end{guidance}\end{question}
%%Hack to correct tcbox behaviour
\color{black}

\section{Reflection: Stage 3}

%%More here

%%Repeated reflection activity
%%Repeated Activity for all reflections
\begin{question}[subtitle={Activity}]
$<$Needs assessing for content and structuring into activity + guidance$>$

This activity has four parts: the first is something you should be doing regularly, but won't make you into a disobedient or indocile thinker. The second, third and fourth may help you get started and keep going.

Part 1: Think about your study this far -- using this book and anything you've done for your dissertation in parallel -- as a journey. More from elsewhere, including   !!.

Part 2: think about yourself and the way you think. How does your desk look? Is it messy or tidy? Do the same for your computer desktop. Is it empty or are there hundreds of files strewn across it? Do you think your tidiness or untidiness will affect the way you do your research? How about how you keep your -- critically important -- bibliographic database which may contain up to a hundred academic\footnote{It's not unknown to have more than a hundred.} and other articles by the time you're finished?

Part 3: think about the context of your research. Which professional pressures are there on you to succeed in solving your research problem? Pressures could come in many forms: financial -- there's a promotion for you at the end of it; peer -- your colleagues know that you are studying will have good expectations of your result and you'll want to prove them right\footnote{Or wrong, depending on the colleague!}. Are you sponsored by your employer? Will you be able to report a negative outcomes to your research, for instance, that there is no solution to our problem using the current technology stack? A negative result is a very good research outcome, even if it tends to satisfy fewer non-academics than a positive result.

Which family pressures do you feel? It's' not unusual that you will have given up a paying role to study, moving the responsibility to provide onto another member of your family. What are their expectations?

Part 4: what's that thought nagging at the back of your mind? Is it ``How will I start?'' Or ``Will I be able to dedicate enough time to this?'' Or ``Can I really do this?''. Or ''Is ``shouldn't I be bringing in a wage rather than studying?''

You may be one of the lucky ones that doesn't have such negative thoughts, but negative thoughts are a very natural part of steps into the unknown. And research is precisely that, a step into the unknown. At least if you are aware of the doubts you naturally have, you can manage them. Think about making even the tiniest of steps forward in your research visible and celebrated! Work with Kansan boards where progress is encouragingly visible as you move a task from the inbox to the outbox. If you have concerns about managing your time, start using one of the many tools out there that break time up into manageable units and help manage it for you. If your concerns are about how to organise your thoughts, look into mind maps, lists, todo lists.

Thinking early and often through reflection is a powerful way of doing better. Do it well and your final report will be better than you will have expected.

It's worth saying that, at the end of what could be an exhausting journey, you will not fully appreciate your achievements. That realisation may have to wait until you are rested, graduated, or some distant time later.

But it will come.

\begin{guidance}
%Hack to correct tcbox behaviour
\color{black}
Something here
\end{guidance}\end{question}
%%Hack to correct tcbox behaviour
\color{black}



