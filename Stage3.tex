\part{The Research}
%%Candidate to move to headers
%\DeclareExpandableDocumentCommand{\tabletitle}{O{2} m}{\\[1ex]\multicolumn{#1}{@{}p{\textwidth}@{}}{\textbf{#2}}\\ }%

%\newcommand{\hwidth}[1]{\makebox[0pt][l]{#1}}
%\DeclareExpandableDocumentCommand{\tabletitle}{m}{\SetCell[c=2]{\textbf{#1}}&\\ }

%%Structures the report table, headings to the left in a list, explanation to the right
%%#1 = caption label

%%Resource type = fulltext cite in itemized list
\DeclareCiteCommand{\resourcelistcite}
  {\usebibmacro{prenote}\begin{itemize}\item[] }
  {\usedriver
     {\DeclareNameAlias{sortname}{default}}
     {\thefield{entrytyper}}
  } 
%     {%
%		\footnote{\thefield{annotation}}
%	 }
  {\item[] \addspace}
  {\end{itemize}\usebibmacro{postnote}}

\newenvironment{ReportStructureTable}[1]{%
	\setlist{nosep} %compress any lists
	\begin{longtable}{@{}p{0.3\tablewidth}>{\vspace{-3pt}}p{0.7\tablewidth}@{}}%%JGH \vspace{-3pt} is a hack to align first lines
	\caption{Report structure and content guidance\label{#1}}\\
	\toprule
	\textbf{Suggested structure} & \textbf{Content guidance} \\
	\midrule%
}%
{\end{longtable}}

\newenvironment{MoreDetails}[0]{\subsection{More details} The following citations provide good starting points for further reading: }{}

%\section{Developing your research design}

By now you will have some mastery of the techniques and tools that you need to \emph{do} research at masters level. You may also have ideas about what you still need to do in the next step\footnote{If not, don't worry – we've got you covered with this chapter!}.

With the skills you have so far gained, you're developing into an independent researcher\footnote{Being an independent researcher isn't one of the examined outcomes of masters research, but if you're feeling confident in your research that's a good thing.} and you may feel that this book holds nothing more for you. 

Stay with us a little longer though: the next sections aren't as long as those that you've studied already – you'll be doing more yourself, honing the skills you've picked up as you go along – but they might help to keep you systematic and on the path to submission. 

%\newcommand{\percentthroughbook}{
%%\thepage
%%\getpagerefnumber{LastPage}
%\intcalcDiv{\intcalcMul{\thepage}{100}}{\getpagerefnumber{LastPage}}\%
%}

You won't be surprised to know that stage 3 comes next; there's another research increment coming.

%%Removed for now
%You're now \(\percentthroughbook{}\) through this book – not long to go before you're finished. Stick with us for a while longer.

%\section{Introducing stage 3}

\todo{LR: update at the end}
\todo{LR: to check all activity titles at the end}

In Stage 3 you will focus on adding detail to both your aim and objectives and your research design. Stage 3 assumes that you have completed your Stage 2 work, and possibly discussed it with your supervisor\footnote{If your proposal still requires some `remedial' work to fully satisfy your course requirements then you should carry that out before moving on.}, particularly your research design choices.

With reference to our 5-stage framework, the activities which are in focus in Stage 3 are recalled in Table~\ref{tab:stage3acts}, which also provides some guidelines for your interaction with your supervisor during this stage.

\begin{ResActtblr}[caption = {Stage 3 Research Activities (15\% of project length)\label{tab:stage3}}]{}
\tableheader
\reportheader{Identifying the research problem}\\
	&Adjust, if needed && &2\% & \\
\reportheader{Reviewing the literature}\\ 
	&Adjust, if needed && &3\% & \\
\reportheader{Setting research aim and objectives}\\ 
	&finalise aim and objectives, and define tasks and deliverables & 10\% & Suitability of tasks and deliverables from objectives \\
\reportheader{\textbf{Choosing the research design}}\\ 
	&Complete research design, with detailed consideration of data and evidence, research strategy, research methods and procedures & 20\% & Suitability of research procedures \\
\reportheader{\textbf{Gathering and analysing evidence}}\\ 
	&Conduct pilot work to test aspects of your research design & 35\% & Scope of your pilot work \\
\reportheader{\textbf{Interpreting and evaluating findings}}\\ 
	&n/a & 0\% & \\
\reportheader{\textbf{Reporting, critical reflection and conclusions}}\\ 
	&Assess research progress and write up Stage 3 report & 25\% & Any further improvements required \\
\reportheader{\textbf{Work planning and risk management}}\\ 
	&At stage start, review work from previous stage and project risk; adjust plan as needed If you have received feedback from supervisor on your previous stage work, adjust plan to include any revision recommended & 5\% & Any major adjustment required \\
\end{ResActtblr}

%\todo{This is a repeated activity through all stages, and so should be templated.}
\begin{question}[subtitle={Activity: Understanding the effort needed in this stage}] Consider Table~\ref{tab:stage3acts} carefully, taking notice of the entries in the `Effort within stage' column. Write down the most time-consuming activities in this stage and what is expected under each.

\begin{solution}Developing your research design further and conducting your pilot work will constitute your major effort in this stage (55\% of the study time in total): your pilot work will be an initial test of some aspects of your research design, including a proof-of-concept application of some of your chosen methods.
\end{solution}\end{question}
%%Hack to correct tcbox behaviour
\color{black}

%\end{document}


\chapter{Defending your claim of new knowledge}

Being able to assert that you have made a contribution to knowledge is the point of structuring your research through a well thought-out research strategy  -- hence, the importance of methodology in research. 

Choosing a good strategy is only the starting point, however. Having made your claim to knowledge at the end of your project, you still need to defend it in your dissertation. That means considering, essentially, everything that could have gone wrong -- any weakness -- with the execution of your research strategy, and explaining how you've dealt with it. 

Introducing potential research weaknesses upfront and ways to deal with them is the purpose of this section: with this information, you can then be more mindful in the choice and execution of your own research strategy.

Figure~\ref{fig:RVs} illustrates the point we are making. At its core is your claimed knowledge contribution at end of your project. Its defence is what you need to argue in your dissertation. Such a defence has to withstand external scrutiny, say that of your examiner or the wider community of scholars, researchers or practitioners your work is intended for. You claim to knowledge is subject to a number of weaknesses (four main types are considered in this section, illustrated as potential `cracks' in your defence), and should you recognise any of them in your research, then your defence should explain how they've been dealt with (illustrated as bandaids over the cracks). The kind of `bandaid' will depend on what you decided to do, one of addressing, avoiding, deferring or ignoring the weakness. If you choose to address it, then some specific kinds of bandaid are available to you: the ones we consider in this section are triangulation, reflexivity and critical review.

\todo{figure to re-draw}

\begin{figure}[hbtp]
\centering{
  \includegraphics[width=\textwidth]{Figures/ResearchVulnerabilities.pdf}
  \caption[]{Research vulnerabilities  \label{fig:RVs}
  	}  
  }
\end{figure}

\section{Weaknesses and ways to deal with them}

We can class weaknesses in claimed knowledge contribution (see figure~\ref{fig:RVs}) as follows:
%
\begin{itemize}
\item validity weaknesses, i.e., the claim you have made to new knowledge isn't sufficiently credible, trustworthy, or accurate to be considered knowledge, or can't be generalised or transferred beyond your study

\item reliability weaknesses, i.e., the procedures that you have used to establish your claim of new knowledge are not dependable, cannot be replicated under the same conditions or are not sufficiently repeatable in other contexts, or the descriptions and interpretations provided are incoherent or inadequate

\item bias weaknesses, i.e., the claim you have made to new knowledge has been affected by your implicit or explicit cognitive biases, or bias affecting human participants in your study, making the new knowledge invalid

\item novelty weaknesses, i.e., the hole in the literature that you claimed existed doesn't actually exist.  If there is no hole, then you cannot have contributed new knowledge --- perhaps you missed some key papers in your literature review or, perhaps in the time that you've taken to complete your research, someone else has made a similar contribution to knowledge as that you claim. Alternatively, while a hole may exist, the novelty of your claim may be in doubt --- perhaps your research was not able to achieve all that you were hoping for. 
\end{itemize}

There are, of course, connections between these types of weakness. In particular, if your research methods are not reliable, then any resulting claim to knowledge is unlikely to be valid. For instance, if the scale you use to measure the weight of an object returns different values every time (it's unreliable), then you can't draw a valid conclusion on the weight of that object. However, reliability is not sufficient for validity. For example, your scale may reliably return the same weight every time, but may overestimates it: in this case, while your scale is reliable, it is inaccurate, so that you still can't draw a valid conclusion on the weight of the object (unless you know precisely by how much your scale overestimates weights). Bias also affects validity. For instance, you may have a preconception of what the outcome of your research should be, so you discard any evidence to the contrary and only retain evidence that confirms your bias. In this case, your conclusions are untrustworthy, hence invalid. It is therefore essential to consider the weaknesses that may affect your research and take action to ensure they will not impact the validity of your claim.

As you will see in the next section, different research strategies are affected differently by these weaknesses. For instance, research based on quantitative objective measurements will focus more on ensuring reliability than research based on subjective interpretations of qualitative information, where the researcher's bias is more likely to have a negative impact. 

 In general, the possible actions you can take to deal with potential weaknesses fall into three options\footnote{There is actually a fourth way, which is to be aware of the weakness but to ignore it. We do not recommend this as your examiner of your dissertation is likely to have detailed understanding of the research strategy you have chosen, including its potential weaknesses, and is likely to pick any methodological omissions up.}:
%
\begin{itemize}
\item avoid the weakness, i.e., choose a research strategy which is not troubled by the weakness. Part of the justification for the choice of research strategy can then be a discussion, if necessary, that the weakness doesn't arise.% This would be reported as part of the research method.

\item address the weakness, i.e., be aware of the weakness during the research and put in place further strengthening research. This might be, for instance, a second of further iteration of the research strategy which addresses discovered weaknesses in earlier research. This would be reported as part of the research design.

\item acknowledge and defer\footnote{Although it may seem to have similar outcomes, this is a much better strategy than simply ignoring the weakness as, although you don't address it, you make the examiner aware that you are aware of it. It can also give you a very neat way of filling out your future work.}, most usually at the end of the research period when the research is complete, i.e., write a reflection on the effect the weakness had on the outcomes and commit to addressing that weakness in future research. This would be reported as part of your \enquote{Discussion} and \enquote{Conclusion and future work} chapters.
\end{itemize}

If you can't avoid a weakness and you can't defer it, you have to address it. Addressing it means that your examiner will have their questions answered about the weaknesses they know occur in the type of research you're doing. Their evaluation will be through the questions they ask of your research and you must be prepared to answer them. 

\subsection{Where to defend your claim}\label{sect:defendingYourClaim}

In your defence of claimed knowledge contribution, you should consider all potential weaknesses in turn – ignoring them leaves yourself open to a negative outcome of expert scrutiny. For each, you should make arguments as to why your claim doesn't suffer from it, or if it does to some extent, that you have dealt with it in a way that ensures there is still a contribution to knowledge arising from your research.

Typically, there are two places at which weaknesses in your claimed knowledge contribution should be discussed:
%
\begin{itemize}
\item in your dissertation, in all cases
\item in any \emph{viva voce} associated with your research course\footnote{As not all masters research have an associated \emph{viva voce}, weaknesses should always be addressed in the dissertation itself. Even if your course does have a \emph{viva voce}, it can be a nerve-racking experience to be confronted by an examiner asking questions to which you have no answer because you haven't thought about it!}
\end{itemize}

In general, an examiner will explore such weaknesses through a number of questions they ask of your dissertation. For each research strategy, many of these questions\footnote{If not all; although examiners will have their own way of asking them!} can be predicted with reference to the types of weaknesses we have discussed above. Somewhere in your dissertation, then, you will need to expose your research strategy weaknesses and argue how your research has addressed them.

Here is an example paragraph taken from an actual dissertation~\parencite{miles2019dispelling} with our commentary on specific points to the right, in the margin:

%\bigskip

\begin{adjustwidth}{4cm}{4cm}
%\subsection*{Limitations of Study}

My observational study focuses solely\footnote{Being specific on which phenomena are studied...} on the external elements of the embouchure and what can be seen in real time with the human eye\footnote{...and on the observations made of them...}, through the recording of video images. My analysis, and the conclusions that come from it, has been made from a purely visual perspective, captured by combinations of camera angles, without needing the use of any complex and expensive technologies\footnote{...thus correcting any expectations of what might have been achieved...}. In embarking on this research project, the initial intention was to measure facial muscle activity using Electromyography. This method proved to be too costly\footnote{...contextual factors prevented more sophisticated observations...} and the heavily mathematic and science based analysis process, out of the current skill set of this researcher\footnote{...and initial investigations reveals how difficult this would be}. Furthermore, due to the significant evidence found in the literature regarding the internal embouchure, the concept of the tongue being a pivotal element in facilitating pitch change has been accepted as fact and deemed unnecessary for further study in this project\footnote{There was no knowledge contribution to be made in this particular area...}. Therefore the ultimate goal of my research is to inform the teaching and learning of brass wind performance, with particular reference to the role of the embouchure\footnote{...and so the knowledge contribution was ...}. With this in mind, it is therefore important that the data obtained through this study be identifiable through the simplest means possible, so that it can be of the most benefit to the brass-playing community\footnote{...and our research goals were set accordingly.}.
\end{adjustwidth}


\begin{question}[subtitle={Activity: Which weaknesses are discussed?}]
	Consider the extract above alongside our comments. Which kinds of weakness does it refer to? How were they dealt with? Which other weaknesses could have been discussed?
\begin{solution}
We found two potential weaknesses which were considered and addressed:
\begin{itemize}
	\item novelty: by being specific on the phenomena studied (the external elements of the embouchure), the text clarifies where the claimed novelty of the research lies. This makes it easy to check against related work in the literature, something the text could have mentioned explicitly
	\item validity: the observation of such phenomena through video images is defended as a valid method in relation to the aim of devising a practical approach to inform teaching and learning. This is in contrast to more sophisticated, but costly, approaches that would have been possible, but deemed unnecessary for the aim of the research.
\end{itemize}	
Other potential weaknesses not discussed are:
\begin{itemize}
	\item reliability: how reliable were the observations? Would another researcher have reached similar conclusions?
	\item validity: the study assumes the embouchure is a key factor in the teaching and learning of a brass instrument. Where does this assumption come from? 
\end{itemize} 	
As this is only a brief extract, it is possible, of course, that these weaknesses were considered and dealt somewhere else in the dissertation. 
\end{solution}
\end{question}

%We go into more detail of the forms of weakness and how they can be treated %– through triangulation, reflexivity and critical literature – 
%below.
%
%
%There are three ways of dealing with the vulnerabilities in your research. You can:
%%
%\begin{itemize}
%\item patch: 
%\end{itemize}
%%
%

%\subsection{Introduction}\footnote{Probably not here.}
%
%There's still a great debate raging about what knowledge actually is. 
%
%The analysis of knowledge involves identifying the components that make up knowledge and understanding the structure of the concept of knowledge. The traditional definition of knowledge is as \emph{justified true belief}. This combines three conditions, usually presented out of the order they written in:
%%
%\begin{itemize}
%\item the \emph{belief} condition: we must \emph{believe} that something is true for it to be knowledge; we cannot know something if we don't believe it to be true, even if it is, in actuality, false;
%\item the \emph{justification} condition: a belief is justified if we have a reason to believing it, i.e., in the words of Wikipedia, \enquote{if another mental state supports it}~\parencite{enwiki:1204227228}, such as \enquote{a sensory experience, a memory, or a second belief}. This says that there is a reason or evidence for believing something;
%\item the \emph{truth} condition: that something \emph{believed} is actually true. Although this condition doesn't prevent from \emph{believing} something that is, in actuality, false, it prevents us \emph{knowing} something that is false. To know something means that that something is true;
%\end{itemize}
%
%\begin{question}[subtitle={Activity: Understanding the effort needed in this stage}]
%Do we want to explore this concept more deeply?
%\end{question}
%
%It's a pretty neat definition that stems from ancient Greece. Unfortunately, it has more recently been argued that it's incorrect. 
%
%The Gettier problem~\parencite{gettier1963is-justified}, for instance, challenges this traditional definition of knowledge. The Gettier problem shows that even if a belief is justified and true, it can still fail to qualify as knowledge. This problem has led to debates about whether the justification condition is sufficient for knowledge, and some have proposed adding additional conditions to the analysis. However, there is also skepticism about whether any analysis of knowledge can fully capture its complexity and nuances. 
%
%This raises real problems for the naive researcher and justifies the need for a lot of additional argument to establish that we have new knowledge.
%
%\begin{question}[subtitle={Optional activity: The Gettier Problem}]
%\textcite{gettier1963is-justified} is only 3 pages long, give it a read and try to understand the examples Gettier uses to question the traditional definition of knowledge.
%\end{question}
%
%The result is that we must be careful to ensure that what we claim to be new knowledge is actually both \emph{new} and \emph{knowledge}.
%
%Against this background of the meaning of knowledge, part of your mission\footnote{...should you choose to accept it.} is to justify your claim to have made a contribution to knowledge.  Given the traditional view of knowledge, for new knowledge you must convince your reader that:
%%
%\begin{enumerate}
%\item the thing you claim to be knowledge is actually true;
%\item that you believe it to be true,
%\item you have a solid justification for believing it,
%\item it hasn't been claimed to be true before.
%\end{enumerate}
%%
%
%Working backwards, the literature review
%
%Let's look at the belief condition first: do you believe your contribution to knowledge to be true. You've expended a great deal of effort in learning, understanding, and writing in convincing yourself that what you have created is believable, and your investment is testimony to the fact that you believe it to be true. Therefore your dissertation is evidence that your contribution to knowledge is something you believe in\todo{Needs to work in validity, reliability, bias and the other elements of this part}.
%
%The thing you ...
%
%You are justified in believing it ...
%
%\subsection{Literature review}
%
%As\todo{refer reader to previous section on this topic.} 
%
%\subsection{In detail: dealing with validity weaknesses}

%Reliability and validity\footnote{Adapted AI summary of \textcite{roberts2006reliability}.} are important – linked – concepts in research that serve to demonstrate the rigour and trustworthiness of both quantitative and qualitative research. Thus, researcher's planning of and reflection on reliability and validity and its reporting should be a part of all research strategies.
%
%Reliability and validity – and the effort needed by you to address them – varies greatly depending on the research strategy chosen. For logical proof, for instance, reliability and validity may be as simple as having a local community of logicians check your work, or even through the use of a computerised proof checker. At another extreme, in grounded theory, for instance, the treatment of reliability and validity will typically form large component of the reported research.
%
%\begin{tblr}{
%colspec={lXX},
%row{1}={font=\bfseries},
%}
%Threat&Meaning&Example\\
%History&An unrelated event influences the outcomes.&A week before the end of the study, all employees are told that there will be layoffs. The participants are stressed on the date of the post-test, and performance may suffer.\\
%Maturation&The outcomes of the study vary as a natural result of time.&Most participants are new to the job at the time of the pre-test. A month later, their productivity has improved as a result of time spent working in the position.\\
%Instrumentation&Different measures are used in pre-test and post-test phases.	&In the pre-test, productivity was measured for 15 minutes, while the post-test was over 30 minutes long.\\
%Testing&The pre-test influences the outcomes of the post-test.	&Participants showed higher productivity at the end of the study because the same test was administered. Due to familiarity, or awareness of the study’s purpose, many participants achieved high results.\\
%\end{tblr}


%\paragraph{Validity} refers to the extent to which a measure accurately represents the concept it claims to measure. It is concerned with the relevance and representativeness of the items or questions in a study. There are different levels of validity, including content validity, criterion-related validity, and construct validity. Content validity focuses on the relevance and representativeness of individual items, while criterion-related validity involves comparing a measure to other similar validated measures. Construct validity examines the underlying theoretical concepts and constructs being measured. Validity is important in ensuring that research findings are accurate and trustworthy.

%Validity\footnote{Adapted AI summary of \textcite{ihantola2011threats}.}, as already mentioned, refers to the extent to which a research study measures what it intends to measure and accurately reflects the concept or phenomenon under investigation. It is a crucial aspect of research quality and involves establishing the credibility, trustworthiness, and accuracy of the findings. 
%
%%Research is vulnerable to claims that it is not valid, and invalid research cannot be a contribution to knowledge. A good researcher will therefore consider each \textit{potential} vulnerability in turn and design their research so that it is robust should that potential vulnerability doesn't threaten their claimed knowledge contribution.
%
%The two main forms of validity are: \emph{internal} validity and \emph{external} validity, with each having a number of weaknesses.
%
%\subsubsection{Confusion reigns: validity} Although, or perhaps because, validity is an extremely important concept, its use as an analytic framework has spawned many different forms. \textcite{wortman1983evaluation}, for instance, say:
%%
%\blockquote{At the level of specific threats to validity, however, the sheer number is both overwhelming and somewhat confusing. Some threats seem a bit esoteric, especially for evaluators (e.g. \enquote{ambiguity about the direction of casual influence} and \enquote{hypothesis-guessing within experimental conditions,} for example); some seem to differ only in small degree (\enquote{compensatory rivalry by respondents receiving less desirable treatments} and \enquote{resentful demoral­ization of respondents receiving less desirable treatments,} for example); and still others (noted above) seem to be miscategorized, thereby blurring the differences among the major validity types. In addition, some of these threats are relevant during the design and planning for an evaluative study while others are more appropriate to the management and conduct of the study (e.g. multiple statistical testing, program implementation, diffusion, etc). Perhaps some consolidation is needed to make the whole structure less cumbersome and \enquote{threatening.}}
%
%Because of this, the treatment of validity below is partial, focussing on three main areas: experiment, observations, and instruments. It would be wise to discuss which expectations your supervisor has of validity\footnote{Schedule a meeting with your supervisor, with the subject \enquote{Validity discussion}.}.
%%
%
%%The validity of a piece of research can be evaluated in many different ways. Those we consider here cover the most often found, and include\footnote{Adapted from \textcite[and enclosing material]{bhandari2023construct}.} internal validity (including), external validity (including), construct validity, content validity, criterion validity, concurrent validity, discriminant validity, face validity, convergent validity, population validity, and predictive validity. 
%
%\paragraph{Internal validity} refers to the extent to which the research design and methodology accurately measure what they are intended to measure. For a conclusion to be internally valid, you need to be able to rule out other explanations (including control, extraneous, and confounding variables) for your results. 
%
%Internal validity applies across research strategies and research instruments, each involving the analysis of a phenomenon, having different interpretations according to each. The various areas of validity are illustrated in figure~\ref{fig:Validity}.
%
%\begin{figure}[hbtp]
%\centering{
%  \includegraphics[width=\textwidth]{figures/InternalValidityTypes}
%  \caption{Types of validity
%  \label{fig:Validity}
%  	}  
%  }
%\end{figure}
%
%\paragraph{Experiment} The forms of validity for experiments\footnote{The acronym for experimental internal validity is \emph{THIS MESS} \parencite{wortman1983evaluation}, hence the addition of \emph{statistical} to \emph{regression toward mean}. Subsequently, other authors have added further validity types; \textcite{cook1979quasi}, for instance, lists 33 potential weaknesses.}, i.e., research designed to establish a cause and effect relationhip, are:
%%
%\begin{itemize}
%\item testing: [if helpful, explanations needed throughout]
%\item history:
%\item instrument change:
%\item (statistical) regression towards mean:
%\item maturation:
%\item experimental mortality:
%\item selection:
%\item selection interaction:
%\end{itemize}
%%
%
%\paragraph{Observations} The forms of validity for observations are:
%%
%\begin{itemize}
%\item selective participation: [explanations needed throughout]
%\item selective recall: 
%\item accentuated perception:
%\end{itemize}
%%
%
%\paragraph{Instruments} The forms of validity for data generation instruments are \parencite[, adapted]{middleton2022the4types}:
%%
%\begin{itemize}
%\item Construct validity: Does the instrument measure the concept that it’s intended to measure?
%\item Content validity: Is the instrument fully representative of what it aims to measure?
%\item Face validity: Does the content of the instument appear to be suitable to its aims?
%\item Criterion validity: Do the results accurately measure the concrete outcome they are designed to measure?
%\end{itemize}
%%
%
%\paragraph{External validity} is the extent to which you can generalise the findings of a study to other situations, people, settings, and conditions \parencite{wortman1983evaluation}. In other words, external validity means the extent to which you can apply your knowledge contribution in a broader context\footnote{In qualitative studies, external validity is also referred to as \emph{transferability}, i.e., how transferable are results.}. 
%
%Although poor external validity may not disqualify your knowledge contribution as novel, it will require you to be extremely careful about its context of applications. Instead, for instance, of being able to say:
%%
%\blockquote{This result applies to all children of school age studying computing}
%%
%you may need to say
%%
%\blockquote{The result applies to all children between 12 and 13 years old, studying computing at the such-and-such a school with teacher A.}
%
%Clearly, the restricted nature of the new knowledge reduces its applicability and use as predictive, for instance.

%External validity breaks down further into population validity and ecological validity.

%%
%\begin{itemize}
%\item Population validity refers to whether you can reasonably generalise the findings from your sample to a larger group of people (the population). Population validity depends on the choice of population and on the extent to which the study sample mirrors that population. Population validity is established by showing that the sample and the population are similar.
%
%\item Ecological validity refers to the extent to which the measures measures how generalisable experimental findings are to the real world, such as situations or settings typical of everyday life. Ecological validity is demonstrated by showing that the restricted research context sufficiently \enquote{mimicked} the real world.
%\end{itemize}
%

%
%\begin{itemize}
%\item Construct validity refers to the extent to which the measures used in the study accurately capture the underlying constructs or concepts of interest.
%
%\item Content validity refers to the extent to which the measures used in the study accurately capture the underlying constructs or concepts of interest.
%
%\item Criterion validity refers to the extent to which the measures used in the study accurately capture the underlying constructs or concepts of interest.
%
%\item Concurrent validity refers to the extent to which the measures used in the study accurately capture the underlying constructs or concepts of interest.
%
%\item Discriminant validity refers to the extent to which the measures used in the study accurately capture the underlying constructs or concepts of interest.
%
%\item Face validity refers to the extent to which the measures used in the study accurately capture the underlying constructs or concepts of interest.
%
%\item Convergent validity refers to the extent to which the measures used in the study accurately capture the underlying constructs or concepts of interest.
%
%\item Predictive validity refers to the extent to which the measures used in the study accurately capture the underlying constructs or concepts of interest.
%\end{itemize}
%%
%



%Validity\footnote{Adapted AI summary of \textcite{wolming2010the-concept}.} is a concept that has evolved over time and has become more complex. Initially, validity was considered to be a fixed property of a test, with adequate correlations between test scores and an external criterion being seen as evidence of validity. However, the definition of validity has now expanded to include different types of validity related to the purpose of the test. These types include \emph{content validity}, \emph{criterion-related validity}, and \emph{construct validity}. 
%
%\paragraph{Content validity} is used for tests that describe an individual researcher's performance on a defined subject and is, for instance, concerned with the relevance and representativeness of items in a questionnaire.
%
%\paragraph{Criterion-related validity} is used for tests that predict future performance. It is established when a research tool can be compared to other similar validated measures for instance, when the results of two questionnaires deliver by two independent samples agree\footnote{If not exactly, then at least within statistical bounds.}.
% 
%\paragraph{Construct validity} is used to make inferences about psychological traits.


\section{Approaches to address weaknesses}

In this section, we consider three common approaches used to address weaknesses in research.

\subsection{Triangulation}


Triangulation \textcite{mathison1988triangulate} consists of using multiple data sources and methods, or even multiple researchers, to develop a comprehensive understanding of a phenomena under study and arrive at a particular conclusion about that phenomenon. Triangulation was introduced in the social sciences in the mid 1950s \textcite{campbell1959convergent}, and since has become an accepted approach across all disciplines, regardless of research paradigm.

The core idea behind triangulation is that if different data and methods converge towards the same conclusion, then it is more likely that such a conclusion is valid, that rival explanations can be dismissed, that the different procedures followed are reliable, and that the effect of any bias is mitigated. In this way, triangulation makes your research more credible, and your claim more defensible. 

However, because triangulation applies many techniques or derives conclusions from many sources, it can result in inconsistent or contradictory findings. So, it is important to understand that triangulation does not necessarily guarantee convergence on a single proposition about a phenomenon. Instead, it provides a rich and complex picture that requires careful interpretation and explanation by the researcher. As a result, triangulation should be used cautiously and researchers should be prepared to explain and make sense of the various outcomes it may produce. Triangulation also adds complexity and requires more time and effort that must be accounted for.

Main kinds of triangulation include \textcite{denzin1978research,patton1999enhancing}:

\paragraph{Data source triangulation} refers simply to using several data sources. These may be the inclusion of multiple participants to interview, or the consideration of a particular phenomenon under different conditions in space and time. For example, in an educational setting, you may wish to measure the efficacy of an educational programme on different student cohorts, possibly over different academic years, or delivered by different educators. With data triangulation you increase the validity of your claim across different contexts, so that your results are more generalisable.

\paragraph{Investigator triangulation} involves several researchers collecting and analysing data\footnote{Because there is more than one researcher involved, it is unlikely that you will be required to perform this form of triangulation in your Masters project. You may, however, be a researcher in the triangulation of another's researcher – your supervisor, for instance – which means that you should be prepared to be involved. Be sure to schedule some time with your supervisor to discuss their needs, should this be the case.}. For instance, you may have different researchers repeating measurements using the same lab equipment and procedures. The involvement of different researchers who independently apply the same techniques to arrive at the same conclusions, increases both reliability and validity of those outcomes, and mitigates against each researcher's bias.  This is particularly important in qualitative research where data are often interpreted rather than measured precisely.

\paragraph{Methodological triangulation} refers to the use of multiple methods in the examination of a phenomenon\footnote{We deal with mixed method research later in this Stage.}. For instance, a neuropsychologist may combine direct observation of human behaviour with neurological data from brain scans to obtain a comprehensive picture of what motivates people to make certain choices. Methodological triangulation allows strengths and weaknesses of different methods to compensate for each other, increasing both reliability and validity. However, it may be difficult to combine results from different methods because of their differing ontological and epistemic stance.

\paragraph{Theory triangulation} refers to the use of different theories or hypotheses to analyse data and interpret phenomena. For instance different motivation theories could be used to study resistance to change in organisations. By employing several theories, findings can be considered from different angles, compensating for possible limitations or biases of each individual theory. 

\begin{question}[subtitle={Activity: Distinguishing different kinds of triangulation}]
Consider each of the following examples and indicate which kind of triangulation they represent:
\begin{itemize}
	\item research on student experience in a university looking at student survey data and students' study results
	\item research on study practice and academic performance, combining an online survey and interviews with a selected number of participants
	\item research on sleeping patterns of the elderly, using data from care homes in the UK
	\item research on volcanos asking vulcanologists around the world to contribute seismological measurements over a period of time. 
\end{itemize}
\begin{solution}
These are example of, respectively:
	\begin{itemize}
	\item data triangulation, in which two different kinds of data are considered  
	\item methodological triangulation, in which two different methods are applied 
	\item data triangulation, in which similar data from different locations are considered
	\item investigator triangulation, in which several researchers are invited to collect and contribute data. Presumably, this also encompasses some data triangulation in the sense that similar data from different locations around the world are collected and analysed.
\end{itemize}
\end{solution}
\end{question}



%Researchers use various methods, such as pilot studies and comparison to other validated measures, to establish reliability and validity. Computerised data analysis packages can also enhance reliability. It is important to balance standardisation with maintaining the context and meaning of the data.

%\subsubsection{Evaluation of Reliability} ...
%
%% Messick's model introduced the idea of social consequences of measurement outcomes as an important aspect of validity. The introduction of social consequences raised debates about the feasibility and inclusion of consequences in the validity framework. Kane's argumentative approach to validity emphasises the practical feasibility and provides guidance on how to allocate research efforts and gauge the validation process. The integration of validity evidence and the need for guidance on prioritising validity questions are also discussed.
%%

%\begin{question}[subtitle={Activity: More on triangulation}]
%Read these sources on triangulation
%\end{question}

\subsection{Reflexivity}

According to \cite{jamieson2023reflexivity}:

\begin{quotation}
	Reflexivity is the act of examining one's own assumption, belief, and judgement systems, and thinking carefully and critically about how these influence the research process. The practice of reflexivity confronts and questions who we are as researchers and how this guides our work.
\end{quotation}

So, reflexivity admits that the researcher isn't an objective, unbiased observer of truth, but someone whose worldviews and subjectivity influences every step of the research process. Through reflexive practice, the researcher can then engage in a more honest and transparent research process, increasing research reliability and mitigating bias.

Note that there is a difference between reflection and reflexivity. Reflection is usually done retrospectively: you could reflect on something that has happened during your study to identify important lessons for the future. In contrast, reflexivity takes place throughout the research process --- before, during and after, hence has the potential to shape it. Also reflection focuses on things you  have done, while reflexivity explores motivations  --- your assumptions, beliefs, biases, etc., behind those actions.

Reflexivity is relevant and applicable to all types of research. Qualitative research has the longest tradition of reflexivity, with qualitative researchers encouraged to examine and openly acknowledge their own beliefs and biases, and their impact on the research. In quantitative research, the acceptance of the importance of reflexivity is growing, and goes alongside an acknowledgement that there are limitations and biases in the scientific method too, so that quantitative research is not a `gold standard' of objectivity.

Reflexivity should be embedded in all steps of the research process. In the early stages, it can apply to the choice of research problem or questions, by guiding the researcher to consider explicitly subjective factors which may explain why that particular choice was made and why the researcher is best placed to research it. In data generation, reflexivity can expose biases and unchecked assumptions which may affect how samples and data sources are selected or participants are recruited. In data analysis and their interpretation, reflexivity may lead to uncover reasons why certain evidence is given more weight or meaning, while other is discarded, for instance due to confirmation bias. In formulating conclusions, reflexivity can support ``thinking about thinking"\footnote{So-called `meta-cognition'}: the process of questioning the way we think to assess how valid and reliable our conclusions are. This is particularly important because while the human brain has the potential for logic and critical thinking, these are not innate skills: rather they need developing, akin to the skills that one must develop to become, say, a proficient musician or mathematician. Psychologists have uncovered that left untrained, our brain tends to make mistakes, which stem from a variety of factors\footnote{A fascinating series of lectures on this topic is "Your Deceptive Mind: A Scientific Guide to Critical Thinking Skills" by Steven Novella.}, including errors in perceptions, flawed memories, heuristic thinking, logical fallacies and cognitive bias. Reflexivity can help us become aware of these tendencies. 

 \begin{question}[subtitle={Activity: Reflexivity practices}]
Conduct a web search on reflexivity practices adopted by researchers. Briefly summarise what they are, and how they are useful. Comment on which of such practices you could adopt in your work.
\begin{solution}
You may have found some or all of the following:
\begin{itemize}
	\item Reflexive writing, such as research journals, diaries, fields notes and memos. These are common tools used by the reflexive researcher at any point in the research process to record assumptions, experiences, observations, perceptions, procedures, and decision points. They are used to bring into focus the researcher's intention and gaps in their knowledge or thinking, as well as interpersonal dynamics, including power ones.
	\item Positionality statement. This is a kind of reflexive writing aimed at describing a KES: Book
Scheduled: 5 Jun 2024 at 09:30 to 12:00, BST
researcher's characteristics (such as age, social class, race, etc.) and beliefs (such as political, philosophical, etc.) which may influence the research.
	\item Narrative autobiography. This is also a kind of reflexive writing focussed on the researcher's life experiences and motivations which may influence the research, particularly the researcher's interaction with participants and understanding of participants' accounts. The aim is to better prepare the researcher for their interaction with participants, so it is best conducted when planning data collection/generation.
	\item Reader-response exercise. This addresses how the researcher's own assumptions and experiences may affect their interpretation of participants' accounts. It consists of including a layer of codes to indicate how the researcher reacts to and interprets participants' accounts in relation to their own background and personal history. As such, this practice is useful during data analysis and interpretation.
	\item Collaborative reflexivity. This entails engaging in reflexivity as part of a research team, with collaborators questioning assumptions and decisions. It assumes mutual trust, and a commitment to ethics and rigorous research, regardless of seniority or status. It applies to all stages of the research process.
\end{itemize}
\end{solution}
\end{question}


\subsection{Returning to the literature}

Addressing novelty weaknesses means returning to your literature review as your research progresses and understanding increases to cast an increasingly critical eye over it, and possibly widen its scope to further related work which may have been published more recently. Each source should be reconsidered for what you thought it originally said and what you now think it says, using any difference\footnote{In the best case, there will, of course, be no difference!} to drive further reflection on your findings, methods, data generation, or even research problem. This process will help you both ensure there continues to be a gap your research can contribute to, and assess the extent of the novel contribution your research can make in relation to related work which has already been published.

As in the example we included at the end of Section~\ref{sect:defendingYourClaim}, while defending your claim or explaining your research design, your reader can be made aware of this process and how it has altered your research. Deepening the critical nature of your literature review allows your reader to understand that you are a reflective researcher and can turn any novelty weakness into a research strength!

%LR -- section commented out (too much of a collection of lists at present); to  consider whether content can be factored into the research strategies instead}
%\subsection{Weakness types  --- this section may be omitted; to decide}
%\todo{this section is either to develop or to remove; to rethink after the research strategy bit is done} In this section we recall many of the common weaknesses under each of the categories we have introduced. All research projects are subject to some or even most of them, so awareness of them will help you better inform your choice of research strategies and the step to take to address them. 
%
%\subsubsection{Novelty weakness}
%
%To identify your research problem, you will have found a hole in current knowledge through the literature you will have surveyed and reflected critically upon. During subsequent research you may find that you were:
%%
%\begin{itemize}
%\item unable to contribute knowledge in that area, or not able to contribute as much as you had initially hoped;
%\item found further sources that had already made a contribution.
%\end{itemize}
%%
%If you do encounter this weakness, you will not be the first – virtually all researchers find that their initial aspirations for a knowledge contribution has to be reduced or altered as their research – and understanding – progresses.
%
%
%
%\subsubsection{Validity weaknesses}
%
%%\subsection{Dealing with validity weaknesses}
%%%[Distinguish and add Validation of evaluation...]
%
%There are a number of recognised validity weaknesses\footnote{These are often called \enquote{threats to validity} in the literature. We prefer \emph{weakness} as it suggests that there is an issue caused by the research strategy application rather than by the environment.} upon which an evaluation of research validity is made. \todo{why this list? where does it come from?}
%%
%\begin{itemize}
%\item Mismatch between quantitative and qualitative samples
%\item Imbalance between an insider's and outsider's views
%\item Insufficient knowledge of research question, theory, hypotheses, statistical tests, and analysis
%\item Occurrence of unrelated events or conditions during data collection
%\item Insufficient or biased knowledge of earlier studies and theories
%\item Lack of descriptive validity of settings and events during data analysis and interpretation
%\item Population, time, and environmental validity in quantitative research
%\item Lack of cognitive and empathy training of researchers
%\item Value or ideologically based conflicts in collaboration between quantitative and qualitative researchers
%\item Difficulty in persuading consumers to value the meta-inferences from both qualitative and quantitative findings.
%\end{itemize}
%%
%
%\subsubsection{Reliability weaknesses} refers to the trustworthiness and consistency of the procedures and data generated in research. It is concerned with the extent to which the results of a study or measure are repeatable in different circumstances. Reliability can be demonstrated through methods such as inter-rater reliability, test-retest reliability, and internal consistency.
%
%\subsubsection{Bias}
%
%[Adapted from \cite{simundic2013bias}]
%
%\textcquote{simundic2013bias}{%
%Bias is any trend or deviation from the truth in data collection, data analysis, interpretation and publication which can cause false conclusions. Bias can occur either intentionally or unintentionally (1). Intention to introduce bias into someone’s research is immoral. Nevertheless, considering the possible consequences of a biased research, it is almost equally irresponsible to conduct and publish a biased research unintentionally.
%
%It is worth pointing out that every study has its confounding variables and limitations. Confounding effect cannot be completely avoided. Every scientist should therefore be aware of all potential sources of bias and undertake all possible actions to reduce and minimise the deviation from the truth. If deviation is still present, authors should confess it in their articles by declaring the known limitations of their work.
%%
%%It is also the responsibility of editors and reviewers to detect any potential bias. If such bias exists, it is up to the editor to decide whether the bias has an important effect on the study conclusions. If that is the case, such articles need to be rejected for publication, because its conclusions are not valid.
%}
%
%\textcite{simundic2013bias} then goes onto detailed four forms of bias: 
%%
%\begin{itemize}
%\item data collection bias: including selection bias, volunteer bias, admission bias, survivor bias, and misclassification bias
%
%\item data analysis bias: including%
%	\begin{itemize}
%	\item data fabrication: reporting non-existing data from experiments that were never done;
%	\item data elimination: eliminating data which do not support a hypothesis (outliers, or even whole subgroups);
%	\item using inappropriate statistical tests to test your data;
%	\item performing multiple testing (\enquote{fishing for P}) by pair-wise comparisons, testing multiple end-points and performing secondary or subgroup analyses, which were not part of the original plan in order \enquote{to find} statistically significant differences regardless of hypothesis.
%	\end{itemize}
%%
%
%\item data interpretation bias: including
%\begin{itemize}
%\item discussing observed differences and associations even if they are not statistically significant (the often used expression is \enquote{borderline significance});
%\item discussing differences which are statistically significant but are not otherwise meaningful;
%\item drawing conclusions about causality, even if the study was not designed as an experiment;
%\item drawing conclusions about values outside the range of observed data (extrapolation);
%\item over-generalisation of the study conclusions to the entire general population, even if a study was confined to the population subset;
%\item Type I (the expected effect is found significant, when actually there is none) and type II (the expected effect is not found significant, when it is actually present) errors.
%\end{itemize}
%
%\item publication bias: including
%%
%\begin{itemize}
%\item funding bias: due to the prevailing number of studies funded by the same company, related to the same scientific question and supporting the interests of the sponsoring company
%\item anti-negative bias: scientific journals are much more likely to accept for publication a study which reports some positive than a study with negative findings. Such behaviour creates false impression in the literature and may cause long-term consequences to the entire scientific community. Also, if negative results would not have so many difficulties to get published, other scientists would not unnecessarily waste their time and financial re- sources by re-running the same experiments.
%\end{itemize}
%\end{itemize}
% LR -- end of commented out section



\chapter{Your research strategy candidate list}\label{sect:standardResearchStrategies}

While your own research strategy will be specific and unique to your project in the way it informs the research you will conduct, standard research strategies have emerged over time, influenced by research paradigms and research practice within specific disciplines. Each of them can be seen as a sort of `recipe' which summarises common ways to conduct academic research: by adopting or combining some of these strategies, you can come up with your own specific instance for your project. 

There are many standard research strategies in the literature, often with many variants: the 12 strategies we consider in this book are discussed in this section. The outcome of working through this section should be your choice of a candidate research strategy that:
%
\begin{itemize}
\item is a good fit for your research problem, i.e., that will allow you to develop a contribution to knowledge arising from your research problem
\item makes the most of your current research skills and resources, i.e., the background knowledge and skills you bring to the research, the time that is available to you, and it fits with your research context.
\item can be evaluated through a list of questions that could be asked of it by a knowledgeable evaluator, such as an examiner.
\end{itemize}
%
From the first two of these, you will gain an understanding of which steps you will be required to take to generate, analyse and interpret research data that, when complete, will make your contribution to knowledge. From the third of these, you'll be able to structure your research report – your dissertation – by describing your answers to the evaluative questions.

The 12 candidate research strategies we consider are listed in Table~\ref{tab:ResearchStrategyChoice}. For each, after a brief description explaining the focus of the strategy, we will:

\todo{compare to structure of each strategy section, at the end of editing}
%
\begin{itemize}
\item describe what kind of knowledge contribution that can be made through it
%\item describes its focus\footnote{meaning?}
%\item describe any variants that exist and the choices that constitute them
\item describe the ways in which data is generated and analysed within the strategy
\item describe how a contribution to knowledge using the strategy will be evaluated
\item ask \enquote{Is this strategy right for me?}
\item provide a number of references that give more detail, if you are seriously considering the strategy
\end{itemize}
%

\newcommand{\tick}{$\square$}
\begin{table}
\caption{Research strategy choice\label{tab:ResearchStrategyChoice}}
\begin{tabulary}{\tablewidth}{rccl}
	\textbf{Research Strategy candidate}&\textbf{Considered}&\textbf{Excluded}&\textbf{Reason excluded}\\\toprule
	Survey&\tick&\tick\\
	Design and Creation&\tick&\tick\\
	Experiment&\tick&\tick\\
	Case study&\tick&\tick\\
	Action research&\tick&\tick\\
	Ethnography&\tick&\tick\\
	Systematic research&\tick&\tick\\
	Grounded theory&\tick&\tick\\
	Phenomenology&\tick&\tick\\
	Simulation&\tick&\tick\\
	Mathematical and logical proof&\tick&\tick\\
	Mixed methods&\tick&\tick
\end{tabulary}
%%Hack to correct tcbox behaviour
\end{table}


This is a lot to digest! Rather than going through all the information about each strategy in turn, we recommend you take the following steps to first reduce your list of candidate strategies from which to arrive at your chosen one. 

\paragraph{Step 1} Consider a strategy, and read its description and type of knowledge contribution that can be made through it. Compare these with your research problem to check whether that research strategy should be a candidate for your project. When you have done this, you should check its tickbox in the first column of Table~\ref{tab:ResearchStrategyChoice} – I've considered the strategy. If there's a clear mismatch with your research problem, you should check the tickbox in the second column – that the research strategy has been excluded – and give a reason why you have excluded it – say, the knowledge contribution it makes is not of the correct form – and you can move onto the next research strategy and repeat this step. The \enquote{Reason excluded} column will be used in the your dissertation to justify your choice of research strategy so think deeply about what you write here – you can use the text of the knowledge contribution and subsequent subsections to frame your reason for excluding it. Whatever you do, don't leave it blank!

\paragraph{Step 2} If you have not excluded the research strategy, then you should read further – next come the methods you would use to generate and analyse data. This gives you another reason to exclude a research strategy – that the data or participants your research needs are not accessible or the methods are not feasible within your project\footnote{Of course, you will need to choose \emph{one} research strategy, so be careful not to exclude something that, perhaps with some adjustment, can be made to work.}. If this analysis leads you to exclude the research strategy, complete\footnote{This time, the reason will be something to do with data generation techniques not accessible.} the second tickbox column and record the reason, then  move on to the next research strategy going back to Step 1.

\paragraph{Step 3} If you have not been able to exclude the research strategy, then you should read the \enquote{Evaluation} section\footnote{Perhaps taking notes on things you haven't immediately understood.}. These are questions you should be able to address once your research is completed, and which you should keep in mind from the very start. If you feel you are unlikely to be able to address them, then this gives you another reason to reject the strategy. Once again, if you have excluded the strategy, tick the box in column 2, record the reason in column 3, then move on to the next strategy and restart the process.

\paragraph{Step 4}  If you have not been able to exclude the research strategy, it's time to look at \enquote{Is this strategy right for me} section. This lists a number of other things you should consider that might lead you to exclude it, particularly in relation to skills you may need, or other features of the strategy which may not align with what you can achieve within your project. If you came to reject the strategy, as before, tick the box in column 2, record the reason in column 3, then move on to the next strategy and restart the process. This section may include alternative strategies you could consider next, otherwise, just proceed through the list.

\paragraph{Step 5} If you have not been able to exclude the research strategy, then look at the \enquote{Further reading} section and record the suggested references in your bibliographical database. You will access these articles later on, to gain a deeper understanding of your candidate research strategies. You can now move on to the next strategy, and restart the process.

We have wrapped up this process in the following activity, which constitutes the most substantial practical work for you to carry on in this sub-section:

\begin{question}[subtitle={Activity: Arriving at your candidate strategy list}]
Copy Table~\ref{tab:ResearchStrategyChoice}	to your favourite word processor or spreadsheet application. Apply the process above until you have considered all the strategies, updating your table as you go along, and recording related references in your bibliographical database. 
\begin{guidance}
The aim of this activity is to help you narrow down the possible choices of candidate strategy for your project, without having to dive deep into the detail of all 12 strategies presented. This is something you will do after you have completed this activity: the references recorded in your database will then provide a starting point for your review of methodology-related literature.
\end{guidance}
\end{question}

Once you have exhausted all the strategies, there are three possible outcomes:
%
\begin{itemize}
\item you find yourself with a single candidate research strategy, in which case you should go for it!

\item you find yourself with a number of candidate research strategies, in which case you need to study more in order to make a choice. You may also like to think about mixing up bits of each to give you your own mixed methods research strategy.

\item you find yourself without a choice, in which case you've probably been too picky... and you should try again – you can't do research without a research strategy and you're unlikely to come up with one not on this list – a completely novel one.
\end{itemize}
%

In all cases, you should discuss the outcome with your supervisor:

\begin{question}[subtitle={Activity: Discussing your choice with your supervisor}]
Arrange a time to talk to your supervisor about the process you have followed to identify possible choices of research strategy for your project, and what the outcome was.	
\begin{guidance}
As an expert in the research process and in your field of study, your supervisor will be able to advise on whether the choices you have made are appropriate, or even recommend strategies you should consider in details.
\end{guidance}
\end{question}

\section{Survey research}

Survey research focuses on collecting, in a standardised and systematic fashion, up-to-date, real-world data from a sample\footnote{A sample is a subset of data from a population of interest. We will return to sampling in Stage 4.} of the population which is the focus of your research. Depending on the population and selected sample, large amounts of data may be collected.

\subsection{Knowledge contribution}

The contribution to knowledge of survey research is to uncover patterns that can be generalised from the sample to the target population.

A typical application of survey research is to predict the outcome of an upcoming election by polling data from a sample of voters.

%\subsection{Variants}

\subsection{Data generation and analysis} 

For your data generation, you need to identify upfront which data you will collect in a standardised matter, your target population and sample. The sample must be representative of the population in the sense that it should reflect accurately population characteristics.

Suggested by the name, a survey – a standardised set of questions administered to a number of respondents – allows the researcher to gather information about a population. Surveys can take many forms, from interviews to questionnaires to focus groups, but authors vary on what they consider appropriate\footnote{Be sure to consider any supplied preparatory reading on the survey research strategy to ensure that you meet your supervisor's (or other's) expectations of what will be appropriate.}. They can be administered via the internet (more traditionally by mail), over the phone, or even face-to-face. Mixed-mode surveys combine these options into more complex instruments, perhaps using a broader but simpler questionnaire to identify potential participants for a deeper face-to-face interview to follow.

In your data analysis, you seek patterns in the sample data collected to arrive at generalisations to the wider population. Statistical analysis is usually applied, possibly complemented by some thematic analysis, if open-ended questions are also included to elicit qualitative data.


\subsection{Evaluation} 

The following questions are typically asked of survey research:

	\begin{enumerate}
	\item Reliability: 
		\begin{itemize}
		\item Are the sampling frame\footnote{The sampling frame is the set of individual units of the population from which the sample is drawn. Such individual units may be participants or data points in a data set.} and sampling techniques\footnote{We will look at sampling in Stage 4.} used adequately explained?
		\item Are the data generation and analysis methods adequately described?
		\item Do the survey questions allow for consistent and dependable measures by different respondents? 
		\item Are significant differences between respondents and non-respondents discussed?
		\end{itemize}
	\item Validity: 
		\begin{itemize} 
		\item Is the sampling frame appropriate? Does it provide sufficient coverage of the target population in terms of its characteristics of interest?
		\item Is the sample representative? \footnote{This relates to the question of whether the sample is sufficiently large and/or as diverse as the population.}
		\item Is the response rate adequate? How were non-respondents handled?
		\item Do they survey questions allow to measure or assess all that is needed? 
		\item Has statistical, or other, analysis been appropriately applied?
		\item Are generalisations made about the target population appropriate? What reasoning chains have led to such generalisations?
		\end{itemize}
	\item Bias: 
		\begin{itemize} 
		\item Is the questionnaire designed as to avoid leading questions, which may have an unduly influence on the respondents? 	
		\end{itemize}

	%\item Are limitations or omissions in relation to any of the points above identified? Have their effect on the research and its outcomes been discussed?
	%\item Has the research strategy been successful in relation to the research aim and objectives	
	\end{enumerate}

\subsection{Is this strategy right for me?}

%\endinput

This strategy sets certain requirements of the researcher for them to be successful. These include that:
	\begin{enumerate}
	\item you must have access to an appropriate population sample, so that a sufficient volume of data can be collected and deep analysis performed. If this is not possible, for instance, because you have limited access to the population, you might like to consider case study research instead.
	\item the phenomena and characteristics of the population which are of interest should be measurable through questions asked through a survey. If this is not the case that then you're not going to be able to make a contribution to knowledge about those phenomena or characteristics, and you might like to consider phenomena that can be measured, or a different population for which those phenomena can be measured.
	\item while this strategy may produce lots of data in a relatively short time, the depth in the data can sometimes be lacking, given the focus on what can be measured. If deeper or more nuanced data is needed, then you may like to consider case study research instead.
	\item conducting a survey means that you'll be analysing phenomena using point data, i.e., data that were collected at a point in time – that time at which the survey was answered. If your research requires longitudinal data, i.e., data that could change over time, then survey research becomes more difficult as you might need two or more surveys to collect the changing data. While it's not impossible to do this, it adds many complications: earlier participants might not be available for later surveys, their mindsets might have changed in the intervening period, etc. If this is the case, then you should consider whether the choice of phenomena is appropriate. Alternatively, you might like to consider one of the experimental research strategies described below.
	\item surveys are not suitable to investigate the mechanisms behind cause and effect relationships, for which you should use an experimental research strategy instead. 
	\item conclusions from survey research rely on the veracity of the responses received, something you can't necessarily take for granted. Even when there is no intention to deceive, people's answers may be inaccurate due to many factors, including a tendency to wishing to provide the `right answers', that is what they may believe is expected, or poor recall of past events or of detailed observations they have made, or even lack of trust which may influence what they are willing to disclose. Triangulation, therefore, may be required to increase validity, but this will add complexity to the strategy. If this is not possible, then other strategies may be advisable, for instance, participant observation through ethnographic research.
	\end{enumerate}
	
\subsection{Further reading}	
To deepen your understanding of this strategy, you can start from \cite{dillman2014internet,oates2008researching,johannesson2014research,kalaian2008encyclopedia}.

\newcommand{\RSActivity}[2]{\begin{question}[subtitle={Activity: Considering #1}]
	Having read the above section, do you consider #1 to be a serious candidate for your research strategy?
%	
\begin{guidance}
	If so, add the following references to your list of reading:
\resourcelistcite{#2}
	\end{guidance}
	\end{question}
%%Hack to correct tcbox behaviour
\color{black}}

%\RSActivity{Survey Research}{dillman2014internet,
%	%oates2008researching,
%	%johannesson2014research,
%	kalaian2008encyclopedia}

\section{Design science research}

The design science research strategy\footnote{AKA Design and creation strategy.} focuses on developing novel solutions to problems, a problem being a need in context. The solution should be an artefact, by which is meant anything designed and constructed by humans: this is a very broad definition, encompassing all that does not exist in nature, including any artificial object, construct, process, policy, model, method, etc.

\subsection{Knowledge contribution}

The contribution to knowledge is that which can be learned from the design and creation of the artefact as the solution to a problem. Knowledge contributions therefore come from an exploration of the problem, of the artefact itself, and its design, development, use, or other characteristics of the real-world problem solving process -- for instance, whether it is linear or iterative,  or the ways in which problem and solution understanding and validation are conducted.

This strategy leads to tangible artefacts which fit real-world contexts, and is particularly suited to emerging and rapidly changing technology-related fields of study, where new problems emerge all the time and known solutions are sparse or become rapidly obsolete, hence necessitating continuous innovation. Lots of research in Computing is an expression of this strategy, for instance designing computational systems able to emulate human cognition, as is the case of AI\footnote{Artificial Intelligence}.

%\subsection{Focus}

%\subsection{Variants}

\subsection{Data generation and analysis}

Data generation is through the problem-solving process of articulating the problem, and designing and constructing the solution artefact, with the interactions between actors (customers, clients, designers, others), technologies and/or knowledge as the source of data. Modelling methods are widely applied, possibly informed by data generation methids, like reviewing existing documents or interviews with stakeholders and experts or observation of people's behaviour. Prototyping is often used to produce proof-of-concept artefacts to test, demonstrate and improve the design. 

Data analysis focuses on knowledge generated in the evaluation of both problem and artefact, including solution characteristics in relation to the extent they address the problem -- the identified need in context. Specific evaluation techniques will depend on the nature of the artefact, and may include problem owner\footnote{By problem owner we mean the person or people who have expressed the need to be addressed and are able to establish whether the solution has met it.}'s validation, various forms of testing, or end-users' evaluation and feedback.

\subsection{Evaluation}

Evaluation of the design and creation research strategy typically consists of the following questions:
					%
\begin{enumerate}
	\item Novelty: 
		\begin{itemize}
			\item What is the novelty in the artefact, its design, development, and/or creation?
			\item To which extent does the artefact address the problem? Have its efficacy and utility been demonstrated? What evidence is provided?
		\end{itemize}
	\item Reliability: 
		\begin{itemize}
			\item Are all stages of the problem solving process discussed, including interactions with stakeholders?
			\item Are the ways data are generated and analysed, both in problem and solution space, adequately described?
		\end{itemize}
	\item Validity: 
		\begin{itemize}
			\item Are appropriate approaches applied in the design and creation of the artefact?
			\item How is the artefact assessed? Are the assessment criteria appropriate and documented? How were they determined?
			\item Which generalisations are made from the design and creation of the artefact? Are they appropriate?
		\end{itemize}
%	\item Are limitations or omissions in relation to any of the points above identified? Have their effect on the research and its outcomes been discussed?
%
%	\item Has the research strategy been successful in relation to the research aim and objectives?
	\end{enumerate}
	

\subsection{Is this strategy right for me?}

%\endinput
For this strategy to be successful:

\begin{enumerate}
\item There must be demonstrable novelty. You must be able to argue that your research does not focus on `normal' design, that is you are not simply re-implementing a solution to a well-known problem through a well-known development process and well-practiced skills\footnote{Learning new skills may be valuable from a personal perspective, but will not, by itself, make a contribution to knowledge – learning them means that they exist already!}. If you cannot clearly identify that novelty, then you will not be able to claim a contribution to knowledge.
\item There should be a problem owner which is separate from the researcher, and who sets the requirements and context for the artefact, with the researcher working on its development for that context to meet those requirements. If you do not have access to a real-world problem owner then this strategy is not applicable.
\item If the problem owner is, say, your employer or a business you are collaborating with, and for which addressing the problem is a matter of urgency, then you must establish whether it is feasible for you to deliver a novel solution in a timely fashion. Research always brings a level of uncertainly so that estimating time to success, or if success is even possible may not be easy. If you can't ensure feasibility within the timescale of your project, then you may need to rethink the problem to address.
\end{enumerate} 


\subsection{Further reading}

To deepen your understanding of this strategy, you can start from \cite{oates2008researching,brocke2020introduction}.


%\RSActivity{Design and creation research strategy}{oates2008researching,brocke2020introduction}

\section{Experimental research}

Experimental research provides a controlled environment in which cause and effect relationships can be investigated, expressed as hypotheses\footnote{A hypothesis is a tentative statement about the relationship between the phenomena to be tested in the experiment.}. The strength of an experiment is that it can reduce the influence of confounding factors on a cause-effect relationship. 
%From wikipedia:

%\textcquote{enwiki:1195800578}{An experiment is a procedure carried out to support or refute a hypothesis, or determine the efficacy or likelihood of something previously untried. Experiments provide insight into cause-and-effect by demonstrating what outcome occurs when a particular factor is manipulated. Experiments vary greatly in goal and scale but always rely on repeatable procedure and logical analysis of the results.}

The potential scope of application of the experimental research strategy is wide, ranging from scientific experiments under laboratory conditions controlled by the researcher to field experiments involving people in a real-world setting in which some factors may be outside the control of the researcher. 

There are pros and cons of each. While laboratory experiments are very reliable due to the high level of control, they can be very artificial, with little or no relation to a real-world context. The opposite is true for field experiments.


\subsection{Knowledge contribution}

The experimental research strategy contributes to knowledge through allowing cause and effect relationships between real-world phenomena to be established. 

For instance, you may run an experiment to test whether the use of mobile phones just before going to sleep disrupts people's sleeping patterns.

%\subsection{Focus}
%
%\subsection{Variants}
%
%There are many variants of the experimental research strategy, including\footnote{Add context of application for each}:
%%
%	\begin{description}
% 	\item [True] see~\textcite[p.126]{oates2008researching}
%	\item [Quasi] see~\textcite[p.133]{oates2008researching}
%	\item [Uncontrolled] see~\textcite[p.134]{oates2008researching}
%	\end{description}
%	%
%\noindent and, even more specialised, for social science applications
%	%
%	\begin{description}
%	\item [One group, pre-test and post-test] see~\textcite[p.135]{oates2008researching}
%	\item [Static group comparison] see~\textcite[p.135]{oates2008researching}
%	\item [Pre-test/post-test control group] see~\textcite[p.135]{oates2008researching}
%	\item [Solomon four-group design] see~\textcite[p.126]{oates2008researching}
%	\item see \textcite{field2002design}
%	\end{description}
%	
%%\endinput
		

\subsection{Data generation and analysis} 

The experimental research strategy revolves around making an intervention within tightly controlled parameters. Observations and measurements are made of before and after the intervention and a comparison is made. Any difference is assumed associated with the intervention made. 

For instance, in establishing a causal relation between the use of mobile phones and sleeping patterns, we could investigate the effect of the blue light emitted by a mobile phone on reducing the production of melatonin: this is the hormone which controls a person's sleep-wake cycle, so that its reduction is likely to disrupt a person's sleeping pattern. We would then measure the amount of melatonin produced by the body (these are our measurements) with and without exposure to the blue light of a mobile phone (this is the intervention), then analyse any difference to establish whether a causal relation exists.

So, you generate data through observations and measurements under different experimental conditions, and analyse your experimental data to explain causal relationships between the factors under study. 

Depending on the complexity of the relationship between cause and effect, more or less complex experimental designs can be used. Those involving an inaccessibly large population of individuals, as might be the case for a medical drug trial, use sophisticated techniques to choose representative samples, as well as sophisticated statistical analysis to test hypotheses.

However, even simpler \enquote{local} cause-effect hypotheses may rely on the availability of a fully equipped scientific laboratory to work.


\subsection{Evaluation}

Typical questions in the evaluation of the experimental research include:
	
	\begin{enumerate}
	\item Reliability: 
		\begin{itemize}
			\item Are the experimental variables manipulated or measured adequately described?
			\item Is there a clear account of what is controlled?
			\item What are the experimental procedures? Are they sufficiently detailed so that the experiment can be repeated by an independent third party?
			\item In a social setting, what information is given about  participants and how they were found?
			\item What information is given about the apparatus and the process used to make measurements?
		\end{itemize}
	\item Validity: 
		\begin{itemize}
			\item Was a hypothesis or predicted outcome of the experiment clearly stated?
			\item If a population sample was selected for the experiment, how representative is it? How was it selected? Which measures were taken to avoid sample bias\footnote{Sample bias occurs when some elements of the population are more likely to be selected than others.}?
			\item If statistical analysis is applied, how adequate is it? Have appropriate statistical tools been used and their use justified?
			\item Are confounding factors or outliers identified and discussed?
			\item Are the statistical and other analyses convincing of the conclusions?
			\item Has the experiment being replicated?
		\end{itemize}
	%	\item Are limitations or omissions in relation to any of the points above identified? Have their effect on the research and its outcomes been discussed?
%	\item Has the research strategy been successful in relation to the research aim and objectives?
	\end{enumerate}

\subsection{Is this strategy right for me?}

Although widely applicable, the experimental research strategy has some counter-indications:
%
	\begin{enumerate}
	\item when testable hypothesis cannot be formulated, concerning the cause-and-effect relationships of interest
	\item when the cause/effect relationship is very complex, for instance, depending on many factors, which cannot be accounted for in an experiment 
	\item when confounding factors and variables cannot be isolated, or no level of control is possible
	\item when the experiment is a one-off and cannot be repeated
	\item if you don't have access to specialised equipment required
	\item if you don't have (or can't develop) statistical analysis skills required
	\end{enumerate}


\subsection{Further reading}

To deepen your understanding of this strategy, you can start from \cite{oates2008researching,
johannesson2014research,field2002design}.

%\RSActivity{experimental research strategy}{%oates2008researching,
%	%johannesson2014research,
%	field2002design}	

\section{Case study research}

Case study research proceeds through the in-depth study of a notable instance of a phenomenon within its real-world context, particularly when it not possible to separate the phenomenon from that context. The study of a single phenomenon requires the researcher to delve deeply into the context of that phenomenon, whether that be a project, an organisation, an engineered system, a policy, an economic or historical setting, or other. Case studies allow the researcher to study complex phenomena where several factors are at play, and to explore alternative meanings and explanations.  

\subsection{Knowledge contribution}

Case studies focus on the `how?' and `why?', so that the knowledge contribution is a detailed insightful account of the phenomenon in its natural context, including when appropriate its relationships with other phenomena, and relevant processes and causal chains. 

What you seek with a case study can span from exploring possible questions or hypotheses for follow-up research, to explaining why certain outcomes have occurred, to investigate changes over time. For instance, an example of case study could be a detailed investigation of the US Equifax social security breach of 2017, in which 143 million of their consumer records were stolen by hackers. This may be descriptive of the chain of events that took place or explicative of why things happened the way they did, or both.

Therefore, case studies come in many forms, including:

	\begin{itemize}
	\item exploratory: in which the researcher explores a research problem sufficiently to be able to conduct a further study. If you're considering studying for a PhD after your Masters research, then this might provide a head-start for your future research
	\item  multiple: in which two or more instances of the phenomenon are chosen, which present both similarities and differences, to provide an even richer analysis of the phenomenon in its context 

	\item longitudinal: in which the researcher considers the state of a phenomenon over time. This offers a natural storytelling context in which change in the phenomenon and/or its context can be analysed.
\end{itemize}

Combinations of the above are also often adopted, allowing even deeper exploration of both relationships between phenomena and how they develop over time or in response to contextual factors.
		
%and have many possible foci \parencite[p.44, adapted]{johannesson2014research}:
%	\begin{itemize} 
%	\item Focus on One Instance: in which a single  The idea is \enquote{To see a World in a Grain of Sand, And a Heaven in a Wild flower, Hold Infinity in the palm of your hand, And Eternity in an hour} as expressed by William Blake (2012).
%	
%	\item Focus on Depth. As much information as possible about the instance studied should be obtained, without shying away from any details.
%	
%	\item Natural Setting. The instance exists before and independently of the research project, and it should be studied in its ordinary context; it should not be moved to, or created in, a laboratory.
%	
%	\item Relationships and Processes. The instance should be studied in a holistic way, taking into account all the relationships and processes within the instance as well as in its environment.
%	
%	\item Multiple Sources and Methods. Multiple information sources should be consulted in order to obtain rich, many-faceted knowledge about the instance; when doing this, different data collection methods could be used, such as interviews and observation.
%	\end{itemize}
					
\subsection{Data collection and analysis}

Case studies require you to collect empirical data from a great variety of sources, and to focus on depth rather than breadth. Therefore, all data collection techniques which allow you to do so may be used, from observation of the phenomena \textit{in situ} and the context and processes in which it participates, to surveys of those that experience the phenomena in context (through interviews, questionnaires, \textit{etc.}), allowing for multiple stakeholder views to be taken into account, to studying forensically existing documents that directly or indirectly describe the phenomena. This will lead to much data to be collected --- mainly qualitative, but also quantitative to some extent, so that their analysis can be very rich and complex.

\subsection{Evaluation}

An experienced researcher evaluating case study research will ask the following questions:

	\begin{enumerate}
	\item Reliability: 
		\begin{itemize}
			\item Has the type of case study conducted been clearly described and justified?
			\item How were ethical considerations taken into account, particularly in relation to participants and confidential information handled?
			\item Were the data generation and analysis methods adequately described?
			\item Are the procedures followed appropriately documented?
		\end{itemize}
	\item Validity: 
		\begin{itemize}	
			\item Have the criteria for choosing the particular case study been described and justified? Is the choice appropriate for the phenomenon studied?	
			\item Did the data generation methods generate the right type of data about the phenomenon in sufficient depth and quantity?
			\item How was a detailed investigation of the phenomena conducted? Was the researcher able to work within the case study context?
			\item Does the research adequately describe the relationships between phenomena and the processes in which the phenomena participate?
			\item Is the data analysis systematic and transparent? Are the steps taken to arrive at  conclusions clearly explained? 
			\item What generalisations were made from the case study research? Are they appropriate for the phenomenon and its context?
		\end{itemize}
	%\item What use of theory of the phenomenon is made in the case study? IS the theory chosen appropriate? If no theory was used, how is the theoretical basis of the research covered?
	
%	\item Are limitations or omissions in relation to any of the points above identified? Have their effect on the research and its outcomes been discussed?
%
%	\item Has the research strategy been successful in relation to the research aim and objectives?
	\end{enumerate}
	
%\endinput
					
\subsection{Is this strategy right for me?}

There are conditions for this strategy to be successful:

\begin{enumerate}
\item Case study research requires you to have access the phenomenon in its context to be able study it holistically and generate rich, detailed descriptions. As an example, if you're not a teacher, it might be difficult to gain access to a classroom to study student/teacher interactions. If access is an issue, then you should consider a different strategy, like systematic research reviews, which work from secondary sources.
\item Access to data sources, such as policy, processes or procedures within an organisation, may rely upon interaction with others. Even if you already have a good relationship with them they might not have the time to assist you sufficiently for your data generation to be successful within the timeline of your project. If time is an issue you should consider alternative sources, or even a different research strategy. 
\item Being embedded within the context of the phenomenon, as might be the case, for instance, of an employee of an organisation, facilitates the investigation of the phenomenon. In this case, however, alternative research strategies are also applicable, such as ethnography or action research.
\item You must have the required knowledge to understand the phenomena under study. For instance studying the processes by which an engine controller in an aircraft is designed may require a detailed understanding of technical documentation, language and even mathematical or computational theories. Acquiring this knowledge from zero as part of your research may not be possible or may consume too much time\footnote{The success of your research will depend critically on climbing any learning curve quickly and successfully, even if that learning curve looks like El Capitan!}. In such cases you should reconsider the phenomenon to study.
\item You need to make a judicious choice of case study to be able to make any generalisations about the phenomenon beyond the particular instance. If you don't have access to a significant instance of the phenomenon and generalisation is an important consideration, then you should consider a different research strategy.
\end{enumerate}


%%
%\begin{enumerate}%[start=0,label={(\bfseries R\arabic*):}]
%\item large amounts of data are needed: use ??? instead
%\item ???
%\end{enumerate}
%			

\subsection{Further reading}

To deepen your understanding of this strategy, you can start from \cite{yin2009case, oates2008researching}.

%\RSActivity{case study research}{oates2008researching}

\section{Action research}
  
Action research focuses on real-world situations for which improvement is sought through participatory and collaborative research. Its focus is on practice change, and continuous learning and improvement via an iterative `plan-act-reflect' cycle which generates both knowledge and action.

\subsection{Knowledge contribution}

Action research should make both a contribution to knowledge \textit{and} to practice: an action researcher strives to generate knowledge and action to address important problems that people experience in their practices, so that the knowledge contributed originates in real-world needs. The researcher is an active participant in the research, rather than solely an observer, alongside other collaborating practitioners: in fact, collaboration and reflexivity are essential elements of this strategy.

The outcomes of action research may be new theories or methods alongside their direct implementation to improve practice within a specific professional or social setting. For example, in an educational setting, where this strategy is widely applied, a group of teachers may come together to study the composition and effectiveness of homework at each school grade, with a view to improve the balance between knowledge-based and practice-based learning.

%\subsection{Focus}

%\subsection{Variants}

\subsection{Data generation and analysis}

Similar to case studies, action research requires you, and your collaborators, to collect empirical data from a variety of sources to gain a deep understanding of the current practical situation to be improved. Common methods which allow you to do so include observations, surveys, focus groups and document reviews. In this case too, both qualitative and quantitative data are collected and analysed.


\subsection{Evaluation}

The evaluation of action research will include the following questions:

	\begin{enumerate}
	\item Reliability: 
		\begin{itemize}
			\item Has the work used an iterative cycle of plan-act-reflect? How many cycles were conducted?
			\item Was the research collaborative? Is the level of collaboration achieved appropriate?		
			\item Were the data generation and analysis methods appropriately described?
		\end{itemize}	
	\item Bias: 
		\begin{itemize}
			\item Have the researcher's personal stake and potential biases been discussed? Was a reflexivity account included?
			\item Is there a reflection on self-delusion and groupthink\footnote{Groupthink is a tendency to conform to majority option to maintain unanimity and avoid confrontation.} or  of the collaborators? How was this mitigated? Was the mitigation successful? If not, what was the outcome?
		\end{itemize}	
	\item Valididty: 
		\begin{itemize}
			\item Was the learning from the plan-act-reflect cycle clearly identified and discussed?
			\item Were the data generation methods appropriate, and was enough data generated?
			\item Were detailed descriptions and accounts of findings provided?
			\item Has the research generated both knowledge and action leading to change? How useful or impactful on practice are they?
			\item Were generalisations made and appropriately supported by evidence, including triangulation? 
		\end{itemize}
%	\item Are limitations or omissions in relation to any of the points above identified? Have their effect on the research and its outcomes been discussed?
%
%	\item Has the research strategy been successful in relation to the research aim and objectives?
	\end{enumerate}
					
\subsection{Is this strategy right for me?}

There are conditions for this strategy to be successful:

					\begin{enumerate}
					\item Action research focuses on action aimed at solving real-world problems in professional and other social contexts. If that's not the case for your research, then you should consider a different strategy.
					
					\item The action researcher is expected to be embedded in the context in which the research takes place, and have a professional stake in addressing the problem beyond the research itself, for instance, as an employee of an organisation. If this is not the case for your project, then you should consider case studies instead.
				
					\item Action research requires the involvement of other practitioners as collaborators in the research. This goes beyond being merely participants in surveys or observations: instead it requires a much higher commitment and continuous involvement in the study. If this is not possible, you should consider case study research instead.
									
					\item Action research works through reflection, and continuous learning and improvement. As such, it does not exhibit the same level of scientific rigour as, for instance, an experiment. If scientific rigour is needed in your research, then you should consider a different strategy. 
					
					\item Action research may not be suitable to study complex causal-effect relationships. If you need to establish one such relationship, then you should consider the experimental strategy instead.
					
					\item Generalisation can also be difficult to achieve with action research. If you need to be able to generalise your research widely, then consider case studies instead.
							
					\item While action research is accepted and commonly applied in some social sciences, like education and healthcare, this may not be the case in your discipline. You should therefore check with your supervisor whether this strategy is acceptable or you should consider a different, strategy. 
					
					%\item Will the organisation in which you are embedded require you to work for them, rather than to conduct research? If so, ensure they are clear that you are not a consultant.
				
					%\item Do envisage issues working with others in complex, problematic and unpredictable real-world situations? If so, consider ???
					\end{enumerate}

\subsection{Further reading}

To deepen your understanding of this strategy, you can start from \cite{oates2008researching,johannesson2014research}.

%\RSActivity{action research}{oates2008researching,johannesson2014research}

%\endinput

\section{Ethnography}

The ethnography research strategy aims to study the culture of a group of people in their natural setting. 

\subsection{Knowledge contribution}

Ethnography contributes to knowledge by providing a cultural characterisation of the group under study. Such characterisation should be one that the group members recognise and find familiar, and should be inclusive of various cultural facets, both social and economical, rather than focusing solely on one specific aspect.

While ethnography was originally developed within the discipline of anthropology, particularly for the study of indigenous populations, it can be applied widely in social settings, for instance to study the work culture within a particular profession or organisation, or the culture of online communities within social media. 

%\subsection{Focus}

%\subsection{Variants}

\subsection{Data generation and analysis}

The researcher is expected to join the group and share what the group members' experience in their natural social setting in order to gain an insider's perspective and arrive at a rich, detailed characterisation. This requires the researcher to make detailed participant observations, appropriately recorded in field notes, accompanied by gathering detailed data through interviews and document reviews, linking them to the existing literature and reflecting on what they have learnt from their own experience, including their state of mind and emotional reactions. Data generation and analysis are predominantly qualitative.

\subsection{Evaluation}

Evaluating ethnography may involve asking the following questions:

	\begin{enumerate}
	\item Reliability: 
		\begin{itemize}
			\item Are field notes sufficiently rich and detailed? Do they capture people’s actions and behaviours, and the motivating reasons?
		\end{itemize}	
	\item Validity:
		\begin{itemize}
			\item Is the cultural characterisation obtained sufficiently rich to account adequately for the group's beliefs, customs, behaviours and interpersonal relations?
			\item Was adequate time spent with the group in their natural setting? What reflection has been done on such time?
			\item Are data appropriately interpreted through a cultural lens?
			\item Is the ethnographic characterisation a standalone description, or has it been linked to theory, other ethnographic studies or issues in other cultures?
		\end{itemize}
	\item Bias: 
		\begin{itemize}
			\item Does the research include a reflexive account of the researcher?
		\end{itemize}		
	
	%\item To which extent is the research presented as an ethnographic construction rather than as a literal description?	
%	\item Are limitations or omissions in relation to any of the points above identified? Have their effect on the research and its outcomes been discussed?
%
%	\item Has the research strategy been successful in relation to the research aim and objectives?
	\end{enumerate}					

\subsection{Is this strategy right for me?}

There are conditions for this strategy to be successful:

	\begin{enumerate}
	\item Ethnography requires you to be a researcher located within the context of your situated research, for instance your employer organisation, where the research is likely to require an extensive amounts of time. If you have yet to have identified the context, or have yet to reach out to obtain permission to proceed, then this requirement may mean that ethnographic research may not be feasible. So, if you are not already close to your context of research, you may wish to consider case study research instead.
	\item Even if you are already located within the context of your ethnographic research, the context must be accepting of an ethnographic approach for your research to be successful. An organisation, for instance, in which there is a culture of strict compartmentalisation may not provide sufficient opportunities for ethnographic research. 
	\item In ethnographic research you allow the culture under study to determine the outcomes of the research, so you should approach it without any preconception or bias. If there is any possibility that you could be biased to a particular outcome – as might happen if you feel you already know the outcome and are simply trying to confirm this – then ethnography is unlikely to lead to a successful outcome for your research. Any competent ethnographer will be particularly sensitive to expressions of such confirmation bias.
	\item Ethnography can lead to rich descriptions of complex social settings, and the characterisation produced may be very deep in representing a particular group culture. However, this may be difficult to generalise to other social groups or settings. If generalisation is an important aspect of your research, then you should consider case studies instead.
	%\item Ethnography is analytical in the extreme. Should you not have an analytical mindset, then ethnography should be avoided.
	\end{enumerate}

\subsection{Further reading}

To deepen your understanding of this strategy, you can start from \cite{oates2008researching,johannesson2014research}.

%https://study.sagepub.com/sites/default/files/Eriksson%20and%20Kovalainen.pdf


%\RSActivity{ethnography}{oates2008researching,johannesson2014research}


\section{Systematic research reviews}

A systematic research review is used to generate new insights from published work, linked to a clearly defined research problem or question.

\subsection{Knowledge contribution}

A systematic research review is meant to advance a field of study by providing insights from across the literature not contained in individual research papers. It uses a rigorous set of criteria to identify, select, and critically appraise relevant research from previously published studies in order to generate a scholarly synthesis of the evidence in relation to an explicit research problem or question. 

For example, a systematic research review could be conducted to generate new insights on the effectiveness of a specific medical treatment, in order to advance evidence-based medicine: published articles on randomised controlled trials for that treatment could be reviewed and a judgement made based on a synthesis of the results from the accumulated body of work.

				
%\subsection{Focus}
%\subsection{Variants}

\subsection{Data collection and analysis}

In a systematic research review you only use evidence from published studies and rely on explicit, reproducible methods to identify the relevant research to review. Specifically, you must decide upfront your research problem\slash question and the set of criteria you will use to select, compare and evaluate those studies, and combine their results. 

The type of analysis you will conduct will depend on the nature of the evidence you are considering and combining. In \textit{narrative reviews}, a narrative synthesis is produced of qualitiative results, while in \textit{meta-analysis}, statistical techniques are used to analyse and combine quantitative results. Combinations of the two are also possible. 
				
\subsection{Evaluation}

Evaluation of the systematic research review will involve answers to the following:

	\begin{enumerate}
	\item Reliability: 
		\begin{itemize}
			\item Are the criteria used to select, exclude, evaluate and combine the published research explicit and reproducible? Were there any deviations from this protocol and, if so, are they explained, justified and documented?
		\end{itemize}
	\item Validity: 
		\begin{itemize}
			\item Has the researcher accessed all relevant published research in the area?
			\item Have the relative strengths and weaknesses of the research reviewed been described? To which extent have conflicts between sources been identified and, when appropriate, resolved?
			\item In combining results from different studies, are significant differences between those studies appropriately acknowledged?
			\item To which extent has a definitive synthesis from the literature been achieved? To which extent are the limits of current knowledge described?
			\item To which extent has precision and/or generalisability been improved through the systematic research review? 
			\item In meta-analysis, to which extent have statistics been used to produce overarching conclusions? Were the studies sufficiently homogeneous for meta-analysis to be feasible?
		\end{itemize}
	\item Bias: 
		\begin{itemize}
		\item In narrative reviews, to which extent potential bias has been acknowledged and mitigation measures applied?
		\end{itemize}
%	\item Are limitations or omissions in relation to any of the points above identified? Have their effect on the research and its outcomes been discussed?
%
%	\item Has the research strategy been successful in relation to the research aim and objectives?

	%\item given the research question, to which extent is the review type the most appropriate ~\parencite[p.142]{pollock2018systematic}?
	
	%\item Have implications for future research and practice been discussed both for the research question and in other areas that were raised by the review?
	
	%\item do you discuss the updating of the review~\parencite[p.142]{pollock2018systematic}?
		\end{enumerate}

%{\parencite[adapted]{wright2007write}: 
%%
%	\begin{enumerate}%[start=0,label={(\bfseries R\arabic*):}]
%%	\item Clarifying the relative strengths and weaknesses of the literature on the question, 
%%	\item Summarising a large amount of literature, 
%%	\item Resolving literature conflicts, 
%%	\item Evaluating the need for large interventions, 
%%	\item Avoiding a redundant unnecessary interventions, 
%%	\item Increasing the statistical power of smaller studies, 
%	\item Improving the precision or identify of smaller interventions, and 
%	\item Improving the generalisability of intervention outcomes.
%	\end{enumerate}
%	%
					
\subsection{Is this strategy right for me?}

You should consider the following points when choosing this strategy:

\begin{enumerate}
	\item A systematic research review is both systematic and extensive in its coverage of the topic of interest. This requires you to have a very good grasp of the subject area in order to establish appropriate criteria for the selection of all relevant published work, and this may lead to a large body of work to review. If you lack such knowledge of the filed of study or the time to conduct an extensive review of the literature, then you should consider a different strategy.
	\item A systematic research review assumes that there is a substantial body of knowledge already published from which new insights can be generated. In emerging research areas, this may not be the case, so that a systemic research review is unlikely to reach any meaningful synthesis. If there is paucity of literature on your chosen topic, then this strategy is not for you.
	\item You need relatively easy access to the academic literature to be able to select the body of work to review, for instance, through a university library with a large research collection in your chosen discipline. If not, you will need to devise alternative ways to access the relevant literature, like contacting the author(s) directly. Although most authors will be happy to send their published research to you, the round trip time can introduce lengthy delays in the research process as you wait for the requested research to arrive. You may also need to be persistent to ensure that a busy author is aware of your research need. If you don't think you will be able to access easily a large proportion of the published work you need, you should rethink your research strategy.
	\item Systematic research reviews are required to be transparent, reliable, and easy to replicate. You will be expected to have stated explicit inclusion and exclusion criteria so that another research would be able to arrive at the same collection of published work, and ensure that no inherent bias has influenced such choice. Choosing appropriate criteria may be difficult and may require you to iterate, starting perhaps from a broader focus, then narrowing it down as your research progresses. This, too, can be time consuming, so if your time is limited, you should consider a more time-efficient strategy for your research, perhaps one which allows you to generate your own primary evidence.
	\end{enumerate}


%	\begin{itemize}
%%	\item unable to consult all relevant literature; if so perhaps you're only trying to complete a preliminary literature review? \parencite[p.414]{andrews2005place}
%%	\item identified focus is too narrow to provide generalisable results~\parencite{wright2007write}
%%	\item difficult to construct strict inclusion and exclusion criteria for the survey
%%	\item no meaningful conclusion reached due to paucity of literature
%%	\item study heterogeneity precludes meta-analysis, the authors of the systematic review need to summarise the findings based on the strength of the individual studies and reach conclusions if indicated~\parencite[p.27]{wright2007write}
%%	\item won't be more comprehensive than what exists~\parencite[p.405]{andrews2005place}
%%	\item there is some latent bias possible~\parencite[p.405]{andrews2005place}
%%	\item it will not transparent or nor replicable~\parencite[p.405]{andrews2005place}
%	\end{itemize}
%	%
								
\subsection{Further reading}

To deepen your understanding of this strategy, you can start from \cite{wright2007write,moher2009preferred,pollock2018systematic}. 

In addition, you should consider the PRISMA statement\footnote{http://prisma-statement.org/prismastatement/checklist.aspx}, a 27-item check-list whose aim is to help authors improve the reporting of systematic reviews and meta-analyses.
					
%\RSActivity{systematic research survey}{wright2007write,moher2009preferred,pollock2018systematic}


%\endinput

\section{Grounded theory}

Grounded theory aims at defining theories\footnote{In simple terms, a theory is a system of ideas intended to explain something.} on social phenomena based\footnote{I.e., `grounded', hence the name!} on empirical data. The intention is for the theory to emerge from the collection and analysis of the data, rather than using the data to confirm or disprove a previously formulated theory, or test a previously formulated hypothesis. 

%\textcquote[3.1.5]{johannesson2014research}{Grounded theory is a research strategy that strives to develop theories through the analysis of empirical data. In contrast to experiments, grounded theory does not start with a hypothesis to be tested but instead with data from which a theory can be generated. Grounded theory also differs from research strategies, such as ethnography, which are content to provide rich descriptions of particular situations, but no theories. Grounded theory challenges a top-down theorising approach, in which the researcher first develops a theory and then checks whether it conforms to empirical data. Instead, grounded theory insists that empirical data is the starting point, upon which theories are to be built. Theory emerges through analysis and is grounded in the data.}

%\endinput

\subsection{Knowledge contribution}

Grounded theory contributes knowledge in the form of theories concerning complex social phenomena, striving to provide explanations of people’s choices and actions grounded in those people's own accounts and interpretations.

For example, grounded theory could be used to formulate theories on what motivates people to join or leave a particular organisation, or why employees may feel fulfilled or frustrated in their workplace.  

%\subsection{Focus}

%\subsection{Variants}

\subsection{Data collection and analysis}

Grounded theory requires the systematic collection and analysis of data without any preconceived belief or theoretical framework. The data are collected, coded and analysed to identify emerging concepts, categories and relationships. The process is iterated with new data used to review and revise those concepts, categories and relationships until no more can be gained from further data collection and analysis. At this point a theory is put forward based on what was derived from the data. In this process, it is essential to be open to multiple explanations, and to explore the data from all angles in order to gain a fresh perspective.


%\textcquote{drew202310-grounded}{Grounded theory aims to understand social phenomena by systematically collecting and analyzing data without preconceived notions or theoretical frameworks. The process involves iterative coding and constant comparison of data to generate concepts, categories, and relationships.}


%\endinput

%Empirical data is extant;~\textcquote[p.13]{strauss1998basics}{In fact, Patton (1990), a qualitative evaluation researcher, made the comment, \enquote{Qualitative evaluation inquiry draws on both critical and creative thinking – both the science and the art of analysis} (p. 434). He went on to provide a list of behaviors that he found useful for promoting creative thinking, something every analyst should keep in mind. These include (a) being open to multiple possibilities; (b) generating a list of options; (c) exploring various possibilities before choosing any one; (d) making use of multiple avenues of expression such as art, music, and metaphors to stimulate thinking; (e) using nonlinear forms of thinking such as going back and forth and circumventing around a subject to get a fresh perspective; (f) diverging from one's usual ways of thinking and working, again to get a fresh perspective; (g) trusting the process and not holding back; (h) not taking shortcuts but rather putting energy and effort into the work; and (i) having fun while doing it (pp. 434–435).~\parencite{patton1990qualitative}}

%\endinput



%\endinput

%\textcquote{corbin1990grounded}{The success of a research project is judged by its products. Except in unusual instances when these are only orally presented, the study design and methods, findings, theoretical formulations, and conclusions are judged through publication. Yet, how are these to be evaluated and by what criteria? When judging qualitative research it is not appropriate, we have asserted, to use criteria ordinarily used to judge the procedures and canons of quantitative studies. It has been one of the aims of this paper to show how the grounded theory approach accepts the usual scientific canons but redefines them carefully to make them appropriate to its specific procedures. In the instance of any grounded theory study, the specific procedures and canons as described above should be part of its evaluation.}

%\endinput

%\textcite{smith1997understanding}
%					\begin{enumerate}%[start=0,label={(\bfseries R\arabic*):}]
%					\item are all processes~\parencite[remove when checked]{strauss1998basics} made explicit? 
%					\item have a range of data sources been used?
%					\item {}[add as general initial item]Was the research question which originated the study included in the final report? 
%					\item were the phenomena to be studied observable in social interaction?
%					\item {}[general]does the literature review allow the reader to identify the issues that the researcher found interesting initially? ~\parencite{smith1997understanding}
%					\item {}[general?]does the literature review provide a backdrop against which the new findings can be evaluated?~\parencite{smith1997understanding}
%					\item have the outcomes been triangulated - has more than one way of arriving at the theory been used?~\parencite{smith1997understanding}
%					\item has a wide range of participants been used? Did the importance of an issue come through repeatedly? Did the participants agree with the analysis process?~\parencite{smith1997understanding}
%					\item has the extent to which the theory is supported by the extant data been described?~\parencite{smith1997understanding} 
%					\item what new insight do the developed theories provide?~\parencite{smith1997understanding} 
%					\item {}[general and specific]have ideas for further study and implication for practice been discussed~\parencite{smith1997understanding}?
%					\end{enumerate} 
%					%
%					and from \textcite[p.425]{corbin1990grounded}
%					%
%					%
%					\begin{itemize}
%					\item [The Research Process:]
%					\begin{enumerate}[label={Criterion \arabic*:}]
%						\item How was the original sample selected? What grounds (selective sampling)?
%						\item What major categories emerged? 
%						\item What were some of the events, incidents, actions, and so on that (as indicators) pointed to some of these major categories?
%						\item On the basis of what categories did theoretical sampling proceed? That is, how did theoretical formulations guide some of the data collection? After the theoretical sampling was done, how representative did these categories prove to be?
%						\item What were some of the hypotheses pertaining to conceptual relations (that is, among categories), and on what grounds were they formulated and tested?
%						\item Were there instances when hypotheses did not hold up against what was actually seen? How were these discrepancies accounted for? How did they affect the hypotheses?
%						\item How and why was the core category selected? Was this selection sudden or gradual, difficult or easy? On what grounds were the final analytic decisions made?
%					\end{enumerate}
%
%					\item [Empirical Grounding of findings]
%\begin{enumerate}[label={Criterion \arabic*:}]


%Strauss & Corbin state that there are four primary requirements for judging a good grounded theory: 1) It should fit the phenomenon, provided it has been carefully derived from diverse data and is adherent to the common reality of the area; 2) It should provide understanding, and be understandable; 3) Because the data is comprehensive, it should provide generality, in that the theory includes extensive variation and is abstract enough to be applicable to a wide variety of contexts; and 4) It should provide control, in the sense of stating the conditions under which the theory applies and describing a reasonable basis for action.

\subsection{Evaluation}

The following questions should be addressed in evaluating grounded theory research:

\begin{enumerate}
	\item Reliability: 
		\begin{itemize}
			\item Was the process followed to arrive at the theory appropriately described? Was it systematic and iterative?
		\end{itemize}
	\item Validity: 
		\begin{itemize}
			\item Were sufficient data collected and described? How relevant were they to the phenomenon under study?	
			\item Which concepts, categories and relationships were generated by the research? How are they grounded in the data? How do they contribute to the theory?
			\item Has the phenomenon been examined under a broad range of conditions and from a variety of perspectives?
 			\item Is the theory plausible in relation to the data? Does it provide sufficient explanation of the phenomenon under study? Is it general enough to account for variation in conditions and context of application?
			\item Can the theory be easily understood by its intended users? How useful is it in helping them understand their social reality and be the basis for action?
		\end{itemize} 
	\item Bias: 
		\begin{itemize}
			\item Is there a reflexivity account of the researcher to guard against possible bias?
		\end{itemize}
%	\item Are limitations or omissions in relation to any of the points above identified? Have their effect on the research and its outcomes been discussed?
%	\item Has the research strategy been successful in relation to the research aim and objectives?
\end{enumerate}

%\textcquote{corbin1990grounded}{Since the basic building blocks of any grounded theory is a set of concepts grounded in the data, the first question to be asked of any publication is: Does it generate (via coding-categorizing activity) or at least use concepts, and what is or are their source or sources? If concepts are drawn from common usage (such as, “uncertainty”) but not put to technical use, then these are not concepts in the sense of being part of a grounded theory, for they are not actually grounded in the data themselves.}
					
					
%\textcquote{corbin1990grounded}{The name of the scientific game is systematic conceptualization through conceptual linkages. So, the questions to ask here of a grounded theory publication are whether such linkages have been made and do they seem to be grounded in the data? Furthermore, are the linkages systematically carried out? As in other qualitative writing, the linkages are unlikely to be presented as a listing of hypotheses or in propositional or other formal terms but will be woven throughout the text of the publication.}
					
									
%\textcquote{corbin1990grounded}{If there are only a few specified conceptual relationships, even if grounded and identified systematically, this leaves something to be desired in terms of the overall grounding of the theory. A grounded theory should be tightly linked, both in terms of categories to their subcategories and between categories in the final integration in terms of the paradigm features conditions, context, action/interaction (strategies) and consequences. Also categories, as mentioned in the body of the paper, should be theoretically dense (have many properties that are dimensionalized). It is the tight linkages, in terms of the paradigm features and density of the categories, that give a theory its explanatory power. Without these, the theory is less than satisfactory.}
					
					
%\textcquote{corbin1990grounded}{Some qualitative studies report only about a single phenomenon and establish only a very few conditions under which it appears, and specify only a few actions/interactions that characterize it, and a limited number or range of consequences. By contrast, a grounded theory monograph should be judged in terms of the range of its variations and the specificity with which these are spelled out in relation to the data that are their source. In a published paper, the range of variations touched upon may be more limited, but the author should at least suggest that the fully study included their specification.}
					
					
%\textcquote{corbin1990grounded}{The grounded theory mode of research requires that the explanatory conditions brought into analysis are not restricted to those that seem to have immediate bearing on the phenomenon under study. That is, the analysis should not be so “microscopic” as to disregard conditions that derive from more “macroscopic” sources: for instance, those such as economic conditions, social movements, trends, cultural values, and so forth.
					
%These also must not simply be listed as background material but directly linked to phenomena through their effect on action/interaction, and through these latter to consequences. Therefore, any grounded theory publication that either omits these broader conditions or fails to explicate their specific connections to the phenomenon(a) under investigation, falls short in its empirical grounding.}

					
%\textcquote{corbin1990grounded}{Identifying and specifying change or movement in the form of process is an important part of grounded theory research. Any change must be linked to the conditions that gave rise to it. Process may be described as stages or phases and also as fluidity or movement of action/interaction over the passage of time in response to prevailing conditions.}
					
										
%\textcquote{corbin1990grounded}{The question of significance is generally thought of in terms of the relative importance of a theory for stimulating further studies and for giving useful explanations of a range of phenomena. We have in mind here, however, the adequacy of a study’s empirical grounding in relation to its actual analysis insofar as this combination of activities succeeds or fails, in some degree, at producing useful theoretical findings. If the researcher simply follows the grounded theory procedures/canons without any imagination or insight into what the data are reflecting - because he or she fails to see what they are really saying except in terms of trivial or well known phenomena - then the published findings can be judged as failing on this criterion. Recollect that there is an interplay between the researcher and the data, and no method, certainly not the grounded theory one, can insure that the interplay will be creative. This depends on three characteristics of the researcher: analytic ability, theoretical sensitivity, and sensitivity to the subtleties of the action/interaction (plus sufficient writing ability to convey the findings). Of course, a creative interplay also depends on the other pole of the re­searcher-data equation: the quality of the data collected or utilized. An unimaginative analysis may in a technical sense be adequately grounded in the data, but actually it is insufficiently grounded for the researcher’s theoretical purposes. This is because the researcher either does not draw on the fuller resources of data or fails to push data collection far enough.}

%\endinput

%\textcquote{corbin1990grounded}{This double set of criteria, for the research process and for the empirical grounding of the theoretical findings, bear directly on the issues of how verified any given grounded theory study is and how this is to be ascertained. When the study is published,, if components of the research process are clearly laid out and if there are sufficient cues in the publication itself, then the presented theory or theoretical formulations can be assessed in terms of degrees of plausibility. We can judge under what conditions the theory might fit with “reality”, give understanding, and be useful (practically and in theoretical terms). Researchers themselves can be rendered more aware of precisely what their operations have been and the possible inadequacies of these operations. In other words, they would be able to identify and convey what were the limitations of their study.}

%\endinput

%\textcquote{drew202310-grounded}{Grounded theory is an innovative way to gather qualitative data that can help introduce new thoughts, theories, and ideas into academic literature. While it has its strength in allowing the “data to do the talking” , it also has some key limitations – namely, often, it leads to results that have already been found in the academic literature. Studies that try to build upon current knowledge by testing new hypotheses are, in general, more laser-focused on ensuring we push current knowledge forward. Nevertheless, a grounded theory approach is very useful in many circumstances, revealing important new information that may not be generated through other approaches. So, overall, this methodology has great value for qualitative researchers, and can be extremely useful, especially when exploring specific case study projects. I also find it to synthesize well with action research projects.}
%

\subsection{Is this strategy right for me?}

In choosing this strategy, you should consider the following:
\begin{enumerate}
\item Grounded theory is about letting the “data to do the talking” \cite{drew202310-grounded}, so you should not have any prior belief, theory or hypothesis you wish to put to test. If that's not the case, other strategies are more appropriate, like case studies, ethnography or experiments.
\item Grounded theory requires you to gather a significant amount of empirical data, making sure you examine a social phenomenon under various conditions and from many perspectives. If you do not have access to such data, then grounded theory cannot get started, and you should consider other strategies.
\item Grounded theory is generally time consuming, given the iterative nature of the process of gathering and analysing data. If time is an issue in your project, then you should choose a more time-efficient strategy, like case studies or experiments. 
\item Grounded theory aims at generating theories concerning social phenomena. If a new theory is not the aim of your work, then you should choose a different strategy, like ethnography or case studies.
\item Grounded theory is particularly useful when there is a paucity of theories in relation to the phenomena of interest. If there are already several theories available, it is less likely grounded theory will be able to contribute something new. In such cases, you should rethink whether a new theory is actually needed or choose a different aim and strategy for your project.
\end{enumerate}
%

								
\subsection{Further reading}

To deepen your understanding of this strategy, you can start from \cite{smith1997understanding,drew202310-grounded,corbin1990grounded,strauss1998basics,gibson2013rediscovering,charmaz2014constructing}.
				
%\RSActivity{grounded theory}{smith1997understanding,drew202310-grounded,corbin1990grounded,strauss1998basics,gibson2013rediscovering,charmaz2014constructing}


\section{Phenomenology}

Phenomenology is a research strategy that focuses on people's conscious experience of a phenomenon, that is how people perceive and give meaning to it, including any feeling and emotions it evokes. 

\subsection{Knowledge contribution}

Phenomenology contributes knowledge by providing insights into people's lived experience, seeking to describe or interpret the essence of a phenomenon from the perspective of the people who have experienced it.

For instance, a phenomenological study of patients emergency care could focus on the experience of nurses and doctors in emergency departments.

%\subsection{Focus}
%
%\subsection{Variants}

\subsection{Data generation and analysis}

Data generation in phenomenology is primarily through in-depth, unstructured interviews and focus groups, which should allow participants to give their own account of their experience and surface key issues, without being influenced by the researcher. These are often complemented by participant observation, in which the researcher is immersed in the day-to-day activities of the study participants, hence sharing their experience of the phenomenon of interest. Audio and video recording, alongside field notes and journals are used to record data.

Data gathering typically results in a large quantity of qualitative data, which are both detailed and unstructured, so that qualitative methods are then needed for their analysis. The analysis process requires the researcher to set aside any preconception, assumption or bias\footnote{This is referred to as 'bracketing'.} and focus solely on the data, considering every participant's statement or expression as equally important and relevant.

\subsection{Evaluation}

The following questions should be asked of phenomenological research:

\begin{enumerate}
	\item Reliability: 
		\begin{itemize}
			\item Are the criteria for selecting the study participants properly explained and justified?
		\end{itemize}
	\item Validity: 
		\begin{itemize}
			\item Is the phenomenon accurately and objectively described?  Is the account provided one that can be recognised by anyone who has experienced that phenomenon?
			\item How have similarities and differences in the participants' experience of the phenomenon accounted for in the study? How are they dealt with in the data analysis, particularly in the coding and categorisation process?
		\end{itemize}
	\item Bias: 
		\begin{itemize}
			\item Is there a reflexive account of how judgments were suspended to focus on the analysis of experience?
		\end{itemize}
%	\item Are limitations or omissions in relation to any of the points above identified? Have their effect on the research and its outcomes been discussed?
%	\item Has the research strategy been successful in relation to the research aim and objectives?
\end{enumerate}

%			\begin{enumerate}
%					\item Will the experience as described be understandable to any reader and can be identified by anyone who has had that particular experience?
%					\item Is the description of the phenomenon clearly presented so that experience differs from other experiences that are similar? {\color{red}what does this mean?}
%					\item Are quotations from the data used to demonstrate the emergence of themes?
%					\item Is there a discussion of discrepancies among participants and how those discrepancies were factored in data analysis?
%					\item Have meaning units, themes, and summaries been described?
%					\item Are meaning units grouped together to form themes?
%					\item Are themes combined to form a composite summary of the phenomenon?
%					\item Are quotes used to support the findings?
%					\item Research participants will have their individual ways of experiencing a certain phenomenon. Have you looked for these common to all or most of the participants and not clustered meaning units together where significant differences exist?
%				\end{enumerate}
%					%
%					and the research process
%					%
%				\begin{enumerate}
%					\item Bracketing/Epoché/Phenomenological reduction - have you discussed how judgments were suspended to focus on analysis of experience. How did you use suspend your judgments to focus on the analysis of participants' experiences?
%					\item Horizon - during data analysis, what was your present experience, your horizon? The horizon cannot be bracketed so you will need to discuss that not everything could have been realized by you, the researcher. This discussion might also lead into a discussion about future research implications in Chapter 5.
%					\item Intentionality - discuss your level of scrutiny of the data you analyzed. How did you keep your focus on the topic you were studying? Perhaps you slowed down and dwelled on each narrative and did not pass over the details of the account as if you understood it already.
%					\item Dasein - How has your Dasein (being-there) affected the research? How did the research affect your Dasein?
%					\item Fore-sight/Fore-conception - What was your preconceived knowledge about the phenomenon you were studying?
%					\item Hermeneutic Circle - How did were your understandings revised as you analyzed the data?
%			\end{enumerate}
%	
					%


%				Phenomenology is sometimes presented as having similarities with ethnography \parencite[p.52]{johannesson2014research}, and this leads us to the following evaluation criteria{\color{red} sense?}:
%					\parencite[p.180, adapted for phenomology]{oates2008researching}%
%					\begin{itemize}
%					\item Does the research focus on lifestyle, meaning, and belief?
%					\item Are the data generation methods that were used described? Did they lead to sufficient data having been collected?
%					\item How long did you spend in the field? Do you think this was long enough?
%					\item Do you describe your approach holistic, semiotic or critical?
%					\item Is the phenomenology a standalone description, or is it linked to theory, other phenomenology or issues in your culture?
%					\item Have you included an account of you as researcher?
%					\item Have you presented an phenomenological construction rather than a literal description?
%					\item What limitations in the phenomenology have you recognised?
%					\item Which other flaws and/or omissions in your reporting of the phenomenology do you admit?
%					\item Overall, how effectively has your phenomenological strategy been?
%					\end{itemize}
%					and \parencite[29m \textit{ff}]{office2020the-phenomenological}
%					%
%						\begin{itemize}
%						\item have you described the \enquote{what, when, where, and how} of the study? What has be done? When the steps were sequenced? Where each step happened? How each step happened?
%						\item have you described where will data be collected? Who collected the data? How often and how much data was collected? How long it took to collect the data? How the data was recorded? (ex: transcriptions, video recordings, audio recordings) Were there follow ups to interviews?
%						\end{itemize}
%						%
%				

\subsection{Is this strategy right for me?}

In choosing this strategy, you should consider the following:

\begin{enumerate}
	\item The focus on phenomenology is lived experience, to uncover what is really like to experience a phenomenon from the perspective of those who have lived through it. If you do not have access to participants who can share their experience, then you should consider a different strategy.
	\item Phenomenology asks you to suspend any prior belief on the phenomenon and only focus on the participants' experience. If you have a theory or hypothesis you want to test, then you should choose a different strategy.
	\item The amount of qualitative data to gather and analyse is considerable, and this can be very time consuming. If time is an issue in your research, then you will need to choose a more time-efficient research strategy.
	\item Phenomenology is about going deep into the experience of a phenomenon, and this constrains the number of participants in your study. If you are more interested in  gaining consensus from a large number of participants, or making general predictions from your sample data, then you should choose something else, like survey research or grounded theory.
	\end{enumerate}

%	\item is your audience expecting scientific rigour? If so, choose ???
%\item does your data (source) allow analysis above the data, or will the outputs be mostly descriptive?
%\item are you short of time for data collection? If so, consider ???
%\item do you appreciate the value of deep philosophical discourse?
%\item ...



\subsection{Further reading}
To deepen your understanding of this strategy, you can start from \cite{merleau1956phenomenology,anderson1991qualitative,smith2018phenomenology,shudak2018phenomenology,academic-educational-materials2019understanding,office2020the-phenomenological,groenewald2004a-phenomenological,hycner1985some}.


\section{Simulation}

The simulation research strategy builds an explicative mechanism to imitate or reproduce the behaviour of a real-world artefact or system.

\subsection{Knowledge contribution}

Simulation contributes knowledge by allowing the study of the simulated artefact or system under different conditions, in order to answer \enquote{What if?} questions, make predictions or gain insights on behaviour or properties, particularly when this can't be easily achieved directly on the real-world artefact or system.

Simulations are used in all disciplines and vary greatly in their purpose, nature and design. For instance: financial simulations are used to study the behaviour of the global stock market; climate simulations to study possible effects of climate change; engineering simulations, to test the properties of materials under different stress conditions; social simulations to study human behaviour in social settings, to name just few examples.  

%\subsection{Focus}
%
%\subsection{Variants}


\subsection{Data generation and analysis}

Data are needed to inform the simulation design. Their kind and how to obtain them will depend strongly on the nature of the artefact or system under study and what your research aim is, so that all known methods for data generation apply. For instance, to simulate a new aircraft design under different wind conditions, you may need to gather data on the physical characteristics of the aircraft and the materials to be used to build it, alongside meteorological data which can be used to perform tests under different simulated conditions. On the other hand, to simulate how size and age of a population may change in future decades within a particular geographical region, you may need to gather data on current population size and age, birth and mortality rates, migration rates in and out of the region, and conditions which may affect them over time. 

Data are also needed to establish measures and criteria to evaluate whether the simulation is sufficiently representative of the artefact or system being simulated. This may involve comparing simulation outputs with empirical data or theoretical predictions, or gathering expert opinions on such outputs, with the aim of establishing the extent expectations are met or significant discrepancies exist. 

Whichever methods you use to gather data for your simulation design, once constructed, the simulation should allow you to generate observations of the simulated artefact or system, both past, present and future, the latter being a unique characteristic of this research strategy\footnote{All other strategies can only look at the past or the present.}. Such observations can then be analysed in order to address the research question, as well as to evaluate the simulation against the established measures or critera. 

%According to \parencite{dooley2017simulation}, the most common types\footnote{Although this list is not exhaustive} include:
%		%
%	\begin{itemize}
%	\item Discrete event simulation, which involves modeling the organizational system as a set of entities evolving over time according to the availability of resources and the triggering of events.
%	\item System dynamics, which involves identifying the key “state” variables that define the behavior of the system, and then relating those variables to one another through coupled, differential equations.
%	\item Agent-based simulation, which involves agents that attempt to maximize their fitness (utility) functions by interacting with other agents and resources; agent behavior is determined by embedded schema which are both interpretive and action-oriented in nature.
%	\end{itemize}
%	%

%Depending on the context in which the simulation is studied or used. May include:
%	%
%	\begin{itemize}
%	\item observations, interviews, questionnaires, documents, in-depth description when supporting social objectives
%	\item accuracy, predicative capability, when used to model real-world processes – think weather forecasting
%	\item ...
%	\end{itemize}
%	%
				
\subsection{Evaluation}

Evaluation questions specific to this strategy include:

\begin{enumerate}
	\item Reliability: 
		\begin{itemize}
			\item Is the simulation design appropriate to address the research problem/answer the research question?
			\item Were simulation performance measures or criteria clearly established? How were they chosen and why?
			\item How was the simulation constructed? Were appropriate computational/mathematical/statistical techniques applied?
		\end{itemize}
	\item Validity: 
		\begin{itemize}
			\item Were appropriate data gathered to inform the simulation design? How were they chosen and why?
			\item How was the simulation tested and improved during its development? Were different testing methods applied and the results documented?
			\item How close are the simulation's outputs or behaviours to the real-world data? How was this established? Do the results make sense?
		\end{itemize}
\end{enumerate}

%For computer simulations:
%
%\begin{itemize}
%\item Good development: has a documented set of requirements been maintained? Has a change control process been implemented?Is there a corresponding document (or version) control process.
%
%\item Has the architecture been documented? What is its relationship to the model?
%
%\item Have a variety of testing methods, including code walk-throughs, scenario testing, and user testing been used to establish code quality?
%
%\item Was a project plan for coding and testing developed?
%
%\item how close is the simulation's behaviour to the \enquote{real} answer? Do the results make sense?
%
%\item has the simulation been compared to any extant quantitative behaviour available? Does it match exactly, distributionally (a variable of interest has statistically similar characteristics), or pattern-wise (variables are generally related to one another in a valid manner, but perhaps differ from reality)?
%
%\item Which experimental set-up was used? Was it appropriate?
%
%\item Have observations from analysis been noted, and results discussed in order to sense-make? Has over-interpretation of the results been avoided so that retrofitting to theories is avoided?
%\end{itemize}
%
%For simulation:
%
%%
%\begin{itemize}
%\item ...
%\end{itemize}
%%

\subsection{Is this strategy right for me?}

Is you are considering the simulation research strategy, then you should consider the following:
\begin{enumerate}
	\item The design and construction of a simulation requires advanced computational skills, often alongside mathematical and statistical skills. Do you already have such skills? If not, you are unlikely to be able to develop them in the time of your project, and should consider other research strategies instead.
	\item Do I have access to the data, and possibly stakeholders, needed for the design and  evaluation of the simulation? Without such data it is unlikely you would be able to build a representative simulation, hence you would not be able to generate valid and reliable results. In such case, you should consider other strategies.
\end{enumerate}

\subsection{Further reading}

To deepen your understanding of this strategy, you can start from \cite{dooley2017simulation}.

\section{Mathematical and logical proof}

\todo{Jon to go through this. I think we need both deductive and inducting proofs}

A mathematical proof is a rigorous argument that demonstrates the truth of a certain proposition starting from certain assumptions. As long as the assumptions are true, then an argument is constructed in a way that guarantees that the proposition is also true. Such argument is termed `deductive' as it starts from the assumptions and arrives at the proposition as the conclusion of the reasoning. 

When the reasoning is carried out within a fully formal logical system, then we have a logical proof.

%\textcquote[3.1.9]{johannesson2014research}{A proof is a rigorous deductive argument that demonstrates the truth of a certain proposition.}


\subsection{Knowledge contribution}

Mathematical and logical proofs contribute knowledge in the form of true propositions, which are the means by which Mathematics functions and grows its scope and applicability.  

\todo{Give example}

Within a mathematical system, such truths are absolute, something which does not hold in any other scientific discipline: even in the natural sciences and by taking a (post-)positivist stance, truths are always only tentative and falsifiable, in that they hold only until new evidence emerges to contradict them.

%\subsection{Focus}
%
%\subsection{Variants}

\subsection{Data generation and analysis}
\todo{rethink data gen. here; add inductive and deductive reasoning}

In this strategy, it does not make sense to talk about data generation and analysis.

Instead, all mathematical disciplines, relies on sets of assumptions and previously proven propositions which are taken as the starting point to generate, through proofs, new propositions, that is new truths.

\subsection{Evaluation}

A new proof is subject to the scrutiny of the community of mathematicians, which employs various means to check both assumptions and reasoning, so as to reach a verdict on the reliability and validity of the proof. 

Such means may include using alternative deductive reasoning to check they can reach the same conclusion, using examples in support of the reasoning, or even recreating a mathematical proof within a fully formal system or using a computer-based automated checker, when applicable. You could adopt some of these approaches to improve or defend the reliability and validity of your proof.

\subsection{Is this strategy right for me?}

If you are considering this strategy, you should ask yourself:

\begin{enumerate}
	\item Mathematical and logical proofs only make sense within research which is amenable to formalisation. Is that the case for your project? If not, then the notion of proof may not apply and you should consider an empirical research strategy instead.
	\item You will need to be a skilled mathematician or logician to come up with a proof that can withstand the scrutiny of the mathematical community. Do you possess such skills? If that's not the case, then this strategy may not be for you\footnote{One way to determine whether your background is suitable would be to read the first few pages of \textcite{lakatos2015proofs} (up to page 9 is available through google books).}. 
	\item Ideally, you should seek formative feedback from an experienced mathematician as you develop your proof, to reduce the chance of mistakes or reasoning pitfalls. Do you have access to such an expert advice? If not, you should discuss with your supervisor to ensure they do have the skills to take up that role.
\end{enumerate}


%Am I trying to predict future behaviours of a system? If so, mathematical, statistical or computational modelling might be a better option.

\subsection{Further reading}

To deepen your understanding of this strategy, you can start from \cite{Kleene1964introduction,lakatos2015proofs,antonini2011examples,johannesson2014research}.

\section{Mixed methods research}

The mixed methods research strategy\footnote{Mixed-method research should not be confused with \textit{multi-method research}, which simply indicates the use of many methods, possibly all qualitative or quantitative.} combines elements of both qualitative and quantitative research, with the aim to increase both breadth and depth of understanding of the phenomenon under study, and corroboration of results, giving more confidence in the conclusions reached. As a result, triangulation is in-built within the strategy.


%\textcquote{johnson2007toward}{Mixed methods research is the type of research in which a researcher or team of researchers combines elements of qualitative and quantitative research approaches (e.g., use of qualitative and quantitative viewpoints, data collection, analysis, inference techniques) for the broad purposes of breadth and depth of understanding and corroboration.}

%\textcquote[p.119, quoting \emph{Valerie Caracelli}]{johnson2007toward}{[P]lanfully [combining] methods of different types (qualitative and quantitative) to provide a more elaborated understanding of the phenomenon of interest (including its context) and, as well, to gain greater confidence in the conclusions generated by the evaluation study.}

\subsection{Knowledge contribution}

The knowledge contribution is the combination of the knowledge contributed by each of the methods applied, appropriately synthesised by considering connections and contradictions between qualitative and quantitative data. 

Mixed method research is particularly suited to interdisciplinary research and to the study of complex situations or social settings, particularly when one kind of method alone would not deliver the desired depth of understanding or richness of results. For example, within urban planning, you may be interested in improving pedestrians' safety, so that a mixed methods study may consider both quantitative data on pedestrian accidents and qualitative data on pedestrians' experiences and perceptions in order to identify both safe and dangerous areas to both learn lessons and plan remedial actions.

%In addition, \textcquote[p.119, quoting \emph{Huey Chen}]{johnson2007toward}{[the methods] can be adapted, altered, or synthesized to fit the research and cost situations of the study (modified form mixed methods).}


%\subsection{Focus}
%
%\subsection{Variants}
%
\subsection{Data generation and analysis}

How data are generated and analysed will depend on how the different methods are combined. Typical combinations include:
\begin{itemize}
	\item parallel, in which separate qualitative and quantitative methods are applied to gather different sets of data. For instance, your collection of pedestrians' accident data, and pedestrian's opinions may occur in parallel, independently of one another, then the results may be analysed and compared.
	\item sequential, in which the methods are applied one after the other, with outcomes from the first used to inform the second. For instance, you could start with the pedestrians' experience, then collect accident data on areas which are perceived as particularly safe or dangerous.
	\item nested (or embedded), in which a quantitative method is applied within a wider qualitative method (or vice versa). For instance, the focus may be primarily on the qualitative pedestrians' experience, within which some statistical analysis is applied, for instance, to look for correlations between such experience and accident data.
\end{itemize}

%Those of the individual methods, combined with those of the methods mixed. The latter form provides for triangulation: \textcquote[p.291]{denzin1978research}{the combination of methodologies in the study of the same phenomenon}: \textcquote[p.3]{webb2000unobtrusive}{If a proposition can survive the onslaught of a series of imperfect measures, with all their irrelevant error, confidence should be placed in it. Of course, this confidence is increased by minimizing error in each instrument and by a reasonable belief in the different and divergent effects of the sources of error.}

\subsection{Evaluation}

Evaluation questions for this strategy will include questions on the specific methods which are combined, alongside the following questions on their combination:

\begin{enumerate}
	\item Reliability: 
		\begin{itemize}
			\item How was the use of mixed methods justified in relation to the phenomena of interest? How has the study benefitted from their combination?
			\item Is the way the methods are combined appropriately described?
		\end{itemize}
	\item Validity: 
		\begin{itemize}
			\item How were connections between qualitative and quantitative findings established?
			\item How were conflicting or mismatched results from the different methods handled?
		\end{itemize}
\end{enumerate}

\subsection{Is this strategy right for me?}

If you are considering mixed methods research, you should take the following into account:

\begin{enumerate}
	\item Mixed methods research requires competence in more than a single research method, which takes time to develop. If your Masters project is your first research project, it is unlikely you will have a developed understanding sufficient to apply the mixed methods research strategy, and you should consider a single method strategy. However, if your work is part of broader mixed methods research, perhaps led by your supervisor, then you may be able to contribute by focusing on the particular method you are being asked to work with.
	\item Collecting and analysing both qualitative and quantitative data requires substantial time and resources. If this is going to be an issue, then a strategy with a single method focus would be a better choice.
\end{enumerate}

\subsection{Further reading}

To deepen your understanding of this strategy, you can start from \cite{johnson2007toward,denzin1978research,webb2000unobtrusive}.


%\subsection{More details}
%
%[Add summary of decision process as a diagram]
%
\chapter{Choosing and drafting your own research strategy}

By now you should have narrowed down your list of candidate strategies to few choices, and possibly discussed those choices with your supervisor. It is now time to learn more about them in order to make a final choice for your project. 

While the content of this section is brief, the two activities it contains are going to be demanding. It is important, however, that you don't skip them as they provide the foundation for your work in Stage 4.

\begin{question}[subtitle={Activity: Choosing your research strategy}]
For each candidate strategy in your list, consider the references you have recorded in your bibliographical database, and review the related literature in order to learn more about the strategy, and how you it may apply practically to your project. 

At the end of the activity reach a decision on which research strategy, or combination of strategies, to adopt.
\begin{guidance}
	The references we have provided are only a starting point, and you should also explore other literature on the topic. There is a vast literature on research methods, so it is easy to get lost. We recommend you look at introductory materials first to gain a broad understanding, then delve into more specialised literature for some of the details. Your supervisor will also be able to suggest appropriate reading.
	
	As you read the literature, you should take notes to augment the summaries we have provided in the previous section, to capture your deeper engagement and growing understanding of each research strategy. By the end of this activity you should have gained a good general understanding of how each strategy reviewed may suit your research.

	 In your review, you should pay particular attention to possible data generation and analysis methods under each strategy, reflecting on which may be most applicable to your project and how, alongside any risk or other factors which may affect their successful application. 
	
	It is not necessary for you to learn the fine details of each method at this point. However, by the end of this activity, you should have a clear idea of which methods you will be focusing on in Stage 4, in which you will engage with the specific procedures to apply those methods in your own project. 
\end{guidance} 
\end{question}


Your final activity in this chapter is a writing task: to draft your chosen research strategy, as a starting point for the narrative you will develop during the remainder of your project and eventually include in your dissertation. We recommend you apply the template in Table~\ref{tab:researchStrategyReport} to structure your writing.

\todo{LR: possibly add columns with reference to relevant sections?}

\begin{table}[htbp]
\small
\caption{How to summarise your research strategy\label{tab:researchStrategyReport}}
\begin{tabulary}{\textwidth}{@{}L|L@{}} 
\toprule
\textbf{Section} & \textbf{Content} \\
\midrule
Choice and justification & Indicate which strategy, or combination of strategies, you have chosen and justify this choice by considering how it aligns with the aim of your research in relation to the way it contributes new knowledge\\
\midrule
Mindset and research paradigm & Discuss how this strategy is consistent with your own mindset, with reference to the research paradigms introduced in Stage 2 \\
\midrule
Reflexive statement & Summarise your own standpoint as a researcher, including specific assumptions, beliefs and potential bias you bring to the research, and the steps you will take ensure they will not weaken your research \\
\midrule
Data and sources & Indicate the kind of data you will need to generate and analyse in your project,  possible data sources you will focus on and how you will gain access \\
\midrule
Data generation and analysis methods & Indicate the selection of methods you intend to apply in your data generation and analysis. Explain why you think they are suitable and feasible \\
\midrule
Ethical issues & Indicate all ethical issues relevant to your chosen research strategy, including listing any regulations you may need to comply with, and explicit permissions you may need to obtain to be able to proceed with your research  \\
\midrule
Research evaluation & Reflect on the potential weaknesses which may affect your chosen strategy, and consider the related evaluation questions from the previous section, alongside other evaluation criteria you may have found in the literature. Highlights those which are most relevant to your project and indicate which actions you may take to address them\\
\bottomrule
\end{tabulary}
\end{table} 

\begin{question}[subtitle={Activity: Sketching your research strategy}]
Apply the template in Table~\ref{tab:researchStrategyReport} to provide a first draft account of your research strategy.
\begin{guidance}
	This first draft will be necessarily tentative, but it should provide a good starting point that you can grow and revise alongside your increasing understanding and practical application of your strategy. 
	
	It is important that you engage with all elements of the template, including the evaluation section: although your full evaluation will only be completed at the end of your project, it is essential that you start thinking about the questions you will need to address. This, in turn, will help you ensure that the steps you take in your strategy application are likely to provide satisfactory answers to those questions.
\end{guidance} 
\end{question}


\chapter{Reflecting and Reporting in Stage 3}

More here

\chapter{Stage 3 Takeaways}

More here




%\begin{figure}[hbtp]
%\centering{
%  \includegraphics[width=0.7\textwidth]{Figures/researchStrategies}
%  \caption{Research strategy choices
%  \label{fig:researchStrategies}
%  	}  
%  }
%\end{figure}

%\section{Research skills audit}
%
%\begin{question}[subtitle={Activity: Skills audit}]
%Something here
%\end{question}


\endinput

%\section{Materials not used}
%
%\subsection{For your chosen research strategy}
%
%\begin{question}[subtitle={Activity: Dissertation structure}]
%Whichever tool you've chosen in which to write your dissertation, create chapters entitled \enquote{research strategy}, \enquote{method}, and \enquote{Evaluation}. 
%
%\begin{guidance}
%	For the research strategy chapter, make notes from the paper you've read on the general form of the research strategy. 
%
%For the method chapter, add details on the methods that are used in the research strategy. For a complex strategy such as ethnography, you may not use all of them, but you will need to be explicit – when you come to complete it – as to which you have excluded and the reasons for their exclusion. 
%
%For the evaluation chapter, create subsections for each of the questions of your chosen research strategy from the lists above.
%\end{guidance}
%	
%\end{question}
%\color{black}
%
%
%
%\subsection{Structuring research}
%
%Like a recipe, research needs to be structured. 
%
%To a large extent, the structure you use depends on your resources: massive research labs with many hundreds, even thousands, of researchers may need to run many different strands of research at the same time\footnote{According to \url{https://www.nature.com/articles/nature.2015.17567}, there are 5,154 authors on the paper \fullcite{aad2015combined}, in a collaboration between ATLAS and CMS, \blockquote{two massive detectors at the Large Hadron Collider (LHC) at CERN, Europe’s particle-physics lab near Geneva, Switzerland}. \textit{George Aad}, the first author, has the perfect surname for an academic.}, each contributing a small part towards an overall research goal. 
%
%Imagine having to manage that collaboration: 5154 researchers working in parallel.
%
%You aren't likely to have access to such resources. That's good, in a way, because you can keep the research simple and your research can be linear: one step after another. 
%
%Given that you're resource limited means we can plot your research as a single line
%%
%\newcommand{\taskline}[0]{\ \rule[0.5ex]{0.8cm}{1pt}\ }
%\[T\taskline T\taskline\cdots\taskline T\taskline T	
%\]
%%
%each node $T$ being one task, taken from one of the entries in Table~\ref{tab:researchTasks}. The design is simple because it's linear, and so there's not much to think about:
%%
%\begin{itemize}
%\item how long the line should be;
%\item what each $T$ will be.
%\end{itemize}
%%
%\todo{Need to talk about relationship between methods, design, tasks, etc}.
%More complex research designs, those involving multiple researchers, for instance, will require some amount of sophisticated project management to ensure that the sequencing of parallel research tasks is done correctly.
%
%The table below introduces more than 30 research tasks – the possible $T$s above – and for each gives a brief introduction and a key reference from which you can find out more\todo{Paraphrase descriptions in the table; more to come here.}.
%
%Your literature review may have thrown up papers with an explicit \enquote{methods} section that describes the research design – how you performed the research\footnote{There are examples from the APA here: \url{https://www.scribbr.com/apa-style/methods-section/}.}\todo{Work this in more}. 
%
%The point of a methods section is to report 
%\begin{stolen}{https://www.scribbr.com/apa-style/methods-section/}
%enough information to understand and replicate your study, including detailed information on the sample, measures, and procedures used.	
%\end{stolen}
%
%You may have come across\todo{complete.}.
%
%\subsection{A simple research design: Experimental Research}
%
%As an example of a simple research design, the shortest research design possible is
%%
%\[R\taskline O\taskline X \taskline O\]
%%
%which means (from the table\footnote{...where you can read more now, or wait until later in this chapter.}):
%%
%\begin{itemize}
%\item [R] Randomise sample from population, then
%\item [O] Observe, then
%\item [X] Change experimental variable, then
%\item [O] Observe, again.
%\end{itemize}
%%
%This form of research design is called \enquote{Experimental research}\footnote{There are simpler research designs} and is what you might think of as the quintessential scientific research – doing an experiment ($X$) on a random sample ($R$) population, a drug trial, for instance and observing the consequences ($O$).
%
%Even though we have called this a simple research design, it doesn't mean that the results that you can obtain by using it will be simple, it could be that the drug you're testing will make amazing strides in curing some illness, improving the lives of millions of people. What we mean by \emph{simple} is simply that there are few steps in the research. Simple doesn't mean, either, that the work that will need to go onto each step is simple, quick, or trivial. Observation, for instance, is an immensely difficult thing to do correctly; in the worst case, it may take many months of work to get to the point where you \enquote{change the experimental variable} – administer the drug, for instance – and so have something to observe.
%
%The text we recommend for experimental design is \textcite{marczyk2005essentials}\footnote{Page~124 is a good place to start in that edition.}, who speak from a social science background. .\todo{Do we need to write more here about the source?} 
%
%\todo{more here, using the above as a template, and the table below as source...}
%
%\subsection{Another research design: Quasi-experimental Research}
%
%More here from \textcite{marczyk2005essentials}.
%
%\subsection{Another research design: Non-experimental Research}
%
%More here from \textcite{marczyk2005essentials}.
%
%\subsubsection{Single-study or Case-study research}
%
%More here from \textcite{yin2009case}.
%
%\subsection{Design Science Research}
%
%More here from \textcite{oates2007researching}.
%
%\subsubsection{Completing your research design}
%
%Add Validity, BIAS, Reporting.
%
%\newcommand{\midtitle}[2]{\SetCell[c=3]{l,clear=preto}{\textbf{Research Strategy: #1}~\parencite{#2}}\\*%
%	Code&Description&Comments\\*%
%	}
%\SetCiteCommand{\parencite}
%\begin{longtblr}[%
%	expand=\midtitle,%\midtitle needs to be expanded so that pattern matching from LaTeX3 can work
%	caption={Research: tasks, codes and descriptions},%caption is an outer key
%	label={tab:researchTasks},%we can add a label
%]{%
%	width=\tablewidth,
%	colspec = {>{\arabic{rownum}–}r|X[2,l]X[7]},%first column right aligned, then 2/7 of remaining width
%%	column{1}={preto={\qquad}},%this doesn't seem to work
%%	row{1} = {font=\bfseries},%first row is bold, but don't need it because of \midtitle
%	measure=vbox,%needed to allow lists, \UseTblrLibrary{varwidth} added above
%	}
%%	Code&Description&Comments\\
%\midtitle{Experimental}{marczyk2005essentials,oates2007researching}
%	EXP&Experiment&\textcquote[p.35]{oates2007researching}{\textbf{Experiment}: focuses on investigating cause and effect relationships, testing hypotheses and seeking to prove or disprove a causal link between a factor and an observed outcome. There is 'before' and 'after' measurement, and all factors that might affect the results are carefully excluded from the study, other than the one factor that is thought will cause the 'after' result. (See Chapter 9.)}\\
%	 R&	Randomise sample from population& \textcquote[p.124]{marczyk2005essentials}{A true experimental design is one in which study participants are randomly assigned to experimental and control groups. We have discussed randomization in previous chapters, so this chapter will simply highlight the importance of randomization in terms of the strength of a research design. Although randomization is typically described using examples such as rolling dice, flipping a coin, or picking a number out of a hat, most studies now rely on the use of random numbers tables to help them assign their research participants (as discussed in Chapters 2 and 3).}\\
%	 O&	Observe phenomenon&\textcquote[p.119]{marczyk2005essentials}{Observation is another versatile approach to data collection. This approach relies on the direct observation of the construct of interest, which is often some type of behavior. In essence, if you can observe it, you can find some way of measuring it. The use of this approach is widespread in a variety of research, educational, and treatment settings.}\\
%	 X&	Change experimental variable&\textcquote[p.127]{marczyk2005essentials}{experimental manipulation (independent variable)}\\
%	 Y&	Change other variable&\textcquote[p.127]{marczyk2005essentials}{experimental manipulation (other variable)}\\
%	\\
%\midtitle{Quasi-experimental}{marczyk2005essentials}
%	 NR&Non-random sampling&\textcquote[p.138]{marczyk2005essentials}{when randomized designs are not feasible, researchers must often make use of quasi-experimental designs. A good rule of thumb is that researchers should attempt to use the most rigorous research design possible, striving to use a randomized experimental design whenever possible (Campbell, 1969).
%	 
%	 Cook and Campbell (1979) present a variety of quasi-experimental designs, which can be divided into two main categories: nonequivalent comparison-group designs and interrupted time-series designs. In this section, we will discuss these two major groups of quasi-experimental designs, followed by a brief overview of single-subjects designs.}\\
%	 REV&	Before the intervention, then after&\textcquote[p.142]{marczyk2005essentials}{\textbf{Reversal Time-Series Design} Also known as an ABA design (detailed on page 145), the reversal time-series design is basically a multi-subject variation of the single-subject reversal design, which will be discussed later in this chapter. The basic goal of this design is to establish causality by presenting and withdrawing an intervention, or independent variable, one to several times while concurrently measuring change in the dependent variable (as depicted in the following). As in the simple time-series design, this design begins with a series of pretests to observe normal fluctuations in baseline. The name “reversal” refers to the idea that causality can be inferred if changes that occur following the presentation of an intervention diminish or “reverse” when the independent variable is withdrawn.}\\
%	 ABA&	Before, Intervention, After&See REV.\\
%	 ABABA&	Iterated ABA&See REV.\\
%	 ABABA...&Further Iterated ABA	&See REV.\\
%	 EC&	Establish control&\textcquote[p.144]{marczyk2005essentials}{As with time-series designs, single-subject designs typically begin by establishing a stable baseline. Establishing a stable baseline involves taking repeated measures of a participant’s behavior (dependent variable) prior to the administration of any intervention to make certain that the participant’s behavior is occurring at a consistent rate. To obtain a stable baseline, the researcher must make special efforts to control all relevant environmental variables that otherwise might affect the participant’s responses. If the researcher does not know, or is uncertain, about which variables are relevant, the researcher must attempt to keep the participant’s environment as constant as possible by maintaining highly controlled conditions.}\\
%	 1P&	Single participant&\textcquote[p.144]{marczyk2005essentials}{Not to be confused with non-experimental single-subject case studies, which are covered later in this chapter, the single-subject experimental design has a long and respected tradition in empirical research. According to Kazdin (2003c), single-subject experiments might be seen as true experiments because they “can demonstrate causal relationships and can rule out or make implausible threats to validity with the same elegance of group research” (p. 273). Similar to other experimental designs, the single subject design seeks to (1) establish that changes in the dependent variable occur following introduction of the independent variable (temporal precedence) and (2) identify differences between study conditions.
%	 
%	 The one way that single-subject designs differ from other experimental designs is in how they establish control, and thereby demonstrate that changes in a dependent variable are not due to extraneous variables. For example, experimental designs rely on randomization to equally distribute extraneous variables and on statistical techniques to control for such factors if they are found. Alternatively, single-subject designs eliminate between-subject variables by using only one participant, and they control for relevant environmental factors by establishing a stable baseline of the dependent variable. If change occurs following the introduction of the intervention, or independent variable, the researcher can reasonably assume that the change was due to the intervention and not to extraneous factors.}\\
%	 SB&	Stable Baseline&See 1P\\
%	 RC&	Retain control of Env&See 1P\\
%	 \\
%\midtitle{Non-experimental}{yin2009case,oates2007researching}
%	CASE&Case Study&\textcquote[p.35]{oates2007researching}{\textbf{Case study}: focuses on one instance of the 'thing' that is to be investigated: an organization, a department, an information system, a discussion forum, a systems developer, a development project, a decision and so on. The aim is to obtain a rich, detailed insight into the 'life' of that case and its complex relationships and processes. (See Chapter 10.)}\\
%	AR&Action research&\textcquote[p.35]{oates2007researching}{\textbf{Action research}: focuses on research into action. The researchers plan to do something in a real-world situation, do it, and then reflect on what happened or was learnt, and then begin another cycle of plan-act-reflect. (See Chapter 11.)}\\
%	ETH&Ethanography&\textcquote[p.35]{oates2007researching}{Ethnography: focuses on understanding the culture and ways of seeing of a particular group of people. The researcher spends time in the field, taking part in the life of the people there, rather than being a detached observer. (See Chapter 12.)}\\
%	CS& 	Choose subject&\textcquote[p.144]{marczyk2005essentials}{single-subject designs eliminate between-subject variables by using only one participant, and they control for relevant environmental factors by establishing a stable baseline of the dependent variable. If change occurs following the introduction of the intervention, or independent variable, the researcher can reasonably assume that the change was due to the intervention and not to extraneous factors.
%	
%	As with time-series designs, single-subject designs typically begin by establishing a stable baseline. Establishing a stable baseline involves taking repeated measures of a participant’s behavior (dependent variable) prior to the administration of any intervention to make certain that the participant’s behavior is occurring at a consistent rate. To obtain a stable baseline, the researcher must make special efforts to control all relevant environmental variables that otherwise might affect the participant’s responses. If the researcher does not know, or is uncertain, about which variables are relevant, the researcher must attempt to keep the participant’s environment as constant as possible by maintaining highly controlled conditions.}\\
%	COMP&	Comprehensive description&\textcquote[p.148]{marczyk2005essentials}{the focus of the case-study approach is on individuality and describing the individual as comprehensively as possible. The case study requires a considerable amount of information, and therefore conclusions are based on a much more detailed and comprehensive set of information than is typically collected by experimental and quasi-experimental studies.}\\
%	IDIP&	In-depth interviews with participants&\textcquote[p.148]{marczyk2005essentials}{Case studies of individual participants often include in-depth interviews with participants ...}\\
%	IDIC&	In-depth interviews with collaterals&\textcquote[p.148]{marczyk2005essentials}{...and collaterals (e.g., friends, family members, colleagues), review of medical records, observation, and excerpts from participants’ personal writings and diaries}\\
%	SUR& Surveys&\textcquote[p.33]{oates2007researching}{\textbf{Survey}: focuses on obtaining the same kinds of data from a large group of people (or events), in a standardized and systematic way. You then look for patterns in the data using statistics so that you can generalize to a larger population than the group you targeted. (See Chapter 7.)}\\
%	RA&	Review of artefacts&\textcquote[p.148]{marczyk2005essentials}{According to Kazdin (1982), the major characteristics of case studies are the following:
%	\begin{itemize}
%		\item They involve the intensive study of an individual, family, group, institution, or other level that can be conceived of as a single unit.
%		\item The information is highly detailed, comprehensive, and typically reported in narrative form as opposed to the quantified scores on a dependent measure.
%		\item They attempt to convey the nuances of the case, including specific contexts, extraneous influences, and special idiosyncratic details.
%		\item The information they examine may be retrospective or archival.
%	\end{itemize}}\\
%%	(RQ)&	Research question??\\
%	PROPS&	Identify propositions&\textcquote[p.28]{yin2009case}{\textbf{Study propositions} [...] each proposition directs attention to something that should be examined within the scope of study.}\\
%	UNITS&	Identify units&\textcquote[p.29]{yin2009case}{\textbf{Unit of analysis} [...] related to the fundamental problem of defining what the \enquote{case} is [... what the primary unit of analysis is].
%	
%Without such questions and propositions, you might be tempted to cover \enquote{everything} about the individual(s), which is impossible to do.}\\
%	LINKS&	Identify how is data linked to propositions&\textcquote[p.34ff]{yin2009case}{be aware of the main choices and how they might suit your case study]}\\
%	CRITS&Which are criteria to interpret findings&\textcquote[p.34]{yin2009case}{Criteria for interpreting a study's findings}\\
%	THD&Theory Development&\textcquote[p.35]{yin2009case}{[including types on p.37]}\\
%	GEN&Generalisation&\textcquote[p.38]{yin2009case}{[including fig 2.2]}\\
%	NAR&Narrative&\textcquote[p.121]{yin2009case}{Certain types of narrative, produced by a case study investigator upon completion of all data collection, also may be considered a formal part of the database and not part of the final case study report. The narrative reflects a special practice that should be used more frequently: to have case study investigators compose open-ended answers to the questions in the case study protocol. This practice has been used on several occasions in multiple-case studies designed by the author (see BOX 24). 
%	
%	[Box 24]
%	
%		In such a situation, each answer represents your attempt to integrate the available evidence and to converge upon the facts of the matter or their tentative interpretation. The process is actually an analytic one and is the start of the case study analysis. }\\
%	{NSC\\NEI\\NID}&Nuance from the specific context/extraneous influences/idiosyncratic details&\textcquote{kazdin1982single}{According to Kazdin(1982), the major characteristics of case studies are the following:
%		\begin{itemize}
%			\item  They involve the intensive study of an individual, family, group, institution, or other level that can be conceived of as a single unit.
%			\item The information is highly detailed, comprehensive, and typically reported in narrative form as opposed to the quantified scores on a dependent measure.
%			\item They attempt to convey the nuances of the case, including specific contexts, extraneous influences, and special idiosyncratic details.
%			\item The information they examine may be retrospective or archival.
%		\end{itemize}}\\ 
%%	NEI&Nuance from extraneous influences\\ 
%%	NID&Nuance from idiosyncratic details\\
%	\\
%\midtitle{Design Science Research}{oates2007researching}
%	D\&C&Design and creation&\textcquote{oates2007researching}{\textbf{Design and creation}: focuses on developing new IT products, or artefacts. Often the new IT product is a computer-based system, but it can also be some element of the development process such as a new construct, model or method. (See Chapter 8.)}\\
%%	PSA&Problem solving awareness&\textcquote[p.111]{oates2007researching}{Awareness is the recognition and articulation of a problem, which can come from studying the literature where authors identify areas for further research, or reading, about new findings in another discipline, or from practitioners or clients expressing the need for something, or from field research or from new developments in technology.}\\
%%	PSS&Problem solving suggestion&\textcquote[p.112]{oates2007researching}{Suggestion involves a creative leap from curiosity about the problem to offering a very tentative idea of how the problem might be addressed}\\
%%	PSD&Problem solving development&\textcquote[p.112]{oates2007researching}{Development is where the tentative design idea is implemented. How this is done depends on the kind of IT artefact being proposed. For example, an algorithm might need the construction of a formal proof. A new user interface embodying novel theories about human cognition will require software development. A systems development method will need to be captured in a manual that can then be followed in a systems development project. A new approach in digital art might require the development of an art portfolio tracing the development of the artist's creative ideas.}\\
%%	PSE&Problem solving evaluation&\textcquote[p.112]{oates2007researching}{Evaluation examines the developed artefact and looks for an assessment of its worth and deviations from expectations.}\\
%%	PSC&Problem solving conclusion&\textcquote[p.112]{oates2007researching}{Conclusion is where the results from the design process are consolidated and written up, and the knowledge gained is identified, together with any loose ends - unexpected or anomalous results that cannot yet be explained and could be the subject of further research.}\\
%	PRU&Problem Understanding&\textcquote[p.49]{hall2017a-design}{gaining an understanding of the real-world environment in which the problem is located, and of the problem owner’s identified need}\\
%	PRV&Problem Validation&\textcquote[p.49]{hall2017a-design}{agreeing with the problem owner that the problem is representative, a form of validation}\\
%	SOU&Solution Understanding&\textcquote[p.49]{hall2017a-design}{producing the solution}\\
%	SOV&Solution Validation&\textcquote[p.49]{hall2017a-design}{convincing the problem owner that the solution meets the agreed recognised need in the agreed real-world environment to their satisfaction, another form of validation}\\
%\midtitle{General}{}
%	LITREV& Literature Review&\textcquote[p.33]{oates2007researching}{literature review in figure 3.1}\\
%	VALID&	Threats to validity&\textcquote[p.40]{yin2009case}{fours (general) tests for validity}\\
%	BIAS&	Reflection on bias&\textcquote[p.72]{yin2009case}{[Avoiding bias for case studies]}\\ 
%	REP&	Reporting&Something here\\
%	TRI&	Triangulation&\textcquote[p.37]{oates2007researching}{The use of more than one data generation method to corroborate findings and enhance their validity is called method triangulation. Many types of triangulation are possible in a research project:
%	%
%\begin{itemize}
%\item Method triangulation: the study uses two or more data generation methods.
%\item Strategy triangulation: the study uses two or more research strategies.
%\item Time triangulation: the study takes place at two or more different points in time.
%\item Space triangulation: the study takes place in two or more different countries or cultures to overcome the parochialism of a study based in just one country or culture.
%\item Investigator triangulation: the study is carried out by two or more researchers who then compare their accounts.
%\item Theoretical triangulation: the study draws on two or more theories rather than one theoretical perspective only.
%\end{itemize}
%%
%
%
%Researching Information Systems and Computing
%Triangulation gives researchers multiple modes of \enquote{attack} on their research question.
%However, researchers differ over whether they should expect triangulation of method or time or space, and so on, to lead to consistency of findings. It depends on their underlying research philosophy (see Chapters 19 and 20 for a detailed explanation). \enquote{Positivists} subscribe to the idea of a single \enquote{truth} or \enquote{reality} and would expect the multiple lines of attack to lead to a consistent set of findings. Interpretivists', on the other hand, do not subscribe to the idea of a single reality, believing any notion of 'reality' to be constructed by individuals and groups, so there are multiple realities for people in our world, and different research approaches are likely to lead to different findings. Interview data about recollections of a meeting and company minutes of the same meeting, for example, are two different \enquote{stories}, created by different people for different audiences. Interpretivists would not always expect to see convergence in the data they generate using triangulation.}\\
%\midtitle{Data Generation}{oates2007researching}
%	INT&Interview&\textcquote[p.36]{oates2007researching}{\textbf{Interview}: a particular kind of conversation between people where, at least at the beginning of the interview if not all the way through, the researcher controls both the agenda and the proceedings and will ask most of the questions. One-to-one and group interviews are possible. (See Chapter 13.)}\\
%	OBS&Observation&\textcquote[p.36]{oates2007researching}{\textbf{Observations}: watching and paying attention to what people actually do, rather than what they report they do. Often involves looking, but it can involve the other senses too: hearing, smelling, touching and tasting. (See Chapter 14.)}\\
%	QUES&Questionnaire&\textcquote[p.36]{oates2007researching}{\textbf{Questionnaire}: a pre-defined set of questions assembled in a pre-determined order. Respondents are asked to answer the questions, often via multiple choice options, thus providing the researcher with data that can be analysed and interpreted. (See Chapter 15.)}\\
%	DOC&Documents&\textcquote[p.36]{oates2007researching}{\textbf{Documents}: documents that already exist prior to the research (for example, policy documents, minutes of meetings and job descriptions) and documents that are made solely for the purposes of the research task (for example, a researcher's logbook or design models). Also includes \enquote{multimedia documents}: visual data sources (for example, photographs, diagrams, videos and animations), aural sources (for example, sounds and music) and electronic sources (for example, websites, computer games and electronic bulletin boards). (See Taspic
%	Chapter 16.)}\\
%	EVAL&Evaluate&\textcquote[p.40]{oates2007researching}{\textbf{Evaluating the Research Process}
%		Now that you know something of the research process, you can start to analyse and evaluate how well other researchers have described their process. Use the Evaluation Guide' below to help you.
%
%EVALUATION GUIDE: RESEARCH PROCESS
%%
%\begin{enumerate}%[start=0,label={(\bfseries R\arabic*):}]
%\item Do the researchers make clear their research question(s)?
%\item Do the researchers explain the theory (ies) they use to conceptualize the research topic?
%\item Do the researchers make clear both their strategy and the data generation method(s) within that strategy?
%\item Do the researchers indicate their criteria for judging the success or usefulness of their work?
%\item Is there a clear process summarized, from the original motivation and literature review through to final outcome(s)? If not, how does that affect your confidence in the research and its reporting?
%\end{enumerate}}\\
%\end{longtblr}
%
%\subsection{Alternative models of the research process}
%
%See \parencite[p.39]{oates2007researching} for two more models for the research process (in the context of Information Systems and Computing)
%%
%\begin{itemize}
%\item Conceptualise, operationalise, generalise;
%\item The SLDC (Software Development Life Cycle) Analogy.
%\end{itemize}
%%
%
%
%
%%As a running example, we'll be working with the following research objectives, which you saw in Example~\ref{ex:machinelearning} in Stage 1\footnote{See page~\pageref{ex:machinelearning}.}\todo{What was the research problem?}:
%
%\subsection{Decomposing objectives into tasks}
%
%You've chosen your research design based on area, and you've got your research objectives from Section~\ref{sect:???}. How do you go about mapping one into the other?
%
%Our suggested template for creating objectives had three components: identify, assess, and recommend.
%%
%%
%\begin{description}
%\item [identify:] literature review; questionnaire, interviews; problem solving awareness, ...
%\item [assess:]	what goes here? problem solving suggestions; interviews; problem solving development; ...
%\item [recommend:] what goes here? problem solving evaluation; problem solving conclusion; validity; bias; ...
%\end{description}
%%
%
%\todo{Turn this example into identifying which research design.}
%
%\begin{example}{Recap: Applying Machine Learning}
%In Stage 2, we refined Clara's research aim, which was:
%%
%\blockquote{to apply Machine Learning (ML) to improve the accuracy of resources and time forecasting in the context of small engineering plants}
%%
%to three following three objectives:\todo[inline]{JGH: needs doing if not already done}
%
%\begin{description}%[start=0,label={(\bfseries R\arabic*):}]
%\item [Objective 1] to identify which ML techniques are applicable to resource and time forecasting in the context of small engineering plants, which will allow us to identify specific ML techniques to be used in the project, to ensure the work is feasible within the time-frame of the project. 
%
%\item [Objective 2] to test the accuracy of forecasting of those techniques which will allow us to investigate and compare how accurate the chosen techniques are in their forecasting application. 
%
%\item [Objective 3] to provide recommendations as to how integrate those techniques effectively in engineering practice in order to improve forecasting accuracy which will allows us to draw some conclusions from the research conducted and make recommendations to improve professional practice.
%\end{description}
%
%Note how those objectives were designed to build on each other and, when successfully completed, they'd contribute to meet the overall aim.
%\end{example}
%
%\begin{example}{Example – cont'd}In our example, the first objective is met once we have identified the relevant ML techniques. There are two complementary ways to do this: to look at the literature and to ask practitioners. As a result, we could break this objective down into the tasks, and deliverables, indicated in the following table
%
%\begin{longtable}{@{}p{0.1\textwidth}@{}p{0.9\textwidth}@{}}
%\caption{Objective 1: to identify which ML techniques...}\\
%\toprule
%\textbf{Task} & \textbf{Deliverable} \\\midrule
%\tabletitle{to identify relevant ML techniques in the academic literature} & a collection of relevant ML techniques reported in the literature \\\\
%\tabletitle{to ask practitioners which techniques they employ} & a collection of relevant ML techniques used in professional practice \\
%\bottomrule
%\end{longtable}
%\end{example}
%
%\begin{question}[subtitle={ACTIVITY: Establishing tasks and deliverables}] Consider your research objectives. For each, identify related tasks and deliverables.\todo[inline]{I don't think I could do this at this point.}
%
%\begin{guidance}You should draw a table similar to that in our running example. You should ensure that the tasks provide a reasonable break down of your objectives into discrete pieces of work.
%\end{guidance}\end{question}
%%%Hack to correct tcbox behaviour
%\color{black}
%
%Your tasks and deliverables don't need to be perfect in stage 3 – there are two more stages to perfect them after all – and are likely to be revised as you progress through your project. However, it is important that you have thought about specific work you will need to carry out to meet your objectives.
%
%\subsubsection{Relating tasks to research methods}
%The way to carry out your tasks and meet your objectives is through the application of research methods.
%
%\begin{example}{EXAMPLE - cont'd }Following on from our previous example, we have extended the table to include an indication and justification of candidate research methods for each task.
%
%\begin{tblr}{colspec={XXXX},
%row{1}={font=\bfseries},
%}
%%\caption{\textbf{Objective 1: to identify which ML techniques...}
%%\hline[1pt]
%Task&Deliverable&Relevant research methods&Justification and feasibility\\
%%\hline[0.5pt]
%\SetCell[c=3]{l}{to investigate the academic literature in order to identify relevant ML techniques}\\ 
%&a collection of relevant ML techniques reported in the literature & review of existing literature & I can access relevant literature via my university library\\
%\SetCell[c=3]{l}{to ask practitioners which techniques they employ}\\ 
%& a collection of relevant ML techniques used in professional practice & questionnaire, possibly followed by interviews & I have access to professional networks, which I could use to distribute the questionnaire, and possibly to recruit participants for follow-up interviews \\
%%\hline[1pt]
%\end{tblr}
%\end{example}
%
%Note that the choice of research methods in relation to your research tasks is an essential part of your research design. In fact, the two influence each other: your objectives and related tasks direct you towards specific research methods, which in turn have to be part of your overall research design.
%
%\begin{question}[subtitle={ACTIVITY: Associating methods to tasks and deliverables}] Extend your tasks and deliverables table with your candidate research methods, including stating why they apply and are feasible for your project. Revise your research design draft from Stage 2 so that is consistent with those choices.
%
%\begin{guidance}
%Refresh your understanding of chosen research methods from the study work you carried out in Stage 2. It is important you keep reviewing your choices with your supervisor.
%\end{guidance}
%\end{question}
%%%Hack to correct tcbox behaviour
%\color{black}
%
%\subsubsection{Research task deliverables}
%
%\todo[inline]{Add something here}
%
%\subsection{Research procedures}
%\todo{What's the relationship to objectives and tasks?}
%
%Once you have chosen the set of research methods you will apply, you must establish exactly how you will do that, something we refer to as \textbf{research procedures}.
%
%Your research procedures will be method specific, in that each method you choose to apply will come with recommended practices, which you will need to contextualise to your own project needs, including your access to participants, data or other kind of evidence. For instance, there are plenty of guidelines in the literature on how to design questionnaires, including which type of questions to include and how to phrase them. There are also recommendations concerning testing the questionnaire design prior to its use, and of course, there are many ways a questionnaire can be administered. In writing your procedures for this research method, you would have to be specific on how each of the above applies in your project.
%
%It is important, therefore, that you master the research methods of your choice, starting by reviewing once again the related academic literature.
%
%\begin{question}[subtitle={Activity: Sketching research procedures}] Consider the research methods you intend to apply, and the related review you conducted in Stage 2. Reconsider those materials, possibly going back to the literature sources, to learn how to apply the methods effectively within your project.
%
%For each method, sketch possible procedures of application, ensuring you make appropriate reference to the literature you have reviewed and best practice guidelines therein.
%
%\begin{guidance}It is possible that the review you conducted in Stage 2 is not sufficient, in which case you will need to extend it to complete this activity.
%
%You should focus on practical aspects of applying the methods, including specific processes and techniques to gather, summarise and present your evidence in your reports.
%
%Depending on the extent you need to review further academic literature, this activity could be quite substantial, so you should set aside up to 20\% of your study time to complete it.
%
%\end{guidance}\end{question}
%%%Hack to correct tcbox behaviour
%\color{black}
%
%\subsection{Assessing validity}
%As your intended research design becomes clearer, you will soon be testing some aspects of it in your pilot work. Before you do that, however, you need to consider if the choices you have made will allow you to gather evidence and derive findings in a systematic, rigorous, repeatable and reliable fashion so as to address your research problem. This is referred to as assessing the overall \textbf{validity} of your research design, which is broken down into the following considerations.
%
%\textbf{Construct validity} asks whether you have put your design together logically by focusing on the relationship between evidence and research problem. Here you ask yourself whether the evidence you will generate through your chosen research design will be accurate and relevant to address your research problem. This tests the logical coherence of your aim, objectives, tasks, methods and deliverables in relation to the research problem and the knowledge gap you intend to address. With construct validity, you are asking: \emph{have I designed my research in the right way?}
%
%\textbf{Internal validity} is concerned with the way you gather and analyse evidence. All research strategies and methods come with recommendations of good practice to ensure that your research is both systematic, repeatable and reliable. In your work, you need to ensure that you follow such practices and are aware of possible pitfalls. For instance, in experimental research you need to control all factors which may effect outcomes beyond those under study: failing to exercise such control will lead to observations and measurements which are unreliable. In assessing internal validity, you should also take into account limitations of human perception and cognition, and any potential personal bias. With internal validity, you are asking: \emph{have I executed my research in the right way?}
%
%\textbf{External validity} relates to the extent you will be able to generalise your findings beyond the immediate context of your research. For instance, you may conduct a case study within a specific organisation, so here you are asking whether and how what you have discovered may apply to other organisations. With external validity, you are asking: \emph{will my research lead to findings that apply somewhere else?}
%
%Anything that gets in the way of validity in research is termed a \textbf{threat to validity}. Different research strategies and methods are exposed to different threats, something you should have encountered in your review of the literature on your chosen methods.
%
%\begin{question}[subtitle={Activity: Assessing validity of research design}] Conduct an initial assessment of your chosen research design in relation to the three kind of validity discussed above. Write down a short summary of your thinking in support of each, and of possible threats to validity you envisage.
%
%\begin{guidance}You may need to refer back to the literature you have reviewed to identify specific threats which apply to your chosen research methods and strategies.
%
%You won't be able to address this in full at this point in your project, particularly the internal validity, which refers to the execution of your research design. Nevertheless, it is important for you to consider validity and possible threats from the onset. You will return to this topic at the end of your project, as part of the overall assessment of your research, to reflect on the validity of your completed research.
%
%\end{guidance}\end{question}
%%%Hack to correct tcbox behaviour
%\color{black}
%
%\subsection{Conducting your pilot work}
%
%Your \textbf{pilot work} will be a small scale test of some of the methods and procedures you will apply in the next stage of your project. Its main function is to help you assess the feasibility of your research design, or at least some aspects of it, and build your confidence in the approach you have chosen.
%
%As such, your pilot work may not contribute directly to your aim and objectives, but it should help you decide whether you can actually do what you have planned to do, or inform how your research design and project plan should change instead.
%
%There are no constraints on what you can do for your pilot work, other than you should exercise some aspect of your research design. It is therefore essential that you agree what you are going to do with your supervisor first.
%
%\begin{question}[subtitle={Activity: Planning and executing your pilot work}] Plan your pilot work and discuss your plan with your supervisor.
%
%Once you have agreed the way forward, execute your plan and write a summary of both its execution and outcomes.
%
%\begin{guidance}
%This is a substantial activity, which will take you up to 35\% of your study time.
%
%Your summary should include:
%%
%\begin{itemize}
%\item an indication of which aspects of your research design your pilot work was concerned with
%\item any methods and procedures applied
%\item any data or evidence gathered, including possible modelling, artefact design or prototyping, appropriately presented and summarised
%\item lessons learnt and any resulting revision to your research design and project plan, particularly in relation to construct and internal validity of your research design.
%\end{itemize}
%%
%To complete this activity successfully, it is essential that you agree your pilot work plan with your supervisor upfront, and discuss your progress on a regular basis.
%\end{guidance}\end{question}
%%%Hack to correct tcbox behaviour
%\color{black}
%
%\subsection{Reporting in Stage 3}
%At the end of Stage 3, you should complete a report, extending that of Stage 2 and covering the work you have carried on in this stage. The structure we suggest and an indication of the contents are shown in Table~\ref{tab:reportStructure}.
%
%%%Report Structure Table is repeated throughout the thesis. This is the template
%%%Format is:
%%%\begin{ReportStructureTable}
%%%	\tabletitle{Section 1: Introduction}\\
%%%	\begin{enumerate}[label={1.\arabic*:}]
%%%	\item Background to the research 
%%%	\item Justification for the research 
%%%	\end{enumerate}
%%%	& This section should provide an introduction to your research topic in its wider context (as background) and your justification of why the research is worth pursuing. It should be well articulated and supported by evidence \\
%%%\end{ReportStructureTable}
%%%Still to do: remove space from above enumerate environment
%%%Sets the chapter across two columns in bold
%\begin{ReportStructureTable}{tab:reportStructure}
%\tabletitle{Title} & Your title should succinctly capture your research problem and aim\\\\
%\tabletitle{Section 1: Introduction}\\
%\begin{enumerate}[label={1.\arabic*:}]
%\item Background to the research 
%\item Justification for the research 
%\end{enumerate}
%& This section should provide an introduction to your research topic in its wider context (as background) and your justification of why the research is worth pursuing. It should be well articulated and supported by evidence \\\\
%\tabletitle{Section 2: Literature review}\\
%\begin{enumerate}[label={2.\arabic*:}]
%\item Review of existing relevant knowledge 
%\item Critical summary, including knowledge gap to be addressed by the research 
%\end{enumerate}
%& Your review should provide a critical account of your in-depth engagement with the academic (and other) relevant literature, including identifying key trends, ideas and possible knowledge gaps. Most of your citations should point to academic articles. Your critical summary should highlight key insights from your review and provide a strong justification for your proposed research. Both coverage and depth of your review matter. You should ensure that your review is well structured, with a logical narrative flow and your arguments are well supported by evidence  \\\\
%\tabletitle{Section 3: Research definition}\\
%\begin{enumerate}[label={3.\arabic*:}]
%\item Problem statement 
%\item Aim, objectives, tasks and deliverables
%\item Knowledge contribution
%\end{enumerate}
%& You should ensure that your research problem is well articulated and appropriate for your course and your personal and professional circumstances, that your aim and objectives are consistent with research problem, that tasks and deliverables break down your objectives appropriately and are clearly related to your chosen research methods, and that the intended knowledge contribution of your research is clearly articulated \\
%\tabletitle{Section 4: Research design}\\\\
%\begin{enumerate}[label={4.\arabic*:}]
%\item Evidence and data 
%\item Research strategy and methods
%\item Research procedures
%\item Ethical, legal and EDI considerations
%\end{enumerate}
%& This section should demonstrated your critical engagement with all elements of research design, including a detailed account of the data and evidence needed in your research, the research methods and research strategies you will to apply, and how you will apply them within your project. Your account should be supported by a clear rationale and insights from the related literature, and appropriately justified in relation to your research problem, aim and objectives. It should also demonstrate your careful consideration of ethical and legal matters, and that your research will comply with your course and university requirements\\\\
%\tabletitle{Section 5: Analysis and interpretation}\\
%\begin{enumerate}[label={5.\arabic*:}]
%\item Pilot work
%\end{enumerate}
%& This section should report on a well thought-out pilot work which clearly and competently test some significant aspect of your research design. It should demonstrate good critical reflection on outcomes and highlight any adjustments needed as a result. \\\\
%\tabletitle{Section 6: Assessment of your proposed research}\\
%\begin{enumerate}[label={6.\arabic*:}]
%\item Qualification fit
%\item Personal and professional fit
%\item Technical skills and resources needed
%\item Statement of feasibility
%\item Personal reflection on research process
%\end{enumerate}
%& In this section you should continue to argue how your research is a good fit across all criteria. You should provide a clear rationale as to why you think what you are proposing is feasible. You should also reflect on your growing understanding of the research process, including key learning and aspects you have found particularly challenging. \\\\
%\tabletitle{Section 7: Planning, scheduling and risk assessment}\\
%\begin{enumerate}[label={7.\arabic*:}]
%\item Key priorities in follow-up stage
%\item Personal and professional fit
%\item Risk assessment
%\end{enumerate}
%& In this section you should reflect on the progress you have made in Stage 2 and establish your priorities for the next stage. You should also review your risk assessment as appropriate.\\\\
%\tabletitle{Section 8: References}\\ & You should keep your growing references in good order and ensure you apply the required bibliographical style consistently. Ideally, you should use a BMT to generate and integrate your references within your report\\\\
%\textbf{Appendix A: Work schedule}& Your revised work plan\\\\
%\textbf{Appendix B: Risk assessment table}& Your revised risk table \\
%\bottomrule
%\end{ReportStructureTable}
%
%\endinput
%
%\begin{question}[subtitle={Activity: Putting your report together}] Using your word processor of choice, and starting from your previous report, complete your Stage 3 report by applying the structure and guidance in Table~\ref{tab:ReviewCrit}, and making good use of your notes and summaries from all related activities you have carried out so far.
%
%\begin{guidance}In this first pass at putting together your report, you should focus primarily on completeness, ensuring that each section includes at least draft content.
%\end{guidance}\end{question}
%%%Hack to correct tcbox behaviour
%\color{black}
%
%As in the previous stages, after you have filled in your report you should review and revise it iteratively until you are happy with your account, and are ready to move on. 
%
%\begin{table}[htbp]
%\begin{minipage}{\linewidth}
%\setlength{\tymax}{0.5\linewidth}
%\centering
%\caption{Criteria to review your report\label{tab:ReviewCrit}}
%\small
%\begin{tabulary}{\tablewidth}{@{}LL@{}} \toprule
% \textbf{Criteria} & \textbf{Prompts} \\
%\midrule
%
% \tabletitle{Completeness} & Are all sections of the suggested structure completed in line with the guidance provided? \\
% \tabletitle{Good academic writing practices} & Have you applied good academic writing practices throughout? \\
% \tabletitle{Logical structure and flow} & Have you structured your narrative appropriately to ensure a logical flow of arguments? \\
% \tabletitle{Supporting references or evidence} & Are your key arguments supported by appropriate references or other evidence? \\
% \tabletitle{Citation and reference style} & Do all your citations and references comply with the required bibliographical style? \\
% \tabletitle{Avoiding plagiarism} & Have you acknowledged the work of others and distinguished it from your own appropriately? \\
% \tabletitle{Standard of English (or any modern language you use)} & Have you proof-read your report carefully to remove all typos and grammatical errors? \\
%\bottomrule
%
%\end{tabulary}
%\end{minipage}
%\end{table}
%
%\begin{question}[subtitle={Activity: Reviewing your report}] Apply the criteria in Table 1 to review your current report and write up a summary of your assessment.
%
%\begin{guidance}For each criteria, consider the related prompts to help you assess your report overall, and write down any further work needed for your next stage.
%\end{guidance}\end{question}
%%%Hack to correct tcbox behaviour
%\color{black}
%
%\subsection{Reflection: Stage 3}
%
%%%More here
%
%%%Repeated reflection activity
%%%Repeated Activity for all reflections
\begin{question}[subtitle={Activity}]
$<$Needs assessing for content and structuring into activity + guidance$>$

This activity has four parts: the first is something you should be doing regularly, but won't make you into a disobedient or indocile thinker. The second, third and fourth may help you get started and keep going.

Part 1: Think about your study this far -- using this book and anything you've done for your dissertation in parallel -- as a journey. More from elsewhere, including   !!.

Part 2: think about yourself and the way you think. How does your desk look? Is it messy or tidy? Do the same for your computer desktop. Is it empty or are there hundreds of files strewn across it? Do you think your tidiness or untidiness will affect the way you do your research? How about how you keep your -- critically important -- bibliographic database which may contain up to a hundred academic\footnote{It's not unknown to have more than a hundred.} and other articles by the time you're finished?

Part 3: think about the context of your research. Which professional pressures are there on you to succeed in solving your research problem? Pressures could come in many forms: financial -- there's a promotion for you at the end of it; peer -- your colleagues know that you are studying will have good expectations of your result and you'll want to prove them right\footnote{Or wrong, depending on the colleague!}. Are you sponsored by your employer? Will you be able to report a negative outcomes to your research, for instance, that there is no solution to our problem using the current technology stack? A negative result is a very good research outcome, even if it tends to satisfy fewer non-academics than a positive result.

Which family pressures do you feel? It's' not unusual that you will have given up a paying role to study, moving the responsibility to provide onto another member of your family. What are their expectations?

Part 4: what's that thought nagging at the back of your mind? Is it ``How will I start?'' Or ``Will I be able to dedicate enough time to this?'' Or ``Can I really do this?''. Or ''Is ``shouldn't I be bringing in a wage rather than studying?''

You may be one of the lucky ones that doesn't have such negative thoughts, but negative thoughts are a very natural part of steps into the unknown. And research is precisely that, a step into the unknown. At least if you are aware of the doubts you naturally have, you can manage them. Think about making even the tiniest of steps forward in your research visible and celebrated! Work with Kansan boards where progress is encouragingly visible as you move a task from the inbox to the outbox. If you have concerns about managing your time, start using one of the many tools out there that break time up into manageable units and help manage it for you. If your concerns are about how to organise your thoughts, look into mind maps, lists, todo lists.

Thinking early and often through reflection is a powerful way of doing better. Do it well and your final report will be better than you will have expected.

It's worth saying that, at the end of what could be an exhausting journey, you will not fully appreciate your achievements. That realisation may have to wait until you are rested, graduated, or some distant time later.

But it will come.

\begin{guidance}
%Hack to correct tcbox behaviour
\color{black}
Something here
\end{guidance}\end{question}
%%Hack to correct tcbox behaviour
\color{black}

